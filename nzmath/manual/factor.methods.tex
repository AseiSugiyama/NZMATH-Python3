\documentclass{report}

%%%%%%%%%%%%%%%%%%%%%%%%%%%%%%%%%%%%%%%%%%%%%%%%%%%%%%%%%%%%%
%
% macros for nzmath manual
%
%%%%%%%%%%%%%%%%%%%%%%%%%%%%%%%%%%%%%%%%%%%%%%%%%%%%%%%%%%%%%
\usepackage{amssymb,amsmath}
\usepackage{color}
\usepackage[dvipdfm,bookmarks=true,bookmarksnumbered=true,%
 pdftitle={NZMATH Users Manual},%
 pdfsubject={Manual for NZMATH Users},%
 pdfauthor={NZMATH Development Group},%
 pdfkeywords={TeX; dvipdfmx; hyperref; color;},%
 colorlinks=true]{hyperref}
\usepackage{fancybox}
\usepackage[T1]{fontenc}
%
\newcommand{\DS}{\displaystyle}
\newcommand{\C}{\clearpage}
\newcommand{\NO}{\noindent}
\newcommand{\negok}{$\dagger$}
\newcommand{\spacing}{\vspace{1pt}\\ }
% software macros
\newcommand{\nzmathzero}{{\footnotesize $\mathbb{N}\mathbb{Z}$}\texttt{MATH}}
\newcommand{\nzmath}{{\nzmathzero}\ }
\newcommand{\pythonzero}{$\mbox{\texttt{Python}}$}
\newcommand{\python}{{\pythonzero}\ }
% link macros
\newcommand{\linkingzero}[1]{{\bf \hyperlink{#1}{#1}}}%module
\newcommand{\linkingone}[2]{{\bf \hyperlink{#1.#2}{#2}}}%module,class/function etc.
\newcommand{\linkingtwo}[3]{{\bf \hyperlink{#1.#2.#3}{#3}}}%module,class,method
\newcommand{\linkedzero}[1]{\hypertarget{#1}{}}
\newcommand{\linkedone}[2]{\hypertarget{#1.#2}{}}
\newcommand{\linkedtwo}[3]{\hypertarget{#1.#2.#3}{}}
\newcommand{\linktutorial}[1]{\href{http://docs.python.org/tutorial/#1}{#1}}
\newcommand{\linktutorialone}[2]{\href{http://docs.python.org/tutorial/#1}{#2}}
\newcommand{\linklibrary}[1]{\href{http://docs.python.org/library/#1}{#1}}
\newcommand{\linklibraryone}[2]{\href{http://docs.python.org/library/#1}{#2}}
\newcommand{\pythonhp}{\href{http://www.python.org/}{\python website}}
\newcommand{\nzmathwiki}{\href{http://nzmath.sourceforge.net/wiki/}{{\nzmathzero}Wiki}}
\newcommand{\nzmathsf}{\href{http://sourceforge.net/projects/nzmath/}{\nzmath Project Page}}
\newcommand{\nzmathtnt}{\href{http://tnt.math.se.tmu.ac.jp/nzmath/}{\nzmath Project Official Page}}
% parameter name
\newcommand{\param}[1]{{\tt #1}}
% function macros
\newcommand{\hiki}[2]{{\tt #1}:\ {\em #2}}
\newcommand{\hikiopt}[3]{{\tt #1}:\ {\em #2}=#3}

\newdimen\hoge
\newdimen\truetextwidth
\newcommand{\func}[3]{%
\setbox0\hbox{#1(#2)}
\hoge=\wd0
\truetextwidth=\textwidth
\advance \truetextwidth by -2\oddsidemargin
\ifdim\hoge<\truetextwidth % short form
{\bf \colorbox{skyyellow}{#1(#2)\ $\to$ #3}}
%
\else % long form
\fcolorbox{skyyellow}{skyyellow}{%
   \begin{minipage}{\textwidth}%
   {\bf #1(#2)\\ %
    \qquad\quad   $\to$\ #3}%
   \end{minipage}%
   }%
\fi%
}

\newcommand{\out}[1]{{\em #1}}
\newcommand{\initialize}{%
  \paragraph{\large \colorbox{skyblue}{Initialize (Constructor)}}%
    \quad\\ %
    \vspace{3pt}\\
}
\newcommand{\method}{\C \paragraph{\large \colorbox{skyblue}{Methods}}}
% Attribute environment
\newenvironment{at}
{%begin
\paragraph{\large \colorbox{skyblue}{Attribute}}
\quad\\
\begin{description}
}%
{%end
\end{description}
}
% Operation environment
\newenvironment{op}
{%begin
\paragraph{\large \colorbox{skyblue}{Operations}}
\quad\\
\begin{table}[h]
\begin{center}
\begin{tabular}{|l|l|}
\hline
operator & explanation\\
\hline
}%
{%end
\hline
\end{tabular}
\end{center}
\end{table}
}
% Examples environment
\newenvironment{ex}%
{%begin
\paragraph{\large \colorbox{skyblue}{Examples}}
\VerbatimEnvironment
\renewcommand{\EveryVerbatim}{\fontencoding{OT1}\selectfont}
\begin{quote}
\begin{Verbatim}
}%
{%end
\end{Verbatim}
\end{quote}
}
%
\definecolor{skyblue}{cmyk}{0.2, 0, 0.1, 0}
\definecolor{skyyellow}{cmyk}{0.1, 0.1, 0.5, 0}
%
%\title{NZMATH User Manual\\ {\large{(for version 1.0)}}}
%\date{}
%\author{}
\begin{document}
%\maketitle
%
\setcounter{tocdepth}{3}
\setcounter{secnumdepth}{3}


\tableofcontents
\C

\chapter{Functions}

%---------- start document ---------- %
 \section{factor.methods -- factoring methods}\linkedzero{factor.methods}

It uses methods of \linkingzero{factor.find} module
or some heavier methods of related modules to find a factor.
Also, classes of \linkingzero{factor.util} module is used to track
the factorization process.
\param{options} are normally passed to the underlying function without modification.

 This module uses the following type:
 \begin{description}
   \item[factorlist]\linkedone{factor.methods}{factorlist}:\\
     \param{factorlist} is a list which consists of pairs {\tt (base, index)}.
     Each pair means \(base^{index}\).
     The product of these terms expresses prime factorization.
 \end{description}
%
  \subsection{factor -- easiest way to factor}\linkedone{factor.methods}{factor}
   \func{factor}
   {%
     \hiki{n}{integer},\ %
     \hikiopt{method}{string}{'default'},\ %
     **\param{options}
   }{%
     \out{\linkingone{factor.methods}{factorlist}}
   }\\
   \spacing
   % document of basic document
   \quad Factor the given positive integer \param{n}.\\
   \spacing
   \quad By default, use several methods internally.\\
   \spacing
   \quad The optional argument \param{method} can be:
   \begin{itemize}
   \item {\tt 'ecm'}: use elliptic curve method.
   \item {\tt 'mpqs'}: use MPQS method.
   \item {\tt 'pmom'}: use \(p-1\) method.
   \item {\tt 'rhomethod'}: use Pollard's \(\rho\) method.
   \item {\tt  'trialDivision'}: use trial division.
   \end{itemize}
   (\negok In fact, the initial letter of method name suffices to specify.)\\
%
  \subsection{ecm -- elliptic curve method}\linkedone{factor.methods}{ecm}
  \func{ecm}
   {%
     \hiki{n}{integer},\ %
     **\param{options}
   }{%
     \out{\linkingone{factor.methods}{factorlist}}
   }\\
   \spacing
   % document of basic document
   \quad Factor the given integer \param{n} by elliptic curve method.\\
   \spacing
   % added document
   (See \linkingone{factor.ecm}{ecm} of \linkingzero{factor.ecm} module.)\\
%
  \subsection{mpqs -- multi-polynomial quadratic sieve method}\linkedone{factor.methods}{mpqs}
  \func{mpqs}
   {%
     \hiki{n}{integer},\ %
     **\param{options}
   }{%
     \out{\linkingone{factor.methods}{factorlist}}
   }\\
   \spacing
   % document of basic document
   \quad Factor the given integer \param{n} by multi-polynomial quadratic sieve method.\\
   \spacing
   % added document
   (See \linkingone{factor.mpqs}{mpqsfind} of \linkingzero{factor.mpqs} module.)\\

  \subsection{pmom -- $p-1$ method}\linkedone{factor.methods}{pmom}
  \func{pmom}
   {%
     \hiki{n}{integer},\ %
     **\param{options}
   }{%
     \out{\linkingone{factor.methods}{factorlist}}
   }\\
   \spacing
   % document of basic document
   \quad Factor the given integer \param{n} by \(p-1\) method.\\
   \spacing
   % added document
   \quad  The method may fail unless n has an appropriate factor for the method.\\
   (See \linkingone{factor.find}{pmom} of \linkingzero{factor.find} module.)\\

  \subsection{rhomethod -- $\rho$ method}\linkedone{factor.methods}{rhomethod}
  \func{rhomethod}
   {%
     \hiki{n}{integer},\ %
     **\param{options}
   }{%
     \out{\linkingone{factor.methods}{factorlist}}
   }\\
   \spacing
   % document of basic document
   \quad Factor the given integer \param{n} by Pollard's \(\rho\) method.\\
   \spacing
   % added document
   \quad The method is a probabilistic method, possibly fails in factorizations.\\
   (See \linkingone{factor.find}{rhomethod} of \linkingzero{factor.find} module.)\\
%
  \subsection{trialDivision -- trial division}\linkedone{factor.methods}{trialDivision}
  \func{trialDivision}
   {%
     \hiki{n}{integer},\ %
     **\param{options}
   }{%
     \out{\linkingone{factor.methods}{factorlist}}
   }\\
   \spacing
   \quad Factor the given integer \param{n} by trial division.\\
   \spacing
   \quad \param{options} for the trial sequence can be either:
   \begin{enumerate}
   \item \param{start} and \param{stop} as range parameters.
   \item \param{iterator} as an iterator of primes.
   \item \param{eratosthenes} as an upper bound to make prime sequence by sieve.
   \end{enumerate}
   If none of the options above are given, the function divides \param{n} by primes from \(2\) to the floor of the square root of \param{n} until a non-trivial factor is found.\\
   (See \linkingone{factor.find}{trialDivision} of \linkingzero{factor.find} module.)\\
%
\begin{ex}
>>> factor.methods.factor(10001)
[(73, 1), (137, 1)]
>>> factor.methods.ecm(1000001)
[(101L, 1), (9901L, 1)]
\end{ex}%Don't indent!(indent causes an error.)
\C

%---------- end document ---------- %

\bibliographystyle{jplain}%use jbibtex
\bibliography{nzmath_references}

\end{document}