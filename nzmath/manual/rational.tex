\documentclass{report}

%%%%%%%%%%%%%%%%%%%%%%%%%%%%%%%%%%%%%%%%%%%%%%%%%%%%%%%%%%%%%
%
% macros for nzmath manual
%
%%%%%%%%%%%%%%%%%%%%%%%%%%%%%%%%%%%%%%%%%%%%%%%%%%%%%%%%%%%%%
\usepackage{amssymb,amsmath}
\usepackage{color}
\usepackage[dvipdfm,bookmarks=true,bookmarksnumbered=true,%
 pdftitle={NZMATH Users Manual},%
 pdfsubject={Manual for NZMATH Users},%
 pdfauthor={NZMATH Development Group},%
 pdfkeywords={TeX; dvipdfmx; hyperref; color;},%
 colorlinks=true]{hyperref}
\usepackage{fancybox}
\usepackage[T1]{fontenc}
%
\newcommand{\DS}{\displaystyle}
\newcommand{\C}{\clearpage}
\newcommand{\NO}{\noindent}
\newcommand{\negok}{$\dagger$}
\newcommand{\spacing}{\vspace{1pt}\\ }
% software macros
\newcommand{\nzmathzero}{{\footnotesize $\mathbb{N}\mathbb{Z}$}\texttt{MATH}}
\newcommand{\nzmath}{{\nzmathzero}\ }
\newcommand{\pythonzero}{$\mbox{\texttt{Python}}$}
\newcommand{\python}{{\pythonzero}\ }
% link macros
\newcommand{\linkingzero}[1]{{\bf \hyperlink{#1}{#1}}}%module
\newcommand{\linkingone}[2]{{\bf \hyperlink{#1.#2}{#2}}}%module,class/function etc.
\newcommand{\linkingtwo}[3]{{\bf \hyperlink{#1.#2.#3}{#3}}}%module,class,method
\newcommand{\linkedzero}[1]{\hypertarget{#1}{}}
\newcommand{\linkedone}[2]{\hypertarget{#1.#2}{}}
\newcommand{\linkedtwo}[3]{\hypertarget{#1.#2.#3}{}}
\newcommand{\linktutorial}[1]{\href{http://docs.python.org/tutorial/#1}{#1}}
\newcommand{\linktutorialone}[2]{\href{http://docs.python.org/tutorial/#1}{#2}}
\newcommand{\linklibrary}[1]{\href{http://docs.python.org/library/#1}{#1}}
\newcommand{\linklibraryone}[2]{\href{http://docs.python.org/library/#1}{#2}}
\newcommand{\pythonhp}{\href{http://www.python.org/}{\python website}}
\newcommand{\nzmathwiki}{\href{http://nzmath.sourceforge.net/wiki/}{{\nzmathzero}Wiki}}
\newcommand{\nzmathsf}{\href{http://sourceforge.net/projects/nzmath/}{\nzmath Project Page}}
\newcommand{\nzmathtnt}{\href{http://tnt.math.metro-u.ac.jp/nzmath/}{\nzmath Project Official Page}}
% parameter name
\newcommand{\param}[1]{{\tt #1}}
% function macros
\newcommand{\hiki}[2]{{\tt #1}:\ {\em #2}}
\newcommand{\hikiopt}[3]{{\tt #1}:\ {\em #2}=#3}

\newdimen\hoge
\newdimen\truetextwidth
\newcommand{\func}[3]{%
\setbox0\hbox{#1(#2)}
\hoge=\wd0
\truetextwidth=\textwidth
\advance \truetextwidth by -2\oddsidemargin
\ifdim\hoge<\truetextwidth % short form
{\bf \colorbox{skyyellow}{#1(#2)\ $\to$ #3}}
%
\else % long form
\fcolorbox{skyyellow}{skyyellow}{%
   \begin{minipage}{\textwidth}%
   {\bf #1(#2)\\ %
    \qquad\quad   $\to$\ #3}%
   \end{minipage}%
   }%
\fi%
}

\newcommand{\out}[1]{{\em #1}}
\newcommand{\initialize}{%
  \paragraph{\large \colorbox{skyblue}{Initialize (Constructor)}}%
    \quad\\ %
    \vspace{3pt}\\
}
\newcommand{\method}{\C \paragraph{\large \colorbox{skyblue}{Methods}}}
% Attribute environment
\newenvironment{at}
{%begin
\paragraph{\large \colorbox{skyblue}{Attribute}}
\quad\\
\begin{description}
}%
{%end
\end{description}
}
% Operation environment
\newenvironment{op}
{%begin
\paragraph{\large \colorbox{skyblue}{Operations}}
\quad\\
\begin{table}[h]
\begin{center}
\begin{tabular}{|l|l|}
\hline
operator & explanation\\
\hline
}%
{%end
\hline
\end{tabular}
\end{center}
\end{table}
}
% Examples environment
\newenvironment{ex}%
{%begin
\paragraph{\large \colorbox{skyblue}{Examples}}
\VerbatimEnvironment
\renewcommand{\EveryVerbatim}{\fontencoding{OT1}\selectfont}
\begin{quote}
\begin{Verbatim}
}%
{%end
\end{Verbatim}
\end{quote}
}
%
\definecolor{skyblue}{cmyk}{0.2, 0, 0.1, 0}
\definecolor{skyyellow}{cmyk}{0.1, 0.1, 0.5, 0}
%
%\title{NZMATH User Manual\\ {\large{(for version 1.0)}}}
%\date{}
%\author{}
\begin{document}
%\maketitle
%
\setcounter{tocdepth}{3}
\setcounter{secnumdepth}{3}


\tableofcontents
\C

\chapter{Classes}

%---------- start document ---------- %
 \section{rational -- integer and rational number}\linkedzero{rational}
rational module provides integer and rational numbers, as class Rational, Integer, RationalField, and IntegerRing.

 \begin{itemize}
   \item {\bf Classes}
   \begin{itemize}
     \item \linkingone{rational}{Integer}
     \item \linkingone{rational}{IntegerRing}
     \item \linkingone{rational}{Rational}
     \item \linkingone{rational}{RationalField}
   \end{itemize}
 \end{itemize}

This module also provides following constants:
\begin{description}
   \item[theIntegerRing]\linkedone{rational}{theIntegerRing}:\\
     \param{theIntegerRing} is represents the ring of rational integers.
     An instance of \linkingone{rational}{IntegerRing}.
   \item[theRationalField]\linkedone{rational}{theRationalField}:\\
     \param{theRationalField} is represents the field of rational numbers.
     An instance of \linkingone{rational}{RationalField}.
 \end{description}

\C

 \subsection{Integer -- integer}\linkedone{rational}{Integer}
 Integer is a class of integer. Since 'int' and 'long' do not return rational for division, it is needed to create a new class.

 This class is a subclass of \linkingone{ring}{CommutativeRingElement} and long.

  \initialize
  \func{Integer}{\hiki{integer}{integer}}{\out{Integer}}\\
  \spacing
  % document of basic document
  \quad Construct a Integer object.
  % added document
  If argument is omitted, the value becomes 0.
  \method
  \subsubsection{getRing -- get ring object}\linkedtwo{rational}{Integer}{getRing}
   \func{getRing}{\param{self}}{\out{IntegerRing}}\\
   \spacing
   % document of basic document
   \quad Return an IntegerRing object.
%
  \subsubsection{actAdditive -- addition of binary addition chain}\linkedtwo{rational}{Integer}{actAdditive}
   \func{actAdditive}{\param{self},\ \hiki{other}{integer}}{\out{Integer}}\\
   \spacing
   % document of basic document
   \quad Act on other additively, i.e. n is expanded to n time additions of \param{other}. Naively, it is:

   \verb|return sum([+other for _ in range(self)])|

   but, here we use a binary addition chain.\\
   \spacing
%
  \subsubsection{actMultiplicative -- multiplication of binary addition chain}\linkedtwo{rational}{Integer}{actMultiplicative}
   \func{actMultiplicative}{\param{self},\ \hiki{other}{integer}}{\out{Integer}}\\
   \spacing
   % document of basic document
   \quad Act on other multiplicatively, i.e. n is expanded to n time multiplications of \param{other}. Naively, it is:

\verb|return reduce(lambda x,y: x*y, [+other for _ in range(self)])|

   but, here we use a binary addition chain.
   \spacing
\C
 \subsection{IntegerRing -- integer ring}\linkedone{rational}{IntegerRing}
 The class is for the ring of rational integers.

 This class is a subclass of \linkingone{ring}{CommutativeRing}.


  \initialize
  \func{IntegerRing}{}{\out{IntegerRing}}\\
  \spacing
  % document of basic document
  \quad Create an instance of IntegerRing. 
  % added document
  You may not want to create an instance, since there is already theIntegerRing.\\
  \begin{at}
    \item[zero]\linkedtwo{integer}{IntegerRing}{zero}:\\ It expresses the additive unit 0. (read only)
    \item[one]\linkedtwo{integer}{IntegerRing}{one}:\\ It expresses the multiplicative unit 1. (read only)
  \end{at}
  \begin{op}
    \verb|x in Z| & return whether an element is in or not.\\
    \verb|repr(Z)| & return representation string.\\
    \verb|str(Z)| & return string.\\
  \end{op}
  \method
  \subsubsection{createElement -- create Integer object}\linkedtwo{rational}{IntegerRing}{createElement}
   \func{createElement}{\param{self},\ \hiki{seed}{integer}}{\out{Integer}}\\
   \spacing
   % document of basic document
   \quad Return an Integer object with \param{seed}. 
   \spacing
   % input, output document
   \quad \param{seed} must be int, long or rational.Integer.\\
%
  \subsubsection{gcd -- greatest common divisor}\linkedtwo{rational}{IntegerRing}{gcd}
   \func{gcd}{\param{self},\ \hiki{n}{integer},\ \hiki{m}{integer}}{\out{Integer}}\\
   \spacing
   % document of basic document
   \quad Return the greatest common divisor of given 2 integers.\\
%
  \subsubsection{extgcd -- extended GCD}\linkedtwo{rational}{IntegerRing}{extgcd}
   \func{extgcd}{\param{self},\ \hiki{n}{integer},\ \hiki{m}{integer}}{\out{Integer}}\\
   \spacing
   % document of basic document
   \quad Return a tuple ($u$, $v$, $d$); they are the greatest common divisor $d$ of two given integers \param{n} and \param{m} and $u$, $v$ such that $d = \mathtt{n}u + \mathtt{m}v$.\\
   \spacing
%
  \subsubsection{lcm -- lowest common multiplier}\linkedtwo{rational}{IntegerRing}{lcm}
   \func{lcm}{\param{self},\ \hiki{n}{integer},\ \hiki{m}{integer}}{\out{Integer}}\\
   \spacing
   % document of basic document
   \quad Return the lowest common multiple of given 2 integers. 
   % added document
   \quad If both are zero, it raises an exception.\\
%
  \subsubsection{getQuotientField -- get rational field object}\linkedtwo{rational}{IntegerRing}{getQuotientField}
   \func{getQuotientField}{\param{self}}{\out{RationalField}}\\
   \spacing
   % document of basic document
   \quad Return the rational field (\linkingone{rational}{RationalField}).\\
%
  \subsubsection{issubring -- subring test}\linkedtwo{rational}{IntegerRing}{issubring}
   \func{issubring}{\param{self},\ \hiki{other}{\linkingone{ring}{Ring}}}{\out{bool}}\\
   \spacing
   % document of basic document
   \quad Report whether another ring contains the integer ring as subring.

   If other is also the integer ring, the output is True. In other cases it depends on the implementation of another ring's issuperring method.\\
   \spacing
%
  \subsubsection{issuperring -- superring test}\linkedtwo{rational}{IntegerRing}{issuperring}
   \func{issuperring}{\param{self},\ \hiki{other}{\linkingone{ring}{Ring}}}{\out{bool}}\\
   \spacing
   % document of basic document
   \quad Report whether the integer ring contains another ring as subring.

If other is also the integer ring, the output is True. In other cases it depends on the implementation of another ring's issubring method.\\
   \spacing

\C
 \subsection{Rational -- rational number}\linkedone{rational}{Rational}
 The class of rational numbers.

  \initialize
  \func{Rational}
       {\hiki{numerator}{numbers},\ 
         \hikiopt{denominator}{numbers}{1}}
       {\out{Integer}}\\
  \spacing
  % document of basic document
  \quad Construct a rational number from:
  \begin{itemize}
  \item integers,
  \item float, or
  \item Rational.
  \end{itemize}
  % added document
  Other objects can be converted if they have toRational methods. Otherwise raise TypeError.

  \method
  \subsubsection{getRing -- get ring object}\linkedtwo{rational}{Rational}{getRing}
   \func{getRing}{\param{self}}{\out{RationalField}}\\
   \spacing
   % document of basic document
   \quad Return a RationalField object.
%
  \subsubsection{decimalString -- represent decimal}\linkedtwo{rational}{Rational}{decimalString}
   \func{decimalString}{\param{self},\ \hiki{N}{integer}}{\out{string}}\\
   \spacing
   % document of basic document
   \quad Return a string of the number to \param{N} decimal places.
   \spacing
%
  \subsubsection{expand -- continued-fraction representation}\linkedtwo{rational}{Rational}{expand}
   \func{expand}
        {\param{self},\ 
          \hiki{base}{integer},\ 
          \hiki{limit}{integer}}
        {\out{string}}\\
   \spacing
   % document of basic document
   \quad Return the nearest rational number whose denominator is a power of \param{base} and at most \param{limit} if \param{base} is positive integer.

   Otherwise, i.e. \param{base}=0, returns the nearest rational number whose denominator is at most \param{limit}.
   \spacing
   % input, output document
   \quad \param{base} must be non-negative integer.\\
%
\C
 \subsection{RationalField -- the rational field}\linkedone{rational}{RationalField}
RationalField is a class of field of rationals. The class has the single instance \linkingone{rational}{theRationalField}.

 This class is a subclass of \linkingone{ring}{QuotientField}.


  \initialize
  \func{RationalField}{}{\out{RationalField}}\\
  \spacing
  % document of basic document
  \quad Create an instance of RationalField. 
  % added document
  You may not want to create an instance, since there is already theRationalField.
  \begin{at}
    \item[zero]\linkedtwo{integer}{RationalField}{zero}:\\ It expresses the additive unit 0, namely Rational(0, 1). (read only)
    \item[one]\linkedtwo{integer}{RationalField}{one}:\\ It expresses the multiplicative unit 1, namely Rational(1, 1). (read only)
  \end{at}
  \begin{op}
    \verb|x in Q| & return whether an element is in or not.\\
    \verb|str(Q)| & return string.\\
  \end{op}
  \method
  \subsubsection{createElement -- create Rational object}\linkedtwo{rational}{RationalField}{createElement}
   \func{createElement}
        {\param{self},\ 
          \hiki{numerator}{integer or \linkingone{rational}{Rational}},\ 
          \hikiopt{denominator}{integer}{1} 
        }{\out{Rational}}\\
   \spacing
   % document of basic document
   \quad Create a Rational object.
%
  \subsubsection{classNumber -- get class number}\linkedtwo{rational}{RationalField}{classNumber}
   \func{classNumber}{\param{self}}{\out{integer}}\\
   \spacing
   % document of basic document
   \quad Return 1, since the class number of the rational field is one.
%
  \subsubsection{getQuotientField -- get rational field object}\linkedtwo{rational}{RationalField}{getQuotientField}
   \func{getQuotientField}{\param{self}}{\out{RationalField}}\\
   \spacing
   % document of basic document
   \quad Return the rational field itself.
%
  \subsubsection{issubring -- subring test}\linkedtwo{rational}{RationalField}{issubring}
   \func{issubring}{\param{self},\ \hiki{other}{\linkingone{ring}{Ring}}}{\out{bool}}\\
   \spacing
   % document of basic document
   \quad Report whether another ring contains the rational field as subring.

   If other is also the rational field, the output is True. In other cases it depends on the implementation of another ring's issuperring method.
   \spacing
%
  \subsubsection{issuperring -- superring test}\linkedtwo{rational}{RationalField}{issuperring}
   \func{issuperring}{\param{self},\ \hiki{other}{\linkingone{ring}{Ring}}}{\out{bool}}\\
   \spacing
   % document of basic document
   \quad Report whether the rational field contains another ring as subring.

If other is also the rational field, the output is True. In other cases it depends on the implementation of another ring's issubring method.
   \spacing
\C

%---------- end document ---------- %

\bibliographystyle{jplain}
\bibliography{nzmath_references}

\end{document}
