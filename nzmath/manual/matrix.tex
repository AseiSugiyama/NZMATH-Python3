\documentclass{report}

%%%%%%%%%%%%%%%%%%%%%%%%%%%%%%%%%%%%%%%%%%%%%%%%%%%%%%%%%%%%%
%
% macros for nzmath manual
%
%%%%%%%%%%%%%%%%%%%%%%%%%%%%%%%%%%%%%%%%%%%%%%%%%%%%%%%%%%%%%
\usepackage{amssymb,amsmath}
\usepackage{color}
\usepackage[dvipdfm,bookmarks=true,bookmarksnumbered=true,%
 pdftitle={NZMATH Users Manual},%
 pdfsubject={Manual for NZMATH Users},%
 pdfauthor={NZMATH Development Group},%
 pdfkeywords={TeX; dvipdfmx; hyperref; color;},%
 colorlinks=true]{hyperref}
\usepackage{fancybox}
\usepackage[T1]{fontenc}
%
\newcommand{\DS}{\displaystyle}
\newcommand{\C}{\clearpage}
\newcommand{\NO}{\noindent}
\newcommand{\negok}{$\dagger$}
\newcommand{\spacing}{\vspace{1pt}\\ }
% software macros
\newcommand{\nzmathzero}{{\footnotesize $\mathbb{N}\mathbb{Z}$}\texttt{MATH}}
\newcommand{\nzmath}{{\nzmathzero}\ }
\newcommand{\pythonzero}{$\mbox{\texttt{Python}}$}
\newcommand{\python}{{\pythonzero}\ }
% link macros
\newcommand{\linkingzero}[1]{{\bf \hyperlink{#1}{#1}}}%module
\newcommand{\linkingone}[2]{{\bf \hyperlink{#1.#2}{#2}}}%module,class/function etc.
\newcommand{\linkingtwo}[3]{{\bf \hyperlink{#1.#2.#3}{#3}}}%module,class,method
\newcommand{\linkedzero}[1]{\hypertarget{#1}{}}
\newcommand{\linkedone}[2]{\hypertarget{#1.#2}{}}
\newcommand{\linkedtwo}[3]{\hypertarget{#1.#2.#3}{}}
\newcommand{\linktutorial}[1]{\href{http://docs.python.org/tutorial/#1}{#1}}
\newcommand{\linktutorialone}[2]{\href{http://docs.python.org/tutorial/#1}{#2}}
\newcommand{\linklibrary}[1]{\href{http://docs.python.org/library/#1}{#1}}
\newcommand{\linklibraryone}[2]{\href{http://docs.python.org/library/#1}{#2}}
\newcommand{\pythonhp}{\href{http://www.python.org/}{\python website}}
\newcommand{\nzmathwiki}{\href{http://nzmath.sourceforge.net/wiki/}{{\nzmathzero}Wiki}}
\newcommand{\nzmathsf}{\href{http://sourceforge.net/projects/nzmath/}{\nzmath Project Page}}
\newcommand{\nzmathtnt}{\href{http://tnt.math.se.tmu.ac.jp/nzmath/}{\nzmath Project Official Page}}
% parameter name
\newcommand{\param}[1]{{\tt #1}}
% function macros
\newcommand{\hiki}[2]{{\tt #1}:\ {\em #2}}
\newcommand{\hikiopt}[3]{{\tt #1}:\ {\em #2}=#3}

\newdimen\hoge
\newdimen\truetextwidth
\newcommand{\func}[3]{%
\setbox0\hbox{#1(#2)}
\hoge=\wd0
\truetextwidth=\textwidth
\advance \truetextwidth by -2\oddsidemargin
\ifdim\hoge<\truetextwidth % short form
{\bf \colorbox{skyyellow}{#1(#2)\ $\to$ #3}}
%
\else % long form
\fcolorbox{skyyellow}{skyyellow}{%
   \begin{minipage}{\textwidth}%
   {\bf #1(#2)\\ %
    \qquad\quad   $\to$\ #3}%
   \end{minipage}%
   }%
\fi%
}

\newcommand{\out}[1]{{\em #1}}
\newcommand{\initialize}{%
  \paragraph{\large \colorbox{skyblue}{Initialize (Constructor)}}%
    \quad\\ %
    \vspace{3pt}\\
}
\newcommand{\method}{\C \paragraph{\large \colorbox{skyblue}{Methods}}}
% Attribute environment
\newenvironment{at}
{%begin
\paragraph{\large \colorbox{skyblue}{Attribute}}
\quad\\
\begin{description}
}%
{%end
\end{description}
}
% Operation environment
\newenvironment{op}
{%begin
\paragraph{\large \colorbox{skyblue}{Operations}}
\quad\\
\begin{table}[h]
\begin{center}
\begin{tabular}{|l|l|}
\hline
operator & explanation\\
\hline
}%
{%end
\hline
\end{tabular}
\end{center}
\end{table}
}
% Examples environment
\newenvironment{ex}%
{%begin
\paragraph{\large \colorbox{skyblue}{Examples}}
\VerbatimEnvironment
\renewcommand{\EveryVerbatim}{\fontencoding{OT1}\selectfont}
\begin{quote}
\begin{Verbatim}
}%
{%end
\end{Verbatim}
\end{quote}
}
%
\definecolor{skyblue}{cmyk}{0.2, 0, 0.1, 0}
\definecolor{skyyellow}{cmyk}{0.1, 0.1, 0.5, 0}
%
%\title{NZMATH User Manual\\ {\large{(for version 1.0)}}}
%\date{}
%\author{}
\begin{document}
%\maketitle
%
\setcounter{tocdepth}{3}
\setcounter{secnumdepth}{3}


\tableofcontents
\C

\chapter{Classes}


%---------- start document ---------- %
 \section{matrix -- matrices}\linkedzero{matrix}
 \begin{itemize}
   \item {\bf Classes}
   \begin{itemize}
     \item \linkingone{matrix}{Matrix}
     \item \linkingone{matrix}{SquareMatrix}
     \item \linkingone{matrix}{RingMatrix}
     \item \linkingone{matrix}{RingSquareMatrix}
     \item \linkingone{matrix}{FieldMatrix}
     \item \linkingone{matrix}{FieldSquareMatrix}
     \item \linkingone{matrix}{MatrixRing}
     \item \linkingone{matrix}{Subspace}
   \end{itemize}
   \item {\bf Functions}
     \begin{itemize}
       \item \linkingone{matrix}{createMatrix}
       \item \linkingone{matrix}{identityMatrix}
       \item \linkingone{matrix}{unitMatrix}
       \item \linkingone{matrix}{zeroMatrix}
     \end{itemize}
 \end{itemize}
 
 The module matrix has also some exception classes.
 \begin{description}
   \item[MatrixSizeError]\linkedone{matrix}{MatrixSizeError}:
     Report contradicting given input to the matrix size.
   \item[VectorsNotIndependent]\linkedone{matrix}{VectorsNotIndependent}:
     Report column vectors are not independent. 
   \item[NoInverseImage]\linkedone{matrix}{NoInverseImage}:
     Report any inverse image does not exist.
   \item[NoInverse]\linkedone{matrix}{NoInverse}:
     Report the matrix is not invertible.
 \end{description}

 This module using following type:
 \begin{description}
   \item[compo]\linkedone{matrix}{compo}:
     \param{compo} must be one of these forms below.\\
     \begin{itemize}
       \item concatenated row lists, such as {\it [1,2]+[3,4]+[5,6]}.
       \item list of row lists, such as {\it [[1,2], [3,4], [5,6]]}.
       \item list of column tuples, such as {\it [(1, 3, 5), (2, 4, 6)]}.
       \item list of vectors whose dimension equals column, such as {\it vector.Vector([1, 3, 5]), vector.Vector([2, 4, 6])}.
     \end{itemize}
     The examples above represent the same matrix form as follows:
     \begin{equation*}
     \begin{array}{cc}
       1 & 2\\
       3 & 4\\
       5 & 6\\
     \end{array}
     \end{equation*}
 \end{description}

\C

 \subsection{Matrix -- matrices}\linkedone{matrix}{Matrix}
 \initialize
  \func{Matrix}{\hiki{row}{integer},\ \hiki{column}{integer},\ \hikiopt{compo}{compo}{0},\ \hikiopt{coeff\_ring}{CommutativeRing}{0}}{\out{Matrix}}\\
  \spacing
  % document of basic document
  \quad Create new matrices object.\\
  \spacing
  % added document
  \quad \negok This constructor automatically changes the class to one of the following class: \linkingone{matrix}{RingMatrix},\ \linkingone{matrix}{RingSquareMatrix},\ \linkingone{matrix}{FieldMatrix},\ \linkingone{matrix}{FieldSquareMatrix}.\\
  \spacing
  % input, output document
  \quad Your input determines the class automatically by examining the matrix size and the coefficient ring.
   \param{row} and \param{column} must be integer, and \param{coeff\_ring} must be an instance of \linkingone{ring}{Ring}.
   Refer to \linkingone{matrix}{compo} for information about \param{compo}.
   If you abbreviate \param{compo}, it will be deemed to all zero list.\\
   The list of expected inputs and outputs is as following:
   \begin{itemize}
     \item Matrix(\param{row},\ \param{column},\ \param{compo},\ \param{coeff\_ring})\\
       $\to$ the \param{row}$\times$\param{column} matrix whose elements are \param{compo} and coefficient ring is \param{coeff\_ring}
     \item Matrix(\param{row},\ \param{column},\ \param{compo})\\
       $\to$ the \param{row}$\times$\param{column} matrix whose elements are \param{compo} (The coefficient ring is automatically determined.)
     \item Matrix(\param{row},\ \param{column},\ \param{coeff\_ring})\\
       $\to$ the \param{row}$\times$\param{column} matrix whose coefficient ring is \param{coeff\_ring} (All elements are $0$ in \param{coeff\_ring}.)
     \item Matrix(\param{row},\ \param{column})\\
       $\to$ the \param{row}$\times$\param{column} matrix (The coefficient matrix is \linkingone{rational}{Integer}. All elements are $0$.)
   \end{itemize}
  \begin{at}
    \item[row]\linkedtwo{matrix}{Matrix}{row}: The row size of the matrix.\\
    \item[column]\linkedtwo{matrix}{Matrix}{column}: The column size of the matrix.\\
    \item[coeff\_ring]\linkedtwo{matrix}{Matrix}{coeff\_ring}: The coefficient ring of the matrix.\\
    \item[compo]\linkedtwo{matrix}{Matrix}{compo}: The elements of the matrix.\\
  \end{at}
  \begin{op}
    \verb+M==N+ & Return whether \param{M} and \param{N} are equal or not.\\
    \verb+M[i, j]+ & Return the coefficient of \param{i}-th row, \param{j}-th column term of matrix \param{M}.\\
    \verb+M[i]+ & Return the vector of \param{i}-th column term of matrix \param{M}.\\
    \verb+M[i, j]=c+ & Replace the coefficient of \param{i}-th row, \param{j}-th column term of matrix \param{M} by \param{c}.\\
    \verb+M[j]=c+ & Replace the vector of \param{i}-th column term of matrix \param{M} by vector \param{c}.\\
    \verb+c in M+ & Check whether some element of \param{M} equals \param{c}.\\
    \verb+repr(M)+ & Return the repr string of the matrix \param{M}.\\
                   & string represents list concatenated row vector lists.\\
    \verb+str(M)+ & Return the str string of the matrix \param{M}.\\
  \end{op}
\begin{ex}
>>> A = matrix.Matrix(2, 3, [1,0,0]+[0,0,0])
>>> A.__class__.__name__
'RingMatrix'
>>> B = matrix.Matrix(2, 3, [1,0,0,0,0,0])
>>> A == B
True
>>> B[1, 1] = 0
>>> A != B
True
>>> B == 0
True
>>> A[1, 1]
1
>>> print repr(A)
[1, 0, 0]+[0, 0, 0]
>>> print str(A)
1 0 0
0 0 0
\end{ex}%Don't indent!
  \method
\subsubsection{map -- apply function to elements}\linkedtwo{matrix}{Matrix}{map}
   \func{map}{\param{self},\ \hiki{function}{function}}{\out{Matrix}}\\
   \spacing
   % document of basic document
   \quad Return the matrix whose elements is applied \param{function} to.\\
   \spacing
   % added document
   \quad \negok The function map is an analogy of built-in function \href{http://docs.python.org/library/functions.html#map}{map}.\\
   \spacing
   % input, output document
   %\quad \param{a} must be int, long or rational.Integer.\\
 \subsubsection{reduce -- reduce elements iteratively}\linkedtwo{matrix}{Matrix}{reduce}
   \func{reduce}{\param{self},\ \hiki{function}{function},\ \hikiopt{initializer}{RingElement}{None}}{\out{RingElement}}\\
   \spacing
   % document of basic document
   \quad Apply \param{function} from upper-left to lower-right, so as to reduce the iterable to a single value.\\
   \spacing
   % added document
   \quad \negok The function map is an analogy of built-in function \href{http://docs.python.org/library/functions.html#reduce}{reduce}.\\
   \spacing
   % input, output document
   %\quad \param{a} must be int, long or rational.Integer.\\
 \subsubsection{copy -- create a copy}\linkedtwo{matrix}{Matrix}{copy}
   \func{copy}{\param{self}}{\out{Matrix}}\\
   \spacing
   % document of basic document
   \quad create a copy of \param{self}.\\
   \spacing
   % added document
   \quad \negok The matrix generated by the function is same matrix to \param{self}, but not same as a instance.\\
   \spacing
   % input, output document
   %\quad \param{a} must be int, long or rational.Integer.\\
  \subsubsection{set -- set compo}\linkedtwo{matrix}{Matrix}{set}
   \func{set}{\param{self},\ \hiki{compo}{compo}}{\out{(None)}}\\
   \spacing
   % document of basic document
   \quad Substitute the list \param{compo} for \linkingtwo{matrix}{Matrix}{compo}.\\
   \spacing
   % added document
   %\quad\\
   %\spacing
   % input, output document
   \quad \param{compo} must be the form of \linkingtwo{matrix}{Matrix}{compo}.\\
  \subsubsection{setRow -- set m-th row vector}\linkedtwo{matrix}{Matrix}{setRow}
   \func{setRow}{\param{self},\ \hiki{m}{integer},\ \hiki{arg}{list/Vector}}{\out{(None)}}\\
   \spacing
   % document of basic document
   \quad Substitute the list/Vector \param{arg} as \param{m}-th row.\\
   \spacing
   % added document
   %\quad\\
   %\spacing
   % input, output document
   \quad The length of \param{arg} must be same to \param{self}.\linkingtwo{matrix}{Matrix}{column}.\\
  \subsubsection{setColumn -- set n-th column vector}\linkedtwo{matrix}{Matrix}{setColumn}
   \func{setColumn}{\param{self},\ \hiki{n}{integer},\ \hiki{arg}{list/Vector}}{\out{(None)}}\\
   \spacing
   % document of basic document
   \quad Substitute the list/Vector \param{arg} as \param{n}-th column.\\
   \spacing
   % added document
   %\quad\\
   %\spacing
   % input, output document
   \quad The length of \param{arg} must be same to \param{self}.\linkingtwo{matrix}{Matrix}{row}.\\
  \subsubsection{getRow -- get i-th row vector}\linkedtwo{matrix}{Matrix}{getRow}
   \func{getRow}{\param{self},\ \hiki{i}{integer}}{\out{Vector}}\\
   \spacing
   % document of basic document
   \quad Return \param{i}-th row in form of \param{self}.\\
   \spacing
   % added document
   %\quad\\
   %\spacing
   % input, output document
   \quad The function returns a row vector (an instance of \linkingone{vector}{Vector}).\\
  \subsubsection{getColumn -- get j-th column vector}\linkedtwo{matrix}{Matrix}{getColumn}
   \func{getColumn}{\param{self},\ \hiki{j}{integer}}{\out{Vector}}\\
   \spacing
   % document of basic document
   \quad Return \param{j}-th column in form of \param{self}.\\
   \spacing
   % added document
   %\quad\\
   %\spacing
   % input, output document
  \subsubsection{swapRow -- swap two row vectors}\linkedtwo{matrix}{Matrix}{swapRow}
   \func{swapRow}{\param{self},\ \hiki{m1}{integer},\ \hiki{m2}{integer}}{\out{(None)}}\\
   \spacing
   % document of basic document
   \quad Swap \param{self}'s \param{m1}-th row vector and \param{m2}-th row one.\\
   \spacing
   % added document
   %\quad\\
   %\spacing
   % input, output document
  \subsubsection{swapColumn -- swap two column vectors}\linkedtwo{matrix}{Matrix}{swapColumn}
   \func{swapColumn}{\param{self},\ \hiki{n1}{integer},\ \hiki{n2}{integer}}{\out{(None)}}\\
   \spacing
   % document of basic document
   \quad Swap \param{self}'s \param{n1}-th column vector and \param{n2}-th column one.\\
   \spacing
   % added document
   %\quad\\
   %\spacing
   % input, output document
  \subsubsection{insertRow -- insert row vectors}\linkedtwo{matrix}{Matrix}{insertRow}
   \func{insertRow}{\param{self},\ \hiki{i}{integer},\ \hiki{arg}{list/Vector/Matrix}}{\out{(None)}}\\
   \spacing
   % document of basic document
   \quad Insert row vectors \param{arg} to \param{i}-th \param{row}.\\
   \spacing
   % added document
   %\quad\\
   %\spacing
   % input, output document
   \param{arg} must be list, \linkingone{vector}{Vector} or \linkingone{matrix}{Matrix}.
    The length (or \linkingtwo{matrix}{Matrix}{column}) of it should be same to the column of \param{self}. 
  \subsubsection{insertColumn -- insert column vectors}\linkedtwo{matrix}{Matrix}{insertColumn}
   \func{insertColumn}{\param{self},\ \hiki{j}{integer},\ \hiki{arg}{list/Vector/Matrix}}{\out{(None)}}\\
   \spacing
   % document of basic document
   \quad Insert column vectors \param{arg} to \param{j}-th \param{column}.\\
   \spacing
   % added document
   %\quad\\
   %\spacing
   % input, output document
   \param{arg} must be list, \linkingone{vector}{Vector} or \linkingone{matrix}{Matrix}.
    The length (or \linkingtwo{matrix}{Matrix}{row}) of it should be same to the row of \param{self}. 
  \subsubsection{extendRow -- extend row vectors}\linkedtwo{matrix}{Matrix}{extendRow}
   \func{extendRow}{\param{self},\ \hiki{arg}{list/Vector/Matrix}}{\out{(None)}}\\
   \spacing
   % document of basic document
   \quad Join \param{self} with row vectors \param{arg} (in vertical way).\\
   \spacing
   % added document
   \quad The function combines \param{self} with the last row vector of \param{self}.
   That is, extendRow(\param{arg}) is same to insertRow(\param{self}.\param{row}+1,\ \param{arg}).\\
   \spacing
   % input, output document
   \param{arg} must be list, \linkingone{vector}{Vector} or \linkingone{matrix}{Matrix}.
    The length (or \linkingtwo{matrix}{Matrix}{column}) of it should be same to the column of \param{self}. 
  \subsubsection{extendColumn -- extend column vectors}\linkedtwo{matrix}{Matrix}{extendColumn}
   \func{extendColumn}{\param{self},\ \hiki{arg}{list/Vector/Matrix}}{\out{(None)}}\\
   \spacing
   % document of basic document
   \quad Join \param{self} with column vectors \param{arg} (in horizontal way).\\
   \spacing
   % added document
   \quad The function combines \param{self} with the last column vector of \param{self}.
   That is, extendColumn(\param{arg}) is same to insertColumn(\param{self}.\param{column}+1,\ \param{arg}).\\
   \spacing
   % input, output document
   \param{arg} must be list, \linkingone{vector}{Vector} or \linkingone{matrix}{Matrix}.
    The length (or \linkingtwo{matrix}{Matrix}{row}) of it should be same to the row of \param{self}.
  \subsubsection{deleteRow -- delete row vector}\linkedtwo{matrix}{Matrix}{deleteRow}
   \func{deleteRow}{\param{self},\ \hiki{i}{integer}}{\out{(None)}}\\
   \spacing
   % document of basic document
   \quad Delete \param{i}-th row vector.\\
   \spacing
   % added document
   %\quad 
   %\spacing
   % input, output document
  \subsubsection{deleteColumn -- delete column vector}\linkedtwo{matrix}{Matrix}{deleteColumn}
   \func{deleteColumn}{\param{self},\ \hiki{j}{integer}}{\out{(None)}}\\
   \spacing
   % document of basic document
   \quad Delete \param{j}-th column vector.\\
   \spacing
   % added document
   %\quad 
   %\spacing
   % input, output document
  \subsubsection{transpose -- transpose matrix}\linkedtwo{matrix}{Matrix}{transpose}
   \func{transpose}{\param{self}}{\out{Matrix}}\\
   \spacing
   % document of basic document
   \quad Return the transpose of \param{self}.\\
   \spacing
   % added document
   %\quad 
   %\spacing
   % input, output document
  \subsubsection{getBlock -- block matrix}\linkedtwo{matrix}{Matrix}{getBlock}
   \func{getBlock}{\param{self},\ \hiki{i}{integer},\ \hiki{j}{integer},\ \hiki{row}{integer},\ \hikiopt{column}{integer}{None}}{\out{Matrix}}\\
   \spacing
   % document of basic document
   \quad Return the \param{row}$\times$\param{column} block matrix from the (\param{i},\ \param{j})-element. \\
   \spacing
   % added document
   %\quad 
   %\spacing
   % input, output document
   If \param{column} is omitted, \param{column} is considered as same value to \param{row}.
  \subsubsection{subMatrix -- submatrix}\linkedtwo{matrix}{Matrix}{subMatrix}
   \func{subMatrix}{\param{self},\ \hiki{I}{integer},\ \hiki{J}{integer}{None}}{\out{Matrix}}\\
   \func{subMatrix}{\param{self},\ \hiki{I}{list},\ \hikiopt{J}{list}{None}}{\out{Matrix}}\\
   \spacing
   % document of basic document
   \quad The function has a twofold significance.
   \begin{itemize}
     \item \param{I} and \param{J} are integer:\\
       Return submatrix deleted \param{I}-th row and \param{J}-th column.
     \item \param{I} and \param{J} are list:\\
       Return the submatrix composed of elements from \param{self} assigned by rows I and columns J, respectively. 
   \end{itemize}
   \quad\\
   \spacing
   % added document
   %\quad 
   %\spacing
   % input, output document
   \quad If \param{J} is omitted, \param{J} is considered as same value to \param{I}.
\begin{ex}
>>> A = matrix.Matrix(2, 3, [1,2,3]+[4,5,6])
>>> A
[1, 2, 3]+[4, 5, 6]
>>> A.map(complex)
[(1+0j), (2+0j), (3+0j)]+[(4+0j), (5+0j), (6+0j)]
>>> A.reduce(max)
6
>>> A.swapRow(1, 2)
>>> A
[4, 5, 6]+[1, 2, 3]
>>> A.extendColumn([-2, -1])
>>> A
[4, 5, 6, -2]+[1, 2, 3, -1]
>>> B = matrix.Matrix(3, 3, [1,2,3]+[4,5,6]+[7,8,9])
>>> B.subMatrix(2, 3)
[1, 2]+[7, 8]
>>> B.subMatrix([2, 3], [1, 2])
[4, 5]+[7, 8]
\end{ex}%Don't indent!
\C

\subsection{SquareMatrix -- square matrices}\linkedone{matrix}{SquareMatrix}
 \initialize
  \func{SquareMatrix}{\hiki{row}{integer},\ \hikiopt{column}{integer}{0},\ \hikiopt{compo}{compo}{0},\ \hikiopt{coeff\_ring}{CommutativeRing}{0}}{\out{SquareMatrix}}\\
  \spacing
  % document of basic document
  \quad Create new square matrices object.\\
  \spacing
  % added document
  \quad SquareMatrix is subclass of \linkingone{matrix}{Matrix}.
  \negok This constructor automatically changes the class to one of the following class: \linkingone{matrix}{RingMatrix},\ \linkingone{matrix}{RingSquareMatrix},\ \linkingone{matrix}{FieldMatrix},\ \linkingone{matrix}{FieldSquareMatrix}.\\
  \spacing
  % input, output document
  \quad Your input determines the class automatically by examining the matrix size and the coefficient ring.
   \param{row} and \param{column} must be integer, and \param{coeff\_ring} must be an instance of \linkingone{ring}{Ring}.
   Refer to \linkingone{matrix}{compo} for information about \param{compo}.
   If you abbreviate \param{compo}, it will be deemed to all zero list.\\
.\\
   The list of expected inputs and outputs is as following:
   \begin{itemize}
     \item Matrix(\param{row},\ \param{compo},\ \param{coeff\_ring})\\
       $\to$ the \param{row} square matrix whose elements are \param{compo} and coefficient ring is \param{coeff\_ring}
     \item Matrix(\param{row},\ \param{compo})\\
       $\to$ the \param{row} square matrix whose elements are \param{compo} (coefficient ring is automatically determined)
     \item Matrix(\param{row},\ \param{coeff\_ring})\\
       $\to$ the \param{row} square matrix whose coefficient ring is \param{coeff\_ring} (All elements are $0$ in \param{coeff\_ring}.)
     \item Matrix(\param{row})\\
       $\to$ the \param{row} square matrix (The coefficient ring is Integer. All elements are $0$.)
   \end{itemize}
   \negok We can initialize as  \linkingone{matrix}{Matrix}, but \param{column} must be same to \param{row} in the case.
\method
\subsubsection{isUpperTriangularMatrix -- check upper triangular}\linkedtwo{matrix}{SquareMatrix}{isUpperTriangularMatrix}
   \func{isUpperTriangularMatrix}{\param{self}}{\out{True/False}}\\
   \spacing
   % document of basic document
   \quad Check whether \param{self} is upper triangular matrix or not.\\
   \spacing
   % added document
   %\quad 
   %\spacing
   % input, output document
   %\quad \param{a} must be int, long or rational.Integer.\\
\subsubsection{isLowerTriangularMatrix -- check lower triangular}\linkedtwo{matrix}{SquareMatrix}{isLowerTriangularMatrix}
   \func{isLowerTriangularMatrix}{\param{self}}{\out{True/False}}\\
   \spacing
   % document of basic document
   \quad Check whether \param{self} is lower triangular matrix or not.\\
   \spacing
   % added document
   %\quad 
   %\spacing
   % input, output document
   %\quad \param{a} must be int, long or rational.Integer.\\
\subsubsection{isDiagonalMatrix -- check diagonal matrix}\linkedtwo{matrix}{SquareMatrix}{isDiagonalMatrix}
   \func{isDiagonalMatrix}{\param{self}}{\out{True/False}}\\
   \spacing
   % document of basic document
   \quad Check whether \param{self} is diagonal matrix or not.\\
   \spacing
   % added document
   %\quad 
   %\spacing
   % input, output document
   %\quad \param{a} must be int, long or rational.Integer.\\
\subsubsection{isScalarMatrix -- check scalar matrix}\linkedtwo{matrix}{SquareMatrix}{isScalarMatrix}
   \func{isScalarMatrix}{\param{self}}{\out{True/False}}\\
   \spacing
   % document of basic document
   \quad Check whether \param{self} is scalar matrix or not.\\
   \spacing
   % added document
   %\quad 
   %\spacing
   % input, output document
   %\quad \param{a} must be int, long or rational.Integer.\\
\subsubsection{isSymmetricMatrix -- check symmetric matrix}\linkedtwo{matrix}{SquareMatrix}{isSymmetricMatrix}
   \func{isSymmetricMatrix}{\param{self}}{\out{True/False}}\\
   \spacing
   % document of basic document
   \quad Check whether \param{self} is symmetric matrix or not.\\
   \spacing
   % added document
   %\quad 
   %\spacing
   % input, output document
   %\quad \param{a} must be int, long or rational.Integer.\\
\begin{ex}
>>> A = matrix.SquareMatrix(3, [1,2,3]+[0,5,6]+[0,0,9])
>>> A.isUpperTriangularMatrix()
True
>>> B = matrix.SquareMatrix(3, [1,0,0]+[0,-2,0]+[0,0,7])
>>> B.isDiagonalMatrix()
True
\end{ex}%Don't indent!
\C

\subsection{RingMatrix -- matrix whose elements belong ring}\linkedone{matrix}{RingMatrix}
  \func{RingMatrix}{\hiki{row}{integer},\ \hiki{column}{integer},\ \hikiopt{compo}{compo}{0},\ \hikiopt{coeff\_ring}{CommutativeRing}{0}}{\out{RingMatrix}}\\
  \spacing
  % document of basic document
  \quad Create matrix whose coefficient ring belongs ring.\\
  \spacing
  % added document
  \quad RingMatrix is subclass of \linkingone{matrix}{Matrix}.
  See Matrix for getting information about the initialization.\\
  \spacing
  % input, output document
  \begin{op}
    \verb|M+N| & Return the sum of matrices \param{M} and \param{N}.\\
    \verb+M-N+ & Return the difference of matrices \param{M} and \param{N}.\\
    \verb+M*N+ & Return the product of \param{M} and \param{N}. \param{N} must be matrix, vector or scalar\\
    \verb+M % d+ & Return \param{M} modulo \param{d}. \param{d} must be nonzero integer.\\
    \verb+-M+ & Return the matrix whose coefficients have inverted signs of \param{M}.\\
    \verb|+M| & Return the copy of \param{M}.\\
  \end{op}
\begin{ex}
>>> A = matrix.Matrix(2, 3, [1,2,3]+[4,5,6])
>>> B = matrix.Matrix(2, 3, [7,8,9]+[0,-1,-2])
>>> A + B
[8, 10, 12]+[4, 4, 4]
>>> A - B
[-6, -6, -6]+[4, 6, 8]
>>> A * B.transpose()
[50, -8]+[122, -17]
>>> -B * vector.Vector([1, -1, 0])
Vector([1, -1])
>>> 2 * A
[2, 4, 6]+[8, 10, 12]
>>> B % 3
[1, 2, 0]+[0, 2, 1]
\end{ex}%Don't indent!
\method
\subsubsection{getCoefficientRing -- get coefficient ring}\linkedtwo{matrix}{RingMatrix}{getCoefficientRing}
   \func{getCoefficientRing}{\param{self}}{\out{CommutativeRing}}\\
   \spacing
   % document of basic document
   \quad Return the coefficient ring of \param{self}.\\
   \spacing
   % added document
   \quad This method checks all elements of \param{self} and set \param{coeff\_ring} to the valid coefficient ring.\\
   \spacing
   % input, output document
   %\quad \param{a} must be int, long or rational.Integer.\\
  \subsubsection{toFieldMatrix -- set field as coefficient ring}\linkedtwo{matrix}{RingMatrix}{toFieldMatrix}
   \func{toFieldMatrix}{\param{self}}{\out{(None)}}\\
   \spacing
   % document of basic document
   \quad Change the class of the matrix to \linkingone{matrix}{FieldMatrix} or \linkingone{matrix}{FieldSquareMatrix}, where the coefficient ring will be the quotient field of the current domain.\\
   \spacing
   % added document
   % \quad 
   %\spacing
   % input, output document
   %\quad \param{a} must be int, long or rational.Integer.\\
  \subsubsection{toSubspace -- regard as vector space}\linkedtwo{matrix}{RingMatrix}{toSubspace}
   \func{toSubspace}{\param{self},\ \hikiopt{isbasis}{True/False}{None}}{\out{(None)}}\\
   \spacing
   % document of basic document
   \quad Change the class of the matrix to \param{Subspace}, where the coefficient ring will be the quotient field of the current domain.\\
   \spacing
   % added document
   % \quad 
   %\spacing
   % input, output document
   %\quad \param{a} must be int, long or rational.Integer.\\
  \subsubsection{hermiteNormalForm (HNF) -- Hermite Normal Form}\linkedtwo{matrix}{RingMatrix}{hermiteNormalForm}\linkedtwo{matrix}{RingMatrix}{HNF}
   \func{hermiteNormalForm}{\param{self}}{\out{RingMatrix}}\\
   \func{HNF}{\param{self}}{\out{RingMatrix}}\\
   \spacing
   % document of basic document
   \quad Return upper triangular Hermite normal form (HNF).\\
   \spacing
   % added document
   % \quad 
   %\spacing
   % input, output document
   %\quad \param{a} must be int, long or rational.Integer.\\
  \subsubsection{exthermiteNormalForm (extHNF) -- extended Hermite Normal Form algorithm}\linkedtwo{matrix}{RingMatrix}{exthermiteNormalForm}
   \func{exthermiteNormalForm}{\param{self}}{\out{(RingSquareMatrix,\ RingMatrix)}}\\
   \func{extHNF}{\param{self}}{\out{(RingSquareMatrix,\ RingMatrix)}}\\
   \spacing
   % document of basic document
   \quad Return Hermite normal form \param{M} and \param{U} satisfied $\param{self}\param{U}=\param{M}$.\\
   \spacing
   % added document
   % \quad 
   %\spacing
   % input, output document
   \quad  The function returns tuple (\param{U},\ \param{M}), where \param{U} is an instance of \linkingone{matrix}{RingSquareMatrix} and \param{M} is an instance of \linkingone{matrix}{RingMatrix}.\\
  \subsubsection{kernelAsModule -- kernel as $\mathbb{Z}$-module}\linkedtwo{matrix}{RingMatrix}{kernelAsModule}
   \func{kernelAsModule}{\param{self}}{\out{RingMatrix}}\\
   \spacing
   % document of basic document
   \quad Return kernel as $\mathbb{Z}$-module.\\
   \spacing
   % added document
   \quad The difference between the function and \linkingtwo{matrix}{FieldMatrix}{kernel} is that each elements of the returned value are integer.\\
   \spacing
   % input, output document
   %\quad
\begin{ex}
>>> A = matrix.Matrix(3, 4, [1,2,3,4,5,6,7,8,9,-1,-2,-3])
>>> print A.hermiteNormalForm()
0 36 29 28
0  0  1  0
0  0  0  1
>>> U, M = A.hermiteNormalForm()
>>> A * U == M
True
>>> B = matrix.Matrix(1, 2, [2, 1])
>>> print B.kernelAsModule()
1
-2
\end{ex}
\C

\subsection{RingSquareMatrix -- square matrix whose elements belong ring}\linkedone{matrix}{RingSquareMatrix}
  \func{RingSquareMatrix}{\hiki{row}{integer},\ \hikiopt{column}{integer}{0},\ \hikiopt{compo}{compo}{0},\ \hikiopt{coeff\_ring}{CommutativeRing}{0}}{\out{RingMatrix}}\\
  \spacing
  % document of basic document
  \quad Create square matrix whose coefficient ring belongs ring.\\
  \spacing
  % added document
  \quad RingSquareMatrix is subclass of \linkingone{matrix}{RingMatrix} and \linkingone{matrix}{SquareMatrix}.
  See SquareMatrix for getting information about the initialization.\\
  \spacing
  % input, output document
  \begin{op}
    \verb|M**c| & Return the \param{c}-th power of matrices \param{M}.\\
  \end{op}
\begin{ex}
>>> A = matrix.RingSquareMatrix(3, [1,2,3]+[4,5,6]+[7,8,9])
>>> A ** 2
[30L, 36L, 42L]+[66L, 81L, 96L]+[102L, 126L, 150L]
\end{ex}%Don't indent!
\method
  \subsubsection{getRing -- get matrix ring}\linkedtwo{matrix}{RingSquareMatrix}{getRing}
   \func{getRing}{\param{self}}{\out{MatrixRing}}\\
   \spacing
   % document of basic document
   \quad Return the \linkingone{matrix}{MatrixRing} belonged to by \param{self}.\\
   \spacing
   % added document
   %\quad 
   %\spacing
   % input, output document
   %\quad \param{a} must be int, long or rational.Integer.\\
  \subsubsection{isOrthogonalMatrix -- check orthogonal matrix}\linkedtwo{matrix}{RingSquareMatrix}{isOrthogonalMatrix}
   \func{isOrthogonalMatrix}{\param{self}}{\out{True/False}}\\
   \spacing
   % document of basic document
   \quad Check whether \param{self} is orthogonal matrix or not.\\
   \spacing
   % added document
   %\quad 
   %\spacing
   % input, output document
   %\quad \param{a} must be int, long or rational.Integer.\\
  \subsubsection{isAlternatingMatrix (isAntiSymmetricMatrix,\ isSkewSymmetricMatrix) -- check alternating matrix}\linkedtwo{matrix}{RingSquareMatrix}{isAlternatingMatrix}
   \func{isAlternatingMatrix}{\param{self}}{\out{True/False}}\\
   \spacing
   % document of basic document
   \quad Check whether \param{self} is alternating matrix or not.\\
   \spacing
   % added document
   %\quad 
   %\spacing
   % input, output document
   %\quad \param{a} must be int, long or rational.Integer.\\
  \subsubsection{isSingular -- check singular matrix}\linkedtwo{matrix}{RingSquareMatrix}{isSingular}
   \func{isSingular}{\param{self}}{\out{True/False}}\\
   \spacing
   % document of basic document
   \quad Check whether \param{self} is singular matrix or not.\\
   \spacing
   % added document
   \quad The function determines whether determinant of \param{self} is $0$.
   Note that the the non-singular matrix does not automatically mean invertible matrix; the nature that the matrix is invertible depends on its coefficient ring.\\
   \spacing
   % input, output document
   %\quad \param{a} must be int, long or rational.Integer.\\
  \subsubsection{trace -- trace}\linkedtwo{matrix}{RingSquareMatrix}{trace}
   \func{trace}{\param{self}}{\out{RingElement}}\\
   \spacing
   % document of basic document
   \quad Return the trace of \param{self}.\\
   \spacing
   % added document
   %\quad 
   % input, output document
   %\quad \param{a} must be int, long or rational.Integer.\\
  \subsubsection{determinant -- determinant}\linkedtwo{matrix}{RingSquareMatrix}{determinant}
   \func{determinant}{\param{self}}{\out{RingElement}}\\
   \spacing
   % document of basic document
   \quad Return the determinant of \param{self}.\\
   \spacing
   % added document
   %\quad 
   % input, output document
   %\quad \param{a} must be int, long or rational.Integer.\\
  \subsubsection{cofactor -- cofactor}\linkedtwo{matrix}{RingSquareMatrix}{cofactor}
   \func{cofactor}{\param{self},\ \hiki{i}{integer},\ \hiki{j}{integer}}{\out{RingElement}}\\
   \spacing
   % document of basic document
   \quad Return the (\param{i},\ \param{j})-cofactor.\\
   \spacing
   % added document
   %\quad 
   % input, output document
   %\quad \param{a} must be int, long or rational.Integer.\\
  \subsubsection{commutator -- commutator}\linkedtwo{matrix}{RingSquareMatrix}{commutator}
   \func{commutator}{\param{self},\ \hiki{N}{RingSquareMatrix element}}{\out{RingSquareMatrix}}\\
   \spacing
   % document of basic document
   \quad Return the commutator for \param{self} and \param{N}.\\
   \spacing
   % added document
   \quad The commutator for \param{M} and \param{N}, which is denoted as $[M,\ N]$, is defined as $[\param{M},\ \param{N}]=\param{M}\param{N}-\param{N}\param{M}$. 
   % input, output document
   %\quad \param{a} must be int, long or rational.Integer.\\
  \subsubsection{characteristicMatrix -- characteristic matrix}\linkedtwo{matrix}{RingSquareMatrix}{characteristicMatrix}
   \func{characteristicMatrix}{\param{self}}{\out{RingSquareMatrix}}\\
   \spacing
   % document of basic document
   \quad Return the characteristic matrix of \param{self}.\\
   \spacing
   % added document
   %\quad 
   % input, output document
   %\quad \param{a} must be int, long or rational.Integer.\\
  \subsubsection{adjugateMatrix -- adjugate matrix}\linkedtwo{matrix}{RingSquareMatrix}{adjugateMatrix}
   \func{adjugateMatrix}{\param{self}}{\out{RingSquareMatrix}}\\
   \spacing
   % document of basic document
   \quad Return the adjugate matrix of \param{self}.\\
   \spacing
   % added document
   \quad The adjugate matrix for \param{M} is the matrix \param{N} such that $\param{M}\param{N}=\param{N}\param{M}=(\det{\param{M}})E$, where $E$ is the identity matrix.\\
   % input, output document
   %\quad \param{a} must be int, long or rational.Integer.\\
  \subsubsection{cofactorMatrix (cofactors) -- cofactor matrix}\linkedtwo{matrix}{RingSquareMatrix}{cofactorMatrix}\linkedtwo{matrix}{RingSquareMatrix}{cofactors}
   \func{cofactorMatrix}{\param{self}}{\out{RingSquareMatrix}}\\
   \func{cofactors}{\param{self}}{\out{RingSquareMatrix}}\\
   \spacing
   % document of basic document
   \quad Return the cofactor matrix of \param{self}.\\
   \spacing
   % added document
   \quad The cofactor matrix for \param{M} is the matrix whose ($i,\ j$) element is  ($i,\ j$)-cofactor of \param{M}.
    The cofactor matrix is same to transpose of the adjugate matrix.\\
   % input, output document
   %\quad \param{a} must be int, long or rational.Integer.\\
  \subsubsection{smithNormalForm (SNF,\ elementary\_divisor) -- Smith Normal Form (SNF)}\linkedtwo{matrix}{RingSquareMatrix}{smithNormalForm}\linkedtwo{matrix}{RingSquareMatrix}{SNF}\linkedtwo{matrix}{RingSquareMatrix}{elementary\_divisor}
   \func{smithNormalForm}{\param{self}}{\out{RingSquareMatrix}}\\
   \func{SNF}{\param{self}}{\out{RingSquareMatrix}}\\
   \func{elementary\_divisor}{\param{self}}{\out{RingSquareMatrix}}\\
   \spacing
   % document of basic document
   \quad Return the list of diagonal elements of the Smith Normal Form (SNF) for \param{self}.\\
   \spacing
   % added document
   \quad The function assumes that \param{self} is non-singular.\\
   % input, output document
   %\quad \param{a} must be int, long or rational.Integer.\\
  \subsubsection{extsmithNormalForm (extSNF) -- Smith Normal Form (SNF)}\linkedtwo{matrix}{RingSquareMatrix}{extsmithNormalForm}\linkedtwo{matrix}{RingSquareMatrix}{extSNF}
   \func{extsmithNormalForm}{\param{self}}{\out{(RingSquareMatrix,\ RingSquareMatrix,\ RingSquareMatrix)}}\\
   \func{extSNF}{\param{self}}{\out{RingSquareMatrix,\ RingSquareMatrix,\ RingSquareMatrix)}}\\
   \spacing
   % document of basic document
   \quad Return the Smith normal form \param{M} for \param{self} and \param{U},\param{V} satisfied $\param{U}\param{self}\param{V}=\param{M}$.\\
   \spacing
   % added document
   % input, output document
   %\quad
\begin{ex}
>>> A = matrix.RingSquareMatrix(3, [3,-5,8]+[-9,2,7]+[6,1,-4])
>>> A.trace()
1L
>>> A.determinant()
-243L
>>> B = matrix.RingSquareMatrix(3, [87,38,80]+[13,6,12]+[65,28,60])
>>> U, V, M = B.extsmithNormalForm()
>>> U * B * V == M
True
>>> print M
4 0 0
0 2 0
0 0 1
>>> B.smithNormalForm()
[4L, 2L, 1L]
\end{ex}
\C

\subsection{FieldMatrix -- matrix whose elements belong field}\linkedone{matrix}{FieldMatrix}
  \func{FieldMatrix}{\hiki{row}{integer},\ \hiki{column}{integer},\ \hikiopt{compo}{compo}{0},\ \hikiopt{coeff\_ring}{CommutativeRing}{0}}{\out{RingMatrix}}\\
  \spacing
  % document of basic document
  \quad Create matrix whose coefficient ring belongs field.\\
  \spacing
  % added document
  \quad FieldMatrix is subclass of \linkingone{matrix}{RingMatrix}.
  See \linkingone{matrix}{Matrix} for getting information about the initialization.\\
  \spacing
  % input, output document
  \begin{op}
    \verb|M/d| & Return the division of \param{M} by \param{d}.\param{d} must be scalar.\\
    \verb|M//d| & Return the division of \param{M} by \param{d}.\param{d} must be scalar.\\
  \end{op}
\begin{ex}
>>> A = matrix.FieldMatrix(3, 3, [1,2,3,4,5,6,7,8,9])
>>> A / 210
1/210 1/105 1/70
2/105  1/42 1/35
1/30  4/105 3/70
\end{ex}%Don't indent!
\method
  \subsubsection{kernel -- kernel}\linkedtwo{matrix}{FieldMatrix}{kernel}
   \func{kernel}{\param{self}}{\out{FieldMatrix}}\\
   \spacing
   % document of basic document
   \quad Return the kernel of \param{self}.\\
   \spacing
   % added document
   %\quad 
   %\spacing
   % input, output document
   \quad The output is the matrix whose column vectors form basis of the kernel.\\
   The function returns None if the kernel do not exist.\\
  \subsubsection{image -- image}\linkedtwo{matrix}{FieldMatrix}{image}
   \func{image}{\param{self}}{\out{FieldMatrix}}\\
   \spacing
   % document of basic document
   \quad Return the image of \param{self}.\\
   \spacing
   % added document
   %\quad 
   %\spacing
   % input, output document
   \quad The output is the matrix whose column vectors form basis of the image.\\
   The function returns None if the kernel do not exist.\\
  \subsubsection{rank -- rank}\linkedtwo{matrix}{FieldMatrix}{rank}
   \func{rank}{\param{self}}{\out{integer}}\\
   \spacing
   % document of basic document
   \quad Return the rank of \param{self}.\\
   \spacing
   % added document
   %\quad 
   %\spacing
   % input, output document
   %\quad \\
  \subsubsection{inverseImage -- inverse image: base solution of linear system}\linkedtwo{matrix}{FieldMatrix}{inverseImage}
   \func{inverseImage}{\param{self},\ \hiki{V}{Vector/RingMatrix}}{\out{RingMatrix}}\\
   \spacing
   % document of basic document
   \quad Return an inverse image of \param{V} by \param{self}.\\
   \spacing
   % added document
   \quad The function returns one solution of the linear equation $\param{self}X=\param{V}$.
   \spacing
   % input, output document
   %\quad \\
  \subsubsection{solve -- solve linear system}\linkedtwo{matrix}{FieldMatrix}{solve}
   \func{solve}{\param{self},\ \hiki{B}{Vector/RingMatrix}}{\out{(RingMatrix,\ RingMatrix)}}\\
   \spacing
   % document of basic document
   \quad Solve $\param{self}X = \param{B}$.\\
   \spacing
   % added document
   \quad The function returns a particular solution \param{sol} and the kernel of \param{self} as a matrix.
   If you only have to obtain the particular solution, use \linkingtwo{matrix}{FieldMatrix}{inverseImage}.
   \spacing
   % input, output document
  \subsubsection{columnEchelonForm -- column echelon form}\linkedtwo{matrix}{FieldMatrix}{columnEchelonForm}
   \func{columnEchelonForm}{\param{self}}{\out{RingMatrix}}\\
   % document of basic document
   \quad Return the column reduced echelon form.\\
   \spacing
   % added document
   %\quad 
   %\spacing
   % input, output document
   %\quad
\begin{ex}
>>> A = matrix.FieldMatrix(2, 3, [1,2,3]+[4,5,6])
>>> print A.kernel
 1/1
-2/1
   1
>>> print A.image()
1 2
4 5
>>> C = matrix.FieldMatrix(4, 3, [1,2,3]+[4,5,6]+[7,8,9]+[-1,-2,-3])
>>> D = matrix.FieldMatrix(4, 2, [1,0]+[7,6]+[13,12]+[-1,0])
>>> print C.inverseImage(D)
 3/1  4/1
-1/1 -2/1
 0/1  0/1
>>> sol, ker = C.solve(D)
>>> C * (sol + ker[0]) == D
True
>>> AA = matrix.FieldMatrix(3, 3, [1,2,3]+[4,5,6]+[7,8,9])
>>> print AA.columnEchelonForm()
0/1 2/1 -1/1
0/1 1/1  0/1
0/1 0/1  1/1
\end{ex}
\C

\subsection{FieldSquareMatrix -- square matrix whose elements belong field}\linkedone{matrix}{FieldSquareMatrix}
  \func{FieldSquareMatrix}{\hiki{row}{integer},\ \hikiopt{column}{integer}{0},\ \hikiopt{compo}{compo}{0},\ \hikiopt{coeff\_ring}{CommutativeRing}{0}}{\out{FieldSquareMatrix}}\\
  \spacing
  % document of basic document
  \quad Create square matrix whose coefficient ring belongs field.\\
  \spacing
  % added document
  \quad FieldSquareMatrix is subclass of \linkingone{matrix}{FieldMatrix} and \linkingone{matrix}{SquareMatrix}.\\
  \negok The function \linkingone{matrix}{RingSquareMatrix}{determinant} is overridden and use different algorithm from one used in \linkingone{matrix}{RingSquareMatrix}{determinant};the function calls \linkingone{matrix}{FieldSquareMatrix}{triangulate}.
  See \linkingone{matrix}{SquareMatrix} for getting information about the initialization.\\
  \spacing
  % input, output document
 \method
  \subsubsection{triangulate - triangulate by elementary row operation}\linkedtwo{matrix}{FieldSquareMatrix}{triangulate}
   \func{triangulate}{\param{self}}{\out{FieldSquareMatrix}}\\
   \spacing
   % document of basic document
   \quad Return an upper triangulated matrix obtained by elementary row operations.\\
   \spacing
   % added document
   %\quad 
   %\spacing
   % input, output document
   %\quad
  \subsubsection{inverse - inverse matrix}\linkedtwo{matrix}{FieldSquareMatrix}{inverse}
   \func{inverse}{\param{self}\ \hikiopt{V}{Vector/RingMatrix}{None}}{\out{FieldSquareMatrix}}\\
   \spacing
   % document of basic document
   \quad Return the inverse of \param{self}.
   If \param{V} is given, then return $\param{self}^(-1)V$.\\ 
   \spacing
   % added document
   \quad \negok If the matrix is not invertible, then raise \linkingone{matrix}{NoInverse}.\\
   \spacing
   % input, output document
   %\quad
  \subsubsection{hessenbergForm - Hessenberg form}\linkedtwo{matrix}{FieldSquareMatrix}{hessenbergForm}
   \func{hessenbergForm}{\param{self}}{\out{FieldSquareMatrix}}\\
   \spacing
   % document of basic document
   \quad Return the Hessenberg form of \param{self}.\\
   \spacing
   % added document
   %\quad 
   %\spacing
   % input, output document
   %\quad
  \subsubsection{LUDecomposition - LU decomposition}\linkedtwo{matrix}{FieldSquareMatrix}{LUDecomposition}
   \func{LUDecomposition}{\param{self}}{\out{(FieldSquareMatrix,\ FieldSquareMatrix)}}\\
   \spacing
   % document of basic document
   \quad Return the lower triangular matrix \param{L} and the upper triangular matrix \param{U} such that $\param{self} == \param{L}\param{U}$.\\
   \spacing
   % added document
   %\quad 
   %\spacing
   % input, output document
   %\quad
\C

\subsection{\negok MatrixRing -- ring of matrices}\linkedone{matrix}{MatrixRing}
  \func{MatrixRing}{\hiki{size}{integer},\ \hiki{scalars}{CommutativeRing}}{\out{MatrixRing}}\\
  \spacing
  % document of basic document
  \quad Create a ring of matrices with given \param{size} and coefficient ring \param{scalars}.\\
  \spacing
  % added document
  \quad MatrixRing is subclass of \linkingone{ring}{Ring}.\\
  \spacing
  % input, output document
 \method
  \subsubsection{unitMatrix - unit matrix}\linkedtwo{matrix}{MatrixRing}{unitMatrix}
   \func{unitMatrix}{\param{self}}{\out{RingSquareMatrix}}\\
   \spacing
   % document of basic document
   \quad Return the unit matrix.\\
   \spacing
   % added document
   %\quad 
   %\spacing
   % input, output document
   %\quad
  \subsubsection{zeroMatrix - zero matrix}\linkedtwo{matrix}{MatrixRing}{zeroMatrix}
   \func{zeroMatrix}{\param{self}}{\out{RingSquareMatrix}}\\
   \spacing
   % document of basic document
   \quad Return the zero matrix.\\
   \spacing
   % added document
   %\quad 
   %\spacing
   % input, output document
   %\quad
\begin{ex}
>>> M = matrix.MatrixRing(3, rational.theIntegerRing)
>>> print M
M_3(Z)
>>> M.unitMatrix()
[1L, 0L, 0L]+[0L, 1L, 0L]+[0L, 0L, 1L]
>>> M.zero
[0L, 0L, 0L]+[0L, 0L, 0L]+[0L, 0L, 0L]
\end{ex}
\C
 \subsubsection{getInstance(class function) - get cached instance}\linkedtwo{matrix}{MatrixRing}{getInstance}
   \func{getInstance}{\param{cls},\ \hiki{size}{integer},\ \hiki{scalars}{CommutativeRing}}{\out{RingSquareMatrix}}\\
   \spacing
   % document of basic document
   \quad Return an instance of MatrixRing of given \param{size} and ring of scalars. \\
   \spacing
   % added document
   \quad The merit of using the method instead of the constructor is that the instances created by the method are cached and reused for efficiency.\\
   \spacing
   % input, output document
   %\quad
\begin{ex}
>>> print MatrixRing.getInstance(3, rational.theIntegerRing)
M_3(Z)
\end{ex}
\C

\subsection{Subspace -- subspace of finite dimensional vector space}\linkedone{matrix}{Subspace}
  \func{Subspace}{\hiki{row}{integer},\ \hikiopt{column}{integer}{0},\ \hikiopt{compo}{compo}{0},\ \hikiopt{coeff\_ring}{CommutativeRing}{0},\ \hikiopt{isbasis}{True/False}{None}}{\out{Subspace}}\\
  \spacing
  % document of basic document
  \quad Create subspace of some finite dimensional vector space over a field.\\
  \spacing
  % added document
  \quad Subspace is subclass of \linkingone{matrix}{FieldMatrix}.\\
  See \linkingone{matrix}{Matrix} for getting information about the initialization.
  The subspace expresses the space generated by column vectors of \param{self}.\\
  \spacing
  % input, output document
  If \param{isbasis} is True, we assume that column vectors are linearly independent.
 \begin{at}
   \item[isbasis] The attribute indicates the linear independence of column vectors, i.e., if they form a basis of the space then \param{isbasis} should be True, otherwise False.
 \end{at}
 \method
  \subsubsection{issubspace - check subspace of self}\linkedtwo{matrix}{Subspace}{triangulate}
   \func{Subspace}{\param{self},\ \hiki{other}{Subspace}}{\out{True/False}}\\
   \spacing
   % document of basic document
   \quad Return True if the subspace instance is a subspace of the \param{other}, or False otherwise.\\
   \spacing
   % added document
   %\quad 
   %\spacing
   % input, output document
   %\quad
  \subsubsection{toBasis - select basis}\linkedtwo{matrix}{Subspace}{toBasis}
   \func{toBasis}{\param{self}}{\out{(None)}}\\
   \spacing
   % document of basic document
   \quad Rewrite \param{self} so that its column vectors form a basis, and set True to its \param{isbasis}.\\
   \spacing
   % added document
   \quad The function does nothing if \param{isbasis} is already True.\\
   \spacing
   % input, output document
   %\quad
  \subsubsection{supplementBasis - to full rank}\linkedtwo{matrix}{Subspace}{supplementBasis}
   \func{supplementBasis}{\param{self}}{\out{Subspace}}\\
   \spacing
   % document of basic document
   \quad Return full rank matrix by supplementing bases for \param{self}.\\
   \spacing
   % added document
   %\quad
   %\spacing
   % input, output document
   %\quad
  \subsubsection{sumOfSubspaces - sum as subspace}\linkedtwo{matrix}{Subspace}{sumOfSubspaces}
   \func{sumOfSubspaces}{\param{self},\ \hiki{other}{Subspace}}{\out{Subspace}}\\
   \spacing
   % document of basic document
   \quad Return a matrix whose columns form a basis for sum of two subspaces.\\
   \spacing
   % added document
   %\quad
   %\spacing
   % input, output document
   %\quad
  \subsubsection{intersectionOfSubspaces - intersection as subspace}\linkedtwo{matrix}{Subspace}{intersectionOfSubspaces}
   \func{intersectionOfSubspaces}{\param{self},\ \hiki{other}{Subspace}}{\out{Subspace}}\\
   \spacing
   % document of basic document
   \quad Return a matrix whose columns form a basis for intersection of two subspaces.\\
   \spacing
   % added document
   %\quad
   %\spacing
   % input, output document
   %\quad
\begin{ex}
>>> A = matrix.Subspace(4, 3, [1,2,3]+[4,5,6]+[7,8,9]+[10,11,12])
>>> A.toBasis()
>>> print A
 1  2
 4  5
 7  8
10 11
>>> B = matrix.Subspace(3, 2, [1,2]+[3,4]+[5,7])
>>> print B.supplementBasis()
1 2 0
3 4 0
5 7 1
>>> C = matrix.Subspace(4, 1, [1,2,3,4])
>>> D = matrix.Subspace(4, 2, [2,-4]+[4,-3]+[6,-2]+[8,-1])
>>> print C.intersectionOfSubspaces(D)
-2/1
-4/1
-6/1
-8/1
\end{ex}
\C

 \subsubsection{fromMatrix(class function) - create subspace}\linkedtwo{matrix}{Subspace}{fromMatrix}
   \func{fromMatrix}{\param{cls},\ \hiki{mat}{FieldMatrix},\ \hikiopt{isbasis}{True/False}{None}}{\out{Subspace}}\\
   \spacing
   % document of basic document
   \quad Create a Subspace instance from a matrix instance \param{mat}, whose class can be any of subclasses of Matrix.\\
   \spacing
   % added document
   \quad Please use this method if you want a Subspace instance for sure.\\
   \spacing
   % input, output document
   %\quad
\C

 \subsection{createMatrix[function] -- create an instance}\linkedone{matrix}{createMatrix}
  \func{createMatrix}{\hiki{row}{integer},\ \hikiopt{column}{integer}{0},\ \hikiopt{compo}{compo}{0},\ \hikiopt{coeff\_ring}{CommutativeRing}{None}}{\param{RingMatrix}}\\
   \spacing
   % document of basic document
   \quad Create an instance of \linkingone{matrix}{RingMatrix}, \linkingone{matrix}{RingSquareMatrix}, \linkingone{matrix}{FieldMatrix} or \linkingone{matrix}{FieldSquareMatrix}.\\
   \spacing
   % added document
   \quad Your input determines the class automatically by examining the matrix size and the coefficient ring.
   See \linkingone{matrix}{Matrix} or \linkingone{matrix}{SquareMatrix} for getting information about the initialization.\\
   \spacing
   % input, output document
 \subsection{identityMatrix(unitMatrix)[function] -- unit matrix}\linkedone{matrix}{identityMatrix}\linkedone{matrix}{unitMatrix}
  \func{identityMatrix}{\hiki{size}{integer},\ \hikiopt{coeff}{CommutativeRing/CommutativeRingElement}{None}}{\param{RingMatrix}}\\
    \func{unitMatrix}{\hiki{size}{integer},\ \hikiopt{coeff}{CommutativeRing/CommutativeRingElement}{None}}{\param{RingMatrix}}\\
   \spacing
   % document of basic document
   \quad Return \param{size}-dimensional unit matrix.\\
   \spacing
   % added document
   \quad \param{coeff} enables us to create matrix not only in integer but in coefficient ring which is determined by coeff. \\
   \spacing
   % input, output document
    \param{coeff} must be an instance of \linkingone{ring}{Ring} or a multiplicative unit (one).
 \subsection{zeroMatrix[function] -- zero matrix}\linkedone{matrix}{zeroMatrix}
  \func{zeroMatrix}{\hiki{row}{integer},\ \hikiopt{column}{0},\ \hikiopt{coeff}{CommutativeRing/CommutativeRingElement}{None}}{\param{RingMatrix}}\\
   \spacing
   % document of basic document
   \quad Return $\param{row}\times\param{column}$ zero matrix.\\
   \spacing
   % added document
   \quad \param{coeff} enables us to create matrix not only in integer but in coefficient ring which is determined by coeff. \\
   \spacing
   % input, output document
    \param{coeff} must be an instance of \linkingone{ring}{Ring} or a additive unit (zero).
    If \param{column} is abbreviated, \param{column} is set same to \param{row}.\begin{ex}
>>> M = matrix.createMatrix(3, [1,2,3]+[4,5,6]+[7,8,9])
>>> print M
1 2 3
4 5 6
7 8 9
>>> O = matrix.zeroMatrix(2, 3, imaginary.ComplexField())
>>> print O
0 + 0j 0 + 0j 0 + 0j
0 + 0j 0 + 0j 0 + 0j
\end{ex}
\C

%---------- end document ---------- %

\bibliographystyle{jplain}%use jbibtex
\bibliography{nzmath_references}

\end{document}

