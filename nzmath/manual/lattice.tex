\documentclass{report}

%%%%%%%%%%%%%%%%%%%%%%%%%%%%%%%%%%%%%%%%%%%%%%%%%%%%%%%%%%%%%
%
% macros for nzmath manual
%
%%%%%%%%%%%%%%%%%%%%%%%%%%%%%%%%%%%%%%%%%%%%%%%%%%%%%%%%%%%%%
\usepackage{amssymb,amsmath}
\usepackage{color}
\usepackage[dvipdfm,bookmarks=true,bookmarksnumbered=true,%
 pdftitle={NZMATH Users Manual},%
 pdfsubject={Manual for NZMATH Users},%
 pdfauthor={NZMATH Development Group},%
 pdfkeywords={TeX; dvipdfmx; hyperref; color;},%
 colorlinks=true]{hyperref}
\usepackage{fancybox}
\usepackage[T1]{fontenc}
%
\newcommand{\DS}{\displaystyle}
\newcommand{\C}{\clearpage}
\newcommand{\NO}{\noindent}
\newcommand{\negok}{$\dagger$}
\newcommand{\spacing}{\vspace{1pt}\\ }
% software macros
\newcommand{\nzmathzero}{{\footnotesize $\mathbb{N}\mathbb{Z}$}\texttt{MATH}}
\newcommand{\nzmath}{{\nzmathzero}\ }
\newcommand{\pythonzero}{$\mbox{\texttt{Python}}$}
\newcommand{\python}{{\pythonzero}\ }
% link macros
\newcommand{\linkingzero}[1]{{\bf \hyperlink{#1}{#1}}}%module
\newcommand{\linkingone}[2]{{\bf \hyperlink{#1.#2}{#2}}}%module,class/function etc.
\newcommand{\linkingtwo}[3]{{\bf \hyperlink{#1.#2.#3}{#3}}}%module,class,method
\newcommand{\linkedzero}[1]{\hypertarget{#1}{}}
\newcommand{\linkedone}[2]{\hypertarget{#1.#2}{}}
\newcommand{\linkedtwo}[3]{\hypertarget{#1.#2.#3}{}}
\newcommand{\linktutorial}[1]{\href{http://docs.python.org/tutorial/#1}{#1}}
\newcommand{\linktutorialone}[2]{\href{http://docs.python.org/tutorial/#1}{#2}}
\newcommand{\linklibrary}[1]{\href{http://docs.python.org/library/#1}{#1}}
\newcommand{\linklibraryone}[2]{\href{http://docs.python.org/library/#1}{#2}}
\newcommand{\pythonhp}{\href{http://www.python.org/}{\python website}}
\newcommand{\nzmathwiki}{\href{http://nzmath.sourceforge.net/wiki/}{{\nzmathzero}Wiki}}
\newcommand{\nzmathsf}{\href{http://sourceforge.net/projects/nzmath/}{\nzmath Project Page}}
\newcommand{\nzmathtnt}{\href{http://tnt.math.se.tmu.ac.jp/nzmath/}{\nzmath Project Official Page}}
% parameter name
\newcommand{\param}[1]{{\tt #1}}
% function macros
\newcommand{\hiki}[2]{{\tt #1}:\ {\em #2}}
\newcommand{\hikiopt}[3]{{\tt #1}:\ {\em #2}=#3}

\newdimen\hoge
\newdimen\truetextwidth
\newcommand{\func}[3]{%
\setbox0\hbox{#1(#2)}
\hoge=\wd0
\truetextwidth=\textwidth
\advance \truetextwidth by -2\oddsidemargin
\ifdim\hoge<\truetextwidth % short form
{\bf \colorbox{skyyellow}{#1(#2)\ $\to$ #3}}
%
\else % long form
\fcolorbox{skyyellow}{skyyellow}{%
   \begin{minipage}{\textwidth}%
   {\bf #1(#2)\\ %
    \qquad\quad   $\to$\ #3}%
   \end{minipage}%
   }%
\fi%
}

\newcommand{\out}[1]{{\em #1}}
\newcommand{\initialize}{%
  \paragraph{\large \colorbox{skyblue}{Initialize (Constructor)}}%
    \quad\\ %
    \vspace{3pt}\\
}
\newcommand{\method}{\C \paragraph{\large \colorbox{skyblue}{Methods}}}
% Attribute environment
\newenvironment{at}
{%begin
\paragraph{\large \colorbox{skyblue}{Attribute}}
\quad\\
\begin{description}
}%
{%end
\end{description}
}
% Operation environment
\newenvironment{op}
{%begin
\paragraph{\large \colorbox{skyblue}{Operations}}
\quad\\
\begin{table}[h]
\begin{center}
\begin{tabular}{|l|l|}
\hline
operator & explanation\\
\hline
}%
{%end
\hline
\end{tabular}
\end{center}
\end{table}
}
% Examples environment
\newenvironment{ex}%
{%begin
\paragraph{\large \colorbox{skyblue}{Examples}}
\VerbatimEnvironment
\renewcommand{\EveryVerbatim}{\fontencoding{OT1}\selectfont}
\begin{quote}
\begin{Verbatim}
}%
{%end
\end{Verbatim}
\end{quote}
}
%
\definecolor{skyblue}{cmyk}{0.2, 0, 0.1, 0}
\definecolor{skyyellow}{cmyk}{0.1, 0.1, 0.5, 0}
%
%\title{NZMATH User Manual\\ {\large{(for version 1.0)}}}
%\date{}
%\author{}
\begin{document}
%\maketitle
%
\setcounter{tocdepth}{3}
\setcounter{secnumdepth}{3}


\tableofcontents
\C

\chapter{Classes}

%---------- start document ---------- %
 \section{lattice -- Lattice}\linkedzero{algfield}
 \begin{itemize}
 \item {\bf Classes}
   \begin{itemize}
   \item \linkingone{lattice}{Lattice}
   \item \linkingone{lattice}{LatticeElement}
   \end{itemize}
 \item {\bf Functions}
   \begin{itemize}
   \item \linkingone{lattice}{LLL}
   \end{itemize}
 \end{itemize}
%
  \subsection{Lattice -- lattice}\linkedone{lattice}{Lattice}
  \initialize
  \func{Lattice}{
    \hiki{basis}{\linkingone{matrix}{Matrix}},\ \hiki{quadraticForm}{\linkingone{matrix}{Matrix}}}{\out{Lattice}}\\
  \spacing
  % document of basic document
  \quad Create Lattice object. \\
  \spacing
  % added document
  \spacing
  % input, output document
  \begin{at}
    \item[basis]\linkedtwo{lattice}{Lattice}{basis}: The basis of \param{self} lattice.
    \item[quadraticForm]\linkedtwo{lattice}{Lattice}{quadraticForm}: The quadratic form of 
  \end{at}
  \begin{op}
  \end{op}
\begin{ex}
\end{ex}%Don't indent!
\C
  \method
  \subsubsection{createElement -- create element}\linkedtwo{lattice}{Lattice}{createElement}
  \func{createElement}{\param{self}, \ \hiki{compo}{list}}{\out{\linkingone{lattice}{LatticeElement}}}\\
  \spacing
  % document of basic document
  \quad Create the element which has coefficients with given \param{compo}. \\
  \spacing
  % add document
  %\spacing
  % input, output document
%
  \subsubsection{bilinearForm -- bilinear form}\linkedtwo{lattice}{Lattice}{bilinearForm}
  \func{bilinearForm}{\param{self}, \ \param{v_1}{}, \, \param{v_2} }{\out{integer}}\\
  \spacing
  % document of basic document
  \quad Return the (polynomial) discriminant of the \param{self}.\linkingtwo{lattice}{Lattice}{polynomial}. \\
  \spacing
  % add document
  \quad \negok The output is not discriminant of the number field itself. \\
  %\spacing
  % input, output document
%
  \subsubsection{isCyclic -- Check whether cyclic lattice or not}\linkedtwo{lattice}{Lattice}{isCyclic}
  \func{isCyclic}{\param{self}}{\out{bool}}}\\
  \spacing
  % document of basic document
  \quad Check whether \param{self} lattice is a cyclic lattice or not.
  \spacing
  % add document
  \quad 
  %\spacing
  % input, output document
%
  \subsubsection{isIdeal -- Check whether ideal lattice or not}\linkedtwo{lattice}{Lattice}{isIdeal}
  \func{signature}{\param{self}}{\out{bool}}\\
  \spacing
  % document of basic document
  \quad Check whether \param{self} lattice is a ideal lattice or not.
  \spacing
  % add document
  %\spacing
  % input, output document
%
\begin{ex}
\end{ex}%Don't indent!
\C
  \subsection{LatticeElement -- Lattice Element}\linkedone{lattice}{LatticeElement}
  \initialize
  \func{Lattice}{
   \hiki{lattice}{list},\
   \hiki{compo}{list},\ 
  }{
   \out{LatticeElement}
  }\\
  \spacing
  % document of basic document
  \spacing
  % added document
  %\spacing
  % input, output document
  \begin{at}
    \item[lattice]\linkedtwo{lattice}{LatticeElement}{lattice}: 
    \item[row]\linkedtwo{lattice}{LatticeElement}{row}: 
    \item[column]\linkedtwo{lattice}{LatticeElement}{column}: 
    \item[compo]\linkedtwo{lattice}{LatticeElement}{compo}: 
  \end{at}
  \begin{op}
  \end{op}
\begin{ex}
\end{ex}%Don't indent!
\C
  \method
  \subsubsection{getLattice -- FInd lattice belongs to}\linkedtwo{lattice}{LatticeElement}{getLattice}
  \func{getLattice}{\param{self}}{\out{\linkingone{lattice}{Lattice}}}\\
  \spacing
  % document of basic document
  \spacing
  % add document
  %\spacing
  % input, output document
%
\begin{ex}
\end{ex}%Don't indent!
\C
  \subsection{LLL(function) -- LLL reduction}\linkedone{lattice}{LLL}
  \func{LLL}{\hiki{M}{\linkingone{matrix}{Matrix}}}{\out{Matrix}, \ \out{Matrix}} \\
  \spacing
  % document of basic document
  \quad Return 
  \spacing
  % add document
  % \spacing
  % input, output document
%
\begin{ex}
>>> M=mat.Matrix(3,3,[1,0,12,0,1,26,0,0,13]);
>>> lat.LLL(M);
([1, 0, 0]+[0, 1, 0]+[0, 0, 13], [1L, 0L, -12L]+[0L, 1L, -26L]+[0L, 0L, 1L])
>>> 
\end{ex}%Don't indent!(indent causes an error.)
\C

%---------- end document ---------- %

\bibliographystyle{jplain}%use jbibtex
\bibliography{nzmath_references}

\end{document}