\documentclass{report}

%%%%%%%%%%%%%%%%%%%%%%%%%%%%%%%%%%%%%%%%%%%%%%%%%%%%%%%%%%%%%
%
% macros for nzmath manual
%
%%%%%%%%%%%%%%%%%%%%%%%%%%%%%%%%%%%%%%%%%%%%%%%%%%%%%%%%%%%%%
\usepackage{amssymb,amsmath}
\usepackage{color}
\usepackage[dvipdfm,bookmarks=true,bookmarksnumbered=true,%
 pdftitle={NZMATH Users Manual},%
 pdfsubject={Manual for NZMATH Users},%
 pdfauthor={NZMATH Development Group},%
 pdfkeywords={TeX; dvipdfmx; hyperref; color;},%
 colorlinks=true]{hyperref}
\usepackage{fancybox}
\usepackage[T1]{fontenc}
%
\newcommand{\DS}{\displaystyle}
\newcommand{\C}{\clearpage}
\newcommand{\NO}{\noindent}
\newcommand{\negok}{$\dagger$}
\newcommand{\spacing}{\vspace{1pt}\\ }
% software macros
\newcommand{\nzmathzero}{{\footnotesize $\mathbb{N}\mathbb{Z}$}\texttt{MATH}}
\newcommand{\nzmath}{{\nzmathzero}\ }
\newcommand{\pythonzero}{$\mbox{\texttt{Python}}$}
\newcommand{\python}{{\pythonzero}\ }
% link macros
\newcommand{\linkingzero}[1]{{\bf \hyperlink{#1}{#1}}}%module
\newcommand{\linkingone}[2]{{\bf \hyperlink{#1.#2}{#2}}}%module,class/function etc.
\newcommand{\linkingtwo}[3]{{\bf \hyperlink{#1.#2.#3}{#3}}}%module,class,method
\newcommand{\linkedzero}[1]{\hypertarget{#1}{}}
\newcommand{\linkedone}[2]{\hypertarget{#1.#2}{}}
\newcommand{\linkedtwo}[3]{\hypertarget{#1.#2.#3}{}}
\newcommand{\linktutorial}[1]{\href{http://docs.python.org/tutorial/#1}{#1}}
\newcommand{\linktutorialone}[2]{\href{http://docs.python.org/tutorial/#1}{#2}}
\newcommand{\linklibrary}[1]{\href{http://docs.python.org/library/#1}{#1}}
\newcommand{\linklibraryone}[2]{\href{http://docs.python.org/library/#1}{#2}}
\newcommand{\pythonhp}{\href{http://www.python.org/}{\python website}}
\newcommand{\nzmathwiki}{\href{http://nzmath.sourceforge.net/wiki/}{{\nzmathzero}Wiki}}
\newcommand{\nzmathsf}{\href{http://sourceforge.net/projects/nzmath/}{\nzmath Project Page}}
\newcommand{\nzmathtnt}{\href{http://tnt.math.metro-u.ac.jp/nzmath/}{\nzmath Project Official Page}}
% parameter name
\newcommand{\param}[1]{{\tt #1}}
% function macros
\newcommand{\hiki}[2]{{\tt #1}:\ {\em #2}}
\newcommand{\hikiopt}[3]{{\tt #1}:\ {\em #2}=#3}

\newdimen\hoge
\newdimen\truetextwidth
\newcommand{\func}[3]{%
\setbox0\hbox{#1(#2)}
\hoge=\wd0
\truetextwidth=\textwidth
\advance \truetextwidth by -2\oddsidemargin
\ifdim\hoge<\truetextwidth % short form
{\bf \colorbox{skyyellow}{#1(#2)\ $\to$ #3}}
%
\else % long form
\fcolorbox{skyyellow}{skyyellow}{%
   \begin{minipage}{\textwidth}%
   {\bf #1(#2)\\ %
    \qquad\quad   $\to$\ #3}%
   \end{minipage}%
   }%
\fi%
}

\newcommand{\out}[1]{{\em #1}}
\newcommand{\initialize}{%
  \paragraph{\large \colorbox{skyblue}{Initialize (Constructor)}}%
    \quad\\ %
    \vspace{3pt}\\
}
\newcommand{\method}{\C \paragraph{\large \colorbox{skyblue}{Methods}}}
% Attribute environment
\newenvironment{at}
{%begin
\paragraph{\large \colorbox{skyblue}{Attribute}}
\quad\\
\begin{description}
}%
{%end
\end{description}
}
% Operation environment
\newenvironment{op}
{%begin
\paragraph{\large \colorbox{skyblue}{Operations}}
\quad\\
\begin{table}[h]
\begin{center}
\begin{tabular}{|l|l|}
\hline
operator & explanation\\
\hline
}%
{%end
\hline
\end{tabular}
\end{center}
\end{table}
}
% Examples environment
\newenvironment{ex}%
{%begin
\paragraph{\large \colorbox{skyblue}{Examples}}
\VerbatimEnvironment
\renewcommand{\EveryVerbatim}{\fontencoding{OT1}\selectfont}
\begin{quote}
\begin{Verbatim}
}%
{%end
\end{Verbatim}
\end{quote}
}
%
\definecolor{skyblue}{cmyk}{0.2, 0, 0.1, 0}
\definecolor{skyyellow}{cmyk}{0.1, 0.1, 0.5, 0}
%
%\title{NZMATH User Manual\\ {\large{(for version 1.0)}}}
%\date{}
%\author{}
\begin{document}
%\maketitle
%
\setcounter{tocdepth}{3}
\setcounter{secnumdepth}{3}


\tableofcontents
\C

\chapter{Functions}

%---------- start document ---------- %
 \section{gcd -- gcd algorithm}\linkedzero{gcd}
%
  \subsection{gcd -- the greatest common divisor}\linkedone{gcd}{gcd}
   \func{gcd}{\hiki{a}{integer},\ \hiki{b}{integer}}{\out{integer}}\\
   \spacing
   % document of basic document
   \quad Return the greatest common divisor of two integers \param{a} and \param{b}.\\
   \spacing
   % added document
   %\spacing
   % input, output document
   \quad \param{a},\ \param{b} must be int, long or \linkingone{rational}{Integer}.
   Even if one of the arguments is negative, the result is non-negative.\\
%
  \subsection{binarygcd -- binary gcd algorithm}\linkedone{gcd}{binarygcd}
   \func{binarygcd}{\hiki{a}{integer},\ \hiki{b}{integer}}{\out{integer}}\\
   \spacing
   % document of basic document
   \quad Return the greatest common divisor of two integers \param{a} and \param{b} by binary gcd algorithm.\\
   \spacing
   % added document
   \quad \negok This function is an alias of \linkingone{arygcd}{binarygcd}\\
   \spacing
   % input, output document
   \quad \param{a},\ \param{b} must be int, long, or \linkingone{rational}{Integer}.\\
%
  \subsection{extgcd -- extended gcd algorithm}\linkedone{gcd}{extgcd}
   \func{extgcd}{\hiki{a}{integer},\ \hiki{b}{integer}}{(\out{integer},\ \out{integer},\ \out{integer})}\\
   \spacing
   % document of basic document
   \quad Return the greatest common divisor $d$ of two integers \param{a} and \param{b} and $u,\ v$ such that $d = \param{a}u + \param{b}v$.\\
   \spacing
   % added document
   %\spacing
   % input, output document
   \quad \param{a},\ \param{b} must be int, long, or \linkingone{rational}{Integer}.\\
   The returned value is a tuple (\param{u},\ \param{v},\ \param{d}).\\
%
  \subsection{lcm -- the least common multiple}\linkedone{gcd}{lcm}
   \func{lcm}{\hiki{a}{integer},\ \hiki{b}{integer}}{\out{integer}}\\
   \spacing
   % document of basic document
   \quad Return the least common multiple of two integers \param{a} and \param{b}.\\
   \spacing
   % added document
   \negok If both \param{a} and \param{b} are zero, then it raises an exception.
   \spacing
   % input, output document
   \quad \param{a},\ \param{b} must be int, long, or \linkingone{rational}{Integer}.\\
%
  \subsection{gcd\_of\_list -- gcd of many integers}\linkedone{gcd}{gcd\_of\_list}
   \func{gcd\_of\_list}{\hiki{integers}{list}}{\out{list}}\\
   \spacing
   % document of basic document
   \quad Return gcd of multiple integers.\\
   \spacing
   % added document
   \quad For given \param{integers} $[x_1,\ldots,x_n]$, return a list $[d,\ [c_1,\ldots,c_n]]$ such that $d=c_1 x_1+\cdots+c_n x_n$, where $d$ is the greatest common divisor of $x_1,\ldots, x_n$.\\
   \spacing
   % input, output document
   \quad \param{integers} is a list which elements are int or long\\
   This function returns $[d,\ [c_1,\ldots,c_n]]$, where $d,\ c_i$ are an integer.\\
%
  \subsection{coprime -- coprime check}\linkedone{gcd}{coprime}
   \func{coprime}{\hiki{a}{integer},\ \hiki{b}{integer}}{\out{bool}}\\
   \spacing
   % document of basic document
   \quad Return True if \param{a} and \param{b} are coprime, False otherwise.\\
   \spacing
   % added document
   %\spacing
   % input, output document
   \quad \param{a},\ \param{b} are int, long, or \linkingone{rational}{integer}.\\
%
  \subsection{pairwise\_coprime -- coprime check of many integers}\linkedone{gcd}{pairwise\_coprime}
   \func{pairwise\_coprime}{\hiki{integers}{list}}{\out{bool}}\\
   \spacing
   % document of basic document
   \quad Return True if all integers in \param{integers} are pairwise coprime, False otherwise.\\
   \spacing
   % added document
   %\spacing
   % input, output document
   \quad \param{integers} is a list which elements are int, long, or \linkingone{rational}{Integer}.\\
%
\begin{ex}
>>> gcd.gcd(12, 18)
6
>>> gcd.gcd(12, -18)
6
>>> gcd.gcd(-12, -18)
6
>>> gcd.extgcd(12, -18)
(-1, -1, 6)
>>> gcd.extgcd(-12, -18)
(1, -1, 6)
>>> gcd.extgcd(0, -18)
(0, -1, 18)
>>> gcd.lcm(12, 18)
36
>>> gcd.lcm(12, -18)
-36
>>> gcd.gcd_of_list([60, 90, 210])
[30, [-1, 1, 0]]
\end{ex}%Don't indent!(indent causes an error.)
\C

%---------- end document ---------- %

\bibliographystyle{jplain}%use jbibtex
\bibliography{nzmath_references}

\end{document}