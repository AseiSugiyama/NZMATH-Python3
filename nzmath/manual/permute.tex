\documentclass{report}

%%%%%%%%%%%%%%%%%%%%%%%%%%%%%%%%%%%%%%%%%%%%%%%%%%%%%%%%%%%%%
%
% macros for nzmath manual
%
%%%%%%%%%%%%%%%%%%%%%%%%%%%%%%%%%%%%%%%%%%%%%%%%%%%%%%%%%%%%%
\usepackage{amssymb,amsmath}
\usepackage{color}
\usepackage[dvipdfm,bookmarks=true,bookmarksnumbered=true,%
 pdftitle={NZMATH Users Manual},%
 pdfsubject={Manual for NZMATH Users},%
 pdfauthor={NZMATH Development Group},%
 pdfkeywords={TeX; dvipdfmx; hyperref; color;},%
 colorlinks=true]{hyperref}
\usepackage{fancybox}
\usepackage[T1]{fontenc}
%
\newcommand{\DS}{\displaystyle}
\newcommand{\C}{\clearpage}
\newcommand{\NO}{\noindent}
\newcommand{\negok}{$\dagger$}
\newcommand{\spacing}{\vspace{1pt}\\ }
% software macros
\newcommand{\nzmathzero}{{\footnotesize $\mathbb{N}\mathbb{Z}$}\texttt{MATH}}
\newcommand{\nzmath}{{\nzmathzero}\ }
\newcommand{\pythonzero}{$\mbox{\texttt{Python}}$}
\newcommand{\python}{{\pythonzero}\ }
% link macros
\newcommand{\linkingzero}[1]{{\bf \hyperlink{#1}{#1}}}%module
\newcommand{\linkingone}[2]{{\bf \hyperlink{#1.#2}{#2}}}%module,class/function etc.
\newcommand{\linkingtwo}[3]{{\bf \hyperlink{#1.#2.#3}{#3}}}%module,class,method
\newcommand{\linkedzero}[1]{\hypertarget{#1}{}}
\newcommand{\linkedone}[2]{\hypertarget{#1.#2}{}}
\newcommand{\linkedtwo}[3]{\hypertarget{#1.#2.#3}{}}
\newcommand{\linktutorial}[1]{\href{http://docs.python.org/tutorial/#1}{#1}}
\newcommand{\linktutorialone}[2]{\href{http://docs.python.org/tutorial/#1}{#2}}
\newcommand{\linklibrary}[1]{\href{http://docs.python.org/library/#1}{#1}}
\newcommand{\linklibraryone}[2]{\href{http://docs.python.org/library/#1}{#2}}
\newcommand{\pythonhp}{\href{http://www.python.org/}{\python website}}
\newcommand{\nzmathwiki}{\href{http://nzmath.sourceforge.net/wiki/}{{\nzmathzero}Wiki}}
\newcommand{\nzmathsf}{\href{http://sourceforge.net/projects/nzmath/}{\nzmath Project Page}}
\newcommand{\nzmathtnt}{\href{http://tnt.math.se.tmu.ac.jp/nzmath/}{\nzmath Project Official Page}}
% parameter name
\newcommand{\param}[1]{{\tt #1}}
% function macros
\newcommand{\hiki}[2]{{\tt #1}:\ {\em #2}}
\newcommand{\hikiopt}[3]{{\tt #1}:\ {\em #2}=#3}

\newdimen\hoge
\newdimen\truetextwidth
\newcommand{\func}[3]{%
\setbox0\hbox{#1(#2)}
\hoge=\wd0
\truetextwidth=\textwidth
\advance \truetextwidth by -2\oddsidemargin
\ifdim\hoge<\truetextwidth % short form
{\bf \colorbox{skyyellow}{#1(#2)\ $\to$ #3}}
%
\else % long form
\fcolorbox{skyyellow}{skyyellow}{%
   \begin{minipage}{\textwidth}%
   {\bf #1(#2)\\ %
    \qquad\quad   $\to$\ #3}%
   \end{minipage}%
   }%
\fi%
}

\newcommand{\out}[1]{{\em #1}}
\newcommand{\initialize}{%
  \paragraph{\large \colorbox{skyblue}{Initialize (Constructor)}}%
    \quad\\ %
    \vspace{3pt}\\
}
\newcommand{\method}{\C \paragraph{\large \colorbox{skyblue}{Methods}}}
% Attribute environment
\newenvironment{at}
{%begin
\paragraph{\large \colorbox{skyblue}{Attribute}}
\quad\\
\begin{description}
}%
{%end
\end{description}
}
% Operation environment
\newenvironment{op}
{%begin
\paragraph{\large \colorbox{skyblue}{Operations}}
\quad\\
\begin{table}[h]
\begin{center}
\begin{tabular}{|l|l|}
\hline
operator & explanation\\
\hline
}%
{%end
\hline
\end{tabular}
\end{center}
\end{table}
}
% Examples environment
\newenvironment{ex}%
{%begin
\paragraph{\large \colorbox{skyblue}{Examples}}
\VerbatimEnvironment
\renewcommand{\EveryVerbatim}{\fontencoding{OT1}\selectfont}
\begin{quote}
\begin{Verbatim}
}%
{%end
\end{Verbatim}
\end{quote}
}
%
\definecolor{skyblue}{cmyk}{0.2, 0, 0.1, 0}
\definecolor{skyyellow}{cmyk}{0.1, 0.1, 0.5, 0}
%
%\title{NZMATH User Manual\\ {\large{(for version 1.0)}}}
%\date{}
%\author{}
\begin{document}
%\maketitle
%
\setcounter{tocdepth}{3}
\setcounter{secnumdepth}{3}


\tableofcontents
\C

\chapter{Classes}

%---------- start document ---------- %
 \section{permute -- permutation (symmetric) group}\linkedzero{permute}
 \begin{itemize}
   \item {\bf Classes}
   \begin{itemize}
     \item \linkingone{permute}{Permute}
     \item \linkingone{permute}{ExPermute}
     \item \linkingone{permute}{PermGroup}
   \end{itemize}
 \end{itemize}

\C

 \subsection{Permute -- element of permutation group}\linkedone{permute}{Permute}
 \initialize
  \func{Permute}{\hiki{value}{list/tuple},\ \hiki{key}{list/tuple}}{Permute}\\
  \func{Permute}{\hiki{val\_key}{dict}}{Permute}\\
  \func{Permute}{\hiki{value}{list/tuple}, \hikiopt{key}{int}{None}}{Permute}\\
  \spacing
  % document of basic document
  \quad Create an element of a permutation group.\\
  \spacing
  % added document
  \quad An instance will be generated with ``normal'' way.
  That is, we input a \param{key}, which is a list of (indexed) all elements from some set, and a \param{value}, which is a list of all permuted elements.\\
  \spacing
  % input, output document
  \quad Normally, you input two lists (or tuples) \param{value} and \param{key} with same length.
  Or you can input \param{val\_key} as a dict whose {\tt values()} is a list ``value'' and {\tt keys()} is a list ``key'' in the sense of above. 
  Also, there are some short-cut for inputting \param{key}:
  \begin{itemize}
    \item If key is $[1,\ 2, \ldots, N]$, you do not have to input \param{key}.
    \item If key is $[0,\ 1, \ldots, N]$, input $0$ as \param{key}.
    \item If key equals the list arranged through \param{value} in ascending order, input $1$.
    \item If key equals the list arranged through \param{value} in descending order, input $-1$.
  \end{itemize}
  \begin{at}
    \item[key]:\linkedtwo{permute}{Permute}{key}\\ It expresses \param{key}.
    \item[data]:\linkedtwo{permute}{Permute}{data}\\ \negok It expresses indexed form of \param{value}.
  \end{at}
  \C
  \begin{op}
    \verb+A==B+ & Check equality for A's value and B's one and A's key and B's one.\\
    \verb+A*B+ & right multiplication (that is, $A \circ B$ with normal mapping way)\\
    \verb+A/B+ & division (that is, $A \circ B^{-1}$)\\
    \verb+A**B+ & powering\\
    \verb+A.inverse()+ & inverse\\
    \verb+A[c]+ & the element of \param{value} corresponding to \param{c} in \param{key}\\
    \verb+A(lst)+ & permute \param{lst} with A\\
  \end{op}
\begin{ex}
>>> p1 = permute.Permute(['b','c','d','a','e'], ['a','b','c','d','e'])
>>> print p1
['a', 'b', 'c', 'd', 'e'] -> ['b', 'c', 'd', 'a', 'e']
>>> p2 = permute.Permute([2, 3, 0, 1, 4], 0)
>>> print p2
[0, 1, 2, 3, 4] -> [2, 3, 0, 1, 4]
>>> p3 = permute.Permute(['c','a','b','e','d'], 1)
>>> print p3
['a', 'b', 'c', 'd', 'e'] -> ['c', 'a', 'b', 'e', 'd']
>>> print p1 * p3
['a', 'b', 'c', 'd', 'e'] -> ['d', 'b', 'c', 'e', 'a']
>>> print p3 * p1
['a', 'b', 'c', 'd', 'e'] -> ['a', 'b', 'e', 'c', 'd']
>>> print p1 ** 4
['a', 'b', 'c', 'd', 'e'] -> ['a', 'b', 'c', 'd', 'e']
>>> p1['d']
'a'
>>> p2([0, 1, 2, 3, 4])
[2, 3, 0, 1, 4]
\end{ex}%Don't indent!
  \method
  \subsubsection{setKey -- change key}\linkedtwo{permute}{Permute}{setKey}
   \func{setKey}{\param{self},\ \hiki{key}{list/tuple}}{\out{Permute}}\\
   \spacing
   % document of basic document
   \quad Set other key.\\
   \spacing
   % added document
   %\spacing
   % input, output document
   \quad \param{key} must be list or tuple with same length to \linkingtwo{permute}{Permute}{key}.\\
 \subsubsection{getValue -- get ``value''}\linkedtwo{permute}{Permute}{getValue}
   \func{getValue}{\param{self}}{\out{list}}\\
   \spacing
   % document of basic document
   \quad Return (not \param{data}) \param{value} of \param{self}.\\
   \spacing
   % added document
   %\spacing
   % input, output document
 \subsubsection{getGroup -- get PermGroup}\linkedtwo{permute}{Permute}{getGroup}
   \func{getGroup}{\param{self}}{\out{PermGroup}}\\
   \spacing
   % document of basic document
   \quad Return \linkingone{permute}{PermGroup} to which \param{self} belongs.\\
   \spacing
   % added document
   %\spacing
   % input, output document
 \subsubsection{numbering -- give the index}\linkedtwo{permute}{Permute}{numbering}
   \func{numbering}{\param{self}}{\out{int}}\\
   \spacing
   % document of basic document
   \quad Number \param{self} in the permutation group. (Slow method)\\
   \spacing
   % added document
   \quad The numbering is made to fit the following inductive definition for dimension of the permutation group.\\
   If numbering of $[\sigma_1,\ \sigma_2,...,\sigma_{n-2},\ \sigma_{n-1}]$ on $(n-1)$-dimension is $k$,
   numbering of $[\sigma_1,\ \sigma_2,...,\sigma_{n-2},\sigma_{n-1},n]$ on $n$-dimension is $k$ and
   numbering of $[\sigma_1,\ \sigma_2,...,\sigma_{n-2},\ n,\ \sigma_{n-1}]$ on $n$-dimension is $k+(n-1)!$, and so on.
   (See \href{http://www32.ocn.ne.jp/~graph_puzzle/2no15.htm}{Room of Points And Lines, part 2, section 15, paragraph 2 (Japanese)})\\
   %\spacing
   % input, output document
 \subsubsection{order -- order of the element}\linkedtwo{permute}{Permute}{order}
   \func{order}{\param{self}}{\out{int/long}}\\
   \spacing
   % document of basic document
   \quad Return order as the element of group.\\
   \spacing
   % added document
   \quad This method is faster than general group method.
   \spacing
   % input, output document
 \subsubsection{ToTranspose -- represent as transpositions}\linkedtwo{permute}{Permute}{ToTranspose}
   \func{ToTranspose}{\param{self}}{\out{ExPermute}}\\
   \spacing
   % document of basic document
   \quad Represent \param{self} as a composition of transpositions.\\
   \spacing
   % added document
   \quad Return the element of \linkingone{permute}{ExPermute} with transpose (2-dimensional cyclic) type.
   It is recursive program, and it would take more time than the method \linkingtwo{permute}{Permute}{ToCyclic}.\\
   \spacing
   % input, output document
 \subsubsection{ToCyclic -- corresponding ExPermute element}\linkedtwo{permute}{Permute}{ToCyclic}
   \func{ToTranspose}{\param{self}}{\out{ExPermute}}\\
   \spacing
   % document of basic document
   \quad Represent \param{self} as a composition of cyclic representations.\\
   \spacing
   % added document
   \quad Return the element of \linkingone{permute}{ExPermute}.
   \negok This method decomposes \param{self} into coprime cyclic permutations, so each cyclic is commutative.
   \spacing
   % input, output document
 \subsubsection{sgn -- sign of the permutation}\linkedtwo{permute}{Permute}{sgn}
   \func{sgn}{\param{self}}{\out{int}}\\
   \spacing
   % document of basic document
   \quad Return the sign of permutation group element.\\
   \spacing
   % added document
   \quad If \param{self} is even permutation, that is, \param{self} can be written as a composition of an even number of transpositions, it returns $1$.
   Otherwise,that is, for odd permutation, it returns $-1$.
   \spacing
   % input, output document
 \subsubsection{types -- type of cyclic representation}\linkedtwo{permute}{Permute}{types}
   \func{types}{\param{self}}{\out{str}}\\
   \spacing
   % document of basic document
   \quad Return cyclic type defined by each cyclic permutation element length.\\
   \spacing
   % added document
   %\spacing
   % input, output document
 \subsubsection{ToMatrix -- permutation matrix}\linkedtwo{permute}{Permute}{ToMatrix}
   \func{ToMatrix}{\param{self}}{\out{\linkingone{matrix}{Matrix}}}\\
   \spacing
   % document of basic document
   \quad Return permutation matrix.\\
   \spacing
   % added document
   \quad The row and column correspond to \param{key}.
   If \param{self} $G$ satisfies $G[a]=b$, then $(a,\ b)$-element of the matrix is $1$.
   Otherwise, the element is $0$.
   %\spacing
   % input, output document
\begin{ex}
>>> p=Permute([2,3,1,5,4])
>>> p.numbering()
28
>>> p.order()
6
>>> p.ToTranspose()
[(4,5)(1,3)(1,2)](5)
>>> p.sgn()
-1
>>> p.ToCyclic()
[(1,2,3)(4,5)](5)
>>> p.types()
'(2,3)type'
>>> print p.ToMatrix()
0 1 0 0 0
0 0 1 0 0
1 0 0 0 0
0 0 0 0 1
0 0 0 1 0
\end{ex}%Don't indent!
\C

 \subsection{ExPermute -- element of permutation group as cyclic representation}\linkedone{permute}{ExPermute}
 \initialize
  \func{ExPermute}{\hiki{dim}{int},\ \hiki{value}{list},\ \hikiopt{key}{list}{None}}{ExPermute}\\
  \spacing
  % document of basic document
  \quad Create an element of a permutation group.\\
  \spacing
  % added document
  \quad An instance will be generated with ``cyclic'' way.
  That is, we input a \param{key}, which is a list of tuples and each tuple expresses a cyclic permutation.
  For example, $(\sigma_1,\ \sigma_2,\ \sigma_3,\ldots,\sigma_k)$ is one-to-one mapping, $\sigma_1 \mapsto \sigma_2,\ \sigma_2 \mapsto \sigma_3,\ldots,\sigma_k \mapsto \sigma_1$.\\
  \spacing
  % input, output document
  \quad \param{dim} must be positive integer, that is, an instance of int, long or \linkedone{rational}{Integer}. 
  \param{key} should be a list whose length equals \param{dim}.
  Input a list of tuples whose elements are in \param{key} as \param{value}.
  Note that you can abbreviate \param{key} if \param{key} has the form $[1,\ 2,\ldots,N]$.
  Also, you can input $0$ as \param{key} if  \param{key} has the form $[0,\ 2,\ldots,N-1]$.
  \begin{at}
    \item{dim}:\linkedtwo{permute}{ExPermute}{dim}\\ It expresses \param{dim}.
    \item[key]:\linkedtwo{permute}{ExPermute}{key}\\ It expresses \param{key}.
    \item[data]:\linkedtwo{permute}{ExPermute}{data}\\ \negok It expresses indexed form of \param{value}.
  \end{at}
  \begin{op}
    \verb+A==B+ & Check equality for A's value and B's one and A's key and B's one.\\
    \verb+A*B+ & right multiplication (that is, $A \circ B$ with normal mapping way)\\
    \verb+A/B+ & division (that is, $A \circ B^{-1}$)\\
    \verb+A**B+ & powering\\
    \verb+A.inverse()+ & inverse\\
    \verb+A[c]+ & the element of \param{value} corresponding to \param{c} in \param{key}\\
    \verb+A(lst)+ & permute \param{lst} with A\\
    \verb+str(A)+ & simple representation. use \linkingtwo{permute}{ExPermute}{simplify}.\\
    \verb+repr(A)+ & representation \\
  \end{op}
\begin{ex}
>>> p1 = permute.ExPermute(5, [('a', 'b')], ['a','b','c','d','e'])
>>> print p1
[('a', 'b')] <['a', 'b', 'c', 'd', 'e']>
>>> p2 = permute.ExPermute(5, [(0, 2), (3, 4, 1)], 0)
>>> print p2
[(0, 2), (1, 3, 4)] <[0, 1, 2, 3, 4]>
>>> p3=nzmath.permute.ExPermute(5,[('b','c')],['a','b','c','d','e'])
>>> print p1 * p3
[('a', 'b'), ('b', 'c')] <['a', 'b', 'c', 'd', 'e']>
>>> print p3 * p1
[('b', 'c'), ('a', 'b')] <['a', 'b', 'c', 'd', 'e']>
>>> p1['c']
'c'
>>> p2([0, 1, 2, 3, 4])
[2, 4, 0, 1, 3]
\end{ex}%Don't indent!
  \method
  \subsubsection{setKey -- change key}\linkedtwo{permute}{ExPermute}{setKey}
   \func{setKey}{\param{self},\ \hiki{key}{list}}{\out{ExPermute}}\\
   \spacing
   % document of basic document
   \quad Set other key.\\
   \spacing
   % added document
   %\spacing
   % input, output document
   \quad \param{key} must be a list whose length equals \linkingtwo{permute}{ExPermute}{dim}.\\
 \subsubsection{getValue -- get ``value''}\linkedtwo{permute}{ExPermute}{getValue}
   \func{getValue}{\param{self}}{\out{list}}\\
   \spacing
   % document of basic document
   \quad Return (not \param{data}) \param{value} of \param{self}.\\
   \spacing
   % added document
   %\spacing
   % input, output document
 \subsubsection{getGroup -- get PermGroup}\linkedtwo{permute}{ExPermute}{getGroup}
   \func{getGroup}{\param{self}}{\out{PermGroup}}\\
   \spacing
   % document of basic document
   \quad Return \linkingone{permute}{PermGroup} to which \param{self} belongs.\\
   \spacing
   % added document
   %\spacing
   % input, output document
 \subsubsection{order -- order of the element}\linkedtwo{permute}{ExPermute}{order}
   \func{order}{\param{self}}{\out{int/long}}\\
   \spacing
   % document of basic document
   \quad Return order as the element of group.\\
   \spacing
   % added document
   \quad This method is faster than general group method.\\
   \spacing
   % input, output document
 \subsubsection{ToNormal -- represent as normal style}\linkedtwo{permute}{ExPermute}{ToNormal}
   \func{ToNormal}{\param{self}}{\out{Permute}}\\
   \spacing
   % document of basic document
   \quad Represent \param{self} as an instance of \linkingone{permute}{Permute}.\\
   \spacing
   % added document
   %\spacing
   % input, output document
 \subsubsection{simplify -- use simple value}\linkedtwo{permute}{ExPermute}{simplify}
   \func{simplify}{\param{self}}{\out{ExPermute}}\\
   \spacing
   % document of basic document
   \quad Return the more simple cyclic element.\\
   \spacing
   % added document
   \quad \negok This method uses \linkingtwo{permute}{ExPermute}{ToNormal} and \linkingtwo{permute}{Permute}{ToCyclic}.
   \spacing
   % input, output document
 \subsubsection{sgn -- sign of the permutation}\linkedtwo{permute}{ExPermute}{sgn}
   \func{sgn}{\param{self}}{\out{int}}\\
   \spacing
   % document of basic document
   \quad Return the sign of permutation group element.\\
   \spacing
   % added document
   \quad If \param{self} is even permutation, that is, \param{self} can be written as a composition of an even number of transpositions, it returns $1$.
   Otherwise,that is, for odd permutation, it returns $-1$.\\
   \spacing
   % input, output document
\begin{ex}
>>> p=permute.ExPermute(5, [(1, 2, 3), (4, 5)])
>>> p.order()
6
>>> print p.ToNormal()
[1, 2, 3, 4, 5] -> [2, 3, 1, 5, 4]
>>> p * p
[(1, 2, 3), (4, 5), (1, 2, 3), (4, 5)] <[1, 2, 3, 4, 5]>
>>> (p * p).simplify()
[(1, 3, 2)] <[1, 2, 3, 4, 5]>
\end{ex}%Don't indent!
\C

 \subsection{PermGroup -- permutation group}\linkedone{permute}{PermGroup}
 \initialize
  \func{PermGroup}{\hiki{key}{int/long}}{PermGroup}\\
  \func{PermGroup}{\hiki{key}{list/tuple}}{PermGroup}\\
  \spacing
  % document of basic document
  \quad Create a permutation group.\\
  \spacing
  % added document
  % \spacing
  % input, output document
  \quad Normally, input list as \param{key}.
  If you input some integer $N$, \param{key} is set as $[1,\ 2,\ldots,N]$. 
  \begin{at}
    \item[key]:\linkedtwo{permute}{PermGroup}{key}\\ It expresses \param{key}.
  \end{at}
  \begin{op}
    \verb+A==B+ & Check equality for A's value and B's one and A's key and B's one.\\
    \verb+card(A)+ & same as \linkingtwo{permute}{PermGroup}{grouporder}\\ 
    \verb+str(A)+ & simple representation\\
    \verb+repr(A)+ & representation \\
  \end{op}
\begin{ex}
>>> p1 = permute.PermGroup(['a','b','c','d','e'])
>>> print p1
['a','b','c','d','e']
>>> card(p1)
120L
\end{ex}%Don't indent!
  \method
  \subsubsection{createElement -- create an element from seed}\linkedtwo{permute}{PermGroup}{createElement}
   \func{createElement}{\param{self},\ \hiki{seed}{list/tuple/dict}}{\out{Permute}}\\
   \func{createElement}{\param{self},\ \hiki{seed}{list}}{\out{ExPermute}}\\
   \spacing
   % document of basic document
   \quad Create new element in \param{self}.\\
   \spacing
   % added document
   %\spacing
   % input, output document
   \quad \param{seed} must be the form of ``value'' on \linkingone{permute}{Permute} or \linkingone{permute}{ExPermute}
 \subsubsection{identity -- group identity}\linkedtwo{permute}{PermGroup}{identity}
   \func{identity}{\param{self}}{\out{Permute}}\\
   \spacing
   % document of basic document
   \quad Return the identity of \param{self} as normal type.\\
   \spacing
   % added document
   \quad For cyclic type, use \linkingtwo{permute}{PermGroup}{identity\_c}.
   \spacing
   % input, output document
 \subsubsection{identity\_c -- group identity as cyclic type}\linkedtwo{permute}{PermGroup}{identity\_c}
   \func{identity\_c}{\param{self}}{\out{ExPermute}}\\
   \spacing
   % document of basic document
   \quad Return permutation group identity as cyclic type.\\
   \spacing
   % added document
   \quad For normal type, use \linkingtwo{permute}{PermGroup}{identity}.
   \spacing
   % input, output document
 \subsubsection{grouporder -- order as group}\linkedtwo{permute}{PermGroup}{grouporder}
   \func{grouporder}{\param{self}}{\out{int/long}}\\
   \spacing
   % document of basic document
   \quad Compute the order of \param{self} as group.\\
   \spacing
   % added document
   %\spacing
   % input, output document
 \subsubsection{randElement -- random permute element}\linkedtwo{permute}{PermGroup}{randElement}
   \func{randElement}{\param{self}}{\out{Permute}}\\
   \spacing
   % document of basic document
   \quad Create random new element as normal type in \param{self}.\\
   \spacing
   % added document
   %\spacing
   % input, output document
\begin{ex}
>>> p=permute.PermGroup(5)
>>> print p.createElement([3, 4, 5, 1, 2])
[1, 2, 3, 4, 5] -> [3, 4, 5, 1, 2]
>>> print p.createElement([(1, 2), (3, 4)])
[(1, 2), (3, 4)] <[1, 2, 3, 4, 5]>
>>> print p.identity()
[1, 2, 3, 4, 5] -> [1, 2, 3, 4, 5]
>>> print p.identity_c()
[] <[1, 2, 3, 4, 5]>
>>> p.grouporder()
120L
>>> print p.randElement()
[1, 2, 3, 4, 5] -> [3, 4, 5, 2, 1]
\end{ex}%Don't indent!
\C
%---------- end document ---------- %

\bibliographystyle{jplain}
\bibliography{nzmath_references}

\end{document}
