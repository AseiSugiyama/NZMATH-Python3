\documentclass{report}

%%%%%%%%%%%%%%%%%%%%%%%%%%%%%%%%%%%%%%%%%%%%%%%%%%%%%%%%%%%%%
%
% macros for nzmath manual
%
%%%%%%%%%%%%%%%%%%%%%%%%%%%%%%%%%%%%%%%%%%%%%%%%%%%%%%%%%%%%%
\usepackage{amssymb,amsmath}
\usepackage{color}
\usepackage[dvipdfm,bookmarks=true,bookmarksnumbered=true,%
 pdftitle={NZMATH Users Manual},%
 pdfsubject={Manual for NZMATH Users},%
 pdfauthor={NZMATH Development Group},%
 pdfkeywords={TeX; dvipdfmx; hyperref; color;},%
 colorlinks=true]{hyperref}
\usepackage{fancybox}
\usepackage[T1]{fontenc}
%
\newcommand{\DS}{\displaystyle}
\newcommand{\C}{\clearpage}
\newcommand{\NO}{\noindent}
\newcommand{\negok}{$\dagger$}
\newcommand{\spacing}{\vspace{1pt}\\ }
% software macros
\newcommand{\nzmathzero}{{\footnotesize $\mathbb{N}\mathbb{Z}$}\texttt{MATH}}
\newcommand{\nzmath}{{\nzmathzero}\ }
\newcommand{\pythonzero}{$\mbox{\texttt{Python}}$}
\newcommand{\python}{{\pythonzero}\ }
% link macros
\newcommand{\linkingzero}[1]{{\bf \hyperlink{#1}{#1}}}%module
\newcommand{\linkingone}[2]{{\bf \hyperlink{#1.#2}{#2}}}%module,class/function etc.
\newcommand{\linkingtwo}[3]{{\bf \hyperlink{#1.#2.#3}{#3}}}%module,class,method
\newcommand{\linkedzero}[1]{\hypertarget{#1}{}}
\newcommand{\linkedone}[2]{\hypertarget{#1.#2}{}}
\newcommand{\linkedtwo}[3]{\hypertarget{#1.#2.#3}{}}
\newcommand{\linktutorial}[1]{\href{http://docs.python.org/tutorial/#1}{#1}}
\newcommand{\linktutorialone}[2]{\href{http://docs.python.org/tutorial/#1}{#2}}
\newcommand{\linklibrary}[1]{\href{http://docs.python.org/library/#1}{#1}}
\newcommand{\linklibraryone}[2]{\href{http://docs.python.org/library/#1}{#2}}
\newcommand{\pythonhp}{\href{http://www.python.org/}{\python website}}
\newcommand{\nzmathwiki}{\href{http://nzmath.sourceforge.net/wiki/}{{\nzmathzero}Wiki}}
\newcommand{\nzmathsf}{\href{http://sourceforge.net/projects/nzmath/}{\nzmath Project Page}}
\newcommand{\nzmathtnt}{\href{http://tnt.math.se.tmu.ac.jp/nzmath/}{\nzmath Project Official Page}}
% parameter name
\newcommand{\param}[1]{{\tt #1}}
% function macros
\newcommand{\hiki}[2]{{\tt #1}:\ {\em #2}}
\newcommand{\hikiopt}[3]{{\tt #1}:\ {\em #2}=#3}

\newdimen\hoge
\newdimen\truetextwidth
\newcommand{\func}[3]{%
\setbox0\hbox{#1(#2)}
\hoge=\wd0
\truetextwidth=\textwidth
\advance \truetextwidth by -2\oddsidemargin
\ifdim\hoge<\truetextwidth % short form
{\bf \colorbox{skyyellow}{#1(#2)\ $\to$ #3}}
%
\else % long form
\fcolorbox{skyyellow}{skyyellow}{%
   \begin{minipage}{\textwidth}%
   {\bf #1(#2)\\ %
    \qquad\quad   $\to$\ #3}%
   \end{minipage}%
   }%
\fi%
}

\newcommand{\out}[1]{{\em #1}}
\newcommand{\initialize}{%
  \paragraph{\large \colorbox{skyblue}{Initialize (Constructor)}}%
    \quad\\ %
    \vspace{3pt}\\
}
\newcommand{\method}{\C \paragraph{\large \colorbox{skyblue}{Methods}}}
% Attribute environment
\newenvironment{at}
{%begin
\paragraph{\large \colorbox{skyblue}{Attribute}}
\quad\\
\begin{description}
}%
{%end
\end{description}
}
% Operation environment
\newenvironment{op}
{%begin
\paragraph{\large \colorbox{skyblue}{Operations}}
\quad\\
\begin{table}[h]
\begin{center}
\begin{tabular}{|l|l|}
\hline
operator & explanation\\
\hline
}%
{%end
\hline
\end{tabular}
\end{center}
\end{table}
}
% Examples environment
\newenvironment{ex}%
{%begin
\paragraph{\large \colorbox{skyblue}{Examples}}
\VerbatimEnvironment
\renewcommand{\EveryVerbatim}{\fontencoding{OT1}\selectfont}
\begin{quote}
\begin{Verbatim}
}%
{%end
\end{Verbatim}
\end{quote}
}
%
\definecolor{skyblue}{cmyk}{0.2, 0, 0.1, 0}
\definecolor{skyyellow}{cmyk}{0.1, 0.1, 0.5, 0}
%
%\title{NZMATH User Manual\\ {\large{(for version 1.0)}}}
%\date{}
%\author{}
\begin{document}
%\maketitle
%
\setcounter{tocdepth}{3}
\setcounter{secnumdepth}{3}


\tableofcontents
\C

\chapter{Classes}


%---------- start document ---------- %
 \section{poly.univar -- univariate polynomial}\linkedzero{poly.univar}
 \begin{itemize}
   \item {\bf Classes}
   \begin{itemize}
     \item \negok\linkingone{poly.univar}{PolynomialInterface}
     \item \negok\linkingone{poly.univar}{BasicPolynomial}
     \item \linkingone{poly.univar}{SortedPolynomial}
   \end{itemize}
 \end{itemize}

 This poly.univar using following type:
 \begin{description}
   \item[polynomial]\linkedone{poly.univar}{polynomial}:\\
     \param{polynomial} is an instance of some descendant class of
     \linkingone{poly.univar}{PolynomialInterface}
     in this context.
 \end{description}

\C

 \subsection{PolynomialInterface -- base class for all univariate polynomials}\linkedone{poly.univar}{PolynomialInterface}
 \initialize
  Since the interface is an abstract class, do not instantiate.\\
  The class is derived from \linkingone{poly.formalsum}{FormalSumContainerInterface}.
  \spacing
%Some of attributes may be treated as a public one.
%  \begin{at}
%  \end{at}
  \begin{op}
    \verb+f * g+ & multiplication\footnote{in FormalSumContainerInterface, there is only scalar multiplication}\\
    \verb+f ** i+ & powering\\
  \end{op} 
  \method
  \subsubsection{differentiate -- formal differentiation}\linkedtwo{poly.univar}{PolynomialInterface}{differentiate}
   \func{differentiate}{\param{self}}{\out{polynomial}}\\
   \spacing
   % document of basic document
   \quad Return the formal differentiation of this polynomial.
 \subsubsection{downshift\_degree -- decreased degree polynomial}\linkedtwo{poly.univar}{PolynomialInterface}{downshift\_degree}
   \func{downshift\_degree}{\param{self},\ \hiki{slide}{integer}}{\out{polynomial}}\\
   \spacing
   % document of basic document
   \quad Return the polynomial obtained by shifting downward all terms
   with degrees of \param{slide.}

   Be careful that if the least degree term has the degree less than
   \param{slide} then the result is not mathematically a
   polynomial. Even in such a case, the method does not raise an
   exception.\\
   \spacing
   % added document
   \quad \negok {\tt f.downshift\_degree(slide)} is equivalent to
   {\tt f.\linkingtwo{poly.univar}{PolynomialInterface}{upshift\_degree}(-slide)}.

 \subsubsection{upshift\_degree -- increased degree polynomial}\linkedtwo{poly.univar}{PolynomialInterface}{upshift\_degree}
   \func{upshift\_degree}{\param{self},\ \hiki{slide}{integer}}{\out{polynomial}}\\
   \spacing
   % document of basic document
   \quad Return the polynomial obtained by shifting upward all terms
   with degrees of \param{slide}.
   \spacing
   % added document
   \quad \negok {\tt f.upshift\_degree(slide)} is equivalent to
   {\tt f.term\_mul((slide, 1))}.

   \subsubsection{ring\_mul -- multiplication in the ring}\linkedtwo{poly.univar}{PolynomialInterface}{ring\_mul}
   \func{ring\_mul}{\param{self},\ \hiki{other}{polynomial}}{\out{polynomial}}\\
   \spacing
   % document of basic document
   Return the result of multiplication with the \param{other} polynomial.

   \subsubsection{scalar\_mul -- multiplication with a scalar}\linkedtwo{poly.univar}{PolynomialInterface}{scalar\_mul}
   \func{scalar\_mul}{\param{self},\ \hiki{scale}{scalar}}{\out{polynomial}}\\
   \spacing
   Return the result of multiplication by scalar \param{scale}.

   \subsubsection{term\_mul -- multiplication with a term}\linkedtwo{poly.univar}{PolynomialInterface}{term\_mul}
   \func{term\_mul}{\param{self},\ \hiki{term}{term}}{\out{polynomial}}\\
   \spacing
   Return the result of multiplication with the given \param{term}.
   The \param{term} can be given as a tuple {\tt (degree, coeff)} or as a {\tt polynomial}.

   \subsubsection{square -- multiplication with itself}\linkedtwo{poly.univar}{PolynomialInterface}{square}
   \func{square}{\param{self}}{\out{polynomial}}\\
   Return the square of this polynomial.

%
 \subsection{BasicPolynomial -- basic implementation of polynomial}\linkedone{poly.univar}{BasicPolynomial}
 Basic polynomial data type.
 There are no concept such as variable name and ring.

  \initialize
  \func{BasicPolynomial}{\hiki{coefficients}{terminit},\ %
    **\hiki{keywords}{dict}}{\out{BasicPolynomial}}\\
  \spacing
  \quad This class inherits and implements \linkingone{poly.univar}{PolynomialInterface}.
  \spacing
  \quad The type of the \param{coefficients} is \linkingone{poly.formalsum}{terminit}.

 \subsection{SortedPolynomial -- polynomial keeping terms sorted}\linkedone{poly.univar}{SortedPolynomial}
 \initialize
  \func{SortedPolynomial}{\hiki{coefficients}{terminit},\ %
    \hikiopt{\_sorted}{bool}{False},\ %
    **\hiki{keywords}{dict}}{\out{SortedPolynomial}}\\
  The class is derived from \linkingone{poly.univar}{PolynomialInterface}.
  \spacing
  \quad The type of the \param{coefficients} is \linkingone{poly.formalsum}{terminit}.
  Optionally \param{\_sorted} can be {\tt True} if the coefficients is an
  already sorted list of terms.

  \method
  \subsubsection{degree -- degree}\linkedtwo{poly.univar}{SortedPolynomial}{degree}
   \func{degree}{\param{self}}{\out{integer}}\\
   \spacing
   % document of basic document
   Return the degree of this polynomial.
   If the polynomial is the zero polynomial, the degree is \(-1\).

  \subsubsection{leading\_coefficient -- the leading coefficient}\linkedtwo{poly.univar}{SortedPolynomial}{leading\_coefficient}
  \func{leading\_coefficient}{\param{self}}{\out{object}}\\

  Return the coefficient of highest degree term.

  \subsubsection{leading\_term -- the leading term}\linkedtwo{poly.univar}{SortedPolynomial}{leading\_term}
  \func{leading\_term}{\param{self}}{\out{tuple}}\\

  Return the leading term as a tuple {\tt (degree, coefficient)}.

  \subsubsection{\negok ring\_mul\_karatsuba -- the leading term}\linkedtwo{poly.univar}{SortedPolynomial}{ring\_mul\_karatsuba}
  \func{ring\_mul\_karatsuba}{\param{self},\ \hiki{other}{polynomial}}{\out{polynomial}}\\

  Multiplication of two polynomials in the same ring.
  Computation is carried out by Karatsuba method.

  This may run faster when degree is higher than 100 or so.
  It is off by default, if you need to use this, do by yourself.
\C

%---------- end document ---------- %

\bibliographystyle{jplain}%use jbibtex
\bibliography{nzmath_references}

\end{document}

