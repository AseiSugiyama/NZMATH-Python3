\documentclass{report}

\documentclass{report}

%%%%%%%%%%%%%%%%%%%%%%%%%%%%%%%%%%%%%%%%%%%%%%%%%%%%%%%%%%%%%
%
% macros for nzmath manual
%
%%%%%%%%%%%%%%%%%%%%%%%%%%%%%%%%%%%%%%%%%%%%%%%%%%%%%%%%%%%%%
\usepackage{amssymb,amsmath}
\usepackage{color}
\usepackage[dvipdfm,bookmarks=true,bookmarksnumbered=true,%
 pdftitle={NZMATH Users Manual},%
 pdfsubject={Manual for NZMATH Users},%
 pdfauthor={NZMATH Development Group},%
 pdfkeywords={TeX; dvipdfmx; hyperref; color;},%
 colorlinks=true]{hyperref}
\usepackage{fancybox}
\usepackage[T1]{fontenc}
%
\newcommand{\DS}{\displaystyle}
\newcommand{\C}{\clearpage}
\newcommand{\NO}{\noindent}
\newcommand{\negok}{$\dagger$}
\newcommand{\spacing}{\vspace{1pt}\\ }
% software macros
\newcommand{\nzmathzero}{{\footnotesize $\mathbb{N}\mathbb{Z}$}\texttt{MATH}}
\newcommand{\nzmath}{{\nzmathzero}\ }
\newcommand{\pythonzero}{$\mbox{\texttt{Python}}$}
\newcommand{\python}{{\pythonzero}\ }
% link macros
\newcommand{\linkingzero}[1]{{\bf \hyperlink{#1}{#1}}}%module
\newcommand{\linkingone}[2]{{\bf \hyperlink{#1.#2}{#2}}}%module,class/function etc.
\newcommand{\linkingtwo}[3]{{\bf \hyperlink{#1.#2.#3}{#3}}}%module,class,method
\newcommand{\linkedzero}[1]{\hypertarget{#1}{}}
\newcommand{\linkedone}[2]{\hypertarget{#1.#2}{}}
\newcommand{\linkedtwo}[3]{\hypertarget{#1.#2.#3}{}}
\newcommand{\linktutorial}[1]{\href{http://docs.python.org/tutorial/#1}{#1}}
\newcommand{\linktutorialone}[2]{\href{http://docs.python.org/tutorial/#1}{#2}}
\newcommand{\linklibrary}[1]{\href{http://docs.python.org/library/#1}{#1}}
\newcommand{\linklibraryone}[2]{\href{http://docs.python.org/library/#1}{#2}}
\newcommand{\pythonhp}{\href{http://www.python.org/}{\python website}}
\newcommand{\nzmathwiki}{\href{http://nzmath.sourceforge.net/wiki/}{{\nzmathzero}Wiki}}
\newcommand{\nzmathsf}{\href{http://sourceforge.net/projects/nzmath/}{\nzmath Project Page}}
\newcommand{\nzmathtnt}{\href{http://tnt.math.metro-u.ac.jp/nzmath/}{\nzmath Project Official Page}}
% parameter name
\newcommand{\param}[1]{{\tt #1}}
% function macros
\newcommand{\hiki}[2]{{\tt #1}:\ {\em #2}}
\newcommand{\hikiopt}[3]{{\tt #1}:\ {\em #2}=#3}

\newdimen\hoge
\newdimen\truetextwidth
\newcommand{\func}[3]{%
\setbox0\hbox{#1(#2)}
\hoge=\wd0
\truetextwidth=\textwidth
\advance \truetextwidth by -2\oddsidemargin
\ifdim\hoge<\truetextwidth % short form
{\bf \colorbox{skyyellow}{#1(#2)\ $\to$ #3}}
%
\else % long form
\fcolorbox{skyyellow}{skyyellow}{%
   \begin{minipage}{\textwidth}%
   {\bf #1(#2)\\ %
    \qquad\quad   $\to$\ #3}%
   \end{minipage}%
   }%
\fi%
}

\newcommand{\out}[1]{{\em #1}}
\newcommand{\initialize}{%
  \paragraph{\large \colorbox{skyblue}{Initialize (Constructor)}}%
    \quad\\ %
    \vspace{3pt}\\
}
\newcommand{\method}{\C \paragraph{\large \colorbox{skyblue}{Methods}}}
% Attribute environment
\newenvironment{at}
{%begin
\paragraph{\large \colorbox{skyblue}{Attribute}}
\quad\\
\begin{description}
}%
{%end
\end{description}
}
% Operation environment
\newenvironment{op}
{%begin
\paragraph{\large \colorbox{skyblue}{Operations}}
\quad\\
\begin{table}[h]
\begin{center}
\begin{tabular}{|l|l|}
\hline
operator & explanation\\
\hline
}%
{%end
\hline
\end{tabular}
\end{center}
\end{table}
}
% Examples environment
\newenvironment{ex}%
{%begin
\paragraph{\large \colorbox{skyblue}{Examples}}
\VerbatimEnvironment
\renewcommand{\EveryVerbatim}{\fontencoding{OT1}\selectfont}
\begin{quote}
\begin{Verbatim}
}%
{%end
\end{Verbatim}
\end{quote}
}
%
\definecolor{skyblue}{cmyk}{0.2, 0, 0.1, 0}
\definecolor{skyyellow}{cmyk}{0.1, 0.1, 0.5, 0}
%
%\title{NZMATH User Manual\\ {\large{(for version 1.0)}}}
%\date{}
%\author{}
\begin{document}
%\maketitle
%
\setcounter{tocdepth}{3}
\setcounter{secnumdepth}{3}


\tableofcontents
\C

\chapter{Classes}


%---------- start document ---------- %
 \negok \section{ring -- for ring object}\linkedzero{ring}
 \begin{itemize}
   \item {\bf Classes}
   \begin{itemize}
     \item \linkingone{ring}{Ring}
     \item \linkingone{ring}{CommutativeRing}
     \item \linkingone{ring}{Field}
     \item \linkingone{ring}{QuotientField}
     \item \linkingone{ring}{RingElement}
     \item \linkingone{ring}{CommutativeRingElement}
     \item \linkingone{ring}{FieldElement}
     \item \linkingone{ring}{QuotientFieldElement}
     \item \linkingone{ring}{Ideal}
     \item \linkingone{ring}{ResidueClassRing}
     \item \linkingone{ring}{ResidueClass}
     \item \linkingone{ring}{CommutativeRingProperties}
   \end{itemize}
   \item {\bf Functions}
     \begin{itemize}
       \item \linkingone{ring}{getRingInstance}
       \item \linkingone{ring}{getRing}
       \item \linkingone{ring}{inverse}
       \item \linkingone{ring}{exact\_division}
     \end{itemize}
 \end{itemize}

\C

 \subsection{\negok Ring -- abstract ring}\linkedone{ring}{Ring}
  %\func{Ring}{(None)}{Ring}\\
  \spacing
  % document of basic document
  \quad Ring is an abstract class which expresses that
    the derived classes are (in mathematical meaning) rings.\\
  \spacing
  % added document
  \quad Definition of ring (in mathematical meaning) is as follows:
  Ring is a structure with addition and multiplication. 
  It is an abelian group with addition, and a monoid with multiplication.
  The multiplication obeys the distributive law.\\
  \spacing
  % input, output document
  This class is abstract and cannot be instantiated.\\
  \begin{at}
    \item[zero]\linkedtwo{ring}{Ring}{zero}  additive unit\\
    \item[one]\linkedtwo{ring}{Ring}{one} multiplicative unit\\
  \end{at}
  \begin{op}
    \verb+A==B+ & Return whether M and N are equal or not.\\
  \end{op} 
  \method
  \subsubsection{createElement -- create an element}\linkedtwo{ring}{Ring}{createElement}
   \func{createElement}{\param{self},\ \hiki{seed}{(undefined)}}{\out{RingElement}}\\
   \spacing
   % document of basic document
   \quad Return an element of the ring with seed.\\
   \spacing
   % added document
   \quad This is an abstract method.\\
   \spacing
   % input, output document
  \subsubsection{getCharacteristic -- characteristic as ring}\linkedtwo{ring}{Ring}{getCharacteristic}
   \func{getCharacteristic}{\param{self}}{\out{integer}}\\
   \spacing
   % document of basic document
   \quad Return the characteristic of the ring.\\
   \spacing
   % added document
   \quad The Characteristic of a ring is the smallest positive integer $n$ 
   s.t. $na=0$ for any element $a$ of the ring, 
   or $0$ if there is no such natural number.\\
   This is an abstract method.\\
   \spacing
   % input, output document
  \subsubsection{issubring -- check subring}\linkedtwo{ring}{Ring}{issubring}
   \func{issubring}{\param{self},\ \hiki{other}{RingElement}}{\out{True/False}}\\
   \spacing
   % document of basic document
   \quad Report whether another ring contains the ring as a subring.\\
   \spacing
   % added document
   \quad This is an abstract method.\\
   \spacing
   % input, output document
  \subsubsection{issuperring -- check superring}\linkedtwo{ring}{Ring}{issuperring}
   \func{issuperring}{\param{self},\ \hiki{other}{RingElement}}{\out{True/False}}\\
   \spacing
   % document of basic document
   \quad Report whether the ring is a superring of another ring.\\
   \spacing
   % added document
   \quad This is an abstract method.\\
   \spacing
   % input, output document
  \subsubsection{getCommonSuperring -- get common ring}\linkedtwo{ring}{Ring}{issuperring}
   \func{getCommonSuperring}{\param{self},\ \hiki{other}{RingElement}}{\out{RingElement}}\\
   \spacing
   % document of basic document
   \quad Return common super ring of self and another ring.\\
   \spacing
   % added document
   \quad This method uses \linkingtwo{ring}{Ring}{issubring},\ \linkingtwo{ring}{Ring}{issuperring}.\\
   \spacing
   % input, output document
\C

 \subsection{\negok CommutativeRing -- abstract commutative ring}\linkedone{ring}{CommutativeRing}
  %\func{CommutativeRing}{(None)}{CommutativeRing}\\
  \spacing
  % document of basic document
  \quad CommutativeRing is an abstract subclass of \linkingone{ring}{Ring} whose multiplication is commutative.\\
  \spacing
  % added document
  \quad CommutativeRing is subclass of \linkingone{ring}{Ring}.\\
  There are some properties of commutative rings, algorithms should be chosen accordingly. To express such properties, there is a class \linkingone{ring}{CommutativeRingProperties}. 
  \\
  \spacing
  % input, output document
  This class is abstract and cannot be instantiated.\\
  \begin{at}
    \item[properties] an instance of \linkingone{ring}{CommutativeRingProperties}
  \end{at}
  \method
  \subsubsection{getQuotientField -- create quotient field}\linkedtwo{ring}{CommutativeRing}{getQuotientField}
   \func{getQuotientField}{\param{self}}{\out{QuotientField}}\\
   \spacing
   % document of basic document
   \quad Return the quotient field of the ring.\\
   \spacing
   % added document
   \quad This is an abstract method.\\
   If quotient field of \param{self} is not available, it should raise exception.
   \spacing
   % input, output document
  \subsubsection{isdomain -- check domain}\linkedtwo{ring}{CommutativeRing}{isdomain}
   \func{isdomain}{\param{self}}{\out{True/False/None}}\\
   \spacing
   % document of basic document
   \quad Return True if the ring is actually a domain,
        False if not, or None if uncertain.\\
   \spacing
   % added document
   %\quad 
   %\spacing
   % input, output document
  \subsubsection{isnoetherian -- check Noetherian domain}\linkedtwo{ring}{CommutativeRing}{isnoetherian}
   \func{isnoetherian}{\param{self}}{\out{True/False/None}}\\
   \spacing
   % document of basic document
   \quad Return True if the ring is actually a Noetherian
        domain, False if not, or None if uncertain.\\
   \spacing
   % added document
   %\quad 
   %\spacing
   % input, output document
  \subsubsection{isufd -- check UFD}\linkedtwo{ring}{CommutativeRing}{isufd}
   \func{isufd}{\param{self}}{\out{True/False/None}}\\
   \spacing
   % document of basic document
   \quad Return True if the ring is actually a unique
        factorization domain (UFD), False if not, or None if uncertain.\\
   \spacing
   % added document
   %\quad 
   %\spacing
   % input, output document
  \subsubsection{ispid -- check PID}\linkedtwo{ring}{CommutativeRing}{ispid}
   \func{ispid}{\param{self}}{\out{True/False/None}}\\
   \spacing
   % document of basic document
   \quad Return True if the ring is actually a principal
        ideal domain (PID), False if not, or None if uncertain.\\
   \spacing
   % added document
   %\quad 
   %\spacing
   % input, output document
  \subsubsection{iseuclidean -- check Euclidean domain}\linkedtwo{ring}{CommutativeRing}{iseuclidean}
   \func{iseuclidean}{\param{self}}{\out{True/False/None}}\\
   \spacing
   % document of basic document
   \quad Return True if the ring is actually a Euclidean
        domain, False if not, or None if uncertain.\\
   \spacing
   % added document
   %\quad 
   %\spacing
   % input, output document
  \subsubsection{isfield -- check field}\linkedtwo{ring}{CommutativeRing}{isfield}
   \func{isfield}{\param{self}}{\out{True/False/None}}\\
   \spacing
   % document of basic document
   \quad Return True if the ring is actually a field,
        False if not, or None if uncertain.\\
   \spacing
   % added document
   %\quad 
   %\spacing
   % input, output document
  \subsubsection{registerModuleAction -- register action as ring}\linkedtwo{ring}{CommutativeRing}{registerModuleAction}
   \func{registerModuleAction}{\param{self},\ \hiki{action\_ring}{RingElement},\ \hiki{action}{function}}{\out{(None)}}\\
   \spacing
   % document of basic document
   \quad Register a ring \param{action\_ring}, which act on the ring through
        \param{action} so the ring be an \param{action\_ring} module.\\
   \spacing
   % added document
   \quad See \linkingtwo{ring}{CommutativeRing}{hasaction},\ \linkingtwo{ring}{CommutativeRing}{getaction}.\\
   \spacing
   % input, output document
  \subsubsection{hasaction -- check if the action has}\linkedtwo{ring}{CommutativeRing}{hasaction}
   \func{hasaction}{\param{self},\ \hiki{action\_ring}{RingElement}}{\out{True/False}}\\
   \spacing
   % document of basic document
   \quad Return True if \param{action\_ring} is registered to provide action.\\
   \spacing
   % added document
   \quad See \linkingtwo{ring}{CommutativeRing}{registerModuleAction},\ \linkingtwo{ring}{CommutativeRing}{getaction}.\\
   \spacing
   % input, output document
  \subsubsection{getaction -- get the registered action }\linkedtwo{ring}{CommutativeRing}{getaction}
   \func{hasaction}{\param{self},\ \hiki{action\_ring}{RingElement}}{\out{function}}\\
   \spacing
   % document of basic document
   \quad Return the registered action for \param{action\_ring}.\\
   \spacing
   % added document
   \quad See \linkingtwo{ring}{CommutativeRing}{registerModuleAction},\ \linkingtwo{ring}{CommutativeRing}{hasaction}.\\
   \spacing
   % input, output document
\C

 \subsection{\negok Field -- abstract field}\linkedone{ring}{Field}
  %\func{Field}{(None)}{Field}\\
  \spacing
  % document of basic document
  \quad Field is an abstract class which expresses that
    the derived classes are (in mathematical meaning) fields,
    i.e., a commutative ring whose multiplicative monoid is a group.
  \spacing
  % added document
  \quad Field is subclass of \linkingone{ring}{CommutativeRing}.
  \linkingtwo{ring}{Ring}{getQuotientField} and \linkingtwo{ring}{CommutativeRing}{isfield} are not abstract (trivial methods).\\
  \spacing
  % input, output document
  This class is abstract and cannot be instantiated.\\
  \method
  \subsubsection{gcd -- gcd}\linkedtwo{ring}{Field}{gcd}
   \func{gcd}{\param{self},\ \hiki{a}{FieldElement},\ \hiki{b}{FieldElement}}{\out{FieldElement}}\\
   \spacing
   % document of basic document
   \quad Return the greatest common divisor of \param{a} and \param{b}.\\
   \spacing
   % added document
   \quad  A field is trivially a UFD and should provide gcd.
   If we can implement an algorithm for computing gcd in an Euclidean domain, 
   we should provide the method corresponding to the algorithm. 
   \\
   \spacing
   % input, output document
\C

 \subsection{\negok QuotientField -- abstract quotient field}\linkedone{ring}{QuotientField}
  %\func{QuotientField}{\hiki{domain}{CommutativeRing Element}}{QuotientField}\\
  \spacing
  % document of basic document
  \quad QuotientField is an abstract class which expresses that
    the derived classes are (in mathematical meaning) quotient fields.\\
  \spacing
  % added document
  \quad \param{self} is the quotient field of \param{domain}.\\
  QuotientField is subclass of \linkingone{ring}{Field}.\\
  In the initialize step, it registers trivial action named as baseaction;
  i.e. it expresses that an element of a domain acts an element of the quotient field by using the multiplication in the domain.\\
  \spacing
  % input, output document
  This class is abstract and cannot be instantiated.\\
  \begin{at}
    \item[basedomain] domain which generates the quotient field \param{self}
  \end{at}
\C

 \subsection{\negok RingElement -- abstract element of ring}\linkedone{ring}{RingElement}
  %\func{RingElement}{\hiki{*args}{(undefined)},\ \hiki{*kwd}{(undefined)}}{RingElement}\\
  \spacing
  % document of basic document
  \quad RingElement is an abstract class for elements of rings.\\
  \spacing
  % added document
  %\quad \\
  %\spacing
  % input, output document
  This class is abstract and cannot be instantiated.\\
  \begin{op}
    \verb+A==B+ & equality (abstract) \\
  \end{op}
  \method
 \subsubsection{getRing -- getRing}\linkedtwo{ring}{RingElement}{getRing}
   \func{getRing}{\param{self}}{\out{Ring}}\\
   \spacing
   % document of basic document
   \quad Return an object of a subclass of Ring,
        to which the element belongs.\\
   \spacing
   % added document
   \quad  This is an abstract method.\\
   \spacing
   % input, output document
\C

 \subsection{\negok CommutativeRingElement -- abstract element of commutative ring}\linkedone{ring}{CommutativeRingElement}
  %\func{CommutativeRingElement}{(None)}{RingElement}\\
  \spacing
  % document of basic document
  \quad CommutativeRingElement is an abstract class for elements of
    commutative rings.\\
  \spacing
  % added document
  %\quad \\
  %\spacing
  % input, output document
  This class is abstract and cannot be instantiated.\\
  \method
 \subsubsection{mul\_module\_action -- apply a module action}\linkedtwo{ring}{CommutativeRingElement}{mul\_module\_action}
   \func{mul\_module\_action}{\param{self},\ \hiki{other}{RingElement}}{\out{(undefined)}}\\
   \spacing
   % document of basic document
   \quad Return the result of a module action.
        other must be in one of the action rings of self's ring.\\
   \spacing
   % added document
   \quad  This method uses \linkingtwo{ring}{RingElement}{getRing},\ \linkingone{ring}{CommutativeRing}{getaction}.
   We should consider that the method is abstract.\\
   \spacing
   % input, output document
 \subsubsection{exact\_division -- division exactly}\linkedtwo{ring}{CommutativeRingElement}{exact\_division}
   \func{exact\_division}{\param{self},\ \hiki{other}{CommutativeRingElement}}{\out{CommutativeRingElement}}\\
   \spacing
   % document of basic document
   \quad In UFD, if \param{other} divides \param{self},
   return the quotient as a UFD element. \\
   \spacing
   % added document
   \quad  The main difference with / is that / may return the
        quotient as an element of quotient field.\\
    Simple cases:
    \begin{itemize}
      \item in a Euclidean domain, 
         if remainder of euclidean division is zero, the division // is exact.
      \item in a field, there's no difference with /.
    \end{itemize}
   If \param{other} doesn't divide self, raise ValueError.
   Though \_\_divmod\_\_ can be used automatically,
   we should consider that the method is abstract.\\
   \spacing
   % input, output document 
\C

 \subsection{\negok FieldElement -- abstract element of field}\linkedone{ring}{FieldElement}
  %\func{FieldElement}{(None)}{FieldElement}\\
  \spacing
  % document of basic document
  \quad FieldElement is an abstract class for elements of
    fields.\\
  \spacing
  % added document
  \quad FieldElement is subclass of \linkingone{ring}{CommutativeRingElement}.
  \linkingtwo{ring}{CommutativeRingElement}{exact\_division} are not abstract (trivial methods).\\
  \spacing
  % input, output document
  This class is abstract and cannot be instantiated.\\
\C

\subsection{\negok QuotientFieldElement -- abstract element of quotient field}\linkedone{ring}{QuotientFieldElement}
  %\initialize
  %\func{QuotientFieldElement}{\hiki{numerator}{CommutativeRingElement},\ \hiki{denominator}{CommutativeRingElement}}{QuotientFieldElement}\\
  \spacing
  % document of basic document
  \quad QuotientFieldElement class is an abstract class to be used as a
    super class of concrete quotient field element classes.\\
  \spacing
  % added document
  \quad QuotientFieldElement is subclass of \linkingone{ring}{FieldElement}.\\
  \param{self} expresses $\frac{\mbox{\param{numerator}}}{\mbox{\param{denominator}}}$ in the quotient field.\\
  \spacing
  % input, output document
  \quad  This class is abstract and should not be instantiated.\\
  \param{denominator} should not be $0$.\\
 \begin{at}
    \item[numerator]\linkedtwo{ring}{QuotientField}{numerator} numerator of \param{self}\\
    \item[denominator]\linkedtwo{ring}{QuotientField}{denominator} denominator of \param{self}\\
  \end{at}
  \begin{op}
    \verb|A+B| & addition\\
    \verb+A-B+ & subtraction\\
    \verb+A*B+ & multiplication\\
    \verb+A**B+ & powering\\
    \verb+A/B+ & division\\
    \verb+-A+ & sign reversion (additive inversion)\\
    \verb+inverse(A)+ & multiplicative inversion\\
    \verb+A==B+ & equality\\
  \end{op}
\C

\subsection{\negok Ideal -- abstract ideal}\linkedone{ring}{Ideal}
  %\initialize
  %\func{Ideal}{\hiki{generators}{list},\ \hiki{aring}{CommutativeRing}}{Ideal}\\
  %\func{Ideal}{\hiki{generators}{CommutativeRingElement},\ \hiki{aring}{CommutativeRing}}{Ideal}\\
  \spacing
  % document of basic document
  \quad Ideal class is an abstract class to represent the finitely
    generated ideals.\\
  \spacing
  % added document
  \quad \negok Because the finitely-generatedness is not a
    restriction for Noetherian rings and in the most cases only
    Noetherian rings are used, it is general enough.\\
  \\
  \spacing
  % input, output document
  \quad  This class is abstract and should not be instantiated.\\
  \param{generators} must be an element of the \param{aring} or a list of elements of the \param{aring}.\\
  If \param{generators} is an element of the \param{aring}, we consider \param{self} is the principal ideal generated by \param{generators}.
 \begin{at}
    \item[ring]\linkedtwo{ring}{Ideal}{ring} the ring belonged to by \param{self}\\
    \item[generators]\linkedtwo{ring}{Ideal}{generators} generators of the ideal \param{self}\\
  \end{at}
  \begin{op}
    \verb|I+J| & addition $\{i+j\ |\ i \in I,\ j \in J\}$\\
    \verb+I*J+ & multiplication $IJ = \{ \sum_{i,j} ij\ |\ i \in I,\  j\in  J\}$\\
    \verb+I==J+ & equality\\
    \verb+e in I+ &  For \param{e} in the ring, to which the ideal \param{I} belongs.\\
  \end{op}
  \method
   \subsubsection{issubset -- check subset}\linkedtwo{ring}{Ideal}{issubset}
   \func{issubset}{\param{self},\ \hiki{other}{Ideal}}{\out{True/False}}\\
   \spacing
   % document of basic document
   \quad Report whether another ideal contains this ideal.\\
   \spacing
   % added document
   We should consider that the method is abstract.\\
   \spacing
   % input, output document
  \subsubsection{issuperset -- check superset}\linkedtwo{ring}{Ideal}{issuperset}
   \func{issuperset}{\param{self},\ \hiki{other}{Ideal}}{\out{True/False}}\\
   \spacing
   % document of basic document
   \quad Report whether this ideal contains another ideal.\\
   \spacing
   % added document
   We should consider that the method is abstract.\\
   \spacing
   % input, output document
  \subsubsection{reduce -- reduction with the ideal}\linkedtwo{ring}{Ideal}{reduce}
  \func{issuperset}{\param{self},\ \hiki{other}{Ideal}}{\out{True/False}}\\
   \spacing
   % document of basic document
   \quad Reduce an element with the ideal to simpler representative.\\
   \spacing
   % added document
   This method is abstract.\\
   \spacing
   % input, output document
\C

\subsection{\negok ResidueClassRing -- abstract residue class ring}\linkedone{ring}{ResidueClassRing}
  \initialize
  \func{ResidueClassRing}{\hiki{ring}{CommutativeRing},\ \hiki{ideal}{Ideal}}{ResidueClassRing}\\
  \spacing
  % document of basic document
  \quad A residue class ring $R/I$, where $R$ is a commutative ring and $I$ is its ideal.
  \spacing
  % added document
  \quad   ResidueClassRing is subclass of \linkingone{ring}{CommutativeRing}.\\
  \linkingtwo{ring}{Ring}{one},\ \linkingtwo{ring}{Ring}{zero} are not abstract (trivial methods).\\
  Because we assume that \param{ring} is Noetherian, so is \param{ring}.
  \spacing
  % input, output document
  \quad  This class is abstract and should not be instantiated.\\
  \param{ring} should be an instance of \linkingone{ring}{CommutativeRing},
  and \param{ideal} must be an instance of \linkingone{ring}{Ideal},
  whose ring attribute points the same ring with the given ring.
 \begin{at}
    \item[ring]\linkedtwo{ring}{ResidueClassRing}{ring} the base ring $R$\\
    \item[ideal]\linkedtwo{ring}{ResidueClassRing}{ideal} the ideal $I$\\
  \end{at}
  \begin{op}
     \verb+A==B+ & equality\\
    \verb+e in A+ &  report whether \param{e} is in the residue ring \param{self}.\\
  \end{op}
\C

\subsection{\negok ResidueClass -- abstract an element of residue class ring}\linkedone{ring}{ResidueClass}
  \initialize
  \func{ResidueClass}{\hiki{x}{CommutativeRingElement},\ \hiki{ideal}{Ideal}}{ResidueClass}\\
  \spacing
  % document of basic document
  \quad Element of residue class ring $x+I$, where $I$ is the modulus ideal
    and $x$ is a representative element.
  \spacing
  % added document
  \quad   ResidueClass is subclass of \linkingone{ring}{CommutativeRingElement}.\\
  \spacing
  % input, output document
  \quad  This class is abstract and should not be instantiated.\\
  \param{ideal} corresponds to the ideal $I$.
  \begin{op}
     \verb|x+y| & addition\\
     \verb+x-y+ & subtraction\\
     \verb+x*y+ & multiplication\\
     \verb+A==B+ & equality\\
  \end{op}
  These operations uses \linkingtwo{ring}{Ideal}{reduce}.
\C

\subsection{\negok CommutativeRingProperties -- properties for CommutativeRingProperties}\linkedone{ring}{CommutativeRingProperties}
  \initialize
  \func{CommutativeRingProperties}{(None)}{CommutativeRingProperties}\\
  \spacing
  % document of basic document
  \quad Boolean properties of ring.\\
  \spacing
  % added document
  \quad Each property can have one of three values; {\it True}, {\it False}, or {\it None}.
    Of course {\it True} is true and {\it False} is false,
    and {\it None} means that the property is
    not set neither directly nor indirectly.\\
    CommutativeRingProperties class treats
  \begin{itemize}
    \item Euclidean (Euclidean domain),
    \item PID (Principal Ideal Domain),
    \item UFD (Unique Factorization Domain),
    \item Noetherian (Noetherian ring (domain)),
    \item field (Field)
  \end{itemize}
  \quad\\
  \spacing
  % input, output document
  \method
  \subsubsection{isfield -- check field}\linkedtwo{ring}{CommutativeRingProperties}{isfield}
   \func{isfield}{\param{self}}{\out{True/False/None}}\\
   \spacing
   % document of basic document
   \quad Return True/False according to the field flag value being set,
        otherwise return None.\\
   \spacing
   % added document
   %\spacing
   % input, output document
  \subsubsection{setIsfield -- set field}\linkedtwo{ring}{CommutativeRingProperties}{setIsfield}
   \func{isfield}{\param{self},\ \hiki{value}{True/False}}{\out{(None)}}\\
   \spacing
   % document of basic document
   \quad Set True/False value to the field flag.\\
   Propagation:
   \begin{itemize}
     \item True $\to$ euclidean\\
   \end{itemize}
   \quad\\
   \spacing
   % added document
   %\spacing
   % input, output document
  \subsubsection{iseuclidean -- check euclidean}\linkedtwo{ring}{CommutativeRingProperties}{iseuclidean}
   \func{iseuclidean}{\param{self}}{\out{True/False/None}}\\
   \spacing
   % document of basic document
   \quad Return True/False according to the euclidean flag value being set,
        otherwise return None.\\
   \spacing
   % added document
   %\spacing
   % input, output document
  \subsubsection{setIseuclidean -- set euclidean}\linkedtwo{ring}{CommutativeRingProperties}{setIseuclidean}
   \func{isfield}{\param{self},\ \hiki{value}{True/False}}{\out{(None)}}\\
   \spacing
   % document of basic document
   \quad Set True/False value to the euclidean flag.\\
   Propagation:
   \begin{itemize}
    \item True $\to$ PID\\
    \item False $\to$ field\\
   \end{itemize}
   \quad\\
   \spacing
   % added document
   %\spacing
   % input, output document
  \subsubsection{ispid -- check PID}\linkedtwo{ring}{CommutativeRingProperties}{ispid}
   \func{ispid}{\param{self}}{\out{True/False/None}}\\
   \spacing
   % document of basic document
   \quad Return True/False according to the PID flag value being set,
        otherwise return None.\\
   \spacing
   % added document
   %\spacing
   % input, output document
  \subsubsection{setIspid -- set PID}\linkedtwo{ring}{CommutativeRingProperties}{setIspid}
   \func{ispid}{\param{self},\ \hiki{value}{True/False}}{\out{(None)}}\\
   \spacing
   % document of basic document
   \quad Set True/False value to the euclidean flag.\\
   Propagation:
   \begin{itemize}
    \item True $\to$ UFD,\ Noetherian\\
    \item False $\to$ euclidean\\
   \end{itemize}
   \quad\\
   \spacing
   % added document
   %\spacing
   % input, output document
 \subsubsection{isufd -- check UFD}\linkedtwo{ring}{CommutativeRingProperties}{isufd}
   \func{isufd}{\param{self}}{\out{True/False/None}}\\
   \spacing
   % document of basic document
   \quad Return True/False according to the UFD flag value being set,
        otherwise return None.\\
   \spacing
   % added document
   %\spacing
   % input, output document
  \subsubsection{setIsufd -- set UFD}\linkedtwo{ring}{CommutativeRingProperties}{setIsufd}
   \func{isufd}{\param{self},\ \hiki{value}{True/False}}{\out{(None)}}\\
   \spacing
   % document of basic document
   \quad Set True/False value to the UFD flag.\\
   Propagation:
   \begin{itemize}
    \item True $\to$ domain\\
    \item False $\to$ PID\\
   \end{itemize}
   \quad\\
   \spacing
   % added document
   %\spacing
   % input, output document
  \subsubsection{isnoetherian -- check Noetherian}\linkedtwo{ring}{CommutativeRingProperties}{isnoetherian}
   \func{isnoetherian}{\param{self}}{\out{True/False/None}}\\
   \spacing
   % document of basic document
   \quad Return True/False according to the Noetherian flag value being set,
        otherwise return None.\\
   \spacing
   % added document
   %\spacing
   % input, output document
  \subsubsection{setIsnoetherian -- set Noetherian}\linkedtwo{ring}{CommutativeRingProperties}{setIsnoetherian}
   \func{isnoetherian}{\param{self},\ \hiki{value}{True/False}}{\out{(None)}}\\
   \spacing
   % document of basic document
   \quad Set True/False value to the Noetherian flag.\\
   Propagation:
   \begin{itemize}
    \item True $\to$ domain\\
    \item False $\to$ PID\\
   \end{itemize}
   \quad\\
   \spacing
   % added document
   %\spacing
   % input, output document
  \subsubsection{isdomain -- check domain}\linkedtwo{ring}{CommutativeRingProperties}{isdomain}
   \func{isdomain}{\param{self}}{\out{True/False/None}}\\
   \spacing
   % document of basic document
   \quad Return True/False according to the domain flag value being set,
        otherwise return None.\\
   \spacing
   % added document
   %\spacing
   % input, output document
  \subsubsection{setIsdomain -- set domain}\linkedtwo{ring}{CommutativeRingProperties}{setIsdomain}
   \func{isdomain}{\param{self},\ \hiki{value}{True/False}}{\out{(None)}}\\
   \spacing
   % document of basic document
   \quad Set True/False value to the domain flag.\\
   Propagation:
   \begin{itemize}
    \item False $\to$ UFD,\ Noetherian\\
   \end{itemize}
   \quad\\
   \spacing
   % added document
   %\spacing
   % input, output document
\C
  \subsection{getRingInstance(function)}\linkedone{ring}{getRingInstance}
  \func{getRingInstance}{\hiki{obj}{RingElement}}{\out{RingElement}}\\
   \spacing
   % document of basic document
   \quad  Return a RingElement instance which equals \param{obj}.\\
   \spacing
   % added document
   \quad Mainly for python built-in objects such as int or float.\\
   \spacing
   % input, output document
  \subsection{getRing(function)}\linkedone{ring}{getRing}
  \func{getRing}{\hiki{obj}{RingElement}}{\out{Ring}}\\
   \spacing
   % document of basic document
   \quad  Return a ring to which \param{obj} belongs.\\
   \spacing
   % added document
   \quad Mainly for python built-in objects such as int or float.\\
   \spacing
   % input, output document
  \subsection{inverse(function)}\linkedone{ring}{inverse}
  \func{inverse}{\hiki{obj}{CommutativeRingElement}}{\out{QuotientFieldElement}}\\
   \spacing
   % document of basic document
   \quad  Return the inverse of \param{obj}.
   The inverse can be in the quotient field,
   if the \param{obj} is an element of non-field domain.\\
   \spacing
   % added document
   \quad Mainly for python built-in objects such as int or float.\\
   \spacing
   % input, output document
  \subsection{exact\_division(function)}\linkedone{ring}{exact\_division}
  \func{exact\_division}{\hiki{self}{RingElement},\ \hiki{other}{RingElement}}{\out{RingElement}}\\
   \spacing
   % document of basic document
   \quad  Return the division of \param{self} by \param{other} if the division is exact.
   \spacing
   % added document
   \quad Mainly for python built-in objects such as int or float.\\
   \spacing
   % input, output document
\begin{ex}
>>> print ring.getRing(3)
Z
>>> print ring.exact_division(6, 3)
2L
\end{ex}%Don't indent!
\C

%---------- end document ---------- %

\bibliographystyle{jplain}%use jbibtex
\bibliography{nzmath_references}

\end{document}

