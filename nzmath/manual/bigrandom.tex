\documentclass{report}

%%%%%%%%%%%%%%%%%%%%%%%%%%%%%%%%%%%%%%%%%%%%%%%%%%%%%%%%%%%%%
%
% macros for nzmath manual
%
%%%%%%%%%%%%%%%%%%%%%%%%%%%%%%%%%%%%%%%%%%%%%%%%%%%%%%%%%%%%%
\usepackage{amssymb,amsmath}
\usepackage{color}
\usepackage[dvipdfm,bookmarks=true,bookmarksnumbered=true,%
 pdftitle={NZMATH Users Manual},%
 pdfsubject={Manual for NZMATH Users},%
 pdfauthor={NZMATH Development Group},%
 pdfkeywords={TeX; dvipdfmx; hyperref; color;},%
 colorlinks=true]{hyperref}
\usepackage{fancybox}
\usepackage[T1]{fontenc}
%
\newcommand{\DS}{\displaystyle}
\newcommand{\C}{\clearpage}
\newcommand{\NO}{\noindent}
\newcommand{\negok}{$\dagger$}
\newcommand{\spacing}{\vspace{1pt}\\ }
% software macros
\newcommand{\nzmathzero}{{\footnotesize $\mathbb{N}\mathbb{Z}$}\texttt{MATH}}
\newcommand{\nzmath}{{\nzmathzero}\ }
\newcommand{\pythonzero}{$\mbox{\texttt{Python}}$}
\newcommand{\python}{{\pythonzero}\ }
% link macros
\newcommand{\linkingzero}[1]{{\bf \hyperlink{#1}{#1}}}%module
\newcommand{\linkingone}[2]{{\bf \hyperlink{#1.#2}{#2}}}%module,class/function etc.
\newcommand{\linkingtwo}[3]{{\bf \hyperlink{#1.#2.#3}{#3}}}%module,class,method
\newcommand{\linkedzero}[1]{\hypertarget{#1}{}}
\newcommand{\linkedone}[2]{\hypertarget{#1.#2}{}}
\newcommand{\linkedtwo}[3]{\hypertarget{#1.#2.#3}{}}
\newcommand{\linktutorial}[1]{\href{http://docs.python.org/tutorial/#1}{#1}}
\newcommand{\linktutorialone}[2]{\href{http://docs.python.org/tutorial/#1}{#2}}
\newcommand{\linklibrary}[1]{\href{http://docs.python.org/library/#1}{#1}}
\newcommand{\linklibraryone}[2]{\href{http://docs.python.org/library/#1}{#2}}
\newcommand{\pythonhp}{\href{http://www.python.org/}{\python website}}
\newcommand{\nzmathwiki}{\href{http://nzmath.sourceforge.net/wiki/}{{\nzmathzero}Wiki}}
\newcommand{\nzmathsf}{\href{http://sourceforge.net/projects/nzmath/}{\nzmath Project Page}}
\newcommand{\nzmathtnt}{\href{http://tnt.math.se.tmu.ac.jp/nzmath/}{\nzmath Project Official Page}}
% parameter name
\newcommand{\param}[1]{{\tt #1}}
% function macros
\newcommand{\hiki}[2]{{\tt #1}:\ {\em #2}}
\newcommand{\hikiopt}[3]{{\tt #1}:\ {\em #2}=#3}

\newdimen\hoge
\newdimen\truetextwidth
\newcommand{\func}[3]{%
\setbox0\hbox{#1(#2)}
\hoge=\wd0
\truetextwidth=\textwidth
\advance \truetextwidth by -2\oddsidemargin
\ifdim\hoge<\truetextwidth % short form
{\bf \colorbox{skyyellow}{#1(#2)\ $\to$ #3}}
%
\else % long form
\fcolorbox{skyyellow}{skyyellow}{%
   \begin{minipage}{\textwidth}%
   {\bf #1(#2)\\ %
    \qquad\quad   $\to$\ #3}%
   \end{minipage}%
   }%
\fi%
}

\newcommand{\out}[1]{{\em #1}}
\newcommand{\initialize}{%
  \paragraph{\large \colorbox{skyblue}{Initialize (Constructor)}}%
    \quad\\ %
    \vspace{3pt}\\
}
\newcommand{\method}{\C \paragraph{\large \colorbox{skyblue}{Methods}}}
% Attribute environment
\newenvironment{at}
{%begin
\paragraph{\large \colorbox{skyblue}{Attribute}}
\quad\\
\begin{description}
}%
{%end
\end{description}
}
% Operation environment
\newenvironment{op}
{%begin
\paragraph{\large \colorbox{skyblue}{Operations}}
\quad\\
\begin{table}[h]
\begin{center}
\begin{tabular}{|l|l|}
\hline
operator & explanation\\
\hline
}%
{%end
\hline
\end{tabular}
\end{center}
\end{table}
}
% Examples environment
\newenvironment{ex}%
{%begin
\paragraph{\large \colorbox{skyblue}{Examples}}
\VerbatimEnvironment
\renewcommand{\EveryVerbatim}{\fontencoding{OT1}\selectfont}
\begin{quote}
\begin{Verbatim}
}%
{%end
\end{Verbatim}
\end{quote}
}
%
\definecolor{skyblue}{cmyk}{0.2, 0, 0.1, 0}
\definecolor{skyyellow}{cmyk}{0.1, 0.1, 0.5, 0}
%
%\title{NZMATH User Manual\\ {\large{(for version 1.0)}}}
%\date{}
%\author{}
\begin{document}
%\maketitle
%
\setcounter{tocdepth}{3}
\setcounter{secnumdepth}{3}


\tableofcontents
\C

\chapter{Functions}

%---------- start document ---------- %
 \section{bigrandom -- random numbers}\linkedzero{bigrandom}
%
 \paragraph{Historical Note}\label{bigrandom_historical_note}

 The module was written for replacement of the \python standard module
 \linklibrary{random}, because in the era of \python 2.2 (prehistorical period of
 \nzmath) the random module raises {\tt OverflowError} for long integer
 arguments for the \linklibraryone{random\#random.randrange}{randrange} function, which is the only function
 having a use case in \nzmath.

 After the creation of \python 2.3, it was theoretically possible to
 use {\tt random.randrange}, since it started to accept long integer
 as its argument. Use of it was, however, not considered, since there
 had been the {\tt bigrandom} module. It was lucky for us. In fall of
 2006, we found a bug in {\tt random.randrange} and reported it (see
 issue tracker); the {\tt random.randrange} accepts long integers but
 returns unreliable result for truly big integers. The bug was fixed
 for \python 2.5.1. You can, therefore, use {\tt random.randrange}
 instead of {\tt bigrandom.randrange} for \python 2.5.1 or higher.

  \subsection{random -- random number generator}\linkedone{bigrandom}{random}
   \func{random}
   {}{{\em float}}\\
   \spacing
   % document of basic document
   \quad Return a random floating point number in the interval \([0, 1)\).\\
   \spacing
   % added document
   \quad This function is an alias to {\tt random.random} in the \python standard library.\\
   % input, output document
%
  \subsection{randrange -- random integer generator}\linkedone{bigrandom}{randrange}
   \func{randrange}
   {%
     \hiki{start}{integer},\ %
     \hikiopt{stop}{integer}{None},\ %
     \hikiopt{step}{integer}{1}
   }{%
     {\em integer}
   }\\
   \spacing
   % document of basic document
   \quad Return a random integer in the range.\\
   \spacing
   % input, output document
   \quad The argument names do not correspond to their roles, but
   users are familiar with the
   \linklibraryone{functions\#range}{range} built-in function of
   \python and understand the semantics.  Calling with one argument
   \(n\), then the result is an integer in the range \([0, n)\) chosen
   randomly.  With two arguments \(n\) and \(m\),
   in \([n, m)\), and with third \(l\), in \([n, m) \cap (n + l\mathbb{Z})\).\\
   \spacing \quad This function is almost the same as {\tt
     random.randrange} in the \python standard library.  See the
   historical note~\ref{bigrandom_historical_note}.
%
\begin{ex}
>>> randrange(4, 10000, 3)
9919L
>>> randrange(4 * 10**60)
31925916908162253969182327491823596145612834799876775114620151L
\end{ex}%Don't indent!(indent causes an error.)

   \subsection{map\_choice -- choice from image of mapping}\linkedone{bigrandom}{map\_choice}
   \func{map\_choice}
   {%
     \hiki{mapping}{function},\ %
     \hiki{upperbound}{integer}
   }{%
     {\em integer}
   }\\
   \spacing
   \quad Return a choice from a set given as the image of the mapping
   from natural numbers (more precisely {\tt range(upperbound)}).  In
   other words, it is equivalent to:
   {\tt random.\linklibraryone{random\#choice}{choice}([mapping(i) for i in range(upperbound)])},
   if \param{upperbound} is small enough for the list size limit.\\
   \spacing
   \quad The \param{mapping} can be a partial function, i.e. it may return
   {\tt None} for some input. However, if the resulting set is empty, it
   will end up with an infinite loop.\\
\C

%---------- end document ---------- %

\bibliographystyle{jplain}%use jbibtex
\bibliography{nzmath_references}

\end{document}