\documentclass{report}

%%%%%%%%%%%%%%%%%%%%%%%%%%%%%%%%%%%%%%%%%%%%%%%%%%%%%%%%%%%%%
%
% macros for nzmath manual
%
%%%%%%%%%%%%%%%%%%%%%%%%%%%%%%%%%%%%%%%%%%%%%%%%%%%%%%%%%%%%%
\usepackage{amssymb,amsmath}
\usepackage{color}
\usepackage[dvipdfm,bookmarks=true,bookmarksnumbered=true,%
 pdftitle={NZMATH Users Manual},%
 pdfsubject={Manual for NZMATH Users},%
 pdfauthor={NZMATH Development Group},%
 pdfkeywords={TeX; dvipdfmx; hyperref; color;},%
 colorlinks=true]{hyperref}
\usepackage{fancybox}
\usepackage[T1]{fontenc}
%
\newcommand{\DS}{\displaystyle}
\newcommand{\C}{\clearpage}
\newcommand{\NO}{\noindent}
\newcommand{\negok}{$\dagger$}
\newcommand{\spacing}{\vspace{1pt}\\ }
% software macros
\newcommand{\nzmathzero}{{\footnotesize $\mathbb{N}\mathbb{Z}$}\texttt{MATH}}
\newcommand{\nzmath}{{\nzmathzero}\ }
\newcommand{\pythonzero}{$\mbox{\texttt{Python}}$}
\newcommand{\python}{{\pythonzero}\ }
% link macros
\newcommand{\linkingzero}[1]{{\bf \hyperlink{#1}{#1}}}%module
\newcommand{\linkingone}[2]{{\bf \hyperlink{#1.#2}{#2}}}%module,class/function etc.
\newcommand{\linkingtwo}[3]{{\bf \hyperlink{#1.#2.#3}{#3}}}%module,class,method
\newcommand{\linkedzero}[1]{\hypertarget{#1}{}}
\newcommand{\linkedone}[2]{\hypertarget{#1.#2}{}}
\newcommand{\linkedtwo}[3]{\hypertarget{#1.#2.#3}{}}
\newcommand{\linktutorial}[1]{\href{http://docs.python.org/tutorial/#1}{#1}}
\newcommand{\linktutorialone}[2]{\href{http://docs.python.org/tutorial/#1}{#2}}
\newcommand{\linklibrary}[1]{\href{http://docs.python.org/library/#1}{#1}}
\newcommand{\linklibraryone}[2]{\href{http://docs.python.org/library/#1}{#2}}
\newcommand{\pythonhp}{\href{http://www.python.org/}{\python website}}
\newcommand{\nzmathwiki}{\href{http://nzmath.sourceforge.net/wiki/}{{\nzmathzero}Wiki}}
\newcommand{\nzmathsf}{\href{http://sourceforge.net/projects/nzmath/}{\nzmath Project Page}}
\newcommand{\nzmathtnt}{\href{http://tnt.math.se.tmu.ac.jp/nzmath/}{\nzmath Project Official Page}}
% parameter name
\newcommand{\param}[1]{{\tt #1}}
% function macros
\newcommand{\hiki}[2]{{\tt #1}:\ {\em #2}}
\newcommand{\hikiopt}[3]{{\tt #1}:\ {\em #2}=#3}

\newdimen\hoge
\newdimen\truetextwidth
\newcommand{\func}[3]{%
\setbox0\hbox{#1(#2)}
\hoge=\wd0
\truetextwidth=\textwidth
\advance \truetextwidth by -2\oddsidemargin
\ifdim\hoge<\truetextwidth % short form
{\bf \colorbox{skyyellow}{#1(#2)\ $\to$ #3}}
%
\else % long form
\fcolorbox{skyyellow}{skyyellow}{%
   \begin{minipage}{\textwidth}%
   {\bf #1(#2)\\ %
    \qquad\quad   $\to$\ #3}%
   \end{minipage}%
   }%
\fi%
}

\newcommand{\out}[1]{{\em #1}}
\newcommand{\initialize}{%
  \paragraph{\large \colorbox{skyblue}{Initialize (Constructor)}}%
    \quad\\ %
    \vspace{3pt}\\
}
\newcommand{\method}{\C \paragraph{\large \colorbox{skyblue}{Methods}}}
% Attribute environment
\newenvironment{at}
{%begin
\paragraph{\large \colorbox{skyblue}{Attribute}}
\quad\\
\begin{description}
}%
{%end
\end{description}
}
% Operation environment
\newenvironment{op}
{%begin
\paragraph{\large \colorbox{skyblue}{Operations}}
\quad\\
\begin{table}[h]
\begin{center}
\begin{tabular}{|l|l|}
\hline
operator & explanation\\
\hline
}%
{%end
\hline
\end{tabular}
\end{center}
\end{table}
}
% Examples environment
\newenvironment{ex}%
{%begin
\paragraph{\large \colorbox{skyblue}{Examples}}
\VerbatimEnvironment
\renewcommand{\EveryVerbatim}{\fontencoding{OT1}\selectfont}
\begin{quote}
\begin{Verbatim}
}%
{%end
\end{Verbatim}
\end{quote}
}
%
\definecolor{skyblue}{cmyk}{0.2, 0, 0.1, 0}
\definecolor{skyyellow}{cmyk}{0.1, 0.1, 0.5, 0}
%
%\title{NZMATH User Manual\\ {\large{(for version 1.0)}}}
%\date{}
%\author{}
\begin{document}
%\maketitle
%
\setcounter{tocdepth}{3}
\setcounter{secnumdepth}{3}


\tableofcontents
\C

\chapter{Classes}

%---------- start document ---------- %
 \section{real -- real numbers and its functions}\linkedzero{real}
The module {\tt real} provides arbitrary precision real numbers and
their utilities. The functions provided are corresponding to the
\linklibrary{math} standard module.



 \begin{itemize}
   \item {\bf Classes}
   \begin{itemize}
     \item \linkingone{real}{RealField}
     \item \linkingone{real}{Real}
     \item \negok \linkingone{real}{Constant}
     \item \negok \linkingone{real}{ExponentialPowerSeries}
     \item \negok \linkingone{real}{AbsoluteError}
     \item \negok \linkingone{real}{RelativeError}
   \end{itemize}
   \item {\bf Functions}
     \begin{itemize}
       \item \linkingone{real}{exp}
       \item \linkingone{real}{sqrt}
       \item \linkingone{real}{log}
       \item \linkingone{real}{log1piter}
       \item \linkingone{real}{piGaussLegendre}
       \item \linkingone{real}{eContinuedFraction}
       \item \linkingone{real}{floor}
       \item \linkingone{real}{ceil}
       \item \linkingone{real}{tranc}
       \item \linkingone{real}{sin}
       \item \linkingone{real}{cos}
       \item \linkingone{real}{tan}
       \item \linkingone{real}{sinh}
       \item \linkingone{real}{cosh}
       \item \linkingone{real}{tanh}
       \item \linkingone{real}{asin}
       \item \linkingone{real}{acos}
       \item \linkingone{real}{atan}
       \item \linkingone{real}{atan2}
       \item \linkingone{real}{hypot}
       \item \linkingone{real}{pow}
       \item \linkingone{real}{degrees}
       \item \linkingone{real}{radians}
       \item \linkingone{real}{fabs}
       \item \linkingone{real}{fmod}
       \item \linkingone{real}{frexp}
       \item \linkingone{real}{ldexp}
       \item \linkingone{real}{EulerTransform}

     \end{itemize}
 \end{itemize}

This module also provides following constants:
\begin{description}
   \item[e]\linkedone{real}{e}:\\
     \param{e} is the base of the natural logarithm function, also called Napier's constant.
   \item[pi]\linkedone{real}{pi}:\\
     \param{pi} is the circular constant, also denoted by $\pi$ .
   \item[Log2]\linkedone{real}{Log2}:\\
     \param{Log2} is the natural logarithm of 2.
   \item[\negok defaultError]\linkedone{real}{defaultError}:\\
     \param{defaultError} is the instance of \linkingone{real}{RelativeError}.
   \item[theRealField]\linkedone{real}{theRealField}:\\
     \param{theRealField} is the instance of \linkingone{real}{RealField}.
 \end{description}

\C
 \subsection{RealField -- field of real numbers}\linkedone{real}{RealField}
 The class is for the field of real numbers. The class has the single instance \linkingone{real}{theRealField}.

 This class is a subclass of \linkingone{ring}{Field}.


  \initialize
  \func{RealField}{}{\out{RealField}}\\
  \spacing
  % document of basic document
  \quad Create an instance of RealField. 
  % added document
  You may not want to create an instance, since there is already \linkingone{real}{theRealField}.
  % \spacing
  % input, output document
  %See \linkingone{module}{point} for \param{point}.
  \begin{at}
    \item[zero]\linkedtwo{real}{RealField}{zero}:\\ It expresses the additive unit 0. (read only)
    \item[one]\linkedtwo{real}{RealField}{one}:\\ It expresses the multiplicative unit 1. (read only)
  \end{at}
  \begin{op}
    \verb|x in R| & membership test; return whether an element is in or not.\\
    \verb|repr(R)| & return representation string.\\
    \verb|str(R)| & return string.\\
  \end{op} 
  \method
%
  \subsubsection{getCharacteristic -- get characteristic}\linkedtwo{real}{RealField}{getCharacteristic}
   \func{getCharacteristic}{\param{self}}{\out{integer}}\\
   \spacing
   % document of basic document
   \quad Return the characteristic, zero.
%
  \subsubsection{issubring -- subring test}\linkedtwo{real}{RealField}{issubring}
   \func{issubring}{\param{self},\ \hiki{aRing}{\linkingone{ring}{Ring}}}{\out{bool}}\\
   \spacing
   % document of basic document
   \quad Report whether another ring contains the real field as subring.
   \spacing
%
  \subsubsection{issuperring -- superring test}\linkedtwo{real}{RealField}{issuperring}
   \func{issuperring}{\param{self},\ \hiki{aRing}{\linkingone{ring}{Ring}}}{\out{bool}}\\
   \spacing
   % document of basic document
   \quad Report whether the real field contains another ring as subring.
   \spacing

\C
 \subsection{Real -- a Real number}\linkedone{real}{Real}
 Real is a class of real number.  This class is only for consistency for other \linkingone{ring}{Ring} object.

 This class is a subclass of \linkingone{ring}{CommutativeRingElement}.

 All implemented operators in this class are delegated to Float type. 
  \initialize
  \func{Real}{\hiki{value}{number}}{\out{Real}}\\
  \spacing
  % document of basic document
  \quad Construct a Real object.
  % added document
  \spacing
  % input, output document
  \param{value} must be int, long, Float or \linkingone{rational}{Rational}.
  \method
  \subsubsection{getRing -- get ring object}\linkedtwo{real}{Real}{getRing}
   \func{getRing}{\param{self}}{\out{RealField}}\\
   \spacing
   % document of basic document
   \quad Return the real field instance.
%
\C
 \subsection{Constant -- real number with error correction}\linkedone{real}{Constant}
 Constant provides constant-like behavior for Float calculation context. It caches the constant value and re-computes for more precision by request.

  Almost every operators are delegated to the cached value.
  \initialize
  \func{Constant}
       {\hiki{getValue}{numbers},\ 
         \hikiopt{err}{Error}{\linkingone{real}{defaultError}}}
       {\out{Constant}}\\
  \spacing
  % document of basic document
  \quad Construct Constant from Float and error.
  % input, output document
  \param{getValue} must be Float, and \param{err} must be \linkingone{real}{AbsoluteError} or \linkingone{real}{RelativeError}.
  \begin{op}
    \verb|S(err)| & Return the value at least as accurate as the given error \param{err}. 

.\\
  \end{op}
\begin{ex}
>>> pi = Constant(piGaussLegendre)
>>> print pi
3.14159265358979
>>> pi + 1
4.14159265358979
>>> pi(RelativeError(0,1,2**100)) # for 100 bit precision
3.1415926535897932384626433832795
\end{ex}%Don't indent!
  \method
  \subsubsection{inverse -- inverse value}\linkedtwo{real}{Constant}{inverse}
   \func{inverse}{\param{self}}{\out{Constant}}\\
   \spacing
   % document of basic document
   \quad Return the inverse of the number.
%
  \subsubsection{toRational -- convert to Rational}\linkedtwo{real}{Constant}{toRational}
   \func{toRational}{\param{self}}{\out{Rational}}\\
   \spacing
   % document of basic document
   \quad Return a rational number approximating the number.
   \spacing
%
\C
 \subsection{ExponentialPowerSeries -- exponential power series}\linkedone{real}{ExponentialPowerSeries}
ExponentialPowerSeries is a class for exponential power series, whose n-th term has form $\frac{a_n{x^n}}{n!}$.
  \initialize
  \func{ExponentialPowerSeries}
       {\hiki{iterator}{iterator}}
       {\out{ExponentialPowerSeries}}\\
  \spacing
  % document of basic document
  \quad Construct an exponential power series with coefficient generated by the given \param{iterator}, which can be an infinite iterator.\\
  \begin{op}
    \verb|S(x,maxerror)| & Return the value of the series with \param{x} assigned.The maximum error \param{maxerror}\\
    & must be given as a \linkingone{real}{RelativeError} or \linkingone{real}{AbsoluteError} instance.\\
  \end{op} 
\begin{ex}
>>> expo = ExponentialPowerSeries(itertools.cycle([1]))
>>> expo(.5, defaultError)
Rational(5434422938503507, 3296144130048000)
\end{ex}%Don't indent!
  \method
  \subsubsection{terms -- generator of terms of series}\linkedtwo{real}{ExponentialPowerSeries}{terms}
   \func{terms}
        {\param{self},\ 
          \hiki{x}{numbers}
        }{\out{ExponentialPowerSeries}}\\
   \spacing
   % document of basic document
   \quad Generator of terms of series with assigned x value.
   \spacing
   % input, output document
   \quad \param{x} must be int, long or Float.\\
%
\C
 \subsection{AbsoluteError -- absolute error}\linkedone{real}{AbsoluteError}
 AbsoluteError is the class of absolute error of real numbers.

 this class is deprecated.

\C
 \subsection{RelativeError -- relative error}\linkedone{real}{RelativeError}
 AbsoluteError is the class of relative error of real numbers.

 this class is deprecated.
\C
  \subsection{exp(function) -- exponential value}\linkedone{real}{exp}
  \func{exp}{\hiki{x}{number}, \ \hikiopt{err}{Error}{\linkingone{real}{defaultError}}}{\out{number}}\\
    \spacing
    % document of basic document
    \quad Return exponential of \param{x}.
    \spacing
    % input, output document
    \quad \param{err} must be \linkingone{real}{AbsoluteError} or \linkingone{real}{RelativeError}.
%
  \subsection{sqrt(function) -- square root}\linkedone{real}{sqrt}
  \func{sqrt}{\hiki{x}{number}, \ \hikiopt{err}{Error}{\linkingone{real}{defaultError}}}{\out{number}}\\
    \spacing
    % document of basic document
    \quad Return square root of \param{x}.
    \spacing
    % input, output document
    \quad \param{err} must be \linkingone{real}{AbsoluteError} or \linkingone{real}{RelativeError}.
%
  \subsection{log(function) -- logarithm}\linkedone{real}{log}
  \func{log}
       {\hiki{x}{number},\ 
         \hikiopt{base}{number}{None},\ 
         \hikiopt{err}{Error}{\linkingone{real}{defaultError}}}
       {\out{number}}\\
    \spacing
    % document of basic document
    \quad Return logarithm of a positive number \param{x}.

    If an additional argument \param{base} is given, it returns logarithm of \param{x} to the \param{base}.
    \spacing
    % input, output document
    \quad \param{err} must be \linkingone{real}{AbsoluteError} or \linkingone{real}{RelativeError}.
%
  \subsection{log1piter(function) -- iterator of log(1+x)}\linkedone{real}{log1piter}
  \func{log1piter}{\hiki{xx}{number}}{\out{iterator}}\\
    \spacing
    % document of basic document
    \quad Return iterator for $\log(1+x)$.
    \spacing
%
  \subsection{piGaussLegendre(function) -- pi by Gauss-Legendre}\linkedone{real}{piGaussLegendre}
  \func{piGaussLegendre}{\hikiopt{err}{Error}{\linkingone{real}{defaultError}}}{\out{number}}\\
    \spacing
    % document of basic document
    \quad Return pi by Gauss-Legendre algorithm.
    \spacing
    % input, output document
    \quad \param{err} must be \linkingone{real}{AbsoluteError} or \linkingone{real}{RelativeError}.
%
  \subsection{eContinuedFraction(function) -- Napier's Constant by continued fraction expansion}\linkedone{real}{eContinuedFraction}
  \func{eContinuedFraction}{\hikiopt{err}{Error}{\linkingone{real}{defaultError}}}{\out{number}}\\
    \spacing
    % document of basic document
    \quad Return the base of natural logarithm \linkingone{real}{e} by continued fraction expansion.
    \spacing
    % input, output document
    \quad \param{err} must be \linkingone{real}{AbsoluteError} or \linkingone{real}{RelativeError}.
%
  \subsection{floor(function) -- floor the number}\linkedone{real}{floor}
  \func{floor}{\hiki{x}{number}}{\out{integer}}\\
    \spacing
    % document of basic document
    \quad Return the biggest integer not more than \param{x}.
    \spacing
%
  \subsection{ceil(function) -- ceil the number}\linkedone{real}{ceil}
  \func{ceil}{\hiki{x}{number}}{\out{integer}}\\
    \spacing
    % document of basic document
    \quad Return the smallest integer not less than \param{x}.
    \spacing
%
  \subsection{tranc(function) -- round-off the number}\linkedone{real}{tranc}
  \func{tranc}{\hiki{x}{number}}{\out{integer}}\\
    \spacing
    % document of basic document
    \quad Return the number of rounded off \param{x}.
    \spacing
%
  \subsection{sin(function) -- sine function}\linkedone{real}{sin}
  \func{sin}{\hiki{x}{number}, \ \hikiopt{err}{Error}{\linkingone{real}{defaultError}}}{\out{number}}\\
    \spacing
    % document of basic document
    \quad Return the sine of \param{x}.
    \spacing
    % input, output document
    \quad \param{err} must be \linkingone{real}{AbsoluteError} or \linkingone{real}{RelativeError}.
%
  \subsection{cos(function) -- cosine function}\linkedone{real}{cos}
  \func{cos}{\hiki{x}{number}, \ \hikiopt{err}{Error}{\linkingone{real}{defaultError}}}{\out{number}}\\
    \spacing
    % document of basic document
    \quad Return the cosine of \param{x}.
    \spacing
    % input, output document
    \quad \param{err} must be \linkingone{real}{AbsoluteError} or \linkingone{real}{RelativeError}.
%
  \subsection{tan(function) -- tangent function}\linkedone{real}{tan}
  \func{tan}{\hiki{x}{number}, \ \hikiopt{err}{Error}{\linkingone{real}{defaultError}}}{\out{number}}\\
    \spacing
    % document of basic document
    \quad Return the tangent of \param{x}.
    \spacing
    % input, output document
    \quad \param{err} must be \linkingone{real}{AbsoluteError} or \linkingone{real}{RelativeError}.
%
  \subsection{sinh(function) -- hyperbolic sine function}\linkedone{real}{sinh}
  \func{sinh}{\hiki{x}{number}, \ \hikiopt{err}{Error}{\linkingone{real}{defaultError}}}{\out{number}}\\
    \spacing
    % document of basic document
    \quad Return the hyperbolic sine of \param{x}.
    \spacing
    % input, output document
    \quad \param{err} must be \linkingone{real}{AbsoluteError} or \linkingone{real}{RelativeError}.
%
  \subsection{cosh(function) -- hyperbolic cosine function}\linkedone{real}{cosh}
  \func{cosh}{\hiki{x}{number}, \ \hikiopt{err}{Error}{\linkingone{real}{defaultError}}}{\out{number}}\\
    \spacing
    % document of basic document
    \quad Return the hyperbolic cosine of \param{x}.
    \spacing
    % input, output document
    \quad \param{err} must be \linkingone{real}{AbsoluteError} or \linkingone{real}{RelativeError}.
%
  \subsection{tanh(function) -- hyperbolic tangent function}\linkedone{real}{tanh}
  \func{tanh}{\hiki{x}{number}, \ \hikiopt{err}{Error}{\linkingone{real}{defaultError}}}{\out{number}}\\
    \spacing
    % document of basic document
    \quad Return the hyperbolic tangent of \param{x}.
    \spacing
    % input, output document
    \quad \param{err} must be \linkingone{real}{AbsoluteError} or \linkingone{real}{RelativeError}.
%
  \subsection{asin(function) -- arc sine function}\linkedone{real}{asin}
  \func{asin}{\hiki{x}{number}, \ \hikiopt{err}{Error}{\linkingone{real}{defaultError}}}{\out{number}}\\
    \spacing
    % document of basic document
    \quad Return the arc sine of \param{x}.
    \spacing
    % input, output document
    \quad \param{err} must be \linkingone{real}{AbsoluteError} or \linkingone{real}{RelativeError}.
%
  \subsection{acos(function) -- arc cosine function}\linkedone{real}{acos}
  \func{acos}{\hiki{x}{number}, \ \hikiopt{err}{Error}{\linkingone{real}{defaultError}}}{\out{number}}\\
    \spacing
    % document of basic document
    \quad Return the arc cosine of \param{x}.
    \spacing
    % input, output document
    \quad \param{err} must be \linkingone{real}{AbsoluteError} or \linkingone{real}{RelativeError}.
%
  \subsection{atan(function) -- arc tangent function}\linkedone{real}{atan}
  \func{atan}{\hiki{x}{number}, \ \hikiopt{err}{Error}{\linkingone{real}{defaultError}}}{\out{number}}\\
    \spacing
    % document of basic document
    \quad Return the arc tangent of \param{x}.
    \spacing
    % input, output document
    \quad \param{err} must be \linkingone{real}{AbsoluteError} or \linkingone{real}{RelativeError}.
%
  \subsection{atan2(function) -- arc tangent function}\linkedone{real}{atan2}
  \func{atan2}{\hiki{y}{number}, \ \hiki{x}{number}, \ \hikiopt{err}{Error}{\linkingone{real}{defaultError}}}{\out{number}}\\
    \spacing
    % document of basic document
    \quad Return the arc tangent of \param{y/x}.

    Unlike \linkingone{real}{atan}\param{(y/x)}, the signs of both \param{x} and \param{y} are considered.
    \spacing
    % added document
    \quad \negok It is unrecommended to obtain the value of \linkingone{real}{pi} with atan2(0,1).
    \spacing
    % input, output document
    \quad \param{err} must be \linkingone{real}{AbsoluteError} or \linkingone{real}{RelativeError}.
%
  \subsection{hypot(function) -- Euclidean distance function}\linkedone{real}{hypot}
  \func{hypot}{\hiki{x}{number}, \ \hiki{y}{number}, \ \hikiopt{err}{Error}{\linkingone{real}{defaultError}}}{\out{number}}\\
    \spacing
    % document of basic document
    \quad Return $\sqrt{x^2+y^2}$.
    \spacing
    % input, output document
    \quad \param{err} must be \linkingone{real}{AbsoluteError} or \linkingone{real}{RelativeError}.
%
  \subsection{pow(function) -- power function}\linkedone{real}{pow}
  \func{pow}{\hiki{x}{number}, \ \hiki{y}{number}, \ \hikiopt{err}{Error}{\linkingone{real}{defaultError}}}{\out{number}}\\
    \spacing
    % document of basic document
    \quad Return \param{y}th power of \param{x}.
    \spacing
    % input, output document
    \quad \param{err} must be \linkingone{real}{AbsoluteError} or \linkingone{real}{RelativeError}.
%
  \subsection{degrees(function) -- convert angle to degree}\linkedone{real}{degrees}
  \func{degrees}{\hiki{rad}{number},\ \hikiopt{err}{Error}{\linkingone{real}{defaultError}}}{\out{number}}\\
    \spacing
    % document of basic document
    \quad Converts angle \param{rad} from radians to degrees.
    \spacing
    % input, output document
    \quad \param{err} must be \linkingone{real}{AbsoluteError} or \linkingone{real}{RelativeError}.
%
  \subsection{radians(function) -- convert angle to radian}\linkedone{real}{radians}
  \func{radians}{\hiki{deg}{number},\ \hikiopt{err}{Error}{\linkingone{real}{defaultError}}}{\out{number}}\\
    \spacing
    % document of basic document
    \quad Converts angle \param{deg} from degrees to radians.
    \spacing
    % input, output document
    \quad \param{err} must be \linkingone{real}{AbsoluteError} or \linkingone{real}{RelativeError}.
%
  \subsection{fabs(function) -- absolute value}\linkedone{real}{fabs}
  \func{fabs}{\hiki{x}{number}}{\out{number}}\\
    \spacing
    % document of basic document
    \quad Return absolute value of \param{x}
    \spacing
%
  \subsection{fmod(function) -- modulo function over real}\linkedone{real}{fmod}
  \func{fmod}{\hiki{x}{number},\ \hiki{y}{number}}{\out{number}}\\
    \spacing
    % document of basic document
    \quad Return $x - ny$, where \param{n} is the quotient of \param{x / y}, rounded towards zero to an integer.
    \spacing
%
  \subsection{frexp(function) -- expression with base and binary exponent}\linkedone{real}{frexp}
  \func{frexp}{\hiki{x}{number}}{(\out{m},\out{e})}\\
    \spacing
    % document of basic document
    \quad Return a tuple \param{(m,e)}, where $x = m \times 2^e$, $1/2 \leq \mathtt{abs(m)} < 1$ and \param{e} is an integer.
    \spacing
    % added document
    \quad \negok This function is provided as the counter-part of math.frexp, but it might not be useful.
%
  \subsection{ldexp(function) -- construct number from base and binary exponent}\linkedone{real}{ldexp}
  \func{ldexp}{\hiki{x}{number},\ \hiki{i}{number}}{\out{number}}\\
    \spacing
    % document of basic document
    \quad Return $x \times 2^i$.
    \spacing
%
  \subsection{EulerTransform(function) -- iterator yields terms of Euler transform}\linkedone{real}{EulerTransform}
  \func{EulerTransform}{\hiki{iterator}{iterator}}{\out{iterator}}\\
    \spacing
    % document of basic document
    \quad Return an iterator which yields terms of Euler transform of the given \param{iterator}.
    \spacing
\C

%---------- end document ---------- %

\bibliographystyle{jplain}
\bibliography{nzmath_references}

\end{document}
