\documentclass{report}

%%%%%%%%%%%%%%%%%%%%%%%%%%%%%%%%%%%%%%%%%%%%%%%%%%%%%%%%%%%%%
%
% macros for nzmath manual
%
%%%%%%%%%%%%%%%%%%%%%%%%%%%%%%%%%%%%%%%%%%%%%%%%%%%%%%%%%%%%%
\usepackage{amssymb,amsmath}
\usepackage{color}
\usepackage[dvipdfm,bookmarks=true,bookmarksnumbered=true,%
 pdftitle={NZMATH Users Manual},%
 pdfsubject={Manual for NZMATH Users},%
 pdfauthor={NZMATH Development Group},%
 pdfkeywords={TeX; dvipdfmx; hyperref; color;},%
 colorlinks=true]{hyperref}
\usepackage{fancybox}
\usepackage[T1]{fontenc}
%
\newcommand{\DS}{\displaystyle}
\newcommand{\C}{\clearpage}
\newcommand{\NO}{\noindent}
\newcommand{\negok}{$\dagger$}
\newcommand{\spacing}{\vspace{1pt}\\ }
% software macros
\newcommand{\nzmathzero}{{\footnotesize $\mathbb{N}\mathbb{Z}$}\texttt{MATH}}
\newcommand{\nzmath}{{\nzmathzero}\ }
\newcommand{\pythonzero}{$\mbox{\texttt{Python}}$}
\newcommand{\python}{{\pythonzero}\ }
% link macros
\newcommand{\linkingzero}[1]{{\bf \hyperlink{#1}{#1}}}%module
\newcommand{\linkingone}[2]{{\bf \hyperlink{#1.#2}{#2}}}%module,class/function etc.
\newcommand{\linkingtwo}[3]{{\bf \hyperlink{#1.#2.#3}{#3}}}%module,class,method
\newcommand{\linkedzero}[1]{\hypertarget{#1}{}}
\newcommand{\linkedone}[2]{\hypertarget{#1.#2}{}}
\newcommand{\linkedtwo}[3]{\hypertarget{#1.#2.#3}{}}
\newcommand{\linktutorial}[1]{\href{http://docs.python.org/tutorial/#1}{#1}}
\newcommand{\linktutorialone}[2]{\href{http://docs.python.org/tutorial/#1}{#2}}
\newcommand{\linklibrary}[1]{\href{http://docs.python.org/library/#1}{#1}}
\newcommand{\linklibraryone}[2]{\href{http://docs.python.org/library/#1}{#2}}
\newcommand{\pythonhp}{\href{http://www.python.org/}{\python website}}
\newcommand{\nzmathwiki}{\href{http://nzmath.sourceforge.net/wiki/}{{\nzmathzero}Wiki}}
\newcommand{\nzmathsf}{\href{http://sourceforge.net/projects/nzmath/}{\nzmath Project Page}}
\newcommand{\nzmathtnt}{\href{http://tnt.math.se.tmu.ac.jp/nzmath/}{\nzmath Project Official Page}}
% parameter name
\newcommand{\param}[1]{{\tt #1}}
% function macros
\newcommand{\hiki}[2]{{\tt #1}:\ {\em #2}}
\newcommand{\hikiopt}[3]{{\tt #1}:\ {\em #2}=#3}

\newdimen\hoge
\newdimen\truetextwidth
\newcommand{\func}[3]{%
\setbox0\hbox{#1(#2)}
\hoge=\wd0
\truetextwidth=\textwidth
\advance \truetextwidth by -2\oddsidemargin
\ifdim\hoge<\truetextwidth % short form
{\bf \colorbox{skyyellow}{#1(#2)\ $\to$ #3}}
%
\else % long form
\fcolorbox{skyyellow}{skyyellow}{%
   \begin{minipage}{\textwidth}%
   {\bf #1(#2)\\ %
    \qquad\quad   $\to$\ #3}%
   \end{minipage}%
   }%
\fi%
}

\newcommand{\out}[1]{{\em #1}}
\newcommand{\initialize}{%
  \paragraph{\large \colorbox{skyblue}{Initialize (Constructor)}}%
    \quad\\ %
    \vspace{3pt}\\
}
\newcommand{\method}{\C \paragraph{\large \colorbox{skyblue}{Methods}}}
% Attribute environment
\newenvironment{at}
{%begin
\paragraph{\large \colorbox{skyblue}{Attribute}}
\quad\\
\begin{description}
}%
{%end
\end{description}
}
% Operation environment
\newenvironment{op}
{%begin
\paragraph{\large \colorbox{skyblue}{Operations}}
\quad\\
\begin{table}[h]
\begin{center}
\begin{tabular}{|l|l|}
\hline
operator & explanation\\
\hline
}%
{%end
\hline
\end{tabular}
\end{center}
\end{table}
}
% Examples environment
\newenvironment{ex}%
{%begin
\paragraph{\large \colorbox{skyblue}{Examples}}
\VerbatimEnvironment
\renewcommand{\EveryVerbatim}{\fontencoding{OT1}\selectfont}
\begin{quote}
\begin{Verbatim}
}%
{%end
\end{Verbatim}
\end{quote}
}
%
\definecolor{skyblue}{cmyk}{0.2, 0, 0.1, 0}
\definecolor{skyyellow}{cmyk}{0.1, 0.1, 0.5, 0}
%
%\title{NZMATH User Manual\\ {\large{(for version 1.0)}}}
%\date{}
%\author{}
\begin{document}
%\maketitle
%
\setcounter{tocdepth}{3}
\setcounter{secnumdepth}{3}

%
%\title{\nzmath User Manual\\ {\large{(for version 1.0)}}}
%\date{}
%\author{}
%\begin{document}
%\maketitle

\tableofcontents
\C

\chapter{Overview}
 \section{Introduction}
 NZMATH\cite{NZMATH} is a number theory oriented calculation system 
mainly developed by the Nakamula laboratory at Tokyo Metropolitan University.
 NZMATH system provides you mathematical, especially number-theoretic computational power.
 It is freely available and distributed under the BSD license.
The most distinctive feature of NZMATH is that it is written entirely 
using a scripting language called Python.

If you want to learn how to start using NZMATH, 
see Installation (section \ref{installation}) and 
Tutorial (section \ref{tutorial}).
%
 \subsection{Philosophy -- Advantages over Other Systems}
 In this section, we discuss philosophy of NZMATH, that is, the advantages of 
NZMATH compared to other similar systems.
%
\subsubsection{Open Source Software}
%
Many computational algebra systems, such as Maple\cite{Maple}, 
Mathematica\cite{Mathematica}, and Magma\cite{Magma} are fare-paying systems.
These non-free systems are not distributed with source codes.
Then, users cannot modify such systems easily.
It narrows these system's potentials 
for users not to take part in developing them.
NZMATH, on the other hand, is an open-source software 
and the source codes are openly available.
Furthermore, NZMATH is distributed under the BSD license.
BSD license claims as-is and redistribution or commercial use are permitted 
provided that these packages retain the copyright notice.
NZMATH users can develop it just as they like.
%
\subsubsection{Speed of Development}
%
We took over developing of SIMATH\cite{SIMATH}, 
which was developed under the leadership of Prof.Zimmer 
at Saarlandes University in Germany.
However, it costs a lot of time and efforts to develop these system.
Almost all systems including SIMATH are 
implemented in C or C++ for execution speed,
but we have to take the time to work memory management, 
construction of an interactive interpreter, 
preparation for multiple precision package and so on.
In this regard, we chose Python which is a modern programming language.
Python provides automatic memory management, a sophisticated interpreter 
and many useful packages.
We can concentrate on development of mathematical matters by using Python.
%
\subsubsection{Bridging the Gap between Users And Developers}
%
KANT/KASH\cite{KANT} and PARI/GP\cite{PARI} are similar systems to NZMATH.
But programming languages for modifying these systems are 
different between users and developers.
We think the gap makes evolution speed of these systems slow.
On the other hand, NZMATH has been developed with Python for bridging this gap. 
Python grammar is easy to understand 
and users can read easily codes written by Python.
And NZMATH, which is one of Python libraries, works on very wide platform 
including UNIX/Linux, Macintosh, Windows, and so forth.
Users can modify the programs and feedback to developers with a light heart.
So developers can absorb their thinking.
Then NZMATH will progress to more flexible user-friendly system.
%
\subsubsection{Link with Other Softwares}
%
NZMATH distributed as a Python library 
enables us to link other Python packages with it.
For example, NZMATH can be used with IPython\cite{IPython}, 
which is a comfortable interactive interpreter.
And it can be linked with matplotlib\cite{matplotlib},
which is a powerful graphic software.
Also mpmath\cite{mpmath}, which is a module for floating-point operation, 
can improve efficiency of NZMATH.
In fact, the module ecpp\linkingzero{ecpp} improves performance with mpmath.
There are many softwares implemented in Python.
Many of these packages are freely available.
Users can use NZMATH with these packages 
and create an unthinkable powerful system.
%
\subsection{Information}
  NZMATH has more than 25 modules.
These modules cover a lot of territory including
elementary number theoretic methods, combinatorial theoretic methods, 
solving equations, primality, factorization, 
multiplicative number theoretic functions, 
matrix, vector, polynomial, rational field, finite field, 
elliptic curve, and so on.
NZMATH manual for users is at:
\begin{verbatim}
http://tnt.math.metro-u.ac.jp/nzmath/manual/
\end{verbatim}
If you are interested in NZMATH, please visit the official website 
below to obtain more information about it.
\begin{verbatim}
http://tnt.math.metro-u.ac.jp/nzmath/
\end{verbatim}
Note that NZMATH can be used even if users do not have any experience of
writing programs in Python.
%
\subsection{Installation}\label{installation}
In this section, we explain how to install NZMATH.
If you use Windows (Windows XP, Windows Vista, Windows 7 etc.) 
as an operating system (OS), 
then see \ref{windows install} ``Install for Windows Users''.
%
\subsubsection{Basic Installation}
There are three steps for installation of NZMATH.

%First, make preparation of computer with the Python installed.
First, check whether Python is installed in the computer.
Python 2.5 or a higher version is needed for NZMATH.
If you do not have a copy of Python, please install it first.
Python is available from \verb+http://www.python.org/+.

Second, download a NZMATH package and expand it.
It is distributed at official web site:
\begin{verbatim}
http://tnt.math.metro-u.ac.jp/nzmath/download
\end{verbatim}
or at sourceforge.net:
\begin{verbatim}
http://sourceforge.net/project/showfiles.php?group_id=171032
\end{verbatim}
The package can be easily extracted, depending on the operating system.
For systems with recent GNU tar, type a single command below:
\begin{verbatim}
    % tar xf NZMATH-*.*.*.tar.gz
\end{verbatim}
where, \verb|%| is a command line prompt. 
With standard tar, type 
\begin{verbatim}
    % gzip -cd NZMATH-*.*.*.tar.gz | tar xf -
\end{verbatim}.
Please read *.*.* as the version number of which you downloaded the package.
For example, if the latest version is 1.0.0, then type the following command.
\begin{verbatim}
    % tar xf NZMATH-1.0.0.tar.gz
\end{verbatim}
Then, a subdirectory named NZMATH-*.*.* is created.

Finally, install NZMATH to the standard python path.
Usually, this can be translated into writing files somewhere 
under \verb+/usr/lib+ or \verb+/usr/local/lib+. 
So the appropriate write permission may be required at this step.
Typically, type commands below:
\begin{verbatim}
    % cd NZMATH-*.*.*
    % su
    # python setup.py install
\end{verbatim}

\subsubsection{Installation for Windows Users}\label{windows install}
We also distribute installation packages for specific platforms.
Especially, we started distributing the installer for Windows in 2007.

Please download the installer (NZMATH-*.*.*.win32Install.exe) from
\begin{verbatim}
http://tnt.math.metro-u.ac.jp/nzmath/download
\end{verbatim}
or at sourceforge.net:
\begin{verbatim}
http://sourceforge.net/project/showfiles.php?group_id=171032
\end{verbatim}

Here, we explain a way of installing NZMATH with the installer.
First please open the installer.
If you use Windows Vista or higher version, 
UAC (User Account Control) may ask if you run the program. click "Allow".
Then the setup window will open.
Following the steps in the setup wizard, 
you can install NZMATH with only three clicks.
%
\subsection{Tutorial}\label{tutorial}
  In this section, we describe how to use NZMATH.
\subsubsection{Sample Session}
  Start your Python interpreter.
  That is, open your command interpreter such as Terminal for MacOS or bash/csh for linux, type the strings ``python'' and press the key Enter.
\begin{ex}
% python
Python 2.6.1 (r261:67515, Jan 14 2009, 10:59:13)
[GCC 4.1.2 20071124 (Red Hat 4.1.2-42)] on linux2
Type "help", "copyright", "credits" or "license" for more information.
>>>
\end{ex}
  For windows users, it normally means opening IDLE (Python GUI), which is a Python software. 
\begin{ex}
Python 2.6.1 (r261:67517, Dec  4 2008, 16:51:00) [MSC v.1500 32 bit (Intel)] on win32
Type "copyright", "credits" or "license()" for more information.

    ****************************************************************
    Personal firewall software may warn about the connection IDLE
    makes to its subprocess using this computer's internal loopback
    interface.  This connection is not visible on any external
    interface and no data is sent to or received from the Internet.
    ****************************************************************
    
IDLE 2.6.1      
>>> 
\end{ex}
Here, '\verb+>>>+' is a Python prompt, which means that the system waits you to input commands.

Then, type:
\begin{ex}
>>> from nzmath import *
>>>
\end{ex}
This command enables you to use all NZMATH features.
If you use only a specific module (the term ``module'' is explained later), for example, prime, type as the following:
\begin{ex}
>>> from nzmath import prime
>>>
\end{ex}
You are ready to use NZMATH.
For example, type the string ``prime.nextPrime(1000)'', 
then you obtain `1009'' 
as the smallest prime among numbers greater than $1000$.
\begin{ex}
>>> prime.nextPrime(1000)
1009
>>>
\end{ex}
``prime'' is a name of a module, which is a NZMATH file including Python codes.
``nextPrime'' is a name of a function, which outputs values after the system executes some processes for inputs.
NZMATH has various functions for mathematical or algorithmic computations.
See \ref{function} Functions.

Also, we can create some mathematical objects.
For example, you may use the module ``matrix''.
If you want to define the matrix
\begin{equation*}
\left(
\begin{array}{rl}
1 & 2\\
5 & 6\\
\end{array}
\right)
\end{equation*}
and compute the square, then type as the following:
\begin{ex}
>>> A = matrix.Matrix(2, 2, [1, 2]+[5, 6])
>>> print A
1 2
5 6
>>> print A ** 2
11 14
35 46
>>>
\end{ex}
``Matrix'' is a name of a class, which is a template of mathematical objects.
See \ref{class} Classes for using NZMATH classes.

The command ``print'' enables us to represent outputs with good-looking forms.
The data structure such as ``[a, b, c, $\cdots$]'' is called list.
Also, we use various Python data structures like 
tuple ``(a, b, c, $\cdots$)'', 
dictionary ``$\{x_1:y_1, x_2:y_2, x_3:y_3, \cdots\}$'' etc.
%These follow syntax of Python programming language.
Note that we do not explain Python's syntax in detail 
because it is not absolutely necessary to use NZMATH.
However, we recommend that you learn Python for developing your potential.
Python grammar are easy to study.
For information on how to use Python, 
see \verb+http://docs.python.org+ or many other documents about Python.
  
 % \subsection{Python $\&$ NZMATH}
 %  \subsubsection{Save to a file}
 \subsection{Note on the Document}
   \paragraph{$\dagger$}
   Some beginnings of lines or blocks such as sections or sentences
    may be marked $\dagger$.
   This means these lines or blocks is for advanced users.
   For example, the class \textit{FiniteFieldElement} %
   (See \linkingone{finitefield}{FinitePrimeFieldElement}) is %
   one of abstract classes in NZMATH, %
   which can be inherited to new classes similar to the finite field.
   \paragraph{[$\cdots$]}
   For example, we may sometimes write as \textit{function(a,b[,c,d])}. 
   It means the argument ``c, d'' or only ``d'' can be discarded.
   Such functions use ``default argument values'', 
   which is one of the feature of Python.\\
   (See 
   \verb+http://docs.python.org/tutorial/controlflow.html#default-argument-values+)
   
   \textbf{Warning}:\ Python also have the feature ``keyword arguments''.
   We have tried to keep the feature in NZMATH too.
   However, some functions cannot be used with this feature 
   because these functions are written 
   expecting that arguments are given in order.
%

\renewcommand\documentclass[2][]{}

\chapter{Basic Utilities}\label{utility}
%\documentclass{report}

\documentclass{report}

%%%%%%%%%%%%%%%%%%%%%%%%%%%%%%%%%%%%%%%%%%%%%%%%%%%%%%%%%%%%%
%
% macros for nzmath manual
%
%%%%%%%%%%%%%%%%%%%%%%%%%%%%%%%%%%%%%%%%%%%%%%%%%%%%%%%%%%%%%
\usepackage{amssymb,amsmath}
\usepackage{color}
\usepackage[dvipdfm,bookmarks=true,bookmarksnumbered=true,%
 pdftitle={NZMATH Users Manual},%
 pdfsubject={Manual for NZMATH Users},%
 pdfauthor={NZMATH Development Group},%
 pdfkeywords={TeX; dvipdfmx; hyperref; color;},%
 colorlinks=true]{hyperref}
\usepackage{fancybox}
\usepackage[T1]{fontenc}
%
\newcommand{\DS}{\displaystyle}
\newcommand{\C}{\clearpage}
\newcommand{\NO}{\noindent}
\newcommand{\negok}{$\dagger$}
\newcommand{\spacing}{\vspace{1pt}\\ }
% software macros
\newcommand{\nzmathzero}{{\footnotesize $\mathbb{N}\mathbb{Z}$}\texttt{MATH}}
\newcommand{\nzmath}{{\nzmathzero}\ }
\newcommand{\pythonzero}{$\mbox{\texttt{Python}}$}
\newcommand{\python}{{\pythonzero}\ }
% link macros
\newcommand{\linkingzero}[1]{{\bf \hyperlink{#1}{#1}}}%module
\newcommand{\linkingone}[2]{{\bf \hyperlink{#1.#2}{#2}}}%module,class/function etc.
\newcommand{\linkingtwo}[3]{{\bf \hyperlink{#1.#2.#3}{#3}}}%module,class,method
\newcommand{\linkedzero}[1]{\hypertarget{#1}{}}
\newcommand{\linkedone}[2]{\hypertarget{#1.#2}{}}
\newcommand{\linkedtwo}[3]{\hypertarget{#1.#2.#3}{}}
\newcommand{\linktutorial}[1]{\href{http://docs.python.org/tutorial/#1}{#1}}
\newcommand{\linktutorialone}[2]{\href{http://docs.python.org/tutorial/#1}{#2}}
\newcommand{\linklibrary}[1]{\href{http://docs.python.org/library/#1}{#1}}
\newcommand{\linklibraryone}[2]{\href{http://docs.python.org/library/#1}{#2}}
\newcommand{\pythonhp}{\href{http://www.python.org/}{\python website}}
\newcommand{\nzmathwiki}{\href{http://nzmath.sourceforge.net/wiki/}{{\nzmathzero}Wiki}}
\newcommand{\nzmathsf}{\href{http://sourceforge.net/projects/nzmath/}{\nzmath Project Page}}
\newcommand{\nzmathtnt}{\href{http://tnt.math.se.tmu.ac.jp/nzmath/}{\nzmath Project Official Page}}
% parameter name
\newcommand{\param}[1]{{\tt #1}}
% function macros
\newcommand{\hiki}[2]{{\tt #1}:\ {\em #2}}
\newcommand{\hikiopt}[3]{{\tt #1}:\ {\em #2}=#3}

\newdimen\hoge
\newdimen\truetextwidth
\newcommand{\func}[3]{%
\setbox0\hbox{#1(#2)}
\hoge=\wd0
\truetextwidth=\textwidth
\advance \truetextwidth by -2\oddsidemargin
\ifdim\hoge<\truetextwidth % short form
{\bf \colorbox{skyyellow}{#1(#2)\ $\to$ #3}}
%
\else % long form
\fcolorbox{skyyellow}{skyyellow}{%
   \begin{minipage}{\textwidth}%
   {\bf #1(#2)\\ %
    \qquad\quad   $\to$\ #3}%
   \end{minipage}%
   }%
\fi%
}

\newcommand{\out}[1]{{\em #1}}
\newcommand{\initialize}{%
  \paragraph{\large \colorbox{skyblue}{Initialize (Constructor)}}%
    \quad\\ %
    \vspace{3pt}\\
}
\newcommand{\method}{\C \paragraph{\large \colorbox{skyblue}{Methods}}}
% Attribute environment
\newenvironment{at}
{%begin
\paragraph{\large \colorbox{skyblue}{Attribute}}
\quad\\
\begin{description}
}%
{%end
\end{description}
}
% Operation environment
\newenvironment{op}
{%begin
\paragraph{\large \colorbox{skyblue}{Operations}}
\quad\\
\begin{table}[h]
\begin{center}
\begin{tabular}{|l|l|}
\hline
operator & explanation\\
\hline
}%
{%end
\hline
\end{tabular}
\end{center}
\end{table}
}
% Examples environment
\newenvironment{ex}%
{%begin
\paragraph{\large \colorbox{skyblue}{Examples}}
\VerbatimEnvironment
\renewcommand{\EveryVerbatim}{\fontencoding{OT1}\selectfont}
\begin{quote}
\begin{Verbatim}
}%
{%end
\end{Verbatim}
\end{quote}
}
%
\definecolor{skyblue}{cmyk}{0.2, 0, 0.1, 0}
\definecolor{skyyellow}{cmyk}{0.1, 0.1, 0.5, 0}
%
%\title{NZMATH User Manual\\ {\large{(for version 1.0)}}}
%\date{}
%\author{}
\begin{document}
%\maketitle
%
\setcounter{tocdepth}{3}
\setcounter{secnumdepth}{3}


\tableofcontents
\C

\chapter{Basic Utilities}


%---------- start document ---------- %
 \section{config -- setting features}\linkedzero{config}
%
All constants in the module can be set in user's config file.
See the~\hyperlink{config.user}{User Settings} section for
more detailed description.

  \subsection{Default Settings}\linkedone{config}{default}

  \subsubsection{Dependencies}\linkedone{config}{dependencies}

  Some third party / platform dependent modules are possibly used, and
  they are configurable.

  \paragraph{HAVE\_MPMATH}\linkedone{config}{HAVE\_MPMATH}

  {\tt mpmath} is a package providing multiprecision math.
  See its \href{http://code.google.com/p/mpmath}{project page}.
  This package is used in \linkingzero{ecpp} module.

  \paragraph{HAVE\_SQLITE3}\linkedone{config}{HAVE\_SQLITE3}

  {\tt sqlite3} is the default database module for \python,
  but it need to be enabled at the build time.

  \paragraph{HAVE\_NET}\linkedone{config}{HAVE_NET}

  Some functions will connect to the Net.
  Desktop machines are usually connected to the Net, but notebooks may
  have connectivity only occasionally.

  \subsubsection{Plug-ins}\linkedone{config}{Plug-ins}

  \paragraph{PLUGIN\_MATH}\linkedone{config}{PLUGIN\_MATH}
  \python standard float/complex types and \linklibrary{math}/\linklibrary{cmath} modules only
  provide fixed precision (double precision), but sometimes
  multiprecision floating point is needed.

  \subsubsection{Assumptions}\linkedone{config}{assumptions}

  Some conjectures are useful for assuring the validity of a faster
  algorithm.

  All assumptions are default to {\tt False}, but you can set them
  {\tt True} if you believe them.

  \paragraph{GRH}\linkedone{config}{GRH}

  Generalized Riemann Hypothesis.  For example, primality test is
  \(O((\log n)^2)\) if GRH is true while \(O((\log n)^6)\) or something
  without it.

  \subsubsection{Files}\linkedone{config}{files}

  \paragraph{DATADIR}\linkedone{config}{DATADIR}

  The directory where \nzmath (static) data files are stored. The
  default will be {\tt os.path.join(sys.prefix, 'share', 'nzmath')} or
  {\tt os.path.join(sys.prefix, 'Data', 'nzmath')} on Windows.

  \subsection{Automatic Configuration}\linkedone{config}{auto}

  The items above can be set automatically by testing the environment.

  \subsubsection{Checks}\linkedone{config}{checks}

  Here are check functions.

  The constants accompanying the check functions which enable the check
  if it is {\tt True}, can be overridden in user settings.

  Both check functions and constants are not exposed.

  \paragraph{check\_mpmath()}\linkedone{config}{check\_mpmath}

  Check whether {\tt mpmath} is available or not.

  constant: {\tt CHECK\_MPMATH}

  \paragraph{check\_sqlite3()}\linkedone{config}{check\_sqlite3}

  Check if {\tt sqlite3} is importable or not.
  {\tt pysqlite2} may be a substitution.

  constant: {\tt CHECK\_SQLITE3}

  \paragraph{check\_net()}\linkedone{config}{check\_net}

  Check the net connection by HTTP call.

  constant: {\tt CHECK\_NET}

  \paragraph{check\_plugin\_math()}\linkedone{config}{check\_plugin\_math}

  Check which math plug-in is available.

  constant: {\tt CHECK\_PLUGIN\_MATH}

  \paragraph{default\_datadir()}\linkedone{config}{default\_datadir}

  Return default value for {\tt DATADIR}.

  This function selects the value from various candidates.
  If this function is called with {\tt DATADIR} set, the value of (previously-defined) {\tt DATADIR} is the first candidate to be returned. Other
  possibilities are, {\tt sys.prefix + 'Data/nzmath'} on Windows, or
  {\tt sys.prefix + 'share/nzmath'} on other platforms.

  Be careful that all the above paths do not exist, the function
  returns {\tt None}.

  constant: {\tt CHECK\_DATADIR}

  \subsection{User Settings}\linkedone{config}{user}

  The module try to load the user's config file named
  {\it nzmathconf.py}. The search path is the following:
  \begin{enumerate}
  \item The directory which is specified by an environment variable
    {\tt NZMATHCONFDIR}.
  \item If the platform is Windows, then
    \begin{enumerate}
    \item If an environment variable {\tt APPDATA} is set, {\tt
        APPDATA/nzmath}.
    \item If, alternatively, an environment variable {\tt USERPROFILE}
      is set,\linebreak {\tt USERPROFILE/Application~Data/nzmath}.
    \end{enumerate}
  \item On other platforms, if an environment variable {\tt HOME} is
    set, {\tt HOME/.nzmath.d}.
  \end{enumerate}

  {\it nzmathconf.py} is a \python script. Users can set the constants
  like {\tt HAVE\_MPMATH}, which will override the default settings. These
  constants, except assumption ones, are automatically set, unless
  constants accompanying the check functions are false (see 
  the~\hyperlink{config.auto}{Automatic Configuration} section above).

%---------- end document ---------- %

\bibliographystyle{jplain}%use jbibtex
\bibliography{nzmath_references}

\end{document}


%\documentclass{report}

\documentclass{report}

%%%%%%%%%%%%%%%%%%%%%%%%%%%%%%%%%%%%%%%%%%%%%%%%%%%%%%%%%%%%%
%
% macros for nzmath manual
%
%%%%%%%%%%%%%%%%%%%%%%%%%%%%%%%%%%%%%%%%%%%%%%%%%%%%%%%%%%%%%
\usepackage{amssymb,amsmath}
\usepackage{color}
\usepackage[dvipdfm,bookmarks=true,bookmarksnumbered=true,%
 pdftitle={NZMATH Users Manual},%
 pdfsubject={Manual for NZMATH Users},%
 pdfauthor={NZMATH Development Group},%
 pdfkeywords={TeX; dvipdfmx; hyperref; color;},%
 colorlinks=true]{hyperref}
\usepackage{fancybox}
\usepackage[T1]{fontenc}
%
\newcommand{\DS}{\displaystyle}
\newcommand{\C}{\clearpage}
\newcommand{\NO}{\noindent}
\newcommand{\negok}{$\dagger$}
\newcommand{\spacing}{\vspace{1pt}\\ }
% software macros
\newcommand{\nzmathzero}{{\footnotesize $\mathbb{N}\mathbb{Z}$}\texttt{MATH}}
\newcommand{\nzmath}{{\nzmathzero}\ }
\newcommand{\pythonzero}{$\mbox{\texttt{Python}}$}
\newcommand{\python}{{\pythonzero}\ }
% link macros
\newcommand{\linkingzero}[1]{{\bf \hyperlink{#1}{#1}}}%module
\newcommand{\linkingone}[2]{{\bf \hyperlink{#1.#2}{#2}}}%module,class/function etc.
\newcommand{\linkingtwo}[3]{{\bf \hyperlink{#1.#2.#3}{#3}}}%module,class,method
\newcommand{\linkedzero}[1]{\hypertarget{#1}{}}
\newcommand{\linkedone}[2]{\hypertarget{#1.#2}{}}
\newcommand{\linkedtwo}[3]{\hypertarget{#1.#2.#3}{}}
\newcommand{\linktutorial}[1]{\href{http://docs.python.org/tutorial/#1}{#1}}
\newcommand{\linktutorialone}[2]{\href{http://docs.python.org/tutorial/#1}{#2}}
\newcommand{\linklibrary}[1]{\href{http://docs.python.org/library/#1}{#1}}
\newcommand{\linklibraryone}[2]{\href{http://docs.python.org/library/#1}{#2}}
\newcommand{\pythonhp}{\href{http://www.python.org/}{\python website}}
\newcommand{\nzmathwiki}{\href{http://nzmath.sourceforge.net/wiki/}{{\nzmathzero}Wiki}}
\newcommand{\nzmathsf}{\href{http://sourceforge.net/projects/nzmath/}{\nzmath Project Page}}
\newcommand{\nzmathtnt}{\href{http://tnt.math.se.tmu.ac.jp/nzmath/}{\nzmath Project Official Page}}
% parameter name
\newcommand{\param}[1]{{\tt #1}}
% function macros
\newcommand{\hiki}[2]{{\tt #1}:\ {\em #2}}
\newcommand{\hikiopt}[3]{{\tt #1}:\ {\em #2}=#3}

\newdimen\hoge
\newdimen\truetextwidth
\newcommand{\func}[3]{%
\setbox0\hbox{#1(#2)}
\hoge=\wd0
\truetextwidth=\textwidth
\advance \truetextwidth by -2\oddsidemargin
\ifdim\hoge<\truetextwidth % short form
{\bf \colorbox{skyyellow}{#1(#2)\ $\to$ #3}}
%
\else % long form
\fcolorbox{skyyellow}{skyyellow}{%
   \begin{minipage}{\textwidth}%
   {\bf #1(#2)\\ %
    \qquad\quad   $\to$\ #3}%
   \end{minipage}%
   }%
\fi%
}

\newcommand{\out}[1]{{\em #1}}
\newcommand{\initialize}{%
  \paragraph{\large \colorbox{skyblue}{Initialize (Constructor)}}%
    \quad\\ %
    \vspace{3pt}\\
}
\newcommand{\method}{\C \paragraph{\large \colorbox{skyblue}{Methods}}}
% Attribute environment
\newenvironment{at}
{%begin
\paragraph{\large \colorbox{skyblue}{Attribute}}
\quad\\
\begin{description}
}%
{%end
\end{description}
}
% Operation environment
\newenvironment{op}
{%begin
\paragraph{\large \colorbox{skyblue}{Operations}}
\quad\\
\begin{table}[h]
\begin{center}
\begin{tabular}{|l|l|}
\hline
operator & explanation\\
\hline
}%
{%end
\hline
\end{tabular}
\end{center}
\end{table}
}
% Examples environment
\newenvironment{ex}%
{%begin
\paragraph{\large \colorbox{skyblue}{Examples}}
\VerbatimEnvironment
\renewcommand{\EveryVerbatim}{\fontencoding{OT1}\selectfont}
\begin{quote}
\begin{Verbatim}
}%
{%end
\end{Verbatim}
\end{quote}
}
%
\definecolor{skyblue}{cmyk}{0.2, 0, 0.1, 0}
\definecolor{skyyellow}{cmyk}{0.1, 0.1, 0.5, 0}
%
%\title{NZMATH User Manual\\ {\large{(for version 1.0)}}}
%\date{}
%\author{}
\begin{document}
%\maketitle
%
\setcounter{tocdepth}{3}
\setcounter{secnumdepth}{3}


\tableofcontents
\C

\chapter{Functions}


%---------- start document ---------- %
 \section{bigrandom -- random numbers}\linkedzero{bigrandom}
%
 \paragraph{Historical Note}\label{bigrandom_historical_note}

 The module was written for replacement of the \python standard module
 \linklibrary{random}, because in the era of \python 2.2 (prehistorical period of
 \nzmath) the random module raises {\tt OverflowError} for long integer
 arguments for the \linklibraryone{random\#random.randrange}{randrange} function, which is the only function
 having a use case in \nzmath.

 After the creation of \python 2.3, it was theoretically possible to
 use {\tt random.randrange}, since it started to accept long integer
 as its argument. Use of it was, however, not considered, since there
 had been the {\tt bigrandom} module. It was lucky for us. In fall of
 2006, we found a bug in {\tt random.randrange} and reported it (see
 issue tracker); the {\tt random.randrange} accepts long integers but
 returns unreliable result for truly big integers. The bug was fixed
 for \python 2.5.1. You can, therefore, use {\tt random.randrange}
 instead of {\tt bigrandom.randrange} for \python 2.5.1 or higher.

  \subsection{random -- random number generator}\linkedone{bigrandom}{random}
   \func{random}
   {}{{\em float}}\\
   \spacing
   % document of basic document
   \quad Return a random floating point number in the interval \([0, 1)\).\\
   \spacing
   % added document
   \quad This function is an alias to {\tt random.random} in the \python standard library.\\
   % input, output document
%
  \subsection{randrange -- random integer generator}\linkedone{bigrandom}{randrange}
   \func{randrange}
   {%
     \hiki{start}{integer},\ %
     \hikiopt{stop}{integer}{None},\ %
     \hikiopt{step}{integer}{1}
   }{%
     {\em integer}
   }\\
   \spacing
   % document of basic document
   \quad Return a random integer in the range.\\
   \spacing
   % input, output document
   \quad The argument names do not correspond to their roles, but
   users are familiar with the
   \linklibraryone{functions\#range}{range} built-in function of
   \python and understand the semantics.  Calling with one argument
   \(n\), then the result is an integer in the range \([0, n)\) chosen
   randomly.  With two arguments \(n\) and \(m\),
   in \([n, m)\), and with third \(l\), in \([n, m) \cap (n + l\mathbb{Z})\).\\
   \spacing \quad This function is almost the same as {\tt
     random.randrange} in the \python standard library.  See the
   historical note~\ref{bigrandom_historical_note}.
%
\begin{ex}
>>> randrange(4, 10000, 3)
9919L
>>> randrange(4 * 10**60)
31925916908162253969182327491823596145612834799876775114620151L
\end{ex}%Don't indent!(indent causes an error.)

   \subsection{map\_choice -- choice from image of mapping}\linkedone{bigrandom}{map\_choice}
   \func{map\_choice}
   {%
     \hiki{mapping}{function},\ %
     \hiki{upperbound}{integer}
   }{%
     {\em integer}
   }\\
   \spacing
   \quad Return a choice from a set given as the image of the mapping
   from natural numbers (more precisely {\tt range(upperbound)}).  In
   other words, it is equivalent to:
   {\tt random.\linklibraryone{random\#choice}{choice}([mapping(i) for i in range(upperbound)])},
   if \param{upperbound} is small enough for the list size limit.\\
   \spacing
   \quad The \param{mapping} can be a partial function, i.e. it may return
   {\tt None} for some input. However, if the resulting set is empty, it
   will end up with an infinite loop.\\
\C

%---------- end document ---------- %

\bibliographystyle{jplain}%use jbibtex
\bibliography{nzmath_references}

\end{document}


%\documentclass{report}

%%%%%%%%%%%%%%%%%%%%%%%%%%%%%%%%%%%%%%%%%%%%%%%%%%%%%%%%%%%%%
%
% macros for nzmath manual
%
%%%%%%%%%%%%%%%%%%%%%%%%%%%%%%%%%%%%%%%%%%%%%%%%%%%%%%%%%%%%%
\usepackage{amssymb,amsmath}
\usepackage{color}
\usepackage[dvipdfm,bookmarks=true,bookmarksnumbered=true,%
 pdftitle={NZMATH Users Manual},%
 pdfsubject={Manual for NZMATH Users},%
 pdfauthor={NZMATH Development Group},%
 pdfkeywords={TeX; dvipdfmx; hyperref; color;},%
 colorlinks=true]{hyperref}
\usepackage{fancybox}
\usepackage[T1]{fontenc}
%
\newcommand{\DS}{\displaystyle}
\newcommand{\C}{\clearpage}
\newcommand{\NO}{\noindent}
\newcommand{\negok}{$\dagger$}
\newcommand{\spacing}{\vspace{1pt}\\ }
% software macros
\newcommand{\nzmathzero}{{\footnotesize $\mathbb{N}\mathbb{Z}$}\texttt{MATH}}
\newcommand{\nzmath}{{\nzmathzero}\ }
\newcommand{\pythonzero}{$\mbox{\texttt{Python}}$}
\newcommand{\python}{{\pythonzero}\ }
% link macros
\newcommand{\linkingzero}[1]{{\bf \hyperlink{#1}{#1}}}%module
\newcommand{\linkingone}[2]{{\bf \hyperlink{#1.#2}{#2}}}%module,class/function etc.
\newcommand{\linkingtwo}[3]{{\bf \hyperlink{#1.#2.#3}{#3}}}%module,class,method
\newcommand{\linkedzero}[1]{\hypertarget{#1}{}}
\newcommand{\linkedone}[2]{\hypertarget{#1.#2}{}}
\newcommand{\linkedtwo}[3]{\hypertarget{#1.#2.#3}{}}
\newcommand{\linktutorial}[1]{\href{http://docs.python.org/tutorial/#1}{#1}}
\newcommand{\linktutorialone}[2]{\href{http://docs.python.org/tutorial/#1}{#2}}
\newcommand{\linklibrary}[1]{\href{http://docs.python.org/library/#1}{#1}}
\newcommand{\linklibraryone}[2]{\href{http://docs.python.org/library/#1}{#2}}
\newcommand{\pythonhp}{\href{http://www.python.org/}{\python website}}
\newcommand{\nzmathwiki}{\href{http://nzmath.sourceforge.net/wiki/}{{\nzmathzero}Wiki}}
\newcommand{\nzmathsf}{\href{http://sourceforge.net/projects/nzmath/}{\nzmath Project Page}}
\newcommand{\nzmathtnt}{\href{http://tnt.math.se.tmu.ac.jp/nzmath/}{\nzmath Project Official Page}}
% parameter name
\newcommand{\param}[1]{{\tt #1}}
% function macros
\newcommand{\hiki}[2]{{\tt #1}:\ {\em #2}}
\newcommand{\hikiopt}[3]{{\tt #1}:\ {\em #2}=#3}

\newdimen\hoge
\newdimen\truetextwidth
\newcommand{\func}[3]{%
\setbox0\hbox{#1(#2)}
\hoge=\wd0
\truetextwidth=\textwidth
\advance \truetextwidth by -2\oddsidemargin
\ifdim\hoge<\truetextwidth % short form
{\bf \colorbox{skyyellow}{#1(#2)\ $\to$ #3}}
%
\else % long form
\fcolorbox{skyyellow}{skyyellow}{%
   \begin{minipage}{\textwidth}%
   {\bf #1(#2)\\ %
    \qquad\quad   $\to$\ #3}%
   \end{minipage}%
   }%
\fi%
}

\newcommand{\out}[1]{{\em #1}}
\newcommand{\initialize}{%
  \paragraph{\large \colorbox{skyblue}{Initialize (Constructor)}}%
    \quad\\ %
    \vspace{3pt}\\
}
\newcommand{\method}{\C \paragraph{\large \colorbox{skyblue}{Methods}}}
% Attribute environment
\newenvironment{at}
{%begin
\paragraph{\large \colorbox{skyblue}{Attribute}}
\quad\\
\begin{description}
}%
{%end
\end{description}
}
% Operation environment
\newenvironment{op}
{%begin
\paragraph{\large \colorbox{skyblue}{Operations}}
\quad\\
\begin{table}[h]
\begin{center}
\begin{tabular}{|l|l|}
\hline
operator & explanation\\
\hline
}%
{%end
\hline
\end{tabular}
\end{center}
\end{table}
}
% Examples environment
\newenvironment{ex}%
{%begin
\paragraph{\large \colorbox{skyblue}{Examples}}
\VerbatimEnvironment
\renewcommand{\EveryVerbatim}{\fontencoding{OT1}\selectfont}
\begin{quote}
\begin{Verbatim}
}%
{%end
\end{Verbatim}
\end{quote}
}
%
\definecolor{skyblue}{cmyk}{0.2, 0, 0.1, 0}
\definecolor{skyyellow}{cmyk}{0.1, 0.1, 0.5, 0}
%
%\title{NZMATH User Manual\\ {\large{(for version 1.0)}}}
%\date{}
%\author{}
\begin{document}
%\maketitle
%
\setcounter{tocdepth}{3}
\setcounter{secnumdepth}{3}


\tableofcontents
\C

\chapter{Functions}

%---------- start document ---------- %
 \section{bigrange -- range-like generator functions}\linkedzero{bigrange}
%
  \subsection{count -- count up}\linkedone{bigrange}{count}
   \func{count}
   {%
     \hikiopt{n}{integer}{0}
   }{%
     {\em iterator}
   }\\
   \spacing
   % document of basic document
   \quad Count up infinitely from \param{n} (default to \(0\)).
   See \linklibrary{itertools}.\linklibraryone{itertools\#count}{count}.\\
   \spacing
   % added document
   %\spacing
   % input, output document
   \quad \param{n} must be int, long or rational.Integer.\\

  \subsection{range -- range-like iterator}\linkedone{bigrange}{range}
   \func{range}
   {%
     \hiki{start}{integer},\ %
     \hikiopt{stop}{integer}{None},\ %
     \hikiopt{step}{integer}{1}
   }{%
     {\em iterator}
   }\\
   \spacing
   % document of basic document
   \quad Return a range-like iterator which generates a finite integer sequence.\\
   \spacing
   % added document
   \quad  It can generate more than
   \linklibrary{sys}.\linklibraryone{sys\#maxint}{maxint} elements,
   which is the limitation of the \linklibraryone{functions\#range}{range}
   built-in function.\\
   \spacing
   % input, output document
   \quad The argument names do not correspond to their roles, but
   users are familiar with the
   \linklibraryone{functions\#range}{range} built-in function of
   \python and understand the semantics.
   Note that the output is not a list.\\
%
\begin{ex}
>>> range(1, 100, 3) # built-in
[1, 4, 7, 10, 13, 16, 19, 22, 25, 28, 31, 34, 37, 40, 43, 46,
 49, 52, 55, 58, 61, 64, 67, 70, 73, 76, 79, 82, 85, 88, 91,
 94, 97]
>>> bigrange.range(1, 100, 3)
<generator object at 0x18f8c8>
\end{ex}%Don't indent!(indent causes an error.)

  \subsection{arithmetic\_progression -- arithmetic progression iterator}\linkedone{bigrange}{arithmetic\_progression}
   \func{arithmetic\_progression}
   {%
     \hiki{init}{integer},\ %
     \hiki{difference}{integer}
   }{%
     {\em iterator}
   }\\
   \spacing
   % document of basic document
   \quad Return an iterator which generates an arithmetic progression
   starting form \param{init} and \param{difference} step.\\

  \subsection{geometric\_progression -- geometric progression iterator}\linkedone{bigrange}{geometric\_progression}
   \func{geometric\_progression}
   {%
     \hiki{init}{integer},\ %
     \hiki{ratio}{integer}
   }{%
     {\em iterator}
   }\\
   \spacing
   % document of basic document
   \quad Return an iterator which generates a geometric progression
   starting form \param{init} and multiplying \param{ratio}.\\

  \subsection{multirange -- multiple range iterator}\linkedone{bigrange}{multirange}
   \func{multirange}
   {%
     \hiki{triples}{list of range triples}
   }{%
     {\em iterator}
   }\\
   \spacing
   % document of basic document
   \quad Return an iterator over Cartesian product of elements of ranges.\\
   \spacing
   % added document
   \quad Be cautious that using multirange usually means you are
   trying to do brute force looping.\\
   \spacing
   % input / output
   \quad The range triples may be doubles {\tt (start, stop)} or single {\tt (stop,)},
   but they have to be always tuples.\\
   
%
\begin{ex}
>>> bigrange.multirange([(1, 10, 3), (1, 10, 4)])
<generator object at 0x18f968>
>>> list(_)
[(1, 1), (1, 5), (1, 9), (4, 1), (4, 5), (4, 9), (7, 1),
 (7, 5), (7, 9)]
\end{ex}%Don't indent!(indent causes an error.)

  \subsection{multirange\_restrictions -- multiple range iterator with restrictions}\linkedone{bigrange}{multirange\_restrictions}
   \func{multirange\_restrictions}
   {%
     \hiki{triples}{list of range triples},\ %
     \hiki{**kwds}{keyword arguments}
   }{%
     {\em iterator}
   }\\
   \spacing
   % document of basic document
   \quad {\tt multirange\_restrictions} is an iterator similar to the
   {\tt multirange} but putting restrictions on each ranges.\\
   \spacing
   % added document
   \quad Restrictions are specified by keyword arguments: \param{ascending},
   \param{descending}, \param{strictly\_ascending} and
   \param{strictly\_descending}.

   A restriction \param{ascending}, for example, is a sequence that
   specifies the indices where the number emitted by the range should
   be greater than or equal to the number at the previous index.
   Other restrictions \param{descending}, \param{strictly\_ascending}
   and \param{strictly\_descending} are similar.  Compare the examples
   below and of \linkingone{bigrange}{multirange}.

%
\begin{ex}
>>> bigrange.multirange_restrictions([(1, 10, 3), (1, 10, 4)], ascending=(1,))
<generator object at 0x18f978>
>>> list(_)
[(1, 1), (1, 5), (1, 9), (4, 5), (4, 9), (7, 9)]
\end{ex}%Don't indent!(indent causes an error.)
\C
%---------- end document ---------- %

\bibliographystyle{jplain}%use jbibtex
\bibliography{nzmath_references}

\end{document}
%\documentclass{report}

%%%%%%%%%%%%%%%%%%%%%%%%%%%%%%%%%%%%%%%%%%%%%%%%%%%%%%%%%%%%%
%
% macros for nzmath manual
%
%%%%%%%%%%%%%%%%%%%%%%%%%%%%%%%%%%%%%%%%%%%%%%%%%%%%%%%%%%%%%
\usepackage{amssymb,amsmath}
\usepackage{color}
\usepackage[dvipdfm,bookmarks=true,bookmarksnumbered=true,%
 pdftitle={NZMATH Users Manual},%
 pdfsubject={Manual for NZMATH Users},%
 pdfauthor={NZMATH Development Group},%
 pdfkeywords={TeX; dvipdfmx; hyperref; color;},%
 colorlinks=true]{hyperref}
\usepackage{fancybox}
\usepackage[T1]{fontenc}
%
\newcommand{\DS}{\displaystyle}
\newcommand{\C}{\clearpage}
\newcommand{\NO}{\noindent}
\newcommand{\negok}{$\dagger$}
\newcommand{\spacing}{\vspace{1pt}\\ }
% software macros
\newcommand{\nzmathzero}{{\footnotesize $\mathbb{N}\mathbb{Z}$}\texttt{MATH}}
\newcommand{\nzmath}{{\nzmathzero}\ }
\newcommand{\pythonzero}{$\mbox{\texttt{Python}}$}
\newcommand{\python}{{\pythonzero}\ }
% link macros
\newcommand{\linkingzero}[1]{{\bf \hyperlink{#1}{#1}}}%module
\newcommand{\linkingone}[2]{{\bf \hyperlink{#1.#2}{#2}}}%module,class/function etc.
\newcommand{\linkingtwo}[3]{{\bf \hyperlink{#1.#2.#3}{#3}}}%module,class,method
\newcommand{\linkedzero}[1]{\hypertarget{#1}{}}
\newcommand{\linkedone}[2]{\hypertarget{#1.#2}{}}
\newcommand{\linkedtwo}[3]{\hypertarget{#1.#2.#3}{}}
\newcommand{\linktutorial}[1]{\href{http://docs.python.org/tutorial/#1}{#1}}
\newcommand{\linktutorialone}[2]{\href{http://docs.python.org/tutorial/#1}{#2}}
\newcommand{\linklibrary}[1]{\href{http://docs.python.org/library/#1}{#1}}
\newcommand{\linklibraryone}[2]{\href{http://docs.python.org/library/#1}{#2}}
\newcommand{\pythonhp}{\href{http://www.python.org/}{\python website}}
\newcommand{\nzmathwiki}{\href{http://nzmath.sourceforge.net/wiki/}{{\nzmathzero}Wiki}}
\newcommand{\nzmathsf}{\href{http://sourceforge.net/projects/nzmath/}{\nzmath Project Page}}
\newcommand{\nzmathtnt}{\href{http://tnt.math.se.tmu.ac.jp/nzmath/}{\nzmath Project Official Page}}
% parameter name
\newcommand{\param}[1]{{\tt #1}}
% function macros
\newcommand{\hiki}[2]{{\tt #1}:\ {\em #2}}
\newcommand{\hikiopt}[3]{{\tt #1}:\ {\em #2}=#3}

\newdimen\hoge
\newdimen\truetextwidth
\newcommand{\func}[3]{%
\setbox0\hbox{#1(#2)}
\hoge=\wd0
\truetextwidth=\textwidth
\advance \truetextwidth by -2\oddsidemargin
\ifdim\hoge<\truetextwidth % short form
{\bf \colorbox{skyyellow}{#1(#2)\ $\to$ #3}}
%
\else % long form
\fcolorbox{skyyellow}{skyyellow}{%
   \begin{minipage}{\textwidth}%
   {\bf #1(#2)\\ %
    \qquad\quad   $\to$\ #3}%
   \end{minipage}%
   }%
\fi%
}

\newcommand{\out}[1]{{\em #1}}
\newcommand{\initialize}{%
  \paragraph{\large \colorbox{skyblue}{Initialize (Constructor)}}%
    \quad\\ %
    \vspace{3pt}\\
}
\newcommand{\method}{\C \paragraph{\large \colorbox{skyblue}{Methods}}}
% Attribute environment
\newenvironment{at}
{%begin
\paragraph{\large \colorbox{skyblue}{Attribute}}
\quad\\
\begin{description}
}%
{%end
\end{description}
}
% Operation environment
\newenvironment{op}
{%begin
\paragraph{\large \colorbox{skyblue}{Operations}}
\quad\\
\begin{table}[h]
\begin{center}
\begin{tabular}{|l|l|}
\hline
operator & explanation\\
\hline
}%
{%end
\hline
\end{tabular}
\end{center}
\end{table}
}
% Examples environment
\newenvironment{ex}%
{%begin
\paragraph{\large \colorbox{skyblue}{Examples}}
\VerbatimEnvironment
\renewcommand{\EveryVerbatim}{\fontencoding{OT1}\selectfont}
\begin{quote}
\begin{Verbatim}
}%
{%end
\end{Verbatim}
\end{quote}
}
%
\definecolor{skyblue}{cmyk}{0.2, 0, 0.1, 0}
\definecolor{skyyellow}{cmyk}{0.1, 0.1, 0.5, 0}
%
%\title{NZMATH User Manual\\ {\large{(for version 1.0)}}}
%\date{}
%\author{}
\begin{document}
%\maketitle
%
\setcounter{tocdepth}{3}
\setcounter{secnumdepth}{3}


\tableofcontents
\C

\chapter{Basic Utilities}


%---------- start document ---------- %
 \section{compatibility -- Keep compatibility between \python versions}\linkedzero{compatibility}
%
This module should be simply imported:\\
{\tt import nzmath.compatibility}\\
then it will do its tasks.

  \subsection{set, frozenset}\linkedone{compatibility}{set}

  The module provides {\tt set} for \python~2.3. \python \(\geq\) 2.4
  have \linklibraryone{stdtypes\#set-types-set-frozenset}{set} in
  built-in namespace, while \python~2.3 has {\tt sets} module and
  {\tt sets.Set}. The {\tt set} the module provides for \python~2.3
  is the {\tt sets.Set}. Similarly, {\tt sets.ImmutableSet} would be
  assigned to {\tt frozenset}. Be careful that the compatibility
  is not perfect.  Note also that \nzmath's recommendation is
  \python~2.5 or higher in 2.x series.

  \subsection{card(virtualset)}\linkedone{compatibility}{card}

  Return cardinality of the virtualset.

  The built-in \linklibraryone{stdfunc\#len}{len()} raises 
  \linklibraryone{exceptions\#exceptions.OverflowError}{OverflowError}
  when the result is greater than
  \linklibrary{sys}.\linklibraryone{sys\#maxint}{maxint}. It is not
  clear this restriction will go away in the future.
  The function {\tt card()} ought to be used instead of {\tt len()} for
  obtaining cardinality of sets or set-like objects in nzmath.

\C
%---------- end document ---------- %

\bibliographystyle{jplain}%use jbibtex
\bibliography{nzmath_references}

\end{document}


%
\chapter{Functions}\label{function}
%\documentclass{report}

\documentclass{report}

%%%%%%%%%%%%%%%%%%%%%%%%%%%%%%%%%%%%%%%%%%%%%%%%%%%%%%%%%%%%%
%
% macros for nzmath manual
%
%%%%%%%%%%%%%%%%%%%%%%%%%%%%%%%%%%%%%%%%%%%%%%%%%%%%%%%%%%%%%
\usepackage{amssymb,amsmath}
\usepackage{color}
\usepackage[dvipdfm,bookmarks=true,bookmarksnumbered=true,%
 pdftitle={NZMATH Users Manual},%
 pdfsubject={Manual for NZMATH Users},%
 pdfauthor={NZMATH Development Group},%
 pdfkeywords={TeX; dvipdfmx; hyperref; color;},%
 colorlinks=true]{hyperref}
\usepackage{fancybox}
\usepackage[T1]{fontenc}
%
\newcommand{\DS}{\displaystyle}
\newcommand{\C}{\clearpage}
\newcommand{\NO}{\noindent}
\newcommand{\negok}{$\dagger$}
\newcommand{\spacing}{\vspace{1pt}\\ }
% software macros
\newcommand{\nzmathzero}{{\footnotesize $\mathbb{N}\mathbb{Z}$}\texttt{MATH}}
\newcommand{\nzmath}{{\nzmathzero}\ }
\newcommand{\pythonzero}{$\mbox{\texttt{Python}}$}
\newcommand{\python}{{\pythonzero}\ }
% link macros
\newcommand{\linkingzero}[1]{{\bf \hyperlink{#1}{#1}}}%module
\newcommand{\linkingone}[2]{{\bf \hyperlink{#1.#2}{#2}}}%module,class/function etc.
\newcommand{\linkingtwo}[3]{{\bf \hyperlink{#1.#2.#3}{#3}}}%module,class,method
\newcommand{\linkedzero}[1]{\hypertarget{#1}{}}
\newcommand{\linkedone}[2]{\hypertarget{#1.#2}{}}
\newcommand{\linkedtwo}[3]{\hypertarget{#1.#2.#3}{}}
\newcommand{\linktutorial}[1]{\href{http://docs.python.org/tutorial/#1}{#1}}
\newcommand{\linktutorialone}[2]{\href{http://docs.python.org/tutorial/#1}{#2}}
\newcommand{\linklibrary}[1]{\href{http://docs.python.org/library/#1}{#1}}
\newcommand{\linklibraryone}[2]{\href{http://docs.python.org/library/#1}{#2}}
\newcommand{\pythonhp}{\href{http://www.python.org/}{\python website}}
\newcommand{\nzmathwiki}{\href{http://nzmath.sourceforge.net/wiki/}{{\nzmathzero}Wiki}}
\newcommand{\nzmathsf}{\href{http://sourceforge.net/projects/nzmath/}{\nzmath Project Page}}
\newcommand{\nzmathtnt}{\href{http://tnt.math.se.tmu.ac.jp/nzmath/}{\nzmath Project Official Page}}
% parameter name
\newcommand{\param}[1]{{\tt #1}}
% function macros
\newcommand{\hiki}[2]{{\tt #1}:\ {\em #2}}
\newcommand{\hikiopt}[3]{{\tt #1}:\ {\em #2}=#3}

\newdimen\hoge
\newdimen\truetextwidth
\newcommand{\func}[3]{%
\setbox0\hbox{#1(#2)}
\hoge=\wd0
\truetextwidth=\textwidth
\advance \truetextwidth by -2\oddsidemargin
\ifdim\hoge<\truetextwidth % short form
{\bf \colorbox{skyyellow}{#1(#2)\ $\to$ #3}}
%
\else % long form
\fcolorbox{skyyellow}{skyyellow}{%
   \begin{minipage}{\textwidth}%
   {\bf #1(#2)\\ %
    \qquad\quad   $\to$\ #3}%
   \end{minipage}%
   }%
\fi%
}

\newcommand{\out}[1]{{\em #1}}
\newcommand{\initialize}{%
  \paragraph{\large \colorbox{skyblue}{Initialize (Constructor)}}%
    \quad\\ %
    \vspace{3pt}\\
}
\newcommand{\method}{\C \paragraph{\large \colorbox{skyblue}{Methods}}}
% Attribute environment
\newenvironment{at}
{%begin
\paragraph{\large \colorbox{skyblue}{Attribute}}
\quad\\
\begin{description}
}%
{%end
\end{description}
}
% Operation environment
\newenvironment{op}
{%begin
\paragraph{\large \colorbox{skyblue}{Operations}}
\quad\\
\begin{table}[h]
\begin{center}
\begin{tabular}{|l|l|}
\hline
operator & explanation\\
\hline
}%
{%end
\hline
\end{tabular}
\end{center}
\end{table}
}
% Examples environment
\newenvironment{ex}%
{%begin
\paragraph{\large \colorbox{skyblue}{Examples}}
\VerbatimEnvironment
\renewcommand{\EveryVerbatim}{\fontencoding{OT1}\selectfont}
\begin{quote}
\begin{Verbatim}
}%
{%end
\end{Verbatim}
\end{quote}
}
%
\definecolor{skyblue}{cmyk}{0.2, 0, 0.1, 0}
\definecolor{skyyellow}{cmyk}{0.1, 0.1, 0.5, 0}
%
%\title{NZMATH User Manual\\ {\large{(for version 1.0)}}}
%\date{}
%\author{}
\begin{document}
%\maketitle
%
\setcounter{tocdepth}{3}
\setcounter{secnumdepth}{3}


\tableofcontents
\C

\chapter{Functions}


%---------- start document ---------- %
\section{arith1 - miscellaneous arithmetic functions}\linkedzero{arith1}

\subsection{floorsqrt -- floor of square root}\linkedone{arith1}{floorsqrt}
\func{floorsqrt}{\hiki{a}{integer/\linkingone{rational}{Rational}}}{\out{integer}}\\
\spacing
% document of basi document
\quad Return the floor of square root of \param{a}.\\ 
%\spacing
% input, output document
%\quad Input number \param{a} must be integer or \linkingone{rational}{Rational}.\\
%
\subsection{floorpowerroot -- floor of some power root}\linkedone{arith1}{floorpowerroot}
\func{floorpowerroot}{\hiki{n}{integer},\ \hiki{k}{integer}}{\out{integer}}\\
\spacing
% document of basi document
\quad Return the floor of \param{k}-th power root of \param{n}.\\
%\spacing
% input, output document
%\quad Input numbers \param{n}, \param{k} must be integer.\\
%
\subsection{legendre - Legendre(Jacobi) Symbol}\linkedone{arith1}{legendre}
\func{legendre}{\hiki{a}{integer},\ \hiki{m}{integer}}{\out{integer}}\\
\spacing
% document of basi document
\quad Return the Legendre symbol or Jacobi symbol $\DS \Bigl(\frac{\param{a}}{\param{m}}\Bigr)$.\\
%\spacing
% input, output document
%\quad Input numbers \param{a}, \param{m} must be integer.\\
%
\subsection{modsqrt -- square root of $a$ for modulo $p$}\linkedone{arith1}{modsqrt}
\func{modsqrt}{\hiki{a}{integer}, \, \hiki{p}{integer}}{\out{integer}}\\
\spacing
% document of basi document
\quad Return one of the square roots of \param{a} for modulo \param{p} if square roots are exist, raise ValueError otherwise.\\
\spacing
% add document
%\spacing
% input, output ducument
\quad \param{p} must be a prime number.\\
%
\subsection{expand -- p-adic expansion}\linkedone{arith1}{expand}
\func{expand}{\hiki{n}{integer}, \, \hiki{m}{integer}}{\out{list}}\\
\spacing
% document of basi document
\quad Return the \param{m}-adic expansion of \param{n}.\\ 
\spacing
% input, output document
\quad \param{n} must be nonnegative integer. \param{m} must be greater than or equal to $2$.  The output is a list of expansion coefficients in ascending order.\\
%
\subsection{inverse -- inverse}\linkedone{arith1}{inverse}
\func{inverse}{\hiki{x}{integer}, \, \hiki{p}{integer}}{\out{integer}}\\
\spacing
% document of basi document
\quad Return the inverse of \param{x} for modulo \param{p}.\\
\spacing
% input, output document
\quad \param{p} must be a prime number.\\
%
\subsection{CRT -- Chinese Reminder Theorem}\linkedone{arith1}{CRT}
\func{CRT}{\hiki{nlist}{list}}{\out{integer}}\\
\spacing
% document of basi document
\quad Return the uniquely determined integer satisfying all modulus
conditions given by \param{nlist}.\\
\spacing
% input, output document
\quad Input list \param{nlist} must be the list of a list consisting of two elements.
The first element is remainder and the second is divisor.
They must be integer.\\
%
\subsection{AGM -- Arithmetic Geometric Mean}\linkedone{arith1}{AGM}
\func{AGM}{\hiki{a}{integer},\ \hiki{b}{integer}}{\out{float}}\\
\spacing
% document of basi document
\quad Return the Arithmetic-Geometric Mean of \param{a} and \param{b}.\\
%\spacing
% input, output document
%\quad Input number \param{a}, \param{b} must be integer.\\ 
%
%\subsection{\_BhaskaraBrouncker}\linkedone{arith1}{\_BhaskaraBrouncker}
%\func{\_BhaskaraBrouncker}{\hiki{n}{integer}}{\out{integer}}\\
%\spacing
% document of basi document
%\quad Return the minimum tuple \param{p}, \param{q} such that, $\param{p}^2
%- \param{n} \param{q}^2 = \pm 1$.\\
%\spacing
% input, output document
%\quad Input number \param{n} must be positive integer.
%
\subsection{vp -- $p$-adic valuation}\linkedone{arith1}{vp}
\func{vp}{\hiki{n}{integer},\ \hiki{p}{integer}, \hikiopt{k}{integer}{0}}{\out{tuple}}\\
\spacing
% document of basi document
\quad Return the \param{p}-adic valuation and other part for \param{n}.\\
\spacing
% added document
\quad \negok If $k$ is given, return the valuation and the other part for $\param{n}p^\param{k}$.\\
% input, output document
%\quad Input number \param{n}, \param{p} must be int, long or \linkingone{rational}{Integer}.
%
\subsection{issquare - Is it square?}\linkedone{arith1}{issquare}
\func{issquare}{\hiki{n}{integer}}{\out{integer}}\\
\spacing
% document of basi document
\quad Check if \param{n} is a square number and return square root
of \param{n} if \param{n} is a square.
Otherwise, return \(0\).\\
%\spacing
% input, output document
%\quad Input number \param{n} must be int, long or \linkingone{rational}{Integer}.
%
\subsection{log -- integer part of logarithm}\linkedone{arith1}{log}
\func{log}{\hiki{n}{integer},\ \hikiopt{base}{integer}{2}}{\out{integer}}\\
\spacing
% document of basi document
\quad Return the integer part of logarithm of \param{n} to the \param{base}.\\
%\spacing
% input, output document
%\quad Input number \param{n}, \param{base} must be int, long or \linkingone{rational}{Integer}.
%
\subsection{product -- product of some numbers}\linkedone{arith1}{product}
\func{product}{\hiki{iterable}{list},\ \hikiopt{init}{integer/\linkingone{rational}{Rational}}{None}}{\out{\hiki{prod}{integer/\linkingone{rational}{Rational}}}}\\
\spacing
% document of basic document
\quad Return the products of all elements in \param{iterable}. \\
\spacing
% added document
\quad If \param{init} is given, the multiplication starts with \param{init} instead of the first element in \param{iterable}.\\
\spacing
% input, output document
\quad Input list \param{iterable} must be list of numbers including integers, \linkingone{rational}{Rational} etc.\\
The output \param{prod} may be determined by the type of elements of \param{iterable} and \param{init}.\\
%
\begin{ex}
>>> arith1.AGM(10, 15)
12.373402181181522
>>> arith1.CRT([[2, 5],[3,7]])
17
>>> arith1.CRT([[2, 5], [3, 7], [5, 11]])
192
>>> arith1.expand(194, 5)
[4, 3, 2, 1]
>>> arith1.vp(54, 3)
(3, 2)
>>> arith1.product([1.5, 2, 2.5])
7.5
>>> arith1.product([3, 4], 2)
24
>>> arith1.product([])
1
\end{ex}

%---------- end document ---------- %

\bibliographystyle{jplain}%use jbibtex
\bibliography{nzmath_references}

\end{document}


%\documentclass{report}

\documentclass{report}

%%%%%%%%%%%%%%%%%%%%%%%%%%%%%%%%%%%%%%%%%%%%%%%%%%%%%%%%%%%%%
%
% macros for nzmath manual
%
%%%%%%%%%%%%%%%%%%%%%%%%%%%%%%%%%%%%%%%%%%%%%%%%%%%%%%%%%%%%%
\usepackage{amssymb,amsmath}
\usepackage{color}
\usepackage[dvipdfm,bookmarks=true,bookmarksnumbered=true,%
 pdftitle={NZMATH Users Manual},%
 pdfsubject={Manual for NZMATH Users},%
 pdfauthor={NZMATH Development Group},%
 pdfkeywords={TeX; dvipdfmx; hyperref; color;},%
 colorlinks=true]{hyperref}
\usepackage{fancybox}
\usepackage[T1]{fontenc}
%
\newcommand{\DS}{\displaystyle}
\newcommand{\C}{\clearpage}
\newcommand{\NO}{\noindent}
\newcommand{\negok}{$\dagger$}
\newcommand{\spacing}{\vspace{1pt}\\ }
% software macros
\newcommand{\nzmathzero}{{\footnotesize $\mathbb{N}\mathbb{Z}$}\texttt{MATH}}
\newcommand{\nzmath}{{\nzmathzero}\ }
\newcommand{\pythonzero}{$\mbox{\texttt{Python}}$}
\newcommand{\python}{{\pythonzero}\ }
% link macros
\newcommand{\linkingzero}[1]{{\bf \hyperlink{#1}{#1}}}%module
\newcommand{\linkingone}[2]{{\bf \hyperlink{#1.#2}{#2}}}%module,class/function etc.
\newcommand{\linkingtwo}[3]{{\bf \hyperlink{#1.#2.#3}{#3}}}%module,class,method
\newcommand{\linkedzero}[1]{\hypertarget{#1}{}}
\newcommand{\linkedone}[2]{\hypertarget{#1.#2}{}}
\newcommand{\linkedtwo}[3]{\hypertarget{#1.#2.#3}{}}
\newcommand{\linktutorial}[1]{\href{http://docs.python.org/tutorial/#1}{#1}}
\newcommand{\linktutorialone}[2]{\href{http://docs.python.org/tutorial/#1}{#2}}
\newcommand{\linklibrary}[1]{\href{http://docs.python.org/library/#1}{#1}}
\newcommand{\linklibraryone}[2]{\href{http://docs.python.org/library/#1}{#2}}
\newcommand{\pythonhp}{\href{http://www.python.org/}{\python website}}
\newcommand{\nzmathwiki}{\href{http://nzmath.sourceforge.net/wiki/}{{\nzmathzero}Wiki}}
\newcommand{\nzmathsf}{\href{http://sourceforge.net/projects/nzmath/}{\nzmath Project Page}}
\newcommand{\nzmathtnt}{\href{http://tnt.math.se.tmu.ac.jp/nzmath/}{\nzmath Project Official Page}}
% parameter name
\newcommand{\param}[1]{{\tt #1}}
% function macros
\newcommand{\hiki}[2]{{\tt #1}:\ {\em #2}}
\newcommand{\hikiopt}[3]{{\tt #1}:\ {\em #2}=#3}

\newdimen\hoge
\newdimen\truetextwidth
\newcommand{\func}[3]{%
\setbox0\hbox{#1(#2)}
\hoge=\wd0
\truetextwidth=\textwidth
\advance \truetextwidth by -2\oddsidemargin
\ifdim\hoge<\truetextwidth % short form
{\bf \colorbox{skyyellow}{#1(#2)\ $\to$ #3}}
%
\else % long form
\fcolorbox{skyyellow}{skyyellow}{%
   \begin{minipage}{\textwidth}%
   {\bf #1(#2)\\ %
    \qquad\quad   $\to$\ #3}%
   \end{minipage}%
   }%
\fi%
}

\newcommand{\out}[1]{{\em #1}}
\newcommand{\initialize}{%
  \paragraph{\large \colorbox{skyblue}{Initialize (Constructor)}}%
    \quad\\ %
    \vspace{3pt}\\
}
\newcommand{\method}{\C \paragraph{\large \colorbox{skyblue}{Methods}}}
% Attribute environment
\newenvironment{at}
{%begin
\paragraph{\large \colorbox{skyblue}{Attribute}}
\quad\\
\begin{description}
}%
{%end
\end{description}
}
% Operation environment
\newenvironment{op}
{%begin
\paragraph{\large \colorbox{skyblue}{Operations}}
\quad\\
\begin{table}[h]
\begin{center}
\begin{tabular}{|l|l|}
\hline
operator & explanation\\
\hline
}%
{%end
\hline
\end{tabular}
\end{center}
\end{table}
}
% Examples environment
\newenvironment{ex}%
{%begin
\paragraph{\large \colorbox{skyblue}{Examples}}
\VerbatimEnvironment
\renewcommand{\EveryVerbatim}{\fontencoding{OT1}\selectfont}
\begin{quote}
\begin{Verbatim}
}%
{%end
\end{Verbatim}
\end{quote}
}
%
\definecolor{skyblue}{cmyk}{0.2, 0, 0.1, 0}
\definecolor{skyyellow}{cmyk}{0.1, 0.1, 0.5, 0}
%
%\title{NZMATH User Manual\\ {\large{(for version 1.0)}}}
%\date{}
%\author{}
\begin{document}
%\maketitle
%
\setcounter{tocdepth}{3}
\setcounter{secnumdepth}{3}


\tableofcontents
\C

\chapter{Functions}


%---------- start document ---------- %
 \section{arygcd -- binary-like gcd algorithms}\linkedzero{arygcd}
%
  \subsection{bit\_num -- the number of bits}\linkedone{arygcd}{bit\_num}
   \func{bit\_num}{\hiki{a}{integer}}{\out{integer}}\\
   \spacing
   % document of basic document
   \quad \param{a}�̃r�b�g���̒l��Ԃ��B\\
   %\spacing
   % added document
   %\spacing
   % input, output document
   %\quad \param{a} must be int, long or \linkingone{rational}{Integer}.\\
%
  \subsection{binarygcd -- gcd by the binary algorithm}\linkedone{arygcd}{binarygcd}
   \func{binarygcd}{\hiki{a}{integer},\ \hiki{b}{integer}}{\out{integer}}\\
   \spacing
   % document of basic document
   \quad�@binary gcd algorithm���g����\param{a},\ \param{b}�̍ő���񐔂̒l��Ԃ��B\\
   %\spacing
   % added document
   %\spacing
   % input, output document
   %\quad \param{a},\ \param{b} must be int, long or \linkingone{rational}{Integer}.\\
%
  \subsection{arygcd\_i -- gcd over gauss-integer}\linkedone{arygcd}{arygcd\_i}
   \func{arygcd\_i}{\hiki{a1}{integer},\ \hiki{a2}{integer},\ \hiki{b1}{integer},\ \hiki{b2}{integer}}{(\out{integer},\ \out{integer})}\\
   \spacing
   % document of basic document
   \quad ��‚�gauss����\param{a1}+\param{a2}$i$,\ \param{b1}+\param{b2}$i$ �̍ő���񐔂̒l��Ԃ��B ``$i$''�͋����B \\
   \spacing
   % added document
   If the output of arygcd\_i(\param{a1}, \param{a2}, \param{b1}, \param{b2}) is (\param{c1}, \param{c2}), then
   the gcd of \param{a1}+\param{a2}$i$ and \ \param{b1}+\param{b2}$i$ equals \param{c1}+\param{c2}$i$.\\
   \negok This function uses $(1+i)$-ary gcd algorithm, which is an generalization of the binary algorithm, proposed by A.Weilert\cite{Weilert}.\\
   %\spacing
   % input, output document
   %\quad \param{a},\ \param{b} must be int, long or \linkingone{rational}{Integer}.\\
%
  \subsection{arygcd\_w -- gcd over Eisenstein-integer}\linkedone{arygcd}{arygcd\_w}
   \func{arygcd\_w}{\hiki{a1}{integer},\ \hiki{a2}{integer},\ \hiki{b1}{integer},\ \hiki{b2}{integer}}{(\out{integer},\ \out{integer})}\\
   \spacing
   % document of basic document
   \quad Eisenstein����\param{a1}+\param{a2}$\omega$,\ \param{b1}+\param{b2}$\omega$�̍ő���񐔂̒l��Ԃ��B
   ``$\omega$''��1 �̋��������B\\
   \spacing
   % added document
   If the output of arygcd\_w(\param{a1}, \param{a2}, \param{b1}, \param{b2}) is (\param{c1}, \param{c2}), then
   the gcd of \param{a1}+\param{a2}$\omega$ and \ \param{b1}+\param{b2}$\omega$ equals \param{c1}+\param{c2}$\omega$.\\
   \negok This functions uses $(1-\omega)$-ary gcd algorithm, which is an generalization of the binary algorithm, proposed by I.B. Damg{\aa}rd and G.S. Frandsen \cite{Dam-Frand}. \\
   %\spacing
   % input, output document
   %\quad \param{a},\ \param{b} must be int, long or \linkingone{rational}{Integer}.\\
%
\begin{ex}
>>> arygcd.binarygcd(32, 48)
16
>>> arygcd_i(1, 13, 13, 9)
(-3, 1)
>>> arygcd_w(2, 13, 33, 15)
(4, 5)
\end{ex}%Don't indent!(indent causes an error.)
\C

%---------- end document ---------- %

\bibliographystyle{jplain}%use jbibtex
\bibliography{nzmath_references}

\end{document}


%\documentclass{report}

\documentclass{report}

%%%%%%%%%%%%%%%%%%%%%%%%%%%%%%%%%%%%%%%%%%%%%%%%%%%%%%%%%%%%%
%
% macros for nzmath manual
%
%%%%%%%%%%%%%%%%%%%%%%%%%%%%%%%%%%%%%%%%%%%%%%%%%%%%%%%%%%%%%
\usepackage{amssymb,amsmath}
\usepackage{color}
\usepackage[dvipdfm,bookmarks=true,bookmarksnumbered=true,%
 pdftitle={NZMATH Users Manual},%
 pdfsubject={Manual for NZMATH Users},%
 pdfauthor={NZMATH Development Group},%
 pdfkeywords={TeX; dvipdfmx; hyperref; color;},%
 colorlinks=true]{hyperref}
\usepackage{fancybox}
\usepackage[T1]{fontenc}
%
\newcommand{\DS}{\displaystyle}
\newcommand{\C}{\clearpage}
\newcommand{\NO}{\noindent}
\newcommand{\negok}{$\dagger$}
\newcommand{\spacing}{\vspace{1pt}\\ }
% software macros
\newcommand{\nzmathzero}{{\footnotesize $\mathbb{N}\mathbb{Z}$}\texttt{MATH}}
\newcommand{\nzmath}{{\nzmathzero}\ }
\newcommand{\pythonzero}{$\mbox{\texttt{Python}}$}
\newcommand{\python}{{\pythonzero}\ }
% link macros
\newcommand{\linkingzero}[1]{{\bf \hyperlink{#1}{#1}}}%module
\newcommand{\linkingone}[2]{{\bf \hyperlink{#1.#2}{#2}}}%module,class/function etc.
\newcommand{\linkingtwo}[3]{{\bf \hyperlink{#1.#2.#3}{#3}}}%module,class,method
\newcommand{\linkedzero}[1]{\hypertarget{#1}{}}
\newcommand{\linkedone}[2]{\hypertarget{#1.#2}{}}
\newcommand{\linkedtwo}[3]{\hypertarget{#1.#2.#3}{}}
\newcommand{\linktutorial}[1]{\href{http://docs.python.org/tutorial/#1}{#1}}
\newcommand{\linktutorialone}[2]{\href{http://docs.python.org/tutorial/#1}{#2}}
\newcommand{\linklibrary}[1]{\href{http://docs.python.org/library/#1}{#1}}
\newcommand{\linklibraryone}[2]{\href{http://docs.python.org/library/#1}{#2}}
\newcommand{\pythonhp}{\href{http://www.python.org/}{\python website}}
\newcommand{\nzmathwiki}{\href{http://nzmath.sourceforge.net/wiki/}{{\nzmathzero}Wiki}}
\newcommand{\nzmathsf}{\href{http://sourceforge.net/projects/nzmath/}{\nzmath Project Page}}
\newcommand{\nzmathtnt}{\href{http://tnt.math.se.tmu.ac.jp/nzmath/}{\nzmath Project Official Page}}
% parameter name
\newcommand{\param}[1]{{\tt #1}}
% function macros
\newcommand{\hiki}[2]{{\tt #1}:\ {\em #2}}
\newcommand{\hikiopt}[3]{{\tt #1}:\ {\em #2}=#3}

\newdimen\hoge
\newdimen\truetextwidth
\newcommand{\func}[3]{%
\setbox0\hbox{#1(#2)}
\hoge=\wd0
\truetextwidth=\textwidth
\advance \truetextwidth by -2\oddsidemargin
\ifdim\hoge<\truetextwidth % short form
{\bf \colorbox{skyyellow}{#1(#2)\ $\to$ #3}}
%
\else % long form
\fcolorbox{skyyellow}{skyyellow}{%
   \begin{minipage}{\textwidth}%
   {\bf #1(#2)\\ %
    \qquad\quad   $\to$\ #3}%
   \end{minipage}%
   }%
\fi%
}

\newcommand{\out}[1]{{\em #1}}
\newcommand{\initialize}{%
  \paragraph{\large \colorbox{skyblue}{Initialize (Constructor)}}%
    \quad\\ %
    \vspace{3pt}\\
}
\newcommand{\method}{\C \paragraph{\large \colorbox{skyblue}{Methods}}}
% Attribute environment
\newenvironment{at}
{%begin
\paragraph{\large \colorbox{skyblue}{Attribute}}
\quad\\
\begin{description}
}%
{%end
\end{description}
}
% Operation environment
\newenvironment{op}
{%begin
\paragraph{\large \colorbox{skyblue}{Operations}}
\quad\\
\begin{table}[h]
\begin{center}
\begin{tabular}{|l|l|}
\hline
operator & explanation\\
\hline
}%
{%end
\hline
\end{tabular}
\end{center}
\end{table}
}
% Examples environment
\newenvironment{ex}%
{%begin
\paragraph{\large \colorbox{skyblue}{Examples}}
\VerbatimEnvironment
\renewcommand{\EveryVerbatim}{\fontencoding{OT1}\selectfont}
\begin{quote}
\begin{Verbatim}
}%
{%end
\end{Verbatim}
\end{quote}
}
%
\definecolor{skyblue}{cmyk}{0.2, 0, 0.1, 0}
\definecolor{skyyellow}{cmyk}{0.1, 0.1, 0.5, 0}
%
%\title{NZMATH User Manual\\ {\large{(for version 1.0)}}}
%\date{}
%\author{}
\begin{document}
%\maketitle
%
\setcounter{tocdepth}{3}
\setcounter{secnumdepth}{3}


\tableofcontents
\C

\chapter{Functions}


%---------- start document ---------- %
 \section{combinatorial -- combinatorial functions}\linkedzero{combinatorial}
%
  \subsection{binomial -- binomial coefficient}\linkedone{combinatorial}{binomial}
   \func{binomial}
   {%
     \hiki{n}{integer},\ %
     \hiki{m}{integer}
   }{%
     \out{integer}
   }\\
   \spacing
   % document of basic document
   \quad Return the binomial coefficient for \param{n} and \param{m}.
   In other words, $\displaystyle{\frac{\param{n} !}{(\param{n} - \param{m}) ! \param{m} !}}$.\\
   \spacing
   % added document
   \negok For convenience, {\tt binomial(n, n+i)} returns \(0\) for positive \(i\), and {\tt binomial(0,0)} returns \(1\).\\
   \spacing
   % input, output document
   \quad \param{n} must be a positive integer and \param{m} must be
   a non-negative integer. \\

  \subsection{combinationIndexGenerator -- iterator for combinations}\linkedone{combinatorial}{combinationIndexGenerator}

   \func{combinationIndexGenerator}{%
     \hiki{n}{integer},\ %
     \hiki{m}{integer}
   }{%
     \out{iterator}
   }\\
   \spacing
   % document of basic document
   \quad Return an iterator which generates indices of \param{m}
   element subsets of \param{n} element set.\\
   \spacing 
   \quad {\tt combination\_index\_generator}\linkedone{combinatorial}{combination\_index\_generator}
   is an alias of {\tt combinationIndexGenerator}.\\

  \subsection{factorial -- factorial}\linkedone{combinatorial}{factorial}
   \func{factorial}{%
     \hiki{n}{integer}
   }{%
     \out{integer}
   }\\
   \spacing
   % document of basic document
   \quad Return \(\param{n}!\) for non-negative integer \param{n}.\\

  \subsection{permutationGenerator -- iterator for permutation}\linkedone{combinatorial}{permutationGenerator}
  \func{permutationGenerator}{%
    \hiki{n}{integer}
  }{%
    \out{iterator}
  }
   \spacing
   \quad Generate all permutations of \param{n} elements as list iterator.\\
   % 
   \spacing 
   \quad The number of generated list is \param{n}'s
   \linkingone{combinatorial}{factorial}, so be careful to use
   big \param{n}.\\
   \spacing 
   \quad {\tt permutation\_generator}\linkedone{combinatorial}{permutation\_generator}
   is an alias of {\tt permutationGenerator}.\\

  \subsection{fallingfactorial -- the falling factorial}\linkedone{combinatorial}{fallingfactorial}
   \func{fallingfactorial}{%
     \hiki{n}{integer},\ %
     \hiki{m}{integer}
   }{%
     \out{integer}
   }\\
   \spacing
   % document of basic document
   \quad Return the falling factorial; \param{n} to the \param{m} falling,
   i.e. \(n(n-1)\cdots(n-m+1)\).\\

  \subsection{risingfactorial -- the rising factorial}\linkedone{combinatorial}{risingfactorial}
   \func{risingfactorial}{%
     \hiki{n}{integer},\ %
     \hiki{m}{integer}
   }{%
     \out{integer}
   }\\
   \spacing
   % document of basic document
   \quad Return the rising factorial; \param{n} to the \param{m} rising,
   i.e.\, \(n(n+1)\cdots(n+m-1)\).\\

  \subsection{multinomial -- the multinomial coefficient}\linkedone{combinatorial}{multinomial}
   \func{multinomial}{%
     \hiki{n}{integer},\ %
     \hiki{parts}{list}
   }{%
     \out{integer}
   }\\
   \spacing
   % document of basic document
   \quad Return the multinomial coefficient.\\
   \spacing
   % input, output document
   \quad \param{parts} must be a sequence of natural numbers and the sum of elements in \param{parts} should be equal to \param{n}.\\

  \subsection{bernoulli -- the Bernoulli number}\linkedone{combinatorial}{bernoulli}
   \func{bernoulli}{%
     \hiki{n}{integer}
   }{%
     \out{Rational}
   }\\
   \spacing
   % document of basic document
   \quad Return the \param{n}-th Bernoulli number.\\

  \subsection{catalan -- the Catalan number}\linkedone{combinatorial}{catalan}
   \func{catalan}{%
     \hiki{n}{integer}
   }{%
     \out{integer}
   }\\
   \spacing
   % document of basic document
   \quad Return the \param{n}-th Catalan number.\\

  \subsection{euler -- the Euler number}\linkedone{combinatorial}{euler}
   \func{euler}{%
     \hiki{n}{integer}
   }{%
     \out{integer}
   }\\
   \spacing
   % document of basic document
   \quad Return the \param{n}-th Euler number.\\

  \subsection{bell -- the Bell number}\linkedone{combinatorial}{bell}
   \func{bell}{%
     \hiki{n}{integer}
   }{%
     \out{integer}
   }\\
   \spacing
   % document of basic document
   \quad Return the \param{n}-th Bell number.\\
   \spacing
   % added document
   \quad The Bell number \(b\) is defined by:
   \[b(n) = \sum_{i=0}^{n} S(n, i),\ \]
   where \(S\) denotes Stirling number of the second kind (\linkingone{combinatorial}{stirling2}).\\

  \subsection{stirling1 -- Stirling number of the first kind}\linkedone{combinatorial}{stirling1}
   \func{stirling1}{%
     \hiki{n}{integer},\ %
     \hiki{m}{integer}
   }{%
     \out{integer}
   }\\
   \spacing
   % document of basic document
   \quad Return Stirling number of the first kind.\\
   \spacing
   % added document
   \quad Let \(s\) denote the Stirling number and \((x)_n\) the falling factorial, then
   \[(x)_n = \sum_{i=0}^{n} s(n,\ i) x^i. \]\\
   \(s\) satisfies the recurrence relation:
   \[s(n,\ m) = s(n-1,\ m-1) - (n-1)S(n-1,\ m)\ .\]\\

  \subsection{stirling2 -- Stirling number of the second kind}\linkedone{combinatorial}{stirling2}
   \func{stirling2}{%
     \hiki{n}{integer},\ %
     \hiki{m}{integer}
   }{%
     \out{integer}
   }\\
   \spacing
   % document of basic document
   \quad Return Stirling number of the second kind.\\
   \spacing
   % added document
   \quad Let \(S\) denote the Stirling number, \((x)_i\) falling factorial, then:
   \[x^n = \sum_{i=0}^{n} S(n,\ i) (x)_i\]
   \(S\) satisfies:
   \[S(n,\ m) = S(n-1,\ m-1) + m S(n-1,\ m)\]\\

  \subsection{partition\_number -- the number of partitions}\linkedone{combinatorial}{partition\_number}
   \func{partition\_number}{%
     \hiki{n}{integer}
   }{%
     \out{integer}
   }\\
   \spacing
   % document of basic document
   \quad Return the number of partitions of \param{n}.\\

  \subsection{partitionGenerator -- iterator for partition}\linkedone{combinatorial}{partitionGenerator}
   \func{partitionGenerator}{%
     \hiki{n}{integer},\ %
     \hikiopt{maxi}{integer}{0}
   }{%
     \out{iterator}
   }\\
   \spacing
   % document of basic document
   \quad Return an iterator which generates partitions of \param{n}.\\
   % input, output document
   \spacing
   \quad If \param{maxi} is given, then summands are limited not to exceed \param{maxi}.\\
   \quad The number of partitions (given by
   \linkingone{combinatorial}{partition\_number}) grows exponentially,
   so be careful to use big \param{n}.\\
   \spacing 
   \quad {\tt partition\_generator}\linkedone{combinatorial}{partition\_generator}
   is an alias of {\tt partitionGenerator}.\\

  \subsection{partition\_conjugate -- the conjugate of partition}\linkedone{combinatorial}{partition\_conjugate}
   \func{partition\_conjugate}{%
     \hiki{partition}{tuple}
   }{%
     \out{tuple}
   }\\
   \spacing
   % document of basic document
   \quad Return the conjugate of \param{partition}.\\

\begin{ex}
>>> combinatorial.binomial(5, 2)
10L
>>> combinatorial.factorial(3)
6L
>>> combinatorial.fallingfactorial(7, 3) == 7 * 6 * 5
True
>>> combinatorial.risingfactorial(7, 3) == 7 * 8 * 9
True
>>> combinatorial.multinomial(7, [2, 2, 3])
210L
>>> for idx in combinatorial.combinationIndexGenerator(5, 3):
...     print idx
...
[0, 1, 2]
[0, 1, 3]
[0, 1, 4]
[0, 2, 3]
[0, 2, 4]
[0, 3, 4]
[1, 2, 3]
[1, 2, 4]
[1, 3, 4]
[2, 3, 4]
>>> for part in combinatorial.partitionGenerator(5):
...     print part
...
(5,)
(4, 1)
(3, 2)
(3, 1, 1)
(2, 2, 1)
(2, 1, 1, 1)
(1, 1, 1, 1, 1)
>>> combinatorial.partition_number(5)
7
>>> def limited_summands(n, maxi):
...     "partition with limited number of summands"
...     for part in combinatorial.partitionGenerator(n, maxi):
...         yield combinatorial.partition_conjugate(part)
...
>>> for part in limited_summands(5, 3):
...     print part
...
(2, 2, 1)
(3, 1, 1)
(3, 2)
(4, 1)
(5,)
\end{ex}%Don't indent!(indent causes an error.)

\C

%---------- end document ---------- %

\bibliographystyle{jplain}%use jbibtex
\bibliography{nzmath_references}

\end{document}


%\documentclass{report}

%%%%%%%%%%%%%%%%%%%%%%%%%%%%%%%%%%%%%%%%%%%%%%%%%%%%%%%%%%%%%
%
% macros for nzmath manual
%
%%%%%%%%%%%%%%%%%%%%%%%%%%%%%%%%%%%%%%%%%%%%%%%%%%%%%%%%%%%%%
\usepackage{amssymb,amsmath}
\usepackage{color}
\usepackage[dvipdfm,bookmarks=true,bookmarksnumbered=true,%
 pdftitle={NZMATH Users Manual},%
 pdfsubject={Manual for NZMATH Users},%
 pdfauthor={NZMATH Development Group},%
 pdfkeywords={TeX; dvipdfmx; hyperref; color;},%
 colorlinks=true]{hyperref}
\usepackage{fancybox}
\usepackage[T1]{fontenc}
%
\newcommand{\DS}{\displaystyle}
\newcommand{\C}{\clearpage}
\newcommand{\NO}{\noindent}
\newcommand{\negok}{$\dagger$}
\newcommand{\spacing}{\vspace{1pt}\\ }
% software macros
\newcommand{\nzmathzero}{{\footnotesize $\mathbb{N}\mathbb{Z}$}\texttt{MATH}}
\newcommand{\nzmath}{{\nzmathzero}\ }
\newcommand{\pythonzero}{$\mbox{\texttt{Python}}$}
\newcommand{\python}{{\pythonzero}\ }
% link macros
\newcommand{\linkingzero}[1]{{\bf \hyperlink{#1}{#1}}}%module
\newcommand{\linkingone}[2]{{\bf \hyperlink{#1.#2}{#2}}}%module,class/function etc.
\newcommand{\linkingtwo}[3]{{\bf \hyperlink{#1.#2.#3}{#3}}}%module,class,method
\newcommand{\linkedzero}[1]{\hypertarget{#1}{}}
\newcommand{\linkedone}[2]{\hypertarget{#1.#2}{}}
\newcommand{\linkedtwo}[3]{\hypertarget{#1.#2.#3}{}}
\newcommand{\linktutorial}[1]{\href{http://docs.python.org/tutorial/#1}{#1}}
\newcommand{\linktutorialone}[2]{\href{http://docs.python.org/tutorial/#1}{#2}}
\newcommand{\linklibrary}[1]{\href{http://docs.python.org/library/#1}{#1}}
\newcommand{\linklibraryone}[2]{\href{http://docs.python.org/library/#1}{#2}}
\newcommand{\pythonhp}{\href{http://www.python.org/}{\python website}}
\newcommand{\nzmathwiki}{\href{http://nzmath.sourceforge.net/wiki/}{{\nzmathzero}Wiki}}
\newcommand{\nzmathsf}{\href{http://sourceforge.net/projects/nzmath/}{\nzmath Project Page}}
\newcommand{\nzmathtnt}{\href{http://tnt.math.se.tmu.ac.jp/nzmath/}{\nzmath Project Official Page}}
% parameter name
\newcommand{\param}[1]{{\tt #1}}
% function macros
\newcommand{\hiki}[2]{{\tt #1}:\ {\em #2}}
\newcommand{\hikiopt}[3]{{\tt #1}:\ {\em #2}=#3}

\newdimen\hoge
\newdimen\truetextwidth
\newcommand{\func}[3]{%
\setbox0\hbox{#1(#2)}
\hoge=\wd0
\truetextwidth=\textwidth
\advance \truetextwidth by -2\oddsidemargin
\ifdim\hoge<\truetextwidth % short form
{\bf \colorbox{skyyellow}{#1(#2)\ $\to$ #3}}
%
\else % long form
\fcolorbox{skyyellow}{skyyellow}{%
   \begin{minipage}{\textwidth}%
   {\bf #1(#2)\\ %
    \qquad\quad   $\to$\ #3}%
   \end{minipage}%
   }%
\fi%
}

\newcommand{\out}[1]{{\em #1}}
\newcommand{\initialize}{%
  \paragraph{\large \colorbox{skyblue}{Initialize (Constructor)}}%
    \quad\\ %
    \vspace{3pt}\\
}
\newcommand{\method}{\C \paragraph{\large \colorbox{skyblue}{Methods}}}
% Attribute environment
\newenvironment{at}
{%begin
\paragraph{\large \colorbox{skyblue}{Attribute}}
\quad\\
\begin{description}
}%
{%end
\end{description}
}
% Operation environment
\newenvironment{op}
{%begin
\paragraph{\large \colorbox{skyblue}{Operations}}
\quad\\
\begin{table}[h]
\begin{center}
\begin{tabular}{|l|l|}
\hline
operator & explanation\\
\hline
}%
{%end
\hline
\end{tabular}
\end{center}
\end{table}
}
% Examples environment
\newenvironment{ex}%
{%begin
\paragraph{\large \colorbox{skyblue}{Examples}}
\VerbatimEnvironment
\renewcommand{\EveryVerbatim}{\fontencoding{OT1}\selectfont}
\begin{quote}
\begin{Verbatim}
}%
{%end
\end{Verbatim}
\end{quote}
}
%
\definecolor{skyblue}{cmyk}{0.2, 0, 0.1, 0}
\definecolor{skyyellow}{cmyk}{0.1, 0.1, 0.5, 0}
%
%\title{NZMATH User Manual\\ {\large{(for version 1.0)}}}
%\date{}
%\author{}
\begin{document}
%\maketitle
%
\setcounter{tocdepth}{3}
\setcounter{secnumdepth}{3}


\tableofcontents
\C

\chapter{Functions}


%---------- start document ---------- %
 \section{cubic\_root -- cubic root, residue, and so on}\linkedzero{cubic\_root}
%
  \subsection{c\_root\_p -- cubic root mod p}\linkedone{cubic\_root}{c\_root\_p}
   \func{c\_root\_p}{\hiki{a}{integer},\ \hiki{p}{integer}}{\out{list}}\\
   \spacing
   % document of basic document
   \quad \param{a}�@\param{p}��\param{a}��3�捪�̒l��Ԃ��B (���Ȃ킿�A $x^3 = \param{a} \pmod{\param{p}}$).\\
   \spacing
   % added document
   %\spacing
   % input, output document
   \quad \param{p}�͑f���B\\
   ���̊֐���\param{a}��3�捪�̂��ׂĂ̒l�����X�g�ŕԂ��B\\
%
  \subsection{c\_residue -- cubic residue mod p}\linkedone{cubic\_root}{c\_residue}
   \func{c\_residue}{\hiki{a}{integer},\ \hiki{p}{integer}}{\out{integer}}\\
   \spacing
   % document of basic document
   \quad �@\param{p}�ŗL����\param{a}���R��ɂȂ��Ă��邩���ׂ�B\\
   \spacing
   % added document
   \quad ����$\param{p}\ |\ \param{a}$�Ȃ�$0$��Ԃ��B
   �܂��A�@\param{p}��\param{a}���R��ɂȂ��Ă���Ȃ��$1$��Ԃ��B
   �����łȂ����(3��ɂȂ��Ă����Ȃ��Ƃ�)$-1$��Ԃ��B\\
   \spacing
   % input, output document
   \quad \param{p}�͑f���B\\
%
  \subsection{c\_symbol -- cubic residue symbol for Eisenstein-integers}\linkedone{cubic\_root}{c\_symbol}
   \func{c\_symbol}{\hiki{a1}{integer},\ \hiki{a2}{integer},\ \hiki{b1}{integer},\ \hiki{b2}{integer}}{\out{integer}}\\
   \spacing
   % document of basic document
   \quad ��‚�Eisenstein�����ł���(Jacobi)������]�L���̒l��Ԃ��B$\left(\frac{\param{a1}+\param{a2}\omega}{\param{b1}+\param{b2}\omega}\right)_3$,
$\omega$�͂P�̂R�捪�̒l�ł���B\\
   \spacing
   % added document
   ����$\param{b1}+\param{b2}\omega$��$\mathbb{Z}[\omega]$�Ɋ܂܂��f���ł���Ȃ�΁A$\param{a1}+\param{a2}\omega$�͗�����]���킩��B\\
   \spacing
   % input, output document
   \quad $\param{b1}+\param{b2}\omega$��$1-\omega$�ɕ������Ȃ��Ɖ��肷��B.\\
%
  \subsection{decomposite\_p -- decomposition to Eisenstein-integers}\linkedone{cubic\_root}{decomposite\_p}
   \func{decomposite\_p}{\hiki{p}{integer}}{(\out{integer},\ \out{integer})}\\
   \spacing
   % document of basic document
   \quad $\mathbb{Z}[\omega]$�Ɋ܂܂��f���̈��\param{p}�̒l��Ԃ��B\\
   \spacing
   % added document
   \quad �����o�͂�(\param{a},\ \param{b})�Ȃ�A$\frac{\param{p}}{\param{a}+\param{b}\omega}$�� $\mathbb{Z}[\omega]$.�Ɋ܂܂��f���ł���B
   ���Ȃ킿\param{p}��$\mathbb{Z}[\omega]$.�Ɋ܂܂��$\param{a}+\param{b}\omega$ and $\param{p}/(\param{a}+\param{b}\omega)$�̓�‚̑f�����ɕ������邱�Ƃ��ł���B\\
   \spacing
   % input, output document
   \quad \param{p}�͗L�������‘f���B
   $\param{p}\equiv 1 \pmod 3$�Ɖ��肷��B\\
%
  \subsection{cornacchia -- solve $x^2+dy^2=p$}\linkedone{cubic\_root}{cornacchia}
   \func{cornacchia}{\hiki{d}{integer},\ \hiki{p}{integer}}{(\out{integer},\ \out{integer})}\\
   \spacing
   % document of basic document
   \quad�@$x^2 + \param{d}y^2 = \param{p}$�̒l��Ԃ��B\\
   \spacing
   % added document
   \quad ���̊֐���Cornacchia�̃A���S���Y�����g�p�B \cite{Cohen1}�Q�ƁB\\
   \spacing
   % input, output document
   \quad \param{p}�͗L�������‘f���B
   \param{d}��$0<\param{d}<\param{p}$�̊֌W���[�����B.
   ���̊֐���$x^2 + \param{d} y^2 = \param{p}$�̒l�Ƃ���(\param{x},\ \param{y})��Ԃ��B\\

\begin{ex}
>>> cubic_root.c_root_p(1, 13)
[1, 3, 9]
>>> cubic_root.c_residue(2, 7)
-1
>>> cubic_root.c_symbol(3, 6, 5, 6)
1
>>> cubic_root.decomposite_p(19)
(2, 5)
>>> cubic_root.cornacchia(5, 29)
(3, 2)
\end{ex}%Don't indent!(indent causes an error.)
\C

%---------- end document ---------- %

\bibliographystyle{jplain}%use jbibtex
\bibliography{nzmath_references}

\end{document}


%\documentclass{report}

%%%%%%%%%%%%%%%%%%%%%%%%%%%%%%%%%%%%%%%%%%%%%%%%%%%%%%%%%%%%%
%
% macros for nzmath manual
%
%%%%%%%%%%%%%%%%%%%%%%%%%%%%%%%%%%%%%%%%%%%%%%%%%%%%%%%%%%%%%
\usepackage{amssymb,amsmath}
\usepackage{color}
\usepackage[dvipdfm,bookmarks=true,bookmarksnumbered=true,%
 pdftitle={NZMATH Users Manual},%
 pdfsubject={Manual for NZMATH Users},%
 pdfauthor={NZMATH Development Group},%
 pdfkeywords={TeX; dvipdfmx; hyperref; color;},%
 colorlinks=true]{hyperref}
\usepackage{fancybox}
\usepackage[T1]{fontenc}
%
\newcommand{\DS}{\displaystyle}
\newcommand{\C}{\clearpage}
\newcommand{\NO}{\noindent}
\newcommand{\negok}{$\dagger$}
\newcommand{\spacing}{\vspace{1pt}\\ }
% software macros
\newcommand{\nzmathzero}{{\footnotesize $\mathbb{N}\mathbb{Z}$}\texttt{MATH}}
\newcommand{\nzmath}{{\nzmathzero}\ }
\newcommand{\pythonzero}{$\mbox{\texttt{Python}}$}
\newcommand{\python}{{\pythonzero}\ }
% link macros
\newcommand{\linkingzero}[1]{{\bf \hyperlink{#1}{#1}}}%module
\newcommand{\linkingone}[2]{{\bf \hyperlink{#1.#2}{#2}}}%module,class/function etc.
\newcommand{\linkingtwo}[3]{{\bf \hyperlink{#1.#2.#3}{#3}}}%module,class,method
\newcommand{\linkedzero}[1]{\hypertarget{#1}{}}
\newcommand{\linkedone}[2]{\hypertarget{#1.#2}{}}
\newcommand{\linkedtwo}[3]{\hypertarget{#1.#2.#3}{}}
\newcommand{\linktutorial}[1]{\href{http://docs.python.org/tutorial/#1}{#1}}
\newcommand{\linktutorialone}[2]{\href{http://docs.python.org/tutorial/#1}{#2}}
\newcommand{\linklibrary}[1]{\href{http://docs.python.org/library/#1}{#1}}
\newcommand{\linklibraryone}[2]{\href{http://docs.python.org/library/#1}{#2}}
\newcommand{\pythonhp}{\href{http://www.python.org/}{\python website}}
\newcommand{\nzmathwiki}{\href{http://nzmath.sourceforge.net/wiki/}{{\nzmathzero}Wiki}}
\newcommand{\nzmathsf}{\href{http://sourceforge.net/projects/nzmath/}{\nzmath Project Page}}
\newcommand{\nzmathtnt}{\href{http://tnt.math.se.tmu.ac.jp/nzmath/}{\nzmath Project Official Page}}
% parameter name
\newcommand{\param}[1]{{\tt #1}}
% function macros
\newcommand{\hiki}[2]{{\tt #1}:\ {\em #2}}
\newcommand{\hikiopt}[3]{{\tt #1}:\ {\em #2}=#3}

\newdimen\hoge
\newdimen\truetextwidth
\newcommand{\func}[3]{%
\setbox0\hbox{#1(#2)}
\hoge=\wd0
\truetextwidth=\textwidth
\advance \truetextwidth by -2\oddsidemargin
\ifdim\hoge<\truetextwidth % short form
{\bf \colorbox{skyyellow}{#1(#2)\ $\to$ #3}}
%
\else % long form
\fcolorbox{skyyellow}{skyyellow}{%
   \begin{minipage}{\textwidth}%
   {\bf #1(#2)\\ %
    \qquad\quad   $\to$\ #3}%
   \end{minipage}%
   }%
\fi%
}

\newcommand{\out}[1]{{\em #1}}
\newcommand{\initialize}{%
  \paragraph{\large \colorbox{skyblue}{Initialize (Constructor)}}%
    \quad\\ %
    \vspace{3pt}\\
}
\newcommand{\method}{\C \paragraph{\large \colorbox{skyblue}{Methods}}}
% Attribute environment
\newenvironment{at}
{%begin
\paragraph{\large \colorbox{skyblue}{Attribute}}
\quad\\
\begin{description}
}%
{%end
\end{description}
}
% Operation environment
\newenvironment{op}
{%begin
\paragraph{\large \colorbox{skyblue}{Operations}}
\quad\\
\begin{table}[h]
\begin{center}
\begin{tabular}{|l|l|}
\hline
operator & explanation\\
\hline
}%
{%end
\hline
\end{tabular}
\end{center}
\end{table}
}
% Examples environment
\newenvironment{ex}%
{%begin
\paragraph{\large \colorbox{skyblue}{Examples}}
\VerbatimEnvironment
\renewcommand{\EveryVerbatim}{\fontencoding{OT1}\selectfont}
\begin{quote}
\begin{Verbatim}
}%
{%end
\end{Verbatim}
\end{quote}
}
%
\definecolor{skyblue}{cmyk}{0.2, 0, 0.1, 0}
\definecolor{skyyellow}{cmyk}{0.1, 0.1, 0.5, 0}
%
%\title{NZMATH User Manual\\ {\large{(for version 1.0)}}}
%\date{}
%\author{}
\begin{document}
%\maketitle
%
\setcounter{tocdepth}{3}
\setcounter{secnumdepth}{3}


\tableofcontents
\C

\chapter{Functions}

%---------- start document ---------- %
 \section{ecpp -- elliptic curve primality proving}\linkedzero{ecpp}
 The module consists of various functions for ECPP (Elliptic Curve Primality Proving).

 It is probable that the module will be refactored in the future so that each function be placed in other modules.\\
\spacing
The ecpp module requires mpmath.

%
  \subsection{ecpp -- elliptic curve primality proving}\linkedone{ecpp}{ecpp}
   \func{ecpp}
   {%
     \hiki{n}{integer},\ %
     \hikiopt{era}{list}{None}%
   }{%
     \out{bool}%
   }\\
   \spacing
   % document of basic document
   \quad Do elliptic curve primality proving.\\
   If \param{n} is prime, return True. Otherwise, return False. \\
   \spacing
   % added document
   The optional argument \param{era} is a list of primes (which stands for ERAtosthenes).\\
   \spacing
   % input, output document
   \quad \param{n} must be a big integer.\\
%
  \subsection{hilbert -- Hilbert class polynomial}\linkedone{ecpp}{hilbert}
   \func{hilbert}
        {\hiki{D}{integer}}
        {\out{(integer, list)}}\\
   \spacing
   % document of basic document
   \quad Return the class number and Hilbert class polynomial for the imaginary quadratic field with fundamental discriminant \param{D}.\\
   \spacing
   % added document
   \quad Note that this function returns Hilbert class polynomial as a list of coefficients.\\
   \negok If the option \linkingone{config}{HAVE\_NET} is set, at first try to retrieve the data in \url{http://hilbert-class-polynomial.appspot.com/}.
   If the data corresponding to \param{D} is not found, compute the Hilbert polynomial directly (for a long time).\\
   \spacing
   % input, output document
   \quad \param{D} must be negative int or long. See \cite{Pomerance}.\\
%
  \subsection{dedekind -- Dedekind's eta function}\linkedone{ecpp}{dedekind}
   \func{dedekind}
        {\hiki{tau}{mpmath.mpc}, \ 
         \hiki{floatpre}{integer}}
        {\out{mpmath.mpc}}\\
   \spacing
   % document of basic document
   \quad Return Dedekind's eta of a complex number \param{tau} in the upper half-plane. \\
   \spacing
   % added document
   \quad Additional argument \param{floatpre} specifies the precision of calculation in decimal digits.\\
   \spacing
   % input, output document
   \quad \param{floatpre} must be positive int. \\
%
  \subsection{cmm -- CM method}\linkedone{ecpp}{cmm}
   \func{cmm}
        {\hiki{p}{integer}}
        {\out{list}}\\
   \spacing
   % document of basic document
   \quad Return curve parameters for CM curves.\\
   \spacing
   % added document
   \quad If you also need its orders, use \linkingone{ecpp}{cmm\_order}.\\
   \spacing
   % input, output document
   \quad A prime \param{p} has to be odd.\\
   This function returns a list of (\param{a},\ \param{b}), where (\param{a},\ \param{b}) expresses Weierstrass' short form.
%
  \subsection{cmm\_order -- CM method with order}\linkedone{ecpp}{cmm\_order}
   \func{cmm\_order}
        {\hiki{p}{integer}}
        {\out{list}}\\
   \spacing
   % document of basic document
   \quad Return curve parameters for CM curves and its orders.\\
   \spacing
   % added document
   \quad If you need only curves, use \linkingone{ecpp}{cmm}.\\
   \spacing
   % input, output document
   \quad A prime \param{p} has to be odd.\\
   This function returns a list of (\param{a},\ \param{b},\ \param{order}), where (\param{a},\ \param{b}) expresses Weierstrass' short form and \param{order} is the order of the curve.
%
  \subsection{cornacchiamodify -- Modified cornacchia algorithm}\linkedone{ecpp}{cornacchiamodify}
   \func{cornacchiamodify}
        {\hiki{d}{integer},\ 
        \hiki{p}{integer}}
        {\out{list}}\\
   \spacing
   % document of basic document
   \quad Return the solution $(u,\ v)$ of $u^2 - \param{d}v^2 = 4\param{p}$.\\
   \spacing
   % added document
   \quad If there is no solution, raise ValueError.\\
   \spacing
   % input, output document
   \quad  \param{p} must be a prime integer and \param{d} be an integer such that $\param{d} < 0$ and $\param{d} > -4\param{p}$ with $\param{d} \equiv 0, 1 \pmod{4}$.\\
%
\begin{ex}
>>> ecpp.ecpp(300000000000000000053)
True
>>> ecpp.hilbert(-7)
(1, [3375, 1])
>>> ecpp.cmm(7)
[(6L, 3L), (5L, 4L)]
>>> ecpp.cornacchiamodify(-7, 29)
(2, 4)
\end{ex}%Don't indent!(indent causes an error.)
\C

%---------- end document ---------- %

\bibliographystyle{jplain}%use jbibtex
\bibliography{nzmath_references}

\end{document}

%\documentclass{report}

\documentclass{report}

%%%%%%%%%%%%%%%%%%%%%%%%%%%%%%%%%%%%%%%%%%%%%%%%%%%%%%%%%%%%%
%
% macros for nzmath manual
%
%%%%%%%%%%%%%%%%%%%%%%%%%%%%%%%%%%%%%%%%%%%%%%%%%%%%%%%%%%%%%
\usepackage{amssymb,amsmath}
\usepackage{color}
\usepackage[dvipdfm,bookmarks=true,bookmarksnumbered=true,%
 pdftitle={NZMATH Users Manual},%
 pdfsubject={Manual for NZMATH Users},%
 pdfauthor={NZMATH Development Group},%
 pdfkeywords={TeX; dvipdfmx; hyperref; color;},%
 colorlinks=true]{hyperref}
\usepackage{fancybox}
\usepackage[T1]{fontenc}
%
\newcommand{\DS}{\displaystyle}
\newcommand{\C}{\clearpage}
\newcommand{\NO}{\noindent}
\newcommand{\negok}{$\dagger$}
\newcommand{\spacing}{\vspace{1pt}\\ }
% software macros
\newcommand{\nzmathzero}{{\footnotesize $\mathbb{N}\mathbb{Z}$}\texttt{MATH}}
\newcommand{\nzmath}{{\nzmathzero}\ }
\newcommand{\pythonzero}{$\mbox{\texttt{Python}}$}
\newcommand{\python}{{\pythonzero}\ }
% link macros
\newcommand{\linkingzero}[1]{{\bf \hyperlink{#1}{#1}}}%module
\newcommand{\linkingone}[2]{{\bf \hyperlink{#1.#2}{#2}}}%module,class/function etc.
\newcommand{\linkingtwo}[3]{{\bf \hyperlink{#1.#2.#3}{#3}}}%module,class,method
\newcommand{\linkedzero}[1]{\hypertarget{#1}{}}
\newcommand{\linkedone}[2]{\hypertarget{#1.#2}{}}
\newcommand{\linkedtwo}[3]{\hypertarget{#1.#2.#3}{}}
\newcommand{\linktutorial}[1]{\href{http://docs.python.org/tutorial/#1}{#1}}
\newcommand{\linktutorialone}[2]{\href{http://docs.python.org/tutorial/#1}{#2}}
\newcommand{\linklibrary}[1]{\href{http://docs.python.org/library/#1}{#1}}
\newcommand{\linklibraryone}[2]{\href{http://docs.python.org/library/#1}{#2}}
\newcommand{\pythonhp}{\href{http://www.python.org/}{\python website}}
\newcommand{\nzmathwiki}{\href{http://nzmath.sourceforge.net/wiki/}{{\nzmathzero}Wiki}}
\newcommand{\nzmathsf}{\href{http://sourceforge.net/projects/nzmath/}{\nzmath Project Page}}
\newcommand{\nzmathtnt}{\href{http://tnt.math.se.tmu.ac.jp/nzmath/}{\nzmath Project Official Page}}
% parameter name
\newcommand{\param}[1]{{\tt #1}}
% function macros
\newcommand{\hiki}[2]{{\tt #1}:\ {\em #2}}
\newcommand{\hikiopt}[3]{{\tt #1}:\ {\em #2}=#3}

\newdimen\hoge
\newdimen\truetextwidth
\newcommand{\func}[3]{%
\setbox0\hbox{#1(#2)}
\hoge=\wd0
\truetextwidth=\textwidth
\advance \truetextwidth by -2\oddsidemargin
\ifdim\hoge<\truetextwidth % short form
{\bf \colorbox{skyyellow}{#1(#2)\ $\to$ #3}}
%
\else % long form
\fcolorbox{skyyellow}{skyyellow}{%
   \begin{minipage}{\textwidth}%
   {\bf #1(#2)\\ %
    \qquad\quad   $\to$\ #3}%
   \end{minipage}%
   }%
\fi%
}

\newcommand{\out}[1]{{\em #1}}
\newcommand{\initialize}{%
  \paragraph{\large \colorbox{skyblue}{Initialize (Constructor)}}%
    \quad\\ %
    \vspace{3pt}\\
}
\newcommand{\method}{\C \paragraph{\large \colorbox{skyblue}{Methods}}}
% Attribute environment
\newenvironment{at}
{%begin
\paragraph{\large \colorbox{skyblue}{Attribute}}
\quad\\
\begin{description}
}%
{%end
\end{description}
}
% Operation environment
\newenvironment{op}
{%begin
\paragraph{\large \colorbox{skyblue}{Operations}}
\quad\\
\begin{table}[h]
\begin{center}
\begin{tabular}{|l|l|}
\hline
operator & explanation\\
\hline
}%
{%end
\hline
\end{tabular}
\end{center}
\end{table}
}
% Examples environment
\newenvironment{ex}%
{%begin
\paragraph{\large \colorbox{skyblue}{Examples}}
\VerbatimEnvironment
\renewcommand{\EveryVerbatim}{\fontencoding{OT1}\selectfont}
\begin{quote}
\begin{Verbatim}
}%
{%end
\end{Verbatim}
\end{quote}
}
%
\definecolor{skyblue}{cmyk}{0.2, 0, 0.1, 0}
\definecolor{skyyellow}{cmyk}{0.1, 0.1, 0.5, 0}
%
%\title{NZMATH User Manual\\ {\large{(for version 1.0)}}}
%\date{}
%\author{}
\begin{document}
%\maketitle
%
\setcounter{tocdepth}{3}
\setcounter{secnumdepth}{3}


\tableofcontents
\C

\chapter{Functions}


%---------- start document ---------- %
 \section{equation -- solving equations, congruences }\linkedzero{equation}

 In the following descriptions, some type aliases are used.
 \begin{description}
   \item[poly\_list]\linkedone{equation}{poly\_list}:\\
     \param{poly\_list} is a list {\tt [a0, a1, \ldots, an]}
     representing a polynomial coefficients in ascending order, i.e.,
     meaning \(a_0 + a_1 X + \cdots + a_n X^n\).  The type of each
     {\tt ai} depends on each function (explained in their descriptions).

   \item[integer]\linkedone{equation}{integer}:\\
     \param{integer} is one of {\it int}, {\it long} or \linkingone{rational}{Integer}.
   \item[complex]\linkedone{equation}{complex}:\\
     \param{complex} includes all number types in the complex field:
     \linkingone{equation}{integer}, {\it float}, {\it complex} of \python,
     \linkingone{rational}{Rational} of \nzmath, etc.\\
 \end{description}
%
  \subsection{e1 -- solve equation with degree 1}\linkedone{equation}{e1}
   \func{e1}{\hiki{f}{\linkingone{equation}{poly\_list}}}{\out{\linkingone{equation}{complex}}}\\
   \spacing
   % document of basic document
   \quad Return the solution of linear equation $ax + b = 0$.\\
   \spacing
   % added document
   %\spacing
   % input, output document
   \quad \param{f} ought to be a \linkingone{equation}{poly\_list} {\tt [b, a]} of \linkingone{equation}{complex}.\\
%
  \subsection{e1\_ZnZ -- solve congruent equation modulo n with degree 1}\linkedone{equation}{e1\_ZnZ}
   \func{e1\_ZnZ}{\hiki{f}{\linkingone{equation}{poly\_list}},\ \hiki{n}{integer}}{\out{integer}}\\
   \spacing
   % document of basic document
   \quad Return the solution of $ax + b  \equiv 0 \pmod{\param{n}}$.\\
   \spacing
   % added document
   %\spacing
   % input, output document
   \quad \param{f} ought to be a \linkingone{equation}{poly\_list} {\tt [b, a]} of \linkingone{equation}{integer}.\\
%
  \subsection{e2 -- solve equation with degree 2}\linkedone{equation}{e2}
   \func{e2}{\hiki{f}{\linkingone{equation}{poly\_list}}}{\out{tuple}}\\
   \spacing
   % document of basic document
   \quad Return the solution of quadratic equation $ax^2 + bx + c = 0$.\\
   \spacing
   % added document
   %\spacing
   % input, output document
   \quad \param{f} ought to be a \linkingone{equation}{poly\_list} {\tt [c, b, a]} of \linkingone{equation}{complex}. \\
   The result tuple will contain exactly 2 roots, even in the case of
   double root.\\
%
  \subsection{e2\_Fp -- solve congruent equation modulo p with degree 2}\linkedone{equation}{e2\_Fp}
   \func{e2\_Fp}{\hiki{f}{\linkingone{equation}{poly\_list}},\ \hiki{p}{integer}}{\out{list}}\\
   \spacing
   % document of basic document
   \quad Return the solution of $ax^2 + bx + c  \equiv 0 \pmod{\param{p}}$.\\
   \spacing
   % added document
   \quad If the same values are returned, then the values are multiple roots. \\
   \spacing
   % input, output document
   \quad \param{f} ought to be a \linkingone{equation}{poly\_list} of 
   \linkingone{equation}{integer}s {\tt [c, b, a]}.
   In addition, \param{p} must be a prime \linkingone{equation}{integer}. \\
%
  \subsection{e3 -- solve equation with degree 3}\linkedone{equation}{e3}
   \func{e3}{\hiki{f}{\linkingone{equation}{poly\_list}}}{\out{list}}\\
   \spacing
   % document of basic document
   \quad Return the solution of cubic equation $ax^3 + bx^2 + cx + d = 0$.\\
   \spacing
   % added document
   %\spacing
   % input, output document
   \quad \param{f} ought to be a \linkingone{equation}{poly\_list} {\tt [d, c, b, a]} of \linkingone{equation}{complex}. \\
   The result tuple will contain exactly 3 roots, even in the case of including
   double roots.\\
%
  \subsection{e3\_Fp -- solve congruent equation modulo p with degree 3}\linkedone{equation}{e3\_Fp}
   \func{e3\_Fp}{\hiki{f}{\linkingone{equation}{poly\_list}},\ \hiki{p}{integer}}{\out{list}}\\
   \spacing
   % document of basic document
   \quad Return the solutions of $ax^3 + bx^2 + cx + d  \equiv 0 \pmod{\param{p}}$.\\
   \spacing
   % added document
   \quad If the same values are returned, then the values are multiple roots. \\
   \spacing
   % input, output document
   \quad \param{f} ought be a \linkingone{equation}{poly\_list} {\tt [d, c, b, a]} of \linkingone{equation}{integer}.
   In addition, \param{p} must be a prime \linkingone{equation}{integer}. \\
  \subsection{Newton -- solve equation using Newton's method}\linkedone{equation}{Newton}
   \func{Newton}{%
     \hiki{f}{\linkingone{equation}{poly\_list}},\ %
     \hikiopt{initial}{\linkingone{equation}{complex}}{1},\ %
     \hikiopt{repeat}{integer}{250}}{\out{complex}}\\
   \spacing
   % document of basic document
   \quad Return one of the approximated roots of $a_nx^n + \cdots + a_1x + a_0=0$.\\
   \spacing
   % added document
   \quad If you want to obtain all roots, then use \linkingone{equation}{SimMethod} instead.\\
   \negok If \param{initial} is a real number but there is no real roots, then this function returns meaningless values. \\
   \spacing
   % input, output document
   \quad \param{f} ought to be a \linkingone{equation}{poly\_list} of
   \linkingone{equation}{complex}.
   \param{initial} is an initial approximation \linkingone{equation}{complex} number.
   \param{repeat} is the number of steps to approximate a root.\\
%
  \subsection{SimMethod -- find all roots simultaneously}\linkedone{equation}{SimMethod}
   \func{SimMethod}{%
     \hiki{f}{\linkingone{equation}{poly\_list}},\ %
     \hikiopt{NewtonInitial}{\linkingone{equation}{complex}}{1},\ %
     \hikiopt{repeat}{integer}{250}}{\out{list}}\\
   \spacing
   % document of basic document
   \quad Return the approximated roots of $a_nx^n + \cdots + a_1x + a_0$.\\
   \spacing
   % added document
   \quad \negok If the equation has multiple root, maybe raise some error. \\
   \spacing
   % input, output document
   \quad \param{f} ought to be a \linkingone{equation}{poly\_list} of
   \linkingone{equation}{complex}.\\
   \param{NewtonInitial} and \param{repeat} will be passed to 
   \linkingone{equation}{Newton} to obtain the first approximations.\\
%
  \subsection{root\_Fp --  solve congruent equation modulo p}\linkedone{equation}{root\_Fp}
   \func{root\_Fp}{\hiki{f}{\linkingone{equation}{poly\_list}},\ \hiki{p}{integer}}{\out{integer}}\\
   \spacing
   % document of basic document
   \quad Return one of the roots of $a_nx^n + \cdots + a_1x + a_0 \equiv 0 \pmod{\param{p}}$. \\
   \spacing
   % added document
   \quad If you want to obtain all roots, then use \linkingone{equation}{allroots\_Fp}.\\
   \spacing
   % input, output document
   \quad \param{f} ought to be a \linkingone{equation}{poly\_list} of
   \linkingone{equation}{integer}.
   In addition, \param{p} must be a prime \linkingone{equation}{integer}. \\
   \quad If there is no root at all, then nothing will be returned.\\
%
  \subsection{allroots\_Fp -- solve congruent equation modulo p}\linkedone{equation}{allroots\_Fp}
   \func{allroots\_Fp}{\hiki{f}{\linkingone{equation}{poly\_list}},\ \hiki{p}{integer}}{\out{integer}}\\
   \spacing
   % document of basic document
   \quad Return all roots of $a_nx^n + \cdots + a_1x + a_0 \equiv 0 \pmod{\param{p}}$. \\
   \spacing
   % added document
   %\spacing
   % input, output document
   \quad \param{f} ought to be a \linkingone{equation}{poly\_list} of
   \linkingone{equation}{integer}.
   In addition, \param{p} must be a prime \linkingone{equation}{integer}. \\
   \quad If there is no root at all, then an empty list will be returned.\\
%
\begin{ex}
>>> equation.e1([1, 2])
-0.5
>>> equation.e1([1j, 2])
-0.5j
>>> equation.e1_ZnZ([3, 2], 5)
1
>>> equation.e2([-3, 1, 1])
(1.3027756377319946, -2.3027756377319948)
>>> equation.e2_Fp([-3, 1, 1], 13)
[6, 6]
>>> equation.e3([1, 1, 2, 1])
[(-0.12256116687665397-0.74486176661974479j), 
(-1.7548776662466921+1.8041124150158794e-16j), 
(-0.12256116687665375+0.74486176661974468j)]
>>> equation.e3_Fp([1, 1, 2, 1], 7)
[3]
>>> equation.Newton([-3, 2, 1, 1])
0.84373427789806899
>>> equation.Newton([-3, 2, 1, 1], 2)
0.84373427789806899
>>> equation.Newton([-3, 2, 1, 1], 2, 1000)
0.84373427789806899
>>> equation.SimMethod([-3, 2, 1, 1])
[(0.84373427789806887+0j), 
(-0.92186713894903438+1.6449263775999723j), 
(-0.92186713894903438-1.6449263775999723j)]
>>> equation.root_Fp([-3, 2, 1, 1], 7)
>>> equation.root_Fp([-3, 2, 1, 1], 11)
9L
>>> equation.allroots_Fp([-3, 2, 1, 1], 7)
[]
>>> equation.allroots_Fp([-3, 2, 1, 1], 11)
[9L]
>>> equation.allroots_Fp([-3, 2, 1, 1], 13)
[3L, 7L, 2L]
\end{ex}%Don't indent!(indent causes an error.)
\C

%---------- end document ---------- %

\bibliographystyle{jplain}%use jbibtex
\bibliography{nzmath_references}

\end{document}


%\documentclass{report}

\documentclass{report}

%%%%%%%%%%%%%%%%%%%%%%%%%%%%%%%%%%%%%%%%%%%%%%%%%%%%%%%%%%%%%
%
% macros for nzmath manual
%
%%%%%%%%%%%%%%%%%%%%%%%%%%%%%%%%%%%%%%%%%%%%%%%%%%%%%%%%%%%%%
\usepackage{amssymb,amsmath}
\usepackage{color}
\usepackage[dvipdfm,bookmarks=true,bookmarksnumbered=true,%
 pdftitle={NZMATH Users Manual},%
 pdfsubject={Manual for NZMATH Users},%
 pdfauthor={NZMATH Development Group},%
 pdfkeywords={TeX; dvipdfmx; hyperref; color;},%
 colorlinks=true]{hyperref}
\usepackage{fancybox}
\usepackage[T1]{fontenc}
%
\newcommand{\DS}{\displaystyle}
\newcommand{\C}{\clearpage}
\newcommand{\NO}{\noindent}
\newcommand{\negok}{$\dagger$}
\newcommand{\spacing}{\vspace{1pt}\\ }
% software macros
\newcommand{\nzmathzero}{{\footnotesize $\mathbb{N}\mathbb{Z}$}\texttt{MATH}}
\newcommand{\nzmath}{{\nzmathzero}\ }
\newcommand{\pythonzero}{$\mbox{\texttt{Python}}$}
\newcommand{\python}{{\pythonzero}\ }
% link macros
\newcommand{\linkingzero}[1]{{\bf \hyperlink{#1}{#1}}}%module
\newcommand{\linkingone}[2]{{\bf \hyperlink{#1.#2}{#2}}}%module,class/function etc.
\newcommand{\linkingtwo}[3]{{\bf \hyperlink{#1.#2.#3}{#3}}}%module,class,method
\newcommand{\linkedzero}[1]{\hypertarget{#1}{}}
\newcommand{\linkedone}[2]{\hypertarget{#1.#2}{}}
\newcommand{\linkedtwo}[3]{\hypertarget{#1.#2.#3}{}}
\newcommand{\linktutorial}[1]{\href{http://docs.python.org/tutorial/#1}{#1}}
\newcommand{\linktutorialone}[2]{\href{http://docs.python.org/tutorial/#1}{#2}}
\newcommand{\linklibrary}[1]{\href{http://docs.python.org/library/#1}{#1}}
\newcommand{\linklibraryone}[2]{\href{http://docs.python.org/library/#1}{#2}}
\newcommand{\pythonhp}{\href{http://www.python.org/}{\python website}}
\newcommand{\nzmathwiki}{\href{http://nzmath.sourceforge.net/wiki/}{{\nzmathzero}Wiki}}
\newcommand{\nzmathsf}{\href{http://sourceforge.net/projects/nzmath/}{\nzmath Project Page}}
\newcommand{\nzmathtnt}{\href{http://tnt.math.se.tmu.ac.jp/nzmath/}{\nzmath Project Official Page}}
% parameter name
\newcommand{\param}[1]{{\tt #1}}
% function macros
\newcommand{\hiki}[2]{{\tt #1}:\ {\em #2}}
\newcommand{\hikiopt}[3]{{\tt #1}:\ {\em #2}=#3}

\newdimen\hoge
\newdimen\truetextwidth
\newcommand{\func}[3]{%
\setbox0\hbox{#1(#2)}
\hoge=\wd0
\truetextwidth=\textwidth
\advance \truetextwidth by -2\oddsidemargin
\ifdim\hoge<\truetextwidth % short form
{\bf \colorbox{skyyellow}{#1(#2)\ $\to$ #3}}
%
\else % long form
\fcolorbox{skyyellow}{skyyellow}{%
   \begin{minipage}{\textwidth}%
   {\bf #1(#2)\\ %
    \qquad\quad   $\to$\ #3}%
   \end{minipage}%
   }%
\fi%
}

\newcommand{\out}[1]{{\em #1}}
\newcommand{\initialize}{%
  \paragraph{\large \colorbox{skyblue}{Initialize (Constructor)}}%
    \quad\\ %
    \vspace{3pt}\\
}
\newcommand{\method}{\C \paragraph{\large \colorbox{skyblue}{Methods}}}
% Attribute environment
\newenvironment{at}
{%begin
\paragraph{\large \colorbox{skyblue}{Attribute}}
\quad\\
\begin{description}
}%
{%end
\end{description}
}
% Operation environment
\newenvironment{op}
{%begin
\paragraph{\large \colorbox{skyblue}{Operations}}
\quad\\
\begin{table}[h]
\begin{center}
\begin{tabular}{|l|l|}
\hline
operator & explanation\\
\hline
}%
{%end
\hline
\end{tabular}
\end{center}
\end{table}
}
% Examples environment
\newenvironment{ex}%
{%begin
\paragraph{\large \colorbox{skyblue}{Examples}}
\VerbatimEnvironment
\renewcommand{\EveryVerbatim}{\fontencoding{OT1}\selectfont}
\begin{quote}
\begin{Verbatim}
}%
{%end
\end{Verbatim}
\end{quote}
}
%
\definecolor{skyblue}{cmyk}{0.2, 0, 0.1, 0}
\definecolor{skyyellow}{cmyk}{0.1, 0.1, 0.5, 0}
%
%\title{NZMATH User Manual\\ {\large{(for version 1.0)}}}
%\date{}
%\author{}
\begin{document}
%\maketitle
%
\setcounter{tocdepth}{3}
\setcounter{secnumdepth}{3}


\tableofcontents
\C

\chapter{Functions}


%---------- start document ---------- %
 \section{gcd -- gcd algorithm}\linkedzero{gcd}
%
  \subsection{gcd -- the greatest common divisor}\linkedone{gcd}{gcd}
   \func{gcd}{\hiki{a}{integer},\ \hiki{b}{integer}}{\out{integer}}\\
   \spacing
   % document of basic document
   \quad Return the greatest common divisor of two integers \param{a} and \param{b}.\\
   \spacing
   % added document
   %\spacing
   % input, output document
   \quad \param{a},\ \param{b} must be int, long or \linkingone{rational}{Integer}.
   Even if one of the arguments is negative, the result is non-negative.\\
%
  \subsection{binarygcd -- binary gcd algorithm}\linkedone{gcd}{binarygcd}
   \func{binarygcd}{\hiki{a}{integer},\ \hiki{b}{integer}}{\out{integer}}\\
   \spacing
   % document of basic document
   \quad Return the greatest common divisor of two integers \param{a} and \param{b} by binary gcd algorithm.\\
   \spacing
   % added document
   \quad \negok This function is an alias of \linkingone{arygcd}{binarygcd}\\
   \spacing
   % input, output document
   \quad \param{a},\ \param{b} must be int, long, or \linkingone{rational}{Integer}.\\
%
  \subsection{extgcd -- extended gcd algorithm}\linkedone{gcd}{extgcd}
   \func{extgcd}{\hiki{a}{integer},\ \hiki{b}{integer}}{(\out{integer},\ \out{integer},\ \out{integer})}\\
   \spacing
   % document of basic document
   \quad Return the greatest common divisor $d$ of two integers \param{a} and \param{b} and $u,\ v$ such that $d = \param{a}u + \param{b}v$.\\
   \spacing
   % added document
   %\spacing
   % input, output document
   \quad \param{a},\ \param{b} must be int, long, or \linkingone{rational}{Integer}.\\
   The returned value is a tuple (\param{u},\ \param{v},\ \param{d}).\\
%
  \subsection{lcm -- the least common multiple}\linkedone{gcd}{lcm}
   \func{lcm}{\hiki{a}{integer},\ \hiki{b}{integer}}{\out{integer}}\\
   \spacing
   % document of basic document
   \quad Return the least common multiple of two integers \param{a} and \param{b}.\\
   \spacing
   % added document
   \negok If both \param{a} and \param{b} are zero, then it raises an exception.
   \spacing
   % input, output document
   \quad \param{a},\ \param{b} must be int, long, or \linkingone{rational}{Integer}.\\
%
  \subsection{gcd\_of\_list -- gcd of many integers}\linkedone{gcd}{gcd\_of\_list}
   \func{gcd\_of\_list}{\hiki{integers}{list}}{\out{list}}\\
   \spacing
   % document of basic document
   \quad Return gcd of multiple integers.\\
   \spacing
   % added document
   \quad For given \param{integers} $[x_1,\ldots,x_n]$, return a list $[d,\ [c_1,\ldots,c_n]]$ such that $d=c_1 x_1+\cdots+c_n x_n$, where $d$ is the greatest common divisor of $x_1,\ldots, x_n$.\\
   \spacing
   % input, output document
   \quad \param{integers} is a list which elements are int or long\\
   This function returns $[d,\ [c_1,\ldots,c_n]]$, where $d,\ c_i$ are an integer.\\
%
  \subsection{coprime -- coprime check}\linkedone{gcd}{coprime}
   \func{coprime}{\hiki{a}{integer},\ \hiki{b}{integer}}{\out{bool}}\\
   \spacing
   % document of basic document
   \quad Return True if \param{a} and \param{b} are coprime, False otherwise.\\
   \spacing
   % added document
   %\spacing
   % input, output document
   \quad \param{a},\ \param{b} are int, long, or \linkingone{rational}{Integer}.\\
%
  \subsection{pairwise\_coprime -- coprime check of many integers}\linkedone{gcd}{pairwise\_coprime}
   \func{pairwise\_coprime}{\hiki{integers}{list}}{\out{bool}}\\
   \spacing
   % document of basic document
   \quad Return True if all integers in \param{integers} are pairwise coprime, False otherwise.\\
   \spacing
   % added document
   %\spacing
   % input, output document
   \quad \param{integers} is a list which elements are int, long, or \linkingone{rational}{Integer}.\\
%
\begin{ex}
>>> gcd.gcd(12, 18)
6
>>> gcd.gcd(12, -18)
6
>>> gcd.gcd(-12, -18)
6
>>> gcd.extgcd(12, -18)
(-1, -1, 6)
>>> gcd.extgcd(-12, -18)
(1, -1, 6)
>>> gcd.extgcd(0, -18)
(0, -1, 18)
>>> gcd.lcm(12, 18)
36
>>> gcd.lcm(12, -18)
-36
>>> gcd.gcd_of_list([60, 90, 210])
[30, [-1, 1, 0]]
\end{ex}%Don't indent!(indent causes an error.)
\C

%---------- end document ---------- %

\bibliographystyle{jplain}%use jbibtex
\bibliography{nzmath_references}

\end{document}


%\documentclass{report}

\documentclass{report}

%%%%%%%%%%%%%%%%%%%%%%%%%%%%%%%%%%%%%%%%%%%%%%%%%%%%%%%%%%%%%
%
% macros for nzmath manual
%
%%%%%%%%%%%%%%%%%%%%%%%%%%%%%%%%%%%%%%%%%%%%%%%%%%%%%%%%%%%%%
\usepackage{amssymb,amsmath}
\usepackage{color}
\usepackage[dvipdfm,bookmarks=true,bookmarksnumbered=true,%
 pdftitle={NZMATH Users Manual},%
 pdfsubject={Manual for NZMATH Users},%
 pdfauthor={NZMATH Development Group},%
 pdfkeywords={TeX; dvipdfmx; hyperref; color;},%
 colorlinks=true]{hyperref}
\usepackage{fancybox}
\usepackage[T1]{fontenc}
%
\newcommand{\DS}{\displaystyle}
\newcommand{\C}{\clearpage}
\newcommand{\NO}{\noindent}
\newcommand{\negok}{$\dagger$}
\newcommand{\spacing}{\vspace{1pt}\\ }
% software macros
\newcommand{\nzmathzero}{{\footnotesize $\mathbb{N}\mathbb{Z}$}\texttt{MATH}}
\newcommand{\nzmath}{{\nzmathzero}\ }
\newcommand{\pythonzero}{$\mbox{\texttt{Python}}$}
\newcommand{\python}{{\pythonzero}\ }
% link macros
\newcommand{\linkingzero}[1]{{\bf \hyperlink{#1}{#1}}}%module
\newcommand{\linkingone}[2]{{\bf \hyperlink{#1.#2}{#2}}}%module,class/function etc.
\newcommand{\linkingtwo}[3]{{\bf \hyperlink{#1.#2.#3}{#3}}}%module,class,method
\newcommand{\linkedzero}[1]{\hypertarget{#1}{}}
\newcommand{\linkedone}[2]{\hypertarget{#1.#2}{}}
\newcommand{\linkedtwo}[3]{\hypertarget{#1.#2.#3}{}}
\newcommand{\linktutorial}[1]{\href{http://docs.python.org/tutorial/#1}{#1}}
\newcommand{\linktutorialone}[2]{\href{http://docs.python.org/tutorial/#1}{#2}}
\newcommand{\linklibrary}[1]{\href{http://docs.python.org/library/#1}{#1}}
\newcommand{\linklibraryone}[2]{\href{http://docs.python.org/library/#1}{#2}}
\newcommand{\pythonhp}{\href{http://www.python.org/}{\python website}}
\newcommand{\nzmathwiki}{\href{http://nzmath.sourceforge.net/wiki/}{{\nzmathzero}Wiki}}
\newcommand{\nzmathsf}{\href{http://sourceforge.net/projects/nzmath/}{\nzmath Project Page}}
\newcommand{\nzmathtnt}{\href{http://tnt.math.se.tmu.ac.jp/nzmath/}{\nzmath Project Official Page}}
% parameter name
\newcommand{\param}[1]{{\tt #1}}
% function macros
\newcommand{\hiki}[2]{{\tt #1}:\ {\em #2}}
\newcommand{\hikiopt}[3]{{\tt #1}:\ {\em #2}=#3}

\newdimen\hoge
\newdimen\truetextwidth
\newcommand{\func}[3]{%
\setbox0\hbox{#1(#2)}
\hoge=\wd0
\truetextwidth=\textwidth
\advance \truetextwidth by -2\oddsidemargin
\ifdim\hoge<\truetextwidth % short form
{\bf \colorbox{skyyellow}{#1(#2)\ $\to$ #3}}
%
\else % long form
\fcolorbox{skyyellow}{skyyellow}{%
   \begin{minipage}{\textwidth}%
   {\bf #1(#2)\\ %
    \qquad\quad   $\to$\ #3}%
   \end{minipage}%
   }%
\fi%
}

\newcommand{\out}[1]{{\em #1}}
\newcommand{\initialize}{%
  \paragraph{\large \colorbox{skyblue}{Initialize (Constructor)}}%
    \quad\\ %
    \vspace{3pt}\\
}
\newcommand{\method}{\C \paragraph{\large \colorbox{skyblue}{Methods}}}
% Attribute environment
\newenvironment{at}
{%begin
\paragraph{\large \colorbox{skyblue}{Attribute}}
\quad\\
\begin{description}
}%
{%end
\end{description}
}
% Operation environment
\newenvironment{op}
{%begin
\paragraph{\large \colorbox{skyblue}{Operations}}
\quad\\
\begin{table}[h]
\begin{center}
\begin{tabular}{|l|l|}
\hline
operator & explanation\\
\hline
}%
{%end
\hline
\end{tabular}
\end{center}
\end{table}
}
% Examples environment
\newenvironment{ex}%
{%begin
\paragraph{\large \colorbox{skyblue}{Examples}}
\VerbatimEnvironment
\renewcommand{\EveryVerbatim}{\fontencoding{OT1}\selectfont}
\begin{quote}
\begin{Verbatim}
}%
{%end
\end{Verbatim}
\end{quote}
}
%
\definecolor{skyblue}{cmyk}{0.2, 0, 0.1, 0}
\definecolor{skyyellow}{cmyk}{0.1, 0.1, 0.5, 0}
%
%\title{NZMATH User Manual\\ {\large{(for version 1.0)}}}
%\date{}
%\author{}
\begin{document}
%\maketitle
%
\setcounter{tocdepth}{3}
\setcounter{secnumdepth}{3}


\tableofcontents
\C

\chapter{Functions}


%---------- start document ---------- %
 \section{multiplicative -- multiplicative number theoretic functions}\linkedzero{multiplicative}
%
All functions of this module accept only positive integers,
unless otherwise noted.
%
  \subsection{euler -- the Euler totient function}\linkedone{multiplicative}{euler}
   \func{euler}
   {%
     \hiki{n}{integer}
   }{%
     \out{integer}
   }\\
   \spacing
   % document of basic document
   \quad Return the number of numbers relatively prime to \param{n}
   and smaller than \param{n}.  In the literature, the function is
   referred often as \(\varphi\).

  \subsection{moebius -- the M\"obius function}\linkedone{multiplicative}{moebius}
   \func{moebius}
   {%
     \hiki{n}{integer}
   }{%
     \out{integer}
   }\\
   \spacing
   % document of basic document
   \quad Return:
   \begin{description}
   \item[-1] if n has odd distinct prime factors,
   \item[1] if n has even distinct prime factors, or
   \item[0] if n has a squared prime factor.
   \end{description}
   In the literature, the function is referred often as \(\mu\).

  \subsection{sigma -- sum of divisor powers)}\linkedone{multiplicative}{sigma}
   \func{sigma}
   {%
     \hiki{m}{integer},\ %
     \hiki{n}{integer}
   }{%
     \out{integer}
   }\\
   \spacing
   % document of basic document
   Return the sum of \param{m}-th powers of the factors of \param{n}.
   The argument \param{m} can be zero, then return the number of factors.
   In the literature, the function is referred often as \(\sigma\).
%
\begin{ex}
>>> multiplicative.euler(1)
1
>>> multiplicative.euler(2)
1
>>> multiplicative.euler(4)
2
>>> multiplicative.euler(5)
4
>>> multiplicative.moebius(1)
1
>>> multiplicative.moebius(2)
-1
>>> multiplicative.moebius(4)
0
>>> multiplicative.moebius(6)
1
>>> multiplicative.sigma(0, 1)
1
>>> multiplicative.sigma(1, 1)
1
>>> multiplicative.sigma(0, 2)
2
>>> multiplicative.sigma(1, 3)
4
>>> multiplicative.sigma(1, 4)
7
>>> multiplicative.sigma(1, 6)
12L
>>> multiplicative.sigma(2, 7)
50
\end{ex}%Don't indent!(indent causes an error.)
\C

%---------- end document ---------- %

\bibliographystyle{jplain}%use jbibtex
\bibliography{nzmath_references}

\end{document}


%\documentclass{report}

%%%%%%%%%%%%%%%%%%%%%%%%%%%%%%%%%%%%%%%%%%%%%%%%%%%%%%%%%%%%%
%
% macros for nzmath manual
%
%%%%%%%%%%%%%%%%%%%%%%%%%%%%%%%%%%%%%%%%%%%%%%%%%%%%%%%%%%%%%
\usepackage{amssymb,amsmath}
\usepackage{color}
\usepackage[dvipdfm,bookmarks=true,bookmarksnumbered=true,%
 pdftitle={NZMATH Users Manual},%
 pdfsubject={Manual for NZMATH Users},%
 pdfauthor={NZMATH Development Group},%
 pdfkeywords={TeX; dvipdfmx; hyperref; color;},%
 colorlinks=true]{hyperref}
\usepackage{fancybox}
\usepackage[T1]{fontenc}
%
\newcommand{\DS}{\displaystyle}
\newcommand{\C}{\clearpage}
\newcommand{\NO}{\noindent}
\newcommand{\negok}{$\dagger$}
\newcommand{\spacing}{\vspace{1pt}\\ }
% software macros
\newcommand{\nzmathzero}{{\footnotesize $\mathbb{N}\mathbb{Z}$}\texttt{MATH}}
\newcommand{\nzmath}{{\nzmathzero}\ }
\newcommand{\pythonzero}{$\mbox{\texttt{Python}}$}
\newcommand{\python}{{\pythonzero}\ }
% link macros
\newcommand{\linkingzero}[1]{{\bf \hyperlink{#1}{#1}}}%module
\newcommand{\linkingone}[2]{{\bf \hyperlink{#1.#2}{#2}}}%module,class/function etc.
\newcommand{\linkingtwo}[3]{{\bf \hyperlink{#1.#2.#3}{#3}}}%module,class,method
\newcommand{\linkedzero}[1]{\hypertarget{#1}{}}
\newcommand{\linkedone}[2]{\hypertarget{#1.#2}{}}
\newcommand{\linkedtwo}[3]{\hypertarget{#1.#2.#3}{}}
\newcommand{\linktutorial}[1]{\href{http://docs.python.org/tutorial/#1}{#1}}
\newcommand{\linktutorialone}[2]{\href{http://docs.python.org/tutorial/#1}{#2}}
\newcommand{\linklibrary}[1]{\href{http://docs.python.org/library/#1}{#1}}
\newcommand{\linklibraryone}[2]{\href{http://docs.python.org/library/#1}{#2}}
\newcommand{\pythonhp}{\href{http://www.python.org/}{\python website}}
\newcommand{\nzmathwiki}{\href{http://nzmath.sourceforge.net/wiki/}{{\nzmathzero}Wiki}}
\newcommand{\nzmathsf}{\href{http://sourceforge.net/projects/nzmath/}{\nzmath Project Page}}
\newcommand{\nzmathtnt}{\href{http://tnt.math.se.tmu.ac.jp/nzmath/}{\nzmath Project Official Page}}
% parameter name
\newcommand{\param}[1]{{\tt #1}}
% function macros
\newcommand{\hiki}[2]{{\tt #1}:\ {\em #2}}
\newcommand{\hikiopt}[3]{{\tt #1}:\ {\em #2}=#3}

\newdimen\hoge
\newdimen\truetextwidth
\newcommand{\func}[3]{%
\setbox0\hbox{#1(#2)}
\hoge=\wd0
\truetextwidth=\textwidth
\advance \truetextwidth by -2\oddsidemargin
\ifdim\hoge<\truetextwidth % short form
{\bf \colorbox{skyyellow}{#1(#2)\ $\to$ #3}}
%
\else % long form
\fcolorbox{skyyellow}{skyyellow}{%
   \begin{minipage}{\textwidth}%
   {\bf #1(#2)\\ %
    \qquad\quad   $\to$\ #3}%
   \end{minipage}%
   }%
\fi%
}

\newcommand{\out}[1]{{\em #1}}
\newcommand{\initialize}{%
  \paragraph{\large \colorbox{skyblue}{Initialize (Constructor)}}%
    \quad\\ %
    \vspace{3pt}\\
}
\newcommand{\method}{\C \paragraph{\large \colorbox{skyblue}{Methods}}}
% Attribute environment
\newenvironment{at}
{%begin
\paragraph{\large \colorbox{skyblue}{Attribute}}
\quad\\
\begin{description}
}%
{%end
\end{description}
}
% Operation environment
\newenvironment{op}
{%begin
\paragraph{\large \colorbox{skyblue}{Operations}}
\quad\\
\begin{table}[h]
\begin{center}
\begin{tabular}{|l|l|}
\hline
operator & explanation\\
\hline
}%
{%end
\hline
\end{tabular}
\end{center}
\end{table}
}
% Examples environment
\newenvironment{ex}%
{%begin
\paragraph{\large \colorbox{skyblue}{Examples}}
\VerbatimEnvironment
\renewcommand{\EveryVerbatim}{\fontencoding{OT1}\selectfont}
\begin{quote}
\begin{Verbatim}
}%
{%end
\end{Verbatim}
\end{quote}
}
%
\definecolor{skyblue}{cmyk}{0.2, 0, 0.1, 0}
\definecolor{skyyellow}{cmyk}{0.1, 0.1, 0.5, 0}
%
%\title{NZMATH User Manual\\ {\large{(for version 1.0)}}}
%\date{}
%\author{}
\begin{document}
%\maketitle
%
\setcounter{tocdepth}{3}
\setcounter{secnumdepth}{3}


\tableofcontents
\C

\chapter{Functions}


%---------- start document ---------- %
 \section{prime -- primality test , prime generation}\linkedzero{prime}
%
  \subsection{trialDivision -- trial division test}\linkedone{prime}{trialDivision}
   \func{trialDivision}
   {\hiki{n}{integer},\ \hikiopt{bound}{integer/float}{0}}{\out{True/False}}\\
   \spacing
   % document of basic document
   \quad Trial division primality test for an odd natural number.\\
   \spacing
   % added document
   %\spacing
   % input, output document
   \quad \param{bound} is a search bound of primes. 
   If it returns \(1\) under the condition that \param{bound} is given and 
   less than the square root of \param{n}, 
   it only means there is no prime factor less than \param{bound}.
%
  \subsection{spsp -- strong pseudo-prime test}\linkedone{prime}{spsp}
   \func{spsp}{\hiki{n}{integer},\ \hiki{base}{integer},\ \hikiopt{s}{integer}{None},\ \hikiopt{t}{integer}{None}}{\out{True/False}}\\
   \spacing
   % document of basic document
   \quad Strong Pseudo-Prime test on base \param{base}.\\
   \spacing
   % added document
   %\quad 
   %\spacing
   % input, output document
   \quad \param{s} and \param{t} are the numbers such that $n-1 = 2^\param{s}\param{t}$ and \param{t} is odd.
%
 \subsection{smallSpsp -- strong pseudo-prime test for small number}\linkedone{prime}{smallSpsp}
   \func{smallSpsp}{\hiki{n}{integer},\ \hikiopt{s}{integer}{None},\ \hikiopt{t}{integer}{None}}{\out{True/False}}\\
   \spacing
   % document of basic document
   \quad Strong Pseudo-Prime test for integer \param{n} less than $10^{12}$.\\
   \spacing
   % added document
   \quad $4$ spsp tests are sufficient to determine whether an integer less than $10^{12}$ is prime or not.
   \spacing
   % input, output document
   \quad \param{s} and \param{t} are the numbers such that $n-1 = 2^\param{s}\param{t}$ and \param{t} is odd.
%
  \subsection{miller -- Miller's primality test}\linkedone{prime}{miller}
   \func{miller}
   {\hiki{n}{integer}}{\out{True/False}}\\
   \spacing
   % document of basic document
   \quad Miller's primality test.\\
   \spacing
   % added document
   \quad This test is valid under GRH. See \linkingzero{config}.
   \spacing
   % input, output document
   %\quad 
%
  \subsection{millerRabin -- Miller-Rabin primality test}\linkedone{prime}{millerRabin}
   \func{millerRabin}
   {\hiki{n}{integer},\ \hikiopt{times}{integer}{20}}{\out{True/False}}\\
   \spacing
   % document of basic document
   \quad Miller's primality test.\\
   \spacing
   % added document
   \quad The difference from \linkingone{prime}{miller} is that 
   the Miller-Rabin method uses fast but probabilistic algorithm.
   On the other hand, \linkingone{prime}{miller} employs deterministic
   algorithm valid under GRH.
   \spacing
   % input, output document
   \quad \param{times} (default to $20$) is the number of repetition.
   The error probability is at most $4^{-\param{times}}$.
%
 \subsection{lpsp -- Lucas test}\linkedone{prime}{lpsp}
   \func{lpsp}
   {\hiki{n}{integer},\ \hiki{a}{integer},\ \hiki{b}{integer}}{\out{True/False}}\\
   \spacing
   % document of basic document
   \quad Lucas Pseudo-Prime test.\\
   \spacing
   % added document
   \quad Return True if \param{n} is a Lucas pseudo-prime of parameters \param{a}, \param{b},
    i.e. with respect to $x^2-\param{a}x+\param{b}$.
   \spacing
   % input, output document
   %\quad 
%
 \subsection{fpsp -- Frobenius test}\linkedone{prime}{fpsp}
   \func{fpsp}
   {\hiki{n}{integer},\ \hiki{a}{integer},\ \hiki{b}{integer}}{\out{True/False}}\\
   \spacing
   % document of basic document
   \quad Frobenius Pseudo-Prime test.\\
   \spacing
   % added document
   \quad Return True if \param{n} is a Frobenius pseudo-prime of parameters \param{a}, \param{b},
    i.e. with respect to $x^2-\param{a}x+\param{b}$.
   \spacing
   % input, output document
   %\quad 
%
 \subsection{by\_primitive\_root -- Lehmer's test}\linkedone{prime}{by\_primitive\_root}
   \func{by\_primitive\_root}
   {\hiki{n}{integer},\ \hiki{divisors}{sequence}}{\out{True/False}}\\
   \spacing
   % document of basic document
   \quad Lehmer's primality test~\cite{Lehmer1927}.\\
   \spacing
   % added document
   \quad Return True iff \param{n} is prime.\\
    The method proves the primality of \param{n} by existence of a primitive
    root.
   \spacing
   % input, output document
   \quad \param{divisors} is a sequence (list, tuple, etc.) of prime divisors
   of $n - 1$.
   %\quad 
%
 \subsection{full\_euler -- Brillhart \& Selfridge's test}\linkedone{prime}{full\_euler}
   \func{full\_euler}
   {\hiki{n}{integer},\ \hiki{divisors}{sequence}}{\out{True/False}}\\
   \spacing
   % document of basic document
   \quad Brillhart \& Selfridge's primality test~\cite{BS1967}.\\
   \spacing
   % added document
   \quad Return True iff \param{n} is prime.\\
    The method proves the primality of \param{n} by the equality
    $\varphi(n) = n - 1$, where $\varphi$ denotes the Euler totient
    (see \linkingone{multiplicative}{euler}).
    It requires a sequence of all prime divisors of $n - 1$.
   \spacing
   % input, output document
   \quad \param{divisors} is a sequence (list, tuple, etc.) of prime divisors
   of $n - 1$.
   %\quad 
   \quad
%
 \subsection{apr -- Jacobi sum test}\linkedone{prime}{apr}
   \func{apr}
   {\hiki{n}{integer}}{\out{True/False}}\\
   \spacing
   % document of basic document
   \quad APR (Adleman-Pomerance-Rumery) primality test or the Jacobi sum test.\\
   \spacing
   % added document
   %\quad 
   %\spacing
   % input, output document
   \quad Assuming \param{n} has no prime factors less than $32$.
    Assuming \param{n} is spsp (strong pseudo-prime) for several bases.
%
 \subsection{primeq -- primality test automatically}\linkedone{prime}{primeq}
   \func{primeq}
   {\hiki{n}{integer}}{\out{True/False}}\\
   \spacing
   % document of basic document
   \quad A convenient function for primality test.\\
   \spacing
   % added document
   \quad It uses one of \linkingone{prime}{trialDivision}, \linkingone{prime}{smallSpsp} or \linkingone{prime}{apr} depending on the size of \param{n}.
   \spacing
   % input, output document
   %\quad 
%
%\subsection{bigprimeq -- primality test automatically}\linkedone{prime}{bigprimeq}
%   \func{bigprimeq}
%   {\hiki{z}{integer}}{\out{True/False}}\\
%   \spacing
   % document of basic document
%   \quad Giving up rigorous proof of primality, return True for a probable prime.
%   \spacing
   % added document
%   \quad 
%   \spacing
   % input, output document
   %\quad 
%
 \subsection{prime -- $n$-th prime number}\linkedone{prime}{prime}
   \func{prime}
   {\hiki{n}{integer}}{\out{integer}}\\
   \spacing
   % document of basic document
   \quad Return the \param{n}-th prime number.\\
   \spacing
   % added document
   %\quad 
   %\spacing
   % input, output document
   %\quad 
%
 \subsection{nextPrime -- generate next prime}\linkedone{prime}{nextPrime}
   \func{nextPrime}
   {\hiki{n}{integer}}{\out{integer}}\\
   \spacing
   % document of basic document
   \quad Return the smallest prime bigger than the given integer \param{n}.
   \spacing
   % added document
   %\quad 
   %\spacing
   % input, output document
   %\quad 
%
 \subsection{randPrime -- generate random prime}\linkedone{prime}{randPrime}
   \func{randPrime}
   {\hiki{n}{integer}}{\out{integer}}\\
   \spacing
   % document of basic document
   \quad Return a random \param{n}-digits prime.\\
   \spacing
   % added document
   %\quad 
   %\spacing
   % input, output document
   %\quad 
%
 \subsection{generator -- generate primes}\linkedone{prime}{generator}
   \func{generator}
   {(None)}{\out{generator}}\\
   \spacing
   % document of basic document
   \quad Generate primes from $2$ to $\infty$ (as generator).\\
   \spacing
   % added document
   %\quad 
   %\spacing
   % input, output document
   %\quad 
%
 \subsection{generator\_eratosthenes -- generate primes using Eratosthenes sieve}\linkedone{prime}{generator\_eratosthenes}
   \func{generator\_eratosthenes}
   {\hiki{n}{integer}}{\out{generator}}\\
   \spacing
   % document of basic document
   \quad Generate primes up to \param{n} using Eratosthenes sieve.\\
   \spacing
   % added document
   %\quad 
   %\spacing
   % input, output document
   %\quad 
%
 \subsection{primonial -- product of primes}\linkedone{prime}{primonial}
   \func{primonial}
   {\hiki{p}{integer}}{\out{integer}}\\
   \spacing
   % document of basic document
   \quad Return the product
   \begin{equation*}
   \prod_{q \in \mathbb{P}_{\le \param{p}}} q=2\cdot 3\cdot 5\cdots \param{p}\ .
   \end{equation*}
   \spacing
   % added document
   %\quad 
   %\spacing
   % input, output document
   %\quad 
%
 \subsection{properDivisors -- proper divisors}\linkedone{prime}{properDivisors}
   \func{properDivisors}
   {\hiki{n}{integer}}{\out{list}}\\
   \spacing
   % document of basic document
   \quad Return proper divisors of \param{n} (all divisors of \param{n} excluding $1$ and \param{n}).\\
   \spacing
   % added document
   \quad  It is only useful for a product of small primes.
   Use \linkingtwo{factor.misc}{FactoredInteger}{proper\_divisors} in a more
   general case.
   \spacing
   % input, output document
   \quad The output is the list of all proper divisors.\\
   \paragraph{DEPRECATION:} This function will be removed in the next release.
   Please use \linkingtwo{factor.misc}{FactoredInteger}{proper\_divisors} instead.\\
%
 \subsection{primitive\_root -- primitive root}\linkedone{prime}{primitive\_root}
   \func{primitive\_root}
   {\hiki{p}{integer}}{\out{integer}}\\
   \spacing
   % document of basic document
   \quad Return a primitive root of \param{p}.\\
   \spacing
   % added document
   %\quad  
   %\spacing
   % input, output document
   \quad \param{p} must be an odd prime.
%
 \subsection{Lucas\_chain -- Lucas sequence}\linkedone{prime}{Lucas\_chain}
   \func{Lucas\_chain}
   {\hiki{n}{integer},\ \hiki{f}{function},\ \hiki{g}{function},\ \hiki{x\_0}{integer},\ \hiki{x\_1}{integer}}{(\out{integer},\ \out{integer})}\\
   \spacing
   % document of basic document
   \quad Return the value of ($x_n$,\ $x_{n+1}$) for the sequnce $\{ x_i \}$ defined as:\\
   \begin{eqnarray*}
      x_{2i} = \param{f}(x_i)\\
      x_{2i+1} = \param{g}(x_i, x_{i+1})\ ,
   \end{eqnarray*}
   where the initial values \param{x\_0},\ \param{x\_1}.\\
   \spacing
   % added document
   %\quad  
   %\spacing
   % input, output document
   \quad \param{f} is the function which can be input as $1$-ary integer.
   \param{g} is the function which can be input as $2$-ary integer.\\
%
\begin{ex}
>>> prime.primeq(131)
True
>>> prime.primeq(133)
False
>>> g = prime.generator()
>>> g.next()
2
>>> g.next()
3
>>> prime.prime(10)
29
>>> prime.nextPrime(100)
101
>>> prime.primitive_root(23)
5
\end{ex}%Don't indent!(indent causes an error.)
\C

%---------- end document ---------- %

\bibliographystyle{jplain}%use jbibtex
\bibliography{nzmath_references}

\end{document}


%%%%%%%%%%%%%%%%%%%%%%%%%%%%%%%%%%%%%%%%%%%%%%%%%%%%%%%%%%%%%%
%
% macros for nzmath manual
%
%%%%%%%%%%%%%%%%%%%%%%%%%%%%%%%%%%%%%%%%%%%%%%%%%%%%%%%%%%%%%
\usepackage{amssymb,amsmath}
\usepackage{color}
\usepackage[dvipdfm,bookmarks=true,bookmarksnumbered=true,%
 pdftitle={NZMATH Users Manual},%
 pdfsubject={Manual for NZMATH Users},%
 pdfauthor={NZMATH Development Group},%
 pdfkeywords={TeX; dvipdfmx; hyperref; color;},%
 colorlinks=true]{hyperref}
\usepackage{fancybox}
\usepackage[T1]{fontenc}
%
\newcommand{\DS}{\displaystyle}
\newcommand{\C}{\clearpage}
\newcommand{\NO}{\noindent}
\newcommand{\negok}{$\dagger$}
\newcommand{\spacing}{\vspace{1pt}\\ }
% software macros
\newcommand{\nzmathzero}{{\footnotesize $\mathbb{N}\mathbb{Z}$}\texttt{MATH}}
\newcommand{\nzmath}{{\nzmathzero}\ }
\newcommand{\pythonzero}{$\mbox{\texttt{Python}}$}
\newcommand{\python}{{\pythonzero}\ }
% link macros
\newcommand{\linkingzero}[1]{{\bf \hyperlink{#1}{#1}}}%module
\newcommand{\linkingone}[2]{{\bf \hyperlink{#1.#2}{#2}}}%module,class/function etc.
\newcommand{\linkingtwo}[3]{{\bf \hyperlink{#1.#2.#3}{#3}}}%module,class,method
\newcommand{\linkedzero}[1]{\hypertarget{#1}{}}
\newcommand{\linkedone}[2]{\hypertarget{#1.#2}{}}
\newcommand{\linkedtwo}[3]{\hypertarget{#1.#2.#3}{}}
\newcommand{\linktutorial}[1]{\href{http://docs.python.org/tutorial/#1}{#1}}
\newcommand{\linktutorialone}[2]{\href{http://docs.python.org/tutorial/#1}{#2}}
\newcommand{\linklibrary}[1]{\href{http://docs.python.org/library/#1}{#1}}
\newcommand{\linklibraryone}[2]{\href{http://docs.python.org/library/#1}{#2}}
\newcommand{\pythonhp}{\href{http://www.python.org/}{\python website}}
\newcommand{\nzmathwiki}{\href{http://nzmath.sourceforge.net/wiki/}{{\nzmathzero}Wiki}}
\newcommand{\nzmathsf}{\href{http://sourceforge.net/projects/nzmath/}{\nzmath Project Page}}
\newcommand{\nzmathtnt}{\href{http://tnt.math.se.tmu.ac.jp/nzmath/}{\nzmath Project Official Page}}
% parameter name
\newcommand{\param}[1]{{\tt #1}}
% function macros
\newcommand{\hiki}[2]{{\tt #1}:\ {\em #2}}
\newcommand{\hikiopt}[3]{{\tt #1}:\ {\em #2}=#3}

\newdimen\hoge
\newdimen\truetextwidth
\newcommand{\func}[3]{%
\setbox0\hbox{#1(#2)}
\hoge=\wd0
\truetextwidth=\textwidth
\advance \truetextwidth by -2\oddsidemargin
\ifdim\hoge<\truetextwidth % short form
{\bf \colorbox{skyyellow}{#1(#2)\ $\to$ #3}}
%
\else % long form
\fcolorbox{skyyellow}{skyyellow}{%
   \begin{minipage}{\textwidth}%
   {\bf #1(#2)\\ %
    \qquad\quad   $\to$\ #3}%
   \end{minipage}%
   }%
\fi%
}

\newcommand{\out}[1]{{\em #1}}
\newcommand{\initialize}{%
  \paragraph{\large \colorbox{skyblue}{Initialize (Constructor)}}%
    \quad\\ %
    \vspace{3pt}\\
}
\newcommand{\method}{\C \paragraph{\large \colorbox{skyblue}{Methods}}}
% Attribute environment
\newenvironment{at}
{%begin
\paragraph{\large \colorbox{skyblue}{Attribute}}
\quad\\
\begin{description}
}%
{%end
\end{description}
}
% Operation environment
\newenvironment{op}
{%begin
\paragraph{\large \colorbox{skyblue}{Operations}}
\quad\\
\begin{table}[h]
\begin{center}
\begin{tabular}{|l|l|}
\hline
operator & explanation\\
\hline
}%
{%end
\hline
\end{tabular}
\end{center}
\end{table}
}
% Examples environment
\newenvironment{ex}%
{%begin
\paragraph{\large \colorbox{skyblue}{Examples}}
\VerbatimEnvironment
\renewcommand{\EveryVerbatim}{\fontencoding{OT1}\selectfont}
\begin{quote}
\begin{Verbatim}
}%
{%end
\end{Verbatim}
\end{quote}
}
%
\definecolor{skyblue}{cmyk}{0.2, 0, 0.1, 0}
\definecolor{skyyellow}{cmyk}{0.1, 0.1, 0.5, 0}
%
%\title{NZMATH User Manual\\ {\large{(for version 1.0)}}}
%\date{}
%\author{}
\begin{document}
%\maketitle
%
\setcounter{tocdepth}{3}
\setcounter{secnumdepth}{3}


\tableofcontents
\C

\chapter{Classes}


%---------- start document ---------- %
 \section{quad -- Imaginary Quadratic Field}\linkedzero{quad}
 \begin{itemize}
 \item {\bf Classes}
   \begin{itemize}
   \item \linkingone{quad}{ReducedQuadraticForm}
   \item \linkingone{quad}{ClassGroup}
   \end{itemize}
 \item {\bf Functions}
   \begin{itemize}
   \item \linkingone{quad}{class\_formula}
   \item \linkingone{quad}{class\_number}
   \item \linkingone{quad}{class\_group}
   \item \linkingone{quad}{class\_number\_bsgs}
   \item \linkingone{quad}{class\_group\_bsgs}
   %\item \linkingone{quad}{disc}
   %\item \linkingone{quad}{compositePDF}
   %\item \linkingone{quad}{reducePDF}
   %\item \linkingone{quad}{sqrPDF}
   %\item \linkingone{quad}{powPDF}
   \end{itemize}
 \end{itemize}
%
  \subsection{ReducedQuadraticForm -- Reduced Quadratic Form Class}\linkedone{quad}{ReducedQuadraticForm}
  \initialize
  \func{ReducedQuadraticForm}{\hiki{f}{list}, \, \hiki{unit}{list}}{\out{ReducedQuadraticForm}}\\
  \spacing
  % document of basic document
  \quad Create ReducedQuadraticForm object.
  \spacing
  % added document
  \spacing
  % input, output document
  \quad \param{f}, \param{unit} must be list of \(3\) integers {\tt [a, b, c]},
  representing a quadratic form \(ax^2+bxy+cy^2\).
  \param{unit} represents the unit form.
  \begin{op}
    \verb|M * N| & Return the composition form of \param{M} and \param{N}. \\
    \verb|M ** a| & Return the $a$-th powering of \param{M}. \\
    \verb|M / N| & Division of form. \\
    \verb|M == N| & Return whether \param{M} and \param{N} are equal or not. \\
    \verb|M != N| & Return whether \param{M} and \param{N} are unequal or not. \\    
  \end{op}
  \method
  \subsubsection{inverse}\linkedtwo{quad}{ReducedQuadraticForm}{inverse}
  \func{inverse}{\param{self}}{\out{ReducedQuadraticForm}}\\
  \spacing
  % document of basic document
  \quad Return the inverse of \param{self}. \\
%
  \subsubsection{disc}\linkedtwo{quad}{ReducedQuadraticForm}{disc}
  \func{disc}{\param{self}}{\out{ReducedQuadraticForm}}\\
  \spacing
  % document of basic document
  \quad Return the discriminant of \param{self}. \\
%
  \subsection{ClassGroup -- Class Group Class}\linkedone{quad}{ClassGroup}
  \initialize
  \func{ClassGroup}{\hiki{disc}{integer}, \, \hiki{cl}{integer}, \, \hikiopt{element}{integer}{None}}{\out{ClassGroup}}\\
  \spacing
  % document of basic document
  \quad Create ClassGroup object.
  \method
   
  \subsection{class\_formula}\linkedone{quad}{class\_formula}
   \func{class\_formula}{\hiki{d}{integer}, \, \hiki{uprbd}{integer}}{\out{integer}}\\
   \spacing
   % document of basic document
   \quad Return the approximation of class number $h$ with discriminant \param{d} using class formula. \\
   \spacing
   % added document
   \quad class formula $\DS h = \frac{\sqrt{|\param{d}|}}{\pi}\prod_{p}\left(1 - \left(\frac{\param{d}}{p}\right) \frac{1}{p}\right)^{-1}$. \\
   \spacing
   % input, output document
   \quad Input number \param{d} must be int, long or \linkingone{rational}{Integer}.  
%
  \subsection{class\_number}\linkedone{quad}{class\_number}
   \func{class\_number}{\hiki{d}{integer}, \, \hikiopt{limit\_of\_d}{integer}{1000000000}}{\out{integer}}\\
   \spacing
   % document of basic document
   \quad Return the class number with the discriminant \param{d} by counting reduced forms. \\
   \spacing
   % added document
   \quad \param{d} is not only fundamental discriminant. \\
   \spacing
   % input, output document
   \quad Input number \param{d} must be int, long or \linkingone{rational}{Integer}. 
%
  \subsection{class\_group}\linkedone{quad}{class\_group}
   \func{class\_group}{\hiki{d}{integer}, \, \hikiopt{limit\_of\_d}{integer}{1000000000}}{\out{integer}}\\
   \spacing
   % document of basic document
   \quad Return the class number and the class group with the discriminant \param{d} by counting reduced forms. \\
   \spacing
   % added document
   \quad \param{d} is not only fundamental discriminant. \\
   \spacing
   % input, output document
   \quad Input number \param{d} must be int, long or \linkingone{rational}{Integer}. 
%
  \subsection{class\_number\_bsgs}\linkedone{quad}{class\_number\_bsgs}
   \func{class\_number\_bsgs}{\hiki{d}{integer}}{\out{integer}}\\
   \spacing
   % document of basic document
   \quad Return the class number with the discriminant \param{d} using Baby-step Giant-step algorithm. \\
   \spacing
   % added document
   \quad \param{d} is not only fundamental discriminant. \\
   \spacing
   % input, output document
   \quad Input number \param{d} must be int, long or \linkingone{rational}{Integer}. 
%
  \subsection{class\_group\_bsgs}\linkedone{quad}{class\_group\_bsgs}
   \func{class\_group\_bsgs}{\hiki{d}{integer}, \, \hiki{cl}{integer}, \, \hiki{qin}{list}}{\out{integer}}\\
   \spacing
   % document of basic document
   \quad Return the construction of the class group of order $p^{exp}$ with the discriminant \param{disc}, where $\param{qin} = [p, exp]$. \\
   \spacing
   % input, output document
   \quad Input number \param{d}, \param{cl} must be int, long or \linkingone{rational}{Integer}. 
%
  %\subsection{disc}\linkedone{quad}{disc}
   %\func{disc}{\hiki{f}{list}}{\out{integer}}\\
   %\spacing
   % document of basic document
   %\quad Return the discriminant of the quadratic form \param{f}.
   %\spacing
   % input, output document
   %\quad \param{f} is a 3-component list just like that of \linkingone{quad}{ReducedQuadraticForm}.
%
  %\subsection{compositePDF}\linkedone{quad}{compositePDF}
   %\func{compositePDF}{\hiki{f\_1}{list}, \, \hiki{f\_2}{list}}{\out{ReducedQuadraticForm}}\\
   %\spacing
   % document of basic document
   %\quad Return the  form of the positive definite form \param{f}.
   %\spacing
   % added document
   %\quad Not only fundamental discriminant.
   %\spacing
   % input, output document
%
  %\subsection{reducePDF}\linkedone{quad}{reducePDF}
   %\func{reducePDF}{\hiki{f}{list}}{\out{ReducedQuadraticForm}}\\
   %\spacing
   % document of basic document
   %\quad Return the reduced form of the positive definite form \param{f}.
   %\spacing
   % added document
   %\quad Not only fundamental discriminant.
   %\spacing
   % input, output document
%
  %\subsection{sqrPDF}\linkedone{quad}{sqrPDF}
   %\func{sqrPDF}{\hiki{f}{list}}{\out{ReducedQuadraticForm}}\\
   %\spacing
   % document of basic document
   %\quad Return the square of the positive definite form \param{f}.
   %\spacing
   % added document
   %\quad Not only fundamental discriminant.
   %\spacing
   % input, output document
%
  %\subsection{powPDF}\linkedone{quad}{powPDF}
   %\func{powPDF}{\hiki{f}{list}}{\out{ReducedQuadraticForm}}\\
   %\spacing
   % document of basic document
   %\quad Return the reduced form of the positive definite form \param{f}.
   %\spacing
   % added document
   %\quad Not only fundamental discriminant.
   %\spacing
   % input, output document
%
  %\subsection{compositePDF}\linkedone{quad}{compositePDF}
   %\func{compositePDF}{\hiki{f\_1}{list}, \, \hiki{f\_2}{list}}{\out{ReducedQuadraticForm}}\\
   %\spacing
   % document of basic document
   %\quad Return the composition of forms \param{f\_1}, \param{f\_2}.
   %\spacing
   % added document
   %\quad Not only fundamental discriminant.
   %\spacing
   % input, output document
%
  %\subsection{unit\_form}\linkedone{quad}{unit\_form}
   %\func{unit\_form}{\hiki{disc}{integer}}{\out{ReducedQuadraticForm}}\\
   %\spacing
   % document of basic document
   %\quad Return generated form with the discriminant \param{disc}.
   %\spacing
   % added document
   %\quad Not only fundamental discriminant.
   %\spacing
   % input, output document
%
  %\subsection{kronecker}\linkedone{quad}{kronecker}
   %\func{kronecker}{\hiki{a}{integer}, \, \hiki{b}{integer}}{\out{integer}}\\
   %\spacing
   % document of basic document
   %\quad Return the Kronecker symbol $\DS (\frac{\param{a}}{\param{b}})$.
   %\spacing
   % added document
   %\quad Not only fundamental discriminant.
   %\spacing
   % input, output document
%
  %\subsection{number\_unit}\linkedone{quad}{kronecker}
   %\func{kronecker}{\hiki{a}{integer}, \, \hiki{b}{integer}}{\out{integer}}\\
   %\spacing
   % document of basic document
   %\quad Return the Kronecker symbol $\frac{\param{a}}{\param{b}}$.
   %\spacing
   % added document
   %\quad Not only fundamental discriminant.
   %\spacing
   % input, output document
%
  %\subsection{crt}\linkedone{quad}{crt}
   %\func{crt}{\hiki{inlist}{list}}{\out{integer}}\\
   %\spacing
   % document of basic document
   %\quad Return the Kronecker symbol $\frac{\param{a}}{\param{b}}$.
   %\spacing
   % added document
   %\quad Not only fundamental discriminant.
   %\spacing
   % input, output document
%
\begin{ex}
>>> quad.class_formula(-1200, 100000)
12
>>> quad.class_number(-1200)
12
>>> quad.class_group(-1200)
(12, [ReducedQuadraticForm(1, 0, 300), ReducedQuadraticForm(3, 0, 100), 
ReducedQuadraticForm(4, 0, 75), ReducedQuadraticForm(12, 0, 25), 
ReducedQuadraticForm(7, 2, 43), ReducedQuadraticForm(7, -2, 43), 
ReducedQuadraticForm(16, 4, 19), ReducedQuadraticForm(16, -4, 19), 
ReducedQuadraticForm(13, 10, 25), ReducedQuadraticForm(13, -10, 25), 
ReducedQuadraticForm(16, 12, 21), ReducedQuadraticForm(16, -12, 21)])
>>> quad.class_number_bsgs(-1200)
12L
>>> quad.class_group_bsgs(-1200, 12, [3, 1])
([ReducedQuadraticForm(16, -12, 21)], [[3L]])
>>> quad.class_group_bsgs(-1200, 12, [2, 2])
([ReducedQuadraticForm(12, 0, 25), ReducedQuadraticForm(4, 0, 75)], 
[[2L], [2L, 0]])
\end{ex}%Don't indent!(indent causes an error.)
\C

%---------- end document ---------- %

\bibliographystyle{jplain}%use jbibtex
\bibliography{nzmath_references}

\end{document}


%%%%%%%%%%%%%%%%%%%%%%%%%%%%%%%%%%%%%%%%%%%%%%%%%%%%%%%%%%%%%%
%
% macros for nzmath manual
%
%%%%%%%%%%%%%%%%%%%%%%%%%%%%%%%%%%%%%%%%%%%%%%%%%%%%%%%%%%%%%
\usepackage{amssymb,amsmath}
\usepackage{color}
\usepackage[dvipdfm,bookmarks=true,bookmarksnumbered=true,%
 pdftitle={NZMATH Users Manual},%
 pdfsubject={Manual for NZMATH Users},%
 pdfauthor={NZMATH Development Group},%
 pdfkeywords={TeX; dvipdfmx; hyperref; color;},%
 colorlinks=true]{hyperref}
\usepackage{fancybox}
\usepackage[T1]{fontenc}
%
\newcommand{\DS}{\displaystyle}
\newcommand{\C}{\clearpage}
\newcommand{\NO}{\noindent}
\newcommand{\negok}{$\dagger$}
\newcommand{\spacing}{\vspace{1pt}\\ }
% software macros
\newcommand{\nzmathzero}{{\footnotesize $\mathbb{N}\mathbb{Z}$}\texttt{MATH}}
\newcommand{\nzmath}{{\nzmathzero}\ }
\newcommand{\pythonzero}{$\mbox{\texttt{Python}}$}
\newcommand{\python}{{\pythonzero}\ }
% link macros
\newcommand{\linkingzero}[1]{{\bf \hyperlink{#1}{#1}}}%module
\newcommand{\linkingone}[2]{{\bf \hyperlink{#1.#2}{#2}}}%module,class/function etc.
\newcommand{\linkingtwo}[3]{{\bf \hyperlink{#1.#2.#3}{#3}}}%module,class,method
\newcommand{\linkedzero}[1]{\hypertarget{#1}{}}
\newcommand{\linkedone}[2]{\hypertarget{#1.#2}{}}
\newcommand{\linkedtwo}[3]{\hypertarget{#1.#2.#3}{}}
\newcommand{\linktutorial}[1]{\href{http://docs.python.org/tutorial/#1}{#1}}
\newcommand{\linktutorialone}[2]{\href{http://docs.python.org/tutorial/#1}{#2}}
\newcommand{\linklibrary}[1]{\href{http://docs.python.org/library/#1}{#1}}
\newcommand{\linklibraryone}[2]{\href{http://docs.python.org/library/#1}{#2}}
\newcommand{\pythonhp}{\href{http://www.python.org/}{\python website}}
\newcommand{\nzmathwiki}{\href{http://nzmath.sourceforge.net/wiki/}{{\nzmathzero}Wiki}}
\newcommand{\nzmathsf}{\href{http://sourceforge.net/projects/nzmath/}{\nzmath Project Page}}
\newcommand{\nzmathtnt}{\href{http://tnt.math.se.tmu.ac.jp/nzmath/}{\nzmath Project Official Page}}
% parameter name
\newcommand{\param}[1]{{\tt #1}}
% function macros
\newcommand{\hiki}[2]{{\tt #1}:\ {\em #2}}
\newcommand{\hikiopt}[3]{{\tt #1}:\ {\em #2}=#3}

\newdimen\hoge
\newdimen\truetextwidth
\newcommand{\func}[3]{%
\setbox0\hbox{#1(#2)}
\hoge=\wd0
\truetextwidth=\textwidth
\advance \truetextwidth by -2\oddsidemargin
\ifdim\hoge<\truetextwidth % short form
{\bf \colorbox{skyyellow}{#1(#2)\ $\to$ #3}}
%
\else % long form
\fcolorbox{skyyellow}{skyyellow}{%
   \begin{minipage}{\textwidth}%
   {\bf #1(#2)\\ %
    \qquad\quad   $\to$\ #3}%
   \end{minipage}%
   }%
\fi%
}

\newcommand{\out}[1]{{\em #1}}
\newcommand{\initialize}{%
  \paragraph{\large \colorbox{skyblue}{Initialize (Constructor)}}%
    \quad\\ %
    \vspace{3pt}\\
}
\newcommand{\method}{\C \paragraph{\large \colorbox{skyblue}{Methods}}}
% Attribute environment
\newenvironment{at}
{%begin
\paragraph{\large \colorbox{skyblue}{Attribute}}
\quad\\
\begin{description}
}%
{%end
\end{description}
}
% Operation environment
\newenvironment{op}
{%begin
\paragraph{\large \colorbox{skyblue}{Operations}}
\quad\\
\begin{table}[h]
\begin{center}
\begin{tabular}{|l|l|}
\hline
operator & explanation\\
\hline
}%
{%end
\hline
\end{tabular}
\end{center}
\end{table}
}
% Examples environment
\newenvironment{ex}%
{%begin
\paragraph{\large \colorbox{skyblue}{Examples}}
\VerbatimEnvironment
\renewcommand{\EveryVerbatim}{\fontencoding{OT1}\selectfont}
\begin{quote}
\begin{Verbatim}
}%
{%end
\end{Verbatim}
\end{quote}
}
%
\definecolor{skyblue}{cmyk}{0.2, 0, 0.1, 0}
\definecolor{skyyellow}{cmyk}{0.1, 0.1, 0.5, 0}
%
%\title{NZMATH User Manual\\ {\large{(for version 1.0)}}}
%\date{}
%\author{}
\begin{document}
%\maketitle
%
\setcounter{tocdepth}{3}
\setcounter{secnumdepth}{3}


\tableofcontents
\C

\chapter{Classes}


%---------- start document ---------- %
 \section{round2 -- the round 2 method}\linkedzero{round2}
 \begin{itemize}
   \item {\bf Classes}
   \begin{itemize}
     \item \linkingone{round2}{ModuleWithDenominator}
   \end{itemize}
   \item {\bf Functions}
     \begin{itemize}
       \item \linkingone{round2}{round2}
       \item \linkingone{round2}{Dedekind}
     \end{itemize}
 \end{itemize}

 The round 2 method is for obtaining the maximal order of a number
 field from an order generated by a root of a defining polynomial of
 the field.

 This implementation of the method is based on \cite{Cohen1}(Algorithm 6.1.8)
 and \cite{Kida}(Chapter 3).

\C

 \subsection{ModuleWithDenominator -- bases of $\mathbb{Z}$-module with denominator.}\linkedone{round2}{ModuleWithDenominator}
 \initialize
  \func{ModuleWithDenominator}{%
    \hiki{basis}{list},\ %
    \hiki{denominator}{integer},\ %
    **\hiki{hints}{dict}%
  }{\out{ModuleWithDenominator}}\\
  \spacing
  % document of basic document
  \quad This class represents bases of $\mathbb{Z}$-module with denominator.
  It is not a general purpose $\mathbb{Z}$-module, you are warned.
  % added document
  %
  % \spacing
  % input, output document
  \quad \param{basis} is a list of integer sequences.\\
  \quad \param{denominator} is a common denominator of all bases.\\
  \quad \negok Optionally you can supply keyword argument \param{dimension} if
  you would like to postpone the initialization of \param{basis}.
  \begin{op}
    \verb|A + B| & sum of two modules\\
    \verb|a * B| & scalar multiplication\\
    \verb|B / d| & divide by an integer\\
  \end{op}
  \method
  \subsubsection{get\_rationals -- get the bases as a list of rationals}\linkedtwo{round2}{ModuleWithDenominator}{get\_rationals}
   \func{get\_rationals}{\param{self}}{\out{list}}\\
   \spacing
   % document of basic document
   \quad Return a list of lists of rational numbers, which is bases
   divided by denominator.\\
   \spacing
 \subsubsection{get\_polynomials -- get the bases as a list of polynomials}\linkedtwo{round2}{ModuleWithDenominator}{get\_polynomials}
   \func{get\_polynomials}{\param{self}}{\out{list}}\\
   \spacing
   % document of basic document
   \quad Return a list of rational polynomials, which is made from
   bases divided by denominator.\\

   \subsubsection{determinant -- determinant of the bases}\linkedtwo{round2}{ModuleWithDenominator}{determinant}
   \func{determinant}{\param{self}}{\out{list}}\\
   \spacing
   \quad Return determinant of the bases (bases ought to be of full
   rank and in Hermite normal form).

\C
  \subsection{round2(function)}\linkedone{round2}{round2}
  \func{round2}{\hiki{minpoly\_coeff}{list}}{(\out{list},\ \out{integer})}\\
   \spacing
   % document of basic document
   \quad Return integral basis of the ring of integers of a field with its
    discriminant.  The field is given by a list of integers, which is
    a polynomial of generating element \(\theta\).  The polynomial ought to
    be monic, in other word, the generating element ought to be an
    algebraic integer.\\
    \quad The integral basis will be given as a list of rational vectors
    with respect to \(\theta\).\\
    %(In other functions, bases are returned in the same fashion.)\\
   \spacing
   \subsection{Dedekind(function)}\linkedone{round2}{Dedekind}
   \func{Dedekind}{%
     \hiki{minpoly\_coeff}{list},\ %
     \hiki{p}{integer},\ %
     \hiki{e}{integer}%
   }{(\out{bool},\ \out{ModuleWithDenominator})}\\
   \spacing
   \quad This is the Dedekind criterion.\\
   \spacing \quad \param{minpoly\_coeff} is an integer list of the
   minimal polynomial of \(\theta\).\\
   \quad \param{p}{\tt **}\param{e} divides the discriminant of the minimal.\\
   \quad The first element of the returned tuple is whether the
   computation about \param{p} is finished or not.\\
  \C

%---------- end document ---------- %

\bibliographystyle{jplain}%use jbibtex
\bibliography{nzmath_references}

\end{document}


%\documentclass{report}

%%%%%%%%%%%%%%%%%%%%%%%%%%%%%%%%%%%%%%%%%%%%%%%%%%%%%%%%%%%%%
%
% macros for nzmath manual
%
%%%%%%%%%%%%%%%%%%%%%%%%%%%%%%%%%%%%%%%%%%%%%%%%%%%%%%%%%%%%%
\usepackage{amssymb,amsmath}
\usepackage{color}
\usepackage[dvipdfm,bookmarks=true,bookmarksnumbered=true,%
 pdftitle={NZMATH Users Manual},%
 pdfsubject={Manual for NZMATH Users},%
 pdfauthor={NZMATH Development Group},%
 pdfkeywords={TeX; dvipdfmx; hyperref; color;},%
 colorlinks=true]{hyperref}
\usepackage{fancybox}
\usepackage[T1]{fontenc}
%
\newcommand{\DS}{\displaystyle}
\newcommand{\C}{\clearpage}
\newcommand{\NO}{\noindent}
\newcommand{\negok}{$\dagger$}
\newcommand{\spacing}{\vspace{1pt}\\ }
% software macros
\newcommand{\nzmathzero}{{\footnotesize $\mathbb{N}\mathbb{Z}$}\texttt{MATH}}
\newcommand{\nzmath}{{\nzmathzero}\ }
\newcommand{\pythonzero}{$\mbox{\texttt{Python}}$}
\newcommand{\python}{{\pythonzero}\ }
% link macros
\newcommand{\linkingzero}[1]{{\bf \hyperlink{#1}{#1}}}%module
\newcommand{\linkingone}[2]{{\bf \hyperlink{#1.#2}{#2}}}%module,class/function etc.
\newcommand{\linkingtwo}[3]{{\bf \hyperlink{#1.#2.#3}{#3}}}%module,class,method
\newcommand{\linkedzero}[1]{\hypertarget{#1}{}}
\newcommand{\linkedone}[2]{\hypertarget{#1.#2}{}}
\newcommand{\linkedtwo}[3]{\hypertarget{#1.#2.#3}{}}
\newcommand{\linktutorial}[1]{\href{http://docs.python.org/tutorial/#1}{#1}}
\newcommand{\linktutorialone}[2]{\href{http://docs.python.org/tutorial/#1}{#2}}
\newcommand{\linklibrary}[1]{\href{http://docs.python.org/library/#1}{#1}}
\newcommand{\linklibraryone}[2]{\href{http://docs.python.org/library/#1}{#2}}
\newcommand{\pythonhp}{\href{http://www.python.org/}{\python website}}
\newcommand{\nzmathwiki}{\href{http://nzmath.sourceforge.net/wiki/}{{\nzmathzero}Wiki}}
\newcommand{\nzmathsf}{\href{http://sourceforge.net/projects/nzmath/}{\nzmath Project Page}}
\newcommand{\nzmathtnt}{\href{http://tnt.math.se.tmu.ac.jp/nzmath/}{\nzmath Project Official Page}}
% parameter name
\newcommand{\param}[1]{{\tt #1}}
% function macros
\newcommand{\hiki}[2]{{\tt #1}:\ {\em #2}}
\newcommand{\hikiopt}[3]{{\tt #1}:\ {\em #2}=#3}

\newdimen\hoge
\newdimen\truetextwidth
\newcommand{\func}[3]{%
\setbox0\hbox{#1(#2)}
\hoge=\wd0
\truetextwidth=\textwidth
\advance \truetextwidth by -2\oddsidemargin
\ifdim\hoge<\truetextwidth % short form
{\bf \colorbox{skyyellow}{#1(#2)\ $\to$ #3}}
%
\else % long form
\fcolorbox{skyyellow}{skyyellow}{%
   \begin{minipage}{\textwidth}%
   {\bf #1(#2)\\ %
    \qquad\quad   $\to$\ #3}%
   \end{minipage}%
   }%
\fi%
}

\newcommand{\out}[1]{{\em #1}}
\newcommand{\initialize}{%
  \paragraph{\large \colorbox{skyblue}{Initialize (Constructor)}}%
    \quad\\ %
    \vspace{3pt}\\
}
\newcommand{\method}{\C \paragraph{\large \colorbox{skyblue}{Methods}}}
% Attribute environment
\newenvironment{at}
{%begin
\paragraph{\large \colorbox{skyblue}{Attribute}}
\quad\\
\begin{description}
}%
{%end
\end{description}
}
% Operation environment
\newenvironment{op}
{%begin
\paragraph{\large \colorbox{skyblue}{Operations}}
\quad\\
\begin{table}[h]
\begin{center}
\begin{tabular}{|l|l|}
\hline
operator & explanation\\
\hline
}%
{%end
\hline
\end{tabular}
\end{center}
\end{table}
}
% Examples environment
\newenvironment{ex}%
{%begin
\paragraph{\large \colorbox{skyblue}{Examples}}
\VerbatimEnvironment
\renewcommand{\EveryVerbatim}{\fontencoding{OT1}\selectfont}
\begin{quote}
\begin{Verbatim}
}%
{%end
\end{Verbatim}
\end{quote}
}
%
\definecolor{skyblue}{cmyk}{0.2, 0, 0.1, 0}
\definecolor{skyyellow}{cmyk}{0.1, 0.1, 0.5, 0}
%
%\title{NZMATH User Manual\\ {\large{(for version 1.0)}}}
%\date{}
%\author{}
\begin{document}
%\maketitle
%
\setcounter{tocdepth}{3}
\setcounter{secnumdepth}{3}


\tableofcontents
\C

\chapter{Functions}


%---------- start document ---------- %
 \section{squarefree -- Squarefreeness tests}\linkedzero{squarefree}

There are two method groups.
A function in one group raises \linkingone{squarefree}{Undetermined} when it cannot determine squarefreeness.
A function in another group returns {\tt None} in such cases.
The latter group of functions have ``\_ternary'' suffix on their names.
We refer a set \(\{{\tt True}, {\tt False}, {\tt None}\}\) as {\it ternary}\linkedone{squarefree}{ternary}.

The parameter type {\it integer}\linkedone{squarefree}{integer} means either {\it int}, {\it long} or \linkingone{rational}{Integer}.

This module provides an exception class.
\begin{description}
  \item[Undetermined]:\ Report undetermined state of calculation.
    The exception will be raised by
    \linkingone{squarefree}{lenstra} or
    \linkingone{squarefree}{trivial\_test}.
\end{description}

\subsection{Definition}

  We define squarefreeness as:\\
  \(n\) is squarefree \(\iff\) there is no prime \(p\) whose square divides \(n\).

\vspace{1em}
\noindent Examples:
  \begin{itemize}
  \item \(0\) is non-squarefree because any square of prime can divide \(0\).
  \item \(1\) is squarefree because there is no prime dividing \(1\).
  \item \(2\), \(3\), \(5\), and any other primes are squarefree.
  \item \(4\), \(8\), \(9\), \(12\), \(16\) are non-squarefree composites.
  \item \(6\), \(10\), \(14\), \(15\), \(21\) are squarefree composites.
\end{itemize}

 \subsection{lenstra -- Lenstra's condition}\linkedone{squarefree}{lenstra}
 \func{lenstra}{\hiki{n}{integer}}{\out{bool}}\\
 \spacing
 % document of basic document
 \quad If return value is True, \param{n} is squarefree.  Otherwise, the
 squarefreeness is still unknown and \linkingone{squarefree}{Undetermined} is raised.
 The algorithm is based on~\cite{Lenstra1979}. \\
 \spacing
 % added document
 \negok The condition is so strong that it seems \param{n} has to be a
 prime or a Carmichael number to satisfy it.\\
 \spacing
 % input, output document
 \quad Input parameter \param{n} ought to be an odd \linkingone{squarefree}{integer}.
 % 
 \subsection{trial\_division -- trial division}\linkedone{squarefree}{trial\_division}
 \func{trial\_division}{\hiki{n}{integer}}{\out{bool}}\\
 \spacing
 % document of basic document
 \quad Check whether \param{n} is squarefree or not. \\
 \spacing
 % added document
 The method is a kind of trial division and inefficient for large numbers. \\
 \spacing
 % input, output document
 \quad Input parameter \param{n} ought to be an \linkingone{squarefree}{integer}.
% 
 \subsection{trivial\_test -- trivial tests}\linkedone{squarefree}{trivial\_test}
 \func{trivial\_test}{\hiki{n}{integer}}{\out{bool}}\\
 \spacing
 % document of basic document
 \quad Check whether \param{n} is squarefree or not.  If the squarefreeness is still unknown, then \linkingone{squarefree}{Undetermined} is raised. \\
 \spacing
 % added document
 This method do anything but factorization including Lenstra's method. \\
 \spacing
 % input, output document
 \quad Input parameter \param{n} ought to be an odd \linkingone{squarefree}{integer}.
% 
 \subsection{viafactor -- via factorization}\linkedone{squarefree}{viafactor}
 \func{viafactor}{\hiki{n}{integer}}{\out{bool}}\\
 \spacing
 % document of basic document
 \quad Check whether \param{n} is squarefree or not. \\
 \spacing
 % added document
 It is obvious that if one knows the prime factorization of the number, he/she can tell whether the number is squarefree or not. \\
 \spacing
 % input, output document
 \quad Input parameter \param{n} ought to be an \linkingone{squarefree}{integer}.
% 
 \subsection{viadecomposition -- via partial factorization}\linkedone{squarefree}{viadecomposition}
 \func{viadecomposition}{\hiki{n}{integer}}{\out{bool}}\\
 \spacing
 % document of basic document
 \quad Test the squarefreeness of \param{n}.
 The return value is either one of {\tt True} or {\tt False};
 {\tt None} never be returned. \\
 \spacing
 % added document
 The method uses partial factorization into squarefree parts,
 if such partial factorization is possible.  In other cases,
 It completely factor \param{n} by trial division.
 \spacing
 % input, output document
 \quad Input parameter \param{n} ought to be an \linkingone{squarefree}{integer}.
% 
 \subsection{lenstra\_ternary -- Lenstra's condition, ternary version}\linkedone{squarefree}{lenstra\_ternary}
 \func{lenstra\_ternary}{\hiki{n}{integer}}{\out{ternary}}\\
 \spacing
 % document of basic document
 \quad Test the squarefreeness of \param{n}. The return value is one of the ternary logical constants.  If return value is {\tt True}, \param{n} is squarefree.  Otherwise, the squarefreeness is still unknown and {\tt None} is returned. \\
 \spacing
 % added document
 \negok The condition is so strong that it seems \param{n} has to be a
 prime or a Carmichael number to satisfy it.\\
 This is a ternary version of \linkingone{squarefree}{lenstra}. \\
 \spacing
 % input, output document
 \quad Input parameter \param{n} ought to be an odd \linkingone{squarefree}{integer}.
 % 
 \subsection{trivial\_test\_ternary -- trivial tests, ternary version}\linkedone{squarefree}{trivial\_test\_ternary}
 \func{trivial\_test\_ternary}{\hiki{n}{integer}}{\out{ternary}}\\
 \spacing
 % document of basic document
 \quad Test the squarefreeness of \param{n}.
 The return value is one of the ternary logical constants. \\
 \spacing
 % added document
 The method uses a series of trivial tests including \linkingone{squarefree}{lenstra\_ternary}. \\
 This is a ternary version of \linkingone{squarefree}{trivial\_test}. \\
 \spacing
 % input, output document
 \quad Input parameter \param{n} ought to be an \linkingone{squarefree}{integer}.
% 
 \subsection{trial\_division\_ternary  -- trial division, ternary version}\linkedone{squarefree}{trial\_division\_ternary}
 \func{trial\_division\_ternary}{\hiki{n}{integer}}{\out{ternary}}\\
 \spacing
 % document of basic document
 \quad Test the squarefreeness of \param{n}.
 The return value is either one of {\tt True} or {\tt False};
 {\tt None} never be returned. \\
 \spacing
 % added document
 The method is a kind of trial division. \\
 This is a ternary version of \linkingone{squarefree}{trial\_division}.\\
 \spacing
 % input, output document
 \quad Input parameter \param{n} ought to be an \linkingone{squarefree}{integer}.
% 
 \subsection{viafactor\_ternary -- via factorization, ternary version}\linkedone{squarefree}{viafactor\_ternary}
 \func{viafactor\_ternary}{\hiki{n}{integer}}{\out{ternary}}\\
 \spacing
 % document of basic document
 \quad Just for symmetry, this function is defined as an alias of \linkingone{squarefree}{viafactor}. \\
 \spacing
 % added document
 \spacing
 % input, output document
 \quad Input parameter \param{n} ought to be an \linkingone{squarefree}{integer}.
% 
\C

%---------- end document ---------- %

\bibliographystyle{jplain}%use jbibtex
\bibliography{nzmath_references}

\end{document}


%
\chapter{Classes}\label{class}
%\documentclass{report}

%%%%%%%%%%%%%%%%%%%%%%%%%%%%%%%%%%%%%%%%%%%%%%%%%%%%%%%%%%%%%
%
% macros for nzmath manual
%
%%%%%%%%%%%%%%%%%%%%%%%%%%%%%%%%%%%%%%%%%%%%%%%%%%%%%%%%%%%%%
\usepackage{amssymb,amsmath}
\usepackage{color}
\usepackage[dvipdfm,bookmarks=true,bookmarksnumbered=true,%
 pdftitle={NZMATH Users Manual},%
 pdfsubject={Manual for NZMATH Users},%
 pdfauthor={NZMATH Development Group},%
 pdfkeywords={TeX; dvipdfmx; hyperref; color;},%
 colorlinks=true]{hyperref}
\usepackage{fancybox}
\usepackage[T1]{fontenc}
%
\newcommand{\DS}{\displaystyle}
\newcommand{\C}{\clearpage}
\newcommand{\NO}{\noindent}
\newcommand{\negok}{$\dagger$}
\newcommand{\spacing}{\vspace{1pt}\\ }
% software macros
\newcommand{\nzmathzero}{{\footnotesize $\mathbb{N}\mathbb{Z}$}\texttt{MATH}}
\newcommand{\nzmath}{{\nzmathzero}\ }
\newcommand{\pythonzero}{$\mbox{\texttt{Python}}$}
\newcommand{\python}{{\pythonzero}\ }
% link macros
\newcommand{\linkingzero}[1]{{\bf \hyperlink{#1}{#1}}}%module
\newcommand{\linkingone}[2]{{\bf \hyperlink{#1.#2}{#2}}}%module,class/function etc.
\newcommand{\linkingtwo}[3]{{\bf \hyperlink{#1.#2.#3}{#3}}}%module,class,method
\newcommand{\linkedzero}[1]{\hypertarget{#1}{}}
\newcommand{\linkedone}[2]{\hypertarget{#1.#2}{}}
\newcommand{\linkedtwo}[3]{\hypertarget{#1.#2.#3}{}}
\newcommand{\linktutorial}[1]{\href{http://docs.python.org/tutorial/#1}{#1}}
\newcommand{\linktutorialone}[2]{\href{http://docs.python.org/tutorial/#1}{#2}}
\newcommand{\linklibrary}[1]{\href{http://docs.python.org/library/#1}{#1}}
\newcommand{\linklibraryone}[2]{\href{http://docs.python.org/library/#1}{#2}}
\newcommand{\pythonhp}{\href{http://www.python.org/}{\python website}}
\newcommand{\nzmathwiki}{\href{http://nzmath.sourceforge.net/wiki/}{{\nzmathzero}Wiki}}
\newcommand{\nzmathsf}{\href{http://sourceforge.net/projects/nzmath/}{\nzmath Project Page}}
\newcommand{\nzmathtnt}{\href{http://tnt.math.se.tmu.ac.jp/nzmath/}{\nzmath Project Official Page}}
% parameter name
\newcommand{\param}[1]{{\tt #1}}
% function macros
\newcommand{\hiki}[2]{{\tt #1}:\ {\em #2}}
\newcommand{\hikiopt}[3]{{\tt #1}:\ {\em #2}=#3}

\newdimen\hoge
\newdimen\truetextwidth
\newcommand{\func}[3]{%
\setbox0\hbox{#1(#2)}
\hoge=\wd0
\truetextwidth=\textwidth
\advance \truetextwidth by -2\oddsidemargin
\ifdim\hoge<\truetextwidth % short form
{\bf \colorbox{skyyellow}{#1(#2)\ $\to$ #3}}
%
\else % long form
\fcolorbox{skyyellow}{skyyellow}{%
   \begin{minipage}{\textwidth}%
   {\bf #1(#2)\\ %
    \qquad\quad   $\to$\ #3}%
   \end{minipage}%
   }%
\fi%
}

\newcommand{\out}[1]{{\em #1}}
\newcommand{\initialize}{%
  \paragraph{\large \colorbox{skyblue}{Initialize (Constructor)}}%
    \quad\\ %
    \vspace{3pt}\\
}
\newcommand{\method}{\C \paragraph{\large \colorbox{skyblue}{Methods}}}
% Attribute environment
\newenvironment{at}
{%begin
\paragraph{\large \colorbox{skyblue}{Attribute}}
\quad\\
\begin{description}
}%
{%end
\end{description}
}
% Operation environment
\newenvironment{op}
{%begin
\paragraph{\large \colorbox{skyblue}{Operations}}
\quad\\
\begin{table}[h]
\begin{center}
\begin{tabular}{|l|l|}
\hline
operator & explanation\\
\hline
}%
{%end
\hline
\end{tabular}
\end{center}
\end{table}
}
% Examples environment
\newenvironment{ex}%
{%begin
\paragraph{\large \colorbox{skyblue}{Examples}}
\VerbatimEnvironment
\renewcommand{\EveryVerbatim}{\fontencoding{OT1}\selectfont}
\begin{quote}
\begin{Verbatim}
}%
{%end
\end{Verbatim}
\end{quote}
}
%
\definecolor{skyblue}{cmyk}{0.2, 0, 0.1, 0}
\definecolor{skyyellow}{cmyk}{0.1, 0.1, 0.5, 0}
%
%\title{NZMATH User Manual\\ {\large{(for version 1.0)}}}
%\date{}
%\author{}
\begin{document}
%\maketitle
%
\setcounter{tocdepth}{3}
\setcounter{secnumdepth}{3}


\tableofcontents
\C

\chapter{Classes}


%---------- start document ---------- %
 \section{algfield -- Algebraic Number Field}\linkedzero{algfield}
 \begin{itemize}
 \item {\bf Classes}
   \begin{itemize}
   \item \linkingone{algfield}{NumberField}
   \item \linkingone{algfield}{BasicAlgNumber}
   \item \linkingone{algfield}{MatAlgNumber}
   \end{itemize}
 \item {\bf Functions}
   \begin{itemize}
   \item \linkingone{algfield}{changetype}
   \item \linkingone{algfield}{disc}
   \item \linkingone{algfield}{fppoly}
%   \item \linkingone{algfield}{prime$\_$decomp}
   \item \linkingone{algfield}{qpoly}
   \item \linkingone{algfield}{zpoly}
   \end{itemize}
 \end{itemize}
%
  \subsection{NumberField -- number field}\linkedone{algfield}{NumberField}
  \initialize
  \func{NumberField}{
    \hiki{f}{list},\
    \hikiopt{precompute}{bool}{False}
  }{
    \out{NumberField}
  }\\
  \spacing
  % document of basic document
  \quad Create NumberField object. \\
  \spacing
  % added document
  \quad This field defined by the polynomial \param{f}. \\
  The class inherits \linkingone{ring}{Field}.\\
  \spacing
  % input, output document
  \quad \param{f}, which expresses coefficients of a polynomial, must be a list of integers.
  \param{f} should be written in ascending order.  \param{f} must be monic irreducible over rational field. \\
  If \param{precompute} is True, all solutions of \param{f} (by \linkingtwo{algfield}{NumberField}{getConj}), the discriminant of \param{f} (by \linkingtwo{algfield}{NumberField}{disc}), the signature (by \linkingtwo{algfield}{NumberField}{signature}) and the field discriminant of the basis of the integer ring (by \linkingtwo{algfield}{NumberField}{integer\_ring}) are precomputed.\\
  \begin{at}
    \item[degree]\linkedtwo{algfield}{NumberField}{degree}: The (absolute) extension degree of the number field.
    \item[polynomial]\linkedtwo{algfield}{NumberField}{polynomial}: The defining polynomial of the number field.
  \end{at}
  \begin{op}
    \verb|K * F| & Return the composite field of \param{K} and \param{F}. \\
    \verb|K == F| & Check whether the equality of \param{K} and \param{F}. \\
  \end{op}
\begin{ex}
>>> K = algfield.NumberField([-2, 0, 1])
>>> L = algfield.NumberField([-3, 0, 1])
>>> print K, L
NumberField([-2, 0, 1]) NumberField([-3, 0, 1])
>>> print K * L
NumberField([1L, 0L, -10L, 0L, 1L])
\end{ex}%Don't indent!
\C
  \method
  \subsubsection{getConj -- roots of polynomial}\linkedtwo{algfield}{NumberField}{getConj}
  \func{getConj}{\param{self}}{\out{list}}\\
  \spacing
  % document of basic document
  \quad Return all (approximate) roots of the \param{self}.\linkingtwo{algfield}{NumberField}{polynomial}. \\
  \spacing
  % add document
  %\spacing
  % input, output document
  \quad The output is a list of (approximate) complex number.\\
%
  \subsubsection{disc -- polynomial discriminant}\linkedtwo{algfield}{NumberField}{disc}
  \func{disc}{\param{self}}{\out{integer}}\\
  \spacing
  % document of basic document
  \quad Return the (polynomial) discriminant of the \param{self}.\linkingtwo{algfield}{NumberField}{polynomial}. \\
  \spacing
  % add document
  \quad \negok The output is not discriminant of the number field itself. \\
  %\spacing
  % input, output document
%
  \subsubsection{integer\_ring -- integer ring}\linkedtwo{algfield}{NumberField}{integer\_ring}
  \func{integer\_ring}{\param{self}}{\out{\linkingone{matrix}{FieldSquareMatrix}}}\\
  \spacing
  % document of basic document
  \quad Return a basis of the ring of integers of \param{self}. \\
  \spacing
  % add document
  \quad \negok The function uses \linkingone{round2}{round2}.\\
  %\spacing
  % input, output document
%
  \subsubsection{field\_discriminant -- discriminant}\linkedtwo{algfield}{NumberField}{field\_discriminant}
  \func{field\_discriminant}{\param{self}}{\out{\linkingone{rational}{Rational}}}\\
  \spacing
  % document of basic document
  \quad Return the field discriminant of \param{self}. \\
  \spacing
  % add document
  \quad \negok The function uses \linkingone{round2}{round2}.\\
  %\spacing
  % input, output document
%
  \subsubsection{basis -- standard basis}\linkedtwo{algfield}{NumberField}{basis}
  \func{basis}{\param{self},\ \hiki{j}{integer}}{\out{\linkingone{algfield}{BasicAlgNumber}}}\\
  \spacing
  % document of basic document
  \quad Return the \param{j}-th basis (over the rational field) of \param{self}. \\
  \spacing
  % add document
  \quad Let $\theta$ be a solution of \param{self}.\linkingtwo{algfield}{NumberField}{polynomial}.
  Then $\theta^j$ is a part of basis of \param{self}, so the method returns them.This basis is called ``standard basis'' or ``power basis''.\\
  %\spacing
  % input, output document
%
  \subsubsection{signature -- signature}\linkedtwo{algfield}{NumberField}{signature}
  \func{signature}{\param{self}}{\out{list}}\\
  \spacing
  % document of basic document
  \quad Return the signature of \param{self}. \\
  \spacing
  % add document
  \quad \negok The method uses Strum's algorithm.\\
  %\spacing
  % input, output document
%
  \subsubsection{POLRED -- polynomial reduction}\linkedtwo{algfield}{NumberField}{POLRED}
  \func{POLRED}{\param{self}}{\out{list}}\\
  \spacing
  % document of basic document
  \quad Return some polynomials defining subfields of \param{self}. \\
  \spacing
  % add document
  \quad \negok ``POLRED'' means ``polynomial reduction''. 
  That is, it finds polynomials whose coefficients are not so large.\\
  %\spacing
  % input, output document
%  
  \subsubsection{isIntBasis -- check integral basis}\linkedtwo{algfield}{NumberField}{isIntBasis}
  \func{isIntBasis}{\param{self}}{\out{bool}}\\
  \spacing
  % document of basic document
  \quad Check whether power basis of \param{self} is also an integral basis of the field. \\
  %\spacing
  % add document
  %\spacing
  % input, output document
%  
  \subsubsection{isGaloisField -- check Galois field}\linkedtwo{algfield}{NumberField}{isGaloisField}
  \func{isGaloisField}{\param{self}}{\out{bool}}\\
  \spacing
  % document of basic document
  \quad Check whether the extension \param{self} over the rational field is Galois. \\
  %\spacing
  % add document
  \quad \negok As it stands, it only checks the signature.\\
  %\spacing
  % input, output document
%  
  \subsubsection{isFieldElement -- check field element}\linkedtwo{algfield}{NumberField}{isFieldElement}
  \func{isFieldElement}{\param{self},\ \hiki{A}{BasicAlgNumber/MatAlgNumber}}{\out{bool}}\\
  \spacing
  % document of basic document
  \quad Check whether \param{A} is an element of the field \param{self}. \\
  %\spacing
  % add document
  %\spacing
  % input, output document
  %\quad The input parameter \param{A} must be an instance of \linkingone{algfield}{BasicAlgNumber} or \linkingone{algfield}{MatAlgNumber}. \\
%  
  \subsubsection{getCharacteristic -- characteristic}\linkedtwo{algfield}{NumberField}{getCharacteristic}
  \func{getCharacteristic}{\param{self}}{\out{integer}}\\
  \spacing
  % document of basic document
  \quad Return the characteristic of \param{self}. \\
  \spacing
  % add document
  \quad It returns always zero. The method is only for ensuring consistency.\\
  %\spacing
  % input, output document
%  
  \subsubsection{createElement -- create an element}\linkedtwo{algfield}{NumberField}{createElement}
  \func{createElement}{\param{self},\ \hiki{seed}{list}}{\out{BasicAlgNumber/MatAlgNumber}}\\
  \spacing
  % document of basic document
  \quad Return an element of \param{self} with \param{seed}. \\
  \spacing
  % add document
  \quad \param{seed} determines the class of returned element.\\
  For example, if \param{seed} forms as $[[e_1, e_2, \ldots, e_n],\ d]$, then it calls \linkingone{algfield}{BasicAlgNumber}.\\
  %\spacing
  % input, output document
%  
\begin{ex}
>>> K = algfield.NumberField([3, 0, 1])
>>> K.getConj()
[-1.7320508075688774j, 1.7320508075688772j]
>>> K.disc()
-12L
>>> print K.integer_ring()
1/1 1/2
0/1 1/2
>>> K.field_discriminant()
Rational(-3, 1)
>>> K.basis(0), K.basis(1)
BasicAlgNumber([[1, 0], 1], [3, 0, 1]) BasicAlgNumber([[0, 1], 1], [3, 0, 1])
>>> K.signature()
(0, 1)
>>> K.POLRED()                     
[IntegerPolynomial([(0, 4L), (1, -2L), (2, 1L)], IntegerRing()), 
IntegerPolynomial([(0, -1L), (1, 1L)], IntegerRing())]
>>> K.isIntBasis()                 
False
\end{ex}%Don't indent!
\C
  \subsection{BasicAlgNumber -- Algebraic Number Class by standard basis}\linkedone{algfield}{BasicAlgNumber}
  \initialize
  \func{BasicAlgNumber}{
   \hiki{valuelist}{list},\
   \hiki{polynomial}{list},\ 
   \hikiopt{precompute}{bool}{False}
  }{
   \out{BasicAlgNumber}
  }\\
  \spacing
  % document of basic document
  \quad Create an algebraic number with standard (power) basis. \\
  \spacing
  % added document
  %\spacing
  % input, output document
  \quad $\param{valuelist} = [[e_1, e_2, \ldots, e_n],\ d]$ means $\DS \frac{1}{d}  (e_1 + e_2 \theta + e_3 \theta^2 + \cdots + e_n \theta^{n-1})$,  where $\theta$ is a solution of the polynomial \param{polynomial}.
  Note that $\langle \theta^i \rangle$ is a (standard) basis of the field defining by \param{polynomial} over the rational field.\\
  \spacing
  \quad $e_i,\ d$ must be integers.
  Also, \param{polynomial} should be list of integers. \\
  If \param{precompute} is True, all solutions of \param{polynomial} (by \linkingtwo{algfield}{BasicAlgNumber}{getConj}), approximation values of all conjugates of \param{self} (by \linkingtwo{algfield}{BasicAlgNumber}{getApprox}) and a polynomial which is a solution of \param{self} (by \linkingtwo{algfield}{BasicAlgNumber}{getCharPoly}) are precomputed. \\
  \begin{at}
    \item[value]\linkedtwo{algfield}{BasicAlgNumber}{value}: The list of numerators (the integer part) and the denominator of \param{self}.
    \item[coeff]\linkedtwo{algfield}{BasicAlgNumber}{coeff}: The coefficients of numerators (the integer part) of \param{self}.
    \item[denom]\linkedtwo{algfield}{BasicAlgNumber}{denom}: The denominator of the algebraic number for standard basis.
    \item[degree]\linkedtwo{algfield}{BasicAlgNumber}{degree}: The degree of extension of the field over the rational field.
    \item[polynomial]\linkedtwo{algfield}{BasicAlgNumber}{polynomial}: The defining polynomial of the field.
    \item[field]\linkedtwo{algfield}{BasicAlgNumber}{field}: The number field in which \param{self} is.
  \end{at}
  \begin{op}
    \verb|a + b| & Return the sum of \param{a} and \param{b}. \\
    \verb|a - b| & Return the subtraction of \param{a} and \param{b}. \\
    \verb| - a| & Return the negation of \param{a}. \\
    \verb|a * b| & Return the product of \param{a} and \param{b}. \\ 
    \verb|a ** k| & Return the \param{k}-th power of \param{a}. \\
    \verb|a / b| & Return the quotient of \param{a} by \param{b}. \\
  \end{op}
\begin{ex}
>>> a = algfield.BasicAlgNumber([[1, 1], 1], [-2, 0, 1])
>>> b = algfield.BasicAlgNumber([[-1, 2], 1], [-2, 0, 1])
>>> print a + b
BasicAlgNumber([[0, 3], 1], [-2, 0, 1])
>>> print a * b
BasicAlgNumber([[3L, 1L], 1], [-2, 0, 1])
>>> print a ** 3
BasicAlgNumber([[7L, 5L], 1], [-2, 0, 1])
>>> a // b
BasicAlgNumber([[5L, 3L], 7L], [-2, 0, 1])
\end{ex}%Don't indent!
\C
  \method
  \subsubsection{inverse -- inverse}\linkedtwo{algfield}{BasicAlgNumber}{inverse}
  \func{inverse}{\param{self}}{\out{BasicAlgNumber}}\\
  \spacing
  % document of basic document
  \quad Return the inverse of \param{self}. \\
  %\spacing
  % add document
  %\spacing
  % input, output document
%
  \subsubsection{getConj -- roots of polynomial}\linkedtwo{algfield}{BasicAlgNumber}{getConj}
  \func{getConj}{\param{self}}{\out{list}}\\
  \spacing
  % document of basic document
  \quad Return all (approximate) roots of \param{self}.\linkingtwo{algfield}{BasicAlgNumber}{polynomial}. \\
  %\spacing
  % add document
  %\spacing
  % input, output document
%
  \subsubsection{getApprox -- approximate conjugates}\linkedtwo{algfield}{BasicAlgNumber}{getApprox}
  \func{getApprox}{\param{self}}{\out{list}}\\
  \spacing
  % document of basic document
  \quad Return all (approximate) conjugates of \param{self}. \\
  %\spacing
  % add document
  %\spacing
  % input, output document
%
  \subsubsection{getCharPoly -- characteristic polynomial}\linkedtwo{algfield}{BasicAlgNumber}{getCharPoly}
  \func{getCharPoly}{\param{self}}{\out{list}}\\
  \spacing
  % document of basic document
  \quad Return the characteristic polynomial of \param{self}. \\
  \spacing
  % add document
  \quad \negok \param{self} is a solution of the characteristic polynomial.\\
  \spacing
  % input, output document
  \quad The output is a list of integers. \\
%
  \subsubsection{getRing -- the field}\linkedtwo{algfield}{BasicAlgNumber}{getRing}
  \func{getRing}{\param{self}}{\out{NumberField}}\\
  \spacing
  % document of basic document
  \quad Return the field which \param{self} belongs to. \\
  %\spacing
  % add document
  %\spacing
  % input, output document
%
  \subsubsection{trace -- trace}\linkedtwo{algfield}{BasicAlgNumber}{trace}
  \func{trace}{\param{self}}{\out{Rational}}\\
  \spacing
  % document of basic document
  \quad Return the trace of \param{self} in the \param{self}.\linkingtwo{algfield}{BasicAlgNumber}{field} over the rational field. \\
  %\spacing
  % add document
  %\spacing
  % input, output document
%
  \subsubsection{norm -- norm}\linkedtwo{algfield}{BasicAlgNumber}{norm}
  \func{norm}{\param{self}}{\out{Rational}}\\
  \spacing
  % document of basic document
  \quad Return the norm of \param{self} in the \param{self}.\linkingtwo{algfield}{BasicAlgNumber}{field} over the rational field. \\
  %\spacing
  % add document
  %\spacing
  % input, output document
%
  \subsubsection{isAlgInteger -- check (algebraic) integer}\linkedtwo{algfield}{BasicAlgNumber}{isAlgInteger}
  \func{isAlgInteger}{\param{self}}{\out{bool}}\\
  \spacing
  % document of basic document
  \quad Check whether \param{self} is an (algebraic) integer or not. \\
  %\spacing
  % add document
  %\spacing
  % input, output document
%
  \subsubsection{ch\_matrix -- obtain MatAlgNumber object}\linkedtwo{algfield}{BasicAlgNumber}{ch\_matrix}
  \func{ch\_matrix}{\param{self}}{\out{MatAlgNumber}}\\
  \spacing
  % document of basic document
  \quad Return \linkingone{algfield}{MatAlgNumber} object corresponding to \param{self}. \\
  %\spacing
  % add document
  %\spacing
  % input, output document
%
\begin{ex}
>>> a = algfield.BasicAlgNumber([[1, 1], 1], [-2, 0, 1])
>>> a.inverse()
BasicAlgNumber([[-1L, 1L], 1L], [-2, 0, 1])
>>> a.getConj()
[(1.4142135623730951+0j), (-1.4142135623730951+0j)]
>>> a.getApprox()
[(2.4142135623730949+0j), (-0.41421356237309515+0j)]
>>> a.getCharPoly()
[-1, -2, 1]
>>> a.getRing()
NumberField([-2, 0, 1])
>>> a.trace(), a.norm()
2 -1
>>> a.isAlgInteger()
True
>>> a.ch_matrix()
MatAlgNumber([1, 1]+[2, 1], [-2, 0, 1])
\end{ex}%Don't indent!
\C
  \subsection{MatAlgNumber -- Algebraic Number Class by matrix representation}\linkedone{algfield}{MatAlgNumber}
  \initialize
  \func{MatAlgNumber}{
    \hiki{coefficient}{list},\
    \hiki{polynomial}{list}
  }{
    \out{MatAlgNumber}
  }\\
  \spacing
  % document of basic document
  \quad Create an algebraic number represented by a matrix. \\
  \spacing
  % added document
  \quad ``matrix representation'' means the matrix $A$ over the rational field such that $(e_1 + e_2 \theta + e_3 \theta^2 + \cdots + e_n \theta^{n-1})(1,\theta,\ldots,\theta^{n-1})^T=A(1,\theta,\ldots,\theta^{n-1})^T$, where $^t$ expresses transpose operation.\\
  \spacing
  % input, output document
  \quad $\param{coefficient} = [e_1, e_2, \ldots, e_n]$ means $e_1 + e_2 \theta + e_3 \theta^2 + \cdots + e_n \theta^{n-1}$,  where $\theta$ is a solution of the polynomial \param{polynomial}.
  Note that $\langle \theta^i \rangle$ is a (standard) basis of the field defining by \param{polynomial} over the rational field.\\
  \param{coefficient} must be a list of (not only integers) rational numbers.
  \param{polynomial} must be a list of integers. \\
  \begin{at}
    \item[coeff]\linkedtwo{algfield}{MatAlgNumber}{coeff}: The coefficients of the algebraic number for standard basis.
    \item[degree]\linkedtwo{algfield}{MatAlgNumber}{degree}: The degree of extension of the field over the rational field.
    \item[matrix]\linkedtwo{algfield}{MatAlgNumber}{matrix}: The representation matrix of the algebraic number.
    \item[polynomial]\linkedtwo{algfield}{MatAlgNumber}{polynomial}: The defining polynomial of the field.
    \item[field]\linkedtwo{algfield}{MatAlgNumber}{field}: The number field in which \param{self} is.
    
  \end{at}
  \begin{op}
    \verb|a + b| & Return the sum of \param{a} and \param{b}. \\
    \verb|a - b| & Return the subtraction of \param{a} and \param{b}. \\
    \verb| - a| & Return the negation of \param{a}. \\
    \verb|a * b| & Return the product of \param{a} and \param{b}. \\
    \verb|a ** k| & Return the \param{k}-th power of \param{a}. \\
    \verb|a / b| & Return the quotient of \param{a} by \param{b}. \\
  \end{op}
\begin{ex}
>>> a = algfield.MatAlgNumber([1, 2], [-2, 0, 1])
>>> b = algfield.MatAlgNumber([-2, 3], [-2, 0, 1])
>>> print a + b
MatAlgNumber([-1, 5]+[10, -1], [-2, 0, 1])
>>> print a * b
MatAlgNumber([10, -1]+[-2, 10], [-2, 0, 1])
>>> print a ** 3
MatAlgNumber([25L, 22L]+[44L, 25L], [-2, 0, 1])
>>> print a / b
MatAlgNumber([Rational(1, 1), Rational(1, 2)]+
[Rational(1, 1), Rational(1, 1)], [-2, 0, 1])
\end{ex}%Don't indent!
\C
  \method
  \subsubsection{inverse -- inverse}\linkedtwo{algfield}{MatAlgNumber}{inverse}
  \func{inverse}{\param{self}}{\out{MatAlgNumber}}\\
  \spacing
  % document of basic document
  \quad Return the inverse of \param{self}. \\
  %\spacing
  % add document
  %\spacing
  % input, output document
%
  \subsubsection{getRing -- the field}\linkedtwo{algfield}{MatAlgNumber}{getRing}
  \func{getRing}{\param{self}}{\out{NumberField}}\\
  \spacing
  % document of basic document
  \quad Return the field which \param{self} belongs to. \\  
  %\spacing
  % add document
  %\spacing
  % input, output document
%
  \subsubsection{trace -- trace}\linkedtwo{algfield}{MatAlgNumber}{trace}
  \func{trace}{\param{self}}{\out{Rational}}\\
  \spacing
  % document of basic document
  \quad Return the trace of \param{self} in the \param{self}.\linkingtwo{algfield}{MatAlgNumber}{field} over the rational field. \\
  %\spacing
  % add document
  %\spacing
  % input, output document
%
  \subsubsection{norm -- norm}\linkedtwo{algfield}{MatAlgNumber}{norm}
  \func{norm}{\param{self}}{\out{Rational}}\\
  \spacing
  % document of basic document
  \quad Return the norm of \param{self} in the \param{self}.\linkingtwo{algfield}{MatAlgNumber}{field} over the rational field. \\
  %\spacing
  % add document
  %\spacing
  % input, output document
%
  \subsubsection{ch\_basic -- obtain BasicAlgNumber object}\linkedtwo{algfield}{BasicAlgNumber}{ch\_basic}
  \func{ch\_basic}{\param{self}}{\out{BasicAlgNumber}}\\
  \spacing
  % document of basic document
  \quad Return \linkingone{algfield}{BasicAlgNumber} object corresponding to \param{self}.\\
  %\spacing
  % add document
  %\spacing
  % input, output document
%
\begin{ex}
>>> a = algfield.MatAlgNumber([1, -1, 1], [-3, 1, 2, 1])
>>> a.inverse()
MatAlgNumber([Rational(2, 3), Rational(4, 9), Rational(1, 9)]+
[Rational(1, 3), Rational(5, 9), Rational(2, 9)]+
[Rational(2, 3), Rational(1, 9), Rational(1, 9)], [-3, 1, 2, 1])
>>> a.trace()
Rational(7, 1)
>>> a.norm()
Rational(27, 1)
>>> a.getRing()
NumberField([-3, 1, 2, 1])
>>> a.ch_basic()
BasicAlgNumber([[1, -1, 1], 1], [-3, 1, 2, 1])
\end{ex}
\C
  \subsection{changetype(function) -- obtain BasicAlgNumber object}\linkedone{algfield}{changetype}
  \func{changetype}{
    \hiki{a}{integer},\ 
    \hikiopt{polynomial}{list}{[0, 1]}
  }{
    \out{BasicAlgNumber}
  } \\
  \func{changetype}{
    \hiki{a}{Rational},\
    \hikiopt{polynomial}{list}{[0, 1]}
  }{
    \out{BasicAlgNumber}
  } \\
  \func{changetype}{
    \hiki{polynomial}{list}
  }{
    \out{BasicAlgNumber}
  } \\
  \spacing
  % document of basic document
%  \quad Return the value of that changed \param{a} into the element in the field defined by the polynomial \param{polynomial}. \\
  \quad Return a BasicAlgNumber object corresponding to \param{a}.
  \spacing
  % add document
  \spacing
  \quad If \param{a} is an integer or an instance of \linkingone{rational}{Rational}, the function returns \linkingone{algfield}{BasicAlgNumber} object whose field is defined by \param{polynomial}.
  If \param{a} is a list, the function returns \linkingone{algfield}{BasicAlgNumber} corresponding to a solution of \param{a}, considering \param{a} as the polynomial.\\
  \spacing
  % input, output document
  The input parameter \param{a} must be an integer, \linkingone{rational}{Rational} or a list of integers.\\
%
  \subsection{disc(function) -- discriminant}\linkedone{algfield}{disc}
  \func{disc}{\hiki{A}{list}}{\out{Rational}} \\
  \spacing
  % document of basic document
  \quad Return the discriminant of $a_i$, where $\param{A} = [a_1, a_2, \cdots, a_n]$. \\
  \spacing
  % add document
  % \spacing
  % input, output document
  \param{$a_i$} must be an instance of \linkingone{algfield}{BasicAlgNumber} or \linkingone{algfield}{MatAlgNumber} defined over a same number field.\\
%
  \subsection{fppoly(function) -- polynomial over finite prime field}\linkedone{algfield}{fppoly}
  \func{fppoly}{\hiki{coeffs}{list},\ \hiki{p}{integer}}{\out{\linkingone{poly.uniutil}{FinitePrimeFieldPolynomial}}} \\
  \spacing
  % document of basic document
  \quad Return the polynomial whose coefficients \param{coeffs} are defined over the prime field $\mathbb{Z}_\param{p}$. \\
  \spacing
  % add document
  % \spacing
  % input, output document
  \quad \param{coeffs} should be a list of integers or of instances of \linkingone{finitefield}{FinitePrimeFieldElement}.\\
%
%  \subsection{prime\_decomp}\linkedone{algfield}{prime\_decomp}
%  \func{prime\_decpomp}{\hiki{p}{integer}, \, \hiki{f}{list}}{\out{BasicAlgNumber}} \\
%  \spacing
  % document of basic document
%  \quad Return prime decomposition of $(\param{p})$ over $\mathbb{Q}[x]/ (\param{f})$ . \\
%  \spacing
  % add document
  % \spacing
  % input, output document
%  \quad If output is $((\alpha, \beta), (\alpha, \gamma))$, this means $\param{p} = (\alpha, \beta)(\alpha, \gamma)$. \\
%
  \subsection{qpoly(function) -- polynomial over rational field}\linkedone{algfield}{qpoly}
  \func{qpoly}{\hiki{coeffs}{list}}{\out{\linkingone{poly.uniutil}{FieldPolynomial}}} \\
  \spacing
  % document of basic document
  \quad Return the polynomial whose coefficients \param{coeffs} are defined over the rational field. \\
  \spacing
  % add document
  % \spacing
  % input, output document
  \quad \param{coeffs} must be a list of integers or instances of \linkingone{rational}{Rational}.\\
%
  \subsection{zpoly(function) -- polynomial over integer ring}\linkedone{algfield}{zpoly}
  \func{zpoly}{\hiki{coeffs}{list}}{\out{\linkingone{poly.uniutil}{IntegerPolynomial}}} \\
  \spacing
  % document of basic document
  \quad Return the polynomial whose coefficients \param{coeffs} are defined over the (rational) integer ring.  \\
  \spacing
  % add document
  % \spacing
  % input, output document
  \quad \param{coeffs} must be a list of integers.\\
%
\begin{ex}
>>> a = algfield.changetype(3, [-2, 0, 1])
>>> b = algfield.BasicAlgNumber([[1, 2], 1], [-2, 0, 1])
>>> A = [a, b]
>>> algfield.disc(A)
288L
\end{ex}%Don't indent!(indent causes an error.)
%>>> algfield.prime_decomp(5, [-3, 1, 2, 1])
%[((BasicAlgNumber([[5, 0, 0], 1], [-3, 1, 2, 1]), 
%BasicAlgNumber([[3L, 1L, 0], 1], [-3, 1, 2, 1])), 1), 
%((BasicAlgNumber([[5, 0, 0], 1], [-3, 1, 2, 1]), 
%BasicAlgNumber([[2L, 1L, 0], 1], [-3, 1, 2, 1])), 2)]
\C

%---------- end document ---------- %

\bibliographystyle{jplain}%use jbibtex
\bibliography{nzmath_references}

\end{document}


%\documentclass{report}

%%%%%%%%%%%%%%%%%%%%%%%%%%%%%%%%%%%%%%%%%%%%%%%%%%%%%%%%%%%%%
%
% macros for nzmath manual
%
%%%%%%%%%%%%%%%%%%%%%%%%%%%%%%%%%%%%%%%%%%%%%%%%%%%%%%%%%%%%%
\usepackage{amssymb,amsmath}
\usepackage{color}
\usepackage[dvipdfm,bookmarks=true,bookmarksnumbered=true,%
 pdftitle={NZMATH Users Manual},%
 pdfsubject={Manual for NZMATH Users},%
 pdfauthor={NZMATH Development Group},%
 pdfkeywords={TeX; dvipdfmx; hyperref; color;},%
 colorlinks=true]{hyperref}
\usepackage{fancybox}
\usepackage[T1]{fontenc}
%
\newcommand{\DS}{\displaystyle}
\newcommand{\C}{\clearpage}
\newcommand{\NO}{\noindent}
\newcommand{\negok}{$\dagger$}
\newcommand{\spacing}{\vspace{1pt}\\ }
% software macros
\newcommand{\nzmathzero}{{\footnotesize $\mathbb{N}\mathbb{Z}$}\texttt{MATH}}
\newcommand{\nzmath}{{\nzmathzero}\ }
\newcommand{\pythonzero}{$\mbox{\texttt{Python}}$}
\newcommand{\python}{{\pythonzero}\ }
% link macros
\newcommand{\linkingzero}[1]{{\bf \hyperlink{#1}{#1}}}%module
\newcommand{\linkingone}[2]{{\bf \hyperlink{#1.#2}{#2}}}%module,class/function etc.
\newcommand{\linkingtwo}[3]{{\bf \hyperlink{#1.#2.#3}{#3}}}%module,class,method
\newcommand{\linkedzero}[1]{\hypertarget{#1}{}}
\newcommand{\linkedone}[2]{\hypertarget{#1.#2}{}}
\newcommand{\linkedtwo}[3]{\hypertarget{#1.#2.#3}{}}
\newcommand{\linktutorial}[1]{\href{http://docs.python.org/tutorial/#1}{#1}}
\newcommand{\linktutorialone}[2]{\href{http://docs.python.org/tutorial/#1}{#2}}
\newcommand{\linklibrary}[1]{\href{http://docs.python.org/library/#1}{#1}}
\newcommand{\linklibraryone}[2]{\href{http://docs.python.org/library/#1}{#2}}
\newcommand{\pythonhp}{\href{http://www.python.org/}{\python website}}
\newcommand{\nzmathwiki}{\href{http://nzmath.sourceforge.net/wiki/}{{\nzmathzero}Wiki}}
\newcommand{\nzmathsf}{\href{http://sourceforge.net/projects/nzmath/}{\nzmath Project Page}}
\newcommand{\nzmathtnt}{\href{http://tnt.math.se.tmu.ac.jp/nzmath/}{\nzmath Project Official Page}}
% parameter name
\newcommand{\param}[1]{{\tt #1}}
% function macros
\newcommand{\hiki}[2]{{\tt #1}:\ {\em #2}}
\newcommand{\hikiopt}[3]{{\tt #1}:\ {\em #2}=#3}

\newdimen\hoge
\newdimen\truetextwidth
\newcommand{\func}[3]{%
\setbox0\hbox{#1(#2)}
\hoge=\wd0
\truetextwidth=\textwidth
\advance \truetextwidth by -2\oddsidemargin
\ifdim\hoge<\truetextwidth % short form
{\bf \colorbox{skyyellow}{#1(#2)\ $\to$ #3}}
%
\else % long form
\fcolorbox{skyyellow}{skyyellow}{%
   \begin{minipage}{\textwidth}%
   {\bf #1(#2)\\ %
    \qquad\quad   $\to$\ #3}%
   \end{minipage}%
   }%
\fi%
}

\newcommand{\out}[1]{{\em #1}}
\newcommand{\initialize}{%
  \paragraph{\large \colorbox{skyblue}{Initialize (Constructor)}}%
    \quad\\ %
    \vspace{3pt}\\
}
\newcommand{\method}{\C \paragraph{\large \colorbox{skyblue}{Methods}}}
% Attribute environment
\newenvironment{at}
{%begin
\paragraph{\large \colorbox{skyblue}{Attribute}}
\quad\\
\begin{description}
}%
{%end
\end{description}
}
% Operation environment
\newenvironment{op}
{%begin
\paragraph{\large \colorbox{skyblue}{Operations}}
\quad\\
\begin{table}[h]
\begin{center}
\begin{tabular}{|l|l|}
\hline
operator & explanation\\
\hline
}%
{%end
\hline
\end{tabular}
\end{center}
\end{table}
}
% Examples environment
\newenvironment{ex}%
{%begin
\paragraph{\large \colorbox{skyblue}{Examples}}
\VerbatimEnvironment
\renewcommand{\EveryVerbatim}{\fontencoding{OT1}\selectfont}
\begin{quote}
\begin{Verbatim}
}%
{%end
\end{Verbatim}
\end{quote}
}
%
\definecolor{skyblue}{cmyk}{0.2, 0, 0.1, 0}
\definecolor{skyyellow}{cmyk}{0.1, 0.1, 0.5, 0}
%
%\title{NZMATH User Manual\\ {\large{(for version 1.0)}}}
%\date{}
%\author{}
\begin{document}
%\maketitle
%
\setcounter{tocdepth}{3}
\setcounter{secnumdepth}{3}


\tableofcontents
\C

\chapter{Classes}

%---------- start document ---------- %
 \section{elliptic -- elliptic class object}\linkedzero{elliptic}
 \begin{itemize}
   \item {\bf Classes}
   \begin{itemize}
     \item \linkingone{elliptic}{ECGeneric}
     \item \linkingone{elliptic}{ECoverQ}
     \item \linkingone{elliptic}{ECoverGF}
   \end{itemize}
   \item {\bf Functions}
     \begin{itemize}
       \item \linkingone{elliptic}{EC}
     \end{itemize}
 \end{itemize}

 This module using following type:
 \begin{description}
   \item[weierstrassform]\linkedone{elliptic}{weierstrassform}:\\
     \param{weierstrassform} is a list ($a_1, a_2, a_3, a_4, a_6$) or ($a_4, a_6$), it represents $E:y^2+a_1xy+a_3y=x^3+a_2x^2+a_4x+a_6$ or $E:y^2=x^3+a_4x+a_6$, respectively.
   \item[infpoint]\linkedone{elliptic}{infpoint}:\\
     \param{infpoint} is the list {\tt [0]}, which represents infinite point on the elliptic curve.
   \item[point]\linkedone{elliptic}{point}:\\
     \param{point} is two-dimensional coordinate list [\param{x}, \param{y}] or \linkingone{elliptic}{infpoint}.
 \end{description}

\C

 \subsection{\negok ECGeneric -- generic elliptic curve class}\linkedone{elliptic}{ECGeneric}
\initialize
  \func{ECGeneric}{
    \hiki{coefficient}{\linkingone{elliptic}{weierstrassform}},\
    \hikiopt{basefield}{Field}{None}
  }{
    \out{ECGeneric}
  }\\
  \spacing
  % document of basic document
  \quad Create an elliptic curve object.\\
  \spacing
  % added document
  \quad The class is for the definition of elliptic curves over general fields.
  Instead of using this class directly, we recommend that you call \linkingone{elliptic}{EC}.\\
  \negok The class precomputes the following values.
  \begin{itemize}
    \item shorter form:\ $y^2=b_2 x^3+b_4x^2+b_6x+b_8$
    \item shortest form:\ $y^2=x^3+c_4x+c_6$
    \item discriminant
    \item j-invariant
  \end{itemize}
  \quad\\
  % input, output document
  \quad All elements of \param{coefficient} must be in \param{basefield}.\\
  See \linkingone{elliptic}{weierstrassform} for more information about \param{coefficient}.
  If discriminant of \param{self} equals $0$, it raises ValueError.
  \begin{at}
    \item[basefield]\linkedtwo{elliptic}{ECGeneric}{basefield}:\\ It expresses the field which each coordinate of all points in \param{self} is on.
(This means not only \param{self} is defined over \param{basefield}.)
    \item[ch]\linkedtwo{elliptic}{ECGeneric}{ch}:\\ It expresses the characteristic of \param{basefield}.
    \item[infpoint]\linkedtwo{elliptic}{ECGeneric}{infpoint}:\\ It expresses infinity point (i.e. [0]).
    \item[a1, a2, a3, a4, a6]\linkedtwo{elliptic}{ECGeneric}{a1, a2, a3, a4, a6}:\\ It expresses the coefficients \param{a1}, \param{a2}, \param{a3}, \param{a4}, \param{a6}.
    \item[b2, b4, b6, b8]\linkedtwo{elliptic}{ECGeneric}{b2, b4, b6, b8}:\\ It expresses the coefficients \param{b2}, \param{b4}, \param{b6}, \param{b8}.
    \item[c4, c6]\linkedtwo{elliptic}{ECGeneric}{c4, c6}:\\ It expresses the coefficients \param{c4}, \param{c6}.
    \item[disc]\linkedtwo{elliptic}{ECGeneric}{disc}:\\ It expresses the discriminant of \param{self}.
    \item[j]\linkedtwo{elliptic}{ECGeneric}{j}:\\ It expresses the j-invariant of \param{self}.
    \item[coefficient]\linkedtwo{elliptic}{ECGeneric}{coefficient}:\\ It expresses the \linkingone{elliptic}{weierstrassform} of \param{self}.
\end{at}
  %\begin{op}
  %\end{op}
  \method
  \subsubsection{simple -- simplify the curve coefficient}\linkedtwo{elliptic}{ECGeneric}{simple}
   \func{simple}{\param{self}}{\out{\linkingone{elliptic}{ECGeneric}}}\\
   \spacing
   % document of basic document
   \quad Return elliptic curve corresponding to the short Weierstrass form of \param{self} by changing the coordinates.\\
   %\spacing
   % added document
   %\quad \negok Note that this function returns integer only.\\
   %\spacing
   % input, output document
   %\quad \param{a} must be int, long or rational.Integer.\\
%
  \subsubsection{changeCurve -- change the curve by coordinate change}\linkedtwo{elliptic}{ECGeneric}{changeCurve}
   \func{changeCurve}{\param{self},\ \hiki{V}{list}}{\out{\linkingone{elliptic}{ECGeneric}}}\\
   \spacing
   % document of basic document
   \quad Return elliptic curve corresponding to the curve obtained by some coordinate change $x=u^2x'+r,\ y=u^3y'+su^2x'+t$.\\
   \spacing
   % added document
   \quad For $u \ne 0$, the coordinate change gives some curve which is \linkingtwo{elliptic}{ECGeneric}{basefield}-isomorphic to \param{self}.\\
   \spacing
   % input, output document
   \quad $V$ must be a list of the form $[u,r,s,t]$, where $u,r,s,t$ are in \linkingtwo{elliptic}{ECGeneric}{basefield}.\\
%
  \subsubsection{changePoint -- change coordinate of point on the curve}\linkedtwo{elliptic}{ECGeneric}{changePoint}
   \func{changePoint}{\param{self},\ \hiki{P}{\linkingone{elliptic}{point}},\ \hiki{V}{list}}{\out{\linkingone{elliptic}{point}}}\\
   \spacing
   % document of basic document
   \quad Return the point corresponding to the point obtained by the coordinate change $x'=(x-r) u^{-2},\ y'=(y-s(x-r)+t)u^{-3}$.\\
   \spacing
   % added document
   \quad Note that the inverse coordinate change is $x=u^2x'+r,\ y=u^3y'+su^2x'+t$.See \linkingtwo{elliptic}{ECGeneric}{changeCurve}.\\
   \spacing
   % input, output document
   \quad $V$ must be a list of the form $[u,r,s,t]$, where $u,r,s,t$ are in \linkingtwo{elliptic}{ECGeneric}{basefield}.$u$ must be non-zero.\\
%
  \subsubsection{coordinateY -- Y-coordinate from X-coordinate}\linkedtwo{elliptic}{ECGeneric}{coordinateY}
   \func{coordinateY}{\param{self},\ \hiki{x}{\linkingone{ring}{FieldElement}}}{\out{\linkingone{ring}{FieldElement} / False}}\\
   \spacing
   % document of basic document
   \quad Return Y-coordinate of the point on \param{self}  whose X-coordinate is \param{x}.\\
   \spacing
   % added document
   %\quad \negok Note that this function returns integer only.\\
   %\spacing
   % input, output document
   \quad The output would be one Y-coordinate (if a coordinate is found).
   If such a Y-coordinate does not exist,  it returns False.\\
%
  \subsubsection{whetherOn -- Check point is on curve }\linkedtwo{elliptic}{ECGeneric}{whetherOn}
   \func{whetherOn}{\param{self},\ \hiki{P}{\linkingone{elliptic}{point}}}{\out{bool}}\\
   \spacing
   % document of basic document
   \quad Check whether the point \param{P} is on \param{self} or not.\\
   \spacing
   % added document
   %\quad \negok Note that this function returns integer only.\\
   %\spacing
   % input, output document
   %\quad Output \param{Y-coordinate} is either one Y-coordinate or 2-component vector containing the Y-coordinates. 
%
  \subsubsection{add -- Point addition on the curve}\linkedtwo{elliptic}{ECGeneric}{add}
   \func{add}{\param{self},\ \hiki{P}{\linkingone{elliptic}{point}},\ \hiki{Q}{\linkingone{elliptic}{point}}}{\out{\linkingone{elliptic}{point}}}\\
   \spacing
   % document of basic document
   \quad Return the sum of the point \param{P} and \param{Q} on \param{self}.\\
   \spacing
   % added document
   %\quad \negok Note that this function returns integer only.\\
   %\spacing
   % input, output document
   %\quad Output \param{Y-coordinate} is either one Y-coordinate or 2-component vector containing the Y-coordinates. 
%
  \subsubsection{sub -- Point subtraction on the curve}\linkedtwo{elliptic}{ECGeneric}{sub}
   \func{sub}{\param{self},\ \hiki{P}{\linkingone{elliptic}{point}},\ \hiki{Q}{\linkingone{elliptic}{point}}}{\out{\linkingone{elliptic}{point}}}\\
   \spacing
   % document of basic document
   \quad Return the subtraction of the point \param{P} from \param{Q} on \param{self}.\\
   \spacing
   % added document
   %\quad \negok Note that this function returns integer only.\\
   %\spacing
   % input, output document
   %\quad Output \param{Y-coordinate} is either one Y-coordinate or 2-component vector containing the Y-coordinates. 
%
  \subsubsection{mul -- Scalar point multiplication on the curve}\linkedtwo{elliptic}{ECGeneric}{mul}
   \func{mul}{\param{self},\ \hiki{k}{integer},\ \hiki{P}{\linkingone{elliptic}{point}}}{\out{\linkingone{elliptic}{point}}}\\
   \spacing
   % document of basic document
   \quad Return the scalar multiplication of the point \param{P} by a scalar \param{k} on \param{self}.\\
   \spacing
   % added document
   %\quad \negok Note that this function returns integer only.\\
   %\spacing
   % input, output document
   %\quad Output \param{Y-coordinate} is either one Y-coordinate or 2-component vector containing the Y-coordinates. 
%
  \subsubsection{divPoly -- division polynomial}\linkedtwo{elliptic}{ECGeneric}{divPoly}
   \func{divPoly}{\param{self},\ \hikiopt{m}{integer}{None}}{\out{\linkingone{poly.uniutil}{FieldPolynomial}/(\hiki{f}{list}, \hiki{H}{integer})}}\\
   \spacing
   % document of basic document
   \quad Return the division polynomial.\\
   \spacing
   % added document
   %\quad \negok Note that this function returns integer only.\\
   %\spacing
   % input, output document
   \quad  If \param{m} is odd, this method returns the usual division polynomial. If \param{m} is even, return the quotient of the usual division polynomial by $2y + a_1 x + a_3$.\\
 \negok If \param{m} is not specified (i.e. \param{m=None}), then return $(\param{f},\ \param{H})$.
 $\param{H}$ is the least prime satisfying $\prod_{2\le l \le H,\ l:prime} l > 4\sqrt{q}$, where $q$ is the order of \linkingtwo{elliptic}{ECGeneric}{basefield}.
 $\param{f}$ is the list of $k$-division polynomials up to $k \le \param{H}$.
 These are used for Schoof's algorithm.\\
%
\C
 \subsection{ECoverQ -- elliptic curve over rational field}\linkedone{elliptic}{ECoverQ}
 The class is for elliptic curves over the rational field $\mathbb{Q}$ (\linkingone{rational}{RationalField} in nzmath.\linkingzero{rational}).\\
 The class is a subclass of \linkingone{elliptic}{ECGeneric}.

\initialize
  \func{ECoverQ}{\hiki{coefficient}{\linkingone{elliptic}{weierstrassform}}}{\out{\linkingone{elliptic}{ECoverQ}}}\\
  \spacing
  % document of basic document
  \quad Create elliptic curve over the rational field.\\
  \spacing
  % added document
  %
  % \spacing
  % input, output document
  \quad All elements of \param{coefficient} must be integer or \linkingone{rational}{Rational}. \\
  See \linkingone{elliptic}{weierstrassform} for more information about \param{coefficient}.\\
\begin{ex}
>>> E = elliptic.ECoverQ([ratinal.Rational(1, 2), 3])
>>> print E.disc
-3896/1
>>> print E.j
1728/487
\end{ex}%Don't indent!
  \method
  \subsubsection{point -- obtain random point on curve}\linkedtwo{elliptic}{ECoverQ}{point}
   \func{point}{\param{self},\ \hikiopt{limit}{integer}{1000}}{\out{\linkingone{elliptic}{point}}}\\
   \spacing
   % document of basic document
   \quad Return a random point on \param{self}.\\
   \spacing
   % added document
   \quad  \param{limit} expresses the time of trying to choose points.
   If failed, raise ValueError.
   \negok Because it is difficult to search the rational point over the rational field, it might raise error with high frequency.\\
   %\spacing
   % input, output document
   %\quad \param{a} must be int, long or rational.Integer.\\
%
\begin{ex}
>>> print E.changeCurve([1, 2, 3, 4])
y ** 2 + 6/1 * x * y + 8/1 * y = x ** 3 - 3/1 * x ** 2 - 23/2 * x - 4/1
>>> E.divPoly(3)
FieldPolynomial([(0, Rational(-1, 4)), (1, Rational(36, 1)), (2, Rational(3, 1)
), (4, Rational(3, 1))], RationalField())
\end{ex}%Don't indent!
\C
 \subsection{ECoverGF -- elliptic curve over finite field}\linkedone{elliptic}{ECoverGF}
 The class is for elliptic curves over a finite field, denoted by $\mathbb{F}_q$ (\linkingone{finitefield}{FiniteField} and its subclasses in nzmath).\\
 The class is a subclass of \linkingone{elliptic}{ECGeneric}.
\initialize
  \func{ECoverGF}{
    \hiki{coefficient}{\linkingone{elliptic}{weierstrassform}},\ 
    \hiki{basefield}{\linkingone{finitefield}{FiniteField}}
    }{
    \out{\linkingone{elliptic}{ECoverGF}}
  }\\
  \spacing
  % document of basic document
  \quad Create elliptic curve over a finite field.\\
  \spacing
  % added document
  %
  % \spacing
  % input, output document
  \quad All elements of \param{coefficient} must be in \param{basefield}.
  \param{basefield} should be an instance of \linkingone{finitefield}{FiniteField}.\\
  See \linkingone{elliptic}{weierstrassform} for more information about \param{coefficient}.
\begin{ex}
>>> E = elliptic.ECoverGF([2, 5], finitefield.FinitePrimeField(11))
>>> print E.j
7 in F_11
>>> E.whetherOn([8, 4])
True
>>> E.add([3, 4], [9, 9])
[FinitePrimeFieldElement(0, 11), FinitePrimeFieldElement(4, 11)]
>>> E.mul(5, [9, 9])
[FinitePrimeFieldElement(0, 11)]
\end{ex}%Don't indent!
  \method
  \subsubsection{point -- find random point on curve}\linkedtwo{elliptic}{ECoverGF}{point}
   \func{point}{\param{self}}{\out{\linkingone{elliptic}{point}}}\\
   \spacing
   % document of basic document
   \quad Return a random point on \param{self}.\\
   \spacing
   % added document
   \quad This method uses a probabilistic algorithm.\\
   %\spacing
   % input, output document
   %\quad \param{a} must be int, long or rational.Integer.\\
%
  \subsubsection{naive -- Frobenius trace by naive method}\linkedtwo{elliptic}{ECoverGF}{naive}
   \func{naive}{\param{self}}{\out{integer}}\\
   \spacing
   % document of basic document
   \quad Return Frobenius trace $t$ by a naive method.\\
   \spacing
   % added document
   \quad \negok The function counts up the Legendre symbols of all rational points on \param{self}.\\
   Frobenius trace of the curve is $t$ such that $\#E(\mathbb{F}_q)=q+1-t$, where $\#E(\mathbb{F}_q)$ stands for the number of points on \param{self} over \param{self}.\linkingtwo{elliptic}{ECGeneric}{basefield} $\mathbb{F}_q$. \\
   \spacing
   % input, output document
   \quad The characteristic of \param{self}.\linkingtwo{elliptic}{ECGeneric}{basefield} must not be $2$ nor $3$.\\
%
  \subsubsection{Shanks\_Mestre -- Frobenius trace by Shanks and Mestre method}\linkedtwo{elliptic}{ECoverGF}{Shanks\_Mestre}
   \func{Shanks\_Mestre}{\param{self}}{\out{integer}}\\
   \spacing
   % document of basic document
   \quad Return Frobenius trace $t$ by Shanks and Mestre method.\\
   \spacing
   % added document
   \quad \negok This uses the method proposed by Shanks and Mestre.
   \negok See Algorithm 7.5.3 of \cite{Pomerance} for more information about the algorithm.\\
   Frobenius trace of the curve is $t$ such that $\#E(\mathbb{F}_q)=q+1-t$, where $\#E(\mathbb{F}_q)$ stands for the number of points on \param{self} over \param{self}.\linkingtwo{elliptic}{ECGeneric}{basefield} $\mathbb{F}_q$. \\
   \spacing
   % input, output document
   \quad \param{self}.\linkingtwo{elliptic}{ECGeneric}{basefield} must be an instance of \linkingone{finitefield}{FinitePrimeField}.\\
%
  \subsubsection{Schoof -- Frobenius trace by Schoof's method}\linkedtwo{elliptic}{ECoverGF}{Schoof}
   \func{Schoof}{\param{self}}{\out{integer}}\\
   \spacing
   % document of basic document
   \quad Return Frobenius trace $t$ by Schoof's method.\\
   \spacing
   % added document
   \quad \negok This uses the method proposed by Schoof.\\
   Frobenius trace of the curve is $t$ such that $\#E(\mathbb{F}_q)=q+1-t$, where $\#E(\mathbb{F}_q)$ stands for the number of points on \param{self} over \param{self}.\linkingtwo{elliptic}{ECGeneric}{basefield} $\mathbb{F}_q$. \\
   \spacing
   % input, output document
%
  \subsubsection{trace -- Frobenius trace}\linkedtwo{elliptic}{ECoverGF}{trace}
   \func{trace}{\param{self},\ \hikiopt{r}{integer}{None}}{\out{integer}}\\
   \spacing
   % document of basic document
   \quad Return Frobenius trace $t$.\\
   \spacing
   % added document
   \quad Frobenius trace of the curve is $t$ such that $\#E(\mathbb{F}_q)=q+1-t$, where $\#E(\mathbb{F}_q)$ stands for the number of points on \param{self} over \param{self}.\linkingtwo{elliptic}{ECGeneric}{basefield} $\mathbb{F}_q$. \\
   If positive $r$ given, it returns $q^r+1-\#E(\mathbb{F}_{q^r})$.\\
   \negok The method selects algorithms by investigating \param{self}.\linkingtwo{elliptic}{ECGeneric}{ch} when \param{self}.\linkingtwo{elliptic}{ECGeneric}{basefield} is an instance of \linkingone{finitefield}{
FinitePrimeField}.
   If \param{ch}<1000, the method uses \linkingtwo{elliptic}{ECoverGF}{naive}. 
   If $10^4<\param{ch}<10^{30}$, the method uses \linkingtwo{elliptic}{ECoverGF}{Shanks\_Mestre}. 
   Otherwise, it uses \linkingtwo{elliptic}{ECoverGF}{Schoof}.\\
   \spacing
   % input, output document
   \quad The parameter $r$ must be positive integer.\\
%
  \subsubsection{order -- order of group of rational points on the curve}\linkedtwo{elliptic}{ECoverGF}{order}
   \func{order}{\param{self},\ \hikiopt{r}{integer}{None}}{\out{integer}}\\
   \spacing
   % document of basic document
   \quad Return order $\#E(\mathbb{F}_q)=q+1-t$. \\
   \spacing
   % added document
   \quad If positive $r$ given, this computes $\#E({\mathbb{F}_q}^r)$ instead. \\
   \negok On the computation of Frobenius trace $t$, the method calls \linkingtwo{elliptic}{ECoverGF}{trace}.\\
   \spacing
   % input, output document
   \quad The parameter $r$ must be positive integer.\\
%
  \subsubsection{pointorder -- order of point on the curve}\linkedtwo{elliptic}{ECoverGF}{pointorder}
   \func{pointorder}{\param{self},\ \hiki{P}{\linkingone{elliptic}{point}},\ \hikiopt{ord\_factor}{list}{None}}{\out{integer}}\\
   \spacing
   % document of basic document
   \quad Return order of a point \param{P}. \\
   \spacing
   % added document
   \quad \negok The method uses factorization of \linkingtwo{elliptic}{ECoverGF}{order}.\\
   If \param{ord\_factor} is given, computation of factorizing the order of \param{self} is omitted and it applies \param{ord\_factor} instead.\\
   \spacing
   % input, output document
   %\quad 
%
  \subsubsection{TatePairing -- Tate Pairing}\linkedtwo{elliptic}{ECoverGF}{TatePairing}
   \func{TatePairing}
        {\param{self},\ 
          \hiki{m}{integer},\ 
          \hiki{P}{\linkingone{elliptic}{point}},\ 
          \hiki{Q}{\linkingone{elliptic}{point}} 
        }
        {\out{\linkingone{finitefield}{FiniteFieldElement}}}\\
   \spacing
   % document of basic document
   \quad Return Tate-Lichetenbaum pairing ${\langle P,\ Q\rangle}_m$.\\
   \spacing
   % added document
   \quad \negok The method uses Miller's algorithm.\\
   The image of the Tate pairing is $\mathbb{F}_q^*/{\mathbb{F}_q^*}^m$, but the method returns an element of $\mathbb{F}_q$, so the value is not uniquely defined.
   If uniqueness is needed, use \linkingtwo{elliptic}{ECoverGF}{TatePairing\_Extend}.\\
   \spacing
   % input, output document
   \quad The point \param{P} has to be a \param{m}-torsion point (i.e. $mP=$[0]).
   Also, the number \param{m} must divide \linkingtwo{elliptic}{ECoverGF}{order}.\\
%
  \subsubsection{TatePairing\_Extend -- Tate Pairing with final exponentiation}\linkedtwo{elliptic}{ECoverGF}{TatePairing\_Extend}
   \func{TatePairing\_Extend}
        {\param{self},\ 
          \hiki{m}{integer},\ 
          \hiki{P}{\linkingone{elliptic}{point}},\ 
          \hiki{Q}{\linkingone{elliptic}{point}} 
        }
        {\out{\linkingone{finitefield}{FiniteFieldElement}}}\\
   \spacing
   % document of basic document
   \quad Return Tate Pairing with final exponentiation, i.e. ${{\langle P,\ Q\rangle}_m}^{(q-1)/m}$.\\
   \spacing
   % added document
   \quad \negok The method calls \linkingtwo{elliptic}{ECoverGF}{TatePairing}.\\
   \spacing
   % input, output document
   \quad The point \param{P} has to be a \param{m}-torsion point (i.e. $mP=$[0]). Also the number \param{m} must divide \linkingtwo{elliptic}{ECoverGF}{order}.\\
   The output is in the group generated by $m$-th root of unity in ${\mathbb{F}_q^*}$.
%
  \subsubsection{WeilPairing -- Weil Pairing }\linkedtwo{elliptic}{ECoverGF}{WeilPairing}
   \func{WeilPairing}
        {\param{self},\ 
          \hiki{m}{integer},\ 
          \hiki{P}{\linkingone{elliptic}{point}},\ 
          \hiki{Q}{\linkingone{elliptic}{point}} 
        }
        {\out{\linkingone{finitefield}{FiniteFieldElement}}}\\
   \spacing
   % document of basic document
   \quad Return Weil pairing  $e_m(P,\ Q)$.\\
   \spacing
   % added document
   \quad \negok The method uses Miller's algorithm.\\
   \spacing
   % input, output document
   \quad The points \param{P} and \param{Q} has to be a \param{m}-torsion point (i.e. $mP=mQ=$[0]).
   Also, the number \param{m} must divide \linkingtwo{elliptic}{ECoverGF}{order}.

   The output is in the group generated by $m$-th root of unity in ${\mathbb{F}_q^*}$.
%
  \subsubsection{BSGS -- point order by Baby-Step and Giant-Step}\linkedtwo{elliptic}{ECoverGF}{BSGS}
   \func{BSGS}
        {\param{self},\ 
          \hiki{P}{\linkingone{elliptic}{point}}
        }
        {\out{integer}}\\
   \spacing
   % document of basic document
   \quad Return order of point $P$ by Baby-Step and Giant-Step method.\\
   \spacing
   % added document
   \quad \negok See \cite{Washington} for more information about the algorithm.\\
   %\spacing
   % input, output document
   %\quad
%
  \subsubsection{DLP\_BSGS -- solve Discrete Logarithm Problem by Baby-Step and Giant-Step}\linkedtwo{elliptic}{ECoverGF}{DLP\_BSGS}
   \func{DLP\_BSGS}
        {\param{self},\ 
          \hiki{n}{integer},\ 
          \hiki{P}{\linkingone{elliptic}{point}},\ 
          \hiki{Q}{\linkingone{elliptic}{point}}
        }
        {\out{\hiki{m}{integer}}}\\
   \spacing
   % document of basic document
   \quad Return \param{m} such that $Q=mP$ by Baby-Step and Giant-Step method.\\
   \spacing
   % added document
   %\spacing
   % input, output document
   \quad The points \param{P} and \param{Q} has to be a \param{n}-torsion point (i.e. $nP=nQ=$[0]). Also, the number \param{n} must divide \linkingtwo{elliptic}{ECoverGF}{order}.\\
   The output \param{m} is an integer.\\
%
  \subsubsection{structure -- structure of group of rational points}\linkedtwo{elliptic}{ECoverGF}{structure}
   \func{structure}
        {\param{self}}
        {\out{\hiki{structure}{tuple}}}\\
   \spacing
   % document of basic document
   \quad Return the group structure of \param{self}.\\
   \spacing
   % added document
   \quad  The structure of $E(\mathbb{F}_q)$ is represented as $\mathbb{Z}/d\mathbb{Z} \times \mathbb{Z}/n\mathbb{Z}$.
   The method uses \linkingtwo{elliptic}{ECoverGF}{WeilPairing}.\\
   \spacing
   % input, output document
   \quad The output \param{structure} is a tuple of positive two integers \param{(d,\  n)}. \param{d} divides \param{n}.
%
  \subsubsection{issupersingular -- check supersingular curve}\linkedtwo{elliptic}{ECoverGF}{issupersingular}
   \func{structure}
        {\param{self}}
        {\out{bool}}\\
   \spacing
   % document of basic document
   \quad Check whether \param{self} is a supersingular curve or not.\\
   \spacing
   % added document
   %\quad  
   %\spacing
   % input, output document
   %\quad
%
\begin{ex}
>>> E=nzmath.elliptic.ECoverGF([2, 5], nzmath.finitefield.FinitePrimeField(11))
>>> E.whetherOn([0, 4])
True
>>> print E.coordinateY(3)
4 in F_11
>>> E.trace()
2
>>> E.order()
10
>>> E.pointorder([3, 4])
10L
>>> E.TatePairing(10, [3, 4], [9, 9])
FinitePrimeFieldElement(3, 11)
>>> E.DLP_BSGS(10, [3, 4], [9, 9])
6
\end{ex}%Don't indent!
\C
  \subsection{EC(function)}\linkedone{elliptic}{EC}
  \func{EC}
       {\hiki{coefficient}{\linkingone{elliptic}{weierstrassform}},\ 
        \hiki{basefield}{\linkingone{ring}{Field}}}
       {\out{\linkingone{elliptic}{ECGeneric}}}\\
   \spacing
   \quad Create an elliptic curve object.\\
   \spacing
   % added document
   % \quad \negok
   % \spacing
   % input, output document
   \quad All elements of \param{coefficient} must be in \param{basefield}.\\
   \param{basefield} must be \linkingone{rational}{RationalField} or \linkingone{finitefield}{FiniteField} or their subclasses. 
  See also \linkingone{elliptic}{weierstrassform} for \param{coefficient}.
%\begin{ex}
%>>> module.hogefunction()
%(0, 0)
%>>>
%\end{ex}%Don't indent!
\C

%---------- end document ---------- %

\bibliographystyle{jplain}
\bibliography{nzmath_references}

\end{document}

%\documentclass{report}

%%%%%%%%%%%%%%%%%%%%%%%%%%%%%%%%%%%%%%%%%%%%%%%%%%%%%%%%%%%%%
%
% macros for nzmath manual
%
%%%%%%%%%%%%%%%%%%%%%%%%%%%%%%%%%%%%%%%%%%%%%%%%%%%%%%%%%%%%%
\usepackage{amssymb,amsmath}
\usepackage{color}
\usepackage[dvipdfm,bookmarks=true,bookmarksnumbered=true,%
 pdftitle={NZMATH Users Manual},%
 pdfsubject={Manual for NZMATH Users},%
 pdfauthor={NZMATH Development Group},%
 pdfkeywords={TeX; dvipdfmx; hyperref; color;},%
 colorlinks=true]{hyperref}
\usepackage{fancybox}
\usepackage[T1]{fontenc}
%
\newcommand{\DS}{\displaystyle}
\newcommand{\C}{\clearpage}
\newcommand{\NO}{\noindent}
\newcommand{\negok}{$\dagger$}
\newcommand{\spacing}{\vspace{1pt}\\ }
% software macros
\newcommand{\nzmathzero}{{\footnotesize $\mathbb{N}\mathbb{Z}$}\texttt{MATH}}
\newcommand{\nzmath}{{\nzmathzero}\ }
\newcommand{\pythonzero}{$\mbox{\texttt{Python}}$}
\newcommand{\python}{{\pythonzero}\ }
% link macros
\newcommand{\linkingzero}[1]{{\bf \hyperlink{#1}{#1}}}%module
\newcommand{\linkingone}[2]{{\bf \hyperlink{#1.#2}{#2}}}%module,class/function etc.
\newcommand{\linkingtwo}[3]{{\bf \hyperlink{#1.#2.#3}{#3}}}%module,class,method
\newcommand{\linkedzero}[1]{\hypertarget{#1}{}}
\newcommand{\linkedone}[2]{\hypertarget{#1.#2}{}}
\newcommand{\linkedtwo}[3]{\hypertarget{#1.#2.#3}{}}
\newcommand{\linktutorial}[1]{\href{http://docs.python.org/tutorial/#1}{#1}}
\newcommand{\linktutorialone}[2]{\href{http://docs.python.org/tutorial/#1}{#2}}
\newcommand{\linklibrary}[1]{\href{http://docs.python.org/library/#1}{#1}}
\newcommand{\linklibraryone}[2]{\href{http://docs.python.org/library/#1}{#2}}
\newcommand{\pythonhp}{\href{http://www.python.org/}{\python website}}
\newcommand{\nzmathwiki}{\href{http://nzmath.sourceforge.net/wiki/}{{\nzmathzero}Wiki}}
\newcommand{\nzmathsf}{\href{http://sourceforge.net/projects/nzmath/}{\nzmath Project Page}}
\newcommand{\nzmathtnt}{\href{http://tnt.math.se.tmu.ac.jp/nzmath/}{\nzmath Project Official Page}}
% parameter name
\newcommand{\param}[1]{{\tt #1}}
% function macros
\newcommand{\hiki}[2]{{\tt #1}:\ {\em #2}}
\newcommand{\hikiopt}[3]{{\tt #1}:\ {\em #2}=#3}

\newdimen\hoge
\newdimen\truetextwidth
\newcommand{\func}[3]{%
\setbox0\hbox{#1(#2)}
\hoge=\wd0
\truetextwidth=\textwidth
\advance \truetextwidth by -2\oddsidemargin
\ifdim\hoge<\truetextwidth % short form
{\bf \colorbox{skyyellow}{#1(#2)\ $\to$ #3}}
%
\else % long form
\fcolorbox{skyyellow}{skyyellow}{%
   \begin{minipage}{\textwidth}%
   {\bf #1(#2)\\ %
    \qquad\quad   $\to$\ #3}%
   \end{minipage}%
   }%
\fi%
}

\newcommand{\out}[1]{{\em #1}}
\newcommand{\initialize}{%
  \paragraph{\large \colorbox{skyblue}{Initialize (Constructor)}}%
    \quad\\ %
    \vspace{3pt}\\
}
\newcommand{\method}{\C \paragraph{\large \colorbox{skyblue}{Methods}}}
% Attribute environment
\newenvironment{at}
{%begin
\paragraph{\large \colorbox{skyblue}{Attribute}}
\quad\\
\begin{description}
}%
{%end
\end{description}
}
% Operation environment
\newenvironment{op}
{%begin
\paragraph{\large \colorbox{skyblue}{Operations}}
\quad\\
\begin{table}[h]
\begin{center}
\begin{tabular}{|l|l|}
\hline
operator & explanation\\
\hline
}%
{%end
\hline
\end{tabular}
\end{center}
\end{table}
}
% Examples environment
\newenvironment{ex}%
{%begin
\paragraph{\large \colorbox{skyblue}{Examples}}
\VerbatimEnvironment
\renewcommand{\EveryVerbatim}{\fontencoding{OT1}\selectfont}
\begin{quote}
\begin{Verbatim}
}%
{%end
\end{Verbatim}
\end{quote}
}
%
\definecolor{skyblue}{cmyk}{0.2, 0, 0.1, 0}
\definecolor{skyyellow}{cmyk}{0.1, 0.1, 0.5, 0}
%
%\title{NZMATH User Manual\\ {\large{(for version 1.0)}}}
%\date{}
%\author{}
\begin{document}
%\maketitle
%
\setcounter{tocdepth}{3}
\setcounter{secnumdepth}{3}


\tableofcontents
\C

\chapter{Classes}


%---------- start document ---------- %
\section{finitefield -- Finite Field}\linkedzero{finitefield}

 \begin{itemize}
   \item {\bf Classes}
   \begin{itemize}
     \item \negok \linkingone{finitefield}{FiniteField}
     \item \negok \linkingone{finitefield}{FiniteFieldElement}
     \item \linkingone{finitefield}{FinitePrimeField}
     \item \linkingone{finitefield}{FinitePrimeFieldElement}
     \item \linkingone{finitefield}{ExtendedField}
     \item \linkingone{finitefield}{ExtendedFieldElement}
   \end{itemize}
   %\item {\bf Functions}
   %  \begin{itemize}
   %    \item \linkingone{rational}{innerProduct}
   %  \end{itemize}
 \end{itemize}

\C

 \subsection{\negok FiniteField -- finite field, abstract}\linkedone{finitefield}{FiniteField}
 
 Abstract class for finite fields. Do not use the class directly, but use the subclasses \linkingone{finitefield}{FinitePrimeField} or \linkingone{finitefield}{ExtendedField}.

 The class is a subclass of \linkingone{ring}{Field}.
\C
 \subsection{\negok FiniteFieldElement -- element in finite field, abstract}\linkedone{finitefield}{FiniteFieldElement}
 Abstract class for finite field elements. Do not use the class directly, but use the subclasses \linkingone{finitefield}{FinitePrimeFieldElement} or \linkingone{finitefield}{ExtendedFieldElement}.

 The class is a subclass of \linkingone{ring}{FieldElement}.

%%   \initialize
%%   \func{IntegerResidueClassRing}{\hiki{modulus}{integer}}{\out{IntegerResidueClassRing}}\\
%%   \spacing
%%   % document of basic document
%%   \quad Create an instance of IntegerResidueClassRing. 
%%   % added document
%%   The argument \param{modulus} = $m$ specifies an ideal $m\mathbb{Z}$.
%%   % \spacing
%%   % input, output document
%%   %See \linkingone{module}{point} for \param{point}.
%%   \begin{at}
%%     \item[zero]\linkedtwo{integer}{IntegerRing}{zero}:\\ It expresses The additive unit 0. (read only)
%%     \item[one]\linkedtwo{integer}{IntegerRing}{one}:\\ It expresses The multiplicative unit 1. (read only)
%%   \end{at}
%%   \begin{op}
%%   %  \verb|+| & Vector sum.\\
%%   %  \verb|-| & Vector subtraction.\\
%%   %  \verb|*| & Scalar multiplication.\\
%%   %  \verb|//| & Scalar division.\\
%%   %  \verb|-(unary)| & element negation.\\
%%     \verb|==| & ring equality.\\
%%   %  \verb|!=| & inequality or not.\\
%%   %  \verb+V[i]+ & Return the coefficient of i-th element of Vector.\\
%%   %  \verb+V[i] = c+ & Replace the coefficient of i-th element of Vector by c.\\
%%     \verb|card| & return cardinality. See also \linkingzero{compatibility} module.\\
%%     \verb|in| & return whether an element is in or not.\\
%%     \verb|repr| & return representation string.\\
%%     \verb|str| & return string.\\
%%   \end{op} 
%% %\begin{ex}
%% %>>> A = vector.Vector([1,2])
%% %>>> A
%% %Vector([1, 2])
%% %>>> A.compo
%% %[1, 2]
%% %>>>
%% %\end{ex}%Don't indent!
\C
 \subsection{FinitePrimeField -- finite prime field}\linkedone{finitefield}{FinitePrimeField}
 
Finite prime field is also known as $\mathbb{F}_p$ or $\mathrm{GF}(p)$. It has prime number cardinality.

 The class is a subclass of \linkingone{finitefield}{FiniteField}.

   \initialize
   \func{FinitePrimeField}
        {\hiki{characteristic}{integer}} 
        {\out{FinitePrimeField}}\\
   \spacing
   % document of basic document
   \quad Create a FinitePrimeField instance with the given \param{characteristic}.
   % added document
   %\spacing
   % input, output document
   \param{characteristic} must be positive prime integer.
   \begin{at}
    \item[zero]\linkedtwo{finitefield}{FinitePrimeField}{zero}:\\ It expresses the additive unit 0. (read only)
    \item[one]\linkedtwo{finitefield}{FinitePrimeField}{one}:\\ It expresses the multiplicative unit 1. (read only)
   \end{at}
   \begin{op}
     \verb|F==G| & equality test.\\
     \verb|x in F| & membership test.\\
     \verb|card(F)| & Cardinality of the field. \\
   \end{op} 
%\begin{ex}
%>>> A = vector.Vector([1,2])
%>>> A
%Vector([1, 2])
%>>> A.compo
%[1, 2]
%>>>
%\end{ex}%Don't indent!
  \method
  \subsubsection{createElement -- create element of finite prime field}\linkedtwo{finitefield}{FinitePrimeField}{createElement}
   \func{createElement}{\param{self},\ \hiki{seed}{integer}}{\out{FinitePrimeFieldElement}}\\
   \spacing
   % document of basic document
   \quad Create \linkingone{finitefield}{FinitePrimeFieldElement} with \param{seed}. 
   %\spacing
   % added document
   %\quad \negok Note that this function returns integer only.\\
   \spacing
   % input, output document
   \quad \param{seed} must be int or long.\\
%
  \subsubsection{getCharacteristic -- get characteristic}\linkedtwo{finitefield}{FinitePrimeField}{getCharacteristic}
   \func{getCharacteristic}{\param{self}}{\out{integer}}\\
   \spacing
   % document of basic document
   \quad Return the characteristic of the field.
   %\spacing
   % added document
   %\quad \negok Note that this function returns integer only.\\
   %\spacing
   % input, output document
   %\quad \param{a} must be int, long or rational.Integer.\\
%
  \subsubsection{issubring -- subring test}\linkedtwo{finitefield}{FinitePrimeField}{issubring}
   \func{issubring}{\param{self},\ \hiki{other}{\linkingone{ring}{Ring}}}{\out{bool}}\\
   \spacing
   % document of basic document
   \quad Report whether another ring contains the field as subring.

   %\spacing
   % added document
   %\quad \negok Note that this function returns integer only.\\
   %\spacing
   % input, output document
   %\quad if \param{as\_column} is True, try to create column matrix.\\
%
  \subsubsection{issuperring -- superring test}\linkedtwo{finitefield}{FinitePrimeField}{issuperring}
   \func{issuperring}{\param{self},\ \hiki{other}{\linkingone{ring}{Ring}}}{\out{bool}}\\
   \spacing
   % document of basic document
   \quad Report whether the field is a superring of another ring.

   Since the field is a prime field, it can be a superring of itself only.

   %\spacing
   % added document
   %\quad \negok Note that this function returns integer only.\\
   %\spacing
   % input, output document
   %\quad if \param{as\_column} is True, try to create column matrix.\\
%\begin{ex}
%>>> A = module.HogeClass((1,2))
%>>> A.hogemethod1(2)
%(2, 4)
%>>>
%\end{ex}%Don't indent!
\C
 \subsection{FinitePrimeFieldElement -- element of finite prime field}\linkedone{finitefield}{FinitePrimeFieldElement}
 The class provides elements of finite prime fields.

 It is a subclass of \linkingone{finitefield}{FiniteFieldElement} and \linkingone{intresidue}{IntegerResidueClass}. 

  \initialize
  \func{FinitePrimeFieldElement}
       {\hiki{representative}{integer},\ 
       \hiki{modulus}{integer}}
       {\out{FinitePrimeFieldElement}}\\
  \spacing
  % document of basic document
  \quad Create element in finite prime field of modulus with residue representative.
  % added document
  \spacing
  % input, output document
  \param{modulus} must be positive prime integer.
  %\begin{at}
  %  \item[compo]\linkedtwo{vector}{Vector}{compo}:\\ It expresses component of Vector.
  %\end{at}
  \begin{op}
    \verb|a+b| & addition.\\
    \verb|a-b| & subtraction.\\
    \verb|a*b| & multiplication.\\
    %\verb|/| & division.\\
    \verb|a**n,pow(a,n)| & power.\\
    \verb|-a| & negation.\\
    \verb|+a| & make a copy.\\
    \verb|a==b| & equality test.\\
    \verb|a!=b| & inequality test.\\
    \verb|repr(a)| & return representation string.\\
    \verb|str(a)| & return string.\\
  \end{op} 
%\begin{ex}
%>>> A = vector.Vector([1,2])
%>>> A
%Vector([1, 2])
%>>> A.compo
%[1, 2]
%>>>
%\end{ex}%Don't indent!
  \method
  \subsubsection{getRing -- get ring object}\linkedtwo{finitefield}{FinitePrimeFieldElement}{getRing}
   \func{getRing}{\param{self}}{\out{FinitePrimeField}}\\
   \spacing
   % document of basic document
   \quad Return an instance of FinitePrimeField to which the element belongs.
   %\spacing
   % added document
   %\quad \negok Note that this function returns integer only.\\
   %\spacing
   % input, output document
   %\quad \param{a} must be int, long or rational.Integer.\\
%
  \subsubsection{order -- order of multiplicative group}\linkedtwo{finitefield}{FinitePrimeFieldElement}{order}
   \func{order}{\param{self}}{\out{integer}}\\
   \spacing
   % document of basic document
   \quad Find and return the order of the element in the multiplicative group of $\mathbb{F}_p$.
   %\spacing
   % added document
   %\quad \negok Note that this function returns integer only.\\
   %\spacing
   % input, output document
   %\quad \param{a} must be int, long or rational.Integer.\\
%
%
%\begin{ex}
%>>> A = module.HogeClass((1,2))
%>>> A.hogemethod1(2)
%(2, 4)
%>>>
%\end{ex}%Don't indent!
\C
 \subsection{ExtendedField -- extended field of finite field}\linkedone{finitefield}{ExtendedField}
 ExtendedField is a class for finite field, whose cardinality $q = p^n$ with a prime $p$ and $n>1$. It is usually called $\mathbb{F}_q$ or $\mathrm{GF}(q)$.

 The class is a subclass of \linkingone{finitefield}{FiniteField}.

   \initialize
   \func{ExtendedField}
        {\hiki{basefield}{FiniteField},\ 
          \hiki{modulus}{FiniteFieldPolynomial}} 
        {\out{ExtendedField}}\\
   \spacing
   % document of basic document
   \quad Create a field extension $\mathtt{basefield}[X]/(\mathtt{modulus}(X))$.

FinitePrimeField instance with the given \param{characteristic}.
   % added document
   %\spacing
   % input, output document
   The \param{modulus} has to be an irreducible polynomial with coefficients in the \param{basefield}.
   \begin{at}
    \item[zero]\linkedtwo{finitefield}{ExtendedField}{zero}:\\ It expresses the additive unit 0. (read only)
    \item[one]\linkedtwo{finitefield}{ExtendedField}{one}:\\ It expresses the multiplicative unit 1. (read only)
   \end{at}
   \begin{op}
     \verb|F==G| & equality or not.\\
     \verb|x in F| & membership test.\\
     \verb|card(F)| & Cardinality of the field. \\
     \verb|repr(F)| & representation string.\\
     \verb|str(F)| & string.\\
   \end{op} 
%\begin{ex}
%>>> A = vector.Vector([1,2])
%>>> A
%Vector([1, 2])
%>>> A.compo
%[1, 2]
%>>>
%\end{ex}%Don't indent!
  \method
  \subsubsection{createElement -- create element of extended field}\linkedtwo{finitefield}{ExtendedField}{createElement}
   \func{createElement}{\param{self},\ \hiki{seed}{extended element seed}}{\out{ExtendedFieldElement}}\\
   \spacing
   % document of basic document
   \quad Create an element of the field from seed. The result is an instance of \linkingone{finitefield}{ExtendedFieldElement}.

   %\spacing
   % added document
   %\quad \negok Note that this function returns integer only.\\
   %\spacing
   % input, output document
   \quad The \param{seed} can be:
 \begin{itemize}
 \item a \linkingone{poly.uniutil}{FinitePrimeFieldPolynomial}
 \item an integer, which will be expanded in card(\param{basefield}) and interpreted as a polynomial. 
 \item \param{basefield} element. 
 \item a list of basefield elements interpreted as a polynomial coefficient.
 \end{itemize}
%
  \subsubsection{getCharacteristic -- get characteristic}\linkedtwo{finitefield}{ExtendedField}{getCharacteristic}
   \func{getCharacteristic}{\param{self}}{\out{integer}}\\
   \spacing
   % document of basic document
   \quad Return the characteristic of the field.
   %\spacing
   % added document
   %\quad \negok Note that this function returns integer only.\\
   %\spacing
   % input, output document
   %\quad \param{a} must be int, long or rational.Integer.\\
%
  \subsubsection{issubring -- subring test}\linkedtwo{finitefield}{ExtendedField}{issubring}
   \func{issubring}{\param{self},\ \hiki{other}{\linkingone{ring}{Ring}}}{\out{bool}}\\
   \spacing
   % document of basic document
   \quad Report whether another ring contains the field as subring.
   %\spacing
   % added document
   %\quad \negok Note that this function returns integer only.\\
   %\spacing
   % input, output document
   %\quad if \param{as\_column} is True, try to create column matrix.\\
%
  \subsubsection{issuperring -- superring test}\linkedtwo{finitefield}{ExtendedField}{issuperring}
   \func{issuperring}{\param{self},\ \hiki{other}{\linkingone{ring}{Ring}}}{\out{bool}}\\
   \spacing
   % document of basic document
   \quad Report whether the field is a superring of another ring.
   %\spacing
   % added document
   %\quad \negok Note that this function returns integer only.\\
   %\spacing
   % input, output document
   %\quad if \param{as\_column} is True, try to create column matrix.\\
%
  \subsubsection{primitive\_element -- generator of multiplicative group}\linkedtwo{finitefield}{ExtendedField}{primitive\_element}
   \func{primitive\_element}{\param{self}}{\out{ExtendedFieldElement}}\\
   \spacing
   % document of basic document
   \quad Return a primitive element of the field, i.e., a generator of the multiplicative group.
   \spacing
   % added document
   %\quad \negok Note that this function returns integer only.\\
   %\spacing
   % input, output document
   %\quad if \param{as\_column} is True, try to create column matrix.\\
%\begin{ex}
%>>> A = module.HogeClass((1,2))
%>>> A.hogemethod1(2)
%(2, 4)
%>>>
%\end{ex}%Don't indent!
\C
 \subsection{ExtendedFieldElement -- element of finite field}\linkedone{finitefield}{ExtendedFieldElement}
 ExtendedFieldElement is a class for an element of $F_q$. 

 The class is a subclass of \linkingone{finitefield}{FiniteFieldElement}.

  \initialize
  \func{ExtendedFieldElement}
       {\hiki{representative}{FiniteFieldPolynomial},\ 
       \hiki{field}{ExtendedField}}
       {\out{ExtendedFieldElement}}\\
  \spacing
  % document of basic document
  \quad Create an element of the finite extended field.
  % added document
  \spacing
  % input, output document
  The argument \param{representative} must be an \linkingone{poly.uniutil}{FiniteFieldPolynomial} has same \param{basefield}.
  Another argument \param{field} must be an instance of ExtendedField.

  %\begin{at}
  %  \item[compo]\linkedtwo{vector}{Vector}{compo}:\\ It expresses component of Vector.
  %\end{at}
  \begin{op}
    \verb|a+b| & addition.\\
    \verb|a-b| & subtraction.\\
    \verb|a*b| & multiplication.\\
    \verb|a/b| & inverse multiplication.\\
    \verb|a**n,pow(a,n)| & power.\\
    \verb|-a| & negation.\\
    \verb|+a| & make a copy.\\
    \verb|a==b| & equality test.\\
    \verb|a!=b| & inequality test.\\
    \verb|repr(a)| & return representation string.\\
    \verb|str(a)| & return string.\\
  \end{op} 
%\begin{ex}
%>>> A = vector.Vector([1,2])
%>>> A
%Vector([1, 2])
%>>> A.compo
%[1, 2]
%>>>
%\end{ex}%Don't indent!
  \method
  \subsubsection{getRing -- get ring object}\linkedtwo{finitefield}{FinitePrimeFieldElement}{getRing}
   \func{getRing}{\param{self}}{\out{FinitePrimeField}}\\
   \spacing
   % document of basic document
   \quad Return an instance of FinitePrimeField to which the element belongs.
   %\spacing
   % added document
   %\quad \negok Note that this function returns integer only.\\
   %\spacing
   % input, output document
   %\quad \param{a} must be int, long or rational.Integer.\\
%
  \subsubsection{inverse -- inverse element}\linkedtwo{finitefield}{ExtendedFieldElement}{inverse}
   \func{inverse}{\param{self}}{\out{ExtendedFieldElement}}\\
   \spacing
   % document of basic document
   \quad Return the inverse element.
   %\spacing
   % added document
   %\quad \negok Note that this function returns integer only.\\
   %\spacing
   % input, output document
   %\quad \param{a} must be int, long or rational.Integer.\\
%
%%   \subsubsection{order -- order of multiplicative group}\linkedtwo{finitefield}{FinitePrimeFieldElement}{order}
%%    \func{order}{\param{self}}{\out{integer}}\\
%%    \spacing
%%    % document of basic document
%%    \quad Find and return the order of the element in the multiplicative group of $F_p$.
%%    %\spacing
%%    % added document
%%    %\quad \negok Note that this function returns integer only.\\
%%    %\spacing
%%    % input, output document
%%    %\quad \param{a} must be int, long or rational.Integer.\\
%
%
%\begin{ex}
%>>> A = module.HogeClass((1,2))
%>>> A.hogemethod1(2)
%(2, 4)
%>>>
%\end{ex}%Don't indent!
\C

%---------- end document ---------- %

\bibliographystyle{jplain}%use jbibtex
\bibliography{nzmath_references}

\end{document}


%\documentclass{report}

%%%%%%%%%%%%%%%%%%%%%%%%%%%%%%%%%%%%%%%%%%%%%%%%%%%%%%%%%%%%%
%
% macros for nzmath manual
%
%%%%%%%%%%%%%%%%%%%%%%%%%%%%%%%%%%%%%%%%%%%%%%%%%%%%%%%%%%%%%
\usepackage{amssymb,amsmath}
\usepackage{color}
\usepackage[dvipdfm,bookmarks=true,bookmarksnumbered=true,%
 pdftitle={NZMATH Users Manual},%
 pdfsubject={Manual for NZMATH Users},%
 pdfauthor={NZMATH Development Group},%
 pdfkeywords={TeX; dvipdfmx; hyperref; color;},%
 colorlinks=true]{hyperref}
\usepackage{fancybox}
\usepackage[T1]{fontenc}
%
\newcommand{\DS}{\displaystyle}
\newcommand{\C}{\clearpage}
\newcommand{\NO}{\noindent}
\newcommand{\negok}{$\dagger$}
\newcommand{\spacing}{\vspace{1pt}\\ }
% software macros
\newcommand{\nzmathzero}{{\footnotesize $\mathbb{N}\mathbb{Z}$}\texttt{MATH}}
\newcommand{\nzmath}{{\nzmathzero}\ }
\newcommand{\pythonzero}{$\mbox{\texttt{Python}}$}
\newcommand{\python}{{\pythonzero}\ }
% link macros
\newcommand{\linkingzero}[1]{{\bf \hyperlink{#1}{#1}}}%module
\newcommand{\linkingone}[2]{{\bf \hyperlink{#1.#2}{#2}}}%module,class/function etc.
\newcommand{\linkingtwo}[3]{{\bf \hyperlink{#1.#2.#3}{#3}}}%module,class,method
\newcommand{\linkedzero}[1]{\hypertarget{#1}{}}
\newcommand{\linkedone}[2]{\hypertarget{#1.#2}{}}
\newcommand{\linkedtwo}[3]{\hypertarget{#1.#2.#3}{}}
\newcommand{\linktutorial}[1]{\href{http://docs.python.org/tutorial/#1}{#1}}
\newcommand{\linktutorialone}[2]{\href{http://docs.python.org/tutorial/#1}{#2}}
\newcommand{\linklibrary}[1]{\href{http://docs.python.org/library/#1}{#1}}
\newcommand{\linklibraryone}[2]{\href{http://docs.python.org/library/#1}{#2}}
\newcommand{\pythonhp}{\href{http://www.python.org/}{\python website}}
\newcommand{\nzmathwiki}{\href{http://nzmath.sourceforge.net/wiki/}{{\nzmathzero}Wiki}}
\newcommand{\nzmathsf}{\href{http://sourceforge.net/projects/nzmath/}{\nzmath Project Page}}
\newcommand{\nzmathtnt}{\href{http://tnt.math.se.tmu.ac.jp/nzmath/}{\nzmath Project Official Page}}
% parameter name
\newcommand{\param}[1]{{\tt #1}}
% function macros
\newcommand{\hiki}[2]{{\tt #1}:\ {\em #2}}
\newcommand{\hikiopt}[3]{{\tt #1}:\ {\em #2}=#3}

\newdimen\hoge
\newdimen\truetextwidth
\newcommand{\func}[3]{%
\setbox0\hbox{#1(#2)}
\hoge=\wd0
\truetextwidth=\textwidth
\advance \truetextwidth by -2\oddsidemargin
\ifdim\hoge<\truetextwidth % short form
{\bf \colorbox{skyyellow}{#1(#2)\ $\to$ #3}}
%
\else % long form
\fcolorbox{skyyellow}{skyyellow}{%
   \begin{minipage}{\textwidth}%
   {\bf #1(#2)\\ %
    \qquad\quad   $\to$\ #3}%
   \end{minipage}%
   }%
\fi%
}

\newcommand{\out}[1]{{\em #1}}
\newcommand{\initialize}{%
  \paragraph{\large \colorbox{skyblue}{Initialize (Constructor)}}%
    \quad\\ %
    \vspace{3pt}\\
}
\newcommand{\method}{\C \paragraph{\large \colorbox{skyblue}{Methods}}}
% Attribute environment
\newenvironment{at}
{%begin
\paragraph{\large \colorbox{skyblue}{Attribute}}
\quad\\
\begin{description}
}%
{%end
\end{description}
}
% Operation environment
\newenvironment{op}
{%begin
\paragraph{\large \colorbox{skyblue}{Operations}}
\quad\\
\begin{table}[h]
\begin{center}
\begin{tabular}{|l|l|}
\hline
operator & explanation\\
\hline
}%
{%end
\hline
\end{tabular}
\end{center}
\end{table}
}
% Examples environment
\newenvironment{ex}%
{%begin
\paragraph{\large \colorbox{skyblue}{Examples}}
\VerbatimEnvironment
\renewcommand{\EveryVerbatim}{\fontencoding{OT1}\selectfont}
\begin{quote}
\begin{Verbatim}
}%
{%end
\end{Verbatim}
\end{quote}
}
%
\definecolor{skyblue}{cmyk}{0.2, 0, 0.1, 0}
\definecolor{skyyellow}{cmyk}{0.1, 0.1, 0.5, 0}
%
%\title{NZMATH User Manual\\ {\large{(for version 1.0)}}}
%\date{}
%\author{}
\begin{document}
%\maketitle
%
\setcounter{tocdepth}{3}
\setcounter{secnumdepth}{3}


\tableofcontents
\C

\chapter{Classes}


%---------- start document ---------- %
 \section{group -- algorithms for finite groups}\linkedzero{group}
 \begin{itemize}
   \item {\bf Classes}
   \begin{itemize}
     \item \linkingone{group}{Group}
     \item \linkingone{group}{GroupElement}
     \item \linkingone{group}{GenerateGroup}
     \item \linkingone{group}{AbelianGenerate}
   \end{itemize}
 \end{itemize}

\C

 \subsection{\negok Group -- group structure}\linkedone{group}{Group}
 \initialize
  \func{Group}{\hiki{value}{class},\ \hikiopt{operation}{int}{-1}}{Group}\\
  \spacing
  % document of basic document
  \quad Create an object which wraps \param{value} (typically a ring or a field)
  only to expose its group structure. \\
  \spacing
  % added document
  \quad The instance has methods defined for (abstract) group.
  For example, \linkingtwo{group}{Group}{identity} returns the identity element of the group from wrapped \param{value}.\\
  \spacing
  % input, output document
  \quad \param{value} must be an instance of a class expresses group structure.
  \param{operation} must be 0 or 1;
  If \param{operation} is 0, \param{value} is regarded as the additive group.
  On the other hand, if \param{operation} is 1, \param{value} is considered as the multiplicative group.
  The default value of \param{operation} is 0.\\
  \negok You can input an instance of \linkingone{group}{Group} itself as \param{value}.
  In this case, the default value of \param{operation} is the attribute \linkingtwo{group}{Group}{operation} of the instance.
  \begin{at}
    \item[entity]\linkedtwo{group}{Group}{entity}:\\ The wrapped object.
    \item[operation]\linkedtwo{group}{Group}{operation}:\\ It expresses the mode of operation;
$0$ means additive, while $1$ means multiplicative.
  \end{at}
  \begin{op}
    \verb+A==B+ & Return whether A and B are equal or not.\\
    \verb+A!=B+ & Check whether A and B are not equal.\\
    \verb+repr(A)+ & representation\\
    \verb+str(A)+ & simple representation\\
  \end{op} 
\begin{ex}
>>> G1=group.Group(finitefield.FinitePrimeField(37), 1)
>>> print G1
F_37
>>> G2=group.Group(intresidue.IntegerResidueClassRing(6), 0)
>>> print G2
Z/6Z
\end{ex}%Don't indent!
  \method
  \subsubsection{setOperation -- change operation}\linkedtwo{group}{Group}{setOperation}
   \func{setOperation}{\param{self},\ \hiki{operation}{int}}{(None)}\\
   \spacing
   % document of basic document
   \quad �Q�̃^�C�v�����@($0$)�܂��͏�@($1$)�ɕς���B\\
   \spacing
   % added document
   %\spacing
   % input, output document
   \quad \param{operation}��0�܂���1�B\\
 \subsubsection{\negok createElement -- generate a GroupElement instance}\linkedtwo{group}{Group}{createElement}
   \func{createElement}{\param{self},\ \param{*value}}{\out{GroupElement}}\\
   \spacing
   % document of basic document
   \quad Return \linkingone{group}{GroupElement} object whose group is \param{self}, initialized with \param{value}.\\
   \spacing
   % added document
   \quad \negok ���̕��@��\param{self}�ƌĂԁBlinkingtwo{group}{Group}{entity}.createElement.\\
   \spacing
   % input, output document
   \quad \param{value} must fit the form of argument for \param{self}.\linkingtwo{group}{Group}{entity}.createElement.\\
  \subsubsection{\negok identity -- identity element}\linkedtwo{group}{Group}{identity}
   \func{identity}{\param{self}}{\param{GroupElement}}\\
   \spacing
   % document of basic document
   \quad �Q�̒P�ʌ��̒l��Ԃ��B\\�@
   \spacing
   % added document
   \quad�@\linkingtwo{group}{Group}{operation}�ɂ���ĂO(���@)�܂��͂P(��@) ��Ԃ��B
   \negok ���̕��@��param{self}.\linkingtwo{group}{Group}{entity}�ƌĂ΂�Ă���Bidentity�܂���\linkingtwo{group}{Group}{entity}�������������Ȃ��Ƃ��͂O���P��Ԃ��B
   \spacing
   % input, output document
  \subsubsection{grouporder -- order of the group}\linkedtwo{group}{Group}{grouporder}
   \func{grouporder}{\param{self}}{\param{long}}\\
   \spacing
   % document of basic document
   \quad param{self}�̗v�f�̌��̒l��Ԃ��B.\\
   \spacing
   % added document
   \quad \negok ���̕��@��\param{self}�ƌĂ΂�Ă���B\linkingtwo{group}{Group}{entity}.grouporder, card or \_\_len\_\_.\\
   �����ł͂��̌Q�͗L���ƍl���A �Ԃ��l��long�^�̐����ł���B
   �������̌Q�������̏ꍇ�A���̕��@�ł͏o�͂͒�`�ł��Ȃ��B
   \spacing
   % input, output document
\begin{ex}
>>> G1=group.Group(finitefield.FinitePrimeField(37), 1)
>>> G1.grouporder()
36
>>> G1.setOperation(0)
>>> print G1.identity()
FinitePrimeField,0 in F_37
>>> G1.grouporder()
37
\end{ex}%Don't indent!
\C

 \subsection{GroupElement -- elements of group structure}\linkedone{group}{GroupElement}
 \initialize
  \func{GroupElement}{\hiki{value}{class},\ \hikiopt{operation}{int}{-1}}{GroupElement}\\
  \spacing
  % document of basic document
  \quad Create an object which wraps \param{value} (typically a ring element or
  a field element) to make it behave as an element of group.\\
  \spacing 
  % added document
  \quad The instance has methods defined for an (abstract) element of group.
  For example, \linkingtwo{group}{GroupElement}{inverse} returns the inverse element of \param{value} as the element of group object.\\
  \spacing
  % input, output document
  \quad \param{value} must be an instance of a class expresses an element of group structure.
  \param{operation} must be $0$ or $1$;
  If \param{operation} is $0$, \param{value} is regarded as the additive group.
  On the other hand, if \param{operation} is $1$, \param{value} is considered as the multiplicative group.
  The default value of \param{operation} is $0$.\\
  \negok You can input an instance of \linkingone{group}{GroupElement} itself as \param{value}.
  In this case, the default value of \param{operation} is the attribute \linkingtwo{group}{GroupElement}{operation} of the instance.
  \begin{at}
    \item[entity]\linkedtwo{group}{GroupElement}{entity}:\\ The wrapped object.
    \item[set]\linkedtwo{group}{GroupElement}{set}:\\ It is an instance of \linkingone{group}{Group}, which expresses the group to which \param{self} belongs.
    \item[operation]\linkedtwo{group}{GroupElement}{operation}:\\ It expresses the mode of operation;
$0$ means additive, while $1$ means multiplicative.
  \end{at}
  \begin{op}
    \verb+A==B+ & Return whether A and B are equal or not.\\
    \verb+A!=B+ & Check whether A and B are not equal.\\
    \verb+A.ope(B)+ & Basic operation (additive $+$, multiplicative $*$)\\
    \verb+A.ope2(n)+ & Extended operation (additive $*$, multiplicative $**$)\\
    \verb+A.inverse()+\linkedtwo{group}{GroupElement}{inverse} & Return the inverse element of \param{self}\\
    \verb+repr(A)+ & representation\\
    \verb+str(A)+ & simple representation\\
  \end{op} 
\begin{ex}
>>> G1=group.GroupElement(finitefield.FinitePrimeFieldElement(18, 37), 1)
>>> print G1
FinitePrimeField,18 in F_37
>>> G2=group.Group(intresidue.IntegerResidueClass(3, 6), 0)
IntegerResidueClass(3, 6)
\end{ex}%Don't indent!
  \method
  \subsubsection{setOperation -- change operation}\linkedtwo{group}{GroupElement}{setOperation}
   \func{setOperation}{\param{self},\ \hiki{operation}{int}}{(None)}\\
   \spacing
   % document of basic document
   \quad �Q�̃^�C�v�����@($0$)�܂��͏�@($1$)�ɕς���B\\
   \spacing
   % added document
   %\spacing
   % input, output document
   \quad \param{operation}��$0$��$1$�B\\
 \subsubsection{\negok getGroup -- generate a Group instance}\linkedtwo{group}{GroupElement}{getGroup}
   \func{getGroup}{\param{self}}{\out{Group}}\\
   \spacing
   % document of basic document
   \quad Return \linkingone{group}{Group} object to which \param{self} belongs.\\
   \spacing
   % added document
   \quad \negok This method calls \param{self}.\linkingtwo{group}{GroupElement}{entity}.getRing or getGroup.\\
   \negok In an initialization of \linkingone{group}{GroupElement}, the attribute \linkingtwo{group}{GroupElement}{set} is set as the value returned from the method.\\
   \spacing
   % input, output document
  \subsubsection{order -- order by factorization method}\linkedtwo{group}{GroupElement}{order}
   \func{order}{\param{self}}{\param{long}}\\
   \spacing
   % document of basic document
   \quad \param{self}�̈ʐ��̒l��Ԃ��B\\
   \spacing
   % added document
   \quad \negok ���̕��@�͌Q�̈ʐ��̈����������g���B\\
   \negok �����ł͂��̌Q�͗L���ƍl���A �Ԃ��l��long�^�̐����ł���B
   \negok ���������̌Q�������Ȃ�΁A���̕��@�̓G���[��Ԃ����L���łȂ��l��Ԃ��B\\
   \spacing
   % input, output document
  \subsubsection{t\_order -- order by baby-step giant-step}\linkedtwo{group}{GroupElement}{t\_order}
   \func{t\_order}{\param{self},\ \hikiopt{v}{int}{2}}{\param{long}}\\
   \spacing
   % document of basic document
   \quad \param{self}�̈ʐ��̒l��Ԃ��B\\
   \spacing
   % added document
   \quad \negok ���̕��@��Terry's baby-step giant-step algorithm���g���B\\
   ���̕��@�͌Q�̈ʐ����g��Ȃ��B
   \param{v}��baby-step�̐�������B
   \negok �����ł͂��̌Q�͗L���ƍl���A �Ԃ��l��long�^�̐����ł���B
   \negok ���������̌Q�������Ȃ�΁A���̕��@�̓G���[��Ԃ����L���łȂ��l��Ԃ��B\\
   \spacing
   % input, output document
   \quad \param{v}��int�^�̐����B\\
\begin{ex}
>>> G1=group.GroupElement(finitefield.FinitePrimeFieldElement(18, 37), 1)
>>> G1.order()
36
>>> G1.t_order()
36
\end{ex}%Don't indent!
\C

 \subsection{\negok GenerateGroup -- group structure with generator}\linkedone{group}{GenerateGroup}
 \initialize
  \func{GenerateGroup}{\hiki{value}{class},\ \hikiopt{operation}{int}{-1}}{GroupElement}\\
  \spacing
  % document of basic document
  \quad Create an object which is generated by \param{value} as the element of group structure. \\
  \spacing
  % added document
  \quad This initializes a group `including' the group elements, not a group with generators, now.
  We do not recommend using this module now.
  The instance has methods defined for an (abstract) element of group.
  For example, \linkingtwo{group}{GroupElement}{inverse} returns the inverse element of \param{value} as the element of group object.\\
  The class inherits the class \linkingone{group}{Group}.\\
  \spacing
  % input, output document
  \quad \param{value} must be a list of generators.
  Each generator should be an instance of a class expresses an element of group structure.
  \param{operation} must be $0$ or $1$;
  If \param{operation} is $0$, \param{value} is regarded as the additive group.
  On the other hand, if \param{operation} is $1$, \param{value} is considered as the multiplicative group.
  The default value of \param{operation} is $0$.\\
\begin{ex}
>>> G1=group.GenerateGroup([intresidue.IntegerResidueClass(2, 20),
... intresidue.IntegerResidueClass(6, 20)])
>>> G1.identity()
intresidue.IntegerResidueClass(0, 20)
\end{ex}%Don't indent!
\C

 \subsection{AbelianGenerate -- abelian group structure with generator}\linkedone{group}{AbelianGenerate}
 \initialize
  \quad \linkingone{group}{GenerateGroup}�̃N���X���p������B\\
  
  \subsubsection{relationLattice -- relation between generators}\linkedtwo{group}{AbelianGenerate}{relationLattice}
   \func{relationLattice}{\param{self}}{\out{\linkingone{matrix}{Matrix}}}\\
   \spacing
   % document of basic document
   \quad �i�q�����֌W�ɂ��鐔�̃��X�g��Ԃ��Bas a square matrix each of whose column vector is a relation basis.\\
   \spacing
   % added document
   \quad �֌W�̌���$V$�� $\prod_{i} \mbox{generator}_i V_i=1$���[�����B
   \spacing
   % input, output document
  \subsubsection{computeStructure -- abelian group structure}\linkedtwo{group}{AbelianGenerate}{computeStructure}
   \func{computeStructure}{\param{self}}{\param{tuple}}\\
   \spacing
   % document of basic document
   \quad �L���A�[�x���Q�\�����v�Z����B\\
   \spacing
   % added document
   \quad�@����\param{self} $G \simeq \oplus_i <h_i>$�ŁA [($h_1$, ord($h_1$)),..($h_n$, ord($h_n$))]�� $^\# G$��Ԃ��B
   $<h_i>$�͕ϐ�$h_i$�̏���Q�B\\
   \spacing
   % input, output document
   \quad �o�͓͂�‚��—v�f�����ŽO�‚̑g�ł���B;
   �ŏ��̗v�f��$h_i$�Ƃ̂��ʐ��̃��X�g�ł���B,
   �܂��A��Ԗڂ̗v�f�͌Q�̈ʐ��ł���B
\begin{ex}
>>> G=AbelianGenerate([intresidue.IntegerResidueClass(2, 20),
... intresidue.IntegerResidueClass(6, 20)])
>>> G.relationLattice()
10 7
 0 1
>>> G.computeStructure()
([IntegerResidueClassRing,IntegerResidueClass(2, 20), 10)], 10L)
\end{ex}%Don't indent!
\C
%---------- end document ---------- %

\bibliographystyle{jplain}%use jbibtex
\bibliography{nzmath_references}

\end{document}


%\documentclass{report}

%%%%%%%%%%%%%%%%%%%%%%%%%%%%%%%%%%%%%%%%%%%%%%%%%%%%%%%%%%%%%
%
% macros for nzmath manual
%
%%%%%%%%%%%%%%%%%%%%%%%%%%%%%%%%%%%%%%%%%%%%%%%%%%%%%%%%%%%%%
\usepackage{amssymb,amsmath}
\usepackage{color}
\usepackage[dvipdfm,bookmarks=true,bookmarksnumbered=true,%
 pdftitle={NZMATH Users Manual},%
 pdfsubject={Manual for NZMATH Users},%
 pdfauthor={NZMATH Development Group},%
 pdfkeywords={TeX; dvipdfmx; hyperref; color;},%
 colorlinks=true]{hyperref}
\usepackage{fancybox}
\usepackage[T1]{fontenc}
%
\newcommand{\DS}{\displaystyle}
\newcommand{\C}{\clearpage}
\newcommand{\NO}{\noindent}
\newcommand{\negok}{$\dagger$}
\newcommand{\spacing}{\vspace{1pt}\\ }
% software macros
\newcommand{\nzmathzero}{{\footnotesize $\mathbb{N}\mathbb{Z}$}\texttt{MATH}}
\newcommand{\nzmath}{{\nzmathzero}\ }
\newcommand{\pythonzero}{$\mbox{\texttt{Python}}$}
\newcommand{\python}{{\pythonzero}\ }
% link macros
\newcommand{\linkingzero}[1]{{\bf \hyperlink{#1}{#1}}}%module
\newcommand{\linkingone}[2]{{\bf \hyperlink{#1.#2}{#2}}}%module,class/function etc.
\newcommand{\linkingtwo}[3]{{\bf \hyperlink{#1.#2.#3}{#3}}}%module,class,method
\newcommand{\linkedzero}[1]{\hypertarget{#1}{}}
\newcommand{\linkedone}[2]{\hypertarget{#1.#2}{}}
\newcommand{\linkedtwo}[3]{\hypertarget{#1.#2.#3}{}}
\newcommand{\linktutorial}[1]{\href{http://docs.python.org/tutorial/#1}{#1}}
\newcommand{\linktutorialone}[2]{\href{http://docs.python.org/tutorial/#1}{#2}}
\newcommand{\linklibrary}[1]{\href{http://docs.python.org/library/#1}{#1}}
\newcommand{\linklibraryone}[2]{\href{http://docs.python.org/library/#1}{#2}}
\newcommand{\pythonhp}{\href{http://www.python.org/}{\python website}}
\newcommand{\nzmathwiki}{\href{http://nzmath.sourceforge.net/wiki/}{{\nzmathzero}Wiki}}
\newcommand{\nzmathsf}{\href{http://sourceforge.net/projects/nzmath/}{\nzmath Project Page}}
\newcommand{\nzmathtnt}{\href{http://tnt.math.se.tmu.ac.jp/nzmath/}{\nzmath Project Official Page}}
% parameter name
\newcommand{\param}[1]{{\tt #1}}
% function macros
\newcommand{\hiki}[2]{{\tt #1}:\ {\em #2}}
\newcommand{\hikiopt}[3]{{\tt #1}:\ {\em #2}=#3}

\newdimen\hoge
\newdimen\truetextwidth
\newcommand{\func}[3]{%
\setbox0\hbox{#1(#2)}
\hoge=\wd0
\truetextwidth=\textwidth
\advance \truetextwidth by -2\oddsidemargin
\ifdim\hoge<\truetextwidth % short form
{\bf \colorbox{skyyellow}{#1(#2)\ $\to$ #3}}
%
\else % long form
\fcolorbox{skyyellow}{skyyellow}{%
   \begin{minipage}{\textwidth}%
   {\bf #1(#2)\\ %
    \qquad\quad   $\to$\ #3}%
   \end{minipage}%
   }%
\fi%
}

\newcommand{\out}[1]{{\em #1}}
\newcommand{\initialize}{%
  \paragraph{\large \colorbox{skyblue}{Initialize (Constructor)}}%
    \quad\\ %
    \vspace{3pt}\\
}
\newcommand{\method}{\C \paragraph{\large \colorbox{skyblue}{Methods}}}
% Attribute environment
\newenvironment{at}
{%begin
\paragraph{\large \colorbox{skyblue}{Attribute}}
\quad\\
\begin{description}
}%
{%end
\end{description}
}
% Operation environment
\newenvironment{op}
{%begin
\paragraph{\large \colorbox{skyblue}{Operations}}
\quad\\
\begin{table}[h]
\begin{center}
\begin{tabular}{|l|l|}
\hline
operator & explanation\\
\hline
}%
{%end
\hline
\end{tabular}
\end{center}
\end{table}
}
% Examples environment
\newenvironment{ex}%
{%begin
\paragraph{\large \colorbox{skyblue}{Examples}}
\VerbatimEnvironment
\renewcommand{\EveryVerbatim}{\fontencoding{OT1}\selectfont}
\begin{quote}
\begin{Verbatim}
}%
{%end
\end{Verbatim}
\end{quote}
}
%
\definecolor{skyblue}{cmyk}{0.2, 0, 0.1, 0}
\definecolor{skyyellow}{cmyk}{0.1, 0.1, 0.5, 0}
%
%\title{NZMATH User Manual\\ {\large{(for version 1.0)}}}
%\date{}
%\author{}
\begin{document}
%\maketitle
%
\setcounter{tocdepth}{3}
\setcounter{secnumdepth}{3}


\tableofcontents
\C

\chapter{Classes}


%---------- start document ---------- %
 \section{imaginary -- complex numbers and its functions}\linkedzero{imaginary}
���̃��W���[��{\tt imaginary}�ł͕��f���Ɉ����B���̊֐��͎��\linklibrary{cmath}�W�����W���[���ƑΉ����Ă���B.

 \begin{itemize}
   \item {\bf Classes}
   \begin{itemize}
     \item \linkingone{imaginary}{ComplexField}
     \item \linkingone{imaginary}{Complex}
     \item \negok \linkingone{imaginary}{ExponentialPowerSeries}
     \item \negok \linkingone{imaginary}{AbsoluteError}
     \item \negok \linkingone{imaginary}{RelativeError}
   \end{itemize}
   \item {\bf Functions}
     \begin{itemize}
       \item \linkingone{imaginary}{exp}
       \item \linkingone{imaginary}{expi}
       \item \linkingone{imaginary}{log}
       \item \linkingone{imaginary}{sin}
       \item \linkingone{imaginary}{cos}
       \item \linkingone{imaginary}{tan}
       \item \linkingone{imaginary}{sinh}
       \item \linkingone{imaginary}{cosh}
       \item \linkingone{imaginary}{tanh}
       \item \linkingone{imaginary}{atanh}
       \item \linkingone{imaginary}{sqrt}

     \end{itemize}
 \end{itemize}

���̃��W���[���͈ȉ��̓��e�������B:
\begin{description}
   \item[e]\linkedone{imaginary}{e}:\\
     This constant is obsolete (Ver 1.1.0).
   \item[pi]\linkedone{imaginary}{pi}:\\
     This constant is obsolete (Ver 1.1.0).
   \item[j]\linkedone{imaginary}{j}:\\
     \param{j} is the imaginary unit.
   \item[theComplexField]\linkedone{imaginary}{theComplexField}:\\
     \param{theComplexField} is the instance of \linkingone{imaginary}{ComplexField}.
 \end{description}

\C
 \subsection{ComplexField -- field of complex numbers}\linkedone{imaginary}{ComplexField}
 �N���X�͕��f����̑̂ł���B ���̃N���X�͈�‚̃C���X�^���X\linkingone{imaginary}{theComplexField}�����B

�@���̃N���X��\linkingone{ring}{Field}�̃T�u�N���X�ł���B

  \initialize
  \func{ComplexField}{}{\out{ComplexField}}\\
  \spacing
  % document of basic document
  \quad ���f���̂̃C���X�^���X�����B 
  % added document
  �����C���X�^���X����肽���Ȃ��ꍇ�́A\linkingone{imaginary}{theComplexField}.
  % \spacing
  % input, output document
  %See \linkingone{module}{point} for \param{point}.
  \begin{at}
    \item[zero]\linkedtwo{imaginary}{ComplexField}{zero}:\\ It expresses The additive unit 0. (read only)
    \item[one]\linkedtwo{imaginary}{ComplexField}{one}:\\ It expresses The multiplicative unit 1. (read only)
  \end{at}
  \begin{op}
  %  \verb|+| & Vector sum.\\
  %  \verb|-| & Vector subtraction.\\
  %  \verb|*| & Scalar multiplication.\\
  %  \verb|//| & Scalar division.\\
  %  \verb|-(unary)| & element negation.\\
  %  \verb|==| & equality or not.\\
  %  \verb|!=| & inequality or not.\\
  %  \verb+V[i]+ & Return the coefficient of i-th element of Vector.\\
  %  \verb+V[i] = c+ & Replace the coefficient of i-th element of Vector by c.\\
  %  \verb|len| & return length of \linkingtwo{vector}{Vector}{compo}.\\
    \verb|in| & membership test; return whether an element is in or not.\\
    \verb|repr| & return representation string.\\
    \verb|str| & return string.\\
  \end{op} 
%\begin{ex}
%>>> A = vector.Vector([1,2])
%>>> A
%Vector([1, 2])
%>>> A.compo
%[1, 2]
%>>>
%\end{ex}%Don't indent!
  \method
   \subsubsection{createElement -- create Imaginary object}\linkedtwo{imaginary}{ComplexField}{createElement}
    \func{createElement}{\param{self},\ \hiki{seed}{integer}}{\out{Integer}}\\
    \spacing
    % document of basic document
    \quad \param{seed}�̕��f���I�u�W�F�N�g��Ԃ��B. 
    \spacing
    % added document
    %\quad \negok ���̊֐��͐����̒l�����Ԃ��Ȃ����Ƃɒ��ӁB\\
    \spacing
    % input, output document
    \quad \param{seed}�͕��f�������f���𖄂ߍ��񂾐��łȂ���΂Ȃ��B\\
%
%
  \subsubsection{getCharacteristic -- get characteristic}\linkedtwo{imaginary}{ComplexField}{getCharacteristic}
   \func{getCharacteristic}{\param{self}}{\out{integer}}\\
   \spacing
   % document of basic document
   \quad �W�����O��Ԃ��B.
   %\spacing
   % added document
   %\quad \negok ���̊֐��͐����̒l�����Ԃ��Ȃ����Ƃɒ��ӁB\\
   %\spacing
   % input, output document
   %\quad \param{a}��int�܂���long�^�B�܂��͗L�����������B\\
%
  \subsubsection{issubring -- subring test}\linkedtwo{imaginary}{ComplexField}{issubring}
   \func{issubring}{\param{self},\ \hiki{aRing}{\linkingone{ring}{Ring}}}{\out{bool}}\\
   \spacing
   % document of basic document
   \quad ���̊‚����f���̏�ɕ����‚Ƃ��Ċ܂܂�Ă��邩�����Ă����B
   \spacing
   % added document
   %\quad \negok ���̊֐��͐����̒l�����Ԃ��Ȃ����Ƃɒ��ӁB\\
   %\spacing
   % input, output document
   %\quad \param{as\_column}���������Ƃ��̓J�����}�g���b�N�X����낤�Ƃ���B\\
%
  \subsubsection{issuperring -- superring test}\linkedtwo{imaginary}{ComplexField}{issuperring}
   \func{issuperring}{\param{self},\ \hiki{aRing}{\linkingone{ring}{Ring}}}{\out{bool}}\\
   \spacing
   % document of basic document
   \quad ���f���̂����̊‚𕔕��‚Ƃ��Ċ܂�ł��邩�����Ă����B
   \spacing
   % added document
   %\quad \negok ���̊֐��͐����̒l�����Ԃ��Ȃ����Ƃɒ��ӁB\\
   %\spacing
   % input, output document
   %\quad ����\param{as\_column}���������Ƃ��̓J�����}�g���b�N�X����낤�Ƃ���B\\
%\begin{ex}
%>>> A = module.HogeClass((1,2))
%>>> A.hogemethod1(2)
%(2, 4)
%>>>
%\end{ex}%Don't indent!

\C
 \subsection{Complex -- a complex number}\linkedone{imaginary}{Complex}
 Complex�Ƃ͕��f���̃N���X�ł���B�ǂ̃C���X�^�X����‚̐������B���܂킿���鐔�̎����Ƌ����ł���B

�@���̃N���X��\linkingone{ring}{FieldElement}�̃T�u�N���X�ł���B

 All implemented operators in this class are delegated to complex type. 
  \initialize
  \func{Complex}
       {\hiki{re}{number}
        \hikiopt{im}{number}{0}
       }
       {\out{Imaginary}}\\
  \spacing
  % document of basic document
  \quad ���f�������B
  % added document
  \spacing
  % input, output document
  \param{re}�͎����ł������ł��\��Ȃ��B����\param{re}��������\param{im}���^�����Ă��Ȃ��ƁA�����͂O�Ƃ������Ƃł���B
  \begin{at}
    \item[real]\linkedtwo{imaginary}{Complex}{real}:\\ ���f���̎���������\���B
    \item[imag]\linkedtwo{imaginary}{Complex}{imag}:\\ ���f���̋���������\���B
  \end{at}
  %\begin{op}
  %  \verb|+| & Vector sum.\\
  %  \verb|-| & Vector subtraction.\\
  %  \verb|*| & Scalar multiplication.\\
  %  \verb|//| & Scalar division.\\
  %  \verb|-(unary)| & element negation.\\
  %  \verb|==| & equality or not.\\
  %  \verb|!=| & inequality or not.\\
  %  \verb+V[i]+ & Return the coefficient of i-th element of Vector.\\
  %  \verb+V[i] = c+ & Replace the coefficient of i-th element of Vector by c.\\
  %  \verb|len| & return length of \linkingtwo{vector}{Vector}{compo}.\\
  %  \verb|repr| & return representation string.\\
  %  \verb|str| & return string of \linkingtwo{vector}{Vector}{compo}.\\
  %\end{op} 
%\begin{ex}
%>>> A = vector.Vector([1,2])
%>>> A
%Vector([1, 2])
%>>> A.compo
%[1, 2]
%>>>
%\end{ex}%Don't indent!
  \method
  \subsubsection{getRing -- get ring object}\linkedtwo{imaginary}{Complex}{getRing}
   \func{getRing}{\param{self}}{\out{ComplexField}}\\
   \spacing
   % document of basic document
   \quad ���f���̂̃C���X�^���X��Ԃ��B
   %\spacing
   % added document
   %\quad \negok ���̊֐��͐����̒l�����Ԃ��Ȃ����Ƃɒ��ӁB\\
   %\spacing
   % input, output document
   %\quad \param{a}��int�܂���long�^�܂��͗L�����������B\\
%
  \subsubsection{arg -- argument of complex}\linkedtwo{imaginary}{Complex}{arg}
   \func{arg}{\param{self}}{\out{radian}}\\
   \spacing
   % document of basic document
   \quad Return the angle between the x-axis and the number in the Gaussian plane.
   %\spacing
   % added document
   %\quad \negok ���̊֐��͐����̒l�����Ԃ��Ȃ����Ƃɒ��ӁB\\
   %\spacing
   % input, output document
   \quad \out{radian}��Float�^�B\
%
  \subsubsection{conjugate -- complex conjugate}\linkedtwo{imaginary}{Complex}{conjugate}
   \func{conjugate}{\param{self}}{\out{Complex}}\\
   \spacing
   % document of basic document
   \quad ���鐔�̕��f�����̒l��Ԃ��B
   %\spacing
   % added document
   %\quad \negok ���̊֐��͐����̒l�����Ԃ��Ȃ����Ƃɒ��ӁB\\
   %\spacing
   % input, output document
   %\quad \out{radian}��Float�^�B\\
%
  \subsubsection{copy -- copied number}\linkedtwo{imaginary}{Complex}{copy}
   \func{copy}{\param{self}}{\out{Complex}}\\
   \spacing
   % document of basic document
   \quad ���鐔���g�̒l��Ԃ��B
   %\spacing
   % added document
   %\quad \negok ���̊֐��͐����̒l�����Ԃ��Ȃ����Ƃɒ��ӁB\\
   %\spacing
   % input, output document
   %\quad \out{radian}��Float�^�B\\
%
  \subsubsection{inverse -- complex inverse}\linkedtwo{imaginary}{Complex}{inverse}
   \func{inverse}{\param{self}}{\out{Complex}}\\
   \spacing
   % document of basic document
   \quad ���鐔�̋t���̒l��Ԃ��B
   \spacing
   % added document
   %\quad \negok ���̊֐��͐����̒l�����Ԃ��Ȃ����Ƃɒ��ӁB\\
   %\spacing
   % input, output document
   \quad�@���͂���Đ����O�̂Ƃ��AZeroDivisionError��Ԃ��B
%
%\begin{ex}
%>>> A = module.HogeClass((1,2))
%>>> A.hogemethod1(2)
%(2, 4)
%>>>
%\end{ex}%Don't indent!
\C
 \subsection{ExponentialPowerSeries -- exponential power series}\linkedone{imaginary}{ExponentialPowerSeries}
  This class is obsolete (Ver 1.1.0).

 \subsection{AbsoluteError -- absolute error}\linkedone{imaginary}{AbsoluteError}
  This class is obsolete (Ver 1.1.0).

 \subsection{RelativeError -- relative error}\linkedone{imaginary}{RelativeError}
  This class is obsolete (Ver 1.1.0).

  \subsection{exp(function) -- exponential value}\linkedone{imaginary}{exp}
   This function is obsolete (Ver 1.1.0).

  \subsection{expi(function) -- imaginary exponential value}\linkedone{imaginary}{expi}
   This function is obsolete (Ver 1.1.0).

  \subsection{log(function) -- logarithm}\linkedone{imaginary}{log}
   This function is obsolete (Ver 1.1.0).

  \subsection{sin(function) -- sine function}\linkedone{imaginary}{sin}
   This function is obsolete (Ver 1.1.0).

  \subsection{cos(function) -- cosine function}\linkedone{imaginary}{cos}
   This function is obsolete (Ver 1.1.0).

  \subsection{tan(function) -- tangent function}\linkedone{imaginary}{tan}
   This function is obsolete (Ver 1.1.0).

  \subsection{sinh(function) -- hyperbolic sine function}\linkedone{imaginary}{sinh}
   This function is obsolete (Ver 1.1.0).

  \subsection{cosh(function) -- hyperbolic cosine function}\linkedone{imaginary}{cosh}
   This function is obsolete (Ver 1.1.0).

  \subsection{tanh(function) -- hyperbolic tangent function}\linkedone{imaginary}{tanh}
   This function is obsolete (Ver 1.1.0).

  \subsection{atanh(function) -- hyperbolic arc tangent function}\linkedone{imaginary}{atanh}
   This function is obsolete (Ver 1.1.0).

  \subsection{sqrt(function) -- square root}\linkedone{imaginary}{sqrt}
   This function is obsolete (Ver 1.1.0).

%
% \begin{ex}
% >>> A = vector.Vector([1,2,3])
% >>> B = vector.Vector([2,1,0])
% >>> vector.innerProduct(A,B)
% 4
% >>>
% \end{ex}%Don't indent!
\C

%---------- end document ---------- %

\bibliographystyle{jplain}%use jbibtex
\bibliography{nzmath_references}

\end{document}


%\documentclass{report}

%%%%%%%%%%%%%%%%%%%%%%%%%%%%%%%%%%%%%%%%%%%%%%%%%%%%%%%%%%%%%
%
% macros for nzmath manual
%
%%%%%%%%%%%%%%%%%%%%%%%%%%%%%%%%%%%%%%%%%%%%%%%%%%%%%%%%%%%%%
\usepackage{amssymb,amsmath}
\usepackage{color}
\usepackage[dvipdfm,bookmarks=true,bookmarksnumbered=true,%
 pdftitle={NZMATH Users Manual},%
 pdfsubject={Manual for NZMATH Users},%
 pdfauthor={NZMATH Development Group},%
 pdfkeywords={TeX; dvipdfmx; hyperref; color;},%
 colorlinks=true]{hyperref}
\usepackage{fancybox}
\usepackage[T1]{fontenc}
%
\newcommand{\DS}{\displaystyle}
\newcommand{\C}{\clearpage}
\newcommand{\NO}{\noindent}
\newcommand{\negok}{$\dagger$}
\newcommand{\spacing}{\vspace{1pt}\\ }
% software macros
\newcommand{\nzmathzero}{{\footnotesize $\mathbb{N}\mathbb{Z}$}\texttt{MATH}}
\newcommand{\nzmath}{{\nzmathzero}\ }
\newcommand{\pythonzero}{$\mbox{\texttt{Python}}$}
\newcommand{\python}{{\pythonzero}\ }
% link macros
\newcommand{\linkingzero}[1]{{\bf \hyperlink{#1}{#1}}}%module
\newcommand{\linkingone}[2]{{\bf \hyperlink{#1.#2}{#2}}}%module,class/function etc.
\newcommand{\linkingtwo}[3]{{\bf \hyperlink{#1.#2.#3}{#3}}}%module,class,method
\newcommand{\linkedzero}[1]{\hypertarget{#1}{}}
\newcommand{\linkedone}[2]{\hypertarget{#1.#2}{}}
\newcommand{\linkedtwo}[3]{\hypertarget{#1.#2.#3}{}}
\newcommand{\linktutorial}[1]{\href{http://docs.python.org/tutorial/#1}{#1}}
\newcommand{\linktutorialone}[2]{\href{http://docs.python.org/tutorial/#1}{#2}}
\newcommand{\linklibrary}[1]{\href{http://docs.python.org/library/#1}{#1}}
\newcommand{\linklibraryone}[2]{\href{http://docs.python.org/library/#1}{#2}}
\newcommand{\pythonhp}{\href{http://www.python.org/}{\python website}}
\newcommand{\nzmathwiki}{\href{http://nzmath.sourceforge.net/wiki/}{{\nzmathzero}Wiki}}
\newcommand{\nzmathsf}{\href{http://sourceforge.net/projects/nzmath/}{\nzmath Project Page}}
\newcommand{\nzmathtnt}{\href{http://tnt.math.se.tmu.ac.jp/nzmath/}{\nzmath Project Official Page}}
% parameter name
\newcommand{\param}[1]{{\tt #1}}
% function macros
\newcommand{\hiki}[2]{{\tt #1}:\ {\em #2}}
\newcommand{\hikiopt}[3]{{\tt #1}:\ {\em #2}=#3}

\newdimen\hoge
\newdimen\truetextwidth
\newcommand{\func}[3]{%
\setbox0\hbox{#1(#2)}
\hoge=\wd0
\truetextwidth=\textwidth
\advance \truetextwidth by -2\oddsidemargin
\ifdim\hoge<\truetextwidth % short form
{\bf \colorbox{skyyellow}{#1(#2)\ $\to$ #3}}
%
\else % long form
\fcolorbox{skyyellow}{skyyellow}{%
   \begin{minipage}{\textwidth}%
   {\bf #1(#2)\\ %
    \qquad\quad   $\to$\ #3}%
   \end{minipage}%
   }%
\fi%
}

\newcommand{\out}[1]{{\em #1}}
\newcommand{\initialize}{%
  \paragraph{\large \colorbox{skyblue}{Initialize (Constructor)}}%
    \quad\\ %
    \vspace{3pt}\\
}
\newcommand{\method}{\C \paragraph{\large \colorbox{skyblue}{Methods}}}
% Attribute environment
\newenvironment{at}
{%begin
\paragraph{\large \colorbox{skyblue}{Attribute}}
\quad\\
\begin{description}
}%
{%end
\end{description}
}
% Operation environment
\newenvironment{op}
{%begin
\paragraph{\large \colorbox{skyblue}{Operations}}
\quad\\
\begin{table}[h]
\begin{center}
\begin{tabular}{|l|l|}
\hline
operator & explanation\\
\hline
}%
{%end
\hline
\end{tabular}
\end{center}
\end{table}
}
% Examples environment
\newenvironment{ex}%
{%begin
\paragraph{\large \colorbox{skyblue}{Examples}}
\VerbatimEnvironment
\renewcommand{\EveryVerbatim}{\fontencoding{OT1}\selectfont}
\begin{quote}
\begin{Verbatim}
}%
{%end
\end{Verbatim}
\end{quote}
}
%
\definecolor{skyblue}{cmyk}{0.2, 0, 0.1, 0}
\definecolor{skyyellow}{cmyk}{0.1, 0.1, 0.5, 0}
%
%\title{NZMATH User Manual\\ {\large{(for version 1.0)}}}
%\date{}
%\author{}
\begin{document}
%\maketitle
%
\setcounter{tocdepth}{3}
\setcounter{secnumdepth}{3}


\tableofcontents
\C

\chapter{Classes}


%---------- start document ---------- %
 \section{intresidue -- integer residue}\linkedzero{intresidue}
intresidue module provides integer residue classes or $\mathbf{Z}/m\mathbf{Z}$.

 \begin{itemize}
   \item {\bf Classes}
   \begin{itemize}
     \item \linkingone{intresidue}{IntegerResidueClass}
     \item \linkingone{intresidue}{IntegerResidueClassRing}
   \end{itemize}
   %\item {\bf Functions}
   %  \begin{itemize}
   %    \item \linkingone{rational}{innerProduct}
   %  \end{itemize}
 \end{itemize}

\C

 \subsection{IntegerResidueClass -- integer residue class}\linkedone{intresidue}{IntegerResidueClass}
 
 This class is a subclass of \linkingone{ring}{CommutativeRingElement}.

  \initialize
  \func{IntegerResidueClass}
       {\hiki{representative}{integer},\ 
       \hiki{modulus}{integer}}
       {\out{Integer}}\\
  \spacing
  % document of basic document
  \quad Create a residue class of modulus with residue representative.
  % added document
  \spacing
  % input, output document
  \param{modulus} must be positive integer.
  %\begin{at}
  %  \item[compo]\linkedtwo{vector}{Vector}{compo}:\\ It expresses component of Vector.
  %\end{at}
  \begin{op}
    \verb|a+b| & addition.\\
    \verb|a-b| & subtraction.\\
    \verb|a*b| & multiplication.\\
    \verb|a/b| & division.\\
    \verb|a**i,pow(a,i)| & power.\\
    \verb|-a| & negation.\\
    \verb|+a| & make a copy.\\
    \verb|a==b| & equality or not.\\
    \verb|a!=b| & inequality or not.\\
    \verb|repr(a)| & return representation string.\\
    \verb|str(a)| & return string.\\
  \end{op} 
%\begin{ex}
%>>> A = vector.Vector([1,2])
%>>> A
%Vector([1, 2])
%>>> A.compo
%[1, 2]
%>>>
%\end{ex}%Don't indent!
  \method
  \subsubsection{getRing -- get ring object}\linkedtwo{intresidue}{IntegerResidueClassRing}{getRing}
   \func{getRing}{\param{self}}{\out{IntegerResidueClassRing}}\\
   \spacing
   % document of basic document
   \quad Return a ring to which it belongs.
   %\spacing
   % added document
   %\quad \negok Note that this function returns integer only.\\
   %\spacing
   % input, output document
   %\quad \param{a} must be int, long or rational.Integer.\\
%
  \subsubsection{getResidue -- get residue}\linkedtwo{intresidue}{IntegerResidueClassRing}{getResidue}
   \func{getResidue}{\param{self}}{\out{integer}}\\
   \spacing
   % document of basic document
   \quad Return the value of residue.
   %\spacing
   % added document
   %\quad \negok Note that this function returns integer only.\\
   %\spacing
   % input, output document
   %\quad \param{a} must be int, long or rational.Integer.\\
%
  \subsubsection{getModulus -- get modulus}\linkedtwo{intresidue}{IntegerResidueClassRing}{getModulus}
   \func{getModulus}{\param{self}}{\out{integer}}\\
   \spacing
   % document of basic document
   \quad Return the value of modulus.
   %\spacing
   % added document
   %\quad \negok Note that this function returns integer only.\\
   %\spacing
   % input, output document
   %\quad \param{a} must be int, long or rational.Integer.\\
%
  \subsubsection{inverse -- inverse element}\linkedtwo{intresidue}{IntegerResidueClassRing}{inverse}
   \func{inverse}{\param{self}}{\out{IntegerResidueClass}}\\
   \spacing
   % document of basic document
   \quad Return the inverse element if it is invertible. Otherwise raise ValueError.
   %\spacing
   % added document
   %\quad \negok Note that this function returns integer only.\\
   %\spacing
   % input, output document
   %\quad \param{a} must be int, long or rational.Integer.\\
%
  \subsubsection{minimumAbsolute -- minimum absolute representative}\linkedtwo{intresidue}{IntegerResidueClassRing}{minimumAbsolute}
   \func{minimumAbsolute}{\param{self}}{\out{\linkingone{rational}{Integer}}}\\
   \spacing
   % document of basic document
   \quad  Return the minimum absolute representative integer of the residue class.
   %\spacing
   % added document
   %\quad \negok Note that this function returns integer only.\\
   %\spacing
   % input, output document
   %\quad \param{a} must be int, long or rational.Integer.\\
%
  \subsubsection{minimumNonNegative -- smallest non-negative representative}\linkedtwo{intresidue}{IntegerResidueClassRing}{minimumNonNegative}
   \func{minimumNonNegative}{\param{self}}{\out{\linkingone{rational}{Integer}}}\\
   \spacing
   % document of basic document
   \quad Return the smallest non-negative representative element of the residue class.
   %\spacing
   % added document
   \quad \negok this method has an alias, named toInteger.\\
   %\spacing
   % input, output document
   %\quad \param{a} must be int, long or rational.Integer.\\
%
%\begin{ex}
%>>> A = module.HogeClass((1,2))
%>>> A.hogemethod1(2)
%(2, 4)
%>>>
%\end{ex}%Don't indent!
\C
 \subsection{IntegerResidueClassRing -- ring of integer residue}\linkedone{intresidue}{IntegerResidueClassRing}
 The class is for rings of integer residue classes.

 This class is a subclass of \linkingone{ring}{CommutativeRing}.


  \initialize
  \func{IntegerResidueClassRing}{\hiki{modulus}{integer}}{\out{IntegerResidueClassRing}}\\
  \spacing
  % document of basic document
  \quad Create an instance of IntegerResidueClassRing. 
  % added document
  The argument \param{modulus} = $m$ specifies an ideal $m\mathbb{Z}$.
  % \spacing
  % input, output document
  %See \linkingone{module}{point} for \param{point}.
  \begin{at}
    \item[zero]\linkedtwo{integer}{IntegerRing}{zero}:\\ It expresses The additive unit 0. (read only)
    \item[one]\linkedtwo{integer}{IntegerRing}{one}:\\ It expresses The multiplicative unit 1. (read only)
  \end{at}
  \begin{op}
  %  \verb|+| & Vector sum.\\
  %  \verb|-| & Vector subtraction.\\
  %  \verb|*| & Scalar multiplication.\\
  %  \verb|//| & Scalar division.\\
  %  \verb|-(unary)| & element negation.\\
    \verb|R==A| & ring equality.\\
  %  \verb|!=| & inequality or not.\\
  %  \verb+V[i]+ & Return the coefficient of i-th element of Vector.\\
  %  \verb+V[i] = c+ & Replace the coefficient of i-th element of Vector by c.\\
    \verb|card(R)| & return cardinality. See also \linkingzero{compatibility} module.\\
    \verb|e in R| & return whether an element is in or not.\\
    \verb|repr(R)| & return representation string.\\
    \verb|str(R)| & return string.\\
  \end{op} 
%\begin{ex}
%>>> A = vector.Vector([1,2])
%>>> A
%Vector([1, 2])
%>>> A.compo
%[1, 2]
%>>>
%\end{ex}%Don't indent!
  \method
  \subsubsection{createElement -- create IntegerResidueClass object}\linkedtwo{intresidue}{IntegerResidueClassRing}{createElement}
   \func{createElement}{\param{self},\ \hiki{seed}{integer}}{\out{Integer}}\\
   \spacing
   % document of basic document
   \quad Return an IntegerResidueClass instance with \param{seed}. 
   %\spacing
   % added document
   %\quad \negok Note that this function returns integer only.\\
   %\spacing
   % input, output document
   %\quad \\
%
  \subsubsection{isfield -- field test}\linkedtwo{intresidue}{IntegerResidueClassRing}{isfield}
   \func{isfield}{\param{self}}{\out{bool}}\\
   \spacing
   % document of basic document
   \quad Return True if the modulus is prime, False if not. Since a finite domain is a field, other ring property tests are merely aliases of isfield; they are isdomain, iseuclidean, isnoetherian, ispid, isufd.
   % added document
   %\quad \negok Note that this function returns integer only.\\
   %\spacing
   % input, output document
   %\quad if \param{as\_column} is True, try to create column matrix.\\
%
  \subsubsection{getInstance -- get instance of IntegerResidueClassRing}\linkedtwo{intresidue}{IntegerResidueClassRing}{getInstance}
   \func{getInstance}{\param{cls},\ \hiki{modulus}{integer}}{\out{IntegerResidueClass}}\\
   \spacing
   % document of basic document
   \quad Return an instance of the class of specified modulus. Since this is a class method, use it as:

\verb|IntegerResidueClassRing.getInstance(3)|

to create a $\mathbb{Z}/3\mathbb{Z}$ object, for example.
%\begin{ex}
%>>> A = module.HogeClass((1,2))
%>>> A.hogemethod1(2)
%(2, 4)
%>>>
%\end{ex}%Don't indent!

\C

%---------- end document ---------- %

\bibliographystyle{jplain}%use jbibtex
\bibliography{nzmath_references}

\end{document}


%\documentclass{report}

%%%%%%%%%%%%%%%%%%%%%%%%%%%%%%%%%%%%%%%%%%%%%%%%%%%%%%%%%%%%%
%
% macros for nzmath manual
%
%%%%%%%%%%%%%%%%%%%%%%%%%%%%%%%%%%%%%%%%%%%%%%%%%%%%%%%%%%%%%
\usepackage{amssymb,amsmath}
\usepackage{color}
\usepackage[dvipdfm,bookmarks=true,bookmarksnumbered=true,%
 pdftitle={NZMATH Users Manual},%
 pdfsubject={Manual for NZMATH Users},%
 pdfauthor={NZMATH Development Group},%
 pdfkeywords={TeX; dvipdfmx; hyperref; color;},%
 colorlinks=true]{hyperref}
\usepackage{fancybox}
\usepackage[T1]{fontenc}
%
\newcommand{\DS}{\displaystyle}
\newcommand{\C}{\clearpage}
\newcommand{\NO}{\noindent}
\newcommand{\negok}{$\dagger$}
\newcommand{\spacing}{\vspace{1pt}\\ }
% software macros
\newcommand{\nzmathzero}{{\footnotesize $\mathbb{N}\mathbb{Z}$}\texttt{MATH}}
\newcommand{\nzmath}{{\nzmathzero}\ }
\newcommand{\pythonzero}{$\mbox{\texttt{Python}}$}
\newcommand{\python}{{\pythonzero}\ }
% link macros
\newcommand{\linkingzero}[1]{{\bf \hyperlink{#1}{#1}}}%module
\newcommand{\linkingone}[2]{{\bf \hyperlink{#1.#2}{#2}}}%module,class/function etc.
\newcommand{\linkingtwo}[3]{{\bf \hyperlink{#1.#2.#3}{#3}}}%module,class,method
\newcommand{\linkedzero}[1]{\hypertarget{#1}{}}
\newcommand{\linkedone}[2]{\hypertarget{#1.#2}{}}
\newcommand{\linkedtwo}[3]{\hypertarget{#1.#2.#3}{}}
\newcommand{\linktutorial}[1]{\href{http://docs.python.org/tutorial/#1}{#1}}
\newcommand{\linktutorialone}[2]{\href{http://docs.python.org/tutorial/#1}{#2}}
\newcommand{\linklibrary}[1]{\href{http://docs.python.org/library/#1}{#1}}
\newcommand{\linklibraryone}[2]{\href{http://docs.python.org/library/#1}{#2}}
\newcommand{\pythonhp}{\href{http://www.python.org/}{\python website}}
\newcommand{\nzmathwiki}{\href{http://nzmath.sourceforge.net/wiki/}{{\nzmathzero}Wiki}}
\newcommand{\nzmathsf}{\href{http://sourceforge.net/projects/nzmath/}{\nzmath Project Page}}
\newcommand{\nzmathtnt}{\href{http://tnt.math.se.tmu.ac.jp/nzmath/}{\nzmath Project Official Page}}
% parameter name
\newcommand{\param}[1]{{\tt #1}}
% function macros
\newcommand{\hiki}[2]{{\tt #1}:\ {\em #2}}
\newcommand{\hikiopt}[3]{{\tt #1}:\ {\em #2}=#3}

\newdimen\hoge
\newdimen\truetextwidth
\newcommand{\func}[3]{%
\setbox0\hbox{#1(#2)}
\hoge=\wd0
\truetextwidth=\textwidth
\advance \truetextwidth by -2\oddsidemargin
\ifdim\hoge<\truetextwidth % short form
{\bf \colorbox{skyyellow}{#1(#2)\ $\to$ #3}}
%
\else % long form
\fcolorbox{skyyellow}{skyyellow}{%
   \begin{minipage}{\textwidth}%
   {\bf #1(#2)\\ %
    \qquad\quad   $\to$\ #3}%
   \end{minipage}%
   }%
\fi%
}

\newcommand{\out}[1]{{\em #1}}
\newcommand{\initialize}{%
  \paragraph{\large \colorbox{skyblue}{Initialize (Constructor)}}%
    \quad\\ %
    \vspace{3pt}\\
}
\newcommand{\method}{\C \paragraph{\large \colorbox{skyblue}{Methods}}}
% Attribute environment
\newenvironment{at}
{%begin
\paragraph{\large \colorbox{skyblue}{Attribute}}
\quad\\
\begin{description}
}%
{%end
\end{description}
}
% Operation environment
\newenvironment{op}
{%begin
\paragraph{\large \colorbox{skyblue}{Operations}}
\quad\\
\begin{table}[h]
\begin{center}
\begin{tabular}{|l|l|}
\hline
operator & explanation\\
\hline
}%
{%end
\hline
\end{tabular}
\end{center}
\end{table}
}
% Examples environment
\newenvironment{ex}%
{%begin
\paragraph{\large \colorbox{skyblue}{Examples}}
\VerbatimEnvironment
\renewcommand{\EveryVerbatim}{\fontencoding{OT1}\selectfont}
\begin{quote}
\begin{Verbatim}
}%
{%end
\end{Verbatim}
\end{quote}
}
%
\definecolor{skyblue}{cmyk}{0.2, 0, 0.1, 0}
\definecolor{skyyellow}{cmyk}{0.1, 0.1, 0.5, 0}
%
%\title{NZMATH User Manual\\ {\large{(for version 1.0)}}}
%\date{}
%\author{}
\begin{document}
%\maketitle
%
\setcounter{tocdepth}{3}
\setcounter{secnumdepth}{3}


\tableofcontents
\C

\chapter{Classes}


%---------- start document ---------- %
 \section{lattice -- Lattice}\linkedzero{lattice}
 \begin{itemize}
 \item {\bf Classes}
   \begin{itemize}
   \item \linkingone{lattice}{Lattice}
   \item \linkingone{lattice}{LatticeElement}
   \end{itemize}
 \item {\bf Functions}
   \begin{itemize}
   \item \linkingone{lattice}{LLL}
   \end{itemize}
 \end{itemize}
%
  \subsection{Lattice -- lattice}\linkedone{lattice}{Lattice}
  \initialize
  \func{Lattice}{
    \hiki{basis}{\linkingone{matrix}{RingSquareMatrix}},\ \hiki{quadraticForm}{\linkingone{matrix}{RingSquareMatrix}}}{\out{Lattice}}\\
  \spacing
  % document of basic document
  \quad Create Lattice object. \\
  \spacing
  % added document
  \spacing
  % input, output document
  \begin{at}
    \item[basis]\linkedtwo{lattice}{Lattice}{basis}: The basis of \param{self} lattice.
    \item[quadraticForm]\linkedtwo{lattice}{Lattice}{quadraticForm}: The quadratic form corresponding the inner product.
  \end{at}
\C
  \method
  \subsubsection{createElement -- create element}\linkedtwo{lattice}{Lattice}{createElement}
  \func{createElement}{\param{self}, \ \hiki{compo}{list}}{\out{\linkingone{lattice}{LatticeElement}}}\\
  \spacing
  % document of basic document
  \quad Create the element which has coefficients with given \param{compo}. \\
  \spacing
  % add document
  %\spacing
  % input, output document
%
  \subsubsection{bilinearForm -- bilinear form}\linkedtwo{lattice}{Lattice}{bilinearForm}
  \func{bilinearForm}{\param{self}, \ \hiki{v\_1}{\linkingone{vector}{Vector}}, \, \hiki{v\_2}{\linkingone{vector}{Vector}} }{\out{integer}}\\
  \spacing
  % document of basic document
  \quad Return the inner product of $v_1$ and $v_2$ with \linkingtwo{lattice}{Lattice}{quadraticForm}. \\
  \spacing
  % add document
  %\spacing
  % input, output document
%
  \subsubsection{isCyclic -- Check whether cyclic lattice or not}\linkedtwo{lattice}{Lattice}{isCyclic}
  \func{isCyclic}{\param{self}}{\out{bool}}\\
  \spacing
  % document of basic document
  \quad Check whether \param{self} lattice is a cyclic lattice or not.
  \spacing
  % add document
  \quad 
  %\spacing
  % input, output document
%
  \subsubsection{isIdeal -- Check whether ideal lattice or not}\linkedtwo{lattice}{Lattice}{isIdeal}
  \func{isIdeal}{\param{self}}{\out{bool}}\\
  \spacing
  % document of basic document
  \quad Check whether \param{self} lattice is an ideal lattice or not.
  \spacing
  % add document
  %\spacing
  % input, output document
\C
  \subsection{LatticeElement -- element of lattice}\linkedone{lattice}{LatticeElement}
  \initialize
  \func{LatticeElement}{
   \hiki{lattice}{\linkingone{lattice}{Lattice}},\
   \hiki{compo}{list},\ 
  }{
   \out{LatticeElement}
  }\\
  \spacing
  % document of basic document
  \quad Create LatticeElement object. \\
  \spacing
  % added document
  \quad Elements of lattices are represented as linear combinations of basis.
  The class inherits \linkingone{matrix}{Matrix}. Then, intances are regarded as $n \times 1$ matrix whose coefficients consist of \param{compo}, where $n$ is the dimension of lattice.\\
  \spacing
  % input, output document
  \quad \param{lattice} is an instance of Lattice object. \param{compo} is coeeficients list of basis.
  %
  \begin{at}
    \item[lattice]\linkedtwo{lattice}{LatticeElement}{lattice}: the lattice which includes \param{self}
  \end{at}
\C
  \method
  \subsubsection{getLattice -- Find lattice belongs to}\linkedtwo{lattice}{LatticeElement}{getLattice}
  \func{getLattice}{\param{self}}{\out{\linkingone{lattice}{Lattice}}}\\
  \spacing
  % document of basic document
  \quad Obtain the Lattice object corresponding to \param{self}. \\
  \spacing
  % add document
  %\spacing
  % input, output document
\C
  \subsection{LLL(function) -- LLL reduction}\linkedone{lattice}{LLL}
  \func{LLL}{\hiki{M}{\linkingone{matrix}{RingSquareMatrix}}}{\out{\hiki{L}{RingSquareMatrix}}, \ \out{\hiki{T}{RingSquareMatrix}}} \\
  \spacing
  % document of basic document
  \quad Return LLL-reduced basis for the given basis \param{M}. \\
  \spacing
  % add document
  % \spacing
  % input, output document
  \quad The output \param{L} is the LLL-reduced basis. \param{T} is the transportation matrix from the original basis to the LLL-reduced basis.
%
\begin{ex}
>>> M=mat.Matrix(3,3,[1,0,12,0,1,26,0,0,13]);
>>> lat.LLL(M);
([1, 0, 0]+[0, 1, 0]+[0, 0, 13], [1L, 0L, -12L]+[0L, 1L, -26L]+[0L, 0L, 1L])
\end{ex}%Don't indent!(indent causes an error.)
\C

%---------- end document ---------- %

\bibliographystyle{jplain}%use jbibtex
\bibliography{nzmath_references}

\end{document}


%%%%%%%%%%%%%%%%%%%%%%%%%%%%%%%%%%%%%%%%%%%%%%%%%%%%%%%%%%%%%%
%
% macros for nzmath manual
%
%%%%%%%%%%%%%%%%%%%%%%%%%%%%%%%%%%%%%%%%%%%%%%%%%%%%%%%%%%%%%
\usepackage{amssymb,amsmath}
\usepackage{color}
\usepackage[dvipdfm,bookmarks=true,bookmarksnumbered=true,%
 pdftitle={NZMATH Users Manual},%
 pdfsubject={Manual for NZMATH Users},%
 pdfauthor={NZMATH Development Group},%
 pdfkeywords={TeX; dvipdfmx; hyperref; color;},%
 colorlinks=true]{hyperref}
\usepackage{fancybox}
\usepackage[T1]{fontenc}
%
\newcommand{\DS}{\displaystyle}
\newcommand{\C}{\clearpage}
\newcommand{\NO}{\noindent}
\newcommand{\negok}{$\dagger$}
\newcommand{\spacing}{\vspace{1pt}\\ }
% software macros
\newcommand{\nzmathzero}{{\footnotesize $\mathbb{N}\mathbb{Z}$}\texttt{MATH}}
\newcommand{\nzmath}{{\nzmathzero}\ }
\newcommand{\pythonzero}{$\mbox{\texttt{Python}}$}
\newcommand{\python}{{\pythonzero}\ }
% link macros
\newcommand{\linkingzero}[1]{{\bf \hyperlink{#1}{#1}}}%module
\newcommand{\linkingone}[2]{{\bf \hyperlink{#1.#2}{#2}}}%module,class/function etc.
\newcommand{\linkingtwo}[3]{{\bf \hyperlink{#1.#2.#3}{#3}}}%module,class,method
\newcommand{\linkedzero}[1]{\hypertarget{#1}{}}
\newcommand{\linkedone}[2]{\hypertarget{#1.#2}{}}
\newcommand{\linkedtwo}[3]{\hypertarget{#1.#2.#3}{}}
\newcommand{\linktutorial}[1]{\href{http://docs.python.org/tutorial/#1}{#1}}
\newcommand{\linktutorialone}[2]{\href{http://docs.python.org/tutorial/#1}{#2}}
\newcommand{\linklibrary}[1]{\href{http://docs.python.org/library/#1}{#1}}
\newcommand{\linklibraryone}[2]{\href{http://docs.python.org/library/#1}{#2}}
\newcommand{\pythonhp}{\href{http://www.python.org/}{\python website}}
\newcommand{\nzmathwiki}{\href{http://nzmath.sourceforge.net/wiki/}{{\nzmathzero}Wiki}}
\newcommand{\nzmathsf}{\href{http://sourceforge.net/projects/nzmath/}{\nzmath Project Page}}
\newcommand{\nzmathtnt}{\href{http://tnt.math.se.tmu.ac.jp/nzmath/}{\nzmath Project Official Page}}
% parameter name
\newcommand{\param}[1]{{\tt #1}}
% function macros
\newcommand{\hiki}[2]{{\tt #1}:\ {\em #2}}
\newcommand{\hikiopt}[3]{{\tt #1}:\ {\em #2}=#3}

\newdimen\hoge
\newdimen\truetextwidth
\newcommand{\func}[3]{%
\setbox0\hbox{#1(#2)}
\hoge=\wd0
\truetextwidth=\textwidth
\advance \truetextwidth by -2\oddsidemargin
\ifdim\hoge<\truetextwidth % short form
{\bf \colorbox{skyyellow}{#1(#2)\ $\to$ #3}}
%
\else % long form
\fcolorbox{skyyellow}{skyyellow}{%
   \begin{minipage}{\textwidth}%
   {\bf #1(#2)\\ %
    \qquad\quad   $\to$\ #3}%
   \end{minipage}%
   }%
\fi%
}

\newcommand{\out}[1]{{\em #1}}
\newcommand{\initialize}{%
  \paragraph{\large \colorbox{skyblue}{Initialize (Constructor)}}%
    \quad\\ %
    \vspace{3pt}\\
}
\newcommand{\method}{\C \paragraph{\large \colorbox{skyblue}{Methods}}}
% Attribute environment
\newenvironment{at}
{%begin
\paragraph{\large \colorbox{skyblue}{Attribute}}
\quad\\
\begin{description}
}%
{%end
\end{description}
}
% Operation environment
\newenvironment{op}
{%begin
\paragraph{\large \colorbox{skyblue}{Operations}}
\quad\\
\begin{table}[h]
\begin{center}
\begin{tabular}{|l|l|}
\hline
operator & explanation\\
\hline
}%
{%end
\hline
\end{tabular}
\end{center}
\end{table}
}
% Examples environment
\newenvironment{ex}%
{%begin
\paragraph{\large \colorbox{skyblue}{Examples}}
\VerbatimEnvironment
\renewcommand{\EveryVerbatim}{\fontencoding{OT1}\selectfont}
\begin{quote}
\begin{Verbatim}
}%
{%end
\end{Verbatim}
\end{quote}
}
%
\definecolor{skyblue}{cmyk}{0.2, 0, 0.1, 0}
\definecolor{skyyellow}{cmyk}{0.1, 0.1, 0.5, 0}
%
%\title{NZMATH User Manual\\ {\large{(for version 1.0)}}}
%\date{}
%\author{}
\begin{document}
%\maketitle
%
\setcounter{tocdepth}{3}
\setcounter{secnumdepth}{3}


\tableofcontents
\C

\chapter{Classes}


%---------- start document ---------- %
 \section{matrix -- matrices}\linkedzero{matrix}
 \begin{itemize}
   \item {\bf Classes}
   \begin{itemize}
     \item \linkingone{matrix}{Matrix}
     \item \linkingone{matrix}{SquareMatrix}
     \item \linkingone{matrix}{RingMatrix}
     \item \linkingone{matrix}{RingSquareMatrix}
     \item \linkingone{matrix}{FieldMatrix}
     \item \linkingone{matrix}{FieldSquareMatrix}
     \item \linkingone{matrix}{MatrixRing}
     \item \linkingone{matrix}{Subspace}
   \end{itemize}
   \item {\bf Functions}
     \begin{itemize}
       \item \linkingone{matrix}{createMatrix}
       \item \linkingone{matrix}{identityMatrix}
       \item \linkingone{matrix}{unitMatrix}
       \item \linkingone{matrix}{zeroMatrix}
     \end{itemize}
 \end{itemize}
 
 The module matrix has also some exception classes.
 \begin{description}
   \item[MatrixSizeError]\linkedone{matrix}{MatrixSizeError}:
     Report contradicting given input to the matrix size.
   \item[VectorsNotIndependent]\linkedone{matrix}{VectorsNotIndependent}:
     Report column vectors are not independent. 
   \item[NoInverseImage]\linkedone{matrix}{NoInverseImage}:
     Report any inverse image does not exist.
   \item[NoInverse]\linkedone{matrix}{NoInverse}:
     Report the matrix is not invertible.
 \end{description}

 This module using following type:
 \begin{description}
   \item[compo]\linkedone{matrix}{compo}:
     \param{compo} must be one of these forms below.\\
     \begin{itemize}
       \item concatenated row lists, such as {\tt [1,2]+[3,4]+[5,6]}.
       \item list of row lists, such as {\tt [[1,2], [3,4], [5,6]]}.
       \item list of column tuples, such as {\tt [(1, 3, 5), (2, 4, 6)]}.
       \item list of vectors whose dimension equals column, such as {\tt vector.Vector([1, 3, 5]), vector.Vector([2, 4, 6])}.
     \end{itemize}
     The examples above represent the same matrix form as follows:
     \begin{equation*}
     \begin{array}{cc}
       1 & 2\\
       3 & 4\\
       5 & 6\\
     \end{array}
     \end{equation*}
 \end{description}

\C

 \subsection{Matrix -- matrices}\linkedone{matrix}{Matrix}
 \initialize
  \func{Matrix}{\hiki{row}{integer},\ \hiki{column}{integer},\ \hikiopt{compo}{compo}{0},\ \hikiopt{coeff\_ring}{CommutativeRing}{0}}{\out{Matrix}}\\
  \spacing
  % document of basic document
  \quad Create new matrices object.\\
  \spacing
  % added document
  \quad \negok This constructor automatically changes the class to one of the following class: \linkingone{matrix}{RingMatrix},\ \linkingone{matrix}{RingSquareMatrix},\ \linkingone{matrix}{FieldMatrix},\ \linkingone{matrix}{FieldSquareMatrix}.\\
  \spacing
  % input, output document
  \quad Your input determines the class automatically by examining the matrix size and the coefficient ring.
   \param{row} and \param{column} must be integer, and \param{coeff\_ring} must be an instance of \linkingone{ring}{Ring}.
   Refer to \linkingone{matrix}{compo} for information about \param{compo}.
   If you abbreviate \param{compo}, it will be deemed to all zero list.\\
   The list of expected inputs and outputs is as following:
   \begin{itemize}
     \item Matrix(\param{row},\ \param{column},\ \param{compo},\ \param{coeff\_ring})\\
       $\to$ the \param{row}$\times$\param{column} matrix whose elements are \param{compo} and coefficient ring is \param{coeff\_ring}
     \item Matrix(\param{row},\ \param{column},\ \param{compo})\\
       $\to$ the \param{row}$\times$\param{column} matrix whose elements are \param{compo} (The coefficient ring is automatically determined.)
     \item Matrix(\param{row},\ \param{column},\ \param{coeff\_ring})\\
       $\to$ the \param{row}$\times$\param{column} matrix whose coefficient ring is \param{coeff\_ring} (All elements are $0$ in \param{coeff\_ring}.)
     \item Matrix(\param{row},\ \param{column})\\
       $\to$ the \param{row}$\times$\param{column} matrix (The coefficient matrix is \linkingone{rational}{Integer}. All elements are $0$.)
   \end{itemize}
  \begin{at}
    \item[row]\linkedtwo{matrix}{Matrix}{row}: The row size of the matrix.\\
    \item[column]\linkedtwo{matrix}{Matrix}{column}: The column size of the matrix.\\
    \item[coeff\_ring]\linkedtwo{matrix}{Matrix}{coeff\_ring}: The coefficient ring of the matrix.\\
    \item[compo]\linkedtwo{matrix}{Matrix}{compo}: The elements of the matrix.\\
  \end{at}
  \begin{op}
    \verb+M==N+ & Return whether \param{M} and \param{N} are equal or not.\\
    \verb+M[i, j]+ & Return the coefficient of \param{i}-th row, \param{j}-th column term of matrix \param{M}.\\
    \verb+M[i]+ & Return the vector of \param{i}-th column term of matrix \param{M}.\\
    \verb+M[i, j]=c+ & Replace the coefficient of \param{i}-th row, \param{j}-th column term of matrix \param{M} by \param{c}.\\
    \verb+M[j]=c+ & Replace the vector of \param{i}-th column term of matrix \param{M} by vector \param{c}.\\
    \verb+c in M+ & Check whether some element of \param{M} equals \param{c}.\\
    \verb+repr(M)+ & Return the repr string of the matrix \param{M}.\\
                   & string represents list concatenated row vector lists.\\
    \verb+str(M)+ & Return the str string of the matrix \param{M}.\\
  \end{op}
\begin{ex}
>>> A = matrix.Matrix(2, 3, [1,0,0]+[0,0,0])
>>> A.__class__.__name__
'RingMatrix'
>>> B = matrix.Matrix(2, 3, [1,0,0,0,0,0])
>>> A == B
True
>>> B[1, 1] = 0
>>> A != B
True
>>> B == 0
True
>>> A[1, 1]
1
>>> print repr(A)
[1, 0, 0]+[0, 0, 0]
>>> print str(A)
1 0 0
0 0 0
\end{ex}%Don't indent!
  \method
\subsubsection{map -- apply function to elements}\linkedtwo{matrix}{Matrix}{map}
   \func{map}{\param{self},\ \hiki{function}{function}}{\out{Matrix}}\\
   \spacing
   % document of basic document
   \quad Return the matrix whose elements is applied \param{function} to.\\
   \spacing
   % added document
   \quad \negok The function map is an analogy of built-in function \href{http://docs.python.org/library/functions.html#map}{map}.\\
   \spacing
   % input, output document
   %\quad \param{a} must be int, long or rational.Integer.\\
 \subsubsection{reduce -- reduce elements iteratively}\linkedtwo{matrix}{Matrix}{reduce}
   \func{reduce}{\param{self},\ \hiki{function}{function},\ \hikiopt{initializer}{RingElement}{None}}{\out{RingElement}}\\
   \spacing
   % document of basic document
   \quad Apply \param{function} from upper-left to lower-right, so as to reduce the iterable to a single value.\\
   \spacing
   % added document
   \quad \negok The function map is an analogy of built-in function \href{http://docs.python.org/library/functions.html#reduce}{reduce}.\\
   \spacing
   % input, output document
   %\quad \param{a} must be int, long or rational.Integer.\\
 \subsubsection{copy -- create a copy}\linkedtwo{matrix}{Matrix}{copy}
   \func{copy}{\param{self}}{\out{Matrix}}\\
   \spacing
   % document of basic document
   \quad create a copy of \param{self}.\\
   \spacing
   % added document
   \quad \negok The matrix generated by the function is same matrix to \param{self}, but not same as a instance.\\
   \spacing
   % input, output document
   %\quad \param{a} must be int, long or rational.Integer.\\
  \subsubsection{set -- set compo}\linkedtwo{matrix}{Matrix}{set}
   \func{set}{\param{self},\ \hiki{compo}{compo}}{\out{(None)}}\\
   \spacing
   % document of basic document
   \quad Substitute the list \param{compo} for \linkingtwo{matrix}{Matrix}{compo}.\\
   \spacing
   % added document
   %\quad\\
   %\spacing
   % input, output document
   \quad \param{compo} must be the form of \linkingtwo{matrix}{Matrix}{compo}.\\
  \subsubsection{setRow -- set m-th row vector}\linkedtwo{matrix}{Matrix}{setRow}
   \func{setRow}{\param{self},\ \hiki{m}{integer},\ \hiki{arg}{list/Vector}}{\out{(None)}}\\
   \spacing
   % document of basic document
   \quad Substitute the list/Vector \param{arg} as \param{m}-th row.\\
   \spacing
   % added document
   %\quad\\
   %\spacing
   % input, output document
   \quad The length of \param{arg} must be same to \param{self}.\linkingtwo{matrix}{Matrix}{column}.\\
  \subsubsection{setColumn -- set n-th column vector}\linkedtwo{matrix}{Matrix}{setColumn}
   \func{setColumn}{\param{self},\ \hiki{n}{integer},\ \hiki{arg}{list/Vector}}{\out{(None)}}\\
   \spacing
   % document of basic document
   \quad Substitute the list/Vector \param{arg} as \param{n}-th column.\\
   \spacing
   % added document
   %\quad\\
   %\spacing
   % input, output document
   \quad The length of \param{arg} must be same to \param{self}.\linkingtwo{matrix}{Matrix}{row}.\\
  \subsubsection{getRow -- get i-th row vector}\linkedtwo{matrix}{Matrix}{getRow}
   \func{getRow}{\param{self},\ \hiki{i}{integer}}{\out{Vector}}\\
   \spacing
   % document of basic document
   \quad Return \param{i}-th row in form of \param{self}.\\
   \spacing
   % added document
   %\quad\\
   %\spacing
   % input, output document
   \quad The function returns a row vector (an instance of \linkingone{vector}{Vector}).\\
  \subsubsection{getColumn -- get j-th column vector}\linkedtwo{matrix}{Matrix}{getColumn}
   \func{getColumn}{\param{self},\ \hiki{j}{integer}}{\out{Vector}}\\
   \spacing
   % document of basic document
   \quad Return \param{j}-th column in form of \param{self}.\\
   \spacing
   % added document
   %\quad\\
   %\spacing
   % input, output document
  \subsubsection{swapRow -- swap two row vectors}\linkedtwo{matrix}{Matrix}{swapRow}
   \func{swapRow}{\param{self},\ \hiki{m1}{integer},\ \hiki{m2}{integer}}{\out{(None)}}\\
   \spacing
   % document of basic document
   \quad Swap \param{self}'s \param{m1}-th row vector and \param{m2}-th row one.\\
   \spacing
   % added document
   %\quad\\
   %\spacing
   % input, output document
  \subsubsection{swapColumn -- swap two column vectors}\linkedtwo{matrix}{Matrix}{swapColumn}
   \func{swapColumn}{\param{self},\ \hiki{n1}{integer},\ \hiki{n2}{integer}}{\out{(None)}}\\
   \spacing
   % document of basic document
   \quad Swap \param{self}'s \param{n1}-th column vector and \param{n2}-th column one.\\
   \spacing
   % added document
   %\quad\\
   %\spacing
   % input, output document
  \subsubsection{insertRow -- insert row vectors}\linkedtwo{matrix}{Matrix}{insertRow}
   \func{insertRow}{\param{self},\ \hiki{i}{integer},\ \hiki{arg}{list/Vector/Matrix}}{\out{(None)}}\\
   \spacing
   % document of basic document
   \quad Insert row vectors \param{arg} to \param{i}-th \param{row}.\\
   \spacing
   % added document
   %\quad\\
   %\spacing
   % input, output document
   \param{arg} must be list, \linkingone{vector}{Vector} or \linkingone{matrix}{Matrix}.
    The length (or \linkingtwo{matrix}{Matrix}{column}) of it should be same to the column of \param{self}. 
  \subsubsection{insertColumn -- insert column vectors}\linkedtwo{matrix}{Matrix}{insertColumn}
   \func{insertColumn}{\param{self},\ \hiki{j}{integer},\ \hiki{arg}{list/Vector/Matrix}}{\out{(None)}}\\
   \spacing
   % document of basic document
   \quad Insert column vectors \param{arg} to \param{j}-th \param{column}.\\
   \spacing
   % added document
   %\quad\\
   %\spacing
   % input, output document
   \param{arg} must be list, \linkingone{vector}{Vector} or \linkingone{matrix}{Matrix}.
    The length (or \linkingtwo{matrix}{Matrix}{row}) of it should be same to the row of \param{self}. 
  \subsubsection{extendRow -- extend row vectors}\linkedtwo{matrix}{Matrix}{extendRow}
   \func{extendRow}{\param{self},\ \hiki{arg}{list/Vector/Matrix}}{\out{(None)}}\\
   \spacing
   % document of basic document
   \quad Join \param{self} with row vectors \param{arg} (in vertical way).\\
   \spacing
   % added document
   \quad The function combines \param{self} with the last row vector of \param{self}.
   That is, extendRow(\param{arg}) is same to insertRow(\param{self}.\param{row}+1,\ \param{arg}).\\
   \spacing
   % input, output document
   \param{arg} must be list, \linkingone{vector}{Vector} or \linkingone{matrix}{Matrix}.
    The length (or \linkingtwo{matrix}{Matrix}{column}) of it should be same to the column of \param{self}. 
  \subsubsection{extendColumn -- extend column vectors}\linkedtwo{matrix}{Matrix}{extendColumn}
   \func{extendColumn}{\param{self},\ \hiki{arg}{list/Vector/Matrix}}{\out{(None)}}\\
   \spacing
   % document of basic document
   \quad Join \param{self} with column vectors \param{arg} (in horizontal way).\\
   \spacing
   % added document
   \quad The function combines \param{self} with the last column vector of \param{self}.
   That is, extendColumn(\param{arg}) is same to insertColumn(\param{self}.\param{column}+1,\ \param{arg}).\\
   \spacing
   % input, output document
   \param{arg} must be list, \linkingone{vector}{Vector} or \linkingone{matrix}{Matrix}.
    The length (or \linkingtwo{matrix}{Matrix}{row}) of it should be same to the row of \param{self}.
  \subsubsection{deleteRow -- delete row vector}\linkedtwo{matrix}{Matrix}{deleteRow}
   \func{deleteRow}{\param{self},\ \hiki{i}{integer}}{\out{(None)}}\\
   \spacing
   % document of basic document
   \quad Delete \param{i}-th row vector.\\
   \spacing
   % added document
   %\quad 
   %\spacing
   % input, output document
  \subsubsection{deleteColumn -- delete column vector}\linkedtwo{matrix}{Matrix}{deleteColumn}
   \func{deleteColumn}{\param{self},\ \hiki{j}{integer}}{\out{(None)}}\\
   \spacing
   % document of basic document
   \quad Delete \param{j}-th column vector.\\
   \spacing
   % added document
   %\quad 
   %\spacing
   % input, output document
  \subsubsection{transpose -- transpose matrix}\linkedtwo{matrix}{Matrix}{transpose}
   \func{transpose}{\param{self}}{\out{Matrix}}\\
   \spacing
   % document of basic document
   \quad Return the transpose of \param{self}.\\
   \spacing
   % added document
   %\quad 
   %\spacing
   % input, output document
  \subsubsection{getBlock -- block matrix}\linkedtwo{matrix}{Matrix}{getBlock}
   \func{getBlock}{\param{self},\ \hiki{i}{integer},\ \hiki{j}{integer},\ \hiki{row}{integer},\ \hikiopt{column}{integer}{None}}{\out{Matrix}}\\
   \spacing
   % document of basic document
   \quad Return the \param{row}$\times$\param{column} block matrix from the (\param{i},\ \param{j})-element. \\
   \spacing
   % added document
   %\quad 
   %\spacing
   % input, output document
   If \param{column} is omitted, \param{column} is considered as same value to \param{row}.
  \subsubsection{subMatrix -- submatrix}\linkedtwo{matrix}{Matrix}{subMatrix}
   \func{subMatrix}{\param{self},\ \hiki{I}{integer},\ \hiki{J}{integer}{None}}{\out{Matrix}}\\
   \func{subMatrix}{\param{self},\ \hiki{I}{list},\ \hikiopt{J}{list}{None}}{\out{Matrix}}\\
   \spacing
   % document of basic document
   \quad The function has a twofold significance.
   \begin{itemize}
     \item \param{I} and \param{J} are integer:\\
       Return submatrix deleted \param{I}-th row and \param{J}-th column.
     \item \param{I} and \param{J} are list:\\
       Return the submatrix composed of elements from \param{self} assigned by rows I and columns J, respectively. 
   \end{itemize}
   \quad\\
   \spacing
   % added document
   %\quad 
   %\spacing
   % input, output document
   \quad If \param{J} is omitted, \param{J} is considered as same value to \param{I}.
\begin{ex}
>>> A = matrix.Matrix(2, 3, [1,2,3]+[4,5,6])
>>> A
[1, 2, 3]+[4, 5, 6]
>>> A.map(complex)
[(1+0j), (2+0j), (3+0j)]+[(4+0j), (5+0j), (6+0j)]
>>> A.reduce(max)
6
>>> A.swapRow(1, 2)
>>> A
[4, 5, 6]+[1, 2, 3]
>>> A.extendColumn([-2, -1])
>>> A
[4, 5, 6, -2]+[1, 2, 3, -1]
>>> B = matrix.Matrix(3, 3, [1,2,3]+[4,5,6]+[7,8,9])
>>> B.subMatrix(2, 3)
[1, 2]+[7, 8]
>>> B.subMatrix([2, 3], [1, 2])
[4, 5]+[7, 8]
\end{ex}%Don't indent!
\C

\subsection{SquareMatrix -- square matrices}\linkedone{matrix}{SquareMatrix}
 \initialize
  \func{SquareMatrix}{\hiki{row}{integer},\ \hikiopt{column}{integer}{0},\ \hikiopt{compo}{compo}{0},\ \hikiopt{coeff\_ring}{CommutativeRing}{0}}{\out{SquareMatrix}}\\
  \spacing
  % document of basic document
  \quad Create new square matrices object.\\
  \spacing
  % added document
  \quad SquareMatrix is subclass of \linkingone{matrix}{Matrix}.
  \negok This constructor automatically changes the class to one of the following class: \linkingone{matrix}{RingMatrix},\ \linkingone{matrix}{RingSquareMatrix},\ \linkingone{matrix}{FieldMatrix},\ \linkingone{matrix}{FieldSquareMatrix}.\\
  \spacing
  % input, output document
  \quad Your input determines the class automatically by examining the matrix size and the coefficient ring.
   \param{row} and \param{column} must be integer, and \param{coeff\_ring} must be an instance of \linkingone{ring}{Ring}.
   Refer to \linkingone{matrix}{compo} for information about \param{compo}.
   If you abbreviate \param{compo}, it will be deemed to all zero list.\\
.\\
   The list of expected inputs and outputs is as following:
   \begin{itemize}
     \item Matrix(\param{row},\ \param{compo},\ \param{coeff\_ring})\\
       $\to$ the \param{row} square matrix whose elements are \param{compo} and coefficient ring is \param{coeff\_ring}
     \item Matrix(\param{row},\ \param{compo})\\
       $\to$ the \param{row} square matrix whose elements are \param{compo} (coefficient ring is automatically determined)
     \item Matrix(\param{row},\ \param{coeff\_ring})\\
       $\to$ the \param{row} square matrix whose coefficient ring is \param{coeff\_ring} (All elements are $0$ in \param{coeff\_ring}.)
     \item Matrix(\param{row})\\
       $\to$ the \param{row} square matrix (The coefficient ring is Integer. All elements are $0$.)
   \end{itemize}
   \negok We can initialize as  \linkingone{matrix}{Matrix}, but \param{column} must be same to \param{row} in the case.
\method
\subsubsection{isUpperTriangularMatrix -- check upper triangular}\linkedtwo{matrix}{SquareMatrix}{isUpperTriangularMatrix}
   \func{isUpperTriangularMatrix}{\param{self}}{\out{True/False}}\\
   \spacing
   % document of basic document
   \quad Check whether \param{self} is upper triangular matrix or not.\\
   \spacing
   % added document
   %\quad 
   %\spacing
   % input, output document
   %\quad \param{a} must be int, long or rational.Integer.\\
\subsubsection{isLowerTriangularMatrix -- check lower triangular}\linkedtwo{matrix}{SquareMatrix}{isLowerTriangularMatrix}
   \func{isLowerTriangularMatrix}{\param{self}}{\out{True/False}}\\
   \spacing
   % document of basic document
   \quad Check whether \param{self} is lower triangular matrix or not.\\
   \spacing
   % added document
   %\quad 
   %\spacing
   % input, output document
   %\quad \param{a} must be int, long or rational.Integer.\\
\subsubsection{isDiagonalMatrix -- check diagonal matrix}\linkedtwo{matrix}{SquareMatrix}{isDiagonalMatrix}
   \func{isDiagonalMatrix}{\param{self}}{\out{True/False}}\\
   \spacing
   % document of basic document
   \quad Check whether \param{self} is diagonal matrix or not.\\
   \spacing
   % added document
   %\quad 
   %\spacing
   % input, output document
   %\quad \param{a} must be int, long or rational.Integer.\\
\subsubsection{isScalarMatrix -- check scalar matrix}\linkedtwo{matrix}{SquareMatrix}{isScalarMatrix}
   \func{isScalarMatrix}{\param{self}}{\out{True/False}}\\
   \spacing
   % document of basic document
   \quad Check whether \param{self} is scalar matrix or not.\\
   \spacing
   % added document
   %\quad 
   %\spacing
   % input, output document
   %\quad \param{a} must be int, long or rational.Integer.\\
\subsubsection{isSymmetricMatrix -- check symmetric matrix}\linkedtwo{matrix}{SquareMatrix}{isSymmetricMatrix}
   \func{isSymmetricMatrix}{\param{self}}{\out{True/False}}\\
   \spacing
   % document of basic document
   \quad Check whether \param{self} is symmetric matrix or not.\\
   \spacing
   % added document
   %\quad 
   %\spacing
   % input, output document
   %\quad \param{a} must be int, long or rational.Integer.\\
\begin{ex}
>>> A = matrix.SquareMatrix(3, [1,2,3]+[0,5,6]+[0,0,9])
>>> A.isUpperTriangularMatrix()
True
>>> B = matrix.SquareMatrix(3, [1,0,0]+[0,-2,0]+[0,0,7])
>>> B.isDiagonalMatrix()
True
\end{ex}%Don't indent!
\C

\subsection{RingMatrix -- matrix whose elements belong ring}\linkedone{matrix}{RingMatrix}
  \func{RingMatrix}{\hiki{row}{integer},\ \hiki{column}{integer},\ \hikiopt{compo}{compo}{0},\ \hikiopt{coeff\_ring}{CommutativeRing}{0}}{\out{RingMatrix}}\\
  \spacing
  % document of basic document
  \quad Create matrix whose coefficient ring belongs ring.\\
  \spacing
  % added document
  \quad RingMatrix is subclass of \linkingone{matrix}{Matrix}.
  See Matrix for getting information about the initialization.\\
  \spacing
  % input, output document
  \begin{op}
    \verb|M+N| & Return the sum of matrices \param{M} and \param{N}.\\
    \verb+M-N+ & Return the difference of matrices \param{M} and \param{N}.\\
    \verb+M*N+ & Return the product of \param{M} and \param{N}. \param{N} must be matrix, vector or scalar\\
    \verb+M % d+ & Return \param{M} modulo \param{d}. \param{d} must be nonzero integer.\\
    \verb+-M+ & Return the matrix whose coefficients have inverted signs of \param{M}.\\
    \verb|+M| & Return the copy of \param{M}.\\
  \end{op}
\begin{ex}
>>> A = matrix.Matrix(2, 3, [1,2,3]+[4,5,6])
>>> B = matrix.Matrix(2, 3, [7,8,9]+[0,-1,-2])
>>> A + B
[8, 10, 12]+[4, 4, 4]
>>> A - B
[-6, -6, -6]+[4, 6, 8]
>>> A * B.transpose()
[50, -8]+[122, -17]
>>> -B * vector.Vector([1, -1, 0])
Vector([1, -1])
>>> 2 * A
[2, 4, 6]+[8, 10, 12]
>>> B % 3
[1, 2, 0]+[0, 2, 1]
\end{ex}%Don't indent!
\method
\subsubsection{getCoefficientRing -- get coefficient ring}\linkedtwo{matrix}{RingMatrix}{getCoefficientRing}
   \func{getCoefficientRing}{\param{self}}{\out{CommutativeRing}}\\
   \spacing
   % document of basic document
   \quad Return the coefficient ring of \param{self}.\\
   \spacing
   % added document
   \quad This method checks all elements of \param{self} and set \param{coeff\_ring} to the valid coefficient ring.\\
   \spacing
   % input, output document
   %\quad \param{a} must be int, long or rational.Integer.\\
  \subsubsection{toFieldMatrix -- set field as coefficient ring}\linkedtwo{matrix}{RingMatrix}{toFieldMatrix}
   \func{toFieldMatrix}{\param{self}}{\out{(None)}}\\
   \spacing
   % document of basic document
   \quad Change the class of the matrix to \linkingone{matrix}{FieldMatrix} or \linkingone{matrix}{FieldSquareMatrix}, where the coefficient ring will be the quotient field of the current domain.\\
   \spacing
   % added document
   % \quad 
   %\spacing
   % input, output document
   %\quad \param{a} must be int, long or rational.Integer.\\
  \subsubsection{toSubspace -- regard as vector space}\linkedtwo{matrix}{RingMatrix}{toSubspace}
   \func{toSubspace}{\param{self},\ \hikiopt{isbasis}{True/False}{None}}{\out{(None)}}\\
   \spacing
   % document of basic document
   \quad Change the class of the matrix to \param{Subspace}, where the coefficient ring will be the quotient field of the current domain.\\
   \spacing
   % added document
   % \quad 
   %\spacing
   % input, output document
   %\quad \param{a} must be int, long or rational.Integer.\\
  \subsubsection{hermiteNormalForm (HNF) -- Hermite Normal Form}\linkedtwo{matrix}{RingMatrix}{hermiteNormalForm}\linkedtwo{matrix}{RingMatrix}{HNF}
   \func{hermiteNormalForm}{\param{self}}{\out{RingMatrix}}\\
   \func{HNF}{\param{self}}{\out{RingMatrix}}\\
   \spacing
   % document of basic document
   \quad Return upper triangular Hermite normal form (HNF).\\
   \spacing
   % added document
   % \quad 
   %\spacing
   % input, output document
   %\quad \param{a} must be int, long or rational.Integer.\\
  \subsubsection{exthermiteNormalForm (extHNF) -- extended Hermite Normal Form algorithm}\linkedtwo{matrix}{RingMatrix}{exthermiteNormalForm}
   \func{exthermiteNormalForm}{\param{self}}{\out{(RingSquareMatrix,\ RingMatrix)}}\\
   \func{extHNF}{\param{self}}{\out{(RingSquareMatrix,\ RingMatrix)}}\\
   \spacing
   % document of basic document
   \quad Return Hermite normal form \param{M} and \param{U} satisfied $\param{self}\param{U}=\param{M}$.\\
   \spacing
   % added document
   % \quad 
   %\spacing
   % input, output document
   \quad  The function returns tuple (\param{U},\ \param{M}), where \param{U} is an instance of \linkingone{matrix}{RingSquareMatrix} and \param{M} is an instance of \linkingone{matrix}{RingMatrix}.\\
  \subsubsection{kernelAsModule -- kernel as $\mathbb{Z}$-module}\linkedtwo{matrix}{RingMatrix}{kernelAsModule}
   \func{kernelAsModule}{\param{self}}{\out{RingMatrix}}\\
   \spacing
   % document of basic document
   \quad Return kernel as $\mathbb{Z}$-module.\\
   \spacing
   % added document
   \quad The difference between the function and \linkingtwo{matrix}{FieldMatrix}{kernel} is that each elements of the returned value are integer.\\
   \spacing
   % input, output document
   %\quad
\begin{ex}
>>> A = matrix.Matrix(3, 4, [1,2,3,4,5,6,7,8,9,-1,-2,-3])
>>> print A.hermiteNormalForm()
0 36 29 28
0  0  1  0
0  0  0  1
>>> U, M = A.hermiteNormalForm()
>>> A * U == M
True
>>> B = matrix.Matrix(1, 2, [2, 1])
>>> print B.kernelAsModule()
1
-2
\end{ex}
\C

\subsection{RingSquareMatrix -- square matrix whose elements belong ring}\linkedone{matrix}{RingSquareMatrix}
  \func{RingSquareMatrix}{\hiki{row}{integer},\ \hikiopt{column}{integer}{0},\ \hikiopt{compo}{compo}{0},\ \hikiopt{coeff\_ring}{CommutativeRing}{0}}{\out{RingMatrix}}\\
  \spacing
  % document of basic document
  \quad Create square matrix whose coefficient ring belongs ring.\\
  \spacing
  % added document
  \quad RingSquareMatrix is subclass of \linkingone{matrix}{RingMatrix} and \linkingone{matrix}{SquareMatrix}.
  See SquareMatrix for getting information about the initialization.\\
  \spacing
  % input, output document
  \begin{op}
    \verb|M**c| & Return the \param{c}-th power of matrices \param{M}.\\
  \end{op}
\begin{ex}
>>> A = matrix.RingSquareMatrix(3, [1,2,3]+[4,5,6]+[7,8,9])
>>> A ** 2
[30L, 36L, 42L]+[66L, 81L, 96L]+[102L, 126L, 150L]
\end{ex}%Don't indent!
\method
  \subsubsection{getRing -- get matrix ring}\linkedtwo{matrix}{RingSquareMatrix}{getRing}
   \func{getRing}{\param{self}}{\out{MatrixRing}}\\
   \spacing
   % document of basic document
   \quad Return the \linkingone{matrix}{MatrixRing} belonged to by \param{self}.\\
   \spacing
   % added document
   %\quad 
   %\spacing
   % input, output document
   %\quad \param{a} must be int, long or rational.Integer.\\
  \subsubsection{isOrthogonalMatrix -- check orthogonal matrix}\linkedtwo{matrix}{RingSquareMatrix}{isOrthogonalMatrix}
   \func{isOrthogonalMatrix}{\param{self}}{\out{True/False}}\\
   \spacing
   % document of basic document
   \quad Check whether \param{self} is orthogonal matrix or not.\\
   \spacing
   % added document
   %\quad 
   %\spacing
   % input, output document
   %\quad \param{a} must be int, long or rational.Integer.\\
  \subsubsection{isAlternatingMatrix (isAntiSymmetricMatrix,\ isSkewSymmetricMatrix) -- check alternating matrix}\linkedtwo{matrix}{RingSquareMatrix}{isAlternatingMatrix}
   \func{isAlternatingMatrix}{\param{self}}{\out{True/False}}\\
   \spacing
   % document of basic document
   \quad Check whether \param{self} is alternating matrix or not.\\
   \spacing
   % added document
   %\quad 
   %\spacing
   % input, output document
   %\quad \param{a} must be int, long or rational.Integer.\\
  \subsubsection{isSingular -- check singular matrix}\linkedtwo{matrix}{RingSquareMatrix}{isSingular}
   \func{isSingular}{\param{self}}{\out{True/False}}\\
   \spacing
   % document of basic document
   \quad Check whether \param{self} is singular matrix or not.\\
   \spacing
   % added document
   \quad The function determines whether determinant of \param{self} is $0$.
   Note that the the non-singular matrix does not automatically mean invertible matrix; the nature that the matrix is invertible depends on its coefficient ring.\\
   \spacing
   % input, output document
   %\quad \param{a} must be int, long or rational.Integer.\\
  \subsubsection{trace -- trace}\linkedtwo{matrix}{RingSquareMatrix}{trace}
   \func{trace}{\param{self}}{\out{RingElement}}\\
   \spacing
   % document of basic document
   \quad Return the trace of \param{self}.\\
   \spacing
   % added document
   %\quad 
   % input, output document
   %\quad \param{a} must be int, long or rational.Integer.\\
  \subsubsection{determinant -- determinant}\linkedtwo{matrix}{RingSquareMatrix}{determinant}
   \func{determinant}{\param{self}}{\out{RingElement}}\\
   \spacing
   % document of basic document
   \quad Return the determinant of \param{self}.\\
   \spacing
   % added document
   %\quad 
   % input, output document
   %\quad \param{a} must be int, long or rational.Integer.\\
  \subsubsection{cofactor -- cofactor}\linkedtwo{matrix}{RingSquareMatrix}{cofactor}
   \func{cofactor}{\param{self},\ \hiki{i}{integer},\ \hiki{j}{integer}}{\out{RingElement}}\\
   \spacing
   % document of basic document
   \quad Return the (\param{i},\ \param{j})-cofactor.\\
   \spacing
   % added document
   %\quad 
   % input, output document
   %\quad \param{a} must be int, long or rational.Integer.\\
  \subsubsection{commutator -- commutator}\linkedtwo{matrix}{RingSquareMatrix}{commutator}
   \func{commutator}{\param{self},\ \hiki{N}{RingSquareMatrix element}}{\out{RingSquareMatrix}}\\
   \spacing
   % document of basic document
   \quad Return the commutator for \param{self} and \param{N}.\\
   \spacing
   % added document
   \quad The commutator for \param{M} and \param{N}, which is denoted as $[M,\ N]$, is defined as $[\param{M},\ \param{N}]=\param{M}\param{N}-\param{N}\param{M}$. 
   % input, output document
   %\quad \param{a} must be int, long or rational.Integer.\\
  \subsubsection{characteristicMatrix -- characteristic matrix}\linkedtwo{matrix}{RingSquareMatrix}{characteristicMatrix}
   \func{characteristicMatrix}{\param{self}}{\out{RingSquareMatrix}}\\
   \spacing
   % document of basic document
   \quad Return the characteristic matrix of \param{self}.\\
   \spacing
   % added document
   %\quad 
   % input, output document
   %\quad \param{a} must be int, long or rational.Integer.\\
  \subsubsection{adjugateMatrix -- adjugate matrix}\linkedtwo{matrix}{RingSquareMatrix}{adjugateMatrix}
   \func{adjugateMatrix}{\param{self}}{\out{RingSquareMatrix}}\\
   \spacing
   % document of basic document
   \quad Return the adjugate matrix of \param{self}.\\
   \spacing
   % added document
   \quad The adjugate matrix for \param{M} is the matrix \param{N} such that $\param{M}\param{N}=\param{N}\param{M}=(\det{\param{M}})E$, where $E$ is the identity matrix.\\
   % input, output document
   %\quad \param{a} must be int, long or rational.Integer.\\
  \subsubsection{cofactorMatrix (cofactors) -- cofactor matrix}\linkedtwo{matrix}{RingSquareMatrix}{cofactorMatrix}\linkedtwo{matrix}{RingSquareMatrix}{cofactors}
   \func{cofactorMatrix}{\param{self}}{\out{RingSquareMatrix}}\\
   \func{cofactors}{\param{self}}{\out{RingSquareMatrix}}\\
   \spacing
   % document of basic document
   \quad Return the cofactor matrix of \param{self}.\\
   \spacing
   % added document
   \quad The cofactor matrix for \param{M} is the matrix whose ($i,\ j$) element is  ($i,\ j$)-cofactor of \param{M}.
    The cofactor matrix is same to transpose of the adjugate matrix.\\
   % input, output document
   %\quad \param{a} must be int, long or rational.Integer.\\
  \subsubsection{smithNormalForm (SNF,\ elementary\_divisor) -- Smith Normal Form (SNF)}\linkedtwo{matrix}{RingSquareMatrix}{smithNormalForm}\linkedtwo{matrix}{RingSquareMatrix}{SNF}\linkedtwo{matrix}{RingSquareMatrix}{elementary\_divisor}
   \func{smithNormalForm}{\param{self}}{\out{RingSquareMatrix}}\\
   \func{SNF}{\param{self}}{\out{RingSquareMatrix}}\\
   \func{elementary\_divisor}{\param{self}}{\out{RingSquareMatrix}}\\
   \spacing
   % document of basic document
   \quad Return the list of diagonal elements of the Smith Normal Form (SNF) for \param{self}.\\
   \spacing
   % added document
   \quad The function assumes that \param{self} is non-singular.\\
   % input, output document
   %\quad \param{a} must be int, long or rational.Integer.\\
  \subsubsection{extsmithNormalForm (extSNF) -- Smith Normal Form (SNF)}\linkedtwo{matrix}{RingSquareMatrix}{extsmithNormalForm}\linkedtwo{matrix}{RingSquareMatrix}{extSNF}
   \func{extsmithNormalForm}{\param{self}}{\out{(RingSquareMatrix,\ RingSquareMatrix,\ RingSquareMatrix)}}\\
   \func{extSNF}{\param{self}}{\out{RingSquareMatrix,\ RingSquareMatrix,\ RingSquareMatrix)}}\\
   \spacing
   % document of basic document
   \quad Return the Smith normal form \param{M} for \param{self} and \param{U},\param{V} satisfied $\param{U}\param{self}\param{V}=\param{M}$.\\
   \spacing
   % added document
   % input, output document
   %\quad
\begin{ex}
>>> A = matrix.RingSquareMatrix(3, [3,-5,8]+[-9,2,7]+[6,1,-4])
>>> A.trace()
1L
>>> A.determinant()
-243L
>>> B = matrix.RingSquareMatrix(3, [87,38,80]+[13,6,12]+[65,28,60])
>>> U, V, M = B.extsmithNormalForm()
>>> U * B * V == M
True
>>> print M
4 0 0
0 2 0
0 0 1
>>> B.smithNormalForm()
[4L, 2L, 1L]
\end{ex}
\C

\subsection{FieldMatrix -- matrix whose elements belong field}\linkedone{matrix}{FieldMatrix}
  \func{FieldMatrix}{\hiki{row}{integer},\ \hiki{column}{integer},\ \hikiopt{compo}{compo}{0},\ \hikiopt{coeff\_ring}{CommutativeRing}{0}}{\out{RingMatrix}}\\
  \spacing
  % document of basic document
  \quad Create matrix whose coefficient ring belongs field.\\
  \spacing
  % added document
  \quad FieldMatrix is subclass of \linkingone{matrix}{RingMatrix}.
  See \linkingone{matrix}{Matrix} for getting information about the initialization.\\
  \spacing
  % input, output document
  \begin{op}
    \verb|M/d| & Return the division of \param{M} by \param{d}.\param{d} must be scalar.\\
    \verb|M//d| & Return the division of \param{M} by \param{d}.\param{d} must be scalar.\\
  \end{op}
\begin{ex}
>>> A = matrix.FieldMatrix(3, 3, [1,2,3,4,5,6,7,8,9])
>>> A / 210
1/210 1/105 1/70
2/105  1/42 1/35
1/30  4/105 3/70
\end{ex}%Don't indent!
\method
  \subsubsection{kernel -- kernel}\linkedtwo{matrix}{FieldMatrix}{kernel}
   \func{kernel}{\param{self}}{\out{FieldMatrix}}\\
   \spacing
   % document of basic document
   \quad Return the kernel of \param{self}.\\
   \spacing
   % added document
   %\quad 
   %\spacing
   % input, output document
   \quad The output is the matrix whose column vectors form basis of the kernel.\\
   The function returns None if the kernel do not exist.\\
  \subsubsection{image -- image}\linkedtwo{matrix}{FieldMatrix}{image}
   \func{image}{\param{self}}{\out{FieldMatrix}}\\
   \spacing
   % document of basic document
   \quad Return the image of \param{self}.\\
   \spacing
   % added document
   %\quad 
   %\spacing
   % input, output document
   \quad The output is the matrix whose column vectors form basis of the image.\\
   The function returns None if the kernel do not exist.\\
  \subsubsection{rank -- rank}\linkedtwo{matrix}{FieldMatrix}{rank}
   \func{rank}{\param{self}}{\out{integer}}\\
   \spacing
   % document of basic document
   \quad Return the rank of \param{self}.\\
   \spacing
   % added document
   %\quad 
   %\spacing
   % input, output document
   %\quad \\
  \subsubsection{inverseImage -- inverse image: base solution of linear system}\linkedtwo{matrix}{FieldMatrix}{inverseImage}
   \func{inverseImage}{\param{self},\ \hiki{V}{Vector/RingMatrix}}{\out{RingMatrix}}\\
   \spacing
   % document of basic document
   \quad Return an inverse image of \param{V} by \param{self}.\\
   \spacing
   % added document
   \quad The function returns one solution of the linear equation $\param{self}X=\param{V}$.
   \spacing
   % input, output document
   %\quad \\
  \subsubsection{solve -- solve linear system}\linkedtwo{matrix}{FieldMatrix}{solve}
   \func{solve}{\param{self},\ \hiki{B}{Vector/RingMatrix}}{\out{(RingMatrix,\ RingMatrix)}}\\
   \spacing
   % document of basic document
   \quad Solve $\param{self}X = \param{B}$.\\
   \spacing
   % added document
   \quad The function returns a particular solution \param{sol} and the kernel of \param{self} as a matrix.
   If you only have to obtain the particular solution, use \linkingtwo{matrix}{FieldMatrix}{inverseImage}.
   \spacing
   % input, output document
  \subsubsection{columnEchelonForm -- column echelon form}\linkedtwo{matrix}{FieldMatrix}{columnEchelonForm}
   \func{columnEchelonForm}{\param{self}}{\out{RingMatrix}}\\
   % document of basic document
   \quad Return the column reduced echelon form.\\
   \spacing
   % added document
   %\quad 
   %\spacing
   % input, output document
   %\quad
\begin{ex}
>>> A = matrix.FieldMatrix(2, 3, [1,2,3]+[4,5,6])
>>> print A.kernel
 1/1
-2/1
   1
>>> print A.image()
1 2
4 5
>>> C = matrix.FieldMatrix(4, 3, [1,2,3]+[4,5,6]+[7,8,9]+[-1,-2,-3])
>>> D = matrix.FieldMatrix(4, 2, [1,0]+[7,6]+[13,12]+[-1,0])
>>> print C.inverseImage(D)
 3/1  4/1
-1/1 -2/1
 0/1  0/1
>>> sol, ker = C.solve(D)
>>> C * (sol + ker[0]) == D
True
>>> AA = matrix.FieldMatrix(3, 3, [1,2,3]+[4,5,6]+[7,8,9])
>>> print AA.columnEchelonForm()
0/1 2/1 -1/1
0/1 1/1  0/1
0/1 0/1  1/1
\end{ex}
\C

\subsection{FieldSquareMatrix -- square matrix whose elements belong field}\linkedone{matrix}{FieldSquareMatrix}
  \func{FieldSquareMatrix}{\hiki{row}{integer},\ \hikiopt{column}{integer}{0},\ \hikiopt{compo}{compo}{0},\ \hikiopt{coeff\_ring}{CommutativeRing}{0}}{\out{FieldSquareMatrix}}\\
  \spacing
  % document of basic document
  \quad Create square matrix whose coefficient ring belongs field.\\
  \spacing
  % added document
  \quad FieldSquareMatrix is subclass of \linkingone{matrix}{FieldMatrix} and \linkingone{matrix}{SquareMatrix}.\\
  \negok The function \linkingone{matrix}{RingSquareMatrix}{determinant} is overridden and use different algorithm from one used in \linkingone{matrix}{RingSquareMatrix}{determinant};the function calls \linkingone{matrix}{FieldSquareMatrix}{triangulate}.
  See \linkingone{matrix}{SquareMatrix} for getting information about the initialization.\\
  \spacing
  % input, output document
 \method
  \subsubsection{triangulate - triangulate by elementary row operation}\linkedtwo{matrix}{FieldSquareMatrix}{triangulate}
   \func{triangulate}{\param{self}}{\out{FieldSquareMatrix}}\\
   \spacing
   % document of basic document
   \quad Return an upper triangulated matrix obtained by elementary row operations.\\
   \spacing
   % added document
   %\quad 
   %\spacing
   % input, output document
   %\quad
  \subsubsection{inverse - inverse matrix}\linkedtwo{matrix}{FieldSquareMatrix}{inverse}
   \func{inverse}{\param{self}\ \hikiopt{V}{Vector/RingMatrix}{None}}{\out{FieldSquareMatrix}}\\
   \spacing
   % document of basic document
   \quad Return the inverse of \param{self}.
   If \param{V} is given, then return $\param{self}^(-1)V$.\\ 
   \spacing
   % added document
   \quad \negok If the matrix is not invertible, then raise \linkingone{matrix}{NoInverse}.\\
   \spacing
   % input, output document
   %\quad
  \subsubsection{hessenbergForm - Hessenberg form}\linkedtwo{matrix}{FieldSquareMatrix}{hessenbergForm}
   \func{hessenbergForm}{\param{self}}{\out{FieldSquareMatrix}}\\
   \spacing
   % document of basic document
   \quad Return the Hessenberg form of \param{self}.\\
   \spacing
   % added document
   %\quad 
   %\spacing
   % input, output document
   %\quad
  \subsubsection{LUDecomposition - LU decomposition}\linkedtwo{matrix}{FieldSquareMatrix}{LUDecomposition}
   \func{LUDecomposition}{\param{self}}{\out{(FieldSquareMatrix,\ FieldSquareMatrix)}}\\
   \spacing
   % document of basic document
   \quad Return the lower triangular matrix \param{L} and the upper triangular matrix \param{U} such that $\param{self} == \param{L}\param{U}$.\\
   \spacing
   % added document
   %\quad 
   %\spacing
   % input, output document
   %\quad
\C

\subsection{\negok MatrixRing -- ring of matrices}\linkedone{matrix}{MatrixRing}
  \func{MatrixRing}{\hiki{size}{integer},\ \hiki{scalars}{CommutativeRing}}{\out{MatrixRing}}\\
  \spacing
  % document of basic document
  \quad Create a ring of matrices with given \param{size} and coefficient ring \param{scalars}.\\
  \spacing
  % added document
  \quad MatrixRing is subclass of \linkingone{ring}{Ring}.\\
  \spacing
  % input, output document
 \method
  \subsubsection{unitMatrix - unit matrix}\linkedtwo{matrix}{MatrixRing}{unitMatrix}
   \func{unitMatrix}{\param{self}}{\out{RingSquareMatrix}}\\
   \spacing
   % document of basic document
   \quad Return the unit matrix.\\
   \spacing
   % added document
   %\quad 
   %\spacing
   % input, output document
   %\quad
  \subsubsection{zeroMatrix - zero matrix}\linkedtwo{matrix}{MatrixRing}{zeroMatrix}
   \func{zeroMatrix}{\param{self}}{\out{RingSquareMatrix}}\\
   \spacing
   % document of basic document
   \quad Return the zero matrix.\\
   \spacing
   % added document
   %\quad 
   %\spacing
   % input, output document
   %\quad
\begin{ex}
>>> M = matrix.MatrixRing(3, rational.theIntegerRing)
>>> print M
M_3(Z)
>>> M.unitMatrix()
[1L, 0L, 0L]+[0L, 1L, 0L]+[0L, 0L, 1L]
>>> M.zero
[0L, 0L, 0L]+[0L, 0L, 0L]+[0L, 0L, 0L]
\end{ex}
\C
 \subsubsection{getInstance(class function) - get cached instance}\linkedtwo{matrix}{MatrixRing}{getInstance}
   \func{getInstance}{\param{cls},\ \hiki{size}{integer},\ \hiki{scalars}{CommutativeRing}}{\out{RingSquareMatrix}}\\
   \spacing
   % document of basic document
   \quad Return an instance of MatrixRing of given \param{size} and ring of scalars. \\
   \spacing
   % added document
   \quad The merit of using the method instead of the constructor is that the instances created by the method are cached and reused for efficiency.\\
   \spacing
   % input, output document
   %\quad
\begin{ex}
>>> print MatrixRing.getInstance(3, rational.theIntegerRing)
M_3(Z)
\end{ex}
\C

\subsection{Subspace -- subspace of finite dimensional vector space}\linkedone{matrix}{Subspace}
  \func{Subspace}{\hiki{row}{integer},\ \hikiopt{column}{integer}{0},\ \hikiopt{compo}{compo}{0},\ \hikiopt{coeff\_ring}{CommutativeRing}{0},\ \hikiopt{isbasis}{True/False}{None}}{\out{Subspace}}\\
  \spacing
  % document of basic document
  \quad Create subspace of some finite dimensional vector space over a field.\\
  \spacing
  % added document
  \quad Subspace is subclass of \linkingone{matrix}{FieldMatrix}.\\
  See \linkingone{matrix}{Matrix} for getting information about the initialization.
  The subspace expresses the space generated by column vectors of \param{self}.\\
  \spacing
  % input, output document
  If \param{isbasis} is True, we assume that column vectors are linearly independent.
 \begin{at}
   \item[isbasis] The attribute indicates the linear independence of column vectors, i.e., if they form a basis of the space then \param{isbasis} should be True, otherwise False.
 \end{at}
 \method
  \subsubsection{issubspace - check subspace of self}\linkedtwo{matrix}{Subspace}{triangulate}
   \func{Subspace}{\param{self},\ \hiki{other}{Subspace}}{\out{True/False}}\\
   \spacing
   % document of basic document
   \quad Return True if the subspace instance is a subspace of the \param{other}, or False otherwise.\\
   \spacing
   % added document
   %\quad 
   %\spacing
   % input, output document
   %\quad
  \subsubsection{toBasis - select basis}\linkedtwo{matrix}{Subspace}{toBasis}
   \func{toBasis}{\param{self}}{\out{(None)}}\\
   \spacing
   % document of basic document
   \quad Rewrite \param{self} so that its column vectors form a basis, and set True to its \param{isbasis}.\\
   \spacing
   % added document
   \quad The function does nothing if \param{isbasis} is already True.\\
   \spacing
   % input, output document
   %\quad
  \subsubsection{supplementBasis - to full rank}\linkedtwo{matrix}{Subspace}{supplementBasis}
   \func{supplementBasis}{\param{self}}{\out{Subspace}}\\
   \spacing
   % document of basic document
   \quad Return full rank matrix by supplementing bases for \param{self}.\\
   \spacing
   % added document
   %\quad
   %\spacing
   % input, output document
   %\quad
  \subsubsection{sumOfSubspaces - sum as subspace}\linkedtwo{matrix}{Subspace}{sumOfSubspaces}
   \func{sumOfSubspaces}{\param{self},\ \hiki{other}{Subspace}}{\out{Subspace}}\\
   \spacing
   % document of basic document
   \quad Return a matrix whose columns form a basis for sum of two subspaces.\\
   \spacing
   % added document
   %\quad
   %\spacing
   % input, output document
   %\quad
  \subsubsection{intersectionOfSubspaces - intersection as subspace}\linkedtwo{matrix}{Subspace}{intersectionOfSubspaces}
   \func{intersectionOfSubspaces}{\param{self},\ \hiki{other}{Subspace}}{\out{Subspace}}\\
   \spacing
   % document of basic document
   \quad Return a matrix whose columns form a basis for intersection of two subspaces.\\
   \spacing
   % added document
   %\quad
   %\spacing
   % input, output document
   %\quad
\begin{ex}
>>> A = matrix.Subspace(4, 3, [1,2,3]+[4,5,6]+[7,8,9]+[10,11,12])
>>> A.toBasis()
>>> print A
 1  2
 4  5
 7  8
10 11
>>> B = matrix.Subspace(3, 2, [1,2]+[3,4]+[5,7])
>>> print B.supplementBasis()
1 2 0
3 4 0
5 7 1
>>> C = matrix.Subspace(4, 1, [1,2,3,4])
>>> D = matrix.Subspace(4, 2, [2,-4]+[4,-3]+[6,-2]+[8,-1])
>>> print C.intersectionOfSubspaces(D)
-2/1
-4/1
-6/1
-8/1
\end{ex}
\C

 \subsubsection{fromMatrix(class function) - create subspace}\linkedtwo{matrix}{Subspace}{fromMatrix}
   \func{fromMatrix}{\param{cls},\ \hiki{mat}{FieldMatrix},\ \hikiopt{isbasis}{True/False}{None}}{\out{Subspace}}\\
   \spacing
   % document of basic document
   \quad Create a Subspace instance from a matrix instance \param{mat}, whose class can be any of subclasses of Matrix.\\
   \spacing
   % added document
   \quad Please use this method if you want a Subspace instance for sure.\\
   \spacing
   % input, output document
   %\quad
\C

 \subsection{createMatrix[function] -- create an instance}\linkedone{matrix}{createMatrix}
  \func{createMatrix}{\hiki{row}{integer},\ \hikiopt{column}{integer}{0},\ \hikiopt{compo}{compo}{0},\ \hikiopt{coeff\_ring}{CommutativeRing}{None}}{\param{RingMatrix}}\\
   \spacing
   % document of basic document
   \quad Create an instance of \linkingone{matrix}{RingMatrix}, \linkingone{matrix}{RingSquareMatrix}, \linkingone{matrix}{FieldMatrix} or \linkingone{matrix}{FieldSquareMatrix}.\\
   \spacing
   % added document
   \quad Your input determines the class automatically by examining the matrix size and the coefficient ring.
   See \linkingone{matrix}{Matrix} or \linkingone{matrix}{SquareMatrix} for getting information about the initialization.\\
   \spacing
   % input, output document
 \subsection{identityMatrix(unitMatrix)[function] -- unit matrix}\linkedone{matrix}{identityMatrix}\linkedone{matrix}{unitMatrix}
  \func{identityMatrix}{\hiki{size}{integer},\ \hikiopt{coeff}{CommutativeRing/CommutativeRingElement}{None}}{\param{RingMatrix}}\\
    \func{unitMatrix}{\hiki{size}{integer},\ \hikiopt{coeff}{CommutativeRing/CommutativeRingElement}{None}}{\param{RingMatrix}}\\
   \spacing
   % document of basic document
   \quad Return \param{size}-dimensional unit matrix.\\
   \spacing
   % added document
   \quad \param{coeff} enables us to create matrix not only in integer but in coefficient ring which is determined by coeff. \\
   \spacing
   % input, output document
    \param{coeff} must be an instance of \linkingone{ring}{Ring} or a multiplicative unit (one).
 \subsection{zeroMatrix[function] -- zero matrix}\linkedone{matrix}{zeroMatrix}
  \func{zeroMatrix}{\hiki{row}{integer},\ \hikiopt{column}{0},\ \hikiopt{coeff}{CommutativeRing/CommutativeRingElement}{None}}{\param{RingMatrix}}\\
   \spacing
   % document of basic document
   \quad Return $\param{row}\times\param{column}$ zero matrix.\\
   \spacing
   % added document
   \quad \param{coeff} enables us to create matrix not only in integer but in coefficient ring which is determined by coeff. \\
   \spacing
   % input, output document
    \param{coeff} must be an instance of \linkingone{ring}{Ring} or a additive unit (zero).
    If \param{column} is abbreviated, \param{column} is set same to \param{row}.\begin{ex}
>>> M = matrix.createMatrix(3, [1,2,3]+[4,5,6]+[7,8,9])
>>> print M
1 2 3
4 5 6
7 8 9
>>> O = matrix.zeroMatrix(2, 3, imaginary.ComplexField())
>>> print O
0 + 0j 0 + 0j 0 + 0j
0 + 0j 0 + 0j 0 + 0j
\end{ex}
\C

%---------- end document ---------- %

\bibliographystyle{jplain}%use jbibtex
\bibliography{nzmath_references}

\end{document}


%%%%%%%%%%%%%%%%%%%%%%%%%%%%%%%%%%%%%%%%%%%%%%%%%%%%%%%%%%%%%%
%
% macros for nzmath manual
%
%%%%%%%%%%%%%%%%%%%%%%%%%%%%%%%%%%%%%%%%%%%%%%%%%%%%%%%%%%%%%
\usepackage{amssymb,amsmath}
\usepackage{color}
\usepackage[dvipdfm,bookmarks=true,bookmarksnumbered=true,%
 pdftitle={NZMATH Users Manual},%
 pdfsubject={Manual for NZMATH Users},%
 pdfauthor={NZMATH Development Group},%
 pdfkeywords={TeX; dvipdfmx; hyperref; color;},%
 colorlinks=true]{hyperref}
\usepackage{fancybox}
\usepackage[T1]{fontenc}
%
\newcommand{\DS}{\displaystyle}
\newcommand{\C}{\clearpage}
\newcommand{\NO}{\noindent}
\newcommand{\negok}{$\dagger$}
\newcommand{\spacing}{\vspace{1pt}\\ }
% software macros
\newcommand{\nzmathzero}{{\footnotesize $\mathbb{N}\mathbb{Z}$}\texttt{MATH}}
\newcommand{\nzmath}{{\nzmathzero}\ }
\newcommand{\pythonzero}{$\mbox{\texttt{Python}}$}
\newcommand{\python}{{\pythonzero}\ }
% link macros
\newcommand{\linkingzero}[1]{{\bf \hyperlink{#1}{#1}}}%module
\newcommand{\linkingone}[2]{{\bf \hyperlink{#1.#2}{#2}}}%module,class/function etc.
\newcommand{\linkingtwo}[3]{{\bf \hyperlink{#1.#2.#3}{#3}}}%module,class,method
\newcommand{\linkedzero}[1]{\hypertarget{#1}{}}
\newcommand{\linkedone}[2]{\hypertarget{#1.#2}{}}
\newcommand{\linkedtwo}[3]{\hypertarget{#1.#2.#3}{}}
\newcommand{\linktutorial}[1]{\href{http://docs.python.org/tutorial/#1}{#1}}
\newcommand{\linktutorialone}[2]{\href{http://docs.python.org/tutorial/#1}{#2}}
\newcommand{\linklibrary}[1]{\href{http://docs.python.org/library/#1}{#1}}
\newcommand{\linklibraryone}[2]{\href{http://docs.python.org/library/#1}{#2}}
\newcommand{\pythonhp}{\href{http://www.python.org/}{\python website}}
\newcommand{\nzmathwiki}{\href{http://nzmath.sourceforge.net/wiki/}{{\nzmathzero}Wiki}}
\newcommand{\nzmathsf}{\href{http://sourceforge.net/projects/nzmath/}{\nzmath Project Page}}
\newcommand{\nzmathtnt}{\href{http://tnt.math.se.tmu.ac.jp/nzmath/}{\nzmath Project Official Page}}
% parameter name
\newcommand{\param}[1]{{\tt #1}}
% function macros
\newcommand{\hiki}[2]{{\tt #1}:\ {\em #2}}
\newcommand{\hikiopt}[3]{{\tt #1}:\ {\em #2}=#3}

\newdimen\hoge
\newdimen\truetextwidth
\newcommand{\func}[3]{%
\setbox0\hbox{#1(#2)}
\hoge=\wd0
\truetextwidth=\textwidth
\advance \truetextwidth by -2\oddsidemargin
\ifdim\hoge<\truetextwidth % short form
{\bf \colorbox{skyyellow}{#1(#2)\ $\to$ #3}}
%
\else % long form
\fcolorbox{skyyellow}{skyyellow}{%
   \begin{minipage}{\textwidth}%
   {\bf #1(#2)\\ %
    \qquad\quad   $\to$\ #3}%
   \end{minipage}%
   }%
\fi%
}

\newcommand{\out}[1]{{\em #1}}
\newcommand{\initialize}{%
  \paragraph{\large \colorbox{skyblue}{Initialize (Constructor)}}%
    \quad\\ %
    \vspace{3pt}\\
}
\newcommand{\method}{\C \paragraph{\large \colorbox{skyblue}{Methods}}}
% Attribute environment
\newenvironment{at}
{%begin
\paragraph{\large \colorbox{skyblue}{Attribute}}
\quad\\
\begin{description}
}%
{%end
\end{description}
}
% Operation environment
\newenvironment{op}
{%begin
\paragraph{\large \colorbox{skyblue}{Operations}}
\quad\\
\begin{table}[h]
\begin{center}
\begin{tabular}{|l|l|}
\hline
operator & explanation\\
\hline
}%
{%end
\hline
\end{tabular}
\end{center}
\end{table}
}
% Examples environment
\newenvironment{ex}%
{%begin
\paragraph{\large \colorbox{skyblue}{Examples}}
\VerbatimEnvironment
\renewcommand{\EveryVerbatim}{\fontencoding{OT1}\selectfont}
\begin{quote}
\begin{Verbatim}
}%
{%end
\end{Verbatim}
\end{quote}
}
%
\definecolor{skyblue}{cmyk}{0.2, 0, 0.1, 0}
\definecolor{skyyellow}{cmyk}{0.1, 0.1, 0.5, 0}
%
%\title{NZMATH User Manual\\ {\large{(for version 1.0)}}}
%\date{}
%\author{}
\begin{document}
%\maketitle
%
\setcounter{tocdepth}{3}
\setcounter{secnumdepth}{3}


\tableofcontents
\C

\chapter{Classes}


%---------- start document ---------- %
 \section{permute -- �u��(�Ώ�)�Q}\linkedzero{permute}
 \begin{itemize}
   \item {\bf Classes}
   \begin{itemize}
     \item \linkingone{permute}{Permute}
     \item \linkingone{permute}{ExPermute}
     \item \linkingone{permute}{PermGroup}
   \end{itemize}
 \end{itemize}

\C

 \subsection{Permute -- �u���Q�̌�}\linkedone{permute}{Permute}
 \initialize
  \func{Permute}{\hiki{value}{list/tuple},\ \hiki{key}{list/tuple}}{Permute}\\
  \func{Permute}{\hiki{val\_key}{dict}}{Permute}\\
  \func{Permute}{\hiki{value}{list/tuple}, \hikiopt{key}{int}{None}}{Permute}\\
  \spacing
  % document of basic document
  \quad �u���Q�̌���V�����쐬.\\
  \spacing
  % added document
  \quad �C���X�^���X��``���ʂ�''���@�ō쐬�����.
  ���Ȃ킿,����W����(�C���f�b�N�X�t����ꂽ)�S�Ă̌��̃��X�g�ł���\param{key}��,�S�Ă̒u�����ꂽ���̃��X�g�ł���\param{value}�����.\\
  \spacing
  % input, output document
  \quad ���ʂ�,���������̃��X�g(�܂��̓^�v��)�ł���\param{value}��\param{key}�����.
  �܂��͏�L�̈Ӗ��ł�``value''�̃��X�g�ł���{\tt values()}, ``key''�̃��X�g�ł���{\tt keys()}�����Ž���\param{val\_key}�Ƃ��ē��͂��邱�Ƃ��ł���. 
  �܂�,\param{key}�̓��͂ɂ͊ȒP�ȕ��@������:
  \begin{itemize}
    \item ����key��$[1,\ 2, \ldots, N]$�Ȃ�,\param{key}����͂���K�v���Ȃ�.
    \item ����key��$[0,\ 1, \ldots, N-1]$�Ȃ�,\param{key}�Ƃ���$0$�����.
    \item ����key��\param{value}�������Ƃ��Đ��񂵂����X�g�Ɠ��������,$1$�����.
    \item ����key��\param{value}���~���Ƃ��Đ��񂵂����X�g�Ɠ��������,$-1$�����.
  \end{itemize}
  \begin{at}
    \item[key]:\linkedtwo{permute}{Permute}{key}\\ \param{key}��\��.
    \item[data]:\linkedtwo{permute}{Permute}{data}\\ \negok \param{value}�̃C���f�b�N�X�t���̌`����\��.
  \end{at}
  \C
  \begin{op}
    \verb+A==B+ & A��value��B��value,������A��key��B��key�����������ǂ����Ԃ�.\\
    \verb+A*B+ & �E��Z(���Ȃ킿,�ʏ�̎ʑ��̉��Z$A \circ B$)\\
    \verb+A/B+ & ���Z(���Ȃ킿,$A \circ B^{-1}$)\\
    \verb+A**B+ & �ׂ���\\
    \verb+A.inverse()+ & �t��\\
    \verb+A[c]+ & \param{key}��\param{c}�ɑΉ�����\param{value}�̌�\\
    \verb+A(lst)+ & A��\param{lst}��u��\\
  \end{op}
\begin{ex}
>>> p1 = permute.Permute(['b','c','d','a','e'], ['a','b','c','d','e'])
>>> print p1
['a', 'b', 'c', 'd', 'e'] -> ['b', 'c', 'd', 'a', 'e']
>>> p2 = permute.Permute([2, 3, 0, 1, 4], 0)
>>> print p2
[0, 1, 2, 3, 4] -> [2, 3, 0, 1, 4]
>>> p3 = permute.Permute(['c','a','b','e','d'], 1)
>>> print p3
['a', 'b', 'c', 'd', 'e'] -> ['c', 'a', 'b', 'e', 'd']
>>> print p1 * p3
['a', 'b', 'c', 'd', 'e'] -> ['d', 'b', 'c', 'e', 'a']
>>> print p3 * p1
['a', 'b', 'c', 'd', 'e'] -> ['a', 'b', 'e', 'c', 'd']
>>> print p1 ** 4
['a', 'b', 'c', 'd', 'e'] -> ['a', 'b', 'c', 'd', 'e']
>>> p1['d']
'a'
>>> p2([0, 1, 2, 3, 4])
[2, 3, 0, 1, 4]
\end{ex}%Don't indent!
  \method
  \subsubsection{setKey -- key��ϊ�}\linkedtwo{permute}{Permute}{setKey}
   \func{setKey}{\param{self},\ \hiki{key}{list/tuple}}{\out{Permute}}\\
   \spacing
   % document of basic document
   \quad ����key��ݒ�.\\
   \spacing
   % added document
   %\spacing
   % input, output document
   \quad \param{key}��\linkingtwo{permute}{Permute}{key}�Ɠ��������̃��X�g�܂��̓^�v���łȂ���΂Ȃ�Ȃ�.\\
 \subsubsection{getValue -- ``value''�𓾂�}\linkedtwo{permute}{Permute}{getValue}
   \func{getValue}{\param{self}}{\out{list}}\\
   \spacing
   % document of basic document
   \quad \param{self}��(\param{data}�łȂ�)\param{value}��Ԃ�.\\
   \spacing
   % added document
   %\spacing
   % input, output document
 \subsubsection{getGroup -- PermGroup�𓾂�}\linkedtwo{permute}{Permute}{getGroup}
   \func{getGroup}{\param{self}}{\out{PermGroup}}\\
   \spacing
   % document of basic document
   \quad \param{self}�̏�������\linkingone{permute}{PermGroup}��Ԃ�.\\
   \spacing
   % added document
   %\spacing
   % input, output document
 \subsubsection{numbering -- �C���f�b�N�X��^����}\linkedtwo{permute}{Permute}{numbering}
   \func{numbering}{\param{self}}{\out{int}}\\
   \spacing
   % document of basic document
   \quad �u���Q��\param{self}�ɐ����߂�. (�x�����\�b�h)\\
   \spacing
   % added document
   \quad ���Ɏ����u���Q�̎����ɂ��A�[�I�Ȓ�`�ɏ]���Ē�߂���.\\
   $(n-1)$�������$[\sigma_1,\ \sigma_2,...,\sigma_{n-2},\ \sigma_{n-1}]$�̔ԍ��t����$k$�Ƃ����,
   $n$�������$[\sigma_1,\ \sigma_2,...,\sigma_{n-2},\sigma_{n-1},n]$�̔ԍ��t����$k$,�܂�
   $n$�������$[\sigma_1,\ \sigma_2,...,\sigma_{n-2},\ n,\ \sigma_{n-1}]$�̔ԍ��t����$k+(n-1)!$,�ȂǂƂȂ�.
   (\href{http://www32.ocn.ne.jp/~graph_puzzle/2no15.htm}{Room of Points And Lines, part 2, section 15, paragraph 2 (Japanese)})\\
   %\spacing
   % input, output document
 \subsubsection{order -- ���̈ʐ�}\linkedtwo{permute}{Permute}{order}
   \func{order}{\param{self}}{\out{int/long}}\\
   \spacing
   % document of basic document
   \quad �Q�̌��Ƃ��Ă̈ʐ���Ԃ�.\\
   \spacing
   % added document
   \quad ���̃��\�b�h�͈�ʂ̌Q�̂����������.
   \spacing
   % input, output document
 \subsubsection{ToTranspose -- �݊��̐ςƂ��ĕ\��}\linkedtwo{permute}{Permute}{ToTranspose}
   \func{ToTranspose}{\param{self}}{\out{ExPermute}}\\
   \spacing
   % document of basic document
   \quad \param{self}���݊��̐ςŕ\��.\\
   \spacing
   % added document
   \quad �݊�(���Ȃ킿�񎟌�����)�̐ςƂ���\linkingone{permute}{ExPermute}�̌���Ԃ�.
   ����͍ċA�v���O�����ł���,\linkingtwo{permute}{Permute}{ToCyclic}���������̎��Ԃ������邾�낤.\\
   \spacing
   % input, output document
 \subsubsection{ToCyclic -- ExPermute�̌��ɑΉ�����}\linkedtwo{permute}{Permute}{ToCyclic}
   \func{ToCyclic}{\param{self}}{\out{ExPermute}}\\
   \spacing
   % document of basic document
   \quad ����\���̐ςƂ���\param{self}��\��.\\
   \spacing
   % added document
   \quad \linkingone{permute}{ExPermute}�̌���Ԃ�.
   \negok ���̃��\�b�h��\param{self}���݂��ɑf�ȏ���u���ɕ�������.����Ă��ꂼ��̏���͉Š�.
   \spacing
   % input, output document
 \subsubsection{sgn -- �u���L��}\linkedtwo{permute}{Permute}{sgn}
   \func{sgn}{\param{self}}{\out{int}}\\
   \spacing
   % document of basic document
   \quad �u���Q�̌��̒u��������Ԃ�.\\
   \spacing
   % added document
   \quad ����\param{self}�����u��,���Ȃ킿,\param{self}�������‚̌݊��̐ςƂ��ď������Ƃ��ł���ꍇ,$1$��Ԃ�.
   �����Ȃ����,���Ȃ킿��u���̏ꍇ,$-1$��Ԃ�.
   \spacing
   % input, output document
 \subsubsection{types -- ����u���̌`��}\linkedtwo{permute}{Permute}{types}
   \func{types}{\param{self}}{\out{list}}\\
   \spacing
   % document of basic document
   \quad ���ꂼ��̏���u���̌��̒����ɂ���Ē�`���ꂽ����u���̌`����Ԃ�.\\
   \spacing
   % added document
   %\spacing
   % input, output document
 \subsubsection{ToMatrix -- �u���s��}\linkedtwo{permute}{Permute}{ToMatrix}
   \func{ToMatrix}{\param{self}}{\out{\linkingone{matrix}{Matrix}}}\\
   \spacing
   % document of basic document
   \quad �u���s���Ԃ�.\\
   \spacing
   % added document
   \quad �s�Ɨ��\param{key}�ɑΉ�����.
   ����\param{self} $G$ �� $G[a]=b$ �𖞂�����,�s���$(a,\ b)$������$1$.
   �����Ȃ���,���̌���$0$.
   %\spacing
   % input, output document
\begin{ex}
>>> p = Permute([2,3,1,5,4])
>>> p.numbering()
28
>>> p.order()
6
>>> p.ToTranspose()
[(4,5)(1,3)(1,2)](5)
>>> p.sgn()
-1
>>> p.ToCyclic()
[(1,2,3)(4,5)](5)
>>> p.types()
'(2,3)type'
>>> print p.ToMatrix()
0 1 0 0 0
0 0 1 0 0
1 0 0 0 0
0 0 0 0 1
0 0 0 1 0
\end{ex}%Don't indent!
\C

 \subsection{ExPermute -- ����\���Ƃ��Ă̒u���Q�̌�}\linkedone{permute}{ExPermute}
 \initialize
  \func{ExPermute}{\hiki{dim}{int},\ \hiki{value}{list},\ \hikiopt{key}{list}{None}}{ExPermute}\\
  \spacing
  % document of basic document
  \quad �V�����u���Q�̌����쐬.\\
  \spacing
  % added document
  \quad �C���X�^���X��``�����'' ���@�ō쐬�����.
  ���Ȃ킿,�e�^�v��������\����\���^�v���̃��X�g�ł���\param{value}�����.
  �Ⴆ��, $(\sigma_1,\ \sigma_2,\ \sigma_3,\ldots,\sigma_k)$��1��1�ʑ�, $\sigma_1 \mapsto \sigma_2,\ \sigma_2 \mapsto \sigma_3,\ldots,\sigma_k \mapsto \sigma_1$.\\
  \spacing
  % input, output document
  \quad \param{dim}�͎��R���łȂ���΂Ȃ�Ȃ�,���Ȃ킿,int,long�܂���\linkingone{rational}{Integer}�̃C���X�^���X. 
  \param{key}��\param{dim}�Ɠ��������̃��X�g�ł���ׂ��ł���.
  ����\param{value}�Ƃ��Ă�\param{key}�ɓ����Ă���^�v���̃��X�g�����.
  \param{key}��$[1,\ 2,\ldots,N]$�Ƃ����`���Ȃ�\param{key}���ȗ����邱�Ƃ��ł��邱�Ƃɒ���.
  �܂�,\param{key}��$[0,\ 1,\ldots,N-1]$�Ƃ����`���Ȃ�\param{key}�Ƃ���$0$����͂��邱�Ƃ��ł���.
  \begin{at}
    \item[dim]:\linkedtwo{permute}{ExPermute}{dim}\\ \param{dim}��\��.
    \item[key]:\linkedtwo{permute}{ExPermute}{key}\\ \param{key}��\��.
    \item[data]:\linkedtwo{permute}{ExPermute}{data}\\ \negok �C���f�b�N�X�̕t����\param{value}�̌`����\��.
  \end{at}
  \begin{op}
    \verb+A==B+ & A��value��B��value,������A��key��B��key�����������ǂ����Ԃ�.\\
    \verb+A*B+ & �E��Z(���Ȃ킿,���ʂ̎ʑ�$A \circ B$)\\
    \verb+A/B+ & ���Z(���Ȃ킿,$A \circ B^{-1}$)\\
    \verb+A**B+ & �ׂ���\\
    \verb+A.inverse()+ & �t��\\
    \verb+A[c]+ & \param{key}��\param{c}�ɑΉ�����\param{value}�̌�\\
    \verb+A(lst)+ & \param{lst}��A�ɒu������\\
    \verb+str(A)+ & �P���ȕ\�L.\linkingtwo{permute}{ExPermute}{simplify}��p����.\\
    \verb+repr(A)+ & �\�L\\
  \end{op}
\begin{ex}
>>> p1 = permute.ExPermute(5, [('a', 'b')], ['a','b','c','d','e'])
>>> print p1
[('a', 'b')] <['a', 'b', 'c', 'd', 'e']>
>>> p2 = permute.ExPermute(5, [(0, 2), (3, 4, 1)], 0)
>>> print p2
[(0, 2), (1, 3, 4)] <[0, 1, 2, 3, 4]>
>>> p3 = permute.ExPermute(5,[('b','c')],['a','b','c','d','e'])
>>> print p1 * p3
[('a', 'b'), ('b', 'c')] <['a', 'b', 'c', 'd', 'e']>
>>> print p3 * p1
[('b', 'c'), ('a', 'b')] <['a', 'b', 'c', 'd', 'e']>
>>> p1['c']
'c'
>>> p2([0, 1, 2, 3, 4])
[2, 4, 0, 1, 3]
\end{ex}%Don't indent!
  \method
  \subsubsection{setKey -- key��ϊ�}\linkedtwo{permute}{ExPermute}{setKey}
   \func{setKey}{\param{self},\ \hiki{key}{list}}{\out{ExPermute}}\\
   \spacing
   % document of basic document
   \quad ����key��ݒ�.\\
   \spacing
   % added document
   %\spacing
   % input, output document
   \quad \param{key}��\linkingtwo{permute}{ExPermute}{dim}�Ɠ��������̃��X�g�łȂ���΂Ȃ�Ȃ�.\\
 \subsubsection{getValue -- ``value''�𓾂�}\linkedtwo{permute}{ExPermute}{getValue}
   \func{getValue}{\param{self}}{\out{list}}\\
   \spacing
   % document of basic document
   \quad \param{self}��(\param{data}�łȂ�)\param{value}��Ԃ�.\\
   \spacing
   % added document
   %\spacing
   % input, output document
 \subsubsection{getGroup -- PermGroup�𓾂�}\linkedtwo{permute}{ExPermute}{getGroup}
   \func{getGroup}{\param{self}}{\out{PermGroup}}\\
   \spacing
   % document of basic document
   \quad \param{self}����������\linkingone{permute}{PermGroup}��Ԃ�.\\
   \spacing
   % added document
   %\spacing
   % input, output document
 \subsubsection{order -- ���̈ʐ�}\linkedtwo{permute}{ExPermute}{order}
   \func{order}{\param{self}}{\out{int/long}}\\
   \spacing
   % document of basic document
   \quad �Q�̌��Ƃ��Ă̈ʐ���Ԃ�.\\
   \spacing
   % added document
   \quad ���̃��\�b�h�͈�ʂ̌Q�̂����������.\\
   \spacing
   % input, output document
 \subsubsection{ToNormal -- ���ʂ̕\��}\linkedtwo{permute}{ExPermute}{ToNormal}
   \func{ToNormal}{\param{self}}{\out{Permute}}\\
   \spacing
   % document of basic document
   \quad \param{self}��\linkingone{permute}{Permute}�̃C���X�^���X�Ƃ��ĕ\��.\\
   \spacing
   % added document
   %\spacing
   % input, output document
 \subsubsection{simplify -- �P���Ȓl���g�p}\linkedtwo{permute}{ExPermute}{simplify}
   \func{simplify}{\param{self}}{\out{ExPermute}}\\
   \spacing
   % document of basic document
   \quad ���P���ȏ���\����Ԃ�.\\
   \spacing
   % added document
   \quad \negok ���̃��\�b�h��\linkingtwo{permute}{ExPermute}{ToNormal}��\linkingtwo{permute}{Permute}{ToCyclic}���g�p.
   \spacing
   % input, output document
 \subsubsection{sgn -- �u������}\linkedtwo{permute}{ExPermute}{sgn}
   \func{sgn}{\param{self}}{\out{int}}\\
   \spacing
   % document of basic document
   \quad �u���Q�̌��̒u��������Ԃ�.\\
   \spacing
   % added document
   \quad ����\param{self}�����u���Ȃ�,���Ȃ킿,\param{self}�������‚̌݊��̐ςƂ��ď������Ƃ��ł���ꍇ,$1$��Ԃ�.
   �����Ȃ���,���Ȃ킿��u���Ȃ�,$-1$��Ԃ�.\\
   \spacing
   % input, output document
\begin{ex}
>>> p = permute.ExPermute(5, [(1, 2, 3), (4, 5)])
>>> p.order()
6
>>> print p.ToNormal()
[1, 2, 3, 4, 5] -> [2, 3, 1, 5, 4]
>>> p * p
[(1, 2, 3), (4, 5), (1, 2, 3), (4, 5)] <[1, 2, 3, 4, 5]>
>>> (p * p).simplify()
[(1, 3, 2)] <[1, 2, 3, 4, 5]>
\end{ex}%Don't indent!
\C

 \subsection{PermGroup -- �u���Q}\linkedone{permute}{PermGroup}
 \initialize
  \func{PermGroup}{\hiki{key}{int/long}}{PermGroup}\\
  \func{PermGroup}{\hiki{key}{list/tuple}}{PermGroup}\\
  \spacing
  % document of basic document
  \quad �V�����u���Q���쐬.\\
  \spacing
  % added document
  % \spacing
  % input, output document
  \quad ���ʂ�,\param{key}�Ƃ��ă��X�g�����.
  �������鐮��$N$����͂�����,\param{key}��$[1,\ 2,\ldots,N]$�Ƃ��Đݒ肳���. 
  \begin{at}
    \item[key]:\linkedtwo{permute}{PermGroup}{key}\\ \param{key}��\��.
  \end{at}
  \begin{op}
    \verb+A==B+ & A��value��B��value,������A��key��B��key�����������ǂ����Ԃ�.\\
    \verb+card(A)+ & \linkingtwo{permute}{PermGroup}{grouporder}�Ɠ���\\ 
    \verb+str(A)+ & �P���ȕ\�L\\
    \verb+repr(A)+ & �\�L\\
  \end{op}
\begin{ex}
>>> p1 = permute.PermGroup(['a','b','c','d','e'])
>>> print p1
['a','b','c','d','e']
>>> card(p1)
120L
\end{ex}%Don't indent!
  \method
  \subsubsection{createElement -- �V�[�h���猳���쐬}\linkedtwo{permute}{PermGroup}{createElement}
   \func{createElement}{\param{self},\ \hiki{seed}{list/tuple/dict}}{\out{Permute}}\\
   \func{createElement}{\param{self},\ \hiki{seed}{list}}{\out{ExPermute}}\\
   \spacing
   % document of basic document
   \quad \param{self}�̐V���������쐬.\\
   \spacing
   % added document
   %\spacing
   % input, output document
   \quad \param{seed}��\linkingone{permute}{Permute}�܂���\linkingone{permute}{ExPermute}��``value''�̌`���łȂ���΂Ȃ�Ȃ�
 \subsubsection{identity -- �P�ʌ�}\linkedtwo{permute}{PermGroup}{identity}
   \func{identity}{\param{self}}{\out{Permute}}\\
   \spacing
   % document of basic document
   \quad ���ʂ̕\����\param{self}�̒P�ʌ���Ԃ�.\\
   \spacing
   % added document
   \quad ����\���̏ꍇ,\linkingtwo{permute}{PermGroup}{identity\_c}���g�p.
   \spacing
   % input, output document
 \subsubsection{identity\_c -- ����\���̒P�ʌ�}\linkedtwo{permute}{PermGroup}{identity\_c}
   \func{identity\_c}{\param{self}}{\out{ExPermute}}\\
   \spacing
   % document of basic document
   \quad ����\���Ƃ��Ēu���Q�̒P�ʌ���Ԃ�.\\
   \spacing
   % added document
   \quad ���ʂ̕\���̏ꍇ,\linkingtwo{permute}{PermGroup}{identity}���g�p.
   \spacing
   % input, output document
 \subsubsection{grouporder -- �Q�̈ʐ�}\linkedtwo{permute}{PermGroup}{grouporder}
   \func{grouporder}{\param{self}}{\out{int/long}}\\
   \spacing
   % document of basic document
   \quad �Q�Ƃ��Ă�\param{self}�̈ʐ����v�Z.\\
   \spacing
   % added document
   %\spacing
   % input, output document
 \subsubsection{randElement -- ����ׂɌ���I��}\linkedtwo{permute}{PermGroup}{randElement}
   \func{randElement}{\param{self}}{\out{Permute}}\\
   \spacing
   % document of basic document
   \quad ���ʂ̕\���Ƃ��Ė���ׂɐV����\param{self}�̌����쐬.\\
   \spacing
   % added document
   %\spacing
   % input, output document
\begin{ex}
>>> p = permute.PermGroup(5)
>>> print p.createElement([3, 4, 5, 1, 2])
[1, 2, 3, 4, 5] -> [3, 4, 5, 1, 2]
>>> print p.createElement([(1, 2), (3, 4)])
[(1, 2), (3, 4)] <[1, 2, 3, 4, 5]>
>>> print p.identity()
[1, 2, 3, 4, 5] -> [1, 2, 3, 4, 5]
>>> print p.identity_c()
[] <[1, 2, 3, 4, 5]>
>>> p.grouporder()
120L
>>> print p.randElement()
[1, 2, 3, 4, 5] -> [3, 4, 5, 2, 1]
\end{ex}%Don't indent!
\C
%---------- end document ---------- %

\bibliographystyle{jplain}%use jbibtex
\bibliography{nzmath_references}

\end{document}


%%%%%%%%%%%%%%%%%%%%%%%%%%%%%%%%%%%%%%%%%%%%%%%%%%%%%%%%%%%%%%
%
% macros for nzmath manual
%
%%%%%%%%%%%%%%%%%%%%%%%%%%%%%%%%%%%%%%%%%%%%%%%%%%%%%%%%%%%%%
\usepackage{amssymb,amsmath}
\usepackage{color}
\usepackage[dvipdfm,bookmarks=true,bookmarksnumbered=true,%
 pdftitle={NZMATH Users Manual},%
 pdfsubject={Manual for NZMATH Users},%
 pdfauthor={NZMATH Development Group},%
 pdfkeywords={TeX; dvipdfmx; hyperref; color;},%
 colorlinks=true]{hyperref}
\usepackage{fancybox}
\usepackage[T1]{fontenc}
%
\newcommand{\DS}{\displaystyle}
\newcommand{\C}{\clearpage}
\newcommand{\NO}{\noindent}
\newcommand{\negok}{$\dagger$}
\newcommand{\spacing}{\vspace{1pt}\\ }
% software macros
\newcommand{\nzmathzero}{{\footnotesize $\mathbb{N}\mathbb{Z}$}\texttt{MATH}}
\newcommand{\nzmath}{{\nzmathzero}\ }
\newcommand{\pythonzero}{$\mbox{\texttt{Python}}$}
\newcommand{\python}{{\pythonzero}\ }
% link macros
\newcommand{\linkingzero}[1]{{\bf \hyperlink{#1}{#1}}}%module
\newcommand{\linkingone}[2]{{\bf \hyperlink{#1.#2}{#2}}}%module,class/function etc.
\newcommand{\linkingtwo}[3]{{\bf \hyperlink{#1.#2.#3}{#3}}}%module,class,method
\newcommand{\linkedzero}[1]{\hypertarget{#1}{}}
\newcommand{\linkedone}[2]{\hypertarget{#1.#2}{}}
\newcommand{\linkedtwo}[3]{\hypertarget{#1.#2.#3}{}}
\newcommand{\linktutorial}[1]{\href{http://docs.python.org/tutorial/#1}{#1}}
\newcommand{\linktutorialone}[2]{\href{http://docs.python.org/tutorial/#1}{#2}}
\newcommand{\linklibrary}[1]{\href{http://docs.python.org/library/#1}{#1}}
\newcommand{\linklibraryone}[2]{\href{http://docs.python.org/library/#1}{#2}}
\newcommand{\pythonhp}{\href{http://www.python.org/}{\python website}}
\newcommand{\nzmathwiki}{\href{http://nzmath.sourceforge.net/wiki/}{{\nzmathzero}Wiki}}
\newcommand{\nzmathsf}{\href{http://sourceforge.net/projects/nzmath/}{\nzmath Project Page}}
\newcommand{\nzmathtnt}{\href{http://tnt.math.se.tmu.ac.jp/nzmath/}{\nzmath Project Official Page}}
% parameter name
\newcommand{\param}[1]{{\tt #1}}
% function macros
\newcommand{\hiki}[2]{{\tt #1}:\ {\em #2}}
\newcommand{\hikiopt}[3]{{\tt #1}:\ {\em #2}=#3}

\newdimen\hoge
\newdimen\truetextwidth
\newcommand{\func}[3]{%
\setbox0\hbox{#1(#2)}
\hoge=\wd0
\truetextwidth=\textwidth
\advance \truetextwidth by -2\oddsidemargin
\ifdim\hoge<\truetextwidth % short form
{\bf \colorbox{skyyellow}{#1(#2)\ $\to$ #3}}
%
\else % long form
\fcolorbox{skyyellow}{skyyellow}{%
   \begin{minipage}{\textwidth}%
   {\bf #1(#2)\\ %
    \qquad\quad   $\to$\ #3}%
   \end{minipage}%
   }%
\fi%
}

\newcommand{\out}[1]{{\em #1}}
\newcommand{\initialize}{%
  \paragraph{\large \colorbox{skyblue}{Initialize (Constructor)}}%
    \quad\\ %
    \vspace{3pt}\\
}
\newcommand{\method}{\C \paragraph{\large \colorbox{skyblue}{Methods}}}
% Attribute environment
\newenvironment{at}
{%begin
\paragraph{\large \colorbox{skyblue}{Attribute}}
\quad\\
\begin{description}
}%
{%end
\end{description}
}
% Operation environment
\newenvironment{op}
{%begin
\paragraph{\large \colorbox{skyblue}{Operations}}
\quad\\
\begin{table}[h]
\begin{center}
\begin{tabular}{|l|l|}
\hline
operator & explanation\\
\hline
}%
{%end
\hline
\end{tabular}
\end{center}
\end{table}
}
% Examples environment
\newenvironment{ex}%
{%begin
\paragraph{\large \colorbox{skyblue}{Examples}}
\VerbatimEnvironment
\renewcommand{\EveryVerbatim}{\fontencoding{OT1}\selectfont}
\begin{quote}
\begin{Verbatim}
}%
{%end
\end{Verbatim}
\end{quote}
}
%
\definecolor{skyblue}{cmyk}{0.2, 0, 0.1, 0}
\definecolor{skyyellow}{cmyk}{0.1, 0.1, 0.5, 0}
%
%\title{NZMATH User Manual\\ {\large{(for version 1.0)}}}
%\date{}
%\author{}
\begin{document}
%\maketitle
%
\setcounter{tocdepth}{3}
\setcounter{secnumdepth}{3}


\tableofcontents
\C

\chapter{Classes}


%---------- start document ---------- %
 \section{rational -- �����ƗL����}\linkedzero{rational}
rational���W���[���̓N���XRational, �N���XInteger, �N���XRationalField, ������ �N���XIntegerRing�Ƃ��Đ����ƗL�������.

 \begin{itemize}
   \item {\bf Classes}
   \begin{itemize}
     \item \linkingone{rational}{Integer}
     \item \linkingone{rational}{IntegerRing}
     \item \linkingone{rational}{Rational}
     \item \linkingone{rational}{RationalField}
   \end{itemize}
 \end{itemize}

���̃��W���[���͂܂��ȉ��̃R���e���c��񋟂���:
\begin{description}
   \item[theIntegerRing]\linkedone{rational}{theIntegerRing}:\\
     \param{theIntegerRing}�͗L�������‚�\��.
     \linkingone{rational}{IntegerRing}�̃C���X�^���X.
   \item[theRationalField]\linkedone{rational}{theRationalField}:\\
     \param{theRationalField}�͗L�����̂�\��.
     \linkingone{rational}{RationalField}�̃C���X�^���X.
 \end{description}

\C

 \subsection{Integer -- ����}\linkedone{rational}{Integer}
 Integer�͐����̃N���X. 'int' �� 'long' �͏��Z�ɂ����ėL������Ԃ��Ȃ��̂�, �V�����N���X���쐬����K�v��������.

 ���̃N���X��\linkingone{ring}{CommutativeRingElement} �� long�̃T�u�N���X.

  \initialize
  \func{Integer}{\hiki{integer}{integer}}{\out{Integer}}\\
  \spacing
  % document of basic document
  \quad Integer�I�u�W�F�N�g���\��.
  % added document
  �����������ȗ����ꂽ��, �l�� 0 �ƂȂ�.
  \method
  \subsubsection{getRing -- ring�I�u�W�F�N�g�𓾂�}\linkedtwo{rational}{Integer}{getRing}
   \func{getRing}{\param{self}}{\out{IntegerRing}}\\
   \spacing
   % document of basic document
   \quad IntegerRing�I�u�W�F�N�g��Ԃ�.
%
  \subsubsection{actAdditive -- 2�i�̉��@���̉��@}\linkedtwo{rational}{Integer}{actAdditive}
   \func{actAdditive}{\param{self},\ \hiki{other}{integer}}{\out{Integer}}\\
   \spacing
   % document of basic document
   \quad other�ɉ��@�I�ɍ�p,���Ȃ킿, n ��\param{other}��n��̉��Z�Ɋg�傳���. ���ʂƂ��Ă͈ȉ��Ɠ���:

   \verb|return sum([+other for _ in range(self)])|

   ������, �����ł�2�i�̉��@�����g��.\\
   \spacing
%
  \subsubsection{actMultiplicative -- 2�i�̉��@���̏�@}\linkedtwo{rational}{Integer}{actMultiplicative}
   \func{actMultiplicative}{\param{self},\ \hiki{other}{integer}}{\out{Integer}}\\
   \spacing
   % document of basic document
   \quad other�ɏ�@�I�ɍ�p����,���Ȃ킿, n ��\param{other}��n��̏�Z�Ɋg�傳���. ���ʂƂ��Ă͈ȉ��Ɠ���:

\verb|return reduce(lambda x,y: x*y, [+other for _ in range(self)])|

   ������, �����ł�2�i�̉��@�����g��.
   \spacing
\C
 \subsection{IntegerRing -- ������}\linkedone{rational}{IntegerRing}
 �L�������‚ɑ΂���N���X.

 ���̃N���X��\linkingone{ring}{CommutativeRing}�̃T�u�N���X.


  \initialize
  \func{IntegerRing}{}{\out{IntegerRing}}\\
  \spacing
  % document of basic document
  \quad IntegerRing�̃C���X�^���X���쐬. 
  % added document
  ���ł�theIntegerRing������̂�,�C���X�^���X���쐬����K�v���Ȃ���������Ȃ�.\\
  \begin{at}
    \item[zero]\linkedtwo{integer}{IntegerRing}{zero}:\\ ���@�̒P�ʌ� 0 ��\��. (�ǂݍ��ݐ�p)
    \item[one]\linkedtwo{integer}{IntegerRing}{one}:\\ ��@�̒P�ʌ� 1 ��\��. (�ǂ݂��ݐ�p)
  \end{at}
  \begin{op}
    \verb|x in Z| & �����܂܂�Ă���ǂ����Ԃ�.\\
    \verb|repr(Z)| & repr�������Ԃ�.\\
    \verb|str(Z)| & str�������Ԃ�.\\
  \end{op}
  \method
  \subsubsection{createElement -- Integer�I�u�W�F�N�g���쐬}\linkedtwo{rational}{IntegerRing}{createElement}
   \func{createElement}{\param{self},\ \hiki{seed}{integer}}{\out{Integer}}\\
   \spacing
   % document of basic document
   \quad \param{seed}�ɑ΂���Integer�I�u�W�F�N�g���쐬. 
   \spacing
   % input, output document
   \quad \param{seed}��int�^, long�^ �܂��� rational.Integer�łȂ���΂Ȃ�Ȃ�.\\
%
  \subsubsection{gcd -- �ő����}\linkedtwo{rational}{IntegerRing}{gcd}
   \func{gcd}{\param{self},\ \hiki{n}{integer},\ \hiki{m}{integer}}{\out{Integer}}\\
   \spacing
   % document of basic document
   \quad �^����ꂽ��‚̐����̍ő���񐔂�Ԃ�.\\
%
  \subsubsection{extgcd -- �g��GCD}\linkedtwo{rational}{IntegerRing}{extgcd}
   \func{extgcd}{\param{self},\ \hiki{n}{integer},\ \hiki{m}{integer}}{\out{Integer}}\\
   \spacing
   % document of basic document
   \quad �^�v��($u$, $v$, $d$)��Ԃ�; �����͗^����ꂽ��‚̐���\param{n} �� \param{m}�̍ő���� $d$ ��, $d = \mathtt{n}u + \mathtt{m}v$�ƂȂ�$u$, $v$.\\
   \spacing
%
  \subsubsection{lcm -- �ŏ����{��}\linkedtwo{rational}{IntegerRing}{lcm}
   \func{lcm}{\param{self},\ \hiki{n}{integer},\ \hiki{m}{integer}}{\out{Integer}}\\
   \spacing
   % document of basic document
   \quad �^����ꂽ��‚̐����̍ŏ����{����Ԃ�. 
   % added document
   \quad ���������Ƃ� 0 �Ȃ�, �G���[���N����.\\
%
  \subsubsection{getQuotientField -- �L�����̃I�u�W�F�N�g�𓾂�}\linkedtwo{rational}{IntegerRing}{getQuotientField}
   \func{getQuotientField}{\param{self}}{\out{RationalField}}\\
   \spacing
   % document of basic document
   \quad �L������(\linkingone{rational}{RationalField})��Ԃ�.\\
%
  \subsubsection{issubring -- �����‚��ǂ�������}\linkedtwo{rational}{IntegerRing}{issubring}
   \func{issubring}{\param{self},\ \hiki{other}{\linkingone{ring}{Ring}}}{\out{bool}}\\
   \spacing
   % document of basic document
   \quad ��������̊‚������‚Ƃ��Đ����‚��܂�ł��邩��.

   ����other�������‚Ȃ�,�o�͂�True. ���̑��̏ꍇ��������̐����‚�issuperring���\�b�h�̂���������Ɉˑ�.\\
   \spacing
%
  \subsubsection{issuperring -- �܂�ł��邩�ǂ�������}\linkedtwo{rational}{IntegerRing}{issuperring}
   \func{issuperring}{\param{self},\ \hiki{other}{\linkingone{ring}{Ring}}}{\out{bool}}\\
   \spacing
   % document of basic document
   \quad �����‚���������̊‚𕔕��ԂƂ��Ċ܂�ł��邩��.

����other�������‚Ȃ�,�o�͂�True. ���̑��̏ꍇ��������̐����‚�issubring���\�b�h�̂���������Ɉˑ�.\\
   \spacing

\C
 \subsection{Rational -- �L����}\linkedone{rational}{Rational}
 �L�����̃N���X.

  \initialize
  \func{Rational}
       {\hiki{numerator}{numbers},\ 
         \hikiopt{denominator}{numbers}{1}}
       {\out{Integer}}\\
  \spacing
  % document of basic document
  \quad �L�����͈ȉ�����\��:
  \begin{itemize}
  \item ����,
  \item float
  \item Rational.
  \end{itemize}
  % added document
  ����toRational���\�b�h�������,���̃I�u�W�F�N�g��ϊ����邱�Ƃ��ł���. �����Ȃ��� TypeError���N����.

  \method
  \subsubsection{getRing -- ring�I�u�W�F�N�g�𓾂�}\linkedtwo{rational}{Rational}{getRing}
   \func{getRing}{\param{self}}{\out{RationalField}}\\
   \spacing
   % document of basic document
   \quad RationalField�I�u�W�F�N�g��Ԃ�.

  \subsubsection{decimalString -- ������\��}\linkedtwo{rational}{Rational}{decimalString}
   \func{decimalString}{\param{self},\ \hiki{N}{integer}}{\out{string}}\\
   \spacing
   % document of basic document
   \quad ������\param{N}���Ƃ����������Ԃ�.
   \spacing

  \subsubsection{expand -- �A�����ɂ��\��}\linkedtwo{rational}{Rational}{expand}
   \func{expand}
        {\param{self},\ 
          \hiki{base}{integer},\ 
          \hiki{limit}{integer}}
        {\out{string}}\\
   \spacing
   % document of basic document
   \quad ����\param{base}�����R���Ȃ�,���ꂪ\param{base}�̍��X\param{limit}��ł���ł��߂��L������Ԃ�.

   �����Ȃ���(���Ȃ킿, \param{base}=0),���ꂪ���X\param{limit}�ł���ł��߂��L������Ԃ�.
   \spacing
   % input, output document
   \quad \param{base}�͕��̐����ł����Ă͂Ȃ�Ȃ�.\\

\C
 \subsection{RationalField -- �L������}\linkedone{rational}{RationalField}
RationalField�͗L�����̂̃N���X. ���̃N���X��\linkingone{rational}{theRationalField}�Ƃ����B��̃C���X�^���X������.

 ���̃N���X��\linkingone{ring}{QuotientField}�̃T�u�N���X.
%  
  \initialize
  \func{RationalField}{}{\out{RationalField}}\\
  \spacing
  % document of basic document
  \quad RationalField�̃C���X�^���X���쐬. 
  % added document
  ���ł�theRationalField������̂�,�C���X�^���X���쐬����K�v�͂Ȃ���������Ȃ�.
  \begin{at}
    \item[zero]\linkedtwo{integer}{RationalField}{zero}:\\ ���@�̒P�ʌ� 0 ��\��, ���Ȃ킿 Rational(0, 1). (�ǂݍ��ݐ�p)
    \item[one]\linkedtwo{integer}{RationalField}{one}:\\ ��@�̒P�ʌ� 1 ��\��, ���Ȃ킿 Rational(1, 1). (�ǂݍ��ݐ�p)
  \end{at}
  \begin{op}
    \verb|x in Q| & �����܂܂�Ă��邩�ǂ����Ԃ�.\\
    \verb|str(Q)| & str�������Ԃ�.\\
  \end{op}
  \method
  \subsubsection{createElement -- Rational�I�u�W�F�N�g��Ԃ�}\linkedtwo{rational}{RationalField}{createElement}
   \func{createElement}
        {\param{self},\ 
          \hiki{numerator}{integer or \linkingone{rational}{Rational}},\ 
          \hikiopt{denominator}{integer}{1} 
        }{\out{Rational}}\\
   \spacing
   % document of basic document
   \quad Rational�I�u�W�F�N�g���쐬.

  \subsubsection{classNumber -- �ސ��𓾂�}\linkedtwo{rational}{RationalField}{classNumber}
   \func{classNumber}{\param{self}}{\out{integer}}\\
   \spacing
   % document of basic document
   \quad �L�����̗̂ސ��� 1 �Ȃ̂�, 1 ��Ԃ�.

  \subsubsection{getQuotientField -- �L�����̃I�u�W�F�N�g��Ԃ�}\linkedtwo{rational}{RationalField}{getQuotientField}
   \func{getQuotientField}{\param{self}}{\out{RationalField}}\\
   \spacing
   % document of basic document
   \quad �L�����̃C���X�^���X��Ԃ�.

  \subsubsection{issubring -- �����‚��ǂ�������}\linkedtwo{rational}{RationalField}{issubring}
   \func{issubring}{\param{self},\ \hiki{other}{\linkingone{ring}{Ring}}}{\out{bool}}\\
   \spacing
   % document of basic document
   \quad ��������̊‚������‚Ƃ��ėL�����̂��܂�ł��邩��.

   ����other���܂��L�����̂Ȃ�, �o�͂�True. ���̏ꍇ���������issuperring���\�b�h�ɂ���������Ɉˑ�.
   \spacing

  \subsubsection{issuperring -- �܂�ł��邩�ǂ�������}\linkedtwo{rational}{RationalField}{issuperring}
   \func{issuperring}{\param{self},\ \hiki{other}{\linkingone{ring}{Ring}}}{\out{bool}}\\
   \spacing
   % document of basic document
   \quad �L�����̂���������̊‚𕔕��‚Ƃ��Ă��܂�ł��邩��.

����other���܂��L�����̂Ȃ�, �o�͂�True. ���̏ꍇ���������issubring���\�b�h�ɂ���������Ɉˑ�.
   \spacing
\C

% ---------- end document ---------- %

\bibliographystyle{jplain}%use jbibtex
\bibliography{nzmath_references}

\end{document}


%\documentclass{report}

%%%%%%%%%%%%%%%%%%%%%%%%%%%%%%%%%%%%%%%%%%%%%%%%%%%%%%%%%%%%%
%
% macros for nzmath manual
%
%%%%%%%%%%%%%%%%%%%%%%%%%%%%%%%%%%%%%%%%%%%%%%%%%%%%%%%%%%%%%
\usepackage{amssymb,amsmath}
\usepackage{color}
\usepackage[dvipdfm,bookmarks=true,bookmarksnumbered=true,%
 pdftitle={NZMATH Users Manual},%
 pdfsubject={Manual for NZMATH Users},%
 pdfauthor={NZMATH Development Group},%
 pdfkeywords={TeX; dvipdfmx; hyperref; color;},%
 colorlinks=true]{hyperref}
\usepackage{fancybox}
\usepackage[T1]{fontenc}
%
\newcommand{\DS}{\displaystyle}
\newcommand{\C}{\clearpage}
\newcommand{\NO}{\noindent}
\newcommand{\negok}{$\dagger$}
\newcommand{\spacing}{\vspace{1pt}\\ }
% software macros
\newcommand{\nzmathzero}{{\footnotesize $\mathbb{N}\mathbb{Z}$}\texttt{MATH}}
\newcommand{\nzmath}{{\nzmathzero}\ }
\newcommand{\pythonzero}{$\mbox{\texttt{Python}}$}
\newcommand{\python}{{\pythonzero}\ }
% link macros
\newcommand{\linkingzero}[1]{{\bf \hyperlink{#1}{#1}}}%module
\newcommand{\linkingone}[2]{{\bf \hyperlink{#1.#2}{#2}}}%module,class/function etc.
\newcommand{\linkingtwo}[3]{{\bf \hyperlink{#1.#2.#3}{#3}}}%module,class,method
\newcommand{\linkedzero}[1]{\hypertarget{#1}{}}
\newcommand{\linkedone}[2]{\hypertarget{#1.#2}{}}
\newcommand{\linkedtwo}[3]{\hypertarget{#1.#2.#3}{}}
\newcommand{\linktutorial}[1]{\href{http://docs.python.org/tutorial/#1}{#1}}
\newcommand{\linktutorialone}[2]{\href{http://docs.python.org/tutorial/#1}{#2}}
\newcommand{\linklibrary}[1]{\href{http://docs.python.org/library/#1}{#1}}
\newcommand{\linklibraryone}[2]{\href{http://docs.python.org/library/#1}{#2}}
\newcommand{\pythonhp}{\href{http://www.python.org/}{\python website}}
\newcommand{\nzmathwiki}{\href{http://nzmath.sourceforge.net/wiki/}{{\nzmathzero}Wiki}}
\newcommand{\nzmathsf}{\href{http://sourceforge.net/projects/nzmath/}{\nzmath Project Page}}
\newcommand{\nzmathtnt}{\href{http://tnt.math.se.tmu.ac.jp/nzmath/}{\nzmath Project Official Page}}
% parameter name
\newcommand{\param}[1]{{\tt #1}}
% function macros
\newcommand{\hiki}[2]{{\tt #1}:\ {\em #2}}
\newcommand{\hikiopt}[3]{{\tt #1}:\ {\em #2}=#3}

\newdimen\hoge
\newdimen\truetextwidth
\newcommand{\func}[3]{%
\setbox0\hbox{#1(#2)}
\hoge=\wd0
\truetextwidth=\textwidth
\advance \truetextwidth by -2\oddsidemargin
\ifdim\hoge<\truetextwidth % short form
{\bf \colorbox{skyyellow}{#1(#2)\ $\to$ #3}}
%
\else % long form
\fcolorbox{skyyellow}{skyyellow}{%
   \begin{minipage}{\textwidth}%
   {\bf #1(#2)\\ %
    \qquad\quad   $\to$\ #3}%
   \end{minipage}%
   }%
\fi%
}

\newcommand{\out}[1]{{\em #1}}
\newcommand{\initialize}{%
  \paragraph{\large \colorbox{skyblue}{Initialize (Constructor)}}%
    \quad\\ %
    \vspace{3pt}\\
}
\newcommand{\method}{\C \paragraph{\large \colorbox{skyblue}{Methods}}}
% Attribute environment
\newenvironment{at}
{%begin
\paragraph{\large \colorbox{skyblue}{Attribute}}
\quad\\
\begin{description}
}%
{%end
\end{description}
}
% Operation environment
\newenvironment{op}
{%begin
\paragraph{\large \colorbox{skyblue}{Operations}}
\quad\\
\begin{table}[h]
\begin{center}
\begin{tabular}{|l|l|}
\hline
operator & explanation\\
\hline
}%
{%end
\hline
\end{tabular}
\end{center}
\end{table}
}
% Examples environment
\newenvironment{ex}%
{%begin
\paragraph{\large \colorbox{skyblue}{Examples}}
\VerbatimEnvironment
\renewcommand{\EveryVerbatim}{\fontencoding{OT1}\selectfont}
\begin{quote}
\begin{Verbatim}
}%
{%end
\end{Verbatim}
\end{quote}
}
%
\definecolor{skyblue}{cmyk}{0.2, 0, 0.1, 0}
\definecolor{skyyellow}{cmyk}{0.1, 0.1, 0.5, 0}
%
%\title{NZMATH User Manual\\ {\large{(for version 1.0)}}}
%\date{}
%\author{}
\begin{document}
%\maketitle
%
\setcounter{tocdepth}{3}
\setcounter{secnumdepth}{3}


\tableofcontents
\C

\chapter{Classes}


%---------- start document ---------- %
 \section{real -- real numbers and its functions}\linkedzero{real}
The module {\tt real} provides arbitrary precision real numbers and
their utilities. The functions provided are corresponding to the
\linklibrary{math} standard module.



 \begin{itemize}
   \item {\bf Classes}
   \begin{itemize}
     \item \linkingone{real}{RealField}
     \item \linkingone{real}{Real}
     \item \negok \linkingone{real}{Constant}
     \item \negok \linkingone{real}{ExponentialPowerSeries}
     \item \negok \linkingone{real}{AbsoluteError}
     \item \negok \linkingone{real}{RelativeError}
   \end{itemize}
   \item {\bf Functions}
     \begin{itemize}
       \item \linkingone{real}{exp}
       \item \linkingone{real}{sqrt}
       \item \linkingone{real}{log}
       \item \linkingone{real}{log1piter}
       \item \linkingone{real}{piGaussLegendre}
       \item \linkingone{real}{eContinuedFraction}
       \item \linkingone{real}{floor}
       \item \linkingone{real}{ceil}
       \item \linkingone{real}{tranc}
       \item \linkingone{real}{sin}
       \item \linkingone{real}{cos}
       \item \linkingone{real}{tan}
       \item \linkingone{real}{sinh}
       \item \linkingone{real}{cosh}
       \item \linkingone{real}{tanh}
       \item \linkingone{real}{asin}
       \item \linkingone{real}{acos}
       \item \linkingone{real}{atan}
       \item \linkingone{real}{atan2}
       \item \linkingone{real}{hypot}
       \item \linkingone{real}{pow}
       \item \linkingone{real}{degrees}
       \item \linkingone{real}{radians}
       \item \linkingone{real}{fabs}
       \item \linkingone{real}{fmod}
       \item \linkingone{real}{frexp}
       \item \linkingone{real}{ldexp}
       \item \linkingone{real}{EulerTransform}

     \end{itemize}
 \end{itemize}

This module also provides following constants:
\begin{description}
   \item[e]\linkedone{real}{e}:\\
     This constant is obsolete (Ver 1.1.0).
   \item[pi]\linkedone{real}{pi}:\\
     This constant is obsolete (Ver 1.1.0).
   \item[Log2]\linkedone{real}{Log2}:\\
     This constant is obsolete (Ver 1.1.0).
   \item[theRealField]\linkedone{real}{theRealField}:\\
     \param{theRealField} is the instance of \linkingone{real}{RealField}.
 \end{description}

\C
 \subsection{RealField -- field of real numbers}\linkedone{real}{RealField}
 The class is for the field of real numbers. The class has the single instance \linkingone{real}{theRealField}.

 This class is a subclass of \linkingone{ring}{Field}.


  \initialize
  \func{RealField}{}{\out{RealField}}\\
  \spacing
  % document of basic document
  \quad Create an instance of RealField. 
  % added document
  You may not want to create an instance, since there is already \linkingone{real}{theRealField}.
  % \spacing
  % input, output document
  %See \linkingone{module}{point} for \param{point}.
  \begin{at}
    \item[zero]\linkedtwo{real}{RealField}{zero}:\\ It expresses the additive unit 0. (read only)
    \item[one]\linkedtwo{real}{RealField}{one}:\\ It expresses the multiplicative unit 1. (read only)
  \end{at}
  \begin{op}
    \verb|x in R| & membership test; return whether an element is in or not.\\
    \verb|repr(R)| & return representation string.\\
    \verb|str(R)| & return string.\\
  \end{op} 
  \method
%
  \subsubsection{getCharacteristic -- get characteristic}\linkedtwo{real}{RealField}{getCharacteristic}
   \func{getCharacteristic}{\param{self}}{\out{integer}}\\
   \spacing
   % document of basic document
   \quad Return the characteristic, zero.
%
  \subsubsection{issubring -- subring test}\linkedtwo{real}{RealField}{issubring}
   \func{issubring}{\param{self},\ \hiki{aRing}{\linkingone{ring}{Ring}}}{\out{bool}}\\
   \spacing
   % document of basic document
   \quad Report whether another ring contains the real field as subring.
   \spacing
%
  \subsubsection{issuperring -- superring test}\linkedtwo{real}{RealField}{issuperring}
   \func{issuperring}{\param{self},\ \hiki{aRing}{\linkingone{ring}{Ring}}}{\out{bool}}\\
   \spacing
   % document of basic document
   \quad Report whether the real field contains another ring as subring.
   \spacing

\C
 \subsection{Real -- a Real number}\linkedone{real}{Real}
 Real is a class of real number.  This class is only for consistency for other \linkingone{ring}{Ring} object.

 This class is a subclass of \linkingone{ring}{CommutativeRingElement}.

 All implemented operators in this class are delegated to Float type. 
  \initialize
  \func{Real}{\hiki{value}{number}}{\out{Real}}\\
  \spacing
  % document of basic document
  \quad Construct a Real object.
  % added document
  \spacing
  % input, output document
  \param{value} must be int, long, Float or \linkingone{rational}{Rational}.
  \method
  \subsubsection{getRing -- get ring object}\linkedtwo{real}{Real}{getRing}
   \func{getRing}{\param{self}}{\out{RealField}}\\
   \spacing
   % document of basic document
   \quad Return the real field instance.
%
\C
 \subsection{Constant -- real number with error correction}\linkedone{real}{Constant}
  This class is obsolete (Ver 1.1.0).

 \subsection{ExponentialPowerSeries -- exponential power series}\linkedone{real}{ExponentialPowerSeries}
  This class is obsolete (Ver 1.1.0).

 \subsection{AbsoluteError -- absolute error}\linkedone{real}{AbsoluteError}
  This class is obsolete (Ver 1.1.0).

 \subsection{RelativeError -- relative error}\linkedone{real}{RelativeError}
  This class is obsolete (Ver 1.1.0).

  \subsection{exp(function) -- exponential value}\linkedone{real}{exp}
   This function is obsolete (Ver 1.1.0).

  \subsection{sqrt(function) -- square root}\linkedone{real}{sqrt}
   This function is obsolete (Ver 1.1.0).

  \subsection{log(function) -- logarithm}\linkedone{real}{log}
   This function is obsolete (Ver 1.1.0).

  \subsection{log1piter(function) -- iterator of log(1+x)}\linkedone{real}{log1piter}
  \func{log1piter}{\hiki{xx}{number}}{\out{iterator}}\\
    \spacing
    % document of basic document
    \quad Return iterator for $\log(1+x)$.
    \spacing
%
  \subsection{piGaussLegendre(function) -- pi by Gauss-Legendre}\linkedone{real}{piGaussLegendre}
   This function is obsolete (Ver 1.1.0).

  \subsection{eContinuedFraction(function) -- Napier's Constant by continued fraction expansion}\linkedone{real}{eContinuedFraction}
   This function is obsolete (Ver 1.1.0).

  \subsection{floor(function) -- floor the number}\linkedone{real}{floor}
  \func{floor}{\hiki{x}{number}}{\out{integer}}\\
    \spacing
    % document of basic document
    \quad Return the biggest integer not more than \param{x}.
    \spacing
%
  \subsection{ceil(function) -- ceil the number}\linkedone{real}{ceil}
  \func{ceil}{\hiki{x}{number}}{\out{integer}}\\
    \spacing
    % document of basic document
    \quad Return the smallest integer not less than \param{x}.
    \spacing
%
  \subsection{tranc(function) -- round-off the number}\linkedone{real}{tranc}
  \func{tranc}{\hiki{x}{number}}{\out{integer}}\\
    \spacing
    % document of basic document
    \quad Return the number of rounded off \param{x}.
    \spacing
%
  \subsection{sin(function) -- sine function}\linkedone{real}{sin}
   This function is obsolete (Ver 1.1.0).

  \subsection{cos(function) -- cosine function}\linkedone{real}{cos}
   This function is obsolete (Ver 1.1.0).

  \subsection{tan(function) -- tangent function}\linkedone{real}{tan}
   This function is obsolete (Ver 1.1.0).

  \subsection{sinh(function) -- hyperbolic sine function}\linkedone{real}{sinh}
   This function is obsolete (Ver 1.1.0).

  \subsection{cosh(function) -- hyperbolic cosine function}\linkedone{real}{cosh}
   This function is obsolete (Ver 1.1.0).

  \subsection{tanh(function) -- hyperbolic tangent function}\linkedone{real}{tanh}
   This function is obsolete (Ver 1.1.0).

  \subsection{asin(function) -- arc sine function}\linkedone{real}{asin}
   This function is obsolete (Ver 1.1.0).

  \subsection{acos(function) -- arc cosine function}\linkedone{real}{acos}
   This function is obsolete (Ver 1.1.0).

  \subsection{atan(function) -- arc tangent function}\linkedone{real}{atan}
   This function is obsolete (Ver 1.1.0).

  \subsection{atan2(function) -- arc tangent function}\linkedone{real}{atan2}
   This function is obsolete (Ver 1.1.0).

  \subsection{hypot(function) -- Euclidean distance function}\linkedone{real}{hypot}
   This function is obsolete (Ver 1.1.0).

  \subsection{pow(function) -- power function}\linkedone{real}{pow}
   This function is obsolete (Ver 1.1.0).

  \subsection{degrees(function) -- convert angle to degree}\linkedone{real}{degrees}
   This function is obsolete (Ver 1.1.0).

  \subsection{radians(function) -- convert angle to radian}\linkedone{real}{radians}
   This function is obsolete (Ver 1.1.0).

  \subsection{fabs(function) -- absolute value}\linkedone{real}{fabs}
  \func{fabs}{\hiki{x}{number}}{\out{number}}\\
    \spacing
    % document of basic document
    \quad Return absolute value of \param{x}
    \spacing
%
  \subsection{fmod(function) -- modulo function over real}\linkedone{real}{fmod}
  \func{fmod}{\hiki{x}{number},\ \hiki{y}{number}}{\out{number}}\\
    \spacing
    % document of basic document
    \quad Return $x - ny$, where \param{n} is the quotient of \param{x / y}, rounded towards zero to an integer.
    \spacing
%
  \subsection{frexp(function) -- expression with base and binary exponent}\linkedone{real}{frexp}
  \func{frexp}{\hiki{x}{number}}{(\out{m},\out{e})}\\
    \spacing
    % document of basic document
    \quad Return a tuple \param{(m,e)}, where $x = m \times 2^e$, $1/2 \leq \mathtt{abs(m)} < 1$ and \param{e} is an integer.
    \spacing
    % added document
    \quad \negok This function is provided as the counter-part of math.frexp, but it might not be useful.
%
  \subsection{ldexp(function) -- construct number from base and binary exponent}\linkedone{real}{ldexp}
  \func{ldexp}{\hiki{x}{number},\ \hiki{i}{number}}{\out{number}}\\
    \spacing
    % document of basic document
    \quad Return $x \times 2^i$.
    \spacing
%
  \subsection{EulerTransform(function) -- iterator yields terms of Euler transform}\linkedone{real}{EulerTransform}
  \func{EulerTransform}{\hiki{iterator}{iterator}}{\out{iterator}}\\
    \spacing
    % document of basic document
    \quad Return an iterator which yields terms of Euler transform of the given \param{iterator}.
    \spacing
\C

%---------- end document ---------- %

\bibliographystyle{jplain}%use jbibtex
\bibliography{nzmath_references}

\end{document}


%\documentclass{report}

%%%%%%%%%%%%%%%%%%%%%%%%%%%%%%%%%%%%%%%%%%%%%%%%%%%%%%%%%%%%%
%
% macros for nzmath manual
%
%%%%%%%%%%%%%%%%%%%%%%%%%%%%%%%%%%%%%%%%%%%%%%%%%%%%%%%%%%%%%
\usepackage{amssymb,amsmath}
\usepackage{color}
\usepackage[dvipdfm,bookmarks=true,bookmarksnumbered=true,%
 pdftitle={NZMATH Users Manual},%
 pdfsubject={Manual for NZMATH Users},%
 pdfauthor={NZMATH Development Group},%
 pdfkeywords={TeX; dvipdfmx; hyperref; color;},%
 colorlinks=true]{hyperref}
\usepackage{fancybox}
\usepackage[T1]{fontenc}
%
\newcommand{\DS}{\displaystyle}
\newcommand{\C}{\clearpage}
\newcommand{\NO}{\noindent}
\newcommand{\negok}{$\dagger$}
\newcommand{\spacing}{\vspace{1pt}\\ }
% software macros
\newcommand{\nzmathzero}{{\footnotesize $\mathbb{N}\mathbb{Z}$}\texttt{MATH}}
\newcommand{\nzmath}{{\nzmathzero}\ }
\newcommand{\pythonzero}{$\mbox{\texttt{Python}}$}
\newcommand{\python}{{\pythonzero}\ }
% link macros
\newcommand{\linkingzero}[1]{{\bf \hyperlink{#1}{#1}}}%module
\newcommand{\linkingone}[2]{{\bf \hyperlink{#1.#2}{#2}}}%module,class/function etc.
\newcommand{\linkingtwo}[3]{{\bf \hyperlink{#1.#2.#3}{#3}}}%module,class,method
\newcommand{\linkedzero}[1]{\hypertarget{#1}{}}
\newcommand{\linkedone}[2]{\hypertarget{#1.#2}{}}
\newcommand{\linkedtwo}[3]{\hypertarget{#1.#2.#3}{}}
\newcommand{\linktutorial}[1]{\href{http://docs.python.org/tutorial/#1}{#1}}
\newcommand{\linktutorialone}[2]{\href{http://docs.python.org/tutorial/#1}{#2}}
\newcommand{\linklibrary}[1]{\href{http://docs.python.org/library/#1}{#1}}
\newcommand{\linklibraryone}[2]{\href{http://docs.python.org/library/#1}{#2}}
\newcommand{\pythonhp}{\href{http://www.python.org/}{\python website}}
\newcommand{\nzmathwiki}{\href{http://nzmath.sourceforge.net/wiki/}{{\nzmathzero}Wiki}}
\newcommand{\nzmathsf}{\href{http://sourceforge.net/projects/nzmath/}{\nzmath Project Page}}
\newcommand{\nzmathtnt}{\href{http://tnt.math.se.tmu.ac.jp/nzmath/}{\nzmath Project Official Page}}
% parameter name
\newcommand{\param}[1]{{\tt #1}}
% function macros
\newcommand{\hiki}[2]{{\tt #1}:\ {\em #2}}
\newcommand{\hikiopt}[3]{{\tt #1}:\ {\em #2}=#3}

\newdimen\hoge
\newdimen\truetextwidth
\newcommand{\func}[3]{%
\setbox0\hbox{#1(#2)}
\hoge=\wd0
\truetextwidth=\textwidth
\advance \truetextwidth by -2\oddsidemargin
\ifdim\hoge<\truetextwidth % short form
{\bf \colorbox{skyyellow}{#1(#2)\ $\to$ #3}}
%
\else % long form
\fcolorbox{skyyellow}{skyyellow}{%
   \begin{minipage}{\textwidth}%
   {\bf #1(#2)\\ %
    \qquad\quad   $\to$\ #3}%
   \end{minipage}%
   }%
\fi%
}

\newcommand{\out}[1]{{\em #1}}
\newcommand{\initialize}{%
  \paragraph{\large \colorbox{skyblue}{Initialize (Constructor)}}%
    \quad\\ %
    \vspace{3pt}\\
}
\newcommand{\method}{\C \paragraph{\large \colorbox{skyblue}{Methods}}}
% Attribute environment
\newenvironment{at}
{%begin
\paragraph{\large \colorbox{skyblue}{Attribute}}
\quad\\
\begin{description}
}%
{%end
\end{description}
}
% Operation environment
\newenvironment{op}
{%begin
\paragraph{\large \colorbox{skyblue}{Operations}}
\quad\\
\begin{table}[h]
\begin{center}
\begin{tabular}{|l|l|}
\hline
operator & explanation\\
\hline
}%
{%end
\hline
\end{tabular}
\end{center}
\end{table}
}
% Examples environment
\newenvironment{ex}%
{%begin
\paragraph{\large \colorbox{skyblue}{Examples}}
\VerbatimEnvironment
\renewcommand{\EveryVerbatim}{\fontencoding{OT1}\selectfont}
\begin{quote}
\begin{Verbatim}
}%
{%end
\end{Verbatim}
\end{quote}
}
%
\definecolor{skyblue}{cmyk}{0.2, 0, 0.1, 0}
\definecolor{skyyellow}{cmyk}{0.1, 0.1, 0.5, 0}
%
%\title{NZMATH User Manual\\ {\large{(for version 1.0)}}}
%\date{}
%\author{}
\begin{document}
%\maketitle
%
\setcounter{tocdepth}{3}
\setcounter{secnumdepth}{3}


\tableofcontents
\C

\chapter{Classes}


%---------- start document ---------- %
 \negok \section{ring -- for ring object}\linkedzero{ring}
 \begin{itemize}
   \item {\bf Classes}
   \begin{itemize}
     \item \linkingone{ring}{Ring}
     \item \linkingone{ring}{CommutativeRing}
     \item \linkingone{ring}{Field}
     \item \linkingone{ring}{QuotientField}
     \item \linkingone{ring}{RingElement}
     \item \linkingone{ring}{CommutativeRingElement}
     \item \linkingone{ring}{FieldElement}
     \item \linkingone{ring}{QuotientFieldElement}
     \item \linkingone{ring}{Ideal}
     \item \linkingone{ring}{ResidueClassRing}
     \item \linkingone{ring}{ResidueClass}
     \item \linkingone{ring}{CommutativeRingProperties}
   \end{itemize}
   \item {\bf Functions}
     \begin{itemize}
       \item \linkingone{ring}{getRingInstance}
       \item \linkingone{ring}{getRing}
       \item \linkingone{ring}{inverse}
       \item \linkingone{ring}{exact\_division}
     \end{itemize}
 \end{itemize}

\C

 \subsection{\negok Ring -- abstract ring}\linkedone{ring}{Ring}
  %\func{Ring}{(None)}{Ring}\\
  %\spacing
  % document of basic document
  \quad Ring is an abstract class which expresses that
    the derived classes are (in mathematical meaning) rings.\\
  \spacing
  % added document
  \quad Definition of ring (in mathematical meaning) is as follows:
  Ring is a structure with addition and multiplication. 
  It is an abelian group with addition, and a monoid with multiplication.
  The multiplication obeys the distributive law.\\
  \spacing
  % input, output document
  This class is abstract and cannot be instantiated.\\
  \begin{at}
    \item[zero]\linkedtwo{ring}{Ring}{zero}  additive unit\\
    \item[one]\linkedtwo{ring}{Ring}{one} multiplicative unit\\
  \end{at}
  \begin{op}
    \verb+A==B+ & Return whether M and N are equal or not.\\
  \end{op} 
  \method
  \subsubsection{createElement -- create an element}\linkedtwo{ring}{Ring}{createElement}
   \func{createElement}{\param{self},\ \hiki{seed}{(undefined)}}{\out{RingElement}}\\
   \spacing
   % document of basic document
   \quad Return an element of the ring with seed.\\
   \spacing
   % added document
   \quad This is an abstract method.\\
   \spacing
   % input, output document
  \subsubsection{getCharacteristic -- characteristic as ring}\linkedtwo{ring}{Ring}{getCharacteristic}
   \func{getCharacteristic}{\param{self}}{\out{integer}}\\
   \spacing
   % document of basic document
   \quad Return the characteristic of the ring.\\
   \spacing
   % added document
   \quad The Characteristic of a ring is the smallest positive integer $n$ 
   s.t. $na=0$ for any element $a$ of the ring, 
   or $0$ if there is no such natural number.\\
   This is an abstract method.\\
   \spacing
   % input, output document
  \subsubsection{issubring -- check subring}\linkedtwo{ring}{Ring}{issubring}
   \func{issubring}{\param{self},\ \hiki{other}{RingElement}}{\out{True/False}}\\
   \spacing
   % document of basic document
   \quad Report whether another ring contains the ring as a subring.\\
   \spacing
   % added document
   \quad This is an abstract method.\\
   \spacing
   % input, output document
  \subsubsection{issuperring -- check superring}\linkedtwo{ring}{Ring}{issuperring}
   \func{issuperring}{\param{self},\ \hiki{other}{RingElement}}{\out{True/False}}\\
   \spacing
   % document of basic document
   \quad Report whether the ring is a superring of another ring.\\
   \spacing
   % added document
   \quad This is an abstract method.\\
   \spacing
   % input, output document
  \subsubsection{getCommonSuperring -- get common ring}\linkedtwo{ring}{Ring}{issuperring}
   \func{getCommonSuperring}{\param{self},\ \hiki{other}{RingElement}}{\out{RingElement}}\\
   \spacing
   % document of basic document
   \quad Return common super ring of self and another ring.\\
   \spacing
   % added document
   \quad This method uses \linkingtwo{ring}{Ring}{issubring},\ \linkingtwo{ring}{Ring}{issuperring}.\\
   \spacing
   % input, output document
\C

 \subsection{\negok CommutativeRing -- abstract commutative ring}\linkedone{ring}{CommutativeRing}
  %\func{CommutativeRing}{(None)}{CommutativeRing}\\
  %\spacing
  % document of basic document
  \quad CommutativeRing is an abstract subclass of \linkingone{ring}{Ring} whose multiplication is commutative.\\
  \spacing
  % added document
  \quad CommutativeRing is subclass of \linkingone{ring}{Ring}.\\
  There are some properties of commutative rings, algorithms should be chosen accordingly. To express such properties, there is a class \linkingone{ring}{CommutativeRingProperties}. 
  \\
  \spacing
  % input, output document
  This class is abstract and cannot be instantiated.\\
  \begin{at}
    \item[properties] an instance of \linkingone{ring}{CommutativeRingProperties}
  \end{at}
  \method
  \subsubsection{getQuotientField -- create quotient field}\linkedtwo{ring}{CommutativeRing}{getQuotientField}
   \func{getQuotientField}{\param{self}}{\out{QuotientField}}\\
   \spacing
   % document of basic document
   \quad Return the quotient field of the ring.\\
   \spacing
   % added document
   \quad This is an abstract method.\\
   If quotient field of \param{self} is not available, it should raise exception.
   \spacing
   % input, output document
  \subsubsection{isdomain -- check domain}\linkedtwo{ring}{CommutativeRing}{isdomain}
   \func{isdomain}{\param{self}}{\out{True/False/None}}\\
   \spacing
   % document of basic document
   \quad Return True if the ring is actually a domain,
        False if not, or None if uncertain.\\
   \spacing
   % added document
   %\quad 
   %\spacing
   % input, output document
  \subsubsection{isnoetherian -- check Noetherian domain}\linkedtwo{ring}{CommutativeRing}{isnoetherian}
   \func{isnoetherian}{\param{self}}{\out{True/False/None}}\\
   \spacing
   % document of basic document
   \quad Return True if the ring is actually a Noetherian
        domain, False if not, or None if uncertain.\\
   \spacing
   % added document
   %\quad 
   %\spacing
   % input, output document
  \subsubsection{isufd -- check UFD}\linkedtwo{ring}{CommutativeRing}{isufd}
   \func{isufd}{\param{self}}{\out{True/False/None}}\\
   \spacing
   % document of basic document
   \quad Return True if the ring is actually a unique
        factorization domain (UFD), False if not, or None if uncertain.\\
   \spacing
   % added document
   %\quad 
   %\spacing
   % input, output document
  \subsubsection{ispid -- check PID}\linkedtwo{ring}{CommutativeRing}{ispid}
   \func{ispid}{\param{self}}{\out{True/False/None}}\\
   \spacing
   % document of basic document
   \quad Return True if the ring is actually a principal
        ideal domain (PID), False if not, or None if uncertain.\\
   \spacing
   % added document
   %\quad 
   %\spacing
   % input, output document
  \subsubsection{iseuclidean -- check Euclidean domain}\linkedtwo{ring}{CommutativeRing}{iseuclidean}
   \func{iseuclidean}{\param{self}}{\out{True/False/None}}\\
   \spacing
   % document of basic document
   \quad Return True if the ring is actually a Euclidean
        domain, False if not, or None if uncertain.\\
   \spacing
   % added document
   %\quad 
   %\spacing
   % input, output document
  \subsubsection{isfield -- check field}\linkedtwo{ring}{CommutativeRing}{isfield}
   \func{isfield}{\param{self}}{\out{True/False/None}}\\
   \spacing
   % document of basic document
   \quad Return True if the ring is actually a field,
        False if not, or None if uncertain.\\
   \spacing
   % added document
   %\quad 
   %\spacing
   % input, output document
  \subsubsection{registerModuleAction -- register action as ring}\linkedtwo{ring}{CommutativeRing}{registerModuleAction}
   \func{registerModuleAction}{\param{self},\ \hiki{action\_ring}{RingElement},\ \hiki{action}{function}}{\out{(None)}}\\
   \spacing
   % document of basic document
   \quad Register a ring \param{action\_ring}, which act on the ring through
        \param{action} so the ring be an \param{action\_ring} module.\\
   \spacing
   % added document
   \quad See \linkingtwo{ring}{CommutativeRing}{hasaction},\ \linkingtwo{ring}{CommutativeRing}{getaction}.\\
   \spacing
   % input, output document
  \subsubsection{hasaction -- check if the action has}\linkedtwo{ring}{CommutativeRing}{hasaction}
   \func{hasaction}{\param{self},\ \hiki{action\_ring}{RingElement}}{\out{True/False}}\\
   \spacing
   % document of basic document
   \quad Return True if \param{action\_ring} is registered to provide action.\\
   \spacing
   % added document
   \quad See \linkingtwo{ring}{CommutativeRing}{registerModuleAction},\ \linkingtwo{ring}{CommutativeRing}{getaction}.\\
   \spacing
   % input, output document
  \subsubsection{getaction -- get the registered action }\linkedtwo{ring}{CommutativeRing}{getaction}
   \func{hasaction}{\param{self},\ \hiki{action\_ring}{RingElement}}{\out{function}}\\
   \spacing
   % document of basic document
   \quad Return the registered action for \param{action\_ring}.\\
   \spacing
   % added document
   \quad See \linkingtwo{ring}{CommutativeRing}{registerModuleAction},\ \linkingtwo{ring}{CommutativeRing}{hasaction}.\\
   \spacing
   % input, output document
\C

 \subsection{\negok Field -- abstract field}\linkedone{ring}{Field}
  %\func{Field}{(None)}{Field}\\
  %\spacing
  % document of basic document
  \quad Field is an abstract class which expresses that
    the derived classes are (in mathematical meaning) fields,
    i.e., a commutative ring whose multiplicative monoid is a group.
  \spacing
  % added document
  \quad Field is subclass of \linkingone{ring}{CommutativeRing}.
  \linkingtwo{ring}{Ring}{getQuotientField} and \linkingtwo{ring}{CommutativeRing}{isfield} are not abstract (trivial methods).\\
  \spacing
  % input, output document
  This class is abstract and cannot be instantiated.\\
  \method
  \subsubsection{gcd -- gcd}\linkedtwo{ring}{Field}{gcd}
   \func{gcd}{\param{self},\ \hiki{a}{FieldElement},\ \hiki{b}{FieldElement}}{\out{FieldElement}}\\
   \spacing
   % document of basic document
   \quad Return the greatest common divisor of \param{a} and \param{b}.\\
   \spacing
   % added document
   \quad  A field is trivially a UFD and should provide gcd.
   If we can implement an algorithm for computing gcd in an Euclidean domain, 
   we should provide the method corresponding to the algorithm. 
   \\
   \spacing
   % input, output document
\C

 \subsection{\negok QuotientField -- abstract quotient field}\linkedone{ring}{QuotientField}
  %\func{QuotientField}{\hiki{domain}{CommutativeRing Element}}{QuotientField}\\
  %\spacing
  % document of basic document
  \quad QuotientField is an abstract class which expresses that
    the derived classes are (in mathematical meaning) quotient fields.\\
  \spacing
  % added document
  \quad \param{self} is the quotient field of \param{domain}.\\
  QuotientField is subclass of \linkingone{ring}{Field}.\\
  In the initialize step, it registers trivial action named as baseaction;
  i.e. it expresses that an element of a domain acts an element of the quotient field by using the multiplication in the domain.\\
  \spacing
  % input, output document
  This class is abstract and cannot be instantiated.\\
  \begin{at}
    \item[basedomain] domain which generates the quotient field \param{self}
  \end{at}
\C

 \subsection{\negok RingElement -- abstract element of ring}\linkedone{ring}{RingElement}
  %\func{RingElement}{\hiki{*args}{(undefined)},\ \hiki{*kwd}{(undefined)}}{RingElement}\\
  %\spacing
  % document of basic document
  \quad RingElement is an abstract class for elements of rings.\\
  \spacing
  % added document
  %\quad \\
  %\spacing
  % input, output document
  This class is abstract and cannot be instantiated.\\
  \begin{op}
    \verb+A==B+ & equality (abstract) \\
  \end{op}
  \method
 \subsubsection{getRing -- getRing}\linkedtwo{ring}{RingElement}{getRing}
   \func{getRing}{\param{self}}{\out{Ring}}\\
   \spacing
   % document of basic document
   \quad Return an object of a subclass of Ring,
        to which the element belongs.\\
   \spacing
   % added document
   \quad  This is an abstract method.\\
   \spacing
   % input, output document
\C

 \subsection{\negok CommutativeRingElement -- abstract element of commutative ring}\linkedone{ring}{CommutativeRingElement}
  %\func{CommutativeRingElement}{(None)}{RingElement}\\
  %\spacing
  % document of basic document
  \quad CommutativeRingElement is an abstract class for elements of
    commutative rings.\\
  \spacing
  % added document
  %\quad \\
  %\spacing
  % input, output document
  This class is abstract and cannot be instantiated.\\
  \method
 \subsubsection{mul\_module\_action -- apply a module action}\linkedtwo{ring}{CommutativeRingElement}{mul\_module\_action}
   \func{mul\_module\_action}{\param{self},\ \hiki{other}{RingElement}}{\out{(undefined)}}\\
   \spacing
   % document of basic document
   \quad Return the result of a module action.
        other must be in one of the action rings of self's ring.\\
   \spacing
   % added document
   \quad  This method uses \linkingtwo{ring}{RingElement}{getRing},\ \linkingone{ring}{CommutativeRing}{getaction}.
   We should consider that the method is abstract.\\
   \spacing
   % input, output document
 \subsubsection{exact\_division -- division exactly}\linkedtwo{ring}{CommutativeRingElement}{exact\_division}
   \func{exact\_division}{\param{self},\ \hiki{other}{CommutativeRingElement}}{\out{CommutativeRingElement}}\\
   \spacing
   % document of basic document
   \quad In UFD, if \param{other} divides \param{self},
   return the quotient as a UFD element. \\
   \spacing
   % added document
   \quad  The main difference with / is that / may return the
        quotient as an element of quotient field.\\
    Simple cases:
    \begin{itemize}
      \item in a Euclidean domain, 
         if remainder of euclidean division is zero, the division // is exact.
      \item in a field, there's no difference with /.
    \end{itemize}
   If \param{other} doesn't divide self, raise ValueError.
   Though \_\_divmod\_\_ can be used automatically,
   we should consider that the method is abstract.\\
   \spacing
   % input, output document 
\C

 \subsection{\negok FieldElement -- abstract element of field}\linkedone{ring}{FieldElement}
  %\func{FieldElement}{(None)}{FieldElement}\\
  %\spacing
  % document of basic document
  \quad FieldElement is an abstract class for elements of
    fields.\\
  \spacing
  % added document
  \quad FieldElement is subclass of \linkingone{ring}{CommutativeRingElement}.
  \linkingtwo{ring}{CommutativeRingElement}{exact\_division} are not abstract (trivial methods).\\
  \spacing
  % input, output document
  This class is abstract and cannot be instantiated.\\
\C

\subsection{\negok QuotientFieldElement -- abstract element of quotient field}\linkedone{ring}{QuotientFieldElement}
  %\initialize
  %\func{QuotientFieldElement}{\hiki{numerator}{CommutativeRingElement},\ \hiki{denominator}{CommutativeRingElement}}{QuotientFieldElement}\\
  %\spacing
  % document of basic document
  \quad QuotientFieldElement class is an abstract class to be used as a
    super class of concrete quotient field element classes.\\
  \spacing
  % added document
  \quad QuotientFieldElement is subclass of \linkingone{ring}{FieldElement}.\\
  \param{self} expresses $\frac{\mbox{\param{numerator}}}{\mbox{\param{denominator}}}$ in the quotient field.\\
  \spacing
  % input, output document
  \quad  This class is abstract and should not be instantiated.\\
  \param{denominator} should not be $0$.\\
 \begin{at}
    \item[numerator]\linkedtwo{ring}{QuotientField}{numerator} numerator of \param{self}\\
    \item[denominator]\linkedtwo{ring}{QuotientField}{denominator} denominator of \param{self}\\
  \end{at}
  \begin{op}
    \verb|A+B| & addition\\
    \verb+A-B+ & subtraction\\
    \verb+A*B+ & multiplication\\
    \verb+A**B+ & powering\\
    \verb+A/B+ & division\\
    \verb+-A+ & sign reversion (additive inversion)\\
    \verb+inverse(A)+ & multiplicative inversion\\
    \verb+A==B+ & equality\\
  \end{op}
\C

\subsection{\negok Ideal -- abstract ideal}\linkedone{ring}{Ideal}
  %\initialize
  %\func{Ideal}{\hiki{generators}{list},\ \hiki{aring}{CommutativeRing}}{Ideal}\\
  %\func{Ideal}{\hiki{generators}{CommutativeRingElement},\ \hiki{aring}{CommutativeRing}}{Ideal}\\
  %\spacing
  % document of basic document
  \quad Ideal class is an abstract class to represent the finitely
    generated ideals.\\
  \spacing
  % added document
  \quad \negok Because the finitely-generatedness is not a
    restriction for Noetherian rings and in the most cases only
    Noetherian rings are used, it is general enough.\\
  \\
  \spacing
  % input, output document
  \quad  This class is abstract and should not be instantiated.\\
  \param{generators} must be an element of the \param{aring} or a list of elements of the \param{aring}.\\
  If \param{generators} is an element of the \param{aring}, we consider \param{self} is the principal ideal generated by \param{generators}.
 \begin{at}
    \item[ring]\linkedtwo{ring}{Ideal}{ring} the ring belonged to by \param{self}\\
    \item[generators]\linkedtwo{ring}{Ideal}{generators} generators of the ideal \param{self}\\
  \end{at}
  \begin{op}
    \verb|I+J| & addition $\{i+j\ |\ i \in I,\ j \in J\}$\\
    \verb+I*J+ & multiplication $IJ = \{ \sum_{i,j} ij\ |\ i \in I,\  j\in  J\}$\\
    \verb+I==J+ & equality\\
    \verb+e in I+ &  For \param{e} in the ring, to which the ideal \param{I} belongs.\\
  \end{op}
  \method
   \subsubsection{issubset -- check subset}\linkedtwo{ring}{Ideal}{issubset}
   \func{issubset}{\param{self},\ \hiki{other}{Ideal}}{\out{True/False}}\\
   \spacing
   % document of basic document
   \quad Report whether another ideal contains this ideal.\\
   \spacing
   % added document
   We should consider that the method is abstract.\\
   \spacing
   % input, output document
  \subsubsection{issuperset -- check superset}\linkedtwo{ring}{Ideal}{issuperset}
   \func{issuperset}{\param{self},\ \hiki{other}{Ideal}}{\out{True/False}}\\
   \spacing
   % document of basic document
   \quad Report whether this ideal contains another ideal.\\
   \spacing
   % added document
   We should consider that the method is abstract.\\
   \spacing
   % input, output document
  \subsubsection{reduce -- reduction with the ideal}\linkedtwo{ring}{Ideal}{reduce}
  \func{issuperset}{\param{self},\ \hiki{other}{Ideal}}{\out{True/False}}\\
   \spacing
   % document of basic document
   \quad Reduce an element with the ideal to simpler representative.\\
   \spacing
   % added document
   This method is abstract.\\
   \spacing
   % input, output document
\C

\subsection{\negok ResidueClassRing -- abstract residue class ring}\linkedone{ring}{ResidueClassRing}
  \initialize
  \func{ResidueClassRing}{\hiki{ring}{CommutativeRing},\ \hiki{ideal}{Ideal}}{ResidueClassRing}\\
  \spacing
  % document of basic document
  \quad A residue class ring $R/I$, where $R$ is a commutative ring and $I$ is its ideal.
  \spacing
  % added document
  \quad   ResidueClassRing is subclass of \linkingone{ring}{CommutativeRing}.\\
  \linkingtwo{ring}{Ring}{one},\ \linkingtwo{ring}{Ring}{zero} are not abstract (trivial methods).\\
  Because we assume that \param{ring} is Noetherian, so is \param{ring}.
  \spacing
  % input, output document
  \quad  This class is abstract and should not be instantiated.\\
  \param{ring} should be an instance of \linkingone{ring}{CommutativeRing},
  and \param{ideal} must be an instance of \linkingone{ring}{Ideal},
  whose ring attribute points the same ring with the given ring.
 \begin{at}
    \item[ring]\linkedtwo{ring}{ResidueClassRing}{ring} the base ring $R$\\
    \item[ideal]\linkedtwo{ring}{ResidueClassRing}{ideal} the ideal $I$\\
  \end{at}
  \begin{op}
     \verb+A==B+ & equality\\
    \verb+e in A+ &  report whether \param{e} is in the residue ring \param{self}.\\
  \end{op}
\C

\subsection{\negok ResidueClass -- abstract an element of residue class ring}\linkedone{ring}{ResidueClass}
  \initialize
  \func{ResidueClass}{\hiki{x}{CommutativeRingElement},\ \hiki{ideal}{Ideal}}{ResidueClass}\\
  \spacing
  % document of basic document
  \quad Element of residue class ring $x+I$, where $I$ is the modulus ideal
    and $x$ is a representative element.
  \spacing
  % added document
  \quad   ResidueClass is subclass of \linkingone{ring}{CommutativeRingElement}.\\
  \spacing
  % input, output document
  \quad  This class is abstract and should not be instantiated.\\
  \param{ideal} corresponds to the ideal $I$.
  \begin{op}
     \verb|x+y| & addition\\
     \verb+x-y+ & subtraction\\
     \verb+x*y+ & multiplication\\
     \verb+A==B+ & equality\\
  \end{op}
  These operations uses \linkingtwo{ring}{Ideal}{reduce}.
\C

\subsection{\negok CommutativeRingProperties -- properties for CommutativeRingProperties}\linkedone{ring}{CommutativeRingProperties}
  \initialize
  \func{CommutativeRingProperties}{(None)}{CommutativeRingProperties}\\
  \spacing
  % document of basic document
  \quad Boolean properties of ring.\\
  \spacing
  % added document
  \quad Each property can have one of three values; {\it True}, {\it False}, or {\it None}.
    Of course {\it True} is true and {\it False} is false,
    and {\it None} means that the property is
    not set neither directly nor indirectly.\\
    CommutativeRingProperties class treats
  \begin{itemize}
    \item Euclidean (Euclidean domain),
    \item PID (Principal Ideal Domain),
    \item UFD (Unique Factorization Domain),
    \item Noetherian (Noetherian ring (domain)),
    \item field (Field)
  \end{itemize}
  \quad\\
  \spacing
  % input, output document
  \method
  \subsubsection{isfield -- check field}\linkedtwo{ring}{CommutativeRingProperties}{isfield}
   \func{isfield}{\param{self}}{\out{True/False/None}}\\
   \spacing
   % document of basic document
   \quad Return True/False according to the field flag value being set,
        otherwise return None.\\
   \spacing
   % added document
   %\spacing
   % input, output document
  \subsubsection{setIsfield -- set field}\linkedtwo{ring}{CommutativeRingProperties}{setIsfield}
   \func{isfield}{\param{self},\ \hiki{value}{True/False}}{\out{(None)}}\\
   \spacing
   % document of basic document
   \quad Set True/False value to the field flag.\\
   Propagation:
   \begin{itemize}
     \item True $\to$ euclidean\\
   \end{itemize}
   \quad\\
   \spacing
   % added document
   %\spacing
   % input, output document
  \subsubsection{iseuclidean -- check euclidean}\linkedtwo{ring}{CommutativeRingProperties}{iseuclidean}
   \func{iseuclidean}{\param{self}}{\out{True/False/None}}\\
   \spacing
   % document of basic document
   \quad Return True/False according to the euclidean flag value being set,
        otherwise return None.\\
   \spacing
   % added document
   %\spacing
   % input, output document
  \subsubsection{setIseuclidean -- set euclidean}\linkedtwo{ring}{CommutativeRingProperties}{setIseuclidean}
   \func{isfield}{\param{self},\ \hiki{value}{True/False}}{\out{(None)}}\\
   \spacing
   % document of basic document
   \quad Set True/False value to the euclidean flag.\\
   Propagation:
   \begin{itemize}
    \item True $\to$ PID\\
    \item False $\to$ field\\
   \end{itemize}
   \quad\\
   \spacing
   % added document
   %\spacing
   % input, output document
  \subsubsection{ispid -- check PID}\linkedtwo{ring}{CommutativeRingProperties}{ispid}
   \func{ispid}{\param{self}}{\out{True/False/None}}\\
   \spacing
   % document of basic document
   \quad Return True/False according to the PID flag value being set,
        otherwise return None.\\
   \spacing
   % added document
   %\spacing
   % input, output document
  \subsubsection{setIspid -- set PID}\linkedtwo{ring}{CommutativeRingProperties}{setIspid}
   \func{ispid}{\param{self},\ \hiki{value}{True/False}}{\out{(None)}}\\
   \spacing
   % document of basic document
   \quad Set True/False value to the euclidean flag.\\
   Propagation:
   \begin{itemize}
    \item True $\to$ UFD,\ Noetherian\\
    \item False $\to$ euclidean\\
   \end{itemize}
   \quad\\
   \spacing
   % added document
   %\spacing
   % input, output document
 \subsubsection{isufd -- check UFD}\linkedtwo{ring}{CommutativeRingProperties}{isufd}
   \func{isufd}{\param{self}}{\out{True/False/None}}\\
   \spacing
   % document of basic document
   \quad Return True/False according to the UFD flag value being set,
        otherwise return None.\\
   \spacing
   % added document
   %\spacing
   % input, output document
  \subsubsection{setIsufd -- set UFD}\linkedtwo{ring}{CommutativeRingProperties}{setIsufd}
   \func{isufd}{\param{self},\ \hiki{value}{True/False}}{\out{(None)}}\\
   \spacing
   % document of basic document
   \quad Set True/False value to the UFD flag.\\
   Propagation:
   \begin{itemize}
    \item True $\to$ domain\\
    \item False $\to$ PID\\
   \end{itemize}
   \quad\\
   \spacing
   % added document
   %\spacing
   % input, output document
  \subsubsection{isnoetherian -- check Noetherian}\linkedtwo{ring}{CommutativeRingProperties}{isnoetherian}
   \func{isnoetherian}{\param{self}}{\out{True/False/None}}\\
   \spacing
   % document of basic document
   \quad Return True/False according to the Noetherian flag value being set,
        otherwise return None.\\
   \spacing
   % added document
   %\spacing
   % input, output document
  \subsubsection{setIsnoetherian -- set Noetherian}\linkedtwo{ring}{CommutativeRingProperties}{setIsnoetherian}
   \func{isnoetherian}{\param{self},\ \hiki{value}{True/False}}{\out{(None)}}\\
   \spacing
   % document of basic document
   \quad Set True/False value to the Noetherian flag.\\
   Propagation:
   \begin{itemize}
    \item True $\to$ domain\\
    \item False $\to$ PID\\
   \end{itemize}
   \quad\\
   \spacing
   % added document
   %\spacing
   % input, output document
  \subsubsection{isdomain -- check domain}\linkedtwo{ring}{CommutativeRingProperties}{isdomain}
   \func{isdomain}{\param{self}}{\out{True/False/None}}\\
   \spacing
   % document of basic document
   \quad Return True/False according to the domain flag value being set,
        otherwise return None.\\
   \spacing
   % added document
   %\spacing
   % input, output document
  \subsubsection{setIsdomain -- set domain}\linkedtwo{ring}{CommutativeRingProperties}{setIsdomain}
   \func{isdomain}{\param{self},\ \hiki{value}{True/False}}{\out{(None)}}\\
   \spacing
   % document of basic document
   \quad Set True/False value to the domain flag.\\
   Propagation:
   \begin{itemize}
    \item False $\to$ UFD,\ Noetherian\\
   \end{itemize}
   \quad\\
   \spacing
   % added document
   %\spacing
   % input, output document
\C
  \subsection{getRingInstance(function)}\linkedone{ring}{getRingInstance}
  \func{getRingInstance}{\hiki{obj}{RingElement}}{\out{RingElement}}\\
   \spacing
   % document of basic document
   \quad  Return a RingElement instance which equals \param{obj}.\\
   \spacing
   % added document
   \quad Mainly for python built-in objects such as int or float.\\
   \spacing
   % input, output document
  \subsection{getRing(function)}\linkedone{ring}{getRing}
  \func{getRing}{\hiki{obj}{RingElement}}{\out{Ring}}\\
   \spacing
   % document of basic document
   \quad  Return a ring to which \param{obj} belongs.\\
   \spacing
   % added document
   \quad Mainly for python built-in objects such as int or float.\\
   \spacing
   % input, output document
  \subsection{inverse(function)}\linkedone{ring}{inverse}
  \func{inverse}{\hiki{obj}{CommutativeRingElement}}{\out{QuotientFieldElement}}\\
   \spacing
   % document of basic document
   \quad  Return the inverse of \param{obj}.
   The inverse can be in the quotient field,
   if the \param{obj} is an element of non-field domain.\\
   \spacing
   % added document
   \quad Mainly for python built-in objects such as int or float.\\
   \spacing
   % input, output document
  \subsection{exact\_division(function)}\linkedone{ring}{exact\_division}
  \func{exact\_division}{\hiki{self}{RingElement},\ \hiki{other}{RingElement}}{\out{RingElement}}\\
   \spacing
   % document of basic document
   \quad  Return the division of \param{self} by \param{other} if the division is exact.
   \spacing
   % added document
   \quad Mainly for python built-in objects such as int or float.\\
   \spacing
   % input, output document
\begin{ex}
>>> print ring.getRing(3)
Z
>>> print ring.exact_division(6, 3)
2L
\end{ex}%Don't indent!
\C

%---------- end document ---------- %

\bibliographystyle{jplain}%use jbibtex
\bibliography{nzmath_references}

\end{document}


%\documentclass{report}

%%%%%%%%%%%%%%%%%%%%%%%%%%%%%%%%%%%%%%%%%%%%%%%%%%%%%%%%%%%%%
%
% macros for nzmath manual
%
%%%%%%%%%%%%%%%%%%%%%%%%%%%%%%%%%%%%%%%%%%%%%%%%%%%%%%%%%%%%%
\usepackage{amssymb,amsmath}
\usepackage{color}
\usepackage[dvipdfm,bookmarks=true,bookmarksnumbered=true,%
 pdftitle={NZMATH Users Manual},%
 pdfsubject={Manual for NZMATH Users},%
 pdfauthor={NZMATH Development Group},%
 pdfkeywords={TeX; dvipdfmx; hyperref; color;},%
 colorlinks=true]{hyperref}
\usepackage{fancybox}
\usepackage[T1]{fontenc}
%
\newcommand{\DS}{\displaystyle}
\newcommand{\C}{\clearpage}
\newcommand{\NO}{\noindent}
\newcommand{\negok}{$\dagger$}
\newcommand{\spacing}{\vspace{1pt}\\ }
% software macros
\newcommand{\nzmathzero}{{\footnotesize $\mathbb{N}\mathbb{Z}$}\texttt{MATH}}
\newcommand{\nzmath}{{\nzmathzero}\ }
\newcommand{\pythonzero}{$\mbox{\texttt{Python}}$}
\newcommand{\python}{{\pythonzero}\ }
% link macros
\newcommand{\linkingzero}[1]{{\bf \hyperlink{#1}{#1}}}%module
\newcommand{\linkingone}[2]{{\bf \hyperlink{#1.#2}{#2}}}%module,class/function etc.
\newcommand{\linkingtwo}[3]{{\bf \hyperlink{#1.#2.#3}{#3}}}%module,class,method
\newcommand{\linkedzero}[1]{\hypertarget{#1}{}}
\newcommand{\linkedone}[2]{\hypertarget{#1.#2}{}}
\newcommand{\linkedtwo}[3]{\hypertarget{#1.#2.#3}{}}
\newcommand{\linktutorial}[1]{\href{http://docs.python.org/tutorial/#1}{#1}}
\newcommand{\linktutorialone}[2]{\href{http://docs.python.org/tutorial/#1}{#2}}
\newcommand{\linklibrary}[1]{\href{http://docs.python.org/library/#1}{#1}}
\newcommand{\linklibraryone}[2]{\href{http://docs.python.org/library/#1}{#2}}
\newcommand{\pythonhp}{\href{http://www.python.org/}{\python website}}
\newcommand{\nzmathwiki}{\href{http://nzmath.sourceforge.net/wiki/}{{\nzmathzero}Wiki}}
\newcommand{\nzmathsf}{\href{http://sourceforge.net/projects/nzmath/}{\nzmath Project Page}}
\newcommand{\nzmathtnt}{\href{http://tnt.math.se.tmu.ac.jp/nzmath/}{\nzmath Project Official Page}}
% parameter name
\newcommand{\param}[1]{{\tt #1}}
% function macros
\newcommand{\hiki}[2]{{\tt #1}:\ {\em #2}}
\newcommand{\hikiopt}[3]{{\tt #1}:\ {\em #2}=#3}

\newdimen\hoge
\newdimen\truetextwidth
\newcommand{\func}[3]{%
\setbox0\hbox{#1(#2)}
\hoge=\wd0
\truetextwidth=\textwidth
\advance \truetextwidth by -2\oddsidemargin
\ifdim\hoge<\truetextwidth % short form
{\bf \colorbox{skyyellow}{#1(#2)\ $\to$ #3}}
%
\else % long form
\fcolorbox{skyyellow}{skyyellow}{%
   \begin{minipage}{\textwidth}%
   {\bf #1(#2)\\ %
    \qquad\quad   $\to$\ #3}%
   \end{minipage}%
   }%
\fi%
}

\newcommand{\out}[1]{{\em #1}}
\newcommand{\initialize}{%
  \paragraph{\large \colorbox{skyblue}{Initialize (Constructor)}}%
    \quad\\ %
    \vspace{3pt}\\
}
\newcommand{\method}{\C \paragraph{\large \colorbox{skyblue}{Methods}}}
% Attribute environment
\newenvironment{at}
{%begin
\paragraph{\large \colorbox{skyblue}{Attribute}}
\quad\\
\begin{description}
}%
{%end
\end{description}
}
% Operation environment
\newenvironment{op}
{%begin
\paragraph{\large \colorbox{skyblue}{Operations}}
\quad\\
\begin{table}[h]
\begin{center}
\begin{tabular}{|l|l|}
\hline
operator & explanation\\
\hline
}%
{%end
\hline
\end{tabular}
\end{center}
\end{table}
}
% Examples environment
\newenvironment{ex}%
{%begin
\paragraph{\large \colorbox{skyblue}{Examples}}
\VerbatimEnvironment
\renewcommand{\EveryVerbatim}{\fontencoding{OT1}\selectfont}
\begin{quote}
\begin{Verbatim}
}%
{%end
\end{Verbatim}
\end{quote}
}
%
\definecolor{skyblue}{cmyk}{0.2, 0, 0.1, 0}
\definecolor{skyyellow}{cmyk}{0.1, 0.1, 0.5, 0}
%
%\title{NZMATH User Manual\\ {\large{(for version 1.0)}}}
%\date{}
%\author{}
\begin{document}
%\maketitle
%
\setcounter{tocdepth}{3}
\setcounter{secnumdepth}{3}


\tableofcontents
\C

\chapter{Classes}


%---------- start document ---------- %
 \section{vector -- vector object and arithmetic}\linkedzero{vector}
 \begin{itemize}
   \item {\bf Classes}
   \begin{itemize}
     \item \linkingone{vector}{Vector}
   \end{itemize}
   \item {\bf Functions}
     \begin{itemize}
       \item \linkingone{vector}{innerProduct}
     \end{itemize}
 \end{itemize}

This module provides an exception class.
\begin{description}
  \item[VectorSizeError]:\ Report vector size is invalid. (Mainly for operations with two vectors.)
\end{description}

\C

 \subsection{Vector -- vector class}\linkedone{vector}{Vector}
 Vector is a class for vector.
  \initialize
  \func{Vector}{\hiki{compo}{list}}{\out{Vector}}\\
  \spacing
  % document of basic document
  \quad Create Vector object from \param{compo}.
  % added document
  %
  % \spacing
  % input, output document
  \param{compo} must be a list of elements which are an integer or an instance of \linkingone{ring}{RingElement}.
  \begin{at}
    \item[compo]\linkedtwo{vector}{Vector}{compo}:\\ It expresses component of vector.
  \end{at}
  \begin{op}
    \verb|u+v| & Vector sum.\\
    \verb|u-v| & Vector subtraction.\\
    \verb|A*v| & Multiplication vector with matrix\\
    \verb|a*v| & or scalar multiplication.\\
    \verb|v//a| & Scalar division.\\
    \verb|v%n| & Reduction each elements of \linkingtwo{vector}{Vector}{compo}\\
    \verb|-v| & element negation.\\
    \verb|u==v| & equality.\\
    \verb|u!=v| & inequality.\\
    \verb+v[i]+ & Return the coefficient of i-th element of Vector.\\
    \verb+v[i] = c+ & Replace the coefficient of i-th element of Vector by c.\\
    \verb|len(v)| & return length of \linkingtwo{vector}{Vector}{compo}.\\
    \verb|repr(v)| & return representation string.\\
    \verb|str(v)| & return string of \linkingtwo{vector}{Vector}{compo}.\\
  \end{op}
  Note that index is 1-origin, which is standard in mathematics field.
\begin{ex}
>>> A = vector.Vector([1, 2])
>>> A
Vector([1, 2])
>>> A.compo
[1, 2]
>>> B = vector.Vector([2, 1])
>>> A + B
Vector([3, 3])
>>> A % 2
Vector([1, 0])
>>> A[1]
1
>>> len(B)
2
\end{ex}%Don't indent!
  \method
  \subsubsection{copy -- copy itself}\linkedtwo{vector}{Vector}{copy}
   \func{copy}{\param{self}}{\out{Vector}}\\
   \spacing
   % document of basic document
   \quad Return copy of \param{self}.\\
   \spacing
   % added document
   %\quad \negok Note that this function returns integer only.\\
   %\spacing
   % input, output document
   %\quad \param{a} must be int, long or rational.Integer.\\
  \subsubsection{set -- set other compo}\linkedtwo{vector}{Vector}{set}
   \func{set}{\param{self},\ \hiki{compo}{list}}{(None)}\\
   \spacing
   % document of basic document
   \quad Substitute \linkingtwo{vector}{Vector}{compo} with \param{compo}.\\
   \spacing
   % added document
   %\quad \negok Note that this function returns integer only.\\
   %\spacing
   % input, output document
   %\quad \param{a} must be int, long or rational.Integer.\\
  \subsubsection{indexOfNoneZero -- first non-zero coordinate}\linkedtwo{vector}{Vector}{indexOfNoneZero}
   \func{indexOfNoneZero}{\param{self}}{integer}\\
   \spacing
   % document of basic document
   \quad Return the first index of non-zero element of \param{self}.\linkingtwo{vector}{Vector}{compo}.\\
   \spacing
   % added document
   \quad \negok Raise ValueError if all elements of \linkingtwo{vector}{Vector}{compo} are zero.\\
   \spacing
   % input, output document
   %\quad \param{a} must be int, long or rational.Integer.\\
  \subsubsection{toMatrix -- convert to Matrix object}\linkedtwo{vector}{Vector}{toMatrix}
   \func{toMatrix}{\param{self},\ \hikiopt{as\_column}{bool}{False}}{\out{Matrix}}\\
   \spacing
   % document of basic document
   \quad Return \linkingone{matrix}{Matrix} object using \linkingone{matrix}{createMatrix} function.\\
   \spacing
   % added document
   %\quad \negok Note that this function returns integer only.\\
   %\spacing
   % input, output document
   \quad If \param{as\_column} is True, create the column matrix with \param{self}.
   Otherwise, create the row matrix.\\
\begin{ex}
>>> A = vector.Vector([0, 4, 5])
>>> A.indexOfNoneZero()
2
>>> print A.toMatrix()
0 4 5
>>> print A.toMatrix()
0
4
5
\end{ex}%Don't indent!
\C
  \subsection{innerProduct(function) -- inner product}\linkedone{vector}{innerProduct}
  \func{innerProduct}{\hiki{bra}{Vector}, \ \hiki{ket}{Vector}}{\out{RingElement}}\\
   \spacing
   % document of basic document
   \quad Return the inner product of \param{bra} and \param{ket}.\\
   \spacing
   % added document
   \quad The function supports Hermitian inner product for elements in the complex number field.\\
   \spacing
   % input, output document
   \quad \negok Note that the returned value depends on type of elements.\\
\begin{ex}
>>> A = vector.Vector([1, 2, 3])
>>> B = vector.Vector([2, 1, 0])
>>> vector.innerProduct(A, B)
4
>>> C = vector.Vector([1+1j, 2+2j, 3+3j])
>>> vector.innerProduct(C, C)
(28+0j)
\end{ex}%Don't indent!
\C

%---------- end document ---------- %

\bibliographystyle{jplain}%use jbibtex
\bibliography{nzmath_references}

\end{document}


%\documentclass{report}

%%%%%%%%%%%%%%%%%%%%%%%%%%%%%%%%%%%%%%%%%%%%%%%%%%%%%%%%%%%%%
%
% macros for nzmath manual
%
%%%%%%%%%%%%%%%%%%%%%%%%%%%%%%%%%%%%%%%%%%%%%%%%%%%%%%%%%%%%%
\usepackage{amssymb,amsmath}
\usepackage{color}
\usepackage[dvipdfm,bookmarks=true,bookmarksnumbered=true,%
 pdftitle={NZMATH Users Manual},%
 pdfsubject={Manual for NZMATH Users},%
 pdfauthor={NZMATH Development Group},%
 pdfkeywords={TeX; dvipdfmx; hyperref; color;},%
 colorlinks=true]{hyperref}
\usepackage{fancybox}
\usepackage[T1]{fontenc}
%
\newcommand{\DS}{\displaystyle}
\newcommand{\C}{\clearpage}
\newcommand{\NO}{\noindent}
\newcommand{\negok}{$\dagger$}
\newcommand{\spacing}{\vspace{1pt}\\ }
% software macros
\newcommand{\nzmathzero}{{\footnotesize $\mathbb{N}\mathbb{Z}$}\texttt{MATH}}
\newcommand{\nzmath}{{\nzmathzero}\ }
\newcommand{\pythonzero}{$\mbox{\texttt{Python}}$}
\newcommand{\python}{{\pythonzero}\ }
% link macros
\newcommand{\linkingzero}[1]{{\bf \hyperlink{#1}{#1}}}%module
\newcommand{\linkingone}[2]{{\bf \hyperlink{#1.#2}{#2}}}%module,class/function etc.
\newcommand{\linkingtwo}[3]{{\bf \hyperlink{#1.#2.#3}{#3}}}%module,class,method
\newcommand{\linkedzero}[1]{\hypertarget{#1}{}}
\newcommand{\linkedone}[2]{\hypertarget{#1.#2}{}}
\newcommand{\linkedtwo}[3]{\hypertarget{#1.#2.#3}{}}
\newcommand{\linktutorial}[1]{\href{http://docs.python.org/tutorial/#1}{#1}}
\newcommand{\linktutorialone}[2]{\href{http://docs.python.org/tutorial/#1}{#2}}
\newcommand{\linklibrary}[1]{\href{http://docs.python.org/library/#1}{#1}}
\newcommand{\linklibraryone}[2]{\href{http://docs.python.org/library/#1}{#2}}
\newcommand{\pythonhp}{\href{http://www.python.org/}{\python website}}
\newcommand{\nzmathwiki}{\href{http://nzmath.sourceforge.net/wiki/}{{\nzmathzero}Wiki}}
\newcommand{\nzmathsf}{\href{http://sourceforge.net/projects/nzmath/}{\nzmath Project Page}}
\newcommand{\nzmathtnt}{\href{http://tnt.math.se.tmu.ac.jp/nzmath/}{\nzmath Project Official Page}}
% parameter name
\newcommand{\param}[1]{{\tt #1}}
% function macros
\newcommand{\hiki}[2]{{\tt #1}:\ {\em #2}}
\newcommand{\hikiopt}[3]{{\tt #1}:\ {\em #2}=#3}

\newdimen\hoge
\newdimen\truetextwidth
\newcommand{\func}[3]{%
\setbox0\hbox{#1(#2)}
\hoge=\wd0
\truetextwidth=\textwidth
\advance \truetextwidth by -2\oddsidemargin
\ifdim\hoge<\truetextwidth % short form
{\bf \colorbox{skyyellow}{#1(#2)\ $\to$ #3}}
%
\else % long form
\fcolorbox{skyyellow}{skyyellow}{%
   \begin{minipage}{\textwidth}%
   {\bf #1(#2)\\ %
    \qquad\quad   $\to$\ #3}%
   \end{minipage}%
   }%
\fi%
}

\newcommand{\out}[1]{{\em #1}}
\newcommand{\initialize}{%
  \paragraph{\large \colorbox{skyblue}{Initialize (Constructor)}}%
    \quad\\ %
    \vspace{3pt}\\
}
\newcommand{\method}{\C \paragraph{\large \colorbox{skyblue}{Methods}}}
% Attribute environment
\newenvironment{at}
{%begin
\paragraph{\large \colorbox{skyblue}{Attribute}}
\quad\\
\begin{description}
}%
{%end
\end{description}
}
% Operation environment
\newenvironment{op}
{%begin
\paragraph{\large \colorbox{skyblue}{Operations}}
\quad\\
\begin{table}[h]
\begin{center}
\begin{tabular}{|l|l|}
\hline
operator & explanation\\
\hline
}%
{%end
\hline
\end{tabular}
\end{center}
\end{table}
}
% Examples environment
\newenvironment{ex}%
{%begin
\paragraph{\large \colorbox{skyblue}{Examples}}
\VerbatimEnvironment
\renewcommand{\EveryVerbatim}{\fontencoding{OT1}\selectfont}
\begin{quote}
\begin{Verbatim}
}%
{%end
\end{Verbatim}
\end{quote}
}
%
\definecolor{skyblue}{cmyk}{0.2, 0, 0.1, 0}
\definecolor{skyyellow}{cmyk}{0.1, 0.1, 0.5, 0}
%
%\title{NZMATH User Manual\\ {\large{(for version 1.0)}}}
%\date{}
%\author{}
\begin{document}
%\maketitle
%
\setcounter{tocdepth}{3}
\setcounter{secnumdepth}{3}


\tableofcontents
\C

\chapter{Functions}


%---------- start document ---------- %
 \section{factor.ecm -- ECM factorization}\linkedzero{factor.ecm}

 This module has curve type constants:
 \begin{description}\linkedone{factor.ecm}{curvetype}
   \item[S]\linkedone{factor.ecm}{S}: aka SUYAMA. Suyama's parameter selection strategy.
   \item[B]\linkedone{factor.ecm}{B}: aka BERNSTEIN. Bernstein's parameter selection strategy.
   \item[A1]\linkedone{factor.ecm}{A1}: aka ASUNCION1. Asuncion's parameter selection strategy variant 1.
   \item[A2]\linkedone{factor.ecm}{A2}: aka ASUNCION2.  ditto 2.
   \item[A3]\linkedone{factor.ecm}{A3}: aka ASUNCION3.  ditto 3.
   \item[A4]\linkedone{factor.ecm}{A4}: aka ASUNCION4.  ditto 4.
   \item[A5]\linkedone{factor.ecm}{A5}: aka ASUNCION5.  ditto 5.
 \end{description}
 See J.S.Asuncion's master thesis~\cite{Janice} for details of each family.

%
  \subsection{ecm -- elliptic curve method}\linkedone{factor.ecm}{ecm}
   \func{ecm}
   {%
     \hiki{n}{integer},\ %
     \hikiopt{curve\_type}{\linkingone{factor.ecm}{curvetype}}{A1},\ %
     \hikiopt{incs}{integer}{3},\ %
     \hikiopt{trials}{integer}{20},\ %
     \hikiopt{verbose}{bool}{False}%
   }{%
     \out{integer}
   }\\
   \spacing
   % document of basic document
   \quad Find a factor of \param{n} by elliptic curve method.\\
   \spacing
   % added document
   If it cannot find non-trivial factor of $n$, then it returns $1$.\\ 
   \spacing
   % input, output document
   \quad \param{curve\_type} should be chosen from \linkingone{factor.ecm}{curvetype} constants above.\\
   \quad The second optional argument \param{incs} specifies a number
   of changes of bounds. The function repeats factorization
   trials several times changing curves with a fixed bounds.\\
   \quad Optional argument \param{trials} can control how quickly move
   on to the next higher bounds.\\
   \quad \param{verbose} toggles verbosity.\\
\C

%---------- end document ---------- %

\bibliographystyle{jplain}%use jbibtex
\bibliography{nzmath_references}

\end{document}


%\documentclass{report}

%%%%%%%%%%%%%%%%%%%%%%%%%%%%%%%%%%%%%%%%%%%%%%%%%%%%%%%%%%%%%
%
% macros for nzmath manual
%
%%%%%%%%%%%%%%%%%%%%%%%%%%%%%%%%%%%%%%%%%%%%%%%%%%%%%%%%%%%%%
\usepackage{amssymb,amsmath}
\usepackage{color}
\usepackage[dvipdfm,bookmarks=true,bookmarksnumbered=true,%
 pdftitle={NZMATH Users Manual},%
 pdfsubject={Manual for NZMATH Users},%
 pdfauthor={NZMATH Development Group},%
 pdfkeywords={TeX; dvipdfmx; hyperref; color;},%
 colorlinks=true]{hyperref}
\usepackage{fancybox}
\usepackage[T1]{fontenc}
%
\newcommand{\DS}{\displaystyle}
\newcommand{\C}{\clearpage}
\newcommand{\NO}{\noindent}
\newcommand{\negok}{$\dagger$}
\newcommand{\spacing}{\vspace{1pt}\\ }
% software macros
\newcommand{\nzmathzero}{{\footnotesize $\mathbb{N}\mathbb{Z}$}\texttt{MATH}}
\newcommand{\nzmath}{{\nzmathzero}\ }
\newcommand{\pythonzero}{$\mbox{\texttt{Python}}$}
\newcommand{\python}{{\pythonzero}\ }
% link macros
\newcommand{\linkingzero}[1]{{\bf \hyperlink{#1}{#1}}}%module
\newcommand{\linkingone}[2]{{\bf \hyperlink{#1.#2}{#2}}}%module,class/function etc.
\newcommand{\linkingtwo}[3]{{\bf \hyperlink{#1.#2.#3}{#3}}}%module,class,method
\newcommand{\linkedzero}[1]{\hypertarget{#1}{}}
\newcommand{\linkedone}[2]{\hypertarget{#1.#2}{}}
\newcommand{\linkedtwo}[3]{\hypertarget{#1.#2.#3}{}}
\newcommand{\linktutorial}[1]{\href{http://docs.python.org/tutorial/#1}{#1}}
\newcommand{\linktutorialone}[2]{\href{http://docs.python.org/tutorial/#1}{#2}}
\newcommand{\linklibrary}[1]{\href{http://docs.python.org/library/#1}{#1}}
\newcommand{\linklibraryone}[2]{\href{http://docs.python.org/library/#1}{#2}}
\newcommand{\pythonhp}{\href{http://www.python.org/}{\python website}}
\newcommand{\nzmathwiki}{\href{http://nzmath.sourceforge.net/wiki/}{{\nzmathzero}Wiki}}
\newcommand{\nzmathsf}{\href{http://sourceforge.net/projects/nzmath/}{\nzmath Project Page}}
\newcommand{\nzmathtnt}{\href{http://tnt.math.se.tmu.ac.jp/nzmath/}{\nzmath Project Official Page}}
% parameter name
\newcommand{\param}[1]{{\tt #1}}
% function macros
\newcommand{\hiki}[2]{{\tt #1}:\ {\em #2}}
\newcommand{\hikiopt}[3]{{\tt #1}:\ {\em #2}=#3}

\newdimen\hoge
\newdimen\truetextwidth
\newcommand{\func}[3]{%
\setbox0\hbox{#1(#2)}
\hoge=\wd0
\truetextwidth=\textwidth
\advance \truetextwidth by -2\oddsidemargin
\ifdim\hoge<\truetextwidth % short form
{\bf \colorbox{skyyellow}{#1(#2)\ $\to$ #3}}
%
\else % long form
\fcolorbox{skyyellow}{skyyellow}{%
   \begin{minipage}{\textwidth}%
   {\bf #1(#2)\\ %
    \qquad\quad   $\to$\ #3}%
   \end{minipage}%
   }%
\fi%
}

\newcommand{\out}[1]{{\em #1}}
\newcommand{\initialize}{%
  \paragraph{\large \colorbox{skyblue}{Initialize (Constructor)}}%
    \quad\\ %
    \vspace{3pt}\\
}
\newcommand{\method}{\C \paragraph{\large \colorbox{skyblue}{Methods}}}
% Attribute environment
\newenvironment{at}
{%begin
\paragraph{\large \colorbox{skyblue}{Attribute}}
\quad\\
\begin{description}
}%
{%end
\end{description}
}
% Operation environment
\newenvironment{op}
{%begin
\paragraph{\large \colorbox{skyblue}{Operations}}
\quad\\
\begin{table}[h]
\begin{center}
\begin{tabular}{|l|l|}
\hline
operator & explanation\\
\hline
}%
{%end
\hline
\end{tabular}
\end{center}
\end{table}
}
% Examples environment
\newenvironment{ex}%
{%begin
\paragraph{\large \colorbox{skyblue}{Examples}}
\VerbatimEnvironment
\renewcommand{\EveryVerbatim}{\fontencoding{OT1}\selectfont}
\begin{quote}
\begin{Verbatim}
}%
{%end
\end{Verbatim}
\end{quote}
}
%
\definecolor{skyblue}{cmyk}{0.2, 0, 0.1, 0}
\definecolor{skyyellow}{cmyk}{0.1, 0.1, 0.5, 0}
%
%\title{NZMATH User Manual\\ {\large{(for version 1.0)}}}
%\date{}
%\author{}
\begin{document}
%\maketitle
%
\setcounter{tocdepth}{3}
\setcounter{secnumdepth}{3}


\tableofcontents
\C

\chapter{Functions}

%---------- start document ---------- %
 \section{factor.find -- find a factor}\linkedzero{factor.find}
%
\quad All methods in this module return one of a factor of given integer.
If it failes to find a non-trivial factor, it returns \(1\).
Note that \(1\) is a factor anyway.

\param{verbose} boolean flag can be specified for verbose reports.
To receive these messages, you have to prepare a logger (see \linklibrary{logging}).

  \subsection{trialDivision -- trial division}\linkedone{factor.find}{trialDivision}
   \func{trialDivision}
   {%
     \hiki{n}{integer},\ %
     **\param{options}
   }{%
     \out{integer}
   }\\
   \spacing
   % document of basic document
   \quad Return a factor of \param{n} by trial divisions.\\
   \spacing
   % input, output document
   \quad \param{options} can be either one of the following:
   \begin{enumerate}
   \item \param{start} and \param{stop} as range parameters.
         In addition to these, \param{step} is also available.
   \item \param{iterator} as an iterator of primes.
   \end{enumerate}
   If \param{options} is not given, the function divides \param{n} by primes from \(2\) to the floor of the square root of \param{n} until a non-trivial factor is found.\\
   \quad \param{verbose} boolean flag can be specified for verbose reports.\\
%
  \subsection{pmom -- $p-1$ method}\linkedone{factor.find}{pmom}
   \func{pmom}{%
     \hiki{n}{integer},\ %
     **\param{options}
   }{%
     \out{integer}
   }\\
   \spacing
   % document of basic document
   \quad Return a factor of \param{n} by the \(p-1\) method.\\
   \spacing
   % added document
   \quad The function tries to find a non-trivial factor of \param{n}
   using Algorithm 8.8.2 (\(p-1\) first stage) of \cite{Cohen1}.
   In the case of \(n = 2^{i}\), the function will not terminate.
   Due to the nature of the method, the method may return the
   trivial factor only.\\
   \spacing
   % input, output document
   \quad \param{verbose} Boolean flag can be specified for verbose reports,
   though it is not so verbose indeed.\\
%
   \subsection{rhomethod -- $\rho$ method}\linkedone{factor.find}{rhomethod}
   \func{rhomethod}{%
     \hiki{n}{integer},\ %
     **\param{options}
   }{%
     \out{integer}
   }\\
   \spacing
   % document of basic document
   \quad Return a factor of \param{n} by Pollard's \(\rho\) method.\\
   \spacing
   % added document
   The implementation refers the explanation in \cite{Pomerance}.
   Due to the nature of the method, a factorization may return the
   trivial factor only.\\
   \spacing
   % input, output document
   \quad \param{verbose} Boolean flag can be specified for verbose reports.\\
\begin{ex}
>>> factor.find.trialDivision(1001)
7
>>> factor.find.trialDivision(1001, start=10, stop=32)
11
>>> factor.find.pmom(1001)
91
>>> import logging
>>> logging.basicConfig()
>>> factor.find.rhomethod(1001, verbose=True)
INFO:nzmath.factor.find:887 748
13
\end{ex}%Don't indent!(indent causes an error.)
\C

%---------- end document ---------- %

\bibliographystyle{jplain}%use jbibtex
\bibliography{nzmath_references}

\end{document}
%\documentclass{report}

\documentclass{report}

%%%%%%%%%%%%%%%%%%%%%%%%%%%%%%%%%%%%%%%%%%%%%%%%%%%%%%%%%%%%%
%
% macros for nzmath manual
%
%%%%%%%%%%%%%%%%%%%%%%%%%%%%%%%%%%%%%%%%%%%%%%%%%%%%%%%%%%%%%
\usepackage{amssymb,amsmath}
\usepackage{color}
\usepackage[dvipdfm,bookmarks=true,bookmarksnumbered=true,%
 pdftitle={NZMATH Users Manual},%
 pdfsubject={Manual for NZMATH Users},%
 pdfauthor={NZMATH Development Group},%
 pdfkeywords={TeX; dvipdfmx; hyperref; color;},%
 colorlinks=true]{hyperref}
\usepackage{fancybox}
\usepackage[T1]{fontenc}
%
\newcommand{\DS}{\displaystyle}
\newcommand{\C}{\clearpage}
\newcommand{\NO}{\noindent}
\newcommand{\negok}{$\dagger$}
\newcommand{\spacing}{\vspace{1pt}\\ }
% software macros
\newcommand{\nzmathzero}{{\footnotesize $\mathbb{N}\mathbb{Z}$}\texttt{MATH}}
\newcommand{\nzmath}{{\nzmathzero}\ }
\newcommand{\pythonzero}{$\mbox{\texttt{Python}}$}
\newcommand{\python}{{\pythonzero}\ }
% link macros
\newcommand{\linkingzero}[1]{{\bf \hyperlink{#1}{#1}}}%module
\newcommand{\linkingone}[2]{{\bf \hyperlink{#1.#2}{#2}}}%module,class/function etc.
\newcommand{\linkingtwo}[3]{{\bf \hyperlink{#1.#2.#3}{#3}}}%module,class,method
\newcommand{\linkedzero}[1]{\hypertarget{#1}{}}
\newcommand{\linkedone}[2]{\hypertarget{#1.#2}{}}
\newcommand{\linkedtwo}[3]{\hypertarget{#1.#2.#3}{}}
\newcommand{\linktutorial}[1]{\href{http://docs.python.org/tutorial/#1}{#1}}
\newcommand{\linktutorialone}[2]{\href{http://docs.python.org/tutorial/#1}{#2}}
\newcommand{\linklibrary}[1]{\href{http://docs.python.org/library/#1}{#1}}
\newcommand{\linklibraryone}[2]{\href{http://docs.python.org/library/#1}{#2}}
\newcommand{\pythonhp}{\href{http://www.python.org/}{\python website}}
\newcommand{\nzmathwiki}{\href{http://nzmath.sourceforge.net/wiki/}{{\nzmathzero}Wiki}}
\newcommand{\nzmathsf}{\href{http://sourceforge.net/projects/nzmath/}{\nzmath Project Page}}
\newcommand{\nzmathtnt}{\href{http://tnt.math.se.tmu.ac.jp/nzmath/}{\nzmath Project Official Page}}
% parameter name
\newcommand{\param}[1]{{\tt #1}}
% function macros
\newcommand{\hiki}[2]{{\tt #1}:\ {\em #2}}
\newcommand{\hikiopt}[3]{{\tt #1}:\ {\em #2}=#3}

\newdimen\hoge
\newdimen\truetextwidth
\newcommand{\func}[3]{%
\setbox0\hbox{#1(#2)}
\hoge=\wd0
\truetextwidth=\textwidth
\advance \truetextwidth by -2\oddsidemargin
\ifdim\hoge<\truetextwidth % short form
{\bf \colorbox{skyyellow}{#1(#2)\ $\to$ #3}}
%
\else % long form
\fcolorbox{skyyellow}{skyyellow}{%
   \begin{minipage}{\textwidth}%
   {\bf #1(#2)\\ %
    \qquad\quad   $\to$\ #3}%
   \end{minipage}%
   }%
\fi%
}

\newcommand{\out}[1]{{\em #1}}
\newcommand{\initialize}{%
  \paragraph{\large \colorbox{skyblue}{Initialize (Constructor)}}%
    \quad\\ %
    \vspace{3pt}\\
}
\newcommand{\method}{\C \paragraph{\large \colorbox{skyblue}{Methods}}}
% Attribute environment
\newenvironment{at}
{%begin
\paragraph{\large \colorbox{skyblue}{Attribute}}
\quad\\
\begin{description}
}%
{%end
\end{description}
}
% Operation environment
\newenvironment{op}
{%begin
\paragraph{\large \colorbox{skyblue}{Operations}}
\quad\\
\begin{table}[h]
\begin{center}
\begin{tabular}{|l|l|}
\hline
operator & explanation\\
\hline
}%
{%end
\hline
\end{tabular}
\end{center}
\end{table}
}
% Examples environment
\newenvironment{ex}%
{%begin
\paragraph{\large \colorbox{skyblue}{Examples}}
\VerbatimEnvironment
\renewcommand{\EveryVerbatim}{\fontencoding{OT1}\selectfont}
\begin{quote}
\begin{Verbatim}
}%
{%end
\end{Verbatim}
\end{quote}
}
%
\definecolor{skyblue}{cmyk}{0.2, 0, 0.1, 0}
\definecolor{skyyellow}{cmyk}{0.1, 0.1, 0.5, 0}
%
%\title{NZMATH User Manual\\ {\large{(for version 1.0)}}}
%\date{}
%\author{}
\begin{document}
%\maketitle
%
\setcounter{tocdepth}{3}
\setcounter{secnumdepth}{3}


\tableofcontents
\C

\chapter{Functions}


%---------- start document ---------- %
 \section{factor.methods -- factoring methods}\linkedzero{factor.methods}

It uses methods of \linkingzero{factor.find} module
or some heavier methods of related modules to find a factor.
Also, classes of \linkingzero{factor.util} module is used to track
the factorization process.
\param{options} are normally passed to the underlying function without modification.

 This module uses the following type:
 \begin{description}
   \item[factorlist]\linkedone{factor.methods}{factorlist}:\\
     \param{factorlist} is a list which consists of pairs {\tt (base, index)}.
     Each pair means \(base^{index}\).
     The product of these terms expresses prime factorization.
 \end{description}
%
  \subsection{factor -- easiest way to factor}\linkedone{factor.methods}{factor}
   \func{factor}
   {%
     \hiki{n}{integer},\ %
     \hikiopt{method}{string}{'default'},\ %
     **\param{options}
   }{%
     \out{\linkingone{factor.methods}{factorlist}}
   }\\
   \spacing
   % document of basic document
   \quad Factor the given positive integer \param{n}.\\
   \spacing
   \quad By default, use several methods internally.\\
   \spacing
   \quad The optional argument \param{method} can be:
   \begin{itemize}
   \item {\tt 'ecm'}: use elliptic curve method.
   \item {\tt 'mpqs'}: use MPQS method.
   \item {\tt 'pmom'}: use \(p-1\) method.
   \item {\tt 'rhomethod'}: use Pollard's \(\rho\) method.
   \item {\tt  'trialDivision'}: use trial division.
   \end{itemize}
   (\negok In fact, the initial letter of method name suffices to specify.)\\
%
  \subsection{ecm -- elliptic curve method}\linkedone{factor.methods}{ecm}
  \func{ecm}
   {%
     \hiki{n}{integer},\ %
     **\param{options}
   }{%
     \out{\linkingone{factor.methods}{factorlist}}
   }\\
   \spacing
   % document of basic document
   \quad Factor the given integer \param{n} by elliptic curve method.\\
   \spacing
   % added document
   (See \linkingone{factor.ecm}{ecm} of \linkingzero{factor.ecm} module.)\\
%
  \subsection{mpqs -- multi-polynomial quadratic sieve method}\linkedone{factor.methods}{mpqs}
  \func{mpqs}
   {%
     \hiki{n}{integer},\ %
     **\param{options}
   }{%
     \out{\linkingone{factor.methods}{factorlist}}
   }\\
   \spacing
   % document of basic document
   \quad Factor the given integer \param{n} by multi-polynomial quadratic sieve method.\\
   \spacing
   % added document
   (See \linkingone{factor.mpqs}{mpqsfind} of \linkingzero{factor.mpqs} module.)\\

  \subsection{pmom -- $p-1$ method}\linkedone{factor.methods}{pmom}
  \func{pmom}
   {%
     \hiki{n}{integer},\ %
     **\param{options}
   }{%
     \out{\linkingone{factor.methods}{factorlist}}
   }\\
   \spacing
   % document of basic document
   \quad Factor the given integer \param{n} by \(p-1\) method.\\
   \spacing
   % added document
   \quad  The method may fail unless n has an appropriate factor for the method.\\
   (See \linkingone{factor.find}{pmom} of \linkingzero{factor.find} module.)\\

  \subsection{rhomethod -- $\rho$ method}\linkedone{factor.methods}{rhomethod}
  \func{rhomethod}
   {%
     \hiki{n}{integer},\ %
     **\param{options}
   }{%
     \out{\linkingone{factor.methods}{factorlist}}
   }\\
   \spacing
   % document of basic document
   \quad Factor the given integer \param{n} by Pollard's \(\rho\) method.\\
   \spacing
   % added document
   \quad The method is a probabilistic method, possibly fails in factorizations.\\
   (See \linkingone{factor.find}{rhomethod} of \linkingzero{factor.find} module.)\\
%
  \subsection{trialDivision -- trial division}\linkedone{factor.methods}{trialDivision}
  \func{trialDivision}
   {%
     \hiki{n}{integer},\ %
     **\param{options}
   }{%
     \out{\linkingone{factor.methods}{factorlist}}
   }\\
   \spacing
   \quad Factor the given integer \param{n} by trial division.\\
   \spacing
   \quad \param{options} for the trial sequence can be either:
   \begin{enumerate}
   \item \param{start} and \param{stop} as range parameters.
   \item \param{iterator} as an iterator of primes.
   \item \param{eratosthenes} as an upper bound to make prime sequence by sieve.
   \end{enumerate}
   If none of the options above are given, the function divides \param{n} by primes from \(2\) to the floor of the square root of \param{n} until a non-trivial factor is found.\\
   (See \linkingone{factor.find}{trialDivision} of \linkingzero{factor.find} module.)\\
%
\begin{ex}
>>> factor.methods.factor(10001)
[(73, 1), (137, 1)]
>>> factor.methods.ecm(1000001)
[(101L, 1), (9901L, 1)]
\end{ex}%Don't indent!(indent causes an error.)
\C

%---------- end document ---------- %

\bibliographystyle{jplain}%use jbibtex
\bibliography{nzmath_references}

\end{document}


%\documentclass{report}

%%%%%%%%%%%%%%%%%%%%%%%%%%%%%%%%%%%%%%%%%%%%%%%%%%%%%%%%%%%%%
%
% macros for nzmath manual
%
%%%%%%%%%%%%%%%%%%%%%%%%%%%%%%%%%%%%%%%%%%%%%%%%%%%%%%%%%%%%%
\usepackage{amssymb,amsmath}
\usepackage{color}
\usepackage[dvipdfm,bookmarks=true,bookmarksnumbered=true,%
 pdftitle={NZMATH Users Manual},%
 pdfsubject={Manual for NZMATH Users},%
 pdfauthor={NZMATH Development Group},%
 pdfkeywords={TeX; dvipdfmx; hyperref; color;},%
 colorlinks=true]{hyperref}
\usepackage{fancybox}
\usepackage[T1]{fontenc}
%
\newcommand{\DS}{\displaystyle}
\newcommand{\C}{\clearpage}
\newcommand{\NO}{\noindent}
\newcommand{\negok}{$\dagger$}
\newcommand{\spacing}{\vspace{1pt}\\ }
% software macros
\newcommand{\nzmathzero}{{\footnotesize $\mathbb{N}\mathbb{Z}$}\texttt{MATH}}
\newcommand{\nzmath}{{\nzmathzero}\ }
\newcommand{\pythonzero}{$\mbox{\texttt{Python}}$}
\newcommand{\python}{{\pythonzero}\ }
% link macros
\newcommand{\linkingzero}[1]{{\bf \hyperlink{#1}{#1}}}%module
\newcommand{\linkingone}[2]{{\bf \hyperlink{#1.#2}{#2}}}%module,class/function etc.
\newcommand{\linkingtwo}[3]{{\bf \hyperlink{#1.#2.#3}{#3}}}%module,class,method
\newcommand{\linkedzero}[1]{\hypertarget{#1}{}}
\newcommand{\linkedone}[2]{\hypertarget{#1.#2}{}}
\newcommand{\linkedtwo}[3]{\hypertarget{#1.#2.#3}{}}
\newcommand{\linktutorial}[1]{\href{http://docs.python.org/tutorial/#1}{#1}}
\newcommand{\linktutorialone}[2]{\href{http://docs.python.org/tutorial/#1}{#2}}
\newcommand{\linklibrary}[1]{\href{http://docs.python.org/library/#1}{#1}}
\newcommand{\linklibraryone}[2]{\href{http://docs.python.org/library/#1}{#2}}
\newcommand{\pythonhp}{\href{http://www.python.org/}{\python website}}
\newcommand{\nzmathwiki}{\href{http://nzmath.sourceforge.net/wiki/}{{\nzmathzero}Wiki}}
\newcommand{\nzmathsf}{\href{http://sourceforge.net/projects/nzmath/}{\nzmath Project Page}}
\newcommand{\nzmathtnt}{\href{http://tnt.math.se.tmu.ac.jp/nzmath/}{\nzmath Project Official Page}}
% parameter name
\newcommand{\param}[1]{{\tt #1}}
% function macros
\newcommand{\hiki}[2]{{\tt #1}:\ {\em #2}}
\newcommand{\hikiopt}[3]{{\tt #1}:\ {\em #2}=#3}

\newdimen\hoge
\newdimen\truetextwidth
\newcommand{\func}[3]{%
\setbox0\hbox{#1(#2)}
\hoge=\wd0
\truetextwidth=\textwidth
\advance \truetextwidth by -2\oddsidemargin
\ifdim\hoge<\truetextwidth % short form
{\bf \colorbox{skyyellow}{#1(#2)\ $\to$ #3}}
%
\else % long form
\fcolorbox{skyyellow}{skyyellow}{%
   \begin{minipage}{\textwidth}%
   {\bf #1(#2)\\ %
    \qquad\quad   $\to$\ #3}%
   \end{minipage}%
   }%
\fi%
}

\newcommand{\out}[1]{{\em #1}}
\newcommand{\initialize}{%
  \paragraph{\large \colorbox{skyblue}{Initialize (Constructor)}}%
    \quad\\ %
    \vspace{3pt}\\
}
\newcommand{\method}{\C \paragraph{\large \colorbox{skyblue}{Methods}}}
% Attribute environment
\newenvironment{at}
{%begin
\paragraph{\large \colorbox{skyblue}{Attribute}}
\quad\\
\begin{description}
}%
{%end
\end{description}
}
% Operation environment
\newenvironment{op}
{%begin
\paragraph{\large \colorbox{skyblue}{Operations}}
\quad\\
\begin{table}[h]
\begin{center}
\begin{tabular}{|l|l|}
\hline
operator & explanation\\
\hline
}%
{%end
\hline
\end{tabular}
\end{center}
\end{table}
}
% Examples environment
\newenvironment{ex}%
{%begin
\paragraph{\large \colorbox{skyblue}{Examples}}
\VerbatimEnvironment
\renewcommand{\EveryVerbatim}{\fontencoding{OT1}\selectfont}
\begin{quote}
\begin{Verbatim}
}%
{%end
\end{Verbatim}
\end{quote}
}
%
\definecolor{skyblue}{cmyk}{0.2, 0, 0.1, 0}
\definecolor{skyyellow}{cmyk}{0.1, 0.1, 0.5, 0}
%
%\title{NZMATH User Manual\\ {\large{(for version 1.0)}}}
%\date{}
%\author{}
\begin{document}
%\maketitle
%
\setcounter{tocdepth}{3}
\setcounter{secnumdepth}{3}


\tableofcontents
\C

\chapter{Classes}

%---------- start document ---------- %
 \section{factor.misc -- miscellaneous functions related factoring}\linkedzero{factor.misc}
 \begin{itemize}
   \item {\bf Functions}
     \begin{itemize}
       \item \linkingone{factor.misc}{allDivisors}
       \item \linkingone{factor.misc}{primeDivisors}
       \item \linkingone{factor.misc}{primePowerTest}
       \item \linkingone{factor.misc}{squarePart}
     \end{itemize}
   \item {\bf Classes}
   \begin{itemize}
     \item \linkingone{factor.misc}{FactoredInteger}
   \end{itemize}
 \end{itemize}
%
  \subsection{allDivisors -- all divisors}\linkedone{factor.misc}{allDivisors}
   \func{allDivisors}{\hiki{n}{integer}}{\out{list}}\\
   \spacing
   % document of basic document
   \quad Return all factors divide \param{n} as a list.\\
%
  \subsection{primeDivisors -- prime divisors}\linkedone{factor.misc}{primeDivisors}
   \func{primeDivisors}{\hiki{n}{integer}}{\out{list}}\\
   \spacing
   % document of basic document
   \quad Return all prime factors divide \param{n} as a list.\\
%
  \subsection{primePowerTest -- prime power test}\linkedone{factor.misc}{primePowerTest}
   \func{primePowerTest}{\hiki{n}{integer}}{(\out{integer},\ \out{integer})}\\
   \spacing
   \quad Judge whether \param{n} is of the form \(p^k\) with a prime \(p\) or not.
   If it is true, then {\tt (p, k)} will be returned, otherwise {\tt (n, 0)}.\\
   \spacing
   \quad This function is based on Algo. 1.7.5 in \cite{Cohen1}.\\
%
  \subsection{squarePart -- square part}\linkedone{factor.misc}{squarePart}
   \func{squarePart}{\hiki{n}{integer}}{\out{integer}}\\
   \spacing
   \quad Return the largest integer whose square divides \param{n}.
%
\begin{ex}
>>> factor.misc.allDivisors(1001)
[1, 7, 11, 13L, 77, 91L, 143L, 1001L]
>>> factor.misc.primeDivisors(100)
[2, 5]
>>> factor.misc.primePowerTest(128)
(2, 7)
>>> factor.misc.squarePart(128)
8L
\end{ex}%Don't indent!(indent causes an error.)
\C

  \subsection{FactoredInteger -- integer with its factorization}\linkedone{factor.misc}{FactoredInteger}
  \initialize
   \func{FactoredInteger}{\hiki{integer}{integer},\ \hikiopt{factors}{dict}{\{\}}}{\out{FactoredInteger}}\\
   \spacing
   \quad Integer with its factorization information.\\
   \spacing
   \quad If \param{factors} is given, it is a dict of type
   {\tt {prime:exponent}} and the product of \(prime^{exponent}\)
   is equal to the \param{integer}. Otherwise, factorization is
   carried out in initialization.\\

   \func{from\_partial\_factorization}{\param{cls},\ \hiki{integer}{integer},\ \hiki{partial}{dict}}{\out{FactoredInteger}}\\
   \spacing
   \quad A class method to create a new \linkingone{factor.misc}{FactoredInteger} object from partial factorization information \param{partial}.\\
   \begin{op}
     \verb+F * G+ & multiplication (other operand can be an int)\\
     \verb+F ** n+ & powering\\
     \verb+F == G+ & equal\\
     \verb+F != G+ & not equal\\
     \verb+F % G+ & remainder (the result is an int)\\
     \verb+F // G+ & same as \linkingtwo{factor.misc}{FactoredInteger}{exact\_division} method\\
     \verb+str(F)+ & string\\
     \verb+int(F)+ & convert to Python integer (forgetting factorization)\\
   \end{op}
   \method
   \subsubsection{is\_divisible\_by}\linkedtwo{factor.misc}{FactoredInteger}{is\_divisible\_by}
   \func{is\_divisible\_by}{\param{self},\ \hiki{other}{integer/\linkingone{factor.misc}{FactoredInteger}}}{\out{bool}}\\
   \spacing
   \quad Return True if \param{other} divides \param{self}.\\

   \subsubsection{exact\_division}\linkedtwo{factor.misc}{FactoredInteger}{exact\_division}
   \func{exact\_division}{\param{self},\ \hiki{other}{integer/\linkingone{factor.misc}{FactoredInteger}}}{\out{\linkingone{factor.misc}{FactoredInteger}}}\\
   \spacing
   \quad Divide by \param{other}. The \param{other} must divide \param{self}.\\

   \subsubsection{divisors}\linkedtwo{factor.misc}{FactoredInteger}{divisors}
   \func{divisors}{\param{self}}{\out{list}}\\
   \spacing
   \quad Return all divisors as a list.\\

   \subsubsection{proper\_divisors}\linkedtwo{factor.misc}{FactoredInteger}{proper\_divisors}
   \func{proper\_divisors}{\param{self}}{\out{list}}\\
   \spacing
   \quad Return all proper divisors (i.e. divisors excluding \(1\) and
   \param{self}) as a list.\\

   \subsubsection{prime\_divisors}\linkedtwo{factor.misc}{FactoredInteger}{prime\_divisors}
   \func{prime\_divisors}{\param{self}}{\out{list}}\\
   \spacing
   \quad Return all prime divisors as a list.\\

   \subsubsection{square\_part}\linkedtwo{factor.misc}{FactoredInteger}{square\_part}
   \func{square\_part}{\param{self},\ \hikiopt{asfactored}{bool}{False}}{\out{integer/\linkingone{factor.misc}{FactoredInteger object}}}\\
   \spacing
   \quad Return the largest integer whose square divides \param{self}.\\
   \spacing
   \quad If an optional argument \param{asfactored} is true,
   then the result is also a \linkingone{factor.misc}{FactoredInteger object}. (default is False)\\

   \subsubsection{squarefree\_part}\linkedtwo{factor.misc}{FactoredInteger}{squarefree\_part}
   \func{squarefree\_part}{\param{self},\ \hikiopt{asfactored}{bool}{False}}{\out{integer/\linkingone{factor.misc}{FactoredInteger object}}}\\
   \spacing
   \quad Return the largest squarefree integer which divides \param{self}.\\
   \spacing
   \quad If an optional argument \param{asfactored} is true,
   then the result is also a \linkingone{factor.misc}{FactoredInteger object} object. (default is False)\\

   \subsubsection{copy}\linkedtwo{factor.misc}{FactoredInteger}{copy}
   \func{copy}{\param{self}}{\out{\linkingone{factor.misc}{FactoredInteger object}}}\\
   \spacing
   \quad Return a copy of the object.\\

%---------- end document ---------- %

\bibliographystyle{jplain}%use jbibtex
\bibliography{nzmath_references}

\end{document}
%\documentclass{report}

\documentclass{report}

%%%%%%%%%%%%%%%%%%%%%%%%%%%%%%%%%%%%%%%%%%%%%%%%%%%%%%%%%%%%%
%
% macros for nzmath manual
%
%%%%%%%%%%%%%%%%%%%%%%%%%%%%%%%%%%%%%%%%%%%%%%%%%%%%%%%%%%%%%
\usepackage{amssymb,amsmath}
\usepackage{color}
\usepackage[dvipdfm,bookmarks=true,bookmarksnumbered=true,%
 pdftitle={NZMATH Users Manual},%
 pdfsubject={Manual for NZMATH Users},%
 pdfauthor={NZMATH Development Group},%
 pdfkeywords={TeX; dvipdfmx; hyperref; color;},%
 colorlinks=true]{hyperref}
\usepackage{fancybox}
\usepackage[T1]{fontenc}
%
\newcommand{\DS}{\displaystyle}
\newcommand{\C}{\clearpage}
\newcommand{\NO}{\noindent}
\newcommand{\negok}{$\dagger$}
\newcommand{\spacing}{\vspace{1pt}\\ }
% software macros
\newcommand{\nzmathzero}{{\footnotesize $\mathbb{N}\mathbb{Z}$}\texttt{MATH}}
\newcommand{\nzmath}{{\nzmathzero}\ }
\newcommand{\pythonzero}{$\mbox{\texttt{Python}}$}
\newcommand{\python}{{\pythonzero}\ }
% link macros
\newcommand{\linkingzero}[1]{{\bf \hyperlink{#1}{#1}}}%module
\newcommand{\linkingone}[2]{{\bf \hyperlink{#1.#2}{#2}}}%module,class/function etc.
\newcommand{\linkingtwo}[3]{{\bf \hyperlink{#1.#2.#3}{#3}}}%module,class,method
\newcommand{\linkedzero}[1]{\hypertarget{#1}{}}
\newcommand{\linkedone}[2]{\hypertarget{#1.#2}{}}
\newcommand{\linkedtwo}[3]{\hypertarget{#1.#2.#3}{}}
\newcommand{\linktutorial}[1]{\href{http://docs.python.org/tutorial/#1}{#1}}
\newcommand{\linktutorialone}[2]{\href{http://docs.python.org/tutorial/#1}{#2}}
\newcommand{\linklibrary}[1]{\href{http://docs.python.org/library/#1}{#1}}
\newcommand{\linklibraryone}[2]{\href{http://docs.python.org/library/#1}{#2}}
\newcommand{\pythonhp}{\href{http://www.python.org/}{\python website}}
\newcommand{\nzmathwiki}{\href{http://nzmath.sourceforge.net/wiki/}{{\nzmathzero}Wiki}}
\newcommand{\nzmathsf}{\href{http://sourceforge.net/projects/nzmath/}{\nzmath Project Page}}
\newcommand{\nzmathtnt}{\href{http://tnt.math.se.tmu.ac.jp/nzmath/}{\nzmath Project Official Page}}
% parameter name
\newcommand{\param}[1]{{\tt #1}}
% function macros
\newcommand{\hiki}[2]{{\tt #1}:\ {\em #2}}
\newcommand{\hikiopt}[3]{{\tt #1}:\ {\em #2}=#3}

\newdimen\hoge
\newdimen\truetextwidth
\newcommand{\func}[3]{%
\setbox0\hbox{#1(#2)}
\hoge=\wd0
\truetextwidth=\textwidth
\advance \truetextwidth by -2\oddsidemargin
\ifdim\hoge<\truetextwidth % short form
{\bf \colorbox{skyyellow}{#1(#2)\ $\to$ #3}}
%
\else % long form
\fcolorbox{skyyellow}{skyyellow}{%
   \begin{minipage}{\textwidth}%
   {\bf #1(#2)\\ %
    \qquad\quad   $\to$\ #3}%
   \end{minipage}%
   }%
\fi%
}

\newcommand{\out}[1]{{\em #1}}
\newcommand{\initialize}{%
  \paragraph{\large \colorbox{skyblue}{Initialize (Constructor)}}%
    \quad\\ %
    \vspace{3pt}\\
}
\newcommand{\method}{\C \paragraph{\large \colorbox{skyblue}{Methods}}}
% Attribute environment
\newenvironment{at}
{%begin
\paragraph{\large \colorbox{skyblue}{Attribute}}
\quad\\
\begin{description}
}%
{%end
\end{description}
}
% Operation environment
\newenvironment{op}
{%begin
\paragraph{\large \colorbox{skyblue}{Operations}}
\quad\\
\begin{table}[h]
\begin{center}
\begin{tabular}{|l|l|}
\hline
operator & explanation\\
\hline
}%
{%end
\hline
\end{tabular}
\end{center}
\end{table}
}
% Examples environment
\newenvironment{ex}%
{%begin
\paragraph{\large \colorbox{skyblue}{Examples}}
\VerbatimEnvironment
\renewcommand{\EveryVerbatim}{\fontencoding{OT1}\selectfont}
\begin{quote}
\begin{Verbatim}
}%
{%end
\end{Verbatim}
\end{quote}
}
%
\definecolor{skyblue}{cmyk}{0.2, 0, 0.1, 0}
\definecolor{skyyellow}{cmyk}{0.1, 0.1, 0.5, 0}
%
%\title{NZMATH User Manual\\ {\large{(for version 1.0)}}}
%\date{}
%\author{}
\begin{document}
%\maketitle
%
\setcounter{tocdepth}{3}
\setcounter{secnumdepth}{3}


\tableofcontents
\C

\chapter{Functions}


%---------- start document ---------- %
 \section{factor.mpqs -- MPQS}\linkedzero{factor.mpqs}
  \subsection{mpqsfind}\linkedone{factor.mpqs}{mpqsfind}
   \func{mpqsfind}{%
     \hiki{n}{integer},\ %
     \hikiopt{s}{integer}{0},\ %
     \hikiopt{f}{integer}{0},\ %
     \hikiopt{m}{integer}{0},\ %
     \hikiopt{verbose}{bool}{False}
   }{\out{integer}}\\
   \spacing
   % document of basic document
   \quad Find a factor of \param{n} by MPQS(multiple-polynomial
   quadratic sieve) method. \\
   \spacing
   \quad  MPQS is suitable for factorizing a large number.\\
   \spacing
   \quad Optional arguments \param{s} is the range of sieve,
   \param{f} is the number of factor base, and \param{m} is multiplier.
   If these are not specified, the function guesses them from \param{n}.\\
%
  \subsection{mpqs}\linkedone{factor.mpqs}{mpqs}
   \func{mpqs}{%
     \hiki{n}{integer},\ %
     \hikiopt{s}{integer}{0},\ %
     \hikiopt{f}{integer}{0},\ %
     \hikiopt{m}{integer}{0}
   }{\out{\linkingone{factor.methods}{factorlist}}}\\
   \spacing
   \quad Factorize n by MPQS method.\\
   \spacing
   \quad Optional arguments are same as \linkingone{factor.mpqs}{mpqsfind}.
%
  \subsection{eratosthenes}\linkedone{factor.mpqs}{eratosthenes}
   \func{eratosthenes}{\hiki{n}{integer}}{\out{list}}\\
   \spacing
   \quad Enumerate the primes up to \param{n}.\\
\C

%---------- end document ---------- %

\bibliographystyle{jplain}%use jbibtex
\bibliography{nzmath_references}

\end{document}


%\documentclass{report}

%%%%%%%%%%%%%%%%%%%%%%%%%%%%%%%%%%%%%%%%%%%%%%%%%%%%%%%%%%%%%
%
% macros for nzmath manual
%
%%%%%%%%%%%%%%%%%%%%%%%%%%%%%%%%%%%%%%%%%%%%%%%%%%%%%%%%%%%%%
\usepackage{amssymb,amsmath}
\usepackage{color}
\usepackage[dvipdfm,bookmarks=true,bookmarksnumbered=true,%
 pdftitle={NZMATH Users Manual},%
 pdfsubject={Manual for NZMATH Users},%
 pdfauthor={NZMATH Development Group},%
 pdfkeywords={TeX; dvipdfmx; hyperref; color;},%
 colorlinks=true]{hyperref}
\usepackage{fancybox}
\usepackage[T1]{fontenc}
%
\newcommand{\DS}{\displaystyle}
\newcommand{\C}{\clearpage}
\newcommand{\NO}{\noindent}
\newcommand{\negok}{$\dagger$}
\newcommand{\spacing}{\vspace{1pt}\\ }
% software macros
\newcommand{\nzmathzero}{{\footnotesize $\mathbb{N}\mathbb{Z}$}\texttt{MATH}}
\newcommand{\nzmath}{{\nzmathzero}\ }
\newcommand{\pythonzero}{$\mbox{\texttt{Python}}$}
\newcommand{\python}{{\pythonzero}\ }
% link macros
\newcommand{\linkingzero}[1]{{\bf \hyperlink{#1}{#1}}}%module
\newcommand{\linkingone}[2]{{\bf \hyperlink{#1.#2}{#2}}}%module,class/function etc.
\newcommand{\linkingtwo}[3]{{\bf \hyperlink{#1.#2.#3}{#3}}}%module,class,method
\newcommand{\linkedzero}[1]{\hypertarget{#1}{}}
\newcommand{\linkedone}[2]{\hypertarget{#1.#2}{}}
\newcommand{\linkedtwo}[3]{\hypertarget{#1.#2.#3}{}}
\newcommand{\linktutorial}[1]{\href{http://docs.python.org/tutorial/#1}{#1}}
\newcommand{\linktutorialone}[2]{\href{http://docs.python.org/tutorial/#1}{#2}}
\newcommand{\linklibrary}[1]{\href{http://docs.python.org/library/#1}{#1}}
\newcommand{\linklibraryone}[2]{\href{http://docs.python.org/library/#1}{#2}}
\newcommand{\pythonhp}{\href{http://www.python.org/}{\python website}}
\newcommand{\nzmathwiki}{\href{http://nzmath.sourceforge.net/wiki/}{{\nzmathzero}Wiki}}
\newcommand{\nzmathsf}{\href{http://sourceforge.net/projects/nzmath/}{\nzmath Project Page}}
\newcommand{\nzmathtnt}{\href{http://tnt.math.se.tmu.ac.jp/nzmath/}{\nzmath Project Official Page}}
% parameter name
\newcommand{\param}[1]{{\tt #1}}
% function macros
\newcommand{\hiki}[2]{{\tt #1}:\ {\em #2}}
\newcommand{\hikiopt}[3]{{\tt #1}:\ {\em #2}=#3}

\newdimen\hoge
\newdimen\truetextwidth
\newcommand{\func}[3]{%
\setbox0\hbox{#1(#2)}
\hoge=\wd0
\truetextwidth=\textwidth
\advance \truetextwidth by -2\oddsidemargin
\ifdim\hoge<\truetextwidth % short form
{\bf \colorbox{skyyellow}{#1(#2)\ $\to$ #3}}
%
\else % long form
\fcolorbox{skyyellow}{skyyellow}{%
   \begin{minipage}{\textwidth}%
   {\bf #1(#2)\\ %
    \qquad\quad   $\to$\ #3}%
   \end{minipage}%
   }%
\fi%
}

\newcommand{\out}[1]{{\em #1}}
\newcommand{\initialize}{%
  \paragraph{\large \colorbox{skyblue}{Initialize (Constructor)}}%
    \quad\\ %
    \vspace{3pt}\\
}
\newcommand{\method}{\C \paragraph{\large \colorbox{skyblue}{Methods}}}
% Attribute environment
\newenvironment{at}
{%begin
\paragraph{\large \colorbox{skyblue}{Attribute}}
\quad\\
\begin{description}
}%
{%end
\end{description}
}
% Operation environment
\newenvironment{op}
{%begin
\paragraph{\large \colorbox{skyblue}{Operations}}
\quad\\
\begin{table}[h]
\begin{center}
\begin{tabular}{|l|l|}
\hline
operator & explanation\\
\hline
}%
{%end
\hline
\end{tabular}
\end{center}
\end{table}
}
% Examples environment
\newenvironment{ex}%
{%begin
\paragraph{\large \colorbox{skyblue}{Examples}}
\VerbatimEnvironment
\renewcommand{\EveryVerbatim}{\fontencoding{OT1}\selectfont}
\begin{quote}
\begin{Verbatim}
}%
{%end
\end{Verbatim}
\end{quote}
}
%
\definecolor{skyblue}{cmyk}{0.2, 0, 0.1, 0}
\definecolor{skyyellow}{cmyk}{0.1, 0.1, 0.5, 0}
%
%\title{NZMATH User Manual\\ {\large{(for version 1.0)}}}
%\date{}
%\author{}
\begin{document}
%\maketitle
%
\setcounter{tocdepth}{3}
\setcounter{secnumdepth}{3}


\tableofcontents
\C

\chapter{Classes}

%---------- start document ---------- %
 \section{factor.util -- utilities for factorization}\linkedzero{factor.util}
 \begin{itemize}
   \item {\bf Classes}
   \begin{itemize}
     \item \linkingone{factor.util}{FactoringInteger}
     \item \linkingone{factor.util}{FactoringMethod}
   \end{itemize}
 \end{itemize}

 This module uses following type:
 \begin{description}
   \item[factorlist]\linkedone{factor.util}{factorlist}:\\
     \param{factorlist} is a list which consists of pairs {\tt (base, index)}.
     Each pair means \(base^{index}\).
     The product of those terms expresses whole prime factorization.
 \end{description}

\C

 \subsection{FactoringInteger -- keeping track of factorization}\linkedone{factor.util}{FactoringInteger}
 \initialize
  \func{FactoringInteger}{\hiki{number}{integer}}{\out{FactoringInteger}}\\
  \spacing
  % document of basic document
  \quad This is the base class for factoring integers.\\
  \spacing
  % added document
  \quad  \param{number} is stored in the attribute \linkingtwo{factor.util}{FactoringInteger}{number}. The factors will be stored in the attribute \linkingtwo{factor.util}{FactoringInteger}{factors}, and primality of factors will be tracked in the attribute \linkingtwo{factor.util}{FactoringInteger}{primality}.\\
  \spacing
  % input/output document
  \quad The given \param{number} must be a composite number.\\
  \begin{at}
    \item[number]\linkedtwo{factor.util}{FactoringInteger}{number}:\\ The composite number.
    \item[factors]\linkedtwo{factor.util}{FactoringInteger}{factors}:\\ Factors known at the time being referred.
    \item[primality]\linkedtwo{factor.util}{FactoringInteger}{primality}:\\ A dictionary of primality information of known factors.
      {\tt True} if the factor is prime, {\tt False} composite, or {\tt None} undetermined.
  \end{at}
%   \begin{op}
%     \verb+A==B+ & Return whether M and N are equal or not.\\
%   \end{op} 
% \begin{ex}
% >>> A = factor.util.FactoringInteger((1,2))
% >>> print A
% (1, 2)
% >>> A.point
% (1, 2)
% >>>
% \end{ex}%Don't indent!
  \method
  \subsubsection{getNextTarget -- next target}\linkedtwo{factor.util}{FactoringInteger}{getNextTarget}
   \func{getNextTarget}{\param{self},\ \hikiopt{cond}{function}{{\tt None}}}{\out{integer}}\\
   \spacing
   % document of basic document
   \quad Return the next target which meets \param{cond}.\\
   \spacing
   % added document
   If \param{cond} is not specified, then the next target is a composite (or undetermined) factor of \linkingtwo{factor.util}{FactoringInteger}{number}.\\
   \spacing
   % input, output document
   \quad \param{cond} should be a binary predicate whose arguments are base and index.\\
   \quad If there is no target factor, \linklibraryone{exceptions\#exceptions.LookupError}{LookupError} will be raised.\\
%
 \subsubsection{getResult -- result of factorization}\linkedtwo{factor.util}{FactoringInteger}{getResult}
   \func{getResult}{\param{self}}{\out{\linkingone{factor.util}{factors}}}\\
   \spacing
   % document of basic document
   \quad Return the currently known factorization of the \linkingtwo{factor.util}{FactoringInteger}{number}.\\
%
 \subsubsection{register -- register a new factor}\linkedtwo{factor.util}{FactoringInteger}{register}
   \func{register}{\param{self},\ \hiki{divisor}{integer},\ \hikiopt{isprime}{bool}{{\tt None}}}{}\\
   \spacing
   Register a \param{divisor} of the \linkingtwo{factor.util}{FactoringInteger}{number} if the \param{divisor} is a true divisor of the number.\\
   \spacing
   %added document
   \quad The number is divided by the \param{divisor} as many times as possible.\\
   \spacing
   % input/output document
   The optional argument \param{isprime} tells the primality of the
   \param{divisor} (default to undetermined).\\
%
 \subsubsection{sortFactors -- sort factors}\linkedtwo{factor.util}{FactoringInteger}{sortFactors}
   \func{sortFactors}{\param{self}}{}
   \spacing
   \quad Sort factors list.\\
   \spacing
    % added document
    \quad This affects the result of \linkingtwo{factor.util}{FactoringInteger}{getResult}.\\
%
\begin{ex}
>>> A = factor.util.FactoringInteger(100)
>>> A.getNextTarget()
100
>>> A.getResult()
[(100, 1)]
>>> A.register(5, True)
>>> A.getResult()
[(5, 2), (4, 1)]
>>> A.sortFactors()
>>> A.getResult()
[(4, 1), (5, 2)
>>> A.primality
{4: None, 5: True}
>>> A.getNextTarget()
4
\end{ex}%Don't indent!
\C

 \subsection{FactoringMethod -- method of factorization}\linkedone{factor.util}{FactoringMethod}
 \initialize
  \func{FactoringMethod}{}{\out{FactoringMethod}}\\
  \spacing
  % document of basic document
  \quad Base class of factoring methods.\\
  \spacing
  % added document
  \quad All methods defined in \linkingzero{factor.methods} are
  implemented as derived classes of this class. The method which users may call is  \linkingtwo{factor.util}{FactoringMethod}{factor} only. 
  Other methods are explained for future implementers of a new factoring method.\\
  \method
  \subsubsection{factor -- do factorization}\linkedtwo{factor.util}{FactoringMethod}{factor}
   \func{factor}{\param{self},\ %
     \hiki{number}{integer},\ %
     \hikiopt{return\_type}{str}{'list'},\ %
     \hikiopt{need\_sort}{bool}{False}
   }{\out{\linkingone{factor.util}{factorlist}}}\\
   \spacing
   % document of basic document
   \quad Return the factorization of the given positive integer \param{number}.
   \spacing
   % input, output document
   \quad The default returned type is a \linkingone{factor.util}{factorlist}.\\
   \quad A keyword option \param{return\_type} can be as the following:
   \begin{enumerate}
   \item {\tt 'list'} for default type (\linkingone{factor.util}{factorlist}).
   \item {\tt 'tracker'} for \linkingone{factor.util}{FactoringInteger}.
   \end{enumerate}
   \quad Another keyword option \param{need\_sort} is Boolean:
   {\tt True} to sort the result.
   This should be specified with {\tt return\_type='list'}.\\
%
  \subsubsection{\negok continue\_factor -- continue factorization}\linkedtwo{factor.util}{FactoringMethod}{continue\_factor}
  \func{continue\_factor}{\param{self},\ %
    \hiki{tracker}{\linkingone{factor.util}{FactoringInteger}},\ %
    \hikiopt{return\_type}{str}{'tracker'},\ %
    \hikiopt{primeq}{func}{\linkingone{prime}{primeq}}
  }{\out{\linkingone{factor.util}{FactoringInteger}}}\\
  \spacing
  \quad Continue factoring of the given \param{tracker} and return the
  result of factorization.\\
  \spacing
  \quad The default returned type is \linkingone{factor.util}{FactoringInteger},
  but if \param{return\_type} is specified as {\tt 'list'}
  then it returns \linkingone{factor.util}{factorlist}.
  The primality is judged by a function specified in \param{primeq}
  optional keyword argument, which default is \linkingone{prime}{primeq}.\\
%
  \subsubsection{\negok find -- find a factor}\linkedtwo{factor.util}{FactoringMethod}{find}
  \func{find}{\param{self},\ %
    \hiki{target}{integer},\ %
    **\param{options}
  }{\out{integer}}\\
  \spacing
  \quad Find a factor from the \param{target} number.\\
  \spacing
  \quad This method has to be overridden, or \linkingtwo{factor.util}{FactoringMethod}{factor} method should be overridden not to call this method.\\
%
  \subsubsection{\negok generate -- generate prime factors}\linkedtwo{factor.util}{FactoringMethod}{generate}
  \func{generate}{\param{self},\ %
    \hiki{target}{integer},\ %
    **\param{options}
  }{\out{integer}}\\
  \spacing
 % basic document
  \quad Generate prime factors of the \param{target} number with their valuations.\\
 \spacing
 % added document
  \quad The method may terminate with yielding {\tt (1, 1)}
  to indicate the factorization is incomplete.\\
  This method has to be overridden, or \linkingtwo{factor.util}{FactoringMethod}{factor} method should be overridden not to call this method.\\
\C

%---------- end document ---------- %

\bibliographystyle{jplain}
\bibliography{nzmath_references}

\end{document}

%\documentclass{report}

%%%%%%%%%%%%%%%%%%%%%%%%%%%%%%%%%%%%%%%%%%%%%%%%%%%%%%%%%%%%%
%
% macros for nzmath manual
%
%%%%%%%%%%%%%%%%%%%%%%%%%%%%%%%%%%%%%%%%%%%%%%%%%%%%%%%%%%%%%
\usepackage{amssymb,amsmath}
\usepackage{color}
\usepackage[dvipdfm,bookmarks=true,bookmarksnumbered=true,%
 pdftitle={NZMATH Users Manual},%
 pdfsubject={Manual for NZMATH Users},%
 pdfauthor={NZMATH Development Group},%
 pdfkeywords={TeX; dvipdfmx; hyperref; color;},%
 colorlinks=true]{hyperref}
\usepackage{fancybox}
\usepackage[T1]{fontenc}
%
\newcommand{\DS}{\displaystyle}
\newcommand{\C}{\clearpage}
\newcommand{\NO}{\noindent}
\newcommand{\negok}{$\dagger$}
\newcommand{\spacing}{\vspace{1pt}\\ }
% software macros
\newcommand{\nzmathzero}{{\footnotesize $\mathbb{N}\mathbb{Z}$}\texttt{MATH}}
\newcommand{\nzmath}{{\nzmathzero}\ }
\newcommand{\pythonzero}{$\mbox{\texttt{Python}}$}
\newcommand{\python}{{\pythonzero}\ }
% link macros
\newcommand{\linkingzero}[1]{{\bf \hyperlink{#1}{#1}}}%module
\newcommand{\linkingone}[2]{{\bf \hyperlink{#1.#2}{#2}}}%module,class/function etc.
\newcommand{\linkingtwo}[3]{{\bf \hyperlink{#1.#2.#3}{#3}}}%module,class,method
\newcommand{\linkedzero}[1]{\hypertarget{#1}{}}
\newcommand{\linkedone}[2]{\hypertarget{#1.#2}{}}
\newcommand{\linkedtwo}[3]{\hypertarget{#1.#2.#3}{}}
\newcommand{\linktutorial}[1]{\href{http://docs.python.org/tutorial/#1}{#1}}
\newcommand{\linktutorialone}[2]{\href{http://docs.python.org/tutorial/#1}{#2}}
\newcommand{\linklibrary}[1]{\href{http://docs.python.org/library/#1}{#1}}
\newcommand{\linklibraryone}[2]{\href{http://docs.python.org/library/#1}{#2}}
\newcommand{\pythonhp}{\href{http://www.python.org/}{\python website}}
\newcommand{\nzmathwiki}{\href{http://nzmath.sourceforge.net/wiki/}{{\nzmathzero}Wiki}}
\newcommand{\nzmathsf}{\href{http://sourceforge.net/projects/nzmath/}{\nzmath Project Page}}
\newcommand{\nzmathtnt}{\href{http://tnt.math.se.tmu.ac.jp/nzmath/}{\nzmath Project Official Page}}
% parameter name
\newcommand{\param}[1]{{\tt #1}}
% function macros
\newcommand{\hiki}[2]{{\tt #1}:\ {\em #2}}
\newcommand{\hikiopt}[3]{{\tt #1}:\ {\em #2}=#3}

\newdimen\hoge
\newdimen\truetextwidth
\newcommand{\func}[3]{%
\setbox0\hbox{#1(#2)}
\hoge=\wd0
\truetextwidth=\textwidth
\advance \truetextwidth by -2\oddsidemargin
\ifdim\hoge<\truetextwidth % short form
{\bf \colorbox{skyyellow}{#1(#2)\ $\to$ #3}}
%
\else % long form
\fcolorbox{skyyellow}{skyyellow}{%
   \begin{minipage}{\textwidth}%
   {\bf #1(#2)\\ %
    \qquad\quad   $\to$\ #3}%
   \end{minipage}%
   }%
\fi%
}

\newcommand{\out}[1]{{\em #1}}
\newcommand{\initialize}{%
  \paragraph{\large \colorbox{skyblue}{Initialize (Constructor)}}%
    \quad\\ %
    \vspace{3pt}\\
}
\newcommand{\method}{\C \paragraph{\large \colorbox{skyblue}{Methods}}}
% Attribute environment
\newenvironment{at}
{%begin
\paragraph{\large \colorbox{skyblue}{Attribute}}
\quad\\
\begin{description}
}%
{%end
\end{description}
}
% Operation environment
\newenvironment{op}
{%begin
\paragraph{\large \colorbox{skyblue}{Operations}}
\quad\\
\begin{table}[h]
\begin{center}
\begin{tabular}{|l|l|}
\hline
operator & explanation\\
\hline
}%
{%end
\hline
\end{tabular}
\end{center}
\end{table}
}
% Examples environment
\newenvironment{ex}%
{%begin
\paragraph{\large \colorbox{skyblue}{Examples}}
\VerbatimEnvironment
\renewcommand{\EveryVerbatim}{\fontencoding{OT1}\selectfont}
\begin{quote}
\begin{Verbatim}
}%
{%end
\end{Verbatim}
\end{quote}
}
%
\definecolor{skyblue}{cmyk}{0.2, 0, 0.1, 0}
\definecolor{skyyellow}{cmyk}{0.1, 0.1, 0.5, 0}
%
%\title{NZMATH User Manual\\ {\large{(for version 1.0)}}}
%\date{}
%\author{}
\begin{document}
%\maketitle
%
\setcounter{tocdepth}{3}
\setcounter{secnumdepth}{3}


\tableofcontents
\C

\chapter{Functions}


%---------- start document ---------- %
 \section{poly.factor -- polynomial factorization}\linkedzero{poly.factor}
 The factor module is for factorizations of integer coefficient univariate polynomials.


 This module using following type:
 \begin{description}
   \item[polynomial]\linkedone{poly.factor}{polynomial}:\\
     \param{polynomial} is the polynomial generated by function poly.uniutil.polynomial. 
 \end{description}

%
  \subsection{brute\_force\_search -- search factorization by brute force}\linkedone{poly.factor}{brute\_force\_search}
   \func{brute\_force\_search}
   {%
     \hiki{f}{poly.uniutil.IntegerPolynomial},\ %
     \hiki{fp\_factors}{list},\ %
     \hiki{q}{integer}%
   }{%
     \out{[factors]}%
   }\\
   \spacing
   % document of basic document
   \quad Find the factorization of \param{f} by searching a factor which is a product of some combination in \param{fp\_factors}. The combination is searched by brute force.
   \spacing
   % added document
   The argument \param{fp\_factors} is a list of poly.uniutil.FinitePrimeFieldPolynomial .
   \spacing
   % input, output document
   %\quad \\
%
  \subsection{divisibility\_test -- divisibility test}\linkedone{poly.factor}{divisibility\_test}
   \func{divisibility\_test}
        {\hiki{f}{polynomial},\ %
         \hiki{g}{polynomial}%
        }
        {\out{bool}}\\
   \spacing
   % document of basic document
   \quad Return Boolean value whether \param{f} is divisible by \param{g} or not, for polynomials.
   \spacing
   % added document
   %\quad \negok Note that this function returns Hilbert class polynomial as a list of coefficients.\\
   %\spacing
   % input, output document
   %\quad \param{D} must be negative int or long. See \cite{Pomerance}.\\
%
  \subsection{minimum\_absolute\_injection -- send coefficients to minimum absolute representation }\linkedone{poly.factor}{minimum\_absolute\_injection}
   \func{minimum\_absolute\_injection}
        {\hiki{f}{polynomial}}
        {\out{F}}\\
   \spacing
   % document of basic document
   \quad Return an integer coefficient polynomial F by injection of a $\mathbf{Z}/p\mathbf{Z}$ coefficient polynomial \param{f} with sending each coefficient to minimum absolute representatives.
   \spacing
   % added document
   %\quad \negok 
   %\spacing
   % input, output document
   \quad The coefficient ring of given polynomial \param{f} must be \linkingone{intresidue}{IntegerResidueClassRing} or \linkingone{finitefield}{FinitePrimeField}.\\
%
  \subsection{padic\_factorization -- p-adic factorization}\linkedone{poly.factor}{padic\_factorization}
   \func{padic\_factorization}
        {\hiki{f}{polynomial}}
        {\out{p}, \out{factors}}\\
   \spacing
   % document of basic document
   \quad Return a prime \param{p} and a p-adic factorization of given integer coefficient squarefree polynomial \param{f}. The result \param{factors} have integer coefficients, injected from $\mathbb{F}_p$ to its minimum absolute representation.
   \spacing
   % added document
   \quad \negok The prime is chosen to be:
   \begin{enumerate}
   \item \param{f} is still squarefree mod \param{p},
   \item the number of factors is not greater than with the successive prime.
   \end{enumerate}
   %\spacing
   % input, output document
   \quad The given polynomial \param{f} must be poly.uniutil.IntegerPolynomial .\\
%
  \subsection{upper\_bound\_of\_coefficient --Landau-Mignotte bound of coefficients}\linkedone{poly.factor}{upper\_bound\_of\_coefficient}
   \func{upper\_bound\_of\_coefficient}
        {\hiki{f}{polynomial}}
        {\out{long}}\\
   \spacing
   % document of basic document
   \quad Compute Landau-Mignotte bound of coefficients of factors, whose degree is no greater than half of the given \param{f}.
   \spacing
   % added document
   %\quad \negok Additional argument \param{floatpre} specifies the precision of calculation in decimal digits.
   %\spacing
   % input, output document
   \quad The given polynomial \param{f} must have integer coefficients.\\
%
  \subsection{zassenhaus -- squarefree integer polynomial factorization by Zassenhaus method}\linkedone{poly.factor}{zassenhaus}
   \func{zassenhaus}
        {\hiki{f}{polynomial}}
        {\out{list of factors f}}\\
   \spacing
   % document of basic document
   \quad Factor a squarefree integer coefficient polynomial \param{f} with Berlekamp-Zassenhaus method.
   \spacing
   % added document
   %\quad 
   %\spacing
   % input, output document
   %\quad output must be list of factors.
%
  \subsection{integerpolynomialfactorization -- Integer polynomial factorization}\linkedone{poly.factor}{integerpolynomialfactorization}
   \func{integerpolynomialfactorization}
        {\hiki{f}{polynomial}}
        {\out{factor}}\\
   \spacing
   % document of basic document
   \quad Factor an integer coefficient polynomial \param{f} with Berlekamp-Zassenhaus method.
   \spacing
   % added document
   %\quad \param{p} must be a prime integer and \param{d} be an integer such that 0 < \param{d} < 4\param{p} with $-\mathtt{d} \equiv 0, 1 \pmod{4}$. 
   \spacing
   % input, output document
   \quad factor output by the form of list of tuples that formed (factor, index). \\
%
%\begin{ex}
%>>> module.func1(1, 0.1, "a", [], (1, 2))
%(2, "b")
%>>> module.func2()
%1
%\end{ex}%Don't indent!(indent causes an error.)
\C

%---------- end document ---------- %

\bibliographystyle{jplain}%use jbibtex
\bibliography{nzmath_references}

\end{document}


%\documentclass{report}

%%%%%%%%%%%%%%%%%%%%%%%%%%%%%%%%%%%%%%%%%%%%%%%%%%%%%%%%%%%%%
%
% macros for nzmath manual
%
%%%%%%%%%%%%%%%%%%%%%%%%%%%%%%%%%%%%%%%%%%%%%%%%%%%%%%%%%%%%%
\usepackage{amssymb,amsmath}
\usepackage{color}
\usepackage[dvipdfm,bookmarks=true,bookmarksnumbered=true,%
 pdftitle={NZMATH Users Manual},%
 pdfsubject={Manual for NZMATH Users},%
 pdfauthor={NZMATH Development Group},%
 pdfkeywords={TeX; dvipdfmx; hyperref; color;},%
 colorlinks=true]{hyperref}
\usepackage{fancybox}
\usepackage[T1]{fontenc}
%
\newcommand{\DS}{\displaystyle}
\newcommand{\C}{\clearpage}
\newcommand{\NO}{\noindent}
\newcommand{\negok}{$\dagger$}
\newcommand{\spacing}{\vspace{1pt}\\ }
% software macros
\newcommand{\nzmathzero}{{\footnotesize $\mathbb{N}\mathbb{Z}$}\texttt{MATH}}
\newcommand{\nzmath}{{\nzmathzero}\ }
\newcommand{\pythonzero}{$\mbox{\texttt{Python}}$}
\newcommand{\python}{{\pythonzero}\ }
% link macros
\newcommand{\linkingzero}[1]{{\bf \hyperlink{#1}{#1}}}%module
\newcommand{\linkingone}[2]{{\bf \hyperlink{#1.#2}{#2}}}%module,class/function etc.
\newcommand{\linkingtwo}[3]{{\bf \hyperlink{#1.#2.#3}{#3}}}%module,class,method
\newcommand{\linkedzero}[1]{\hypertarget{#1}{}}
\newcommand{\linkedone}[2]{\hypertarget{#1.#2}{}}
\newcommand{\linkedtwo}[3]{\hypertarget{#1.#2.#3}{}}
\newcommand{\linktutorial}[1]{\href{http://docs.python.org/tutorial/#1}{#1}}
\newcommand{\linktutorialone}[2]{\href{http://docs.python.org/tutorial/#1}{#2}}
\newcommand{\linklibrary}[1]{\href{http://docs.python.org/library/#1}{#1}}
\newcommand{\linklibraryone}[2]{\href{http://docs.python.org/library/#1}{#2}}
\newcommand{\pythonhp}{\href{http://www.python.org/}{\python website}}
\newcommand{\nzmathwiki}{\href{http://nzmath.sourceforge.net/wiki/}{{\nzmathzero}Wiki}}
\newcommand{\nzmathsf}{\href{http://sourceforge.net/projects/nzmath/}{\nzmath Project Page}}
\newcommand{\nzmathtnt}{\href{http://tnt.math.se.tmu.ac.jp/nzmath/}{\nzmath Project Official Page}}
% parameter name
\newcommand{\param}[1]{{\tt #1}}
% function macros
\newcommand{\hiki}[2]{{\tt #1}:\ {\em #2}}
\newcommand{\hikiopt}[3]{{\tt #1}:\ {\em #2}=#3}

\newdimen\hoge
\newdimen\truetextwidth
\newcommand{\func}[3]{%
\setbox0\hbox{#1(#2)}
\hoge=\wd0
\truetextwidth=\textwidth
\advance \truetextwidth by -2\oddsidemargin
\ifdim\hoge<\truetextwidth % short form
{\bf \colorbox{skyyellow}{#1(#2)\ $\to$ #3}}
%
\else % long form
\fcolorbox{skyyellow}{skyyellow}{%
   \begin{minipage}{\textwidth}%
   {\bf #1(#2)\\ %
    \qquad\quad   $\to$\ #3}%
   \end{minipage}%
   }%
\fi%
}

\newcommand{\out}[1]{{\em #1}}
\newcommand{\initialize}{%
  \paragraph{\large \colorbox{skyblue}{Initialize (Constructor)}}%
    \quad\\ %
    \vspace{3pt}\\
}
\newcommand{\method}{\C \paragraph{\large \colorbox{skyblue}{Methods}}}
% Attribute environment
\newenvironment{at}
{%begin
\paragraph{\large \colorbox{skyblue}{Attribute}}
\quad\\
\begin{description}
}%
{%end
\end{description}
}
% Operation environment
\newenvironment{op}
{%begin
\paragraph{\large \colorbox{skyblue}{Operations}}
\quad\\
\begin{table}[h]
\begin{center}
\begin{tabular}{|l|l|}
\hline
operator & explanation\\
\hline
}%
{%end
\hline
\end{tabular}
\end{center}
\end{table}
}
% Examples environment
\newenvironment{ex}%
{%begin
\paragraph{\large \colorbox{skyblue}{Examples}}
\VerbatimEnvironment
\renewcommand{\EveryVerbatim}{\fontencoding{OT1}\selectfont}
\begin{quote}
\begin{Verbatim}
}%
{%end
\end{Verbatim}
\end{quote}
}
%
\definecolor{skyblue}{cmyk}{0.2, 0, 0.1, 0}
\definecolor{skyyellow}{cmyk}{0.1, 0.1, 0.5, 0}
%
%\title{NZMATH User Manual\\ {\large{(for version 1.0)}}}
%\date{}
%\author{}
\begin{document}
%\maketitle
%
\setcounter{tocdepth}{3}
\setcounter{secnumdepth}{3}


\tableofcontents
\C

\chapter{Classes}


%---------- start document ---------- %
 \section{poly.formalsum -- formal sum}\linkedzero{poly.formalsum}
 \begin{itemize}
   \item {\bf Classes}
   \begin{itemize}
     \item \negok\linkingone{poly.formalsum}{FormalSumContainerInterface}
     \item \linkingone{poly.formalsum}{DictFormalSum}
     \item \negok \linkingone{poly.formalsum}{ListFormalSum}
   \end{itemize}
 \end{itemize}

 The formal sum is mathematically a finite sum of terms,
 A term consists of two parts: coefficient and base.
 All coefficients in a formal sum are in a common ring,
 while bases are arbitrary.

 Two formal sums can be added in the following way.
 If there are terms with common base, they are fused into a new
 term with the same base and coefficients added.

 A coefficient can be looked up from the base. If the specified base
 does not appear in the formal sum, it is null.

 We refer the following for convenience as {\tt terminit}:
 \begin{description}
   \item[terminit]\linkedone{poly.formalsum}{terminit}:\\
     \param{terminit} means one of types to initialize
     \linklibraryone{stdtypes\#dict}{dict}.  The dictionary
     constructed from it will be considered as a mapping from bases to
     coefficients.
 \end{description}

\paragraph{Note for beginner}
You may need USE only \linkingone{poly.formalsum}{DictFormalSum},
but may have to READ the description of
\linkingone{poly.formalsum}{FormalSumContainerInterface} because
interface (all method names and their semantics) is defined in it.


\C
%
 \subsection{FormalSumContainerInterface -- interface class}\linkedone{poly.formalsum}{FormalSumContainerInterface}
  \initialize
  Since the interface is an abstract class, do not instantiate.\\
  \spacing
  % document of basic document
  \quad The interface defines what ``formal sum'' is.
  Derived classes must provide the following operations and methods.
  \begin{op}
    \verb/f + g/ & addition\\
    \verb/f - g/ & subtraction\\
    \verb/-f/ & negation\\
    \verb/+f/ & new copy\\
    \verb/f * a, a * f/ & multiplication by scalar {\tt a}\\
    \verb/f == g/ & equality\\
    \verb/f != g/ & inequality\\
    \verb/f[b]/	& get coefficient corresponding to a base {\tt b}\\
    \verb/b in f/ & return whether base {\tt b} is in {\tt f}\\
    \verb/len(f)/ & number of terms\\
    \verb/hash(f)/ & hash\\
  \end{op}
  \method
  \subsubsection{construct\_with\_default -- copy-constructing}\linkedtwo{poly.formalsum}{FormalSumContainerInterface}{construct\_with\_default}
   \func{construct\_with\_default}{\param{self},\ %
   \hiki{maindata}{terminit}}{\out{FormalSumContainerInterface}}\\
   \spacing
   % document of basic document
   \quad Create a new formal sum of the same class with \param{self},
   with given only the \param{maindata} and use copy of \param{self}'s
   data if necessary.
   \spacing
%
  \subsubsection{iterterms -- iterator of terms}\linkedtwo{poly.formalsum}{FormalSumContainerInterface}{iterterms}
   \func{iterterms}{\param{self}}{\out{iterator}}\\
   \spacing
   % document of basic document
   \quad Return an iterator of the terms.
   \spacing
   % input, output document
   \quad Each term yielded from iterators is a {\tt (base, coefficient)} pair.\\
 \subsubsection{itercoefficients -- iterator of coefficients}\linkedtwo{poly.formalsum}{FormalSumContainerInterface}{itercoefficients}
   \func{itercoefficients}{\param{self}}{\out{iterator}}\\
   \spacing
   % document of basic document
   \quad Return an iterator of the coefficients.\\
 \subsubsection{iterbases -- iterator of bases}\linkedtwo{poly.formalsum}{FormalSumContainerInterface}{iterbases}
   \func{iterbases}{\param{self}}{\out{iterator}}\\
   \spacing
   % document of basic document
   \quad Return an iterator of the bases.\\
  \subsubsection{terms -- list of terms}\linkedtwo{poly.formalsum}{FormalSumContainerInterface}{terms}
   \func{terms}{\param{self}}{\out{list}}\\
   \spacing
   % document of basic document
   \quad Return a list of the terms.
   \spacing
   % input, output document
   \quad Each term in returned lists is a {\tt (base, coefficient)} pair.\\
 \subsubsection{coefficients -- list of coefficients}\linkedtwo{poly.formalsum}{FormalSumContainerInterface}{coefficients}
   \func{coefficients}{\param{self}}{\out{list}}\\
   \spacing
   % document of basic document
   \quad Return a list of the coefficients.\\
 \subsubsection{bases -- list of bases}\linkedtwo{poly.formalsum}{FormalSumContainerInterface}{bases}
   \func{bases}{\param{self}}{\out{list}}\\
   \spacing
   % document of basic document
   \quad Return a list of the bases.\\
  \subsubsection{terms\_map -- list of terms}\linkedtwo{poly.formalsum}{FormalSumContainerInterface}{terms\_map}
   \func{terms\_map}{\param{self},\ %
   \hiki{func}{function}}{\out{FormalSumContainerInterface}}\\
   \spacing
   % document of basic document
   \quad Map on terms, i.e., create a new formal sum by applying \param{func}
   to each term.
   \spacing
   % input, output document
   \quad \param{func} has to accept two parameters {\tt base} and
   {\tt coefficient}, then return a new term pair.
 \subsubsection{coefficients\_map -- list of coefficients}\linkedtwo{poly.formalsum}{FormalSumContainerInterface}{coefficients\_map}
   \func{coefficients\_map}{\param{self}}{\out{FormalSumContainerInterface}}\\
   \spacing
   % document of basic document
   \quad Map on coefficients, i.e., create a new formal sum by applying
   \param{func} to each coefficient.\\
   \spacing
   % input, output document
   \quad \param{func} has to accept one parameters {\tt coefficient},
   then return a new coefficient.
 \subsubsection{bases\_map -- list of bases}\linkedtwo{poly.formalsum}{FormalSumContainerInterface}{bases\_map}
   \func{bases\_map}{\param{self}}{\out{FormalSumContainerInterface}}\\
   \spacing
   % document of basic document
   \quad Map on bases, i.e., create a new formal sum by applying
   \param{func} to each base.\\
   \spacing
   % input, output document
   \quad \param{func} has to accept one parameters {\tt base},
   then return a new base.

\C
%
 \subsection{DictFormalSum -- formal sum implemented with dictionary}\linkedone{poly.formalsum}{DictFormalSum}
  % document of basic document
  A formal sum implementation based on {\tt dict}.\\
  \spacing
  % added document
  \quad This class inherits \linkingone{poly.formalsum}{FormalSumContainerInterface}.
  All methods of the interface are implemented.
 \initialize
  \func{DictFormalSum}{\hiki{args}{terminit},\ %
  \hikiopt{defaultvalue}{RingElement}{None}}{\out{DictFormalSum}}\\
  \spacing
  \quad See \linkingone{poly.formalsum}{terminit} for type of \param{args}.
  It makes a mapping from bases to coefficients.\\
  \quad The optional argument \param{defaultvalue} is the default value for
  {\tt \_\_getitem\_\_}, i.e., if there is no term with the specified
  base, a look up attempt returns the \param{defaultvalue}.  It is,
  thus, an element of the ring to which other coefficients belong.

 \subsection{ListFormalSum -- formal sum implemented with list}\linkedone{poly.formalsum}{ListFormalSum}
  % document of basic document
   A formal sum implementation based on list.\\
  \spacing
  % added document
  \quad This class inherits \linkingone{poly.formalsum}{FormalSumContainerInterface}.
  All methods of the interface are implemented.
 \initialize
  \func{ListFormalSum}{\hiki{args}{terminit},\ %
  \hikiopt{defaultvalue}{RingElement}{None}}{\out{ListFormalSum}}\\
  \spacing
  \quad See \linkingone{poly.formalsum}{terminit} for type of \param{args}.
  It makes a mapping from bases to coefficients.\\
  \quad The optional argument \param{defaultvalue} is the default value for
  {\tt \_\_getitem\_\_}, i.e., if there is no term with the specified
  base, a look up attempt returns the \param{defaultvalue}.  It is,
  thus, an element of the ring to which other coefficients belong.
\C
%---------- end document ---------- %

\bibliographystyle{jplain}%use jbibtex
\bibliography{nzmath_references}

\end{document}


%\documentclass{report}

%%%%%%%%%%%%%%%%%%%%%%%%%%%%%%%%%%%%%%%%%%%%%%%%%%%%%%%%%%%%%
%
% macros for nzmath manual
%
%%%%%%%%%%%%%%%%%%%%%%%%%%%%%%%%%%%%%%%%%%%%%%%%%%%%%%%%%%%%%
\usepackage{amssymb,amsmath}
\usepackage{color}
\usepackage[dvipdfm,bookmarks=true,bookmarksnumbered=true,%
 pdftitle={NZMATH Users Manual},%
 pdfsubject={Manual for NZMATH Users},%
 pdfauthor={NZMATH Development Group},%
 pdfkeywords={TeX; dvipdfmx; hyperref; color;},%
 colorlinks=true]{hyperref}
\usepackage{fancybox}
\usepackage[T1]{fontenc}
%
\newcommand{\DS}{\displaystyle}
\newcommand{\C}{\clearpage}
\newcommand{\NO}{\noindent}
\newcommand{\negok}{$\dagger$}
\newcommand{\spacing}{\vspace{1pt}\\ }
% software macros
\newcommand{\nzmathzero}{{\footnotesize $\mathbb{N}\mathbb{Z}$}\texttt{MATH}}
\newcommand{\nzmath}{{\nzmathzero}\ }
\newcommand{\pythonzero}{$\mbox{\texttt{Python}}$}
\newcommand{\python}{{\pythonzero}\ }
% link macros
\newcommand{\linkingzero}[1]{{\bf \hyperlink{#1}{#1}}}%module
\newcommand{\linkingone}[2]{{\bf \hyperlink{#1.#2}{#2}}}%module,class/function etc.
\newcommand{\linkingtwo}[3]{{\bf \hyperlink{#1.#2.#3}{#3}}}%module,class,method
\newcommand{\linkedzero}[1]{\hypertarget{#1}{}}
\newcommand{\linkedone}[2]{\hypertarget{#1.#2}{}}
\newcommand{\linkedtwo}[3]{\hypertarget{#1.#2.#3}{}}
\newcommand{\linktutorial}[1]{\href{http://docs.python.org/tutorial/#1}{#1}}
\newcommand{\linktutorialone}[2]{\href{http://docs.python.org/tutorial/#1}{#2}}
\newcommand{\linklibrary}[1]{\href{http://docs.python.org/library/#1}{#1}}
\newcommand{\linklibraryone}[2]{\href{http://docs.python.org/library/#1}{#2}}
\newcommand{\pythonhp}{\href{http://www.python.org/}{\python website}}
\newcommand{\nzmathwiki}{\href{http://nzmath.sourceforge.net/wiki/}{{\nzmathzero}Wiki}}
\newcommand{\nzmathsf}{\href{http://sourceforge.net/projects/nzmath/}{\nzmath Project Page}}
\newcommand{\nzmathtnt}{\href{http://tnt.math.se.tmu.ac.jp/nzmath/}{\nzmath Project Official Page}}
% parameter name
\newcommand{\param}[1]{{\tt #1}}
% function macros
\newcommand{\hiki}[2]{{\tt #1}:\ {\em #2}}
\newcommand{\hikiopt}[3]{{\tt #1}:\ {\em #2}=#3}

\newdimen\hoge
\newdimen\truetextwidth
\newcommand{\func}[3]{%
\setbox0\hbox{#1(#2)}
\hoge=\wd0
\truetextwidth=\textwidth
\advance \truetextwidth by -2\oddsidemargin
\ifdim\hoge<\truetextwidth % short form
{\bf \colorbox{skyyellow}{#1(#2)\ $\to$ #3}}
%
\else % long form
\fcolorbox{skyyellow}{skyyellow}{%
   \begin{minipage}{\textwidth}%
   {\bf #1(#2)\\ %
    \qquad\quad   $\to$\ #3}%
   \end{minipage}%
   }%
\fi%
}

\newcommand{\out}[1]{{\em #1}}
\newcommand{\initialize}{%
  \paragraph{\large \colorbox{skyblue}{Initialize (Constructor)}}%
    \quad\\ %
    \vspace{3pt}\\
}
\newcommand{\method}{\C \paragraph{\large \colorbox{skyblue}{Methods}}}
% Attribute environment
\newenvironment{at}
{%begin
\paragraph{\large \colorbox{skyblue}{Attribute}}
\quad\\
\begin{description}
}%
{%end
\end{description}
}
% Operation environment
\newenvironment{op}
{%begin
\paragraph{\large \colorbox{skyblue}{Operations}}
\quad\\
\begin{table}[h]
\begin{center}
\begin{tabular}{|l|l|}
\hline
operator & explanation\\
\hline
}%
{%end
\hline
\end{tabular}
\end{center}
\end{table}
}
% Examples environment
\newenvironment{ex}%
{%begin
\paragraph{\large \colorbox{skyblue}{Examples}}
\VerbatimEnvironment
\renewcommand{\EveryVerbatim}{\fontencoding{OT1}\selectfont}
\begin{quote}
\begin{Verbatim}
}%
{%end
\end{Verbatim}
\end{quote}
}
%
\definecolor{skyblue}{cmyk}{0.2, 0, 0.1, 0}
\definecolor{skyyellow}{cmyk}{0.1, 0.1, 0.5, 0}
%
%\title{NZMATH User Manual\\ {\large{(for version 1.0)}}}
%\date{}
%\author{}
\begin{document}
%\maketitle
%
\setcounter{tocdepth}{3}
\setcounter{secnumdepth}{3}


\tableofcontents
\C

\chapter{Classes}


%---------- start document ---------- %
 \section{poly.hensel -- Hensel lift}\linkedzero{poly.hensel}
 \begin{itemize}
   \item {\bf Classes}
   \begin{itemize}
     \item \negok \linkingone{poly.hensel}{HenselLiftPair}
     \item \negok \linkingone{poly.hensel}{HenselLiftMulti}
     \item \negok \linkingone{poly.hensel}{HenselLiftSimultaneously}
   \end{itemize}
   \item {\bf Functions}
     \begin{itemize}
       \item \linkingone{poly.hensel}{lift\_upto}
     \end{itemize}
 \end{itemize}

 In this module document, {\em polynomial} means integer polynomial.
\C

 \subsection{HenselLiftPair -- Hensel lift for a pair}\linkedone{poly.hensel}{HenselLiftPair}
 \initialize
  \func{HenselLiftPair}{%
    \hiki{f}{polynomial},
    \hiki{a1}{polynomial},
    \hiki{a2}{polynomial},
    \hiki{u1}{polynomial},
    \hiki{u2}{polynomial},
    \hiki{p}{integer},
    \hikiopt{q}{integer}{p}}{\out{HenselLiftPair}}\\
  \spacing
  % document of basic document
  \quad This object keeps integer polynomial pair which will be lifted by Hensel's lemma.
  % added document
  %
  \spacing
  % input, output document
  \quad The argument should satisfy the following preconditions:
  \begin{itemize}
  \item \param{f}, \param{a1} and \param{a2} are monic
  \item {\tt \param{f} == \param{a1}*\param{a2} (mod \param{q})}
  \item {\tt \param{a1}*\param{u1} + \param{a2}*\param{u2} == 1 (mod \param{p})}
  \item \param{p} divides \param{q} and both are positive
  \end{itemize}
  \func{from\_factors}{%
    \hiki{f}{polynomial},
    \hiki{a1}{polynomial},
    \hiki{a2}{polynomial},
    \hiki{p}{integer}}{\out{HenselLiftPair}}\\
  \spacing
  \quad This is a class method to create and return an instance of {\tt HenselLiftPair}.
  You do not have to precompute {\tt u1} and {\tt u2} for the default constructor; they will be prepared for you from other arguments.\\
  \spacing
  % input, output document
  \quad The argument should satisfy the following preconditions:
  \begin{itemize}
  \item \param{f}, \param{a1} and \param{a2} are monic
  \item {\tt \param{f} == \param{a1}*\param{a2} (mod \param{p})}
  \item \param{p} is prime
  \end{itemize}
  \begin{at}
    \item[point]\linkedtwo{poly.hensel}{HenselLiftPair}{factors}:\\
      factors {\tt a1} and {\tt a2} as a list.
  \end{at}
  \method
  \subsubsection{lift -- lift one step}\linkedtwo{poly.hensel}{HenselLiftPair}{lift}
  \func{lift}{\param{self}}{}\\
  \spacing
  \quad Lift polynomials by so-called the quadratic method.
  \subsubsection{lift\_factors -- lift {\tt a1} and {\tt a2}}\linkedtwo{poly.hensel}{HenselLiftPair}{lift\_factors}
   \func{lift\_factors}{\param{self}}{}\\
   \spacing
   % document of basic document
   \quad Update factors by lifted integer coefficient polynomials {\tt Ai}'s:
   \begin{itemize}
   \item {\tt f == A1 * A2 (mod p * q)}
   \item {\tt Ai == ai (mod q)} \((i = 1, 2)\)
   \end{itemize}
   Moreover, {\tt q} is updated to {\tt p * q}.
   \spacing
   % added document
   \quad \negok The preconditions which should be automatically satisfied:
   \begin{itemize}
   \item {\tt f == a1*a2 (mod q)}
   \item {\tt a1*u1 + a2*u2 == 1 (mod p)}
   \item {\tt p} divides {\tt q}
   \end{itemize}
   \subsubsection{lift\_ladder -- lift {\tt u1} and {\tt u2}}\linkedtwo{poly.hensel}{HenselLiftPair}{lift\_ladder}
   \func{lift\_ladder}{\param{self}}{}\\
   \spacing
   % document of basic document
   \quad Update {\tt u1} and {\tt u2} with {\tt U1} and {\tt U2}:
   \begin{itemize}
   \item {\tt a1*U1 + a2*U2 == 1 (mod p**2)}
   \item {\tt Ui == ui (mod p)} \((i = 1, 2)\)
   \end{itemize}
   Then, update {\tt p} to {\tt p**2}.
   \spacing
   % added document
   \quad \negok The preconditions which should be automatically satisfied:
   \begin{itemize}
   \item {\tt a1*u1 + a2*u2 == 1 (mod p)}
   \end{itemize}

\subsection{HenselLiftMulti -- Hensel lift for multiple polynomials}\linkedone{poly.hensel}{HenselLiftMulti}
 \initialize
  \func{HenselLiftMulti}{%
    \hiki{f}{polynomial},
    \hiki{factors}{list},
    \hiki{ladder}{tuple},
    \hiki{p}{integer},
    \hikiopt{q}{integer}{p}}{\out{HenselLiftMulti}}\\
  \spacing
  % document of basic document
  \quad This object keeps integer polynomial factors which will be lifted by Hensel's lemma.
  If the number of factors is just two, then you should use \linkingone{poly.hensel}{HenselLiftPair}.
  % added document
  %
  \spacing
  % input, output document
  \quad \param{factors} is a list of polynomials;
  we refer those polynomials as {\tt a1}, {\tt a2}, \(\ldots\)
  \param{ladder} is a tuple of two lists {\tt sis} and {\tt tis},
  both lists consist polynomials.
  We refer polynomials in {\tt sis} as {\tt s1}, {\tt s2}, \(\ldots\),
  and those in {\tt tis} as {\tt t1}, {\tt t2}, \(\ldots\)
  Moreover, we define {\tt bi} as the product of {\tt aj}'s for
  \(i < j\).
  \quad The argument should satisfy the following preconditions:
  \begin{itemize}
  \item \param{f} and all of \param{factors} are monic
  \item {\tt \param{f} == \param{a1}*...*\param{ar} (mod \param{q})}
  \item {\tt ai*si + bi*ti == 1 (mod \param{p})} \((i = 1,2,\ldots,r)\)
  \item \param{p} divides \param{q} and both are positive
  \end{itemize}
%
  \func{from\_factors}{%
    \hiki{f}{polynomial},
    \hiki{factors}{list},
    \hiki{p}{integer}}{\out{HenselLiftMulti}}\\
  \spacing
  \quad This is a class method to create and return an instance of {\tt HenselLiftMulti}.
  You do not have to precompute {\tt ladder} for the default constructor; they will be prepared for you from other arguments.\\
  \spacing
  % input, output document
  \quad The argument should satisfy the following preconditions:
  \begin{itemize}
  \item \param{f} and all of \param{factors} are monic
  \item {\tt \param{f} == \param{a1}*...*\param{ar} (mod \param{q})}
  \item \param{p} is prime
  \end{itemize}
  \begin{at}
    \item[point]\linkedtwo{poly.hensel}{HenselLiftMulti}{factors}:\\
      factors {\tt ai}s as a list.
  \end{at}
  \method
  \subsubsection{lift -- lift one step}\linkedtwo{poly.hensel}{HenselLiftMulti}{lift}
  \func{lift}{\param{self}}{}\\
  \spacing
  \quad Lift polynomials by so-called the quadratic method.
  \subsubsection{lift\_factors -- lift factors}\linkedtwo{poly.hensel}{HenselLiftMulti}{lift\_factors}
  \func{lift\_factors}{\param{self}}{}\\
  \spacing
  % document of basic document
  \quad Update factors by lifted integer coefficient polynomials {\tt Ai}s:
  \begin{itemize}
  \item {\tt f == A1*...*Ar (mod p * q)}
  \item {\tt Ai == ai (mod q)} \((i = 1, \ldots, r)\)
  \end{itemize}
  Moreover, {\tt q} is updated to {\tt p * q}.
  \spacing
  % added document
  \quad \negok The preconditions which should be automatically satisfied:
  \begin{itemize}
  \item {\tt f == a1*...*ar (mod q)}
  \item {\tt ai*si + bi*ti == 1 (mod p)} \((i = 1,\ldots, r)\)
  \item {\tt p} divides {\tt q}
  \end{itemize}
  \subsubsection{lift\_ladder -- lift {\tt u1} and {\tt u2}}\linkedtwo{poly.hensel}{HenselLiftMulti}{lift\_ladder}
  \func{lift\_ladder}{\param{self}}{}\\
  \spacing
  % document of basic document
  \quad Update {\tt si}s and {\tt ti}s with {\tt Si}s and {\tt Ti}s:
  \begin{itemize}
  \item {\tt a1*Si + bi*Ti == 1 (mod p**2)}
  \item {\tt Si == si (mod p)} \((i = 1, \ldots, r)\)
  \item {\tt Ti == ti (mod p)} \((i = 1, \ldots, r)\)
  \end{itemize}
  Then, update {\tt p} to {\tt p**2}.
  \spacing
  % added document
  \quad \negok The preconditions which should be automatically satisfied:
  \begin{itemize}
  \item {\tt ai*si + bi*ti == 1 (mod p)} \((i = 1,\ldots, r)\)
  \end{itemize}
%

\subsection{HenselLiftSimultaneously}\linkedone{poly.hensel}{HenselLiftSimultaneously}

  The method explained in~\cite{ColEnc}.\\
  \quad \negok Keep these invariants:
  \begin{itemize}
  \item     {\tt ai}s, {\tt pi} and {\tt gi}s are monic
  \item     {\tt f == g1*...*gr (mod p)}
  \item     {\tt f == d0 + d1*p + d2*p**2 +...+ dk*p**k}
  \item     {\tt hi == g(i+1)*...*gr}
  \item     {\tt 1 == gi*si + hi*ti (mod p)} \((i = 1 ,\ldots, r)\)
  \item     \(\deg\)({\tt si}) \(<\) \(\deg\)({\tt hi}),
    \(\deg\)({\tt ti}) \(<\) \(\deg\)({\tt gi}) \((i = 1 ,\ldots, r)\)
  \item     {\tt p} divides {\tt q}
  \item     {\tt f == l1*...*lr (mod q/p)}
  \item     {\tt f == a1*...*ar (mod q)}
  \item     {\tt ui == ai*yi + bi*zi (mod p)} \((i = 1, \ldots, r)\)
  \end{itemize}

 \initialize
  \func{HenselLiftSimultaneously}{%
    \hiki{target}{polynomial},
    \hiki{factors}{list},
    \hiki{cofactors}{list},
    \hiki{bases}{list},
    \hiki{p}{integer}}{\out{HenselLiftSimultaneously}}\\
  \spacing
  % document of basic document
  \quad This object keeps integer polynomial factors which will be lifted by Hensel's lemma.\\
  \spacing
  \quad {\tt f = \param{target}}, {\tt gi} in \param{factors},
  {\tt hi}s in \param{cofactors} and {\tt si}s and {\tt ti}s are in \param{bases}.
%
  \func{from\_factors}{%
    \hiki{target}{polynomial},\ %
    \hiki{factors}{list},\ %
    \hiki{p}{integer},\ %
    \hikiopt{ubound}{integer}{\linklibraryone{sys\#maxint}{sys.maxint}}}{%
    \out{HenselLiftSimultaneously}}
  \spacing
  % document of basic document
  \quad This is a class method to create and return an instance of {\tt HenselLiftSimultaneously}, whose factors are lifted by \linkingone{poly.hensel}{HenselLiftMulti} upto \param{ubound} if it is smaller than {\tt sys.maxint}, or upto {\tt sys.maxint} otherwise.
  You do not have to precompute auxiliary polynomials for the default
  constructor; they will be prepared for you from other arguments.\\
  \spacing
  \quad {\tt f = \param{target}}, {\tt gi}s in \param{factors}.
%
  \method
  \subsubsection{lift -- lift one step}\linkedtwo{poly.hensel}{HenselLiftSimultaneously}{lift}
  \func{lift}{\param{self}}{}\\
  \spacing
  The lift. You should call this method only.
  \subsubsection{first\_lift -- the first step}\linkedtwo{poly.hensel}{HenselLiftSimultaneously}{first\_lift}
  \func{first\_lift}{\param{self}}{}\\
  \spacing
  \quad Start lifting.\\
  {\tt f == l1*l2*...*lr (mod p**2)}\\
  Initialize {\tt di}s, {\tt ui}s, {\tt yi}s and {\tt zi}s.
  Update {\tt ai}s, {\tt bi}s.
  Then, update {\tt q} with {\tt p**2}.
  \subsubsection{general\_lift -- next step}\linkedtwo{poly.hensel}{HenselLiftSimultaneously}{general\_lift}
  \func{general\_lift}{\param{self}}{}\\
  \spacing
  \quad Continue lifting.\\
  {\tt f == a1*a2*...*ar (mod p*q)}\\
  Initialize {\tt ai}s, {\tt ubi}s, {\tt yi}s and {\tt zi}s.
  Then, update {\tt q} with {\tt p*q.}

  \subsection{lift\_upto -- main function}\linkedone{poly.hensel}{lift\_upto}
  \func{lift\_upto}{\param{self},\ %
  \hiki{target}{polynomial},\  %
  \hiki{factors}{list},\ %
  \hiki{p}{integer},\ %
  \hiki{bound}{integer}}{\out{tuple}}\\
\spacing
\quad Hensel lift \param{factors} mod \param{p} of \param{target} upto \param{bound}
and return {\tt factors} mod {\tt q} and the {\tt q} itself.\\
\quad These preconditions should be satisfied:
\begin{itemize}
\item \param{target} is monic.
\item {\tt \param{target} == product(\param{factors}) mod \param{p}}
\end{itemize}
\quad The result {\tt (factors, q)} satisfies the following postconditions:
\begin{itemize}
\item there exist \(k\) s.t. {\tt q == \param{p}**k >= \param{bound}} and
\item {\tt \param{target} == product(factors) mod q}
\end{itemize}

\C

%---------- end document ---------- %

\bibliographystyle{jplain}%use jbibtex
\bibliography{nzmath_references}

\end{document}


%%%%%%%%%%%%%%%%%%%%%%%%%%%%%%%%%%%%%%%%%%%%%%%%%%%%%%%%%%%%%%
%
% macros for nzmath manual
%
%%%%%%%%%%%%%%%%%%%%%%%%%%%%%%%%%%%%%%%%%%%%%%%%%%%%%%%%%%%%%
\usepackage{amssymb,amsmath}
\usepackage{color}
\usepackage[dvipdfm,bookmarks=true,bookmarksnumbered=true,%
 pdftitle={NZMATH Users Manual},%
 pdfsubject={Manual for NZMATH Users},%
 pdfauthor={NZMATH Development Group},%
 pdfkeywords={TeX; dvipdfmx; hyperref; color;},%
 colorlinks=true]{hyperref}
\usepackage{fancybox}
\usepackage[T1]{fontenc}
%
\newcommand{\DS}{\displaystyle}
\newcommand{\C}{\clearpage}
\newcommand{\NO}{\noindent}
\newcommand{\negok}{$\dagger$}
\newcommand{\spacing}{\vspace{1pt}\\ }
% software macros
\newcommand{\nzmathzero}{{\footnotesize $\mathbb{N}\mathbb{Z}$}\texttt{MATH}}
\newcommand{\nzmath}{{\nzmathzero}\ }
\newcommand{\pythonzero}{$\mbox{\texttt{Python}}$}
\newcommand{\python}{{\pythonzero}\ }
% link macros
\newcommand{\linkingzero}[1]{{\bf \hyperlink{#1}{#1}}}%module
\newcommand{\linkingone}[2]{{\bf \hyperlink{#1.#2}{#2}}}%module,class/function etc.
\newcommand{\linkingtwo}[3]{{\bf \hyperlink{#1.#2.#3}{#3}}}%module,class,method
\newcommand{\linkedzero}[1]{\hypertarget{#1}{}}
\newcommand{\linkedone}[2]{\hypertarget{#1.#2}{}}
\newcommand{\linkedtwo}[3]{\hypertarget{#1.#2.#3}{}}
\newcommand{\linktutorial}[1]{\href{http://docs.python.org/tutorial/#1}{#1}}
\newcommand{\linktutorialone}[2]{\href{http://docs.python.org/tutorial/#1}{#2}}
\newcommand{\linklibrary}[1]{\href{http://docs.python.org/library/#1}{#1}}
\newcommand{\linklibraryone}[2]{\href{http://docs.python.org/library/#1}{#2}}
\newcommand{\pythonhp}{\href{http://www.python.org/}{\python website}}
\newcommand{\nzmathwiki}{\href{http://nzmath.sourceforge.net/wiki/}{{\nzmathzero}Wiki}}
\newcommand{\nzmathsf}{\href{http://sourceforge.net/projects/nzmath/}{\nzmath Project Page}}
\newcommand{\nzmathtnt}{\href{http://tnt.math.se.tmu.ac.jp/nzmath/}{\nzmath Project Official Page}}
% parameter name
\newcommand{\param}[1]{{\tt #1}}
% function macros
\newcommand{\hiki}[2]{{\tt #1}:\ {\em #2}}
\newcommand{\hikiopt}[3]{{\tt #1}:\ {\em #2}=#3}

\newdimen\hoge
\newdimen\truetextwidth
\newcommand{\func}[3]{%
\setbox0\hbox{#1(#2)}
\hoge=\wd0
\truetextwidth=\textwidth
\advance \truetextwidth by -2\oddsidemargin
\ifdim\hoge<\truetextwidth % short form
{\bf \colorbox{skyyellow}{#1(#2)\ $\to$ #3}}
%
\else % long form
\fcolorbox{skyyellow}{skyyellow}{%
   \begin{minipage}{\textwidth}%
   {\bf #1(#2)\\ %
    \qquad\quad   $\to$\ #3}%
   \end{minipage}%
   }%
\fi%
}

\newcommand{\out}[1]{{\em #1}}
\newcommand{\initialize}{%
  \paragraph{\large \colorbox{skyblue}{Initialize (Constructor)}}%
    \quad\\ %
    \vspace{3pt}\\
}
\newcommand{\method}{\C \paragraph{\large \colorbox{skyblue}{Methods}}}
% Attribute environment
\newenvironment{at}
{%begin
\paragraph{\large \colorbox{skyblue}{Attribute}}
\quad\\
\begin{description}
}%
{%end
\end{description}
}
% Operation environment
\newenvironment{op}
{%begin
\paragraph{\large \colorbox{skyblue}{Operations}}
\quad\\
\begin{table}[h]
\begin{center}
\begin{tabular}{|l|l|}
\hline
operator & explanation\\
\hline
}%
{%end
\hline
\end{tabular}
\end{center}
\end{table}
}
% Examples environment
\newenvironment{ex}%
{%begin
\paragraph{\large \colorbox{skyblue}{Examples}}
\VerbatimEnvironment
\renewcommand{\EveryVerbatim}{\fontencoding{OT1}\selectfont}
\begin{quote}
\begin{Verbatim}
}%
{%end
\end{Verbatim}
\end{quote}
}
%
\definecolor{skyblue}{cmyk}{0.2, 0, 0.1, 0}
\definecolor{skyyellow}{cmyk}{0.1, 0.1, 0.5, 0}
%
%\title{NZMATH User Manual\\ {\large{(for version 1.0)}}}
%\date{}
%\author{}
\begin{document}
%\maketitle
%
\setcounter{tocdepth}{3}
\setcounter{secnumdepth}{3}


\tableofcontents
\C

\chapter{Classes}


%---------- start document ---------- %
 \section{poly.multiutil -- ���ϐ��������ɑ΂��郆�[�e�B���e�B}\linkedzero{poly.multiutil}
 \begin{itemize}
   \item {\bf Classes}
     \begin{itemize}
     \item \linkingone{poly.multiutil}{RingPolynomial}
     \item \linkingone{poly.multiutil}{DomainPolynomial}
     \item \linkingone{poly.multiutil}{UniqueFactorizationDomainPolynomial}
     \item OrderProvider
     \item NestProvider
     \item PseudoDivisionProvider
     \item GcdProvider
     \item RingElementProvider
     \end{itemize}
   \item {\bf Functions}
     \begin{itemize}
     \item \linkingone{poly.multiutil}{polynomial}
     \end{itemize}
 \end{itemize}

\C

 \subsection{RingPolynomial}\linkedone{poly.multiutil}{RingPolynomial}
 �Š��ŒW�������ˆ�ʂ̑�����.

 \initialize
   \func{RingPolynomial}{%
    \hiki{coefficients}{terminit},\ %
    **\hiki{keywords}{dict}}{\out{RingPolynomial}}\\
  \spacing
  % document of basic document
  \quad \param{keywords}�͈ȉ����܂܂Ȃ���΂Ȃ�Ȃ�:
  \begin{description}
  \item[coeffring] �Š���({\it CommutativeRing})
  \item[number\_of\_variables] �ϐ��̐�({\it integer})
  \item[order] ������({\it TermOrder})
  \end{description}
  \quad ���̃N���X��\linkingone{poly.multivar}{BasicPolynomial},
  \linkingone{poly.multiutil}{OrderProvider},
  \linkingone{poly.multiutil}{NestProvider} and
  \linkingone{poly.multiutil}{RingElementProvider}���p������.
%
  \begin{at}
    \item[order]\linkedtwo{poly.multiutil}{RingPolynomial}{order}:\\
      ������.
  \end{at}
%
  \method
  \subsubsection{getRing}\linkedtwo{poly.multiutil}{RingPolynomial}{getRing}
  \func{getRing}{\param{self}}{\out{Ring}}\\
  \spacing
  \quad ����������������{\tt Ring}�̃T�u�N���X�̃I�u�W�F�N�g��Ԃ�.\\
  (���̃��\�b�h��RingElementProvider���̒�`���I�[�o�[���C�h����)

  \subsubsection{getCoefficientRing}\linkedtwo{poly.multiutil}{RingPolynomial}{getCoefficientRing}
  \func{getCoefficientRing}{\param{self}}{\out{Ring}}\\
  \spacing
  \quad ���ׂĂ̌W������������{\tt Ring}�̃T�u�N���X�̃I�u�W�F�N�g��Ԃ�.\\
  (���̃��\�b�h��RingElementProvider���̒�`���I�[�o�[���C�h����)

  \subsubsection{leading\_variable}\linkedtwo{poly.multiutil}{RingPolynomial}{leading\_variable}
  \func{leading\_variable}{\param{self}}{\out{integer}}\\
  \spacing
  \quad ��ϐ�(�S�Ă̑S������1�̍��̒��ł̎區)�̈ʒu��Ԃ�.\\
  �區�͌��ʂƂ��č������ɕω�����.�������͑���{\tt order}�ɂ���Ďw�肳���.\\
  (���̃��\�b�h��NestProvider����p�������)

  \subsubsection{nest}\linkedtwo{poly.multiutil}{RingPolynomial}{nest}
  \func{nest}{\param{self},\ \hiki{outer}{integer},\ \hiki{coeffring}{CommutativeRing}}{\out{polynomial}}\\
  \spacing
  \quad �^����ꂽ�ʒu�̕ϐ�\param{outer}�������o�����Ƃɂ�葽�������l�X�g.\\
  (���̃��\�b�h��NestProvider����p�������)

  \subsubsection{unnest}\linkedtwo{poly.multiutil}{RingPolynomial}{unnest}
  \func{nest}{\param{self},\ \hiki{q}{polynomial},\ \hiki{outer}{integer},\ \hiki{coeffring}{CommutativeRing}}{\out{polynomial}}\\
  \spacing
  \quad �^����ꂽ�ʒu�̕ϐ�\param{outer}��}�����邱�Ƃɂ��l�X�g���ꂽ������\param{q}���A���l�X�g���܂�.\\
  (���̃��\�b�h��NestProvider����p������܂�)

 \subsection{DomainPolynomial}\linkedone{poly.multiutil}{DomainPolynomial}
 ����̌W�������‘�����.
 \initialize
   \func{DomainPolynomial}{%
    \hiki{coefficients}{terminit},\ %
    **\hiki{keywords}{dict}}{\out{DomainPolynomial}}\\
  \spacing
  % document of basic document
  \quad \param{keywords}�͈ȉ����܂܂Ȃ���΂Ȃ�Ȃ�:
  \begin{description}
  \item[coeffring] �Š���({\it CommutativeRing})
  \item[number\_of\_variables] �ϐ��̐�({\it integer})
  \item[order] ������({\it TermOrder})
  \end{description}
  \quad ���̃N���X��\linkingone{poly.multiutil}{RingPolynomial}��\linkingone{poly.multiutil}{PseudoDivisionProvider}���p������.
%
  \begin{op}
    \verb+f / g+ & ���Z(���ʂ͗L���֐�)\\
  \end{op}
  \method

  \subsubsection{pseudo\_divmod}\linkedtwo{poly.multiutil}{DomainPolynomial}{pseudo\_divmod}
  \func{pseudo\_divmod}{\param{self},\ \hiki{other}{polynomial}}{\out{polynomial}}\\
  \spacing
  \quad �ȉ��ƂȂ鑽����\(Q\), \(R\) ��Ԃ�:
  \[d^{deg(self) - deg(other) + 1} self = other \times Q + R\]
  �Œ�l�Ƃ���\param{other}�̎�W���ł���\(d\).\\
  \quad ���ʂƂ��Ď�W���͍��̌W���ɕς��. �������͑���{\tt order}�ɂ���Ďw�肳���.\\
  (���̃��\�b�h��PseudoDivisionProvider����p�������.)

  \subsubsection{pseudo\_floordiv}\linkedtwo{poly.multiutil}{DomainPolynomial}{pseudo\_floordiv}
  \func{pseudo\_floordiv}{\param{self},\ \hiki{other}{polynomial}}{\out{polynomial}}\\
  \spacing
  \quad �ȉ��ƂȂ鑽����\(Q\) ��Ԃ�:
  \[d^{deg(self) - deg(other) + 1} self = other \times Q + R\]
  �Œ�l�Ƃ���\param{other}�̎�W��\(d\) �� ������\(R\).\\

  ���ʂƂ��Ď�W���͍������ɕς��.
  �������͑���{\tt order}�ɂ���Ďw�肳���.\\
  (���̃��\�b�h��PseudoDivisionProvider����p�������.)

  \subsubsection{pseudo\_mod}\linkedtwo{poly.multiutil}{DomainPolynomial}{pseudo\_mod}
  \func{pseudo\_mod}{\param{self},\ \hiki{other}{polynomial}}{\out{polynomial}}\\
  \quad �ȉ��ƂȂ鑽����\(R\) ��Ԃ�:
  \[ d^{deg(self) - deg(other) + 1} \times self = other \times Q + R\]
  \(d\) �� \param{other}�̎�W����\(Q\) �͑�����.\\
  \quad ���ʂƂ��Ď�W���͍��̈ʐ��ɕς��.�������͑���{\tt order}�ɂ���Ďw�肳���.\\
  (���̃��\�b�h��PseudoDivisionProvider����p�������.)

  \subsubsection{exact\_division}\linkedtwo{poly.multiutil}{DomainPolynomial}{exact\_division}
  \func{exact\_division}{\param{self},\ \hiki{other}{polynomial}}{\out{polynomial}}\\
  \spacing
  \quad (����؂��Ƃ��̂�)���Z�ŏ���Ԃ�.\\
  (���̃��\�b�h��PseudoDivisionProvider����p�������.)

 \subsection{UniqueFactorizationDomainPolynomial}\linkedone{poly.multiutil}{UniqueFactorizationDomainPolynomial}
 ��ӕ��𐹈�(UFD)�W�������‘�����.
 \initialize
   \func{UniqueFactorizationDomainPolynomial}{%
    \hiki{coefficients}{terminit},\ %
    **\hiki{keywords}{dict}}{\out{UniqueFactorizationDomainPolynomial}}\\
  \spacing
  % document of basic document
  \quad \param{keywords}�͈ȉ����܂܂Ȃ���΂Ȃ�Ȃ�:
  \begin{description}
  \item[coeffring] �Š���({\it CommutativeRing})
  \item[number\_of\_variables] �ϐ��̐�({\it integer})
  \item[order] ������({\it TermOrder})
  \end{description}
  \quad ���̃N���X��\linkingone{poly.multiutil}{DomainPolynomial}��\linkingone{poly.multiutil}{GcdProvider}���p������.
  \method
  \subsubsection{gcd}\linkedtwo{poly.multiutil}{UniqueFactorizationDomainPolynomial}{gcd}
  \func{gcd}{\param{self},\ \hiki{other}{polynomial}}{\out{polynomial}}\\
  \quad gcd��Ԃ�.�l�X�g���ꂽ��������gcd���g����.\\
  (���̃��\�b�h��GcdProvider����p�������.)

  \subsubsection{resultant}\linkedtwo{poly.multiutil}{UniqueFactorizationDomainPolynomial}{resultant}
  \func{resultant}{\param{self},\ \hiki{other}{polynomial},\ \hiki{var}{integer}}{\out{polynomial}}\\
  \quad ���̈ʒu\param{var}�ɂ���Ďw�肳�ꂽ�ϐ��ɂ‚��Ă�,������̓�‚̑������̏I������Ԃ�.

% Functions
 \subsection{polynomial -- ���܂��܂ȑ������ɑ΂���t�@�N�g���֐�}\linkedone{poly.multiutil}{polynomial}
  \func{polynomial}{\hiki{coefficients}{terminit},\ \hiki{coeffring}{CommutativeRing}, \hikiopt{number\_of\_variables}{integer}{None}}{\out{polynomial}}\\
   \spacing
   % document of basic document
   \quad ��������Ԃ�.\\
   \spacing
   \quad \negok �֐����Ă΂��O�Ɏ��̐ݒ�����邱�Ƃɂ��,�W���‚��瑽�����̌^��I�ԕ��@���I�[�o�[���C�h�ł���:\\
   {\tt special\_ring\_table[coeffring\_type] = polynomial\_type}\\.

 \subsection{prepare\_indeterminates -- �s�茳�A���錾}\linkedone{poly.multiutil}{prepare\_indeterminates}
 \func{prepare\_indeterminates}{\hiki{names}{string},\ \hiki{ctx}{dict},\ %
   \hikiopt{coeffring}{CoefficientRing}{None}}{\out{None}}\\
 \spacing
 \quad �s�茳��\param{names}�ɂ���ĕ�����ꂽ��Ԃ���,�s�茳��\���ϐ���p�ӂ���.  ���ʂ͎���\param{ctx}�Ɋi�[�����.\\
 \quad �ϐ��͂����ɗp�ӂ����ׂ��ł���,�����Ȃ��ΊԈ�����ϐ��̃G�C���A�X���v�Z��x�������������邾�낤.\\
 \quad �����C�ӈ�����\param{coeffring}���^�����Ȃ����,�s�茳�͐����W���������Ƃ��ď����������.\\

\begin{ex}
>>> prepare_indeterminates("X Y Z", globals())
>>> Y
UniqueFactorizationDomainPolynomial({(0, 1, 0): 1})
\end{ex}
\C

%---------- end document ---------- %

\bibliographystyle{jplain}%use jbibtex
\bibliography{nzmath_references}

\end{document}


%\documentclass{report}

%%%%%%%%%%%%%%%%%%%%%%%%%%%%%%%%%%%%%%%%%%%%%%%%%%%%%%%%%%%%%
%
% macros for nzmath manual
%
%%%%%%%%%%%%%%%%%%%%%%%%%%%%%%%%%%%%%%%%%%%%%%%%%%%%%%%%%%%%%
\usepackage{amssymb,amsmath}
\usepackage{color}
\usepackage[dvipdfm,bookmarks=true,bookmarksnumbered=true,%
 pdftitle={NZMATH Users Manual},%
 pdfsubject={Manual for NZMATH Users},%
 pdfauthor={NZMATH Development Group},%
 pdfkeywords={TeX; dvipdfmx; hyperref; color;},%
 colorlinks=true]{hyperref}
\usepackage{fancybox}
\usepackage[T1]{fontenc}
%
\newcommand{\DS}{\displaystyle}
\newcommand{\C}{\clearpage}
\newcommand{\NO}{\noindent}
\newcommand{\negok}{$\dagger$}
\newcommand{\spacing}{\vspace{1pt}\\ }
% software macros
\newcommand{\nzmathzero}{{\footnotesize $\mathbb{N}\mathbb{Z}$}\texttt{MATH}}
\newcommand{\nzmath}{{\nzmathzero}\ }
\newcommand{\pythonzero}{$\mbox{\texttt{Python}}$}
\newcommand{\python}{{\pythonzero}\ }
% link macros
\newcommand{\linkingzero}[1]{{\bf \hyperlink{#1}{#1}}}%module
\newcommand{\linkingone}[2]{{\bf \hyperlink{#1.#2}{#2}}}%module,class/function etc.
\newcommand{\linkingtwo}[3]{{\bf \hyperlink{#1.#2.#3}{#3}}}%module,class,method
\newcommand{\linkedzero}[1]{\hypertarget{#1}{}}
\newcommand{\linkedone}[2]{\hypertarget{#1.#2}{}}
\newcommand{\linkedtwo}[3]{\hypertarget{#1.#2.#3}{}}
\newcommand{\linktutorial}[1]{\href{http://docs.python.org/tutorial/#1}{#1}}
\newcommand{\linktutorialone}[2]{\href{http://docs.python.org/tutorial/#1}{#2}}
\newcommand{\linklibrary}[1]{\href{http://docs.python.org/library/#1}{#1}}
\newcommand{\linklibraryone}[2]{\href{http://docs.python.org/library/#1}{#2}}
\newcommand{\pythonhp}{\href{http://www.python.org/}{\python website}}
\newcommand{\nzmathwiki}{\href{http://nzmath.sourceforge.net/wiki/}{{\nzmathzero}Wiki}}
\newcommand{\nzmathsf}{\href{http://sourceforge.net/projects/nzmath/}{\nzmath Project Page}}
\newcommand{\nzmathtnt}{\href{http://tnt.math.se.tmu.ac.jp/nzmath/}{\nzmath Project Official Page}}
% parameter name
\newcommand{\param}[1]{{\tt #1}}
% function macros
\newcommand{\hiki}[2]{{\tt #1}:\ {\em #2}}
\newcommand{\hikiopt}[3]{{\tt #1}:\ {\em #2}=#3}

\newdimen\hoge
\newdimen\truetextwidth
\newcommand{\func}[3]{%
\setbox0\hbox{#1(#2)}
\hoge=\wd0
\truetextwidth=\textwidth
\advance \truetextwidth by -2\oddsidemargin
\ifdim\hoge<\truetextwidth % short form
{\bf \colorbox{skyyellow}{#1(#2)\ $\to$ #3}}
%
\else % long form
\fcolorbox{skyyellow}{skyyellow}{%
   \begin{minipage}{\textwidth}%
   {\bf #1(#2)\\ %
    \qquad\quad   $\to$\ #3}%
   \end{minipage}%
   }%
\fi%
}

\newcommand{\out}[1]{{\em #1}}
\newcommand{\initialize}{%
  \paragraph{\large \colorbox{skyblue}{Initialize (Constructor)}}%
    \quad\\ %
    \vspace{3pt}\\
}
\newcommand{\method}{\C \paragraph{\large \colorbox{skyblue}{Methods}}}
% Attribute environment
\newenvironment{at}
{%begin
\paragraph{\large \colorbox{skyblue}{Attribute}}
\quad\\
\begin{description}
}%
{%end
\end{description}
}
% Operation environment
\newenvironment{op}
{%begin
\paragraph{\large \colorbox{skyblue}{Operations}}
\quad\\
\begin{table}[h]
\begin{center}
\begin{tabular}{|l|l|}
\hline
operator & explanation\\
\hline
}%
{%end
\hline
\end{tabular}
\end{center}
\end{table}
}
% Examples environment
\newenvironment{ex}%
{%begin
\paragraph{\large \colorbox{skyblue}{Examples}}
\VerbatimEnvironment
\renewcommand{\EveryVerbatim}{\fontencoding{OT1}\selectfont}
\begin{quote}
\begin{Verbatim}
}%
{%end
\end{Verbatim}
\end{quote}
}
%
\definecolor{skyblue}{cmyk}{0.2, 0, 0.1, 0}
\definecolor{skyyellow}{cmyk}{0.1, 0.1, 0.5, 0}
%
%\title{NZMATH User Manual\\ {\large{(for version 1.0)}}}
%\date{}
%\author{}
\begin{document}
%\maketitle
%
\setcounter{tocdepth}{3}
\setcounter{secnumdepth}{3}


\tableofcontents
\C

\chapter{Classes}


%---------- start document ---------- %
 \section{poly.multivar -- multivariate polynomial}\linkedzero{poly.multivar}
 \begin{itemize}
   \item {\bf Classes}
   \begin{itemize}
     \item \negok \linkingone{poly.multivar}{PolynomialInterface}
     \item \negok \linkingone{poly.multivar}{BasicPolynomial}
     \item \linkingone{poly.multivar}{TermIndices}
   \end{itemize}
 \end{itemize}

\C

 \subsection{PolynomialInterface -- base class for all multivariate polynomials}\linkedone{poly.multivar}{PolynomialInterface}
  Since the interface is an abstract class, do not instantiate.\\

  % No Documentation Yet
%
 \subsection{BasicPolynomial -- basic implementation of polynomial}\linkedone{poly.multivar}{BasicPolynomial}
  Basic polynomial data type.

  % No Documentation Yet
%
 \subsection{TermIndices -- Indices of terms of multivariate polynomials}\linkedone{poly.multivar}{TermIndices}
  It is a tuple-like object.
  \initialize
  \func{TermIndices}{\hiki{indices}{tuple}}{\out{TermIndices}}\\
  \spacing
  \quad The constructor does not check the validity of indices, such
  as integerness, nonnegativity, etc.
  \begin{op}
    \verb/t == u/ & equality\\
    \verb/t != u/ & inequality\\
    \verb/t + u/ & (componentwise) addition\\
    \verb/t - u/ & (componentwise) subtraction\\
    \verb/t * a/ & (componentwise) multiplication by scalar {\tt a}\\
    \verb/t <= u, t < u, t >= u, t > u/ & ordering\\
    \verb/t[k]/ & {\tt k}-th index\\
    \verb/len(t)/ & length\\
    \verb/iter(t)/ & iterator\\
    \verb/hash(t)/ & hash\\
  \end{op}
  \method
  \subsubsection{pop}\linkedtwo{poly.multivar}{TermIndices}{pop}
   \func{pop}{\param{self},\ \hiki{pos}{integer}}{(\out{integer}, \out{TermIndices})}\\
   \spacing
   % document of basic document
   \quad Return the index at pos and a new TermIndices object as the
   omitting-the-pos indices.

  \subsubsection{gcd}\linkedtwo{poly.multivar}{TermIndices}{gcd}
  \func{gcd}{\param{self},\ \hiki{other}{TermIndices}}{\out{TermIndices}}\\
  \quad Return the ``gcd'' of two indices.

  \subsubsection{lcm}\linkedtwo{poly.multivar}{TermIndices}{lcm}
  \func{lcm}{\param{self},\ \hiki{other}{TermIndices}}{\out{TermIndices}}\\
  \quad Return the ``lcm'' of two indices.

\C

%---------- end document ---------- %

\bibliographystyle{jplain}%use jbibtex
\bibliography{nzmath_references}

\end{document}


%%%%%%%%%%%%%%%%%%%%%%%%%%%%%%%%%%%%%%%%%%%%%%%%%%%%%%%%%%%%%%
%
% macros for nzmath manual
%
%%%%%%%%%%%%%%%%%%%%%%%%%%%%%%%%%%%%%%%%%%%%%%%%%%%%%%%%%%%%%
\usepackage{amssymb,amsmath}
\usepackage{color}
\usepackage[dvipdfm,bookmarks=true,bookmarksnumbered=true,%
 pdftitle={NZMATH Users Manual},%
 pdfsubject={Manual for NZMATH Users},%
 pdfauthor={NZMATH Development Group},%
 pdfkeywords={TeX; dvipdfmx; hyperref; color;},%
 colorlinks=true]{hyperref}
\usepackage{fancybox}
\usepackage[T1]{fontenc}
%
\newcommand{\DS}{\displaystyle}
\newcommand{\C}{\clearpage}
\newcommand{\NO}{\noindent}
\newcommand{\negok}{$\dagger$}
\newcommand{\spacing}{\vspace{1pt}\\ }
% software macros
\newcommand{\nzmathzero}{{\footnotesize $\mathbb{N}\mathbb{Z}$}\texttt{MATH}}
\newcommand{\nzmath}{{\nzmathzero}\ }
\newcommand{\pythonzero}{$\mbox{\texttt{Python}}$}
\newcommand{\python}{{\pythonzero}\ }
% link macros
\newcommand{\linkingzero}[1]{{\bf \hyperlink{#1}{#1}}}%module
\newcommand{\linkingone}[2]{{\bf \hyperlink{#1.#2}{#2}}}%module,class/function etc.
\newcommand{\linkingtwo}[3]{{\bf \hyperlink{#1.#2.#3}{#3}}}%module,class,method
\newcommand{\linkedzero}[1]{\hypertarget{#1}{}}
\newcommand{\linkedone}[2]{\hypertarget{#1.#2}{}}
\newcommand{\linkedtwo}[3]{\hypertarget{#1.#2.#3}{}}
\newcommand{\linktutorial}[1]{\href{http://docs.python.org/tutorial/#1}{#1}}
\newcommand{\linktutorialone}[2]{\href{http://docs.python.org/tutorial/#1}{#2}}
\newcommand{\linklibrary}[1]{\href{http://docs.python.org/library/#1}{#1}}
\newcommand{\linklibraryone}[2]{\href{http://docs.python.org/library/#1}{#2}}
\newcommand{\pythonhp}{\href{http://www.python.org/}{\python website}}
\newcommand{\nzmathwiki}{\href{http://nzmath.sourceforge.net/wiki/}{{\nzmathzero}Wiki}}
\newcommand{\nzmathsf}{\href{http://sourceforge.net/projects/nzmath/}{\nzmath Project Page}}
\newcommand{\nzmathtnt}{\href{http://tnt.math.se.tmu.ac.jp/nzmath/}{\nzmath Project Official Page}}
% parameter name
\newcommand{\param}[1]{{\tt #1}}
% function macros
\newcommand{\hiki}[2]{{\tt #1}:\ {\em #2}}
\newcommand{\hikiopt}[3]{{\tt #1}:\ {\em #2}=#3}

\newdimen\hoge
\newdimen\truetextwidth
\newcommand{\func}[3]{%
\setbox0\hbox{#1(#2)}
\hoge=\wd0
\truetextwidth=\textwidth
\advance \truetextwidth by -2\oddsidemargin
\ifdim\hoge<\truetextwidth % short form
{\bf \colorbox{skyyellow}{#1(#2)\ $\to$ #3}}
%
\else % long form
\fcolorbox{skyyellow}{skyyellow}{%
   \begin{minipage}{\textwidth}%
   {\bf #1(#2)\\ %
    \qquad\quad   $\to$\ #3}%
   \end{minipage}%
   }%
\fi%
}

\newcommand{\out}[1]{{\em #1}}
\newcommand{\initialize}{%
  \paragraph{\large \colorbox{skyblue}{Initialize (Constructor)}}%
    \quad\\ %
    \vspace{3pt}\\
}
\newcommand{\method}{\C \paragraph{\large \colorbox{skyblue}{Methods}}}
% Attribute environment
\newenvironment{at}
{%begin
\paragraph{\large \colorbox{skyblue}{Attribute}}
\quad\\
\begin{description}
}%
{%end
\end{description}
}
% Operation environment
\newenvironment{op}
{%begin
\paragraph{\large \colorbox{skyblue}{Operations}}
\quad\\
\begin{table}[h]
\begin{center}
\begin{tabular}{|l|l|}
\hline
operator & explanation\\
\hline
}%
{%end
\hline
\end{tabular}
\end{center}
\end{table}
}
% Examples environment
\newenvironment{ex}%
{%begin
\paragraph{\large \colorbox{skyblue}{Examples}}
\VerbatimEnvironment
\renewcommand{\EveryVerbatim}{\fontencoding{OT1}\selectfont}
\begin{quote}
\begin{Verbatim}
}%
{%end
\end{Verbatim}
\end{quote}
}
%
\definecolor{skyblue}{cmyk}{0.2, 0, 0.1, 0}
\definecolor{skyyellow}{cmyk}{0.1, 0.1, 0.5, 0}
%
%\title{NZMATH User Manual\\ {\large{(for version 1.0)}}}
%\date{}
%\author{}
\begin{document}
%\maketitle
%
\setcounter{tocdepth}{3}
\setcounter{secnumdepth}{3}


\tableofcontents
\C

\chapter{Classes}


%---------- start document ---------- %
 \section{poly.ratfunc -- rational function}\linkedzero{poly.ratfunc}
 \begin{itemize}
   \item {\bf Classes}
   \begin{itemize}
     \item \linkingone{poly.ratfunc}{RationalFunction}
   \end{itemize}
 \end{itemize}

 A rational function is a ratio of two polynomials.

 Please don't expect this module is useful.
 It just provides an acceptable container for polynomial division.
\C

 \subsection{RationalFunction -- rational function class}\linkedone{poly.ratfunc}{RationalFunction}
 \initialize
  \func{RationalFunction}{\hiki{numerator}{polynomial},\ \hikiopt{denominator}{polynomial}{1}}{\out{RationalFunction}}\\
  \spacing
  % document of basic document
  \quad Make a rational function with the given \param{numerator} and
  \param{denominator}.
  % added document
  %
  % \spacing
  % input, output document
  If the \param{numerator} is a {\tt RationalFunction} instance and
  \param{denominator} is not given, then make a copy.
  If the \param{numerator} is a kind of polynomial, then make a rational
  function whose numerator is the given polynomial.
  Additionally, if \param{denominator} is also given,
  the denominator is set to its values, otherwise the denominator is 1.
  \begin{at}
    \item[numerator]\linkedtwo{poly.ratfunc}{RationalFunction}{numerator}:\\
      polynomial.
    \item[denominator]\linkedtwo{poly.ratfunc}{RationalFunction}{denominator}:\\
      polynomial.
  \end{at}
  \begin{op}
    \verb+A==B+ & Return whether {\tt A} and {\tt B} are equal or not.\\
    \verb+str(A)+ & Return readable string.\\
    \verb+repr(A)+ & Return string representing {\tt A}'s structure.\\
  \end{op}
  \method
  \subsubsection{getRing -- get rational function field}\linkedtwo{poly.ratfunc}{RationalFunction}{getRing}
   \func{getRing}{\param{self}}{\linkingone{poly.ring}{RationalFunctionField}}\\
   \spacing
   % document of basic document
   \quad Return the rational function field to which the rational function belongs.
\C

%---------- end document ---------- %

\bibliographystyle{jplain}%use jbibtex
\bibliography{nzmath_references}

\end{document}


%%%%%%%%%%%%%%%%%%%%%%%%%%%%%%%%%%%%%%%%%%%%%%%%%%%%%%%%%%%%%%
%
% macros for nzmath manual
%
%%%%%%%%%%%%%%%%%%%%%%%%%%%%%%%%%%%%%%%%%%%%%%%%%%%%%%%%%%%%%
\usepackage{amssymb,amsmath}
\usepackage{color}
\usepackage[dvipdfm,bookmarks=true,bookmarksnumbered=true,%
 pdftitle={NZMATH Users Manual},%
 pdfsubject={Manual for NZMATH Users},%
 pdfauthor={NZMATH Development Group},%
 pdfkeywords={TeX; dvipdfmx; hyperref; color;},%
 colorlinks=true]{hyperref}
\usepackage{fancybox}
\usepackage[T1]{fontenc}
%
\newcommand{\DS}{\displaystyle}
\newcommand{\C}{\clearpage}
\newcommand{\NO}{\noindent}
\newcommand{\negok}{$\dagger$}
\newcommand{\spacing}{\vspace{1pt}\\ }
% software macros
\newcommand{\nzmathzero}{{\footnotesize $\mathbb{N}\mathbb{Z}$}\texttt{MATH}}
\newcommand{\nzmath}{{\nzmathzero}\ }
\newcommand{\pythonzero}{$\mbox{\texttt{Python}}$}
\newcommand{\python}{{\pythonzero}\ }
% link macros
\newcommand{\linkingzero}[1]{{\bf \hyperlink{#1}{#1}}}%module
\newcommand{\linkingone}[2]{{\bf \hyperlink{#1.#2}{#2}}}%module,class/function etc.
\newcommand{\linkingtwo}[3]{{\bf \hyperlink{#1.#2.#3}{#3}}}%module,class,method
\newcommand{\linkedzero}[1]{\hypertarget{#1}{}}
\newcommand{\linkedone}[2]{\hypertarget{#1.#2}{}}
\newcommand{\linkedtwo}[3]{\hypertarget{#1.#2.#3}{}}
\newcommand{\linktutorial}[1]{\href{http://docs.python.org/tutorial/#1}{#1}}
\newcommand{\linktutorialone}[2]{\href{http://docs.python.org/tutorial/#1}{#2}}
\newcommand{\linklibrary}[1]{\href{http://docs.python.org/library/#1}{#1}}
\newcommand{\linklibraryone}[2]{\href{http://docs.python.org/library/#1}{#2}}
\newcommand{\pythonhp}{\href{http://www.python.org/}{\python website}}
\newcommand{\nzmathwiki}{\href{http://nzmath.sourceforge.net/wiki/}{{\nzmathzero}Wiki}}
\newcommand{\nzmathsf}{\href{http://sourceforge.net/projects/nzmath/}{\nzmath Project Page}}
\newcommand{\nzmathtnt}{\href{http://tnt.math.se.tmu.ac.jp/nzmath/}{\nzmath Project Official Page}}
% parameter name
\newcommand{\param}[1]{{\tt #1}}
% function macros
\newcommand{\hiki}[2]{{\tt #1}:\ {\em #2}}
\newcommand{\hikiopt}[3]{{\tt #1}:\ {\em #2}=#3}

\newdimen\hoge
\newdimen\truetextwidth
\newcommand{\func}[3]{%
\setbox0\hbox{#1(#2)}
\hoge=\wd0
\truetextwidth=\textwidth
\advance \truetextwidth by -2\oddsidemargin
\ifdim\hoge<\truetextwidth % short form
{\bf \colorbox{skyyellow}{#1(#2)\ $\to$ #3}}
%
\else % long form
\fcolorbox{skyyellow}{skyyellow}{%
   \begin{minipage}{\textwidth}%
   {\bf #1(#2)\\ %
    \qquad\quad   $\to$\ #3}%
   \end{minipage}%
   }%
\fi%
}

\newcommand{\out}[1]{{\em #1}}
\newcommand{\initialize}{%
  \paragraph{\large \colorbox{skyblue}{Initialize (Constructor)}}%
    \quad\\ %
    \vspace{3pt}\\
}
\newcommand{\method}{\C \paragraph{\large \colorbox{skyblue}{Methods}}}
% Attribute environment
\newenvironment{at}
{%begin
\paragraph{\large \colorbox{skyblue}{Attribute}}
\quad\\
\begin{description}
}%
{%end
\end{description}
}
% Operation environment
\newenvironment{op}
{%begin
\paragraph{\large \colorbox{skyblue}{Operations}}
\quad\\
\begin{table}[h]
\begin{center}
\begin{tabular}{|l|l|}
\hline
operator & explanation\\
\hline
}%
{%end
\hline
\end{tabular}
\end{center}
\end{table}
}
% Examples environment
\newenvironment{ex}%
{%begin
\paragraph{\large \colorbox{skyblue}{Examples}}
\VerbatimEnvironment
\renewcommand{\EveryVerbatim}{\fontencoding{OT1}\selectfont}
\begin{quote}
\begin{Verbatim}
}%
{%end
\end{Verbatim}
\end{quote}
}
%
\definecolor{skyblue}{cmyk}{0.2, 0, 0.1, 0}
\definecolor{skyyellow}{cmyk}{0.1, 0.1, 0.5, 0}
%
%\title{NZMATH User Manual\\ {\large{(for version 1.0)}}}
%\date{}
%\author{}
\begin{document}
%\maketitle
%
\setcounter{tocdepth}{3}
\setcounter{secnumdepth}{3}


\tableofcontents
\C

\chapter{Classes}


%---------- start document ---------- %
 \section{poly.ring -- ��������}\linkedzero{poly.ring}
 \begin{itemize}
   \item {\bf Classes}
   \begin{itemize}
     \item \linkingone{poly.ring}{PolynomialRing}
     \item \linkingone{poly.ring}{RationalFunctionField}
     \item \linkingone{poly.ring}{PolynomialIdeal}
   \end{itemize}
 \end{itemize}

\C

 \subsection{PolynomialRing -- ��������}\linkedone{poly.ring}{PolynomialRing}
 uni-/multivariate polynomial rings�̂��߂̃N���X.
 \linkingone{ring}{CommutativeRing}�̂��߂̃T�u�N���X.

 \initialize
 \func{PolynomialRing}{%
   \hiki{coeffring}{CommutativeRing},\ %
   \hikiopt{number\_of\_variables}{integer}{1}}{\out{PolynomialRing}}\\
 \spacing
 \quad \param{coeffring}�͌W����.
 \param{number\_of\_variables}�͕ϐ��̐�.
 �������̒l�� \(1\) ���傫�����,���̊‚͑��ϐ��������ɑ΂������.
  \begin{at}
    \item[zero]\linkedtwo{poly.ring}{PolynomialRing}{zero}:\\ ���\(0\).
    \item[one]\linkedtwo{poly.ring}{PolynomialRing}{one}:\\ ���\(1\).
  \end{at}
 \method
 \subsubsection{getInstance -- �N���X���\�b�h}\linkedtwo{poly.ring}{PolynomialRing}{getInstance}
 \func{getInstance}{%
   \hiki{coeffring}{CommutativeRing},\ %
   \hiki{number\_of\_variables}{integer}}{\out{PolynomialRing}}\\
 \spacing
 \quad �W����\param{coeffring}�ƕϐ��̐�\param{number\_of\_variables}�����‘������‚̃C���X�^���X��Ԃ�.

 \subsubsection{getCoefficientRing}\linkedtwo{poly.ring}{PolynomialRing}{getCoefficientRing}
 \func{getCoefficientRing}{}{CommutativeRing}

 \subsubsection{getQuotientField}\linkedtwo{poly.ring}{PolynomialRing}{getQuotientField}
 \func{getQuotientField}{}{Field}

 \subsubsection{issubring}\linkedtwo{poly.ring}{PolynomialRing}{issubring}
 \func{issubring}{\hiki{other}{Ring}}{\out{bool}}

 \subsubsection{issuperring}\linkedtwo{poly.ring}{PolynomialRing}{issuperring}
 \func{issuperring}{\hiki{other}{Ring}}{\out{bool}}

 \subsubsection{getCharacteristic}\linkedtwo{poly.ring}{PolynomialRing}{getCharacteristic}
 \func{getCharacteristic}{}{\out{integer}}

 \subsubsection{createElement}\linkedtwo{poly.ring}{PolynomialRing}{createElement}
 \func{createElement}{seed}{\out{polynomial}}\\
 \quad ��������Ԃ�. \param{seed}�͑�����,
 �W���‚̌�,�܂��͈�ϐ�/���ϐ��������̍ŏ��̈����ɓK�������̃f�[�^�ł��蓾��.

 \subsubsection{gcd}\linkedtwo{poly.ring}{PolynomialRing}{gcd}
 \func{gcd}{a, b}{\out{polynomial}}\\
 \quad (�”\�Ȃ��)�^����ꂽ�������̍ő���񐔂�Ԃ�.
 �������͑������‚ɓ����Ă��Ȃ���΂Ȃ�Ȃ�.
 �����W���‚��̂Ȃ��,���̌��ʂ̓��j�b�N������.

 \subsubsection{isdomain}\linkedtwo{poly.ring}{PolynomialRing}{isdomain}
 \subsubsection{iseuclidean}\linkedtwo{poly.ring}{PolynomialRing}{iseuclidean}
 \subsubsection{isnoetherian}\linkedtwo{poly.ring}{PolynomialRing}{isnoetherian}
 \subsubsection{ispid}\linkedtwo{poly.ring}{PolynomialRing}{ispid}
 \subsubsection{isufd}\linkedtwo{poly.ring}{PolynomialRing}{isufd}

 \linkingone{ring}{CommutativeRing}����p�����ꂽ.

 \subsection{RationalFunctionField -- �L���֐���}\linkedone{poly.ring}{RationalFunctionField}
  \initialize
  \func{RationalFunctionField}{%
    \hiki{field}{Field},\ 
    \hiki{number\_of\_variables}{integer}}{%
    \out{RationalFunctionField}}\\
  \spacing
  \quad �L���֐��̂Ɋւ���N���X. 
  \linkingone{ring}{QuotientField}�̃T�u�N���X.\\
  \spacing
  \quad \param{field}��\linkingone{ring}{Field}�̃I�u�W�F�N�g�ł���ׂ��ł���W����.
  \param{number\_of\_variables}�͕ϐ��̐�.\\
  \spacing
  \begin{at}
    \item[zero]\linkedtwo{poly.ring}{RationalFunctionField}{zero}:\\ �̏��\(0\).
    \item[one]\linkedtwo{poly.ring}{RationalFunctionField}{one}:\\ �̏��\(1\).
  \end{at}
%
  \method
  \subsubsection{getInstance -- �N���X���\�b�h}\linkedtwo{poly.ring}{RationalFunctionField}{getInstance}
  \func{getInstance}{%
    \hiki{coefffield}{Field},\ %
    \hiki{number\_of\_variables}{integer}}{\out{RationalFunctionField}}\\
  \spacing
  \quad �W����\param{coefffield}�ƕϐ��̐�\param{number\_of\_variables}������{\tt RationalFunctionField}�̃C���X�^���X��Ԃ�.

  \subsubsection{createElement}\linkedtwo{poly.ring}{RationalFunctionField}{createElement}
  \func{createElement}{*\hiki{seedarg}{list}, **\hiki{seedkwd}{dict}}{\out{RationalFunction}}\\

  \subsubsection{getQuotientField}\linkedtwo{poly.ring}{RationalFunctionField}{getQuotientField}
  \func{getQuotientField}{}{\out{Field}}

  \subsubsection{issubring}\linkedtwo{poly.ring}{RationalFunctionField}{issubring}
  \func{issubring}{\hiki{other}{Ring}}{\out{bool}}\\

  \subsubsection{issuperring}\linkedtwo{poly.ring}{RationalFunctionField}{issuperring}
  \func{issuperring}{\hiki{other}{Ring}}{\out{bool}}\\

  \subsubsection{unnest}\linkedtwo{poly.ring}{RationalFunctionField}{unnest}
  \func{unnest}{}{\out{RationalFunctionField}}\\
  \spacing
  \quad ����self��{\tt RationalFunctionField}�Ƀl�X�g����Ă�����,���Ȃ킿���̌W���̂��܂�{\tt RationalFunctionField}�Ȃ�,���\�b�h�͈�i�K�A���l�X�g���ꂽ{\tt RationalFunctionField}��Ԃ�.\\
  \quad �Ⴆ��:
\begin{ex}
>>> RationalFunctionField(RationalFunctionField(Q, 1), 1).unnest()
RationalFunctionField(Q, 2)
\end{ex}

  \subsubsection{gcd}\linkedtwo{poly.ring}{RationalFunctionField}{gcd}
  \func{gcd}{\hiki{a}{RationalFunction},\ \hiki{b}{RationalFunction}}{%
  \out{RationalFunction}}\\
  \spacing
  \quad  \linkingone{ring}{Field}����p�������.

 \subsubsection{isdomain}\linkedtwo{poly.ring}{RationalFunctionField}{isdomain}
 \subsubsection{iseuclidean}\linkedtwo{poly.ring}{RationalFunctionField}{iseuclidean}
 \subsubsection{isnoetherian}\linkedtwo{poly.ring}{RationalFunctionField}{isnoetherian}
 \subsubsection{ispid}\linkedtwo{poly.ring}{RationalFunctionField}{ispid}
 \subsubsection{isufd}\linkedtwo{poly.ring}{RationalFunctionField}{isufd}

 \linkingone{ring}{CommutativeRing}����p�������.

  \subsection{PolynomialIdeal -- �������‚̃C�f�A��}\linkedone{poly.ring}{PolynomialIdeal}

  �������‚̃C�f�A����\��\linkingone{ring}{Ideal}�̃T�u�N���X.
  
  \initialize
  \func{PolynomialIdeal}{%
    \hiki{generators}{list},\ %
    \hiki{polyring}{PolynomialRing}}{\out{PolynomialIdeal}}\\
  \spacing
  \quad \param{generators}�ɂ���Đ�������鑽������\param{polyring}�̃C�f�A����\���V�����I�u�W�F�N�g���쐬.
  \begin{op}
    \verb/in/ & �܂܂�Ă��邩�̃e�X�g\\
    \verb/==/ & �����C�f�A����?\\
    \verb/!=/ & �قȂ�C�f�A����?\\
    \verb/+/ & �a\\
    \verb/*/ & ��\\
  \end{op}
  \method
  \subsubsection{reduce}\linkedtwo{poly.ring}{PolynomialIdeal}{reduce}
  \func{reduce}{\hiki{element}{polynomial}}{\out{polynomial}}\\
  \spacing
  \quad �C�f�A����@�Ƃ���\param{element}�̏�].

  \subsubsection{issubset}\linkedtwo{poly.ring}{PolynomialIdeal}{issubset}
  \func{issubset}{\hiki{other}{set}}{\out{bool}}\\

  \subsubsection{issuperset}\linkedtwo{poly.ring}{PolynomialIdeal}{issuperset}
  \func{issuperset}{\hiki{other}{set}}{\out{bool}}\\

\C

%---------- end document ---------- %

\bibliographystyle{jplain}%use jbibtex
\bibliography{nzmath_references}

\end{document}


%%%%%%%%%%%%%%%%%%%%%%%%%%%%%%%%%%%%%%%%%%%%%%%%%%%%%%%%%%%%%%
%
% macros for nzmath manual
%
%%%%%%%%%%%%%%%%%%%%%%%%%%%%%%%%%%%%%%%%%%%%%%%%%%%%%%%%%%%%%
\usepackage{amssymb,amsmath}
\usepackage{color}
\usepackage[dvipdfm,bookmarks=true,bookmarksnumbered=true,%
 pdftitle={NZMATH Users Manual},%
 pdfsubject={Manual for NZMATH Users},%
 pdfauthor={NZMATH Development Group},%
 pdfkeywords={TeX; dvipdfmx; hyperref; color;},%
 colorlinks=true]{hyperref}
\usepackage{fancybox}
\usepackage[T1]{fontenc}
%
\newcommand{\DS}{\displaystyle}
\newcommand{\C}{\clearpage}
\newcommand{\NO}{\noindent}
\newcommand{\negok}{$\dagger$}
\newcommand{\spacing}{\vspace{1pt}\\ }
% software macros
\newcommand{\nzmathzero}{{\footnotesize $\mathbb{N}\mathbb{Z}$}\texttt{MATH}}
\newcommand{\nzmath}{{\nzmathzero}\ }
\newcommand{\pythonzero}{$\mbox{\texttt{Python}}$}
\newcommand{\python}{{\pythonzero}\ }
% link macros
\newcommand{\linkingzero}[1]{{\bf \hyperlink{#1}{#1}}}%module
\newcommand{\linkingone}[2]{{\bf \hyperlink{#1.#2}{#2}}}%module,class/function etc.
\newcommand{\linkingtwo}[3]{{\bf \hyperlink{#1.#2.#3}{#3}}}%module,class,method
\newcommand{\linkedzero}[1]{\hypertarget{#1}{}}
\newcommand{\linkedone}[2]{\hypertarget{#1.#2}{}}
\newcommand{\linkedtwo}[3]{\hypertarget{#1.#2.#3}{}}
\newcommand{\linktutorial}[1]{\href{http://docs.python.org/tutorial/#1}{#1}}
\newcommand{\linktutorialone}[2]{\href{http://docs.python.org/tutorial/#1}{#2}}
\newcommand{\linklibrary}[1]{\href{http://docs.python.org/library/#1}{#1}}
\newcommand{\linklibraryone}[2]{\href{http://docs.python.org/library/#1}{#2}}
\newcommand{\pythonhp}{\href{http://www.python.org/}{\python website}}
\newcommand{\nzmathwiki}{\href{http://nzmath.sourceforge.net/wiki/}{{\nzmathzero}Wiki}}
\newcommand{\nzmathsf}{\href{http://sourceforge.net/projects/nzmath/}{\nzmath Project Page}}
\newcommand{\nzmathtnt}{\href{http://tnt.math.se.tmu.ac.jp/nzmath/}{\nzmath Project Official Page}}
% parameter name
\newcommand{\param}[1]{{\tt #1}}
% function macros
\newcommand{\hiki}[2]{{\tt #1}:\ {\em #2}}
\newcommand{\hikiopt}[3]{{\tt #1}:\ {\em #2}=#3}

\newdimen\hoge
\newdimen\truetextwidth
\newcommand{\func}[3]{%
\setbox0\hbox{#1(#2)}
\hoge=\wd0
\truetextwidth=\textwidth
\advance \truetextwidth by -2\oddsidemargin
\ifdim\hoge<\truetextwidth % short form
{\bf \colorbox{skyyellow}{#1(#2)\ $\to$ #3}}
%
\else % long form
\fcolorbox{skyyellow}{skyyellow}{%
   \begin{minipage}{\textwidth}%
   {\bf #1(#2)\\ %
    \qquad\quad   $\to$\ #3}%
   \end{minipage}%
   }%
\fi%
}

\newcommand{\out}[1]{{\em #1}}
\newcommand{\initialize}{%
  \paragraph{\large \colorbox{skyblue}{Initialize (Constructor)}}%
    \quad\\ %
    \vspace{3pt}\\
}
\newcommand{\method}{\C \paragraph{\large \colorbox{skyblue}{Methods}}}
% Attribute environment
\newenvironment{at}
{%begin
\paragraph{\large \colorbox{skyblue}{Attribute}}
\quad\\
\begin{description}
}%
{%end
\end{description}
}
% Operation environment
\newenvironment{op}
{%begin
\paragraph{\large \colorbox{skyblue}{Operations}}
\quad\\
\begin{table}[h]
\begin{center}
\begin{tabular}{|l|l|}
\hline
operator & explanation\\
\hline
}%
{%end
\hline
\end{tabular}
\end{center}
\end{table}
}
% Examples environment
\newenvironment{ex}%
{%begin
\paragraph{\large \colorbox{skyblue}{Examples}}
\VerbatimEnvironment
\renewcommand{\EveryVerbatim}{\fontencoding{OT1}\selectfont}
\begin{quote}
\begin{Verbatim}
}%
{%end
\end{Verbatim}
\end{quote}
}
%
\definecolor{skyblue}{cmyk}{0.2, 0, 0.1, 0}
\definecolor{skyyellow}{cmyk}{0.1, 0.1, 0.5, 0}
%
%\title{NZMATH User Manual\\ {\large{(for version 1.0)}}}
%\date{}
%\author{}
\begin{document}
%\maketitle
%
\setcounter{tocdepth}{3}
\setcounter{secnumdepth}{3}


\tableofcontents
\C

\chapter{Classes}


%---------- start document ---------- %
 \section{poly.termorder -- term orders}\linkedzero{poly.termorder}
 \begin{itemize}
   \item {\bf Classes}
     \begin{itemize}
     \item \negok \linkingone{poly.termorder}{TermOrderInterface}
     \item \negok \linkingone{poly.termorder}{UnivarTermOrder}
     \item \linkingone{poly.termorder}{MultivarTermOrder}
     \end{itemize}
   \item {\bf Functions}
     \begin{itemize}
       \item \linkingone{poly.termorder}{weight\_order}
     \end{itemize}
 \end{itemize}

\C

 \subsection{TermOrderInterface -- interface of term order}\linkedone{poly.termorder}{TermOrderInterface}
 \initialize
  \func{TermOrderInterface}{\hiki{comparator}{function}}{\out{TermOrderInterface}}\\
  \spacing
  \quad A term order is primarily a function, which determines
  precedence between two terms (or monomials). By the precedence, all
  terms are ordered.\\
  \quad More precisely in terms of \python, a term order accepts two
  tuples of integers, each of which represents power indices of the
  term, and returns 0, 1 or -1 just like \linklibraryone{stdfuncs\#cmp}{cmp}
  built-in function.\\
  \quad A {\tt TermOrder} object provides not only the precedence function,
  but also a method to format a string for a polynomial, to tell
  degree, leading coefficients, etc.\\
  \spacing
  \quad \param{comparator} accepts two tuple-like objects of integers,
  each of which represents power indices of the term, and returns 0, 1
  or -1 just like {\tt cmp} built-in function.\\
  \spacing
  This class is abstract and should not be instantiated.
  The methods below have to be overridden.
  \method
  \subsubsection{cmp}\linkedtwo{poly.termorder}{TermOrderInterface}{cmp}
  \func{cmp}{\param{self},\ \hiki{left}{tuple},\ \hiki{right}{tuple}}{\out{integer}}\\
  \spacing
  \quad Compare two index tuples \param{left} and \param{right} and
  determine precedence.

  \subsubsection{format}\linkedtwo{poly.termorder}{TermOrderInterface}{format}
  \func{format}{\param{self},\ \hiki{polynom}{polynomial},\ **\hiki{keywords}{dict}}{\out{string}}\\
  \spacing
  \quad Return the formatted string of the polynomial \param{polynom}.

  \subsubsection{leading\_coefficient}\linkedtwo{poly.termorder}{TermOrderInterface}{leading\_coefficient}
  \func{leading\_coefficient}{\param{self},\ \hiki{polynom}{polynomial}}{\out{CommutativeRingElement}}\\
  \spacing
  \quad Return the leading coefficient of polynomial \param{polynom}
  with respect to the term order.

  \subsubsection{leading\_term}\linkedtwo{poly.termorder}{TermOrderInterface}{leading\_term}
  \func{leading\_term}{\param{self},\ \hiki{polynom}{polynomial}}{\out{tuple}}\\
  \spacing
  \quad Return the leading term of polynomial \param{polynom} as tuple
  of {\tt (degree index, coefficient)} with respect to the term order.

  \subsection{UnivarTermOrder -- term order for univariate polynomials}\linkedone{poly.termorder}{UnivarTermOrder}
  \initialize
  \func{UnivarTermOrder}{\hiki{comparator}{function}}{\out{UnivarTermOrder}}\\
  \spacing
  \quad There is one unique term order for univariate polynomials.
  It's known as degree.\\
  \quad One thing special to univariate case is that powers are not tuples
  but bare integers.
  According to the fact, method signatures also need be translated from
  the definitions in TermOrderInterface, but its easy, and we omit
  some explanations.\\
  \spacing
  \quad \param{comparator} can be any callable that accepts two integers
  and returns 0, 1 or -1 just like {\tt cmp}, i.e. if they are equal
  it returns 0, first one is greater 1, and otherwise -1.
  Theoretically acceptable comparator is only the {\tt cmp} function.\\
  \spacing
  \quad This class inherits \linkingone{poly.termorder}{TermOrderInterface}.
  \method
  \subsubsection{format}\linkedtwo{poly.termorder}{UnivarTermOrder}{format}
  \func{format}{\param{self},\ \hiki{polynom}{polynomial},\ %
    \hikiopt{varname}{string}{'X'}, \hikiopt{reverse}{bool}{False}}{%
    \out{string}}\\
  \spacing
  \quad Return the formatted string of the polynomial \param{polynom}.\\
  \spacing
  \begin{itemize}
  \item \param{polynom} must be a univariate polynomial.
  \item \param{varname} can be set to the name of the variable.
  \item \param{reverse} can be either {\tt True} or {\tt False}.
    If it's {\tt True}, terms appear in reverse (descending) order.
  \end{itemize}

  \subsubsection{degree}\linkedtwo{poly.termorder}{UnivarTermOrder}{degree}
  \func{degree}{\param{self},\ \hiki{polynom}{polynomial}}{\out{integer}}\\
  \spacing
  \quad Return the degree of the polynomial \param{polynom}.

  \subsubsection{tail\_degree}\linkedtwo{poly.termorder}{UnivarTermOrder}{tail\_degree}
  \func{tail\_degree}{\param{self},\ \hiki{polynom}{polynomial}}{\out{integer}}\\
  \spacing
  \quad Return the least degree among all terms of the \param{polynom}.\\
  \spacing
  \quad This method is {\em experimental}.

  \subsection{MultivarTermOrder -- term order for multivariate polynomials}\linkedone{poly.termorder}{MultivarTermOrder}
  \initialize
  \func{MultivarTermOrder}{\hiki{comparator}{function}}{\out{MultivarTermOrder}}\\
  \spacing
  \quad This class inherits \linkingone{poly.termorder}{TermOrderInterface}.

  \method
  \subsubsection{format}\linkedtwo{poly.termorder}{MultivarTermOrder}{format}
  \func{format}{\param{self},\ \hiki{polynom}{polynomial},\ %
    \hikiopt{varname}{tuple}{None}, \hikiopt{reverse}{bool}{False},\ %
    **\hiki{kwds}{dict}}{%
    \out{string}}\\
  \spacing
  \quad Return the formatted string of the polynomial \param{polynom}.\\
  \spacing
  \quad  An additional argument \param{varnames} is required to name
  variables.\\

  \begin{itemize}
  \item \param{polynom} is a multivariate polynomial.
  \item \param{varnames} is the sequence of the variable names.
  \item \param{reverse} can be either {\tt True} or {\tt False}.
    If it's {\tt True}, terms appear in reverse (descending) order.
  \end{itemize}

  \subsection{weight\_order -- weight order}\linkedone{poly.termorder}{weight\_order}
  \func{weight\_order}{\hiki{weight}{sequence},\ \hikiopt{tie\_breaker}{function}{None}}{\out{function}}\\
  \spacing
   % document of basic document
   \quad Return a comparator of weight ordering by \param{weight}.\\
   \spacing
   \quad Let \(w\) denote the \param{weight}.
   The weight ordering is defined for arguments \(x\) and \(y\) that
   \(x < y\) if \(w\cdot x < w\cdot y\) or \(w\cdot x == w\cdot y\) and 
   tie breaker tells \(x < y\).\\
   \spacing
   \quad The option \param{tie\_breaker} is another comparator that will
   be used if dot products with the weight vector leaves arguments tie.
   If the option is {\tt None} (default) and a tie breaker is indeed necessary
   to order given arguments, a {\tt TypeError} is raised.

\begin{ex}
>>> w = termorder.MultivarTermOrder(
...     termorder.weight_order((6, 3, 1), cmp))
>>> w.cmp((1, 0, 0), (0, 1, 2))
1
\end{ex}%Don't indent!
\C

%---------- end document ---------- %

\bibliographystyle{jplain}%use jbibtex
\bibliography{nzmath_references}

\end{document}


%\documentclass{report}

%%%%%%%%%%%%%%%%%%%%%%%%%%%%%%%%%%%%%%%%%%%%%%%%%%%%%%%%%%%%%
%
% macros for nzmath manual
%
%%%%%%%%%%%%%%%%%%%%%%%%%%%%%%%%%%%%%%%%%%%%%%%%%%%%%%%%%%%%%
\usepackage{amssymb,amsmath}
\usepackage{color}
\usepackage[dvipdfm,bookmarks=true,bookmarksnumbered=true,%
 pdftitle={NZMATH Users Manual},%
 pdfsubject={Manual for NZMATH Users},%
 pdfauthor={NZMATH Development Group},%
 pdfkeywords={TeX; dvipdfmx; hyperref; color;},%
 colorlinks=true]{hyperref}
\usepackage{fancybox}
\usepackage[T1]{fontenc}
%
\newcommand{\DS}{\displaystyle}
\newcommand{\C}{\clearpage}
\newcommand{\NO}{\noindent}
\newcommand{\negok}{$\dagger$}
\newcommand{\spacing}{\vspace{1pt}\\ }
% software macros
\newcommand{\nzmathzero}{{\footnotesize $\mathbb{N}\mathbb{Z}$}\texttt{MATH}}
\newcommand{\nzmath}{{\nzmathzero}\ }
\newcommand{\pythonzero}{$\mbox{\texttt{Python}}$}
\newcommand{\python}{{\pythonzero}\ }
% link macros
\newcommand{\linkingzero}[1]{{\bf \hyperlink{#1}{#1}}}%module
\newcommand{\linkingone}[2]{{\bf \hyperlink{#1.#2}{#2}}}%module,class/function etc.
\newcommand{\linkingtwo}[3]{{\bf \hyperlink{#1.#2.#3}{#3}}}%module,class,method
\newcommand{\linkedzero}[1]{\hypertarget{#1}{}}
\newcommand{\linkedone}[2]{\hypertarget{#1.#2}{}}
\newcommand{\linkedtwo}[3]{\hypertarget{#1.#2.#3}{}}
\newcommand{\linktutorial}[1]{\href{http://docs.python.org/tutorial/#1}{#1}}
\newcommand{\linktutorialone}[2]{\href{http://docs.python.org/tutorial/#1}{#2}}
\newcommand{\linklibrary}[1]{\href{http://docs.python.org/library/#1}{#1}}
\newcommand{\linklibraryone}[2]{\href{http://docs.python.org/library/#1}{#2}}
\newcommand{\pythonhp}{\href{http://www.python.org/}{\python website}}
\newcommand{\nzmathwiki}{\href{http://nzmath.sourceforge.net/wiki/}{{\nzmathzero}Wiki}}
\newcommand{\nzmathsf}{\href{http://sourceforge.net/projects/nzmath/}{\nzmath Project Page}}
\newcommand{\nzmathtnt}{\href{http://tnt.math.se.tmu.ac.jp/nzmath/}{\nzmath Project Official Page}}
% parameter name
\newcommand{\param}[1]{{\tt #1}}
% function macros
\newcommand{\hiki}[2]{{\tt #1}:\ {\em #2}}
\newcommand{\hikiopt}[3]{{\tt #1}:\ {\em #2}=#3}

\newdimen\hoge
\newdimen\truetextwidth
\newcommand{\func}[3]{%
\setbox0\hbox{#1(#2)}
\hoge=\wd0
\truetextwidth=\textwidth
\advance \truetextwidth by -2\oddsidemargin
\ifdim\hoge<\truetextwidth % short form
{\bf \colorbox{skyyellow}{#1(#2)\ $\to$ #3}}
%
\else % long form
\fcolorbox{skyyellow}{skyyellow}{%
   \begin{minipage}{\textwidth}%
   {\bf #1(#2)\\ %
    \qquad\quad   $\to$\ #3}%
   \end{minipage}%
   }%
\fi%
}

\newcommand{\out}[1]{{\em #1}}
\newcommand{\initialize}{%
  \paragraph{\large \colorbox{skyblue}{Initialize (Constructor)}}%
    \quad\\ %
    \vspace{3pt}\\
}
\newcommand{\method}{\C \paragraph{\large \colorbox{skyblue}{Methods}}}
% Attribute environment
\newenvironment{at}
{%begin
\paragraph{\large \colorbox{skyblue}{Attribute}}
\quad\\
\begin{description}
}%
{%end
\end{description}
}
% Operation environment
\newenvironment{op}
{%begin
\paragraph{\large \colorbox{skyblue}{Operations}}
\quad\\
\begin{table}[h]
\begin{center}
\begin{tabular}{|l|l|}
\hline
operator & explanation\\
\hline
}%
{%end
\hline
\end{tabular}
\end{center}
\end{table}
}
% Examples environment
\newenvironment{ex}%
{%begin
\paragraph{\large \colorbox{skyblue}{Examples}}
\VerbatimEnvironment
\renewcommand{\EveryVerbatim}{\fontencoding{OT1}\selectfont}
\begin{quote}
\begin{Verbatim}
}%
{%end
\end{Verbatim}
\end{quote}
}
%
\definecolor{skyblue}{cmyk}{0.2, 0, 0.1, 0}
\definecolor{skyyellow}{cmyk}{0.1, 0.1, 0.5, 0}
%
%\title{NZMATH User Manual\\ {\large{(for version 1.0)}}}
%\date{}
%\author{}
\begin{document}
%\maketitle
%
\setcounter{tocdepth}{3}
\setcounter{secnumdepth}{3}


\tableofcontents
\C

\chapter{Classes}


%---------- start document ---------- %
 \section{poly.uniutil -- univariate utilities}\linkedzero{poly.uniutil}
 \begin{itemize}
   \item {\bf Classes}
   \begin{itemize}
     \item \linkingone{poly.uniutil}{RingPolynomial}
     \item \linkingone{poly.uniutil}{DomainPolynomial}
     \item \linkingone{poly.uniutil}{UniqueFactorizationDomainPolynomial}
     \item \linkingone{poly.uniutil}{IntegerPolynomial}
     \item \linkingone{poly.uniutil}{FieldPolynomial}
     \item \linkingone{poly.uniutil}{FinitePrimeFieldPolynomial}
     \item OrderProvider
     \item DivisionProvider
     \item PseudoDivisionProvider
     \item ContentProvider
     \item SubresultantGcdProvider
     \item PrimeCharacteristicFunctionsProvider
     \item VariableProvider
     \item RingElementProvider
   \end{itemize}
   \item {\bf Functions}
     \begin{itemize}
       \item \linkingone{poly.uniutil}{polynomial}
     \end{itemize}
 \end{itemize}

\C

 \subsection{RingPolynomial -- polynomial over commutative ring}\linkedone{poly.uniutil}{RingPolynomial}

 \initialize
  \func{RingPolynomial}{%
    \hiki{coefficients}{terminit},\ %
    \hiki{coeffring}{CommutativeRing},\ %
    **\hiki{keywords}{dict}}{\out{RingPolynomial object}}\\
  \spacing
  % document of basic document
  \quad Initialize a polynomial over the given commutative ring \param{coeffring}.\\
  \spacing
  % added document
  \quad This class inherits from \linkingone{poly.univar}{SortedPolynomial},
  \linkingone{poly.uniutil}{OrderProvider} and \linkingone{poly.uniutil}{RingElementProvider}.\\
  \spacing
  % input, output document
  \quad The type of the \param{coefficients} is \linkingone{poly.formalsum}{terminit}.
  \param{coeffring} is an instance of descendant of \linkingone{ring}{CommutativeRing}.\\
  \method
  \subsubsection{getRing}\linkedtwo{poly.uniutil}{RingPolynomial}{getRing}
  \func{getRing}{\param{self}}{\out{Ring}}\\
  \spacing
  \quad Return an object of a subclass of {\tt Ring},
  to which the polynomial belongs.\\
  (This method overrides the definition in RingElementProvider)

  \subsubsection{getCoefficientRing}\linkedtwo{poly.uniutil}{RingPolynomial}{getCoefficientRing}
  \func{getCoefficientRing}{\param{self}}{\out{Ring}}\\
  \spacing
  \quad Return an object of a subclass of {\tt Ring},
  to which the all coefficients belong.\\
  (This method overrides the definition in RingElementProvider)

  \subsubsection{shift\_degree\_to}\linkedtwo{poly.uniutil}{RingPolynomial}{shift\_degree\_to}
  \func{shift\_degree\_to}{\param{self}, \hiki{degree}{integer}}{\out{polynomial}}\\
  \spacing
  \quad Return polynomial whose degree is the given \param{degree}.
  More precisely, let \(f(X) = a_0 + ... + a_n X^n\), then
  {\tt f.shift\_degree\_to(m)} returns:
  \begin{itemize}
  \item zero polynomial, if f is zero polynomial
  \item \(a_{n-m} + ... + a_n X^m\), if \(0 \leq m < n\)
  \item \(a_0 X^{m-n} + ... + a_n X^m\), if \(m \geq n\)
  \end{itemize}
  (This method is inherited from OrderProvider)

  \subsubsection{split\_at}\linkedtwo{poly.uniutil}{RingPolynomial}{split\_at}
  \func{split\_at}{\param{self},\ \hiki{degree}{integer}}{\out{polynomial}}\\
  \spacing
  \quad Return tuple of two polynomials, which are split at the
  given degree. The term of the given degree, if exists, belongs to
  the lower degree polynomial.\\
  (This method is inherited from OrderProvider)  

 \subsection{DomainPolynomial -- polynomial over domain}\linkedone{poly.uniutil}{DomainPolynomial}
 \initialize
  \func{DomainPolynomial}{%
    \hiki{coefficients}{terminit},\ %
    \hiki{coeffring}{CommutativeRing},\ %
    **\hiki{keywords}{dict}}{\out{DomainPolynomial object}}\\
  \spacing
  % document of basic document
  \quad Initialize a polynomial over the given domain \param{coeffring}.\\
  \spacing
  % added document
  \quad In addition to the basic polynomial operations,
  it has pseudo division methods.\\
  \spacing
  \quad This class inherits \linkingone{poly.uniutil}{RingPolynomial} and
  \linkingone{poly.uniutil}{PseudoDivisionProvider}.
  \spacing
  % input, output document
  \quad The type of the \param{coefficients} is \linkingone{poly.formalsum}{terminit}.
  \param{coeffring} is an instance of descendant of \linkingone{ring}{CommutativeRing} which satisfies {\tt coeffring.isdomain()}.\\
  \spacing
  \method
  \subsubsection{pseudo\_divmod}\linkedtwo{poly.uniutil}{DomainPolynomial}{pseudo\_divmod}
  \func{pseudo\_divmod}{\param{self},\ \hiki{other}{polynomial}}{\out{tuple}}\\
  \spacing
  \quad Return a tuple {\tt (Q, R)},
  where \(Q\), \(R\) are polynomials such that:
  \[ d^{deg(f) - deg(other) + 1} f = other \times Q + R,\]
  where \(d\) is the leading coefficient of \param{other}.\\
  (This method is inherited from PseudoDivisionProvider)

  \subsubsection{pseudo\_floordiv}\linkedtwo{poly.uniutil}{DomainPolynomial}{pseudo\_floordiv}
  \func{pseudo\_floordiv}{\param{self},\ \hiki{other}{polynomial}}{\out{polynomial}}\\
  \spacing
  \quad Return a polynomial \(Q\) such that:
  \[ d^{deg(f) - deg(other) + 1} f = other \times Q + R,\]
  where \(d\) is the leading coefficient of \param{other}.\\
  (This method is inherited from PseudoDivisionProvider)

  \subsubsection{pseudo\_mod}\linkedtwo{poly.uniutil}{DomainPolynomial}{pseudo\_mod}
  \func{pseudo\_mod}{\param{self},\ \hiki{other}{polynomial}}{\out{polynomial}}\\
  \spacing
  \quad Return a polynomial \(R\) such that:
  \[ d^{deg(f) - deg(other) + 1} f = other \times Q + R,\]
  where \(d\) is the leading coefficient of \param{other}.\\
  (This method is inherited from PseudoDivisionProvider)

  \subsubsection{exact\_division}\linkedtwo{poly.uniutil}{DomainPolynomial}{exact\_division}
  \func{exact\_division}{\param{self},\ \hiki{other}{polynomial}}{\out{polynomial}}\\
  \spacing
  \quad Return quotient of exact division.\\
  (This method is inherited from PseudoDivisionProvider)

  \subsubsection{scalar\_exact\_division}\linkedtwo{poly.uniutil}{DomainPolynomial}{scalar\_exact\_division}
  \func{scalar\_exact\_division}{\param{self},\ \hiki{scale}{CommutativeRingElement}}{\out{polynomial}}\\
  \spacing
  \quad Return quotient by \param{scale} which can divide each
  coefficient exactly.\\
  (This method is inherited from PseudoDivisionProvider)

  \subsubsection{discriminant}\linkedtwo{poly.uniutil}{DomainPolynomial}{discriminant}
  \func{discriminant}{\param{self}}{\out{CommutativeRingElement}}\\
  \spacing
  \quad Return discriminant of the polynomial.

  \subsubsection{to\_field\_polynomial}\linkedtwo{poly.uniutil}{DomainPolynomial}{to\_field\_polynomial}
  \func{to\_field\_polynomial}{\param{self}}{\out{FieldPolynomial}}\\
  \spacing
  \quad Return a {\tt FieldPolynomial} object obtained by embedding
  the polynomial ring over the domain \(D\) to over the quotient
  field of \(D\).

 \subsection{UniqueFactorizationDomainPolynomial -- polynomial over UFD}\linkedone{poly.uniutil}{UniqueFactorizationDomainPolynomial}
 \initialize
  \func{UniqueFactorizationDomainPolynomial}{%
    \hiki{coefficients}{terminit},\ %
    \hiki{coeffring}{CommutativeRing},\ %
    **\hiki{keywords}{dict}}{\out{UniqueFactorizationDomainPolynomial object}}\\
  \spacing
  % document of basic document
  \quad Initialize a polynomial over the given UFD \param{coeffring}.\\
  \spacing
  % added document
  \quad This class inherits from \linkingone{poly.uniutil}{DomainPolynomial},
  \linkingone{poly.uniutil}{SubresultantGcdProvider} and \linkingone{poly.uniutil}{ContentProvider}.\\
  \spacing
  % input, output document
  \quad The type of the \param{coefficients} is \linkingone{poly.formalsum}{terminit}.
  \param{coeffring} is an instance of descendant of \linkingone{ring}{CommutativeRing} which satisfies {\tt coeffring.isufd()}.

  \subsubsection{content}\linkedtwo{poly.uniutil}{UniqueFactorizationDomainPolynomial}{content}
  \func{content}{\param{self}}{\out{CommutativeRingElement}}\\
  \spacing
  \quad Return content of the polynomial.\\
  (This method is inherited from ContentProvider)

  \subsubsection{primitive\_part}\linkedtwo{poly.uniutil}{UniqueFactorizationDomainPolynomial}{primitive\_part}
  \func{primitive\_part}{\param{self}}{\out{UniqueFactorizationDomainPolynomial}}\\
  \spacing
  \quad Return the primitive part of the polynomial.\\
  (This method is inherited from ContentProvider)

  \subsubsection{subresultant\_gcd}\linkedtwo{poly.uniutil}{UniqueFactorizationDomainPolynomial}{subresultant\_gcd}
  \func{subresultant\_gcd}{\param{self},\ \hiki{other}{polynomial}}{\out{UniqueFactorizationDomainPolynomial}}\\
  \spacing
  \quad Return the greatest common divisor of given polynomials.
  They must be in the polynomial ring and its coefficient ring must be a UFD.\\
  (This method is inherited from SubresultantGcdProvider)\\
  Reference: \cite{Cohen1}{Algorithm 3.3.1}

  \subsubsection{subresultant\_extgcd}\linkedtwo{poly.uniutil}{UniqueFactorizationDomainPolynomial}{subresultant\_extgcd}
  \func{subresultant\_extgcd}{\param{self},\ \hiki{other}{polynomial}}{\out{tuple}}\\
  \spacing
  \quad Return {\tt (A, B, P)} s.t. \(A\times self + B \times other=P\),
  where \(P\) is the greatest common divisor of given polynomials.
  They must be in the polynomial ring and its coefficient ring must be a UFD.\\
  Reference: \cite{Kida}{p.18}\\
  (This method is inherited from SubresultantGcdProvider)

  \subsubsection{resultant}\linkedtwo{poly.uniutil}{UniqueFactorizationDomainPolynomial}{resultant}
  \func{resultant}{\param{self}, \hiki{other}{polynomial}}{\out{polynomial}}\\
  \quad Return the resultant of \param{self} and \param{other}.\\
  (This method is inherited from SubresultantGcdProvider)

 \subsection{IntegerPolynomial -- polynomial over ring of rational integers}\linkedone{poly.uniutil}{IntegerPolynomial}
 \initialize
  \func{IntegerPolynomial}{%
    \hiki{coefficients}{terminit},\ %
    \hiki{coeffring}{CommutativeRing},\ %
    **\hiki{keywords}{dict}}{\out{IntegerPolynomial object}}\\
  \spacing
  % document of basic document
  \quad Initialize a polynomial over the given commutative ring \param{coeffring}.\\
  \spacing
  % added document
  \quad This class is required because special initialization must be
  done for built-in int/long.\\
  \spacing
  \quad This class inherits from \linkingone{poly.uniutil}{UniqueFactorizationDomainPolynomial}.\\
  \spacing
  % input, output document
  \quad The type of the \param{coefficients} is \linkingone{poly.formalsum}{terminit}.
  \param{coeffring} is an instance of \linkingone{rational}{IntegerRing}.
  You have to give the rational integer ring, though it seems redundant.

 \subsection{FieldPolynomial -- polynomial over field}\linkedone{poly.uniutil}{FieldPolynomial}
 \initialize
  \func{FieldPolynomial}{%
    \hiki{coefficients}{terminit},\ %
    \hiki{coeffring}{Field},\ %
    **\hiki{keywords}{dict}}{\out{FieldPolynomial object}}\\
  \spacing
  % document of basic document
  \quad Initialize a polynomial over the given field \param{coeffring}.\\
  \spacing
  % added document
  \quad Since the polynomial ring over field is a Euclidean domain,
  it provides divisions.\\
  \spacing
  \quad This class inherits from \linkingone{poly.uniutil}{RingPolynomial},
  \linkingone{poly.uniutil}{DivisionProvider} and \linkingone{poly.uniutil}{ContentProvider}.\\
  \spacing
  % input, output document
  \quad The type of the \param{coefficients} is \linkingone{poly.formalsum}{terminit}.
  \param{coeffring} is an instance of descendant of \linkingone{ring}{Field}.\\
%
  \begin{op}
    \verb+f // g+ & quotient of floor division\\
    \verb+f % g+ & remainder\\
    \verb+divmod(f, g)+ & quotient and remainder\\
    \verb+f / g+ & division in rational function field\\
  \end{op}
  \method

  \subsubsection{content}\linkedtwo{poly.uniutil}{FieldPolynomial}{content}
  \func{content}{\param{self}}{\out{FieldElement}}\\
  \spacing
  \quad Return content of the polynomial.\\
  (This method is inherited from ContentProvider)

  \subsubsection{primitive\_part}\linkedtwo{poly.uniutil}{FieldPolynomial}{primitive\_part}
  \func{primitive\_part}{\param{self}}{\out{polynomial}}\\
  \spacing
  \quad Return the primitive part of the polynomial.\\
  (This method is inherited from ContentProvider)

  \subsubsection{mod}\linkedtwo{poly.uniutil}{FieldPolynomial}{mod}
  \func{mod}{\param{self},\ \hiki{dividend}{polynomial}}{\out{polynomial}}\\
  \spacing
  \quad Return \(dividend \bmod self\).\\
  (This method is inherited from DivisionProvider)

  \subsubsection{scalar\_exact\_division}\linkedtwo{poly.uniutil}{FieldPolynomial}{scalar\_exact\_division}
  \func{scalar\_exact\_division}{\param{self},\ \hiki{scale}{FieldElement}}{\out{polynomial}}\\
  \spacing
  \quad Return quotient by \param{scale} which can divide each
  coefficient exactly.\\
  (This method is inherited from DivisionProvider)

  \subsubsection{gcd}\linkedtwo{poly.uniutil}{FieldPolynomial}{gcd}
  \func{gcd}{\param{self},\ \hiki{other}{polynomial}}{\out{polynomial}}\\
  \spacing
  \quad Return a greatest common divisor of self and other.\\
  \spacing
  \quad Returned polynomial is always monic.\\
  (This method is inherited from DivisionProvider)

  \subsubsection{extgcd}\linkedtwo{poly.uniutil}{FieldPolynomial}{extgcd}
  \func{extgcd}{\param{self},\ \hiki{other}{polynomial}}{\out{tuple}}\\
  \spacing
  \quad Return a tuple {\tt (u, v, d)}; they are the greatest common
  divisor \(d\) of two polynomials \param{self} and \param{other} and
  \(u\), \(v\) such that
  \[ d = self \times u + other \times v\]
  \spacing
  See \linkingone{gcd}{extgcd}.\\
  (This method is inherited from DivisionProvider)

 \subsection{FinitePrimeFieldPolynomial -- polynomial over finite prime field}\linkedone{poly.uniutil}{FinitePrimeFieldPolynomial}
 \initialize
  \func{FinitePrimeFieldPolynomial}{%
    \hiki{coefficients}{terminit},\ %
    \hiki{coeffring}{FinitePrimeField},\ %
    **\hiki{keywords}{dict}}{\out{FinitePrimeFieldPolynomial object}}\\
  \spacing
  % document of basic document
  \quad Initialize a polynomial over the given commutative ring \param{coeffring}.\\
  \spacing
  % added document
  \quad This class inherits from \linkingone{poly.uniutil}{FieldPolynomial} and
  \linkingone{poly.uniutil}{PrimeCharacteristicFunctionsProvider}.\\
  \spacing
  % input, output document
  \quad The type of the \param{coefficients} is \linkingone{poly.formalsum}{terminit}.
  \param{coeffring} is an instance of descendant of \linkingone{finitefield}{FinitePrimeField}.
  \method
  \subsubsection{mod\_pow -- powering with modulus}\linkedtwo{poly.uniutil}{FinitePrimeFieldPolynomial}{mod\_pow}
  \func{mod\_pow}{\param{self},\ \hiki{polynom}{polynomial}, \hiki{index}{integer}}{\out{polynomial}}\\
  \spacing
  \quad Return \(polynom ^ {index} \bmod self\).\\
  \spacing
  \quad Note that \param{self} is the modulus.\\
  (This method is inherited from PrimeCharacteristicFunctionsProvider)

  \subsubsection{pthroot}\linkedtwo{poly.uniutil}{FinitePrimeFieldPolynomial}{pthroot}
  \func{pthroot}{\param{self}}{\out{polynomial}}\\
  \spacing
  \quad Return a polynomial obtained by sending \(X^p\) to \(X\),
  where \(p\) is the characteristic. If the polynomial does not consist
  of \(p\)-th powered terms only, result is nonsense.\\
  (This method is inherited from PrimeCharacteristicFunctionsProvider)

  \subsubsection{squarefree\_decomposition}\linkedtwo{poly.uniutil}{FinitePrimeFieldPolynomial}{squarefree\_decomposition}
  \func{squarefree\_decomposition}{\param{self}}{\out{dict}}\\
  \spacing
  \quad Return the square free decomposition of the polynomial.\\
  \spacing
  \quad The return value is a dict whose keys are integers and values are
  corresponding powered factors.  For example, If
\begin{ex}
>>> A = A1 * A2**2
>>> A.squarefree_decomposition()
{1: A1, 2: A2}.
\end{ex}%Don't indent!
  (This method is inherited from PrimeCharacteristicFunctionsProvider)

  \subsubsection{distinct\_degree\_decomposition}\linkedtwo{poly.uniutil}{FinitePrimeFieldPolynomial}{distinct\_degree\_decomposition}
  \func{distinct\_degree\_decomposition}{\param{self}}{\out{dict}}\\
  \spacing
  \quad Return the distinct degree factorization of the polynomial.\\
  \spacing
  \quad The return value is a dict whose keys are integers and values are
  corresponding product of factors of the degree. For example, if
  \(A = A1 \times A2\), and all irreducible factors of \(A1\) having
  degree \(1\) and all irreducible factors of \(A2\) having degree \(2\),
  then the result is: {\tt \{1: A1, 2: A2\}}.\\

  \quad The given polynomial must be square free, and its coefficient
  ring must be a finite field.\\
  (This method is inherited from PrimeCharacteristicFunctionsProvider)

  \subsubsection{split\_same\_degrees}\linkedtwo{poly.uniutil}{FinitePrimeFieldPolynomial}{split\_same\_degrees}
  \func{split\_same\_degrees}{\param{self}, \hiki{degree}}{\out{list}}\\
  \spacing
  \quad Return the irreducible factors of the polynomial.\\
  \spacing
  \quad The polynomial must be a product of irreducible factors of
  the given degree.\\
  (This method is inherited from PrimeCharacteristicFunctionsProvider)

  \subsubsection{factor}\linkedtwo{poly.uniutil}{FinitePrimeFieldPolynomial}{factor}
  \func{factor}{\param{self}}{\out{list}}\\
  \spacing
  \quad Factor the polynomial.\\
  \spacing
  \quad The returned value is a list of tuples whose first component
  is a factor and second component is its multiplicity.\\
  (This method is inherited from PrimeCharacteristicFunctionsProvider)

  \subsubsection{isirreducible}\linkedtwo{poly.uniutil}{FinitePrimeFieldPolynomial}{isirreducible}
  \func{isirreducible}{\param{self}}{\out{bool}}\\
  \quad If the polynomial is irreducible return {\tt True},
  otherwise {\tt False}.\\
  (This method is inherited from PrimeCharacteristicFunctionsProvider)


  \subsection{polynomial -- factory function for various polynomials}\linkedone{poly.uniutil}{polynomial}
  \func{polynomial}{\hiki{coefficients}{terminit},\ \hiki{coeffring}{CommutativeRing}}{\out{polynomial}}\\
   \spacing
   % document of basic document
   \quad Return a polynomial.\\
   \spacing
   \quad \negok One can override the way to choose a polynomial type
   from a coefficient ring, by setting:\\
   {\tt special\_ring\_table[coeffring\_type] = polynomial\_type}\\
   before the function call.
\C

%---------- end document ---------- %

\bibliographystyle{jplain}%use jbibtex
\bibliography{nzmath_references}

\end{document}


%\documentclass{report}

%%%%%%%%%%%%%%%%%%%%%%%%%%%%%%%%%%%%%%%%%%%%%%%%%%%%%%%%%%%%%
%
% macros for nzmath manual
%
%%%%%%%%%%%%%%%%%%%%%%%%%%%%%%%%%%%%%%%%%%%%%%%%%%%%%%%%%%%%%
\usepackage{amssymb,amsmath}
\usepackage{color}
\usepackage[dvipdfm,bookmarks=true,bookmarksnumbered=true,%
 pdftitle={NZMATH Users Manual},%
 pdfsubject={Manual for NZMATH Users},%
 pdfauthor={NZMATH Development Group},%
 pdfkeywords={TeX; dvipdfmx; hyperref; color;},%
 colorlinks=true]{hyperref}
\usepackage{fancybox}
\usepackage[T1]{fontenc}
%
\newcommand{\DS}{\displaystyle}
\newcommand{\C}{\clearpage}
\newcommand{\NO}{\noindent}
\newcommand{\negok}{$\dagger$}
\newcommand{\spacing}{\vspace{1pt}\\ }
% software macros
\newcommand{\nzmathzero}{{\footnotesize $\mathbb{N}\mathbb{Z}$}\texttt{MATH}}
\newcommand{\nzmath}{{\nzmathzero}\ }
\newcommand{\pythonzero}{$\mbox{\texttt{Python}}$}
\newcommand{\python}{{\pythonzero}\ }
% link macros
\newcommand{\linkingzero}[1]{{\bf \hyperlink{#1}{#1}}}%module
\newcommand{\linkingone}[2]{{\bf \hyperlink{#1.#2}{#2}}}%module,class/function etc.
\newcommand{\linkingtwo}[3]{{\bf \hyperlink{#1.#2.#3}{#3}}}%module,class,method
\newcommand{\linkedzero}[1]{\hypertarget{#1}{}}
\newcommand{\linkedone}[2]{\hypertarget{#1.#2}{}}
\newcommand{\linkedtwo}[3]{\hypertarget{#1.#2.#3}{}}
\newcommand{\linktutorial}[1]{\href{http://docs.python.org/tutorial/#1}{#1}}
\newcommand{\linktutorialone}[2]{\href{http://docs.python.org/tutorial/#1}{#2}}
\newcommand{\linklibrary}[1]{\href{http://docs.python.org/library/#1}{#1}}
\newcommand{\linklibraryone}[2]{\href{http://docs.python.org/library/#1}{#2}}
\newcommand{\pythonhp}{\href{http://www.python.org/}{\python website}}
\newcommand{\nzmathwiki}{\href{http://nzmath.sourceforge.net/wiki/}{{\nzmathzero}Wiki}}
\newcommand{\nzmathsf}{\href{http://sourceforge.net/projects/nzmath/}{\nzmath Project Page}}
\newcommand{\nzmathtnt}{\href{http://tnt.math.se.tmu.ac.jp/nzmath/}{\nzmath Project Official Page}}
% parameter name
\newcommand{\param}[1]{{\tt #1}}
% function macros
\newcommand{\hiki}[2]{{\tt #1}:\ {\em #2}}
\newcommand{\hikiopt}[3]{{\tt #1}:\ {\em #2}=#3}

\newdimen\hoge
\newdimen\truetextwidth
\newcommand{\func}[3]{%
\setbox0\hbox{#1(#2)}
\hoge=\wd0
\truetextwidth=\textwidth
\advance \truetextwidth by -2\oddsidemargin
\ifdim\hoge<\truetextwidth % short form
{\bf \colorbox{skyyellow}{#1(#2)\ $\to$ #3}}
%
\else % long form
\fcolorbox{skyyellow}{skyyellow}{%
   \begin{minipage}{\textwidth}%
   {\bf #1(#2)\\ %
    \qquad\quad   $\to$\ #3}%
   \end{minipage}%
   }%
\fi%
}

\newcommand{\out}[1]{{\em #1}}
\newcommand{\initialize}{%
  \paragraph{\large \colorbox{skyblue}{Initialize (Constructor)}}%
    \quad\\ %
    \vspace{3pt}\\
}
\newcommand{\method}{\C \paragraph{\large \colorbox{skyblue}{Methods}}}
% Attribute environment
\newenvironment{at}
{%begin
\paragraph{\large \colorbox{skyblue}{Attribute}}
\quad\\
\begin{description}
}%
{%end
\end{description}
}
% Operation environment
\newenvironment{op}
{%begin
\paragraph{\large \colorbox{skyblue}{Operations}}
\quad\\
\begin{table}[h]
\begin{center}
\begin{tabular}{|l|l|}
\hline
operator & explanation\\
\hline
}%
{%end
\hline
\end{tabular}
\end{center}
\end{table}
}
% Examples environment
\newenvironment{ex}%
{%begin
\paragraph{\large \colorbox{skyblue}{Examples}}
\VerbatimEnvironment
\renewcommand{\EveryVerbatim}{\fontencoding{OT1}\selectfont}
\begin{quote}
\begin{Verbatim}
}%
{%end
\end{Verbatim}
\end{quote}
}
%
\definecolor{skyblue}{cmyk}{0.2, 0, 0.1, 0}
\definecolor{skyyellow}{cmyk}{0.1, 0.1, 0.5, 0}
%
%\title{NZMATH User Manual\\ {\large{(for version 1.0)}}}
%\date{}
%\author{}
\begin{document}
%\maketitle
%
\setcounter{tocdepth}{3}
\setcounter{secnumdepth}{3}


\tableofcontents
\C

\chapter{Classes}


%---------- start document ---------- %
 \section{poly.univar -- univariate polynomial}\linkedzero{poly.univar}
 \begin{itemize}
   \item {\bf Classes}
   \begin{itemize}
     \item \negok\linkingone{poly.univar}{PolynomialInterface}
     \item \negok\linkingone{poly.univar}{BasicPolynomial}
     \item \linkingone{poly.univar}{SortedPolynomial}
   \end{itemize}
 \end{itemize}

 This poly.univar using following type:
 \begin{description}
   \item[polynomial]\linkedone{poly.univar}{polynomial}:\\
     \param{polynomial} is an instance of some descendant class of
     \linkingone{poly.univar}{PolynomialInterface}
     in this context.
 \end{description}

\C

 \subsection{PolynomialInterface -- base class for all univariate polynomials}\linkedone{poly.univar}{PolynomialInterface}
 \initialize
  Since the interface is an abstract class, do not instantiate.\\
  The class is derived from \linkingone{poly.formalsum}{FormalSumContainerInterface}.
  \spacing
%Some of attributes may be treated as a public one.
%  \begin{at}
%  \end{at}
  \begin{op}
    \verb+f * g+ & multiplication\footnote{in FormalSumContainerInterface, there is only scalar multiplication}\\
    \verb+f ** i+ & powering\\
  \end{op} 
  \method
  \subsubsection{differentiate -- formal differentiation}\linkedtwo{poly.univar}{PolynomialInterface}{differentiate}
   \func{differentiate}{\param{self}}{\out{polynomial}}\\
   \spacing
   % document of basic document
   \quad Return the formal differentiation of this polynomial.
 \subsubsection{downshift\_degree -- decreased degree polynomial}\linkedtwo{poly.univar}{PolynomialInterface}{downshift\_degree}
   \func{downshift\_degree}{\param{self},\ \hiki{slide}{integer}}{\out{polynomial}}\\
   \spacing
   % document of basic document
   \quad Return the polynomial obtained by shifting downward all terms
   with degrees of \param{slide.}

   Be careful that if the least degree term has the degree less than
   \param{slide} then the result is not mathematically a
   polynomial. Even in such a case, the method does not raise an
   exception.\\
   \spacing
   % added document
   \quad \negok {\tt f.downshift\_degree(slide)} is equivalent to
   {\tt f.\linkingtwo{poly.univar}{PolynomialInterface}{upshift\_degree}(-slide)}.

 \subsubsection{upshift\_degree -- increased degree polynomial}\linkedtwo{poly.univar}{PolynomialInterface}{upshift\_degree}
   \func{upshift\_degree}{\param{self},\ \hiki{slide}{integer}}{\out{polynomial}}\\
   \spacing
   % document of basic document
   \quad Return the polynomial obtained by shifting upward all terms
   with degrees of \param{slide}.
   \spacing
   % added document
   \quad \negok {\tt f.upshift\_degree(slide)} is equivalent to
   {\tt f.term\_mul((slide, 1))}.

   \subsubsection{ring\_mul -- multiplication in the ring}\linkedtwo{poly.univar}{PolynomialInterface}{ring\_mul}
   \func{ring\_mul}{\param{self},\ \hiki{other}{polynomial}}{\out{polynomial}}\\
   \spacing
   % document of basic document
   Return the result of multiplication with the \param{other} polynomial.

   \subsubsection{scalar\_mul -- multiplication with a scalar}\linkedtwo{poly.univar}{PolynomialInterface}{scalar\_mul}
   \func{scalar\_mul}{\param{self},\ \hiki{scale}{scalar}}{\out{polynomial}}\\
   \spacing
   Return the result of multiplication by scalar \param{scale}.

   \subsubsection{term\_mul -- multiplication with a term}\linkedtwo{poly.univar}{PolynomialInterface}{term\_mul}
   \func{term\_mul}{\param{self},\ \hiki{term}{term}}{\out{polynomial}}\\
   \spacing
   Return the result of multiplication with the given \param{term}.
   The \param{term} can be given as a tuple {\tt (degree, coeff)} or as a {\tt polynomial}.

   \subsubsection{square -- multiplication with itself}\linkedtwo{poly.univar}{PolynomialInterface}{square}
   \func{square}{\param{self}}{\out{polynomial}}\\
   Return the square of this polynomial.

%
 \subsection{BasicPolynomial -- basic implementation of polynomial}\linkedone{poly.univar}{BasicPolynomial}
 Basic polynomial data type.
 There are no concept such as variable name and ring.

  \initialize
  \func{BasicPolynomial}{\hiki{coefficients}{terminit},\ %
    **\hiki{keywords}{dict}}{\out{BasicPolynomial}}\\
  \spacing
  \quad This class inherits and implements \linkingone{poly.univar}{PolynomialInterface}.
  \spacing
  \quad The type of the \param{coefficients} is \linkingone{poly.formalsum}{terminit}.

 \subsection{SortedPolynomial -- polynomial keeping terms sorted}\linkedone{poly.univar}{SortedPolynomial}
 \initialize
  \func{SortedPolynomial}{\hiki{coefficients}{terminit},\ %
    \hikiopt{\_sorted}{bool}{False},\ %
    **\hiki{keywords}{dict}}{\out{SortedPolynomial}}\\
  The class is derived from \linkingone{poly.univar}{PolynomialInterface}.
  \spacing
  \quad The type of the \param{coefficients} is \linkingone{poly.formalsum}{terminit}.
  Optionally \param{\_sorted} can be {\tt True} if the coefficients is an
  already sorted list of terms.

  \method
  \subsubsection{degree -- degree}\linkedtwo{poly.univar}{SortedPolynomial}{degree}
   \func{degree}{\param{self}}{\out{integer}}\\
   \spacing
   % document of basic document
   Return the degree of this polynomial.
   If the polynomial is the zero polynomial, the degree is \(-1\).

  \subsubsection{leading\_coefficient -- the leading coefficient}\linkedtwo{poly.univar}{SortedPolynomial}{leading\_coefficient}
  \func{leading\_coefficient}{\param{self}}{\out{object}}\\

  Return the coefficient of highest degree term.

  \subsubsection{leading\_term -- the leading term}\linkedtwo{poly.univar}{SortedPolynomial}{leading\_term}
  \func{leading\_term}{\param{self}}{\out{tuple}}\\

  Return the leading term as a tuple {\tt (degree, coefficient)}.

  \subsubsection{\negok ring\_mul\_karatsuba -- the leading term}\linkedtwo{poly.univar}{SortedPolynomial}{ring\_mul\_karatsuba}
  \func{ring\_mul\_karatsuba}{\param{self},\ \hiki{other}{polynomial}}{\out{polynomial}}\\

  Multiplication of two polynomials in the same ring.
  Computation is carried out by Karatsuba method.

  This may run faster when degree is higher than 100 or so.
  It is off by default, if you need to use this, do by yourself.
\C

%---------- end document ---------- %

\bibliographystyle{jplain}%use jbibtex
\bibliography{nzmath_references}

\end{document}


 
\bibliographystyle{jplain}%use jbibtex
\bibliography{nzmath_references}

\end{document}
