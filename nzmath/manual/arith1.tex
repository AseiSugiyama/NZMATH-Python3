\documentclass{report}

%%%%%%%%%%%%%%%%%%%%%%%%%%%%%%%%%%%%%%%%%%%%%%%%%%%%%%%%%%%%%
%
% macros for nzmath manual
%
%%%%%%%%%%%%%%%%%%%%%%%%%%%%%%%%%%%%%%%%%%%%%%%%%%%%%%%%%%%%%
\usepackage{amssymb,amsmath}
\usepackage{color}
\usepackage[dvipdfm,bookmarks=true,bookmarksnumbered=true,%
 pdftitle={NZMATH Users Manual},%
 pdfsubject={Manual for NZMATH Users},%
 pdfauthor={NZMATH Development Group},%
 pdfkeywords={TeX; dvipdfmx; hyperref; color;},%
 colorlinks=true]{hyperref}
\usepackage{fancybox}
\usepackage[T1]{fontenc}
%
\newcommand{\DS}{\displaystyle}
\newcommand{\C}{\clearpage}
\newcommand{\NO}{\noindent}
\newcommand{\negok}{$\dagger$}
\newcommand{\spacing}{\vspace{1pt}\\ }
% software macros
\newcommand{\nzmathzero}{{\footnotesize $\mathbb{N}\mathbb{Z}$}\texttt{MATH}}
\newcommand{\nzmath}{{\nzmathzero}\ }
\newcommand{\pythonzero}{$\mbox{\texttt{Python}}$}
\newcommand{\python}{{\pythonzero}\ }
% link macros
\newcommand{\linkingzero}[1]{{\bf \hyperlink{#1}{#1}}}%module
\newcommand{\linkingone}[2]{{\bf \hyperlink{#1.#2}{#2}}}%module,class/function etc.
\newcommand{\linkingtwo}[3]{{\bf \hyperlink{#1.#2.#3}{#3}}}%module,class,method
\newcommand{\linkedzero}[1]{\hypertarget{#1}{}}
\newcommand{\linkedone}[2]{\hypertarget{#1.#2}{}}
\newcommand{\linkedtwo}[3]{\hypertarget{#1.#2.#3}{}}
\newcommand{\linktutorial}[1]{\href{http://docs.python.org/tutorial/#1}{#1}}
\newcommand{\linktutorialone}[2]{\href{http://docs.python.org/tutorial/#1}{#2}}
\newcommand{\linklibrary}[1]{\href{http://docs.python.org/library/#1}{#1}}
\newcommand{\linklibraryone}[2]{\href{http://docs.python.org/library/#1}{#2}}
\newcommand{\pythonhp}{\href{http://www.python.org/}{\python website}}
\newcommand{\nzmathwiki}{\href{http://nzmath.sourceforge.net/wiki/}{{\nzmathzero}Wiki}}
\newcommand{\nzmathsf}{\href{http://sourceforge.net/projects/nzmath/}{\nzmath Project Page}}
\newcommand{\nzmathtnt}{\href{http://tnt.math.se.tmu.ac.jp/nzmath/}{\nzmath Project Official Page}}
% parameter name
\newcommand{\param}[1]{{\tt #1}}
% function macros
\newcommand{\hiki}[2]{{\tt #1}:\ {\em #2}}
\newcommand{\hikiopt}[3]{{\tt #1}:\ {\em #2}=#3}

\newdimen\hoge
\newdimen\truetextwidth
\newcommand{\func}[3]{%
\setbox0\hbox{#1(#2)}
\hoge=\wd0
\truetextwidth=\textwidth
\advance \truetextwidth by -2\oddsidemargin
\ifdim\hoge<\truetextwidth % short form
{\bf \colorbox{skyyellow}{#1(#2)\ $\to$ #3}}
%
\else % long form
\fcolorbox{skyyellow}{skyyellow}{%
   \begin{minipage}{\textwidth}%
   {\bf #1(#2)\\ %
    \qquad\quad   $\to$\ #3}%
   \end{minipage}%
   }%
\fi%
}

\newcommand{\out}[1]{{\em #1}}
\newcommand{\initialize}{%
  \paragraph{\large \colorbox{skyblue}{Initialize (Constructor)}}%
    \quad\\ %
    \vspace{3pt}\\
}
\newcommand{\method}{\C \paragraph{\large \colorbox{skyblue}{Methods}}}
% Attribute environment
\newenvironment{at}
{%begin
\paragraph{\large \colorbox{skyblue}{Attribute}}
\quad\\
\begin{description}
}%
{%end
\end{description}
}
% Operation environment
\newenvironment{op}
{%begin
\paragraph{\large \colorbox{skyblue}{Operations}}
\quad\\
\begin{table}[h]
\begin{center}
\begin{tabular}{|l|l|}
\hline
operator & explanation\\
\hline
}%
{%end
\hline
\end{tabular}
\end{center}
\end{table}
}
% Examples environment
\newenvironment{ex}%
{%begin
\paragraph{\large \colorbox{skyblue}{Examples}}
\VerbatimEnvironment
\renewcommand{\EveryVerbatim}{\fontencoding{OT1}\selectfont}
\begin{quote}
\begin{Verbatim}
}%
{%end
\end{Verbatim}
\end{quote}
}
%
\definecolor{skyblue}{cmyk}{0.2, 0, 0.1, 0}
\definecolor{skyyellow}{cmyk}{0.1, 0.1, 0.5, 0}
%
%\title{NZMATH User Manual\\ {\large{(for version 1.0)}}}
%\date{}
%\author{}
\begin{document}
%\maketitle
%
\setcounter{tocdepth}{3}
\setcounter{secnumdepth}{3}


\tableofcontents
\C

\chapter{Functions}

%---------- start document ---------- %
\section{arith1 - miscellaneous arithmetic functions}\linkedzero{arith1}

\subsection{floorsqrt -- floor of square root}\linkedone{arith1}{floorsqrt}
\func{floorsqrt}{\hiki{a}{integer/\linkingone{rational}{Rational}}}{\out{integer}}\\
\spacing
% document of basi document
\quad Return the floor of square root of \param{a}.\\ 
%\spacing
% input, output document
%\quad Input number \param{a} must be integer or \linkingone{rational}{Rational}.\\
%
\subsection{floorpowerroot -- floor of some power root}\linkedone{arith1}{floorpowerroot}
\func{floorpowerroot}{\hiki{n}{integer},\ \hiki{k}{integer}}{\out{integer}}\\
\spacing
% document of basi document
\quad Return the floor of \param{k}-th power root of \param{n}.\\
%\spacing
% input, output document
%\quad Input numbers \param{n}, \param{k} must be integer.\\
%
\subsection{legendre - Legendre(Jacobi) Symbol}\linkedone{arith1}{legendre}
\func{legendre}{\hiki{a}{integer},\ \hiki{m}{integer}}{\out{integer}}\\
\spacing
% document of basi document
\quad Return the Legendre symbol or Jacobi symbol $\DS \Bigl(\frac{\param{a}}{\param{m}}\Bigr)$.\\
%\spacing
% input, output document
%\quad Input numbers \param{a}, \param{m} must be integer.\\
%
\subsection{modsqrt -- square root of $a$ for modulo $p$}\linkedone{arith1}{modsqrt}
\func{modsqrt}{\hiki{a}{integer}, \, \hiki{p}{integer}}{\out{integer}}\\
\spacing
% document of basi document
\quad Return one of the square roots of \param{a} for modulo \param{p} if square roots are exist, raise ValueError otherwise.\\
\spacing
% add document
%\spacing
% input, output ducument
\quad \param{p} must be a prime number.\\
%
\subsection{expand -- p-adic expansion}\linkedone{arith1}{expand}
\func{expand}{\hiki{n}{integer}, \, \hiki{m}{integer}}{\out{list}}\\
\spacing
% document of basi document
\quad Return the \param{m}-adic expansion of \param{n}.\\ 
\spacing
% input, output document
\quad \param{n} must be nonnegative integer. \param{m} must be greater than or equal to $2$.  The output is a list of expansion coefficients in ascending order.\\
%
\subsection{inverse -- inverse}\linkedone{arith1}{inverse}
\func{inverse}{\hiki{x}{integer}, \, \hiki{p}{integer}}{\out{integer}}\\
\spacing
% document of basi document
\quad Return the inverse of \param{x} for modulo \param{p}.\\
\spacing
% input, output document
\quad \param{p} must be a prime number.\\
%
\subsection{CRT -- Chinese Reminder Theorem}\linkedone{arith1}{CRT}
\func{CRT}{\hiki{nlist}{list}}{\out{integer}}\\
\spacing
% document of basi document
\quad Return the uniquely determined integer satisfying all modulus
conditions given by \param{nlist}.\\
\spacing
% input, output document
\quad Input list \param{nlist} must be the list of a list consisting of two elements.
The first element is remainder and the second is divisor.
They must be integer.\\
%
\subsection{AGM -- Arithmetic Geometric Mean}\linkedone{arith1}{AGM}
\func{AGM}{\hiki{a}{integer},\ \hiki{b}{integer}}{\out{float}}\\
\spacing
% document of basi document
\quad Return the Arithmetic-Geometric Mean of \param{a} and \param{b}.\\
%\spacing
% input, output document
%\quad Input number \param{a}, \param{b} must be integer.\\ 
%
%\subsection{\_BhaskaraBrouncker}\linkedone{arith1}{\_BhaskaraBrouncker}
%\func{\_BhaskaraBrouncker}{\hiki{n}{integer}}{\out{integer}}\\
%\spacing
% document of basi document
%\quad Return the minimum tuple \param{p}, \param{q} such that, $\param{p}^2
%- \param{n} \param{q}^2 = \pm 1$.\\
%\spacing
% input, output document
%\quad Input number \param{n} must be positive integer.
%
\subsection{vp -- $p$-adic valuation}\linkedone{arith1}{vp}
\func{vp}{\hiki{n}{integer},\ \hiki{p}{integer}, \hikiopt{k}{integer}{0}}{\out{tuple}}\\
\spacing
% document of basi document
\quad Return the \param{p}-adic valuation and other part for \param{n}.\\
\spacing
% added document
\quad \negok If $k$ is given, return the valuation and the other part for $\param{n}p^\param{k}$.\\
% input, output document
%\quad Input number \param{n}, \param{p} must be int, long or \linkingone{rational}{Integer}.
%
\subsection{issquare - Is it square?}\linkedone{arith1}{issquare}
\func{issquare}{\hiki{n}{integer}}{\out{integer}}\\
\spacing
% document of basi document
\quad Check if \param{n} is a square number and return square root
of \param{n} if \param{n} is a square.
Otherwise, return \(0\).\\
%\spacing
% input, output document
%\quad Input number \param{n} must be int, long or \linkingone{rational}{Integer}.
%
\subsection{log -- integer part of logarithm}\linkedone{arith1}{log}
\func{log}{\hiki{n}{integer},\ \hikiopt{base}{integer}{2}}{\out{integer}}\\
\spacing
% document of basi document
\quad Return the integer part of logarithm of \param{n} to the \param{base}.\\
%\spacing
% input, output document
%\quad Input number \param{n}, \param{base} must be int, long or \linkingone{rational}{Integer}.
%
\subsection{product -- product of some numbers}\linkedone{arith1}{product}
\func{product}{\hiki{iterable}{list},\ \hikiopt{init}{integer/\linkingone{rational}{Rational}}{None}}{\out{\hiki{prod}{integer/\linkingone{rational}{Rational}}}}\\
\spacing
% document of basic document
\quad Return the products of all elements in \param{iterable}. \\
\spacing
% added document
\quad If \param{init} is given, the multiplication starts with \param{init} instead of the first element in \param{iterable}.\\
\spacing
% input, output document
\quad Input list \param{iterable} must be list of numbers including integers, \linkingone{rational}{Rational} etc.\\
The output \param{prod} may be determined by the type of elements of \param{iterable} and \param{init}.\\
%
\begin{ex}
>>> arith1.AGM(10, 15)
12.373402181181522
>>> arith1.CRT([[2, 5],[3,7]])
17
>>> arith1.CRT([[2, 5], [3, 7], [5, 11]])
192
>>> arith1.expand(194, 5)
[4, 3, 2, 1]
>>> arith1.vp(54, 3)
(3, 2)
>>> arith1.product([1.5, 2, 2.5])
7.5
>>> arith1.product([3, 4], 2)
24
>>> arith1.product([])
1
\end{ex}

%---------- end document ---------- %

%\bibliographystyle{jplain}
%\bibliography{nzamth_refereces}

\end{document}