\documentclass{report}

%%%%%%%%%%%%%%%%%%%%%%%%%%%%%%%%%%%%%%%%%%%%%%%%%%%%%%%%%%%%%
%
% macros for nzmath manual
%
%%%%%%%%%%%%%%%%%%%%%%%%%%%%%%%%%%%%%%%%%%%%%%%%%%%%%%%%%%%%%
\usepackage{amssymb,amsmath}
\usepackage{color}
\usepackage[dvipdfm,bookmarks=true,bookmarksnumbered=true,%
 pdftitle={NZMATH Users Manual},%
 pdfsubject={Manual for NZMATH Users},%
 pdfauthor={NZMATH Development Group},%
 pdfkeywords={TeX; dvipdfmx; hyperref; color;},%
 colorlinks=true]{hyperref}
\usepackage{fancybox}
\usepackage[T1]{fontenc}
%
\newcommand{\DS}{\displaystyle}
\newcommand{\C}{\clearpage}
\newcommand{\NO}{\noindent}
\newcommand{\negok}{$\dagger$}
\newcommand{\spacing}{\vspace{1pt}\\ }
% software macros
\newcommand{\nzmathzero}{{\footnotesize $\mathbb{N}\mathbb{Z}$}\texttt{MATH}}
\newcommand{\nzmath}{{\nzmathzero}\ }
\newcommand{\pythonzero}{$\mbox{\texttt{Python}}$}
\newcommand{\python}{{\pythonzero}\ }
% link macros
\newcommand{\linkingzero}[1]{{\bf \hyperlink{#1}{#1}}}%module
\newcommand{\linkingone}[2]{{\bf \hyperlink{#1.#2}{#2}}}%module,class/function etc.
\newcommand{\linkingtwo}[3]{{\bf \hyperlink{#1.#2.#3}{#3}}}%module,class,method
\newcommand{\linkedzero}[1]{\hypertarget{#1}{}}
\newcommand{\linkedone}[2]{\hypertarget{#1.#2}{}}
\newcommand{\linkedtwo}[3]{\hypertarget{#1.#2.#3}{}}
\newcommand{\linktutorial}[1]{\href{http://docs.python.org/tutorial/#1}{#1}}
\newcommand{\linktutorialone}[2]{\href{http://docs.python.org/tutorial/#1}{#2}}
\newcommand{\linklibrary}[1]{\href{http://docs.python.org/library/#1}{#1}}
\newcommand{\linklibraryone}[2]{\href{http://docs.python.org/library/#1}{#2}}
\newcommand{\pythonhp}{\href{http://www.python.org/}{\python website}}
\newcommand{\nzmathwiki}{\href{http://nzmath.sourceforge.net/wiki/}{{\nzmathzero}Wiki}}
\newcommand{\nzmathsf}{\href{http://sourceforge.net/projects/nzmath/}{\nzmath Project Page}}
\newcommand{\nzmathtnt}{\href{http://tnt.math.se.tmu.ac.jp/nzmath/}{\nzmath Project Official Page}}
% parameter name
\newcommand{\param}[1]{{\tt #1}}
% function macros
\newcommand{\hiki}[2]{{\tt #1}:\ {\em #2}}
\newcommand{\hikiopt}[3]{{\tt #1}:\ {\em #2}=#3}

\newdimen\hoge
\newdimen\truetextwidth
\newcommand{\func}[3]{%
\setbox0\hbox{#1(#2)}
\hoge=\wd0
\truetextwidth=\textwidth
\advance \truetextwidth by -2\oddsidemargin
\ifdim\hoge<\truetextwidth % short form
{\bf \colorbox{skyyellow}{#1(#2)\ $\to$ #3}}
%
\else % long form
\fcolorbox{skyyellow}{skyyellow}{%
   \begin{minipage}{\textwidth}%
   {\bf #1(#2)\\ %
    \qquad\quad   $\to$\ #3}%
   \end{minipage}%
   }%
\fi%
}

\newcommand{\out}[1]{{\em #1}}
\newcommand{\initialize}{%
  \paragraph{\large \colorbox{skyblue}{Initialize (Constructor)}}%
    \quad\\ %
    \vspace{3pt}\\
}
\newcommand{\method}{\C \paragraph{\large \colorbox{skyblue}{Methods}}}
% Attribute environment
\newenvironment{at}
{%begin
\paragraph{\large \colorbox{skyblue}{Attribute}}
\quad\\
\begin{description}
}%
{%end
\end{description}
}
% Operation environment
\newenvironment{op}
{%begin
\paragraph{\large \colorbox{skyblue}{Operations}}
\quad\\
\begin{table}[h]
\begin{center}
\begin{tabular}{|l|l|}
\hline
operator & explanation\\
\hline
}%
{%end
\hline
\end{tabular}
\end{center}
\end{table}
}
% Examples environment
\newenvironment{ex}%
{%begin
\paragraph{\large \colorbox{skyblue}{Examples}}
\VerbatimEnvironment
\renewcommand{\EveryVerbatim}{\fontencoding{OT1}\selectfont}
\begin{quote}
\begin{Verbatim}
}%
{%end
\end{Verbatim}
\end{quote}
}
%
\definecolor{skyblue}{cmyk}{0.2, 0, 0.1, 0}
\definecolor{skyyellow}{cmyk}{0.1, 0.1, 0.5, 0}
%
%\title{NZMATH User Manual\\ {\large{(for version 1.0)}}}
%\date{}
%\author{}
\begin{document}
%\maketitle
%
\setcounter{tocdepth}{3}
\setcounter{secnumdepth}{3}


\tableofcontents
\C

\chapter{Classes}


%---------- start document ---------- %
 \section{algfield -- Algebraic Number Field}\linkedzero{algfield}
 \begin{itemize}
 \item {\bf Classes}
   \begin{itemize}
   \item \linkingone{algfield}{NumberField}
   \item \linkingone{algfield}{BasicAlgNumber}
   \item \linkingone{algfield}{MatAlgNumber}
   \end{itemize}
 \item {\bf Functions}
   \begin{itemize}
   \item \linkingone{algfield}{changetype}
   \item \linkingone{algfield}{disc}
   \item \linkingone{algfield}{fppoly}
%   \item \linkingone{algfield}{prime$\_$decomp}
   \item \linkingone{algfield}{qpoly}
   \item \linkingone{algfield}{zpoly}
   \end{itemize}
 \end{itemize}
%
  \subsection{NumberField -- number field}\linkedone{algfield}{NumberField}
  \initialize
  \func{NumberField}{
    \hiki{f}{list},\
    \hikiopt{precompute}{bool}{False}
  }{
    \out{NumberField}
  }\\
  \spacing
  % document of basic document
  \quad Create NumberField object. \\
  \spacing
  % added document
  \quad This field defined by the polynomial \param{f}. \\
  The class inherits \linkingone{ring}{Field}.\\
  \spacing
  % input, output document
  \quad \param{f}, which expresses coefficients of a polynomial, must be a list of integers.
  \param{f} should be written in ascending order.  \param{f} must be monic irreducible over rational field. \\
  If \param{precompute} is True, all solutions of \param{f} (by \linkingtwo{algfield}{NumberField}{getConj}), the discriminant of \param{f} (by \linkingtwo{algfield}{NumberField}{disc}), the signature (by \linkingtwo{algfield}{NumberField}{signature}) and the field discriminant of the basis of the integer ring (by \linkingtwo{algfield}{NumberField}{integer\_ring}) are precomputed.\\
  \begin{at}
    \item[degree]\linkedtwo{algfield}{NumberField}{degree}: The (absolute) extension degree of the number field.
    \item[polynomial]\linkedtwo{algfield}{NumberField}{polynomial}: The defining polynomial of the number field.
  \end{at}
  \begin{op}
    \verb|K * F| & Return the composite field of \param{K} and \param{F}. \\
    \verb|K == F| & Check whether the equality of \param{K} and \param{F}. \\
  \end{op}
\begin{ex}
>>> K = algfield.NumberField([-2, 0, 1])
>>> L = algfield.NumberField([-3, 0, 1])
>>> print K, L
NumberField([-2, 0, 1]) NumberField([-3, 0, 1])
>>> print K * L
NumberField([1L, 0L, -10L, 0L, 1L])
\end{ex}%Don't indent!
\C
  \method
  \subsubsection{getConj -- roots of polynomial}\linkedtwo{algfield}{NumberField}{getConj}
  \func{getConj}{\param{self}}{\out{list}}\\
  \spacing
  % document of basic document
  \quad Return all (approximate) roots of the \param{self}.\linkingtwo{algfield}{NumberField}{polynomial}. \\
  \spacing
  % add document
  %\spacing
  % input, output document
  \quad The output is a list of (approximate) complex number.\\
%
  \subsubsection{disc -- polynomial discriminant}\linkedtwo{algfield}{NumberField}{disc}
  \func{disc}{\param{self}}{\out{integer}}\\
  \spacing
  % document of basic document
  \quad Return the (polynomial) discriminant of the \param{self}.\linkingtwo{algfield}{NumberField}{polynomial}. \\
  \spacing
  % add document
  \quad \negok The output is not discriminant of the number field itself. \\
  %\spacing
  % input, output document
%
  \subsubsection{integer\_ring -- integer ring}\linkedtwo{algfield}{NumberField}{integer\_ring}
  \func{integer\_ring}{\param{self}}{\out{\linkingone{matrix}{FieldSquareMatrix}}}\\
  \spacing
  % document of basic document
  \quad Return a basis of the ring of integers of \param{self}. \\
  \spacing
  % add document
  \quad \negok The function uses \linkingone{round2}{round2}.\\
  %\spacing
  % input, output document
%
  \subsubsection{field\_discriminant -- discriminant}\linkedtwo{algfield}{NumberField}{field\_discriminant}
  \func{field\_discriminant}{\param{self}}{\out{\linkingone{rational}{Rational}}}\\
  \spacing
  % document of basic document
  \quad Return the field discriminant of \param{self}. \\
  \spacing
  % add document
  \quad \negok The function uses \linkingone{round2}{round2}.\\
  %\spacing
  % input, output document
%
  \subsubsection{basis -- standard basis}\linkedtwo{algfield}{NumberField}{basis}
  \func{basis}{\param{self},\ \hiki{j}{integer}}{\out{\linkingone{algfield}{BasicAlgNumber}}}\\
  \spacing
  % document of basic document
  \quad Return the \param{j}-th basis (over the rational field) of \param{self}. \\
  \spacing
  % add document
  \quad Let $\theta$ be a solution of \param{self}.\linkingtwo{algfield}{NumberField}{polynomial}.
  Then $\theta^j$ is a part of basis of \param{self}, so the method returns them.This basis is called ``standard basis'' or ``power basis''.\\
  %\spacing
  % input, output document
%
  \subsubsection{signature -- signature}\linkedtwo{algfield}{NumberField}{signature}
  \func{signature}{\param{self}}{\out{list}}\\
  \spacing
  % document of basic document
  \quad Return the signature of \param{self}. \\
  \spacing
  % add document
  \quad \negok The method uses Strum's algorithm.\\
  %\spacing
  % input, output document
%
  \subsubsection{POLRED -- polynomial reduction}\linkedtwo{algfield}{NumberField}{POLRED}
  \func{POLRED}{\param{self}}{\out{list}}\\
  \spacing
  % document of basic document
  \quad Return some polynomials defining subfields of \param{self}. \\
  \spacing
  % add document
  \quad \negok ``POLRED'' means ``polynomial reduction''. 
  That is, it finds polynomials whose coefficients are not so large.\\
  %\spacing
  % input, output document
%  
  \subsubsection{isIntBasis -- check integral basis}\linkedtwo{algfield}{NumberField}{isIntBasis}
  \func{isIntBasis}{\param{self}}{\out{bool}}\\
  \spacing
  % document of basic document
  \quad Check whether power basis of \param{self} is also an integral basis of the field. \\
  %\spacing
  % add document
  %\spacing
  % input, output document
%  
  \subsubsection{isGaloisField -- check Galois field}\linkedtwo{algfield}{NumberField}{isGaloisField}
  \func{isGaloisField}{\param{self}}{\out{bool}}\\
  \spacing
  % document of basic document
  \quad Check whether the extension \param{self} over the rational field is Galois. \\
  %\spacing
  % add document
  \quad \negok As it stands, it only checks the signature.\\
  %\spacing
  % input, output document
%  
  \subsubsection{isFieldElement -- check field element}\linkedtwo{algfield}{NumberField}{isFieldElement}
  \func{isFieldElement}{\param{self},\ \hiki{A}{BasicAlgNumber/MatAlgNumber}}{\out{bool}}\\
  \spacing
  % document of basic document
  \quad Check whether \param{A} is an element of the field \param{self}. \\
  %\spacing
  % add document
  %\spacing
  % input, output document
  %\quad The input parameter \param{A} must be an instance of \linkingone{algfield}{BasicAlgNumber} or \linkingone{algfield}{MatAlgNumber}. \\
%  
  \subsubsection{getCharacteristic -- characteristic}\linkedtwo{algfield}{NumberField}{getCharacteristic}
  \func{getCharacteristic}{\param{self}}{\out{integer}}\\
  \spacing
  % document of basic document
  \quad Return the characteristic of \param{self}. \\
  \spacing
  % add document
  \quad It returns always zero. The method is only for ensuring consistency.\\
  %\spacing
  % input, output document
%  
  \subsubsection{createElement -- create an element}\linkedtwo{algfield}{NumberField}{createElement}
  \func{createElement}{\param{self},\ \hiki{seed}{list}}{\out{BasicAlgNumber/MatAlgNumber}}\\
  \spacing
  % document of basic document
  \quad Return an element of \param{self} with \param{seed}. \\
  \spacing
  % add document
  \quad \param{seed} determines the class of returned element.\\
  For example, if \param{seed} forms as $[[e_1, e_2, \ldots, e_n],\ d]$, then it calls \linkingone{algfield}{BasicAlgNumber}.\\
  %\spacing
  % input, output document
%  
\begin{ex}
>>> K = algfield.NumberField([3, 0, 1])
>>> K.getConj()
[-1.7320508075688774j, 1.7320508075688772j]
>>> K.disc()
-12L
>>> print K.integer_ring()
1/1 1/2
0/1 1/2
>>> K.field_discriminant()
Rational(-3, 1)
>>> K.basis(0), K.basis(1)
BasicAlgNumber([[1, 0], 1], [3, 0, 1]) BasicAlgNumber([[0, 1], 1], [3, 0, 1])
>>> K.signature()
(0, 1)
>>> K.POLRED()                     
[IntegerPolynomial([(0, 4L), (1, -2L), (2, 1L)], IntegerRing()), 
IntegerPolynomial([(0, -1L), (1, 1L)], IntegerRing())]
>>> K.isIntBasis()                 
False
\end{ex}%Don't indent!
\C
  \subsection{BasicAlgNumber -- Algebraic Number Class by standard basis}\linkedone{algfield}{BasicAlgNumber}
  \initialize
  \func{BasicAlgNumber}{
   \hiki{valuelist}{list},\
   \hiki{polynomial}{list},\ 
   \hikiopt{precompute}{bool}{False}
  }{
   \out{BasicAlgNumber}
  }\\
  \spacing
  % document of basic document
  \quad Create an algebraic number with standard (power) basis. \\
  \spacing
  % added document
  %\spacing
  % input, output document
  \quad $\param{valuelist} = [[e_1, e_2, \ldots, e_n],\ d]$ means $\DS \frac{1}{d}  (e_1 + e_2 \theta + e_3 \theta^2 + \cdots + e_n \theta^{n-1})$,  where $\theta$ is a solution of the polynomial \param{polynomial}.
  Note that $\langle \theta^i \rangle$ is a (standard) basis of the field defining by \param{polynomial} over the rational field.\\
  \spacing
  \quad $e_i,\ d$ must be integers.
  Also, \param{polynomial} should be list of integers. \\
  If \param{precompute} is True, all solutions of \param{polynomial} (by \linkingtwo{algfield}{BasicAlgNumber}{getConj}), approximation values of all conjugates of \param{self} (by \linkingtwo{algfield}{BasicAlgNumber}{getApprox}) and a polynomial which is a solution of \param{self} (by \linkingtwo{algfield}{BasicAlgNumber}{getCharPoly}) are precomputed. \\
  \begin{at}
    \item[value]\linkedtwo{algfield}{BasicAlgNumber}{value}: The list of numerators (the integer part) and the denominator of \param{self}.
    \item[coeff]\linkedtwo{algfield}{BasicAlgNumber}{coeff}: The coefficients of numerators (the integer part) of \param{self}.
    \item[denom]\linkedtwo{algfield}{BasicAlgNumber}{denom}: The denominator of the algebraic number for standard basis.
    \item[degree]\linkedtwo{algfield}{BasicAlgNumber}{degree}: The degree of extension of the field over the rational field.
    \item[polynomial]\linkedtwo{algfield}{BasicAlgNumber}{polynomial}: The defining polynomial of the field.
    \item[field]\linkedtwo{algfield}{BasicAlgNumber}{field}: The number field in which \param{self} is.
  \end{at}
  \begin{op}
    \verb|a + b| & Return the sum of \param{a} and \param{b}. \\
    \verb|a - b| & Return the subtraction of \param{a} and \param{b}. \\
    \verb| - a| & Return the negation of \param{a}. \\
    \verb|a * b| & Return the product of \param{a} and \param{b}. \\ 
    \verb|a ** k| & Return the \param{k}-th power of \param{a}. \\
    \verb|a / b| & Return the quotient of \param{a} by \param{b}. \\
  \end{op}
\begin{ex}
>>> a = algfield.BasicAlgNumber([[1, 1], 1], [-2, 0, 1])
>>> b = algfield.BasicAlgNumber([[-1, 2], 1], [-2, 0, 1])
>>> print a + b
BasicAlgNumber([[0, 3], 1], [-2, 0, 1])
>>> print a * b
BasicAlgNumber([[3L, 1L], 1], [-2, 0, 1])
>>> print a ** 3
BasicAlgNumber([[7L, 5L], 1], [-2, 0, 1])
>>> a // b
BasicAlgNumber([[5L, 3L], 7L], [-2, 0, 1])
\end{ex}%Don't indent!
\C
  \method
  \subsubsection{inverse -- inverse}\linkedtwo{algfield}{BasicAlgNumber}{inverse}
  \func{inverse}{\param{self}}{\out{BasicAlgNumber}}\\
  \spacing
  % document of basic document
  \quad Return the inverse of \param{self}. \\
  %\spacing
  % add document
  %\spacing
  % input, output document
%
  \subsubsection{getConj -- roots of polynomial}\linkedtwo{algfield}{BasicAlgNumber}{getConj}
  \func{getConj}{\param{self}}{\out{list}}\\
  \spacing
  % document of basic document
  \quad Return all (approximate) roots of \param{self}.\linkingtwo{algfield}{BasicAlgNumber}{polynomial}. \\
  %\spacing
  % add document
  %\spacing
  % input, output document
%
  \subsubsection{getApprox -- approximate conjugates}\linkedtwo{algfield}{BasicAlgNumber}{getApprox}
  \func{getApprox}{\param{self}}{\out{list}}\\
  \spacing
  % document of basic document
  \quad Return all (approximate) conjugates of \param{self}. \\
  %\spacing
  % add document
  %\spacing
  % input, output document
%
  \subsubsection{getCharPoly -- characteristic polynomial}\linkedtwo{algfield}{BasicAlgNumber}{getCharPoly}
  \func{getCharPoly}{\param{self}}{\out{list}}\\
  \spacing
  % document of basic document
  \quad Return the characteristic polynomial of \param{self}. \\
  \spacing
  % add document
  \quad \negok \param{self} is a solution of the characteristic polynomial.\\
  \spacing
  % input, output document
  \quad The output is a list of integers. \\
%
  \subsubsection{getRing -- the field}\linkedtwo{algfield}{BasicAlgNumber}{getRing}
  \func{getRing}{\param{self}}{\out{NumberField}}\\
  \spacing
  % document of basic document
  \quad Return the field which \param{self} belongs to. \\
  %\spacing
  % add document
  %\spacing
  % input, output document
%
  \subsubsection{trace -- trace}\linkedtwo{algfield}{BasicAlgNumber}{trace}
  \func{trace}{\param{self}}{\out{Rational}}\\
  \spacing
  % document of basic document
  \quad Return the trace of \param{self} in the \param{self}.\linkingtwo{algfield}{BasicAlgNumber}{field} over the rational field. \\
  %\spacing
  % add document
  %\spacing
  % input, output document
%
  \subsubsection{norm -- norm}\linkedtwo{algfield}{BasicAlgNumber}{norm}
  \func{norm}{\param{self}}{\out{Rational}}\\
  \spacing
  % document of basic document
  \quad Return the norm of \param{self} in the \param{self}.\linkingtwo{algfield}{BasicAlgNumber}{field} over the rational field. \\
  %\spacing
  % add document
  %\spacing
  % input, output document
%
  \subsubsection{isAlgInteger -- check (algebraic) integer}\linkedtwo{algfield}{BasicAlgNumber}{isAlgInteger}
  \func{isAlgInteger}{\param{self}}{\out{bool}}\\
  \spacing
  % document of basic document
  \quad Check whether \param{self} is an (algebraic) integer or not. \\
  %\spacing
  % add document
  %\spacing
  % input, output document
%
  \subsubsection{ch\_matrix -- obtain MatAlgNumber object}\linkedtwo{algfield}{BasicAlgNumber}{ch\_matrix}
  \func{ch\_matrix}{\param{self}}{\out{MatAlgNumber}}\\
  \spacing
  % document of basic document
  \quad Return \linkingone{algfield}{MatAlgNumber} object corresponding to \param{self}. \\
  %\spacing
  % add document
  %\spacing
  % input, output document
%
\begin{ex}
>>> a = algfield.BasicAlgNumber([[1, 1], 1], [-2, 0, 1])
>>> a.inverse()
BasicAlgNumber([[-1L, 1L], 1L], [-2, 0, 1])
>>> a.getConj()
[(1.4142135623730951+0j), (-1.4142135623730951+0j)]
>>> a.getApprox()
[(2.4142135623730949+0j), (-0.41421356237309515+0j)]
>>> a.getCharPoly()
[-1, -2, 1]
>>> a.getRing()
NumberField([-2, 0, 1])
>>> a.trace(), a.norm()
2 -1
>>> a.isAlgInteger()
True
>>> a.ch_matrix()
MatAlgNumber([1, 1]+[2, 1], [-2, 0, 1])
\end{ex}%Don't indent!
\C
  \subsection{MatAlgNumber -- Algebraic Number Class by matrix representation}\linkedone{algfield}{MatAlgNumber}
  \initialize
  \func{MatAlgNumber}{
    \hiki{coefficient}{list},\
    \hiki{polynomial}{list}
  }{
    \out{MatAlgNumber}
  }\\
  \spacing
  % document of basic document
  \quad Create an algebraic number represented by a matrix. \\
  \spacing
  % added document
  \quad ``matrix representation'' means the matrix $A$ over the rational field such that $(e_1 + e_2 \theta + e_3 \theta^2 + \cdots + e_n \theta^{n-1})(1,\theta,\ldots,\theta^{n-1})^T=A(1,\theta,\ldots,\theta^{n-1})^T$, where $^t$ expresses transpose operation.\\
  \spacing
  % input, output document
  \quad $\param{coefficient} = [e_1, e_2, \ldots, e_n]$ means $e_1 + e_2 \theta + e_3 \theta^2 + \cdots + e_n \theta^{n-1}$,  where $\theta$ is a solution of the polynomial \param{polynomial}.
  Note that $\langle \theta^i \rangle$ is a (standard) basis of the field defining by \param{polynomial} over the rational field.\\
  \param{coefficient} must be a list of (not only integers) rational numbers.
  \param{polynomial} must be a list of integers. \\
  \begin{at}
    \item[coeff]\linkedtwo{algfield}{MatAlgNumber}{coeff}: The coefficients of the algebraic number for standard basis.
    \item[degree]\linkedtwo{algfield}{MatAlgNumber}{degree}: The degree of extension of the field over the rational field.
    \item[matrix]\linkedtwo{algfield}{MatAlgNumber}{matrix}: The representation matrix of the algebraic number.
    \item[polynomial]\linkedtwo{algfield}{MatAlgNumber}{polynomial}: The defining polynomial of the field.
    \item[field]\linkedtwo{algfield}{MatAlgNumber}{field}: The number field in which \param{self} is.
    
  \end{at}
  \begin{op}
    \verb|a + b| & Return the sum of \param{a} and \param{b}. \\
    \verb|a - b| & Return the subtraction of \param{a} and \param{b}. \\
    \verb| - a| & Return the negation of \param{a}. \\
    \verb|a * b| & Return the product of \param{a} and \param{b}. \\
    \verb|a ** k| & Return the \param{k}-th power of \param{a}. \\
    \verb|a / b| & Return the quotient of \param{a} by \param{b}. \\
  \end{op}
\begin{ex}
>>> a = algfield.MatAlgNumber([1, 2], [-2, 0, 1])
>>> b = algfield.MatAlgNumber([-2, 3], [-2, 0, 1])
>>> print a + b
MatAlgNumber([-1, 5]+[10, -1], [-2, 0, 1])
>>> print a * b
MatAlgNumber([10, -1]+[-2, 10], [-2, 0, 1])
>>> print a ** 3
MatAlgNumber([25L, 22L]+[44L, 25L], [-2, 0, 1])
>>> print a / b
MatAlgNumber([Rational(1, 1), Rational(1, 2)]+
[Rational(1, 1), Rational(1, 1)], [-2, 0, 1])
\end{ex}%Don't indent!
\C
  \method
  \subsubsection{inverse -- inverse}\linkedtwo{algfield}{MatAlgNumber}{inverse}
  \func{inverse}{\param{self}}{\out{MatAlgNumber}}\\
  \spacing
  % document of basic document
  \quad Return the inverse of \param{self}. \\
  %\spacing
  % add document
  %\spacing
  % input, output document
%
  \subsubsection{getRing -- the field}\linkedtwo{algfield}{MatAlgNumber}{getRing}
  \func{getRing}{\param{self}}{\out{NumberField}}\\
  \spacing
  % document of basic document
  \quad Return the field which \param{self} belongs to. \\  
  %\spacing
  % add document
  %\spacing
  % input, output document
%
  \subsubsection{trace -- trace}\linkedtwo{algfield}{MatAlgNumber}{trace}
  \func{trace}{\param{self}}{\out{Rational}}\\
  \spacing
  % document of basic document
  \quad Return the trace of \param{self} in the \param{self}.\linkingtwo{algfield}{MatAlgNumber}{field} over the rational field. \\
  %\spacing
  % add document
  %\spacing
  % input, output document
%
  \subsubsection{norm -- norm}\linkedtwo{algfield}{MatAlgNumber}{norm}
  \func{norm}{\param{self}}{\out{Rational}}\\
  \spacing
  % document of basic document
  \quad Return the norm of \param{self} in the \param{self}.\linkingtwo{algfield}{MatAlgNumber}{field} over the rational field. \\
  %\spacing
  % add document
  %\spacing
  % input, output document
%
  \subsubsection{ch\_basic -- obtain BasicAlgNumber object}\linkedtwo{algfield}{BasicAlgNumber}{ch\_basic}
  \func{ch\_basic}{\param{self}}{\out{BasicAlgNumber}}\\
  \spacing
  % document of basic document
  \quad Return \linkingone{algfield}{BasicAlgNumber} object corresponding to \param{self}.\\
  %\spacing
  % add document
  %\spacing
  % input, output document
%
\begin{ex}
>>> a = algfield.MatAlgNumber([1, -1, 1], [-3, 1, 2, 1])
>>> a.inverse()
MatAlgNumber([Rational(2, 3), Rational(4, 9), Rational(1, 9)]+
[Rational(1, 3), Rational(5, 9), Rational(2, 9)]+
[Rational(2, 3), Rational(1, 9), Rational(1, 9)], [-3, 1, 2, 1])
>>> a.trace()
Rational(7, 1)
>>> a.norm()
Rational(27, 1)
>>> a.getRing()
NumberField([-3, 1, 2, 1])
>>> a.ch_basic()
BasicAlgNumber([[1, -1, 1], 1], [-3, 1, 2, 1])
\end{ex}
\C
  \subsection{changetype(function) -- obtain BasicAlgNumber object}\linkedone{algfield}{changetype}
  \func{changetype}{
    \hiki{a}{integer},\ 
    \hikiopt{polynomial}{list}{[0, 1]}
  }{
    \out{BasicAlgNumber}
  } \\
  \func{changetype}{
    \hiki{a}{Rational},\
    \hikiopt{polynomial}{list}{[0, 1]}
  }{
    \out{BasicAlgNumber}
  } \\
  \func{changetype}{
    \hiki{polynomial}{list}
  }{
    \out{BasicAlgNumber}
  } \\
  \spacing
  % document of basic document
%  \quad Return the value of that changed \param{a} into the element in the field defined by the polynomial \param{polynomial}. \\
  \quad Return a BasicAlgNumber object corresponding to \param{a}.
  \spacing
  % add document
  \spacing
  \quad If \param{a} is an integer or an instance of \linkingone{rational}{Rational}, the function returns \linkingone{algfield}{BasicAlgNumber} object whose field is defined by \param{polynomial}.
  If \param{a} is a list, the function returns \linkingone{algfield}{BasicAlgNumber} corresponding to a solution of \param{a}, considering \param{a} as the polynomial.\\
  \spacing
  % input, output document
  The input parameter \param{a} must be an integer, \linkingone{rational}{Rational} or a list of integers.\\
%
  \subsection{disc(function) -- discriminant}\linkedone{algfield}{disc}
  \func{disc}{\hiki{A}{list}}{\out{Rational}} \\
  \spacing
  % document of basic document
  \quad Return the discriminant of $a_i$, where $\param{A} = [a_1, a_2, \cdots, a_n]$. \\
  \spacing
  % add document
  % \spacing
  % input, output document
  \param{$a_i$} must be an instance of \linkingone{algfield}{BasicAlgNumber} or \linkingone{algfield}{MatAlgNumber} defined over a same number field.\\
%
  \subsection{fppoly(function) -- polynomial over finite prime field}\linkedone{algfield}{fppoly}
  \func{fppoly}{\hiki{coeffs}{list},\ \hiki{p}{integer}}{\out{\linkingone{poly.uniutil}{FinitePrimeFieldPolynomial}}} \\
  \spacing
  % document of basic document
  \quad Return the polynomial whose coefficients \param{coeffs} are defined over the prime field $\mathbb{Z}_\param{p}$. \\
  \spacing
  % add document
  % \spacing
  % input, output document
  \quad \param{coeffs} should be a list of integers or of instances of \linkingone{finitefield}{FinitePrimeFieldElement}.\\
%
%  \subsection{prime\_decomp}\linkedone{algfield}{prime\_decomp}
%  \func{prime\_decpomp}{\hiki{p}{integer}, \, \hiki{f}{list}}{\out{BasicAlgNumber}} \\
%  \spacing
  % document of basic document
%  \quad Return prime decomposition of $(\param{p})$ over $\mathbb{Q}[x]/ (\param{f})$ . \\
%  \spacing
  % add document
  % \spacing
  % input, output document
%  \quad If output is $((\alpha, \beta), (\alpha, \gamma))$, this means $\param{p} = (\alpha, \beta)(\alpha, \gamma)$. \\
%
  \subsection{qpoly(function) -- polynomial over rational field}\linkedone{algfield}{qpoly}
  \func{qpoly}{\hiki{coeffs}{list}}{\out{\linkingone{poly.uniutil}{FieldPolynomial}}} \\
  \spacing
  % document of basic document
  \quad Return the polynomial whose coefficients \param{coeffs} are defined over the rational field. \\
  \spacing
  % add document
  % \spacing
  % input, output document
  \quad \param{coeffs} must be a list of integers or instances of \linkingone{rational}{Rational}.\\
%
  \subsection{zpoly(function) -- polynomial over integer ring}\linkedone{algfield}{zpoly}
  \func{zpoly}{\hiki{coeffs}{list}}{\out{\linkingone{poly.uniutil}{IntegerPolynomial}}} \\
  \spacing
  % document of basic document
  \quad Return the polynomial whose coefficients \param{coeffs} are defined over the (rational) integer ring.  \\
  \spacing
  % add document
  % \spacing
  % input, output document
  \quad \param{coeffs} must be a list of integers.\\
%
\begin{ex}
>>> a = algfield.changetype(3, [-2, 0, 1])
>>> b = algfield.BasicAlgNumber([[1, 2], 1], [-2, 0, 1])
>>> A = [a, b]
>>> algfield.disc(A)
288L
\end{ex}%Don't indent!(indent causes an error.)
%>>> algfield.prime_decomp(5, [-3, 1, 2, 1])
%[((BasicAlgNumber([[5, 0, 0], 1], [-3, 1, 2, 1]), 
%BasicAlgNumber([[3L, 1L, 0], 1], [-3, 1, 2, 1])), 1), 
%((BasicAlgNumber([[5, 0, 0], 1], [-3, 1, 2, 1]), 
%BasicAlgNumber([[2L, 1L, 0], 1], [-3, 1, 2, 1])), 2)]
\C

%---------- end document ---------- %

\bibliographystyle{jplain}%use jbibtex
\bibliography{nzmath_references}

\end{document}

