\documentclass{report}

\documentclass{report}

%%%%%%%%%%%%%%%%%%%%%%%%%%%%%%%%%%%%%%%%%%%%%%%%%%%%%%%%%%%%%
%
% macros for nzmath manual
%
%%%%%%%%%%%%%%%%%%%%%%%%%%%%%%%%%%%%%%%%%%%%%%%%%%%%%%%%%%%%%
\usepackage{amssymb,amsmath}
\usepackage{color}
\usepackage[dvipdfm,bookmarks=true,bookmarksnumbered=true,%
 pdftitle={NZMATH Users Manual},%
 pdfsubject={Manual for NZMATH Users},%
 pdfauthor={NZMATH Development Group},%
 pdfkeywords={TeX; dvipdfmx; hyperref; color;},%
 colorlinks=true]{hyperref}
\usepackage{fancybox}
\usepackage[T1]{fontenc}
%
\newcommand{\DS}{\displaystyle}
\newcommand{\C}{\clearpage}
\newcommand{\NO}{\noindent}
\newcommand{\negok}{$\dagger$}
\newcommand{\spacing}{\vspace{1pt}\\ }
% software macros
\newcommand{\nzmathzero}{{\footnotesize $\mathbb{N}\mathbb{Z}$}\texttt{MATH}}
\newcommand{\nzmath}{{\nzmathzero}\ }
\newcommand{\pythonzero}{$\mbox{\texttt{Python}}$}
\newcommand{\python}{{\pythonzero}\ }
% link macros
\newcommand{\linkingzero}[1]{{\bf \hyperlink{#1}{#1}}}%module
\newcommand{\linkingone}[2]{{\bf \hyperlink{#1.#2}{#2}}}%module,class/function etc.
\newcommand{\linkingtwo}[3]{{\bf \hyperlink{#1.#2.#3}{#3}}}%module,class,method
\newcommand{\linkedzero}[1]{\hypertarget{#1}{}}
\newcommand{\linkedone}[2]{\hypertarget{#1.#2}{}}
\newcommand{\linkedtwo}[3]{\hypertarget{#1.#2.#3}{}}
\newcommand{\linktutorial}[1]{\href{http://docs.python.org/tutorial/#1}{#1}}
\newcommand{\linktutorialone}[2]{\href{http://docs.python.org/tutorial/#1}{#2}}
\newcommand{\linklibrary}[1]{\href{http://docs.python.org/library/#1}{#1}}
\newcommand{\linklibraryone}[2]{\href{http://docs.python.org/library/#1}{#2}}
\newcommand{\pythonhp}{\href{http://www.python.org/}{\python website}}
\newcommand{\nzmathwiki}{\href{http://nzmath.sourceforge.net/wiki/}{{\nzmathzero}Wiki}}
\newcommand{\nzmathsf}{\href{http://sourceforge.net/projects/nzmath/}{\nzmath Project Page}}
\newcommand{\nzmathtnt}{\href{http://tnt.math.se.tmu.ac.jp/nzmath/}{\nzmath Project Official Page}}
% parameter name
\newcommand{\param}[1]{{\tt #1}}
% function macros
\newcommand{\hiki}[2]{{\tt #1}:\ {\em #2}}
\newcommand{\hikiopt}[3]{{\tt #1}:\ {\em #2}=#3}

\newdimen\hoge
\newdimen\truetextwidth
\newcommand{\func}[3]{%
\setbox0\hbox{#1(#2)}
\hoge=\wd0
\truetextwidth=\textwidth
\advance \truetextwidth by -2\oddsidemargin
\ifdim\hoge<\truetextwidth % short form
{\bf \colorbox{skyyellow}{#1(#2)\ $\to$ #3}}
%
\else % long form
\fcolorbox{skyyellow}{skyyellow}{%
   \begin{minipage}{\textwidth}%
   {\bf #1(#2)\\ %
    \qquad\quad   $\to$\ #3}%
   \end{minipage}%
   }%
\fi%
}

\newcommand{\out}[1]{{\em #1}}
\newcommand{\initialize}{%
  \paragraph{\large \colorbox{skyblue}{Initialize (Constructor)}}%
    \quad\\ %
    \vspace{3pt}\\
}
\newcommand{\method}{\C \paragraph{\large \colorbox{skyblue}{Methods}}}
% Attribute environment
\newenvironment{at}
{%begin
\paragraph{\large \colorbox{skyblue}{Attribute}}
\quad\\
\begin{description}
}%
{%end
\end{description}
}
% Operation environment
\newenvironment{op}
{%begin
\paragraph{\large \colorbox{skyblue}{Operations}}
\quad\\
\begin{table}[h]
\begin{center}
\begin{tabular}{|l|l|}
\hline
operator & explanation\\
\hline
}%
{%end
\hline
\end{tabular}
\end{center}
\end{table}
}
% Examples environment
\newenvironment{ex}%
{%begin
\paragraph{\large \colorbox{skyblue}{Examples}}
\VerbatimEnvironment
\renewcommand{\EveryVerbatim}{\fontencoding{OT1}\selectfont}
\begin{quote}
\begin{Verbatim}
}%
{%end
\end{Verbatim}
\end{quote}
}
%
\definecolor{skyblue}{cmyk}{0.2, 0, 0.1, 0}
\definecolor{skyyellow}{cmyk}{0.1, 0.1, 0.5, 0}
%
%\title{NZMATH User Manual\\ {\large{(for version 1.0)}}}
%\date{}
%\author{}
\begin{document}
%\maketitle
%
\setcounter{tocdepth}{3}
\setcounter{secnumdepth}{3}


\tableofcontents
\C

\chapter{Basic Utilities}


%---------- start document ---------- %
 \section{config -- setting features}\linkedzero{config}
%
All constants in the module can be set in user's config file.
See the~\hyperlink{config.user}{User Settings} section for
more detailed description.

  \subsection{Default Settings}\linkedone{config}{default}

  \subsubsection{Dependencies}\linkedone{config}{dependencies}

  Some third party / platform dependent modules are possibly used, and
  they are configurable.

  \paragraph{HAVE\_MPMATH}\linkedone{config}{HAVE\_MPMATH}

  {\tt mpmath} is a package providing multiprecision math.
  See its \href{http://code.google.com/p/mpmath}{project page}.
  This package is used in \linkingzero{ecpp} module.

  \paragraph{HAVE\_SQLITE3}\linkedone{config}{HAVE\_SQLITE3}

  {\tt sqlite3} is the default database module for \python,
  but it need to be enabled at the build time.

  \paragraph{HAVE\_NET}\linkedone{config}{HAVE_NET}

  Some functions will connect to the Net.
  Desktop machines are usually connected to the Net, but notebooks may
  have connectivity only occasionally.

  \subsubsection{Plug-ins}\linkedone{config}{Plug-ins}

  \paragraph{PLUGIN\_MATH}\linkedone{config}{PLUGIN\_MATH}
  \python standard float/complex types and \linklibrary{math}/\linklibrary{cmath} modules only
  provide fixed precision (double precision), but sometimes
  multiprecision floating point is needed.

  \subsubsection{Assumptions}\linkedone{config}{assumptions}

  Some conjectures are useful for assuring the validity of a faster
  algorithm.

  All assumptions are default to {\tt False}, but you can set them
  {\tt True} if you believe them.

  \paragraph{GRH}\linkedone{config}{GRH}

  Generalized Riemann Hypothesis.  For example, primality test is
  \(O((\log n)^2)\) if GRH is true while \(O((\log n)^6)\) or something
  without it.

  \subsubsection{Files}\linkedone{config}{files}

  \paragraph{DATADIR}\linkedone{config}{DATADIR}

  The directory where \nzmath (static) data files are stored. The
  default will be {\tt os.path.join(sys.prefix, 'share', 'nzmath')} or
  {\tt os.path.join(sys.prefix, 'Data', 'nzmath')} on Windows.

  \subsection{Automatic Configuration}\linkedone{config}{auto}

  The items above can be set automatically by testing the environment.

  \subsubsection{Checks}\linkedone{config}{checks}

  Here are check functions.

  The constants accompanying the check functions which enable the check
  if it is {\tt True}, can be overridden in user settings.

  Both check functions and constants are not exposed.

  \paragraph{check\_mpmath()}\linkedone{config}{check\_mpmath}

  Check whether {\tt mpmath} is available or not.

  constant: {\tt CHECK\_MPMATH}

  \paragraph{check\_sqlite3()}\linkedone{config}{check\_sqlite3}

  Check if {\tt sqlite3} is importable or not.
  {\tt pysqlite2} may be a substitution.

  constant: {\tt CHECK\_SQLITE3}

  \paragraph{check\_net()}\linkedone{config}{check\_net}

  Check the net connection by HTTP call.

  constant: {\tt CHECK\_NET}

  \paragraph{check\_plugin\_math()}\linkedone{config}{check\_plugin\_math}

  Check which math plug-in is available.

  constant: {\tt CHECK\_PLUGIN\_MATH}

  \paragraph{default\_datadir()}\linkedone{config}{default\_datadir}

  Return default value for {\tt DATADIR}.

  This function selects the value from various candidates.
  If this function is called with {\tt DATADIR} set, the value of (previously-defined) {\tt DATADIR} is the first candidate to be returned. Other
  possibilities are, {\tt sys.prefix + 'Data/nzmath'} on Windows, or
  {\tt sys.prefix + 'share/nzmath'} on other platforms.

  Be careful that all the above paths do not exist, the function
  returns {\tt None}.

  constant: {\tt CHECK\_DATADIR}

  \subsection{User Settings}\linkedone{config}{user}

  The module try to load the user's config file named
  {\it nzmathconf.py}. The search path is the following:
  \begin{enumerate}
  \item The directory which is specified by an environment variable
    {\tt NZMATHCONFDIR}.
  \item If the platform is Windows, then
    \begin{enumerate}
    \item If an environment variable {\tt APPDATA} is set, {\tt
        APPDATA/nzmath}.
    \item If, alternatively, an environment variable {\tt USERPROFILE}
      is set,\linebreak {\tt USERPROFILE/Application~Data/nzmath}.
    \end{enumerate}
  \item On other platforms, if an environment variable {\tt HOME} is
    set, {\tt HOME/.nzmath.d}.
  \end{enumerate}

  {\it nzmathconf.py} is a \python script. Users can set the constants
  like {\tt HAVE\_MPMATH}, which will override the default settings. These
  constants, except assumption ones, are automatically set, unless
  constants accompanying the check functions are false (see 
  the~\hyperlink{config.auto}{Automatic Configuration} section above).

%---------- end document ---------- %

\bibliographystyle{jplain}%use jbibtex
\bibliography{nzmath_references}

\end{document}

