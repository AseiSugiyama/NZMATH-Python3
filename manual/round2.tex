\documentclass{report}

%%%%%%%%%%%%%%%%%%%%%%%%%%%%%%%%%%%%%%%%%%%%%%%%%%%%%%%%%%%%%
%
% macros for nzmath manual
%
%%%%%%%%%%%%%%%%%%%%%%%%%%%%%%%%%%%%%%%%%%%%%%%%%%%%%%%%%%%%%
\usepackage{amssymb,amsmath}
\usepackage{color}
\usepackage[dvipdfm,bookmarks=true,bookmarksnumbered=true,%
 pdftitle={NZMATH Users Manual},%
 pdfsubject={Manual for NZMATH Users},%
 pdfauthor={NZMATH Development Group},%
 pdfkeywords={TeX; dvipdfmx; hyperref; color;},%
 colorlinks=true]{hyperref}
\usepackage{fancybox}
\usepackage[T1]{fontenc}
%
\newcommand{\DS}{\displaystyle}
\newcommand{\C}{\clearpage}
\newcommand{\NO}{\noindent}
\newcommand{\negok}{$\dagger$}
\newcommand{\spacing}{\vspace{1pt}\\ }
% software macros
\newcommand{\nzmathzero}{{\footnotesize $\mathbb{N}\mathbb{Z}$}\texttt{MATH}}
\newcommand{\nzmath}{{\nzmathzero}\ }
\newcommand{\pythonzero}{$\mbox{\texttt{Python}}$}
\newcommand{\python}{{\pythonzero}\ }
% link macros
\newcommand{\linkingzero}[1]{{\bf \hyperlink{#1}{#1}}}%module
\newcommand{\linkingone}[2]{{\bf \hyperlink{#1.#2}{#2}}}%module,class/function etc.
\newcommand{\linkingtwo}[3]{{\bf \hyperlink{#1.#2.#3}{#3}}}%module,class,method
\newcommand{\linkedzero}[1]{\hypertarget{#1}{}}
\newcommand{\linkedone}[2]{\hypertarget{#1.#2}{}}
\newcommand{\linkedtwo}[3]{\hypertarget{#1.#2.#3}{}}
\newcommand{\linktutorial}[1]{\href{http://docs.python.org/tutorial/#1}{#1}}
\newcommand{\linktutorialone}[2]{\href{http://docs.python.org/tutorial/#1}{#2}}
\newcommand{\linklibrary}[1]{\href{http://docs.python.org/library/#1}{#1}}
\newcommand{\linklibraryone}[2]{\href{http://docs.python.org/library/#1}{#2}}
\newcommand{\pythonhp}{\href{http://www.python.org/}{\python website}}
\newcommand{\nzmathwiki}{\href{http://nzmath.sourceforge.net/wiki/}{{\nzmathzero}Wiki}}
\newcommand{\nzmathsf}{\href{http://sourceforge.net/projects/nzmath/}{\nzmath Project Page}}
\newcommand{\nzmathtnt}{\href{http://tnt.math.se.tmu.ac.jp/nzmath/}{\nzmath Project Official Page}}
% parameter name
\newcommand{\param}[1]{{\tt #1}}
% function macros
\newcommand{\hiki}[2]{{\tt #1}:\ {\em #2}}
\newcommand{\hikiopt}[3]{{\tt #1}:\ {\em #2}=#3}

\newdimen\hoge
\newdimen\truetextwidth
\newcommand{\func}[3]{%
\setbox0\hbox{#1(#2)}
\hoge=\wd0
\truetextwidth=\textwidth
\advance \truetextwidth by -2\oddsidemargin
\ifdim\hoge<\truetextwidth % short form
{\bf \colorbox{skyyellow}{#1(#2)\ $\to$ #3}}
%
\else % long form
\fcolorbox{skyyellow}{skyyellow}{%
   \begin{minipage}{\textwidth}%
   {\bf #1(#2)\\ %
    \qquad\quad   $\to$\ #3}%
   \end{minipage}%
   }%
\fi%
}

\newcommand{\out}[1]{{\em #1}}
\newcommand{\initialize}{%
  \paragraph{\large \colorbox{skyblue}{Initialize (Constructor)}}%
    \quad\\ %
    \vspace{3pt}\\
}
\newcommand{\method}{\C \paragraph{\large \colorbox{skyblue}{Methods}}}
% Attribute environment
\newenvironment{at}
{%begin
\paragraph{\large \colorbox{skyblue}{Attribute}}
\quad\\
\begin{description}
}%
{%end
\end{description}
}
% Operation environment
\newenvironment{op}
{%begin
\paragraph{\large \colorbox{skyblue}{Operations}}
\quad\\
\begin{table}[h]
\begin{center}
\begin{tabular}{|l|l|}
\hline
operator & explanation\\
\hline
}%
{%end
\hline
\end{tabular}
\end{center}
\end{table}
}
% Examples environment
\newenvironment{ex}%
{%begin
\paragraph{\large \colorbox{skyblue}{Examples}}
\VerbatimEnvironment
\renewcommand{\EveryVerbatim}{\fontencoding{OT1}\selectfont}
\begin{quote}
\begin{Verbatim}
}%
{%end
\end{Verbatim}
\end{quote}
}
%
\definecolor{skyblue}{cmyk}{0.2, 0, 0.1, 0}
\definecolor{skyyellow}{cmyk}{0.1, 0.1, 0.5, 0}
%
%\title{NZMATH User Manual\\ {\large{(for version 1.0)}}}
%\date{}
%\author{}
\begin{document}
%\maketitle
%
\setcounter{tocdepth}{3}
\setcounter{secnumdepth}{3}


\tableofcontents
\C

\chapter{Classes}


%---------- start document ---------- %
 \section{round2 -- the round 2 method}\linkedzero{round2}
 \begin{itemize}
   \item {\bf Classes}
   \begin{itemize}
     \item \linkingone{round2}{ModuleWithDenominator}
   \end{itemize}
   \item {\bf Functions}
     \begin{itemize}
       \item \linkingone{round2}{round2}
       \item \linkingone{round2}{Dedekind}
     \end{itemize}
 \end{itemize}

 The round 2 method is for obtaining the maximal order of a number
 field from an order generated by a root of a defining polynomial of
 the field.

 This implementation of the method is based on \cite{Cohen1}(Algorithm 6.1.8)
 and \cite{Kida}(Chapter 3).

\C

 \subsection{ModuleWithDenominator -- bases of $\mathbb{Z}$-module with denominator.}\linkedone{round2}{ModuleWithDenominator}
 \initialize
  \func{ModuleWithDenominator}{%
    \hiki{basis}{list},\ %
    \hiki{denominator}{integer},\ %
    **\hiki{hints}{dict}%
  }{\out{ModuleWithDenominator}}\\
  \spacing
  % document of basic document
  \quad This class represents bases of $\mathbb{Z}$-module with denominator.
  It is not a general purpose $\mathbb{Z}$-module, you are warned.
  % added document
  %
  % \spacing
  % input, output document
  \quad \param{basis} is a list of integer sequences.\\
  \quad \param{denominator} is a common denominator of all bases.\\
  \quad \negok Optionally you can supply keyword argument \param{dimension} if
  you would like to postpone the initialization of \param{basis}.
  \begin{op}
    \verb|A + B| & sum of two modules\\
    \verb|a * B| & scalar multiplication\\
    \verb|B / d| & divide by an integer\\
  \end{op}
  \method
  \subsubsection{get\_rationals -- get the bases as a list of rationals}\linkedtwo{round2}{ModuleWithDenominator}{get\_rationals}
   \func{get\_rationals}{\param{self}}{\out{list}}\\
   \spacing
   % document of basic document
   \quad Return a list of lists of rational numbers, which is bases
   divided by denominator.\\
   \spacing
 \subsubsection{get\_polynomials -- get the bases as a list of polynomials}\linkedtwo{round2}{ModuleWithDenominator}{get\_polynomials}
   \func{get\_polynomials}{\param{self}}{\out{list}}\\
   \spacing
   % document of basic document
   \quad Return a list of rational polynomials, which is made from
   bases divided by denominator.\\

   \subsubsection{determinant -- determinant of the bases}\linkedtwo{round2}{ModuleWithDenominator}{determinant}
   \func{determinant}{\param{self}}{\out{list}}\\
   \spacing
   \quad Return determinant of the bases (bases ought to be of full
   rank and in Hermite normal form).

\C
  \subsection{round2(function)}\linkedone{round2}{round2}
  \func{round2}{\hiki{minpoly\_coeff}{list}}{(\out{list},\ \out{integer})}\\
   \spacing
   % document of basic document
   \quad Return integral basis of the ring of integers of a field with its
    discriminant.  The field is given by a list of integers, which is
    a polynomial of generating element \(\theta\).  The polynomial ought to
    be monic, in other word, the generating element ought to be an
    algebraic integer.\\
    \quad The integral basis will be given as a list of rational vectors
    with respect to \(\theta\).\\
    %(In other functions, bases are returned in the same fashion.)\\
   \spacing
   \subsection{Dedekind(function)}\linkedone{round2}{Dedekind}
   \func{Dedekind}{%
     \hiki{minpoly\_coeff}{list},\ %
     \hiki{p}{integer},\ %
     \hiki{e}{integer}%
   }{(\out{bool},\ \out{ModuleWithDenominator})}\\
   \spacing
   \quad This is the Dedekind criterion.\\
   \spacing \quad \param{minpoly\_coeff} is an integer list of the
   minimal polynomial of \(\theta\).\\
   \quad \param{p}{\tt **}\param{e} divides the discriminant of the minimal.\\
   \quad The first element of the returned tuple is whether the
   computation about \param{p} is finished or not.\\
  \C

%---------- end document ---------- %

\bibliographystyle{jplain}%use jbibtex
\bibliography{nzmath_references}

\end{document}

