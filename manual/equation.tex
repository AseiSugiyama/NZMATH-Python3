\documentclass{report}

\documentclass{report}

%%%%%%%%%%%%%%%%%%%%%%%%%%%%%%%%%%%%%%%%%%%%%%%%%%%%%%%%%%%%%
%
% macros for nzmath manual
%
%%%%%%%%%%%%%%%%%%%%%%%%%%%%%%%%%%%%%%%%%%%%%%%%%%%%%%%%%%%%%
\usepackage{amssymb,amsmath}
\usepackage{color}
\usepackage[dvipdfm,bookmarks=true,bookmarksnumbered=true,%
 pdftitle={NZMATH Users Manual},%
 pdfsubject={Manual for NZMATH Users},%
 pdfauthor={NZMATH Development Group},%
 pdfkeywords={TeX; dvipdfmx; hyperref; color;},%
 colorlinks=true]{hyperref}
\usepackage{fancybox}
\usepackage[T1]{fontenc}
%
\newcommand{\DS}{\displaystyle}
\newcommand{\C}{\clearpage}
\newcommand{\NO}{\noindent}
\newcommand{\negok}{$\dagger$}
\newcommand{\spacing}{\vspace{1pt}\\ }
% software macros
\newcommand{\nzmathzero}{{\footnotesize $\mathbb{N}\mathbb{Z}$}\texttt{MATH}}
\newcommand{\nzmath}{{\nzmathzero}\ }
\newcommand{\pythonzero}{$\mbox{\texttt{Python}}$}
\newcommand{\python}{{\pythonzero}\ }
% link macros
\newcommand{\linkingzero}[1]{{\bf \hyperlink{#1}{#1}}}%module
\newcommand{\linkingone}[2]{{\bf \hyperlink{#1.#2}{#2}}}%module,class/function etc.
\newcommand{\linkingtwo}[3]{{\bf \hyperlink{#1.#2.#3}{#3}}}%module,class,method
\newcommand{\linkedzero}[1]{\hypertarget{#1}{}}
\newcommand{\linkedone}[2]{\hypertarget{#1.#2}{}}
\newcommand{\linkedtwo}[3]{\hypertarget{#1.#2.#3}{}}
\newcommand{\linktutorial}[1]{\href{http://docs.python.org/tutorial/#1}{#1}}
\newcommand{\linktutorialone}[2]{\href{http://docs.python.org/tutorial/#1}{#2}}
\newcommand{\linklibrary}[1]{\href{http://docs.python.org/library/#1}{#1}}
\newcommand{\linklibraryone}[2]{\href{http://docs.python.org/library/#1}{#2}}
\newcommand{\pythonhp}{\href{http://www.python.org/}{\python website}}
\newcommand{\nzmathwiki}{\href{http://nzmath.sourceforge.net/wiki/}{{\nzmathzero}Wiki}}
\newcommand{\nzmathsf}{\href{http://sourceforge.net/projects/nzmath/}{\nzmath Project Page}}
\newcommand{\nzmathtnt}{\href{http://tnt.math.se.tmu.ac.jp/nzmath/}{\nzmath Project Official Page}}
% parameter name
\newcommand{\param}[1]{{\tt #1}}
% function macros
\newcommand{\hiki}[2]{{\tt #1}:\ {\em #2}}
\newcommand{\hikiopt}[3]{{\tt #1}:\ {\em #2}=#3}

\newdimen\hoge
\newdimen\truetextwidth
\newcommand{\func}[3]{%
\setbox0\hbox{#1(#2)}
\hoge=\wd0
\truetextwidth=\textwidth
\advance \truetextwidth by -2\oddsidemargin
\ifdim\hoge<\truetextwidth % short form
{\bf \colorbox{skyyellow}{#1(#2)\ $\to$ #3}}
%
\else % long form
\fcolorbox{skyyellow}{skyyellow}{%
   \begin{minipage}{\textwidth}%
   {\bf #1(#2)\\ %
    \qquad\quad   $\to$\ #3}%
   \end{minipage}%
   }%
\fi%
}

\newcommand{\out}[1]{{\em #1}}
\newcommand{\initialize}{%
  \paragraph{\large \colorbox{skyblue}{Initialize (Constructor)}}%
    \quad\\ %
    \vspace{3pt}\\
}
\newcommand{\method}{\C \paragraph{\large \colorbox{skyblue}{Methods}}}
% Attribute environment
\newenvironment{at}
{%begin
\paragraph{\large \colorbox{skyblue}{Attribute}}
\quad\\
\begin{description}
}%
{%end
\end{description}
}
% Operation environment
\newenvironment{op}
{%begin
\paragraph{\large \colorbox{skyblue}{Operations}}
\quad\\
\begin{table}[h]
\begin{center}
\begin{tabular}{|l|l|}
\hline
operator & explanation\\
\hline
}%
{%end
\hline
\end{tabular}
\end{center}
\end{table}
}
% Examples environment
\newenvironment{ex}%
{%begin
\paragraph{\large \colorbox{skyblue}{Examples}}
\VerbatimEnvironment
\renewcommand{\EveryVerbatim}{\fontencoding{OT1}\selectfont}
\begin{quote}
\begin{Verbatim}
}%
{%end
\end{Verbatim}
\end{quote}
}
%
\definecolor{skyblue}{cmyk}{0.2, 0, 0.1, 0}
\definecolor{skyyellow}{cmyk}{0.1, 0.1, 0.5, 0}
%
%\title{NZMATH User Manual\\ {\large{(for version 1.0)}}}
%\date{}
%\author{}
\begin{document}
%\maketitle
%
\setcounter{tocdepth}{3}
\setcounter{secnumdepth}{3}


\tableofcontents
\C

\chapter{Functions}


%---------- start document ---------- %
 \section{equation -- solving equations, congruences }\linkedzero{equation}

 In the following descriptions, some type aliases are used.
 \begin{description}
   \item[poly\_list]\linkedone{equation}{poly\_list}:\\
     \param{poly\_list} is a list {\tt [a0, a1, \ldots, an]}
     representing a polynomial coefficients in ascending order, i.e.,
     meaning \(a_0 + a_1 X + \cdots + a_n X^n\).  The type of each
     {\tt ai} depends on each function (explained in their descriptions).

   \item[integer]\linkedone{equation}{integer}:\\
     \param{integer} is one of {\it int}, {\it long} or \linkingone{rational}{Integer}.
   \item[complex]\linkedone{equation}{complex}:\\
     \param{complex} includes all number types in the complex field:
     \linkingone{equation}{integer}, {\it float}, {\it complex} of \python,
     \linkingone{rational}{Rational} of \nzmath, etc.\\
 \end{description}
%
  \subsection{e1 -- solve equation with degree 1}\linkedone{equation}{e1}
   \func{e1}{\hiki{f}{\linkingone{equation}{poly\_list}}}{\out{\linkingone{equation}{complex}}}\\
   \spacing
   % document of basic document
   \quad Return the solution of linear equation $ax + b = 0$.\\
   \spacing
   % added document
   %\spacing
   % input, output document
   \quad \param{f} ought to be a \linkingone{equation}{poly\_list} {\tt [b, a]} of \linkingone{equation}{complex}.\\
%
  \subsection{e1\_ZnZ -- solve congruent equation modulo n with degree 1}\linkedone{equation}{e1\_ZnZ}
   \func{e1\_ZnZ}{\hiki{f}{\linkingone{equation}{poly\_list}},\ \hiki{n}{integer}}{\out{integer}}\\
   \spacing
   % document of basic document
   \quad Return the solution of $ax + b  \equiv 0 \pmod{\param{n}}$.\\
   \spacing
   % added document
   %\spacing
   % input, output document
   \quad \param{f} ought to be a \linkingone{equation}{poly\_list} {\tt [b, a]} of \linkingone{equation}{integer}.\\
%
  \subsection{e2 -- solve equation with degree 2}\linkedone{equation}{e2}
   \func{e2}{\hiki{f}{\linkingone{equation}{poly\_list}}}{\out{tuple}}\\
   \spacing
   % document of basic document
   \quad Return the solution of quadratic equation $ax^2 + bx + c = 0$.\\
   \spacing
   % added document
   %\spacing
   % input, output document
   \quad \param{f} ought to be a \linkingone{equation}{poly\_list} {\tt [c, b, a]} of \linkingone{equation}{complex}. \\
   The result tuple will contain exactly 2 roots, even in the case of
   double root.\\
%
  \subsection{e2\_Fp -- solve congruent equation modulo p with degree 2}\linkedone{equation}{e2\_Fp}
   \func{e2\_Fp}{\hiki{f}{\linkingone{equation}{poly\_list}},\ \hiki{p}{integer}}{\out{list}}\\
   \spacing
   % document of basic document
   \quad Return the solution of $ax^2 + bx + c  \equiv 0 \pmod{\param{p}}$.\\
   \spacing
   % added document
   \quad If the same values are returned, then the values are multiple roots. \\
   \spacing
   % input, output document
   \quad \param{f} ought to be a \linkingone{equation}{poly\_list} of 
   \linkingone{equation}{integer}s {\tt [c, b, a]}.
   In addition, \param{p} must be a prime \linkingone{equation}{integer}. \\
%
  \subsection{e3 -- solve equation with degree 3}\linkedone{equation}{e3}
   \func{e3}{\hiki{f}{\linkingone{equation}{poly\_list}}}{\out{list}}\\
   \spacing
   % document of basic document
   \quad Return the solution of cubic equation $ax^3 + bx^2 + cx + d = 0$.\\
   \spacing
   % added document
   %\spacing
   % input, output document
   \quad \param{f} ought to be a \linkingone{equation}{poly\_list} {\tt [d, c, b, a]} of \linkingone{equation}{complex}. \\
   The result tuple will contain exactly 3 roots, even in the case of including
   double roots.\\
%
  \subsection{e3\_Fp -- solve congruent equation modulo p with degree 3}\linkedone{equation}{e3\_Fp}
   \func{e3\_Fp}{\hiki{f}{\linkingone{equation}{poly\_list}},\ \hiki{p}{integer}}{\out{list}}\\
   \spacing
   % document of basic document
   \quad Return the solutions of $ax^3 + bx^2 + cx + d  \equiv 0 \pmod{\param{p}}$.\\
   \spacing
   % added document
   \quad If the same values are returned, then the values are multiple roots. \\
   \spacing
   % input, output document
   \quad \param{f} ought be a \linkingone{equation}{poly\_list} {\tt [d, c, b, a]} of \linkingone{equation}{integer}.
   In addition, \param{p} must be a prime \linkingone{equation}{integer}. \\
  \subsection{Newton -- solve equation using Newton's method}\linkedone{equation}{Newton}
   \func{Newton}{%
     \hiki{f}{\linkingone{equation}{poly\_list}},\ %
     \hikiopt{initial}{\linkingone{equation}{complex}}{1},\ %
     \hikiopt{repeat}{integer}{250}}{\out{complex}}\\
   \spacing
   % document of basic document
   \quad Return one of the approximated roots of $a_nx^n + \cdots + a_1x + a_0=0$.\\
   \spacing
   % added document
   \quad If you want to obtain all roots, then use \linkingone{equation}{SimMethod} instead.\\
   \negok If \param{initial} is a real number but there is no real roots, then this function returns meaningless values. \\
   \spacing
   % input, output document
   \quad \param{f} ought to be a \linkingone{equation}{poly\_list} of
   \linkingone{equation}{complex}.
   \param{initial} is an initial approximation \linkingone{equation}{complex} number.
   \param{repeat} is the number of steps to approximate a root.\\
%
  \subsection{SimMethod -- find all roots simultaneously}\linkedone{equation}{SimMethod}
   \func{SimMethod}{%
     \hiki{f}{\linkingone{equation}{poly\_list}},\ %
     \hikiopt{NewtonInitial}{\linkingone{equation}{complex}}{1},\ %
     \hikiopt{repeat}{integer}{250}}{\out{list}}\\
   \spacing
   % document of basic document
   \quad Return the approximated roots of $a_nx^n + \cdots + a_1x + a_0$.\\
   \spacing
   % added document
   \quad \negok If the equation has multiple root, maybe raise some error. \\
   \spacing
   % input, output document
   \quad \param{f} ought to be a \linkingone{equation}{poly\_list} of
   \linkingone{equation}{complex}.\\
   \param{NewtonInitial} and \param{repeat} will be passed to 
   \linkingone{equation}{Newton} to obtain the first approximations.\\
%
  \subsection{root\_Fp --  solve congruent equation modulo p}\linkedone{equation}{root\_Fp}
   \func{root\_Fp}{\hiki{f}{\linkingone{equation}{poly\_list}},\ \hiki{p}{integer}}{\out{integer}}\\
   \spacing
   % document of basic document
   \quad Return one of the roots of $a_nx^n + \cdots + a_1x + a_0 \equiv 0 \pmod{\param{p}}$. \\
   \spacing
   % added document
   \quad If you want to obtain all roots, then use \linkingone{equation}{allroots\_Fp}.\\
   \spacing
   % input, output document
   \quad \param{f} ought to be a \linkingone{equation}{poly\_list} of
   \linkingone{equation}{integer}.
   In addition, \param{p} must be a prime \linkingone{equation}{integer}. \\
   \quad If there is no root at all, then nothing will be returned.\\
%
  \subsection{allroots\_Fp -- solve congruent equation modulo p}\linkedone{equation}{allroots\_Fp}
   \func{allroots\_Fp}{\hiki{f}{\linkingone{equation}{poly\_list}},\ \hiki{p}{integer}}{\out{integer}}\\
   \spacing
   % document of basic document
   \quad Return all roots of $a_nx^n + \cdots + a_1x + a_0 \equiv 0 \pmod{\param{p}}$. \\
   \spacing
   % added document
   %\spacing
   % input, output document
   \quad \param{f} ought to be a \linkingone{equation}{poly\_list} of
   \linkingone{equation}{integer}.
   In addition, \param{p} must be a prime \linkingone{equation}{integer}. \\
   \quad If there is no root at all, then an empty list will be returned.\\
%
\begin{ex}
>>> equation.e1([1, 2])
-0.5
>>> equation.e1([1j, 2])
-0.5j
>>> equation.e1_ZnZ([3, 2], 5)
1
>>> equation.e2([-3, 1, 1])
(1.3027756377319946, -2.3027756377319948)
>>> equation.e2_Fp([-3, 1, 1], 13)
[6, 6]
>>> equation.e3([1, 1, 2, 1])
[(-0.12256116687665397-0.74486176661974479j), 
(-1.7548776662466921+1.8041124150158794e-16j), 
(-0.12256116687665375+0.74486176661974468j)]
>>> equation.e3_Fp([1, 1, 2, 1], 7)
[3]
>>> equation.Newton([-3, 2, 1, 1])
0.84373427789806899
>>> equation.Newton([-3, 2, 1, 1], 2)
0.84373427789806899
>>> equation.Newton([-3, 2, 1, 1], 2, 1000)
0.84373427789806899
>>> equation.SimMethod([-3, 2, 1, 1])
[(0.84373427789806887+0j), 
(-0.92186713894903438+1.6449263775999723j), 
(-0.92186713894903438-1.6449263775999723j)]
>>> equation.root_Fp([-3, 2, 1, 1], 7)
>>> equation.root_Fp([-3, 2, 1, 1], 11)
9L
>>> equation.allroots_Fp([-3, 2, 1, 1], 7)
[]
>>> equation.allroots_Fp([-3, 2, 1, 1], 11)
[9L]
>>> equation.allroots_Fp([-3, 2, 1, 1], 13)
[3L, 7L, 2L]
\end{ex}%Don't indent!(indent causes an error.)
\C

%---------- end document ---------- %

\bibliographystyle{jplain}%use jbibtex
\bibliography{nzmath_references}

\end{document}

