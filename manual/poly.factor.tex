\documentclass{report}

\documentclass{report}

%%%%%%%%%%%%%%%%%%%%%%%%%%%%%%%%%%%%%%%%%%%%%%%%%%%%%%%%%%%%%
%
% macros for nzmath manual
%
%%%%%%%%%%%%%%%%%%%%%%%%%%%%%%%%%%%%%%%%%%%%%%%%%%%%%%%%%%%%%
\usepackage{amssymb,amsmath}
\usepackage{color}
\usepackage[dvipdfm,bookmarks=true,bookmarksnumbered=true,%
 pdftitle={NZMATH Users Manual},%
 pdfsubject={Manual for NZMATH Users},%
 pdfauthor={NZMATH Development Group},%
 pdfkeywords={TeX; dvipdfmx; hyperref; color;},%
 colorlinks=true]{hyperref}
\usepackage{fancybox}
\usepackage[T1]{fontenc}
%
\newcommand{\DS}{\displaystyle}
\newcommand{\C}{\clearpage}
\newcommand{\NO}{\noindent}
\newcommand{\negok}{$\dagger$}
\newcommand{\spacing}{\vspace{1pt}\\ }
% software macros
\newcommand{\nzmathzero}{{\footnotesize $\mathbb{N}\mathbb{Z}$}\texttt{MATH}}
\newcommand{\nzmath}{{\nzmathzero}\ }
\newcommand{\pythonzero}{$\mbox{\texttt{Python}}$}
\newcommand{\python}{{\pythonzero}\ }
% link macros
\newcommand{\linkingzero}[1]{{\bf \hyperlink{#1}{#1}}}%module
\newcommand{\linkingone}[2]{{\bf \hyperlink{#1.#2}{#2}}}%module,class/function etc.
\newcommand{\linkingtwo}[3]{{\bf \hyperlink{#1.#2.#3}{#3}}}%module,class,method
\newcommand{\linkedzero}[1]{\hypertarget{#1}{}}
\newcommand{\linkedone}[2]{\hypertarget{#1.#2}{}}
\newcommand{\linkedtwo}[3]{\hypertarget{#1.#2.#3}{}}
\newcommand{\linktutorial}[1]{\href{http://docs.python.org/tutorial/#1}{#1}}
\newcommand{\linktutorialone}[2]{\href{http://docs.python.org/tutorial/#1}{#2}}
\newcommand{\linklibrary}[1]{\href{http://docs.python.org/library/#1}{#1}}
\newcommand{\linklibraryone}[2]{\href{http://docs.python.org/library/#1}{#2}}
\newcommand{\pythonhp}{\href{http://www.python.org/}{\python website}}
\newcommand{\nzmathwiki}{\href{http://nzmath.sourceforge.net/wiki/}{{\nzmathzero}Wiki}}
\newcommand{\nzmathsf}{\href{http://sourceforge.net/projects/nzmath/}{\nzmath Project Page}}
\newcommand{\nzmathtnt}{\href{http://tnt.math.se.tmu.ac.jp/nzmath/}{\nzmath Project Official Page}}
% parameter name
\newcommand{\param}[1]{{\tt #1}}
% function macros
\newcommand{\hiki}[2]{{\tt #1}:\ {\em #2}}
\newcommand{\hikiopt}[3]{{\tt #1}:\ {\em #2}=#3}

\newdimen\hoge
\newdimen\truetextwidth
\newcommand{\func}[3]{%
\setbox0\hbox{#1(#2)}
\hoge=\wd0
\truetextwidth=\textwidth
\advance \truetextwidth by -2\oddsidemargin
\ifdim\hoge<\truetextwidth % short form
{\bf \colorbox{skyyellow}{#1(#2)\ $\to$ #3}}
%
\else % long form
\fcolorbox{skyyellow}{skyyellow}{%
   \begin{minipage}{\textwidth}%
   {\bf #1(#2)\\ %
    \qquad\quad   $\to$\ #3}%
   \end{minipage}%
   }%
\fi%
}

\newcommand{\out}[1]{{\em #1}}
\newcommand{\initialize}{%
  \paragraph{\large \colorbox{skyblue}{Initialize (Constructor)}}%
    \quad\\ %
    \vspace{3pt}\\
}
\newcommand{\method}{\C \paragraph{\large \colorbox{skyblue}{Methods}}}
% Attribute environment
\newenvironment{at}
{%begin
\paragraph{\large \colorbox{skyblue}{Attribute}}
\quad\\
\begin{description}
}%
{%end
\end{description}
}
% Operation environment
\newenvironment{op}
{%begin
\paragraph{\large \colorbox{skyblue}{Operations}}
\quad\\
\begin{table}[h]
\begin{center}
\begin{tabular}{|l|l|}
\hline
operator & explanation\\
\hline
}%
{%end
\hline
\end{tabular}
\end{center}
\end{table}
}
% Examples environment
\newenvironment{ex}%
{%begin
\paragraph{\large \colorbox{skyblue}{Examples}}
\VerbatimEnvironment
\renewcommand{\EveryVerbatim}{\fontencoding{OT1}\selectfont}
\begin{quote}
\begin{Verbatim}
}%
{%end
\end{Verbatim}
\end{quote}
}
%
\definecolor{skyblue}{cmyk}{0.2, 0, 0.1, 0}
\definecolor{skyyellow}{cmyk}{0.1, 0.1, 0.5, 0}
%
%\title{NZMATH User Manual\\ {\large{(for version 1.0)}}}
%\date{}
%\author{}
\begin{document}
%\maketitle
%
\setcounter{tocdepth}{3}
\setcounter{secnumdepth}{3}


\tableofcontents
\C

\chapter{Functions}


%---------- start document ---------- %
 \section{poly.factor -- polynomial factorization}\linkedzero{poly.factor}
 The factor module is for factorizations of integer coefficient univariate polynomials.


 This module using following type:
 \begin{description}
   \item[polynomial]\linkedone{poly.factor}{polynomial}:\\
     \param{polynomial} is the polynomial generated by function poly.uniutil.polynomial. 
 \end{description}

%
  \subsection{brute\_force\_search -- search factorization by brute force}\linkedone{poly.factor}{brute\_force\_search}
   \func{brute\_force\_search}
   {%
     \hiki{f}{poly.uniutil.IntegerPolynomial},\ %
     \hiki{fp\_factors}{list},\ %
     \hiki{q}{integer}%
   }{%
     \out{[factors]}%
   }\\
   \spacing
   % document of basic document
   \quad Find the factorization of \param{f} by searching a factor which is a product of some combination in \param{fp\_factors}. The combination is searched by brute force.
   \spacing
   % added document
   The argument \param{fp\_factors} is a list of poly.uniutil.FinitePrimeFieldPolynomial .
   \spacing
   % input, output document
   %\quad \\
%
  \subsection{divisibility\_test -- divisibility test}\linkedone{poly.factor}{divisibility\_test}
   \func{divisibility\_test}
        {\hiki{f}{polynomial},\ %
         \hiki{g}{polynomial}%
        }
        {\out{bool}}\\
   \spacing
   % document of basic document
   \quad Return Boolean value whether \param{f} is divisible by \param{g} or not, for polynomials.
   \spacing
   % added document
   %\quad \negok Note that this function returns Hilbert class polynomial as a list of coefficients.\\
   %\spacing
   % input, output document
   %\quad \param{D} must be negative int or long. See \cite{Pomerance}.\\
%
  \subsection{minimum\_absolute\_injection -- send coefficients to minimum absolute representation }\linkedone{poly.factor}{minimum\_absolute\_injection}
   \func{minimum\_absolute\_injection}
        {\hiki{f}{polynomial}}
        {\out{F}}\\
   \spacing
   % document of basic document
   \quad Return an integer coefficient polynomial F by injection of a $\mathbf{Z}/p\mathbf{Z}$ coefficient polynomial \param{f} with sending each coefficient to minimum absolute representatives.
   \spacing
   % added document
   %\quad \negok 
   %\spacing
   % input, output document
   \quad The coefficient ring of given polynomial \param{f} must be \linkingone{intresidue}{IntegerResidueClassRing} or \linkingone{finitefield}{FinitePrimeField}.\\
%
  \subsection{padic\_factorization -- p-adic factorization}\linkedone{poly.factor}{padic\_factorization}
   \func{padic\_factorization}
        {\hiki{f}{polynomial}}
        {\out{p}, \out{factors}}\\
   \spacing
   % document of basic document
   \quad Return a prime \param{p} and a p-adic factorization of given integer coefficient squarefree polynomial \param{f}. The result \param{factors} have integer coefficients, injected from $\mathbb{F}_p$ to its minimum absolute representation.
   \spacing
   % added document
   \quad \negok The prime is chosen to be:
   \begin{enumerate}
   \item \param{f} is still squarefree mod \param{p},
   \item the number of factors is not greater than with the successive prime.
   \end{enumerate}
   %\spacing
   % input, output document
   \quad The given polynomial \param{f} must be poly.uniutil.IntegerPolynomial .\\
%
  \subsection{upper\_bound\_of\_coefficient --Landau-Mignotte bound of coefficients}\linkedone{poly.factor}{upper\_bound\_of\_coefficient}
   \func{upper\_bound\_of\_coefficient}
        {\hiki{f}{polynomial}}
        {\out{long}}\\
   \spacing
   % document of basic document
   \quad Compute Landau-Mignotte bound of coefficients of factors, whose degree is no greater than half of the given \param{f}.
   \spacing
   % added document
   %\quad \negok Additional argument \param{floatpre} specifies the precision of calculation in decimal digits.
   %\spacing
   % input, output document
   \quad The given polynomial \param{f} must have integer coefficients.\\
%
  \subsection{zassenhaus -- squarefree integer polynomial factorization by Zassenhaus method}\linkedone{poly.factor}{zassenhaus}
   \func{zassenhaus}
        {\hiki{f}{polynomial}}
        {\out{list of factors f}}\\
   \spacing
   % document of basic document
   \quad Factor a squarefree integer coefficient polynomial \param{f} with Berlekamp-Zassenhaus method.
   \spacing
   % added document
   %\quad 
   %\spacing
   % input, output document
   %\quad output must be list of factors.
%
  \subsection{integerpolynomialfactorization -- Integer polynomial factorization}\linkedone{poly.factor}{integerpolynomialfactorization}
   \func{integerpolynomialfactorization}
        {\hiki{f}{polynomial}}
        {\out{factor}}\\
   \spacing
   % document of basic document
   \quad Factor an integer coefficient polynomial \param{f} with Berlekamp-Zassenhaus method.
   \spacing
   % added document
   %\quad \param{p} must be a prime integer and \param{d} be an integer such that 0 < \param{d} < 4\param{p} with $-\mathtt{d} \equiv 0, 1 \pmod{4}$. 
   \spacing
   % input, output document
   \quad factor output by the form of list of tuples that formed (factor, index). \\
%
%\begin{ex}
%>>> module.func1(1, 0.1, "a", [], (1, 2))
%(2, "b")
%>>> module.func2()
%1
%\end{ex}%Don't indent!(indent causes an error.)
\C

%---------- end document ---------- %

\bibliographystyle{jplain}%use jbibtex
\bibliography{nzmath_references}

\end{document}

