\documentclass{report}

\documentclass{report}

%%%%%%%%%%%%%%%%%%%%%%%%%%%%%%%%%%%%%%%%%%%%%%%%%%%%%%%%%%%%%
%
% macros for nzmath manual
%
%%%%%%%%%%%%%%%%%%%%%%%%%%%%%%%%%%%%%%%%%%%%%%%%%%%%%%%%%%%%%
\usepackage{amssymb,amsmath}
\usepackage{color}
\usepackage[dvipdfm,bookmarks=true,bookmarksnumbered=true,%
 pdftitle={NZMATH Users Manual},%
 pdfsubject={Manual for NZMATH Users},%
 pdfauthor={NZMATH Development Group},%
 pdfkeywords={TeX; dvipdfmx; hyperref; color;},%
 colorlinks=true]{hyperref}
\usepackage{fancybox}
\usepackage[T1]{fontenc}
%
\newcommand{\DS}{\displaystyle}
\newcommand{\C}{\clearpage}
\newcommand{\NO}{\noindent}
\newcommand{\negok}{$\dagger$}
\newcommand{\spacing}{\vspace{1pt}\\ }
% software macros
\newcommand{\nzmathzero}{{\footnotesize $\mathbb{N}\mathbb{Z}$}\texttt{MATH}}
\newcommand{\nzmath}{{\nzmathzero}\ }
\newcommand{\pythonzero}{$\mbox{\texttt{Python}}$}
\newcommand{\python}{{\pythonzero}\ }
% link macros
\newcommand{\linkingzero}[1]{{\bf \hyperlink{#1}{#1}}}%module
\newcommand{\linkingone}[2]{{\bf \hyperlink{#1.#2}{#2}}}%module,class/function etc.
\newcommand{\linkingtwo}[3]{{\bf \hyperlink{#1.#2.#3}{#3}}}%module,class,method
\newcommand{\linkedzero}[1]{\hypertarget{#1}{}}
\newcommand{\linkedone}[2]{\hypertarget{#1.#2}{}}
\newcommand{\linkedtwo}[3]{\hypertarget{#1.#2.#3}{}}
\newcommand{\linktutorial}[1]{\href{http://docs.python.org/tutorial/#1}{#1}}
\newcommand{\linktutorialone}[2]{\href{http://docs.python.org/tutorial/#1}{#2}}
\newcommand{\linklibrary}[1]{\href{http://docs.python.org/library/#1}{#1}}
\newcommand{\linklibraryone}[2]{\href{http://docs.python.org/library/#1}{#2}}
\newcommand{\pythonhp}{\href{http://www.python.org/}{\python website}}
\newcommand{\nzmathwiki}{\href{http://nzmath.sourceforge.net/wiki/}{{\nzmathzero}Wiki}}
\newcommand{\nzmathsf}{\href{http://sourceforge.net/projects/nzmath/}{\nzmath Project Page}}
\newcommand{\nzmathtnt}{\href{http://tnt.math.se.tmu.ac.jp/nzmath/}{\nzmath Project Official Page}}
% parameter name
\newcommand{\param}[1]{{\tt #1}}
% function macros
\newcommand{\hiki}[2]{{\tt #1}:\ {\em #2}}
\newcommand{\hikiopt}[3]{{\tt #1}:\ {\em #2}=#3}

\newdimen\hoge
\newdimen\truetextwidth
\newcommand{\func}[3]{%
\setbox0\hbox{#1(#2)}
\hoge=\wd0
\truetextwidth=\textwidth
\advance \truetextwidth by -2\oddsidemargin
\ifdim\hoge<\truetextwidth % short form
{\bf \colorbox{skyyellow}{#1(#2)\ $\to$ #3}}
%
\else % long form
\fcolorbox{skyyellow}{skyyellow}{%
   \begin{minipage}{\textwidth}%
   {\bf #1(#2)\\ %
    \qquad\quad   $\to$\ #3}%
   \end{minipage}%
   }%
\fi%
}

\newcommand{\out}[1]{{\em #1}}
\newcommand{\initialize}{%
  \paragraph{\large \colorbox{skyblue}{Initialize (Constructor)}}%
    \quad\\ %
    \vspace{3pt}\\
}
\newcommand{\method}{\C \paragraph{\large \colorbox{skyblue}{Methods}}}
% Attribute environment
\newenvironment{at}
{%begin
\paragraph{\large \colorbox{skyblue}{Attribute}}
\quad\\
\begin{description}
}%
{%end
\end{description}
}
% Operation environment
\newenvironment{op}
{%begin
\paragraph{\large \colorbox{skyblue}{Operations}}
\quad\\
\begin{table}[h]
\begin{center}
\begin{tabular}{|l|l|}
\hline
operator & explanation\\
\hline
}%
{%end
\hline
\end{tabular}
\end{center}
\end{table}
}
% Examples environment
\newenvironment{ex}%
{%begin
\paragraph{\large \colorbox{skyblue}{Examples}}
\VerbatimEnvironment
\renewcommand{\EveryVerbatim}{\fontencoding{OT1}\selectfont}
\begin{quote}
\begin{Verbatim}
}%
{%end
\end{Verbatim}
\end{quote}
}
%
\definecolor{skyblue}{cmyk}{0.2, 0, 0.1, 0}
\definecolor{skyyellow}{cmyk}{0.1, 0.1, 0.5, 0}
%
%\title{NZMATH User Manual\\ {\large{(for version 1.0)}}}
%\date{}
%\author{}
\begin{document}
%\maketitle
%
\setcounter{tocdepth}{3}
\setcounter{secnumdepth}{3}


\tableofcontents
\C

\chapter{Functions}


%---------- start document ---------- %
 \section{factor.find -- find a factor}\linkedzero{factor.find}
%
\quad All methods in this module return one of a factor of given integer.
If it failes to find a non-trivial factor, it returns \(1\).
Note that \(1\) is a factor anyway.

\param{verbose} boolean flag can be specified for verbose reports.
To receive these messages, you have to prepare a logger (see \linklibrary{logging}).

  \subsection{trialDivision -- trial division}\linkedone{factor.find}{trialDivision}
   \func{trialDivision}
   {%
     \hiki{n}{integer},\ %
     **\param{options}
   }{%
     \out{integer}
   }\\
   \spacing
   % document of basic document
   \quad Return a factor of \param{n} by trial divisions.\\
   \spacing
   % input, output document
   \quad \param{options} can be either one of the following:
   \begin{enumerate}
   \item \param{start} and \param{stop} as range parameters.
         In addition to these, \param{step} is also available.
   \item \param{iterator} as an iterator of primes.
   \end{enumerate}
   If \param{options} is not given, the function divides \param{n} by primes from \(2\) to the floor of the square root of \param{n} until a non-trivial factor is found.\\
   \quad \param{verbose} boolean flag can be specified for verbose reports.\\
%
  \subsection{pmom -- $p-1$ method}\linkedone{factor.find}{pmom}
   \func{pmom}{%
     \hiki{n}{integer},\ %
     **\param{options}
   }{%
     \out{integer}
   }\\
   \spacing
   % document of basic document
   \quad Return a factor of \param{n} by the \(p-1\) method.\\
   \spacing
   % added document
   \quad The function tries to find a non-trivial factor of \param{n}
   using Algorithm 8.8.2 (\(p-1\) first stage) of \cite{Cohen1}.
   In the case of \(n = 2^{i}\), the function will not terminate.
   Due to the nature of the method, the method may return the
   trivial factor only.\\
   \spacing
   % input, output document
   \quad \param{verbose} Boolean flag can be specified for verbose reports,
   though it is not so verbose indeed.\\
%
   \subsection{rhomethod -- $\rho$ method}\linkedone{factor.find}{rhomethod}
   \func{rhomethod}{%
     \hiki{n}{integer},\ %
     **\param{options}
   }{%
     \out{integer}
   }\\
   \spacing
   % document of basic document
   \quad Return a factor of \param{n} by Pollard's \(\rho\) method.\\
   \spacing
   % added document
   The implementation refers the explanation in \cite{Pomerance}.
   Due to the nature of the method, a factorization may return the
   trivial factor only.\\
   \spacing
   % input, output document
   \quad \param{verbose} Boolean flag can be specified for verbose reports.\\
\begin{ex}
>>> factor.find.trialDivision(1001)
7
>>> factor.find.trialDivision(1001, start=10, stop=32)
11
>>> factor.find.pmom(1001)
91
>>> import logging
>>> logging.basicConfig()
>>> factor.find.rhomethod(1001, verbose=True)
INFO:nzmath.factor.find:887 748
13
\end{ex}%Don't indent!(indent causes an error.)
\C

%---------- end document ---------- %

\bibliographystyle{jplain}%use jbibtex
\bibliography{nzmath_references}

\end{document}

