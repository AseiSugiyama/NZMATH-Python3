\documentclass{report}

%%%%%%%%%%%%%%%%%%%%%%%%%%%%%%%%%%%%%%%%%%%%%%%%%%%%%%%%%%%%%
%
% macros for nzmath manual
%
%%%%%%%%%%%%%%%%%%%%%%%%%%%%%%%%%%%%%%%%%%%%%%%%%%%%%%%%%%%%%
\usepackage{amssymb,amsmath}
\usepackage{color}
\usepackage[dvipdfm,bookmarks=true,bookmarksnumbered=true,%
 pdftitle={NZMATH Users Manual},%
 pdfsubject={Manual for NZMATH Users},%
 pdfauthor={NZMATH Development Group},%
 pdfkeywords={TeX; dvipdfmx; hyperref; color;},%
 colorlinks=true]{hyperref}
\usepackage{fancybox}
\usepackage[T1]{fontenc}
%
\newcommand{\DS}{\displaystyle}
\newcommand{\C}{\clearpage}
\newcommand{\NO}{\noindent}
\newcommand{\negok}{$\dagger$}
\newcommand{\spacing}{\vspace{1pt}\\ }
% software macros
\newcommand{\nzmathzero}{{\footnotesize $\mathbb{N}\mathbb{Z}$}\texttt{MATH}}
\newcommand{\nzmath}{{\nzmathzero}\ }
\newcommand{\pythonzero}{$\mbox{\texttt{Python}}$}
\newcommand{\python}{{\pythonzero}\ }
% link macros
\newcommand{\linkingzero}[1]{{\bf \hyperlink{#1}{#1}}}%module
\newcommand{\linkingone}[2]{{\bf \hyperlink{#1.#2}{#2}}}%module,class/function etc.
\newcommand{\linkingtwo}[3]{{\bf \hyperlink{#1.#2.#3}{#3}}}%module,class,method
\newcommand{\linkedzero}[1]{\hypertarget{#1}{}}
\newcommand{\linkedone}[2]{\hypertarget{#1.#2}{}}
\newcommand{\linkedtwo}[3]{\hypertarget{#1.#2.#3}{}}
\newcommand{\linktutorial}[1]{\href{http://docs.python.org/tutorial/#1}{#1}}
\newcommand{\linktutorialone}[2]{\href{http://docs.python.org/tutorial/#1}{#2}}
\newcommand{\linklibrary}[1]{\href{http://docs.python.org/library/#1}{#1}}
\newcommand{\linklibraryone}[2]{\href{http://docs.python.org/library/#1}{#2}}
\newcommand{\pythonhp}{\href{http://www.python.org/}{\python website}}
\newcommand{\nzmathwiki}{\href{http://nzmath.sourceforge.net/wiki/}{{\nzmathzero}Wiki}}
\newcommand{\nzmathsf}{\href{http://sourceforge.net/projects/nzmath/}{\nzmath Project Page}}
\newcommand{\nzmathtnt}{\href{http://tnt.math.metro-u.ac.jp/nzmath/}{\nzmath Project Official Page}}
% parameter name
\newcommand{\param}[1]{{\tt #1}}
% function macros
\newcommand{\hiki}[2]{{\tt #1}:\ {\em #2}}
\newcommand{\hikiopt}[3]{{\tt #1}:\ {\em #2}=#3}

\newdimen\hoge
\newdimen\truetextwidth
\newcommand{\func}[3]{%
\setbox0\hbox{#1(#2)}
\hoge=\wd0
\truetextwidth=\textwidth
\advance \truetextwidth by -2\oddsidemargin
\ifdim\hoge<\truetextwidth % short form
{\bf \colorbox{skyyellow}{#1(#2)\ $\to$ #3}}
%
\else % long form
\fcolorbox{skyyellow}{skyyellow}{%
   \begin{minipage}{\textwidth}%
   {\bf #1(#2)\\ %
    \qquad\quad   $\to$\ #3}%
   \end{minipage}%
   }%
\fi%
}

\newcommand{\out}[1]{{\em #1}}
\newcommand{\initialize}{%
  \paragraph{\large \colorbox{skyblue}{Initialize (Constructor)}}%
    \quad\\ %
    \vspace{3pt}\\
}
\newcommand{\method}{\C \paragraph{\large \colorbox{skyblue}{Methods}}}
% Attribute environment
\newenvironment{at}
{%begin
\paragraph{\large \colorbox{skyblue}{Attribute}}
\quad\\
\begin{description}
}%
{%end
\end{description}
}
% Operation environment
\newenvironment{op}
{%begin
\paragraph{\large \colorbox{skyblue}{Operations}}
\quad\\
\begin{table}[h]
\begin{center}
\begin{tabular}{|l|l|}
\hline
operator & explanation\\
\hline
}%
{%end
\hline
\end{tabular}
\end{center}
\end{table}
}
% Examples environment
\newenvironment{ex}%
{%begin
\paragraph{\large \colorbox{skyblue}{Examples}}
\VerbatimEnvironment
\renewcommand{\EveryVerbatim}{\fontencoding{OT1}\selectfont}
\begin{quote}
\begin{Verbatim}
}%
{%end
\end{Verbatim}
\end{quote}
}
%
\definecolor{skyblue}{cmyk}{0.2, 0, 0.1, 0}
\definecolor{skyyellow}{cmyk}{0.1, 0.1, 0.5, 0}
%
%\title{NZMATH User Manual\\ {\large{(for version 1.0)}}}
%\date{}
%\author{}
\begin{document}
%\maketitle
%
\setcounter{tocdepth}{3}
\setcounter{secnumdepth}{3}


\tableofcontents
\C

\chapter{Functions}


%---------- start document ---------- %
 \section{cubic\_root -- cubic root, residue, and so on}\linkedzero{cubic\_root}
%
  \subsection{c\_root\_p -- cubic root mod p}\linkedone{cubic\_root}{c\_root\_p}
   \func{c\_root\_p}{\hiki{a}{integer},\ \hiki{p}{integer}}{\out{list}}\\
   \spacing
   % document of basic document
   \quad Return the cubic root of \param{a} modulo prime \param{p}. (i.e. solutions of the equation $x^3 = \param{a} \pmod{\param{p}}$).\\
   \spacing
   % added document
   %\spacing
   % input, output document
   \quad \param{p} must be a prime integer.\\
   This function returns the list of all cubic roots of \param{a}.\\
%
  \subsection{c\_residue -- cubic residue mod p}\linkedone{cubic\_root}{c\_residue}
   \func{c\_residue}{\hiki{a}{integer},\ \hiki{p}{integer}}{\out{integer}}\\
   \spacing
   % document of basic document
   \quad Check whether the rational integer \param{a} is cubic residue modulo prime \param{p}.\\
   \spacing
   % added document
   \quad If $\param{p}\ |\ \param{a}$, then this function returns $0$,
   elif \param{a} is cubic residue modulo \param{p}, then it returns $1$,
   otherwise (i.e. cubic non-residue), it returns $-1$.\\
   \spacing
   % input, output document
   \quad \param{p} must be a prime integer.\\
%
  \subsection{c\_symbol -- cubic residue symbol for Eisenstein-integers}\linkedone{cubic\_root}{c\_symbol}
   \func{c\_symbol}{\hiki{a1}{integer},\ \hiki{a2}{integer},\ \hiki{b1}{integer},\ \hiki{b2}{integer}}{\out{integer}}\\
   \spacing
   % document of basic document
   \quad Return the (Jacobi) cubic residue symbol of two Eisenstein-integers $\left(\frac{\param{a1}+\param{a2}\omega}{\param{b1}+\param{b2}\omega}\right)_3$,
where $\omega$ is a primitive cubic root of unity.\\
   \spacing
   % added document
   If $\param{b1}+\param{b2}\omega$ is a prime in $\mathbb{Z}[\omega]$, it shows $\param{a1}+\param{a2}\omega$ is cubic residue or not.\\
   \spacing
   % input, output document
   \quad We assume that $\param{b1}+\param{b2}\omega$ is not divisible $1-\omega$.\\
%
  \subsection{decomposite\_p -- decomposition to Eisenstein-integers}\linkedone{cubic\_root}{decomposite\_p}
   \func{decomposite\_p}{\hiki{p}{integer}}{(\out{integer},\ \out{integer})}\\
   \spacing
   % document of basic document
   \quad Return one of prime factors of \param{p} in $\mathbb{Z}[\omega]$.\\
   \spacing
   % added document
   \quad If the output is (\param{a},\ \param{b}), then $\frac{\param{p}}{\param{a}+\param{b}\omega}$ is a prime in $\mathbb{Z}[\omega]$.
   In other words, \param{p} decomposes into two prime factors $\param{a}+\param{b}\omega$ and $\param{p}/(\param{a}+\param{b}\omega)$ in $\mathbb{Z}[\omega]$.\\
   \spacing
   % input, output document
   \quad \param{p} must be a prime rational integer.
   We assume that $\param{p}\equiv 1 \pmod 3$.\\
%
  \subsection{cornacchia -- solve $x^2+dy^2=p$}\linkedone{cubic\_root}{cornacchia}
   \func{cornacchia}{\hiki{d}{integer},\ \hiki{p}{integer}}{(\out{integer},\ \out{integer})}\\
   \spacing
   % document of basic document
   \quad Return the solution of $x^2 + \param{d}y^2 = \param{p}$.\\
   \spacing
   % added document
   \quad This function uses Cornacchia's algorithm. See \cite{Cohen1}.\\
   \spacing
   % input, output document
   \quad \param{p} must be prime rational integer.
   \param{d} must be satisfied with the condition $0<\param{d}<\param{p}$.
   This function returns (\param{x},\ \param{y}) as one of solutions of the equation $x^2 + \param{d} y^2 = \param{p}$.\\
%
\begin{ex}
>>> cubic_root.c_root_p(1, 13)
[1, 3, 9]
>>> cubic_root.c_residue(2, 7)
-1
>>> cubic_root.c_symbol(3, 6, 5, 6)
1
>>> cubic_root.decomposite_p(19)
(2, 5)
>>> cubic_root.cornacchia(5, 29)
(3, 2)
\end{ex}%Don't indent!(indent causes an error.)
\C

%---------- end document ---------- %

\bibliographystyle{jplain}%use jbibtex
\bibliography{nzmath_references}

\end{document}

