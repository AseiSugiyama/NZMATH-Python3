\documentclass{report}

\documentclass{report}

%%%%%%%%%%%%%%%%%%%%%%%%%%%%%%%%%%%%%%%%%%%%%%%%%%%%%%%%%%%%%
%
% macros for nzmath manual
%
%%%%%%%%%%%%%%%%%%%%%%%%%%%%%%%%%%%%%%%%%%%%%%%%%%%%%%%%%%%%%
\usepackage{amssymb,amsmath}
\usepackage{color}
\usepackage[dvipdfm,bookmarks=true,bookmarksnumbered=true,%
 pdftitle={NZMATH Users Manual},%
 pdfsubject={Manual for NZMATH Users},%
 pdfauthor={NZMATH Development Group},%
 pdfkeywords={TeX; dvipdfmx; hyperref; color;},%
 colorlinks=true]{hyperref}
\usepackage{fancybox}
\usepackage[T1]{fontenc}
%
\newcommand{\DS}{\displaystyle}
\newcommand{\C}{\clearpage}
\newcommand{\NO}{\noindent}
\newcommand{\negok}{$\dagger$}
\newcommand{\spacing}{\vspace{1pt}\\ }
% software macros
\newcommand{\nzmathzero}{{\footnotesize $\mathbb{N}\mathbb{Z}$}\texttt{MATH}}
\newcommand{\nzmath}{{\nzmathzero}\ }
\newcommand{\pythonzero}{$\mbox{\texttt{Python}}$}
\newcommand{\python}{{\pythonzero}\ }
% link macros
\newcommand{\linkingzero}[1]{{\bf \hyperlink{#1}{#1}}}%module
\newcommand{\linkingone}[2]{{\bf \hyperlink{#1.#2}{#2}}}%module,class/function etc.
\newcommand{\linkingtwo}[3]{{\bf \hyperlink{#1.#2.#3}{#3}}}%module,class,method
\newcommand{\linkedzero}[1]{\hypertarget{#1}{}}
\newcommand{\linkedone}[2]{\hypertarget{#1.#2}{}}
\newcommand{\linkedtwo}[3]{\hypertarget{#1.#2.#3}{}}
\newcommand{\linktutorial}[1]{\href{http://docs.python.org/tutorial/#1}{#1}}
\newcommand{\linktutorialone}[2]{\href{http://docs.python.org/tutorial/#1}{#2}}
\newcommand{\linklibrary}[1]{\href{http://docs.python.org/library/#1}{#1}}
\newcommand{\linklibraryone}[2]{\href{http://docs.python.org/library/#1}{#2}}
\newcommand{\pythonhp}{\href{http://www.python.org/}{\python website}}
\newcommand{\nzmathwiki}{\href{http://nzmath.sourceforge.net/wiki/}{{\nzmathzero}Wiki}}
\newcommand{\nzmathsf}{\href{http://sourceforge.net/projects/nzmath/}{\nzmath Project Page}}
\newcommand{\nzmathtnt}{\href{http://tnt.math.metro-u.ac.jp/nzmath/}{\nzmath Project Official Page}}
% parameter name
\newcommand{\param}[1]{{\tt #1}}
% function macros
\newcommand{\hiki}[2]{{\tt #1}:\ {\em #2}}
\newcommand{\hikiopt}[3]{{\tt #1}:\ {\em #2}=#3}

\newdimen\hoge
\newdimen\truetextwidth
\newcommand{\func}[3]{%
\setbox0\hbox{#1(#2)}
\hoge=\wd0
\truetextwidth=\textwidth
\advance \truetextwidth by -2\oddsidemargin
\ifdim\hoge<\truetextwidth % short form
{\bf \colorbox{skyyellow}{#1(#2)\ $\to$ #3}}
%
\else % long form
\fcolorbox{skyyellow}{skyyellow}{%
   \begin{minipage}{\textwidth}%
   {\bf #1(#2)\\ %
    \qquad\quad   $\to$\ #3}%
   \end{minipage}%
   }%
\fi%
}

\newcommand{\out}[1]{{\em #1}}
\newcommand{\initialize}{%
  \paragraph{\large \colorbox{skyblue}{Initialize (Constructor)}}%
    \quad\\ %
    \vspace{3pt}\\
}
\newcommand{\method}{\C \paragraph{\large \colorbox{skyblue}{Methods}}}
% Attribute environment
\newenvironment{at}
{%begin
\paragraph{\large \colorbox{skyblue}{Attribute}}
\quad\\
\begin{description}
}%
{%end
\end{description}
}
% Operation environment
\newenvironment{op}
{%begin
\paragraph{\large \colorbox{skyblue}{Operations}}
\quad\\
\begin{table}[h]
\begin{center}
\begin{tabular}{|l|l|}
\hline
operator & explanation\\
\hline
}%
{%end
\hline
\end{tabular}
\end{center}
\end{table}
}
% Examples environment
\newenvironment{ex}%
{%begin
\paragraph{\large \colorbox{skyblue}{Examples}}
\VerbatimEnvironment
\renewcommand{\EveryVerbatim}{\fontencoding{OT1}\selectfont}
\begin{quote}
\begin{Verbatim}
}%
{%end
\end{Verbatim}
\end{quote}
}
%
\definecolor{skyblue}{cmyk}{0.2, 0, 0.1, 0}
\definecolor{skyyellow}{cmyk}{0.1, 0.1, 0.5, 0}
%
%\title{NZMATH User Manual\\ {\large{(for version 1.0)}}}
%\date{}
%\author{}
\begin{document}
%\maketitle
%
\setcounter{tocdepth}{3}
\setcounter{secnumdepth}{3}


\tableofcontents
\C

\chapter{Classes}


%---------- start document ---------- %
 \section{poly.termorder -- term orders}\linkedzero{poly.termorder}
 \begin{itemize}
   \item {\bf Classes}
     \begin{itemize}
     \item \negok \linkingone{poly.termorder}{TermOrderInterface}
     \item \negok \linkingone{poly.termorder}{UnivarTermOrder}
     \item \linkingone{poly.termorder}{MultivarTermOrder}
     \end{itemize}
   \item {\bf Functions}
     \begin{itemize}
       \item \linkingone{poly.termorder}{weight\_order}
     \end{itemize}
 \end{itemize}

\C

 \subsection{TermOrderInterface -- interface of term order}\linkedone{poly.termorder}{TermOrderInterface}
 \initialize
  \func{TermOrderInterface}{\hiki{comparator}{function}}{\out{TermOrderInterface}}\\
  \spacing
  \quad A term order is primarily a function, which determines
  precedence between two terms (or monomials). By the precedence, all
  terms are ordered.\\
  \quad More precisely in terms of \python, a term order accepts two
  tuples of integers, each of which represents power indices of the
  term, and returns 0, 1 or -1 just like \linklibraryone{stdfuncs\#cmp}{cmp}
  built-in function.\\
  \quad A {\tt TermOrder} object provides not only the precedence function,
  but also a method to format a string for a polynomial, to tell
  degree, leading coefficients, etc.\\
  \spacing
  \quad \param{comparator} accepts two tuple-like objects of integers,
  each of which represents power indices of the term, and returns 0, 1
  or -1 just like {\tt cmp} built-in function.\\
  \spacing
  This class is abstract and should not be instantiated.
  The methods below have to be overridden.
  \method
  \subsubsection{cmp}\linkedtwo{poly.termorder}{TermOrderInterface}{cmp}
  \func{cmp}{\param{self},\ \hiki{left}{tuple},\ \hiki{right}{tuple}}{\out{integer}}\\
  \spacing
  \quad Compare two index tuples \param{left} and \param{right} and
  determine precedence.

  \subsubsection{format}\linkedtwo{poly.termorder}{TermOrderInterface}{format}
  \func{format}{\param{self},\ \hiki{polynom}{polynomial},\ **\hiki{keywords}{dict}}{\out{string}}\\
  \spacing
  \quad Return the formatted string of the polynomial \param{polynom}.

  \subsubsection{leading\_coefficient}\linkedtwo{poly.termorder}{TermOrderInterface}{leading\_coefficient}
  \func{leading\_coefficient}{\param{self},\ \hiki{polynom}{polynomial}}{\out{CommutativeRingElement}}\\
  \spacing
  \quad Return the leading coefficient of polynomial \param{polynom}
  with respect to the term order.

  \subsubsection{leading\_term}\linkedtwo{poly.termorder}{TermOrderInterface}{leading\_term}
  \func{leading\_term}{\param{self},\ \hiki{polynom}{polynomial}}{\out{tuple}}\\
  \spacing
  \quad Return the leading term of polynomial \param{polynom} as tuple
  of {\tt (degree index, coefficient)} with respect to the term order.

  \subsection{UnivarTermOrder -- term order for univariate polynomials}\linkedone{poly.termorder}{UnivarTermOrder}
  \initialize
  \func{UnivarTermOrder}{\hiki{comparator}{function}}{\out{UnivarTermOrder}}\\
  \spacing
  \quad There is one unique term order for univariate polynomials.
  It's known as degree.\\
  \quad One thing special to univariate case is that powers are not tuples
  but bare integers.
  According to the fact, method signatures also need be translated from
  the definitions in TermOrderInterface, but its easy, and we omit
  some explanations.\\
  \spacing
  \quad \param{comparator} can be any callable that accepts two integers
  and returns 0, 1 or -1 just like {\tt cmp}, i.e. if they are equal
  it returns 0, first one is greater 1, and otherwise -1.
  Theoretically acceptable comparator is only the {\tt cmp} function.\\
  \spacing
  \quad This class inherits \linkingone{poly.termorder}{TermOrderInterface}.
  \method
  \subsubsection{format}\linkedtwo{poly.termorder}{UnivarTermOrder}{format}
  \func{format}{\param{self},\ \hiki{polynom}{polynomial},\ %
    \hikiopt{varname}{string}{'X'}, \hikiopt{reverse}{bool}{False}}{%
    \out{string}}\\
  \spacing
  \quad Return the formatted string of the polynomial \param{polynom}.\\
  \spacing
  \begin{itemize}
  \item \param{polynom} must be a univariate polynomial.
  \item \param{varname} can be set to the name of the variable.
  \item \param{reverse} can be either {\tt True} or {\tt False}.
    If it's {\tt True}, terms appear in reverse (descending) order.
  \end{itemize}

  \subsubsection{degree}\linkedtwo{poly.termorder}{UnivarTermOrder}{degree}
  \func{degree}{\param{self},\ \hiki{polynom}{polynomial}}{\out{integer}}\\
  \spacing
  \quad Return the degree of the polynomial \param{polynom}.

  \subsubsection{tail\_degree}\linkedtwo{poly.termorder}{UnivarTermOrder}{tail\_degree}
  \func{tail\_degree}{\param{self},\ \hiki{polynom}{polynomial}}{\out{integer}}\\
  \spacing
  \quad Return the least degree among all terms of the \param{polynom}.\\
  \spacing
  \quad This method is {\em experimental}.

  \subsection{MultivarTermOrder -- term order for multivariate polynomials}\linkedone{poly.termorder}{MultivarTermOrder}
  \initialize
  \func{MultivarTermOrder}{\hiki{comparator}{function}}{\out{MultivarTermOrder}}\\
  \spacing
  \quad This class inherits \linkingone{poly.termorder}{TermOrderInterface}.

  \method
  \subsubsection{format}\linkedtwo{poly.termorder}{MultivarTermOrder}{format}
  \func{format}{\param{self},\ \hiki{polynom}{polynomial},\ %
    \hikiopt{varname}{tuple}{None}, \hikiopt{reverse}{bool}{False},\ %
    **\hiki{kwds}{dict}}{%
    \out{string}}\\
  \spacing
  \quad Return the formatted string of the polynomial \param{polynom}.\\
  \spacing
  \quad  An additional argument \param{varnames} is required to name
  variables.\\

  \begin{itemize}
  \item \param{polynom} is a multivariate polynomial.
  \item \param{varnames} is the sequence of the variable names.
  \item \param{reverse} can be either {\tt True} or {\tt False}.
    If it's {\tt True}, terms appear in reverse (descending) order.
  \end{itemize}

  \subsection{weight\_order -- weight order}\linkedone{poly.termorder}{weight\_order}
  \func{weight\_order}{\hiki{weight}{sequence},\ \hikiopt{tie\_breaker}{function}{None}}{\out{function}}\\
  \spacing
   % document of basic document
   \quad Return a comparator of weight ordering by \param{weight}.\\
   \spacing
   \quad Let \(w\) denote the \param{weight}.
   The weight ordering is defined for arguments \(x\) and \(y\) that
   \(x < y\) if \(w\cdot x < w\cdot y\) or \(w\cdot x == w\cdot y\) and 
   tie breaker tells \(x < y\).\\
   \spacing
   \quad The option \param{tie\_breaker} is another comparator that will
   be used if dot products with the weight vector leaves arguments tie.
   If the option is {\tt None} (default) and a tie breaker is indeed necessary
   to order given arguments, a {\tt TypeError} is raised.

\begin{ex}
>>> w = termorder.MultivarTermOrder(
...     termorder.weight_order((6, 3, 1), cmp))
>>> w.cmp((1, 0, 0), (0, 1, 2))
1
\end{ex}%Don't indent!
\C

%---------- end document ---------- %

\bibliographystyle{jplain}%use jbibtex
\bibliography{nzmath_references}

\end{document}

