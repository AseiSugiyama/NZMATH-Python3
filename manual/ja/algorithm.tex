\documentclass{report}

%%%%%%%%%%%%%%%%%%%%%%%%%%%%%%%%%%%%%%%%%%%%%%%%%%%%%%%%%%%%%
%
% macros for nzmath manual
%
%%%%%%%%%%%%%%%%%%%%%%%%%%%%%%%%%%%%%%%%%%%%%%%%%%%%%%%%%%%%%
\usepackage{amssymb,amsmath}
\usepackage{color}
\usepackage[dvipdfm,bookmarks=true,bookmarksnumbered=true,%
 pdftitle={NZMATH Users Manual},%
 pdfsubject={Manual for NZMATH Users},%
 pdfauthor={NZMATH Development Group},%
 pdfkeywords={TeX; dvipdfmx; hyperref; color;},%
 colorlinks=true]{hyperref}
\usepackage{fancybox}
\usepackage[T1]{fontenc}
%
\newcommand{\DS}{\displaystyle}
\newcommand{\C}{\clearpage}
\newcommand{\NO}{\noindent}
\newcommand{\negok}{$\dagger$}
\newcommand{\spacing}{\vspace{1pt}\\ }
% software macros
\newcommand{\nzmathzero}{{\footnotesize $\mathbb{N}\mathbb{Z}$}\texttt{MATH}}
\newcommand{\nzmath}{{\nzmathzero}\ }
\newcommand{\pythonzero}{$\mbox{\texttt{Python}}$}
\newcommand{\python}{{\pythonzero}\ }
% link macros
\newcommand{\linkingzero}[1]{{\bf \hyperlink{#1}{#1}}}%module
\newcommand{\linkingone}[2]{{\bf \hyperlink{#1.#2}{#2}}}%module,class/function etc.
\newcommand{\linkingtwo}[3]{{\bf \hyperlink{#1.#2.#3}{#3}}}%module,class,method
\newcommand{\linkedzero}[1]{\hypertarget{#1}{}}
\newcommand{\linkedone}[2]{\hypertarget{#1.#2}{}}
\newcommand{\linkedtwo}[3]{\hypertarget{#1.#2.#3}{}}
\newcommand{\linktutorial}[1]{\href{http://docs.python.org/tutorial/#1}{#1}}
\newcommand{\linktutorialone}[2]{\href{http://docs.python.org/tutorial/#1}{#2}}
\newcommand{\linklibrary}[1]{\href{http://docs.python.org/library/#1}{#1}}
\newcommand{\linklibraryone}[2]{\href{http://docs.python.org/library/#1}{#2}}
\newcommand{\pythonhp}{\href{http://www.python.org/}{\python website}}
\newcommand{\nzmathwiki}{\href{http://nzmath.sourceforge.net/wiki/}{{\nzmathzero}Wiki}}
\newcommand{\nzmathsf}{\href{http://sourceforge.net/projects/nzmath/}{\nzmath Project Page}}
\newcommand{\nzmathtnt}{\href{http://tnt.math.se.tmu.ac.jp/nzmath/}{\nzmath Project Official Page}}
% parameter name
\newcommand{\param}[1]{{\tt #1}}
% function macros
\newcommand{\hiki}[2]{{\tt #1}:\ {\em #2}}
\newcommand{\hikiopt}[3]{{\tt #1}:\ {\em #2}=#3}

\newdimen\hoge
\newdimen\truetextwidth
\newcommand{\func}[3]{%
\setbox0\hbox{#1(#2)}
\hoge=\wd0
\truetextwidth=\textwidth
\advance \truetextwidth by -2\oddsidemargin
\ifdim\hoge<\truetextwidth % short form
{\bf \colorbox{skyyellow}{#1(#2)\ $\to$ #3}}
%
\else % long form
\fcolorbox{skyyellow}{skyyellow}{%
   \begin{minipage}{\textwidth}%
   {\bf #1(#2)\\ %
    \qquad\quad   $\to$\ #3}%
   \end{minipage}%
   }%
\fi%
}

\newcommand{\out}[1]{{\em #1}}
\newcommand{\initialize}{%
  \paragraph{\large \colorbox{skyblue}{Initialize (Constructor)}}%
    \quad\\ %
    \vspace{3pt}\\
}
\newcommand{\method}{\C \paragraph{\large \colorbox{skyblue}{Methods}}}
% Attribute environment
\newenvironment{at}
{%begin
\paragraph{\large \colorbox{skyblue}{Attribute}}
\quad\\
\begin{description}
}%
{%end
\end{description}
}
% Operation environment
\newenvironment{op}
{%begin
\paragraph{\large \colorbox{skyblue}{Operations}}
\quad\\
\begin{table}[h]
\begin{center}
\begin{tabular}{|l|l|}
\hline
operator & explanation\\
\hline
}%
{%end
\hline
\end{tabular}
\end{center}
\end{table}
}
% Examples environment
\newenvironment{ex}%
{%begin
\paragraph{\large \colorbox{skyblue}{Examples}}
\VerbatimEnvironment
\renewcommand{\EveryVerbatim}{\fontencoding{OT1}\selectfont}
\begin{quote}
\begin{Verbatim}
}%
{%end
\end{Verbatim}
\end{quote}
}
%
\definecolor{skyblue}{cmyk}{0.2, 0, 0.1, 0}
\definecolor{skyyellow}{cmyk}{0.1, 0.1, 0.5, 0}
%
%\title{NZMATH User Manual\\ {\large{(for version 1.0)}}}
%\date{}
%\author{}
\begin{document}
%\maketitle
%
\setcounter{tocdepth}{3}
\setcounter{secnumdepth}{3}


\tableofcontents
\C

\chapter{Functions}


%---------- start document ---------- %
 \section{algorithm -- basic number theoretic algorithms}\linkedzero{algorithm}
%
  \subsection{digital\_method -- univariate polynomial evaluation}\linkedone{algorithm}{digital\_method}
   \func{digital\_method}
   {%
     \hiki{coefficients}{list},\ %
     \hiki{val}{object},\ %
     \hiki{add}{function},\ %
     \hiki{mul}{function},\ %
     \hiki{act}{function},\ %
     \hiki{power}{function},\ %
     \hiki{zero}{object},\ %
     \hiki{one}{object}
   }{%
     \out{object}%
   }\\
   \spacing
   % document of basic document
   \quad Evaluate a univariate polynomial corresponding to \param{coefficients} at \param{val}.\\
   \spacing
   % added document
   %\spacing
   % input, output document
   \quad  If the polynomial corresponding to \param{coefficients} is of $R$-coefficients for some ring $R$, 
   then \param{val} should be in an $R$-algebra $D$.\\
   \param{coefficients} should be a descending ordered list of tuples $(d,\ c)$, 
   where $d$ is an integer which expresses the degree and $c$ is an element of $R$ which expresses the coefficient. 
   All operations 'add', 'mul', 'act', 'power', 'zero', 'one' should be explicitly given, where:\\
      'add' means addition $(D \times D \to D)$,
      'mul' multiplication $(D \times D \to D)$,
      'act' action of $R$ $(R \times D \to D)$,
      'power' powering $(D \times \mathbf{Z} \to D)$,
      'zero' the additive unit (an constant) in $D$ and
      'one', the multiplicative unit (an constant) in $D$.
%
  \subsection{digital\_method\_func -- function of univariate polynomial evaluation}\linkedone{algorithm}{digital\_method\_func}
   \func{digital\_method}
   {%
     \hiki{add}{function},\ %
     \hiki{mul}{function},\ %
     \hiki{act}{function},\ %
     \hiki{power}{function},\ %
     \hiki{zero}{object},\ %
     \hiki{one}{object}
   }{%
     \out{function}%
   }\\
   \spacing
   % document of basic document
   \quad Return a function which evaluates polynomial corresponding to 'coefficients' at 'val' 
from an iterator 'coefficients' and an object 'val'.\\
   \spacing
   % added document
   %\spacing
   % input, output document
   \quad  All operations 'add', 'mul', 'act', 'power', 'zero', 'one' should be inputted 
   in a manner similar to \linkingone{algorithm}{digital\_method}.\\
%
  \subsection{rl\_binary\_powering -- right-left powering}\linkedone{algorithm}{rl\_binary\_powering}
   \func{rl\_binary\_powering}
   {%
     \hiki{element}{object},\ %
     \hiki{index}{integer},\ %
     \hiki{mul}{function},\ %
     \hikiopt{square}{function}{None},\ %
     \hikiopt{one}{object}{None},\ %
   }{%
     \out{object}%
   }\\
   \spacing
   % document of basic document
   \quad Return \param{element} to the \param{index} power by using right-left binary method.\\
   \spacing
   % added document
   %\spacing
   % input, output document
   \quad  \param{index} should be a non-negative integer.
   If \param{square} is None, \param{square} is defined by using \param{mul}.\\
%
  \subsection{lr\_binary\_powering -- left-right powering}\linkedone{algorithm}{lr\_binary\_powering}
   \func{lr\_binary\_powering}
   {%
     \hiki{element}{object},\ %
     \hiki{index}{integer},\ %
     \hiki{mul}{function},\ %
     \hikiopt{square}{function}{None},\ %
     \hikiopt{one}{object}{None},\ %
   }{%
     \out{object}%
   }\\
   \spacing
   % document of basic document
   \quad Return \param{element} to the \param{index} power by using left-right binary method.\\
   \spacing
   % added document
   %\spacing
   % input, output document
   \quad  \param{index} should be a non-negative integer.
   If \param{square} is None, \param{square} is defined by using \param{mul}.\\
%
  \subsection{window\_powering -- window powering}\linkedone{algorithm}{window\_powering}
   \func{window\_powering}
   {%
     \hiki{element}{object},\ %
     \hiki{index}{integer},\ %
     \hiki{mul}{function},\ %
     \hikiopt{square}{function}{None},\ %
     \hikiopt{one}{object}{None},\ %
   }{%
     \out{object}%
   }\\
   \spacing
   % document of basic document
   \quad Return \param{element} to the \param{index} power by using small-window method.\\
   \spacing
   % added document
   \quad The window size is selected by average analystic optimization.\\
   \spacing
   % input, output document
   \quad  \param{index} should be a non-negative integer.
   If \param{square} is None, \param{square} is defined by using \param{mul}.\\
%
  \subsection{powering\_func -- function of powering}\linkedone{algorithm}{powering\_func}
   \func{powering\_func}
   {%
     \hiki{mul}{function},\ %
     \hikiopt{square}{function}{None},\ %
     \hikiopt{one}{object}{None},\ %
     \hikiopt{type}{integer}{0}
   }{%
     \out{function}%
   }\\
   \spacing
   % document of basic document
   \quad Return a function which computes 'element' to the 'index' power from an object 'element' and an integer 'index'.\\
   \spacing
   % added document
   %\spacing
   % input, output document
   \quad If \param{square} is None, \param{square} is defined by using \param{mul}.
   \param{type} should be an integer which means one of the following:\\
   0;\ \linkingone{algorithm}{rl\_binary\_powering}\\
   1;\ \linkingone{algorithm}{lr\_binary\_powering}\\
   2;\ \linkingone{algorithm}{window\_powering}\\
\\
%
\begin{ex}
>>> d_func = algorithm.digital_method_func(
... lambda a,b:a+b, lambda a,b:a*b, lambda i,a:i*a, lambda a,i:a**i, 
... matrix.zeroMatrix(3,0), matrix.identityMatrix(3,1)
... )
>>> coefficients = [(2,1), (1,2), (0,1)] # X^2+2*X+I
>>> A = matrix.SquareMatrix(3, [1,2,3]+[4,5,6]+[7,8,9])
>>> d_func(coefficients, A) # A**2+2*A+I
[33L, 40L, 48L]+[74L, 92L, 108L]+[116L, 142L, 169L]
>>> p_func = algorithm.powering_func(lambda a,b:a*b, type=2)
>>> p_func(A, 10) # A**10 by window method
[132476037840L, 162775103256L, 193074168672L]+[300005963406L, 368621393481L,
 437236823556L]+[467535888972L, 574467683706L, 681399478440L]

\end{ex}%Don't indent!(indent causes an error.)
\C

%---------- end document ---------- %

\bibliographystyle{jplain}%use jbibtex
\bibliography{nzmath_references}

\end{document}

