\documentclass{report}

\documentclass{report}

%%%%%%%%%%%%%%%%%%%%%%%%%%%%%%%%%%%%%%%%%%%%%%%%%%%%%%%%%%%%%
%
% macros for nzmath manual
%
%%%%%%%%%%%%%%%%%%%%%%%%%%%%%%%%%%%%%%%%%%%%%%%%%%%%%%%%%%%%%
\usepackage{amssymb,amsmath}
\usepackage{color}
\usepackage[dvipdfm,bookmarks=true,bookmarksnumbered=true,%
 pdftitle={NZMATH Users Manual},%
 pdfsubject={Manual for NZMATH Users},%
 pdfauthor={NZMATH Development Group},%
 pdfkeywords={TeX; dvipdfmx; hyperref; color;},%
 colorlinks=true]{hyperref}
\usepackage{fancybox}
\usepackage[T1]{fontenc}
%
\newcommand{\DS}{\displaystyle}
\newcommand{\C}{\clearpage}
\newcommand{\NO}{\noindent}
\newcommand{\negok}{$\dagger$}
\newcommand{\spacing}{\vspace{1pt}\\ }
% software macros
\newcommand{\nzmathzero}{{\footnotesize $\mathbb{N}\mathbb{Z}$}\texttt{MATH}}
\newcommand{\nzmath}{{\nzmathzero}\ }
\newcommand{\pythonzero}{$\mbox{\texttt{Python}}$}
\newcommand{\python}{{\pythonzero}\ }
% link macros
\newcommand{\linkingzero}[1]{{\bf \hyperlink{#1}{#1}}}%module
\newcommand{\linkingone}[2]{{\bf \hyperlink{#1.#2}{#2}}}%module,class/function etc.
\newcommand{\linkingtwo}[3]{{\bf \hyperlink{#1.#2.#3}{#3}}}%module,class,method
\newcommand{\linkedzero}[1]{\hypertarget{#1}{}}
\newcommand{\linkedone}[2]{\hypertarget{#1.#2}{}}
\newcommand{\linkedtwo}[3]{\hypertarget{#1.#2.#3}{}}
\newcommand{\linktutorial}[1]{\href{http://docs.python.org/tutorial/#1}{#1}}
\newcommand{\linktutorialone}[2]{\href{http://docs.python.org/tutorial/#1}{#2}}
\newcommand{\linklibrary}[1]{\href{http://docs.python.org/library/#1}{#1}}
\newcommand{\linklibraryone}[2]{\href{http://docs.python.org/library/#1}{#2}}
\newcommand{\pythonhp}{\href{http://www.python.org/}{\python website}}
\newcommand{\nzmathwiki}{\href{http://nzmath.sourceforge.net/wiki/}{{\nzmathzero}Wiki}}
\newcommand{\nzmathsf}{\href{http://sourceforge.net/projects/nzmath/}{\nzmath Project Page}}
\newcommand{\nzmathtnt}{\href{http://tnt.math.metro-u.ac.jp/nzmath/}{\nzmath Project Official Page}}
% parameter name
\newcommand{\param}[1]{{\tt #1}}
% function macros
\newcommand{\hiki}[2]{{\tt #1}:\ {\em #2}}
\newcommand{\hikiopt}[3]{{\tt #1}:\ {\em #2}=#3}

\newdimen\hoge
\newdimen\truetextwidth
\newcommand{\func}[3]{%
\setbox0\hbox{#1(#2)}
\hoge=\wd0
\truetextwidth=\textwidth
\advance \truetextwidth by -2\oddsidemargin
\ifdim\hoge<\truetextwidth % short form
{\bf \colorbox{skyyellow}{#1(#2)\ $\to$ #3}}
%
\else % long form
\fcolorbox{skyyellow}{skyyellow}{%
   \begin{minipage}{\textwidth}%
   {\bf #1(#2)\\ %
    \qquad\quad   $\to$\ #3}%
   \end{minipage}%
   }%
\fi%
}

\newcommand{\out}[1]{{\em #1}}
\newcommand{\initialize}{%
  \paragraph{\large \colorbox{skyblue}{Initialize (Constructor)}}%
    \quad\\ %
    \vspace{3pt}\\
}
\newcommand{\method}{\C \paragraph{\large \colorbox{skyblue}{Methods}}}
% Attribute environment
\newenvironment{at}
{%begin
\paragraph{\large \colorbox{skyblue}{Attribute}}
\quad\\
\begin{description}
}%
{%end
\end{description}
}
% Operation environment
\newenvironment{op}
{%begin
\paragraph{\large \colorbox{skyblue}{Operations}}
\quad\\
\begin{table}[h]
\begin{center}
\begin{tabular}{|l|l|}
\hline
operator & explanation\\
\hline
}%
{%end
\hline
\end{tabular}
\end{center}
\end{table}
}
% Examples environment
\newenvironment{ex}%
{%begin
\paragraph{\large \colorbox{skyblue}{Examples}}
\VerbatimEnvironment
\renewcommand{\EveryVerbatim}{\fontencoding{OT1}\selectfont}
\begin{quote}
\begin{Verbatim}
}%
{%end
\end{Verbatim}
\end{quote}
}
%
\definecolor{skyblue}{cmyk}{0.2, 0, 0.1, 0}
\definecolor{skyyellow}{cmyk}{0.1, 0.1, 0.5, 0}
%
%\title{NZMATH User Manual\\ {\large{(for version 1.0)}}}
%\date{}
%\author{}
\begin{document}
%\maketitle
%
\setcounter{tocdepth}{3}
\setcounter{secnumdepth}{3}


\tableofcontents
\C

\chapter{Classes}


%---------- start document ---------- %
\section{finitefield -- Finite Field}\linkedzero{finitefield}

 \begin{itemize}
   \item {\bf Classes}
   \begin{itemize}
     \item \negok \linkingone{finitefield}{FiniteField}
     \item \negok \linkingone{finitefield}{FiniteFieldElement}
     \item \linkingone{finitefield}{FinitePrimeField}
     \item \linkingone{finitefield}{FinitePrimeFieldElement}
     \item \linkingone{finitefield}{ExtendedField}
     \item \linkingone{finitefield}{ExtendedFieldElement}
   \end{itemize}
   %\item {\bf Functions}
   %  \begin{itemize}
   %    \item \linkingone{rational}{innerProduct}
   %  \end{itemize}
 \end{itemize}

\C

 \subsection{\negok FiniteField -- finite field, abstract}\linkedone{finitefield}{FiniteField}
 
 Abstract class for finite fields. Do not use the class directly, but use the subclasses \linkingone{finitefield}{FinitePrimeField} or \linkingone{finitefield}{ExtendedField}.

 The class is a subclass of \linkingone{ring}{Field}.
\C
 \subsection{\negok FiniteFieldElement -- element in finite field, abstract}\linkedone{finitefield}{FiniteFieldElement}
 Abstract class for finite field elements. Do not use the class directly, but use the subclasses \linkingone{finitefield}{FinitePrimeFieldElement} or \linkingone{finitefield}{ExtendedFieldElement}.

 The class is a subclass of \linkingone{ring}{FieldElement}.

%%   \initialize
%%   \func{IntegerResidueClassRing}{\hiki{modulus}{integer}}{\out{IntegerResidueClassRing}}\\
%%   \spacing
%%   % document of basic document
%%   \quad Create an instance of IntegerResidueClassRing. 
%%   % added document
%%   The argument \param{modulus} = $m$ specifies an ideal $m\mathbb{Z}$.
%%   % \spacing
%%   % input, output document
%%   %See \linkingone{module}{point} for \param{point}.
%%   \begin{at}
%%     \item[zero]\linkedtwo{integer}{IntegerRing}{zero}:\\ It expresses The additive unit 0. (read only)
%%     \item[one]\linkedtwo{integer}{IntegerRing}{one}:\\ It expresses The multiplicative unit 1. (read only)
%%   \end{at}
%%   \begin{op}
%%   %  \verb|+| & Vector sum.\\
%%   %  \verb|-| & Vector subtraction.\\
%%   %  \verb|*| & Scalar multiplication.\\
%%   %  \verb|//| & Scalar division.\\
%%   %  \verb|-(unary)| & element negation.\\
%%     \verb|==| & ring equality.\\
%%   %  \verb|!=| & inequality or not.\\
%%   %  \verb+V[i]+ & Return the coefficient of i-th element of Vector.\\
%%   %  \verb+V[i] = c+ & Replace the coefficient of i-th element of Vector by c.\\
%%     \verb|card| & return cardinality. See also \linkingzero{compatibility} module.\\
%%     \verb|in| & return whether an element is in or not.\\
%%     \verb|repr| & return representation string.\\
%%     \verb|str| & return string.\\
%%   \end{op} 
%% %\begin{ex}
%% %>>> A = vector.Vector([1,2])
%% %>>> A
%% %Vector([1, 2])
%% %>>> A.compo
%% %[1, 2]
%% %>>>
%% %\end{ex}%Don't indent!
\C
 \subsection{FinitePrimeField -- finite prime field}\linkedone{finitefield}{FinitePrimeField}
 
Finite prime field is also known as $\mathbb{F}_p$ or $\mathrm{GF}(p)$. It has prime number cardinality.

 The class is a subclass of \linkingone{finitefield}{FiniteField}.

   \initialize
   \func{FinitePrimeField}
        {\hiki{characteristic}{integer}} 
        {\out{FinitePrimeField}}\\
   \spacing
   % document of basic document
   \quad Create a FinitePrimeField instance with the given \param{characteristic}.
   % added document
   %\spacing
   % input, output document
   \param{characteristic} must be positive prime integer.
   \begin{at}
    \item[zero]\linkedtwo{finitefield}{FinitePrimeField}{zero}:\\ It expresses the additive unit 0. (read only)
    \item[one]\linkedtwo{finitefield}{FinitePrimeField}{one}:\\ It expresses the multiplicative unit 1. (read only)
   \end{at}
   \begin{op}
     \verb|F==G| & equality test.\\
     \verb|x in F| & membership test.\\
     \verb|card(F)| & Cardinality of the field. \\
   \end{op} 
%\begin{ex}
%>>> A = vector.Vector([1,2])
%>>> A
%Vector([1, 2])
%>>> A.compo
%[1, 2]
%>>>
%\end{ex}%Don't indent!
  \method
  \subsubsection{createElement -- create element of finite prime field}\linkedtwo{finitefield}{FinitePrimeField}{createElement}
   \func{createElement}{\param{self},\ \hiki{seed}{integer}}{\out{FinitePrimeFieldElement}}\\
   \spacing
   % document of basic document
   \quad Create \linkingone{finitefield}{FinitePrimeFieldElement} with \param{seed}. 
   %\spacing
   % added document
   %\quad \negok Note that this function returns integer only.\\
   \spacing
   % input, output document
   \quad \param{seed} must be int or long.\\
%
  \subsubsection{getCharacteristic -- get characteristic}\linkedtwo{finitefield}{FinitePrimeField}{getCharacteristic}
   \func{getCharacteristic}{\param{self}}{\out{integer}}\\
   \spacing
   % document of basic document
   \quad Return the characteristic of the field.
   %\spacing
   % added document
   %\quad \negok Note that this function returns integer only.\\
   %\spacing
   % input, output document
   %\quad \param{a} must be int, long or rational.Integer.\\
%
  \subsubsection{issubring -- subring test}\linkedtwo{finitefield}{FinitePrimeField}{issubring}
   \func{issubring}{\param{self},\ \hiki{other}{\linkingone{ring}{Ring}}}{\out{bool}}\\
   \spacing
   % document of basic document
   \quad Report whether another ring contains the field as subring.

   %\spacing
   % added document
   %\quad \negok Note that this function returns integer only.\\
   %\spacing
   % input, output document
   %\quad if \param{as\_column} is True, try to create column matrix.\\
%
  \subsubsection{issuperring -- superring test}\linkedtwo{finitefield}{FinitePrimeField}{issuperring}
   \func{issuperring}{\param{self},\ \hiki{other}{\linkingone{ring}{Ring}}}{\out{bool}}\\
   \spacing
   % document of basic document
   \quad Report whether the field is a superring of another ring.

   Since the field is a prime field, it can be a superring of itself only.

   %\spacing
   % added document
   %\quad \negok Note that this function returns integer only.\\
   %\spacing
   % input, output document
   %\quad if \param{as\_column} is True, try to create column matrix.\\
%\begin{ex}
%>>> A = module.HogeClass((1,2))
%>>> A.hogemethod1(2)
%(2, 4)
%>>>
%\end{ex}%Don't indent!
\C
 \subsection{FinitePrimeFieldElement -- element of finite prime field}\linkedone{finitefield}{FinitePrimeFieldElement}
 The class provides elements of finite prime fields.

 It is a subclass of \linkingone{finitefield}{FiniteFieldElement} and \linkingone{intresidue}{IntegerResidueClass}. 

  \initialize
  \func{FinitePrimeFieldElement}
       {\hiki{representative}{integer},\ 
       \hiki{modulus}{integer}}
       {\out{FinitePrimeFieldElement}}\\
  \spacing
  % document of basic document
  \quad Create element in finite prime field of modulus with residue representative.
  % added document
  \spacing
  % input, output document
  \param{modulus} must be positive prime integer.
  %\begin{at}
  %  \item[compo]\linkedtwo{vector}{Vector}{compo}:\\ It expresses component of Vector.
  %\end{at}
  \begin{op}
    \verb|a+b| & addition.\\
    \verb|a-b| & subtraction.\\
    \verb|a*b| & multiplication.\\
    %\verb|/| & division.\\
    \verb|a**n,pow(a,n)| & power.\\
    \verb|-a| & negation.\\
    \verb|+a| & make a copy.\\
    \verb|a==b| & equality test.\\
    \verb|a!=b| & inequality test.\\
    \verb|repr(a)| & return representation string.\\
    \verb|str(a)| & return string.\\
  \end{op} 
%\begin{ex}
%>>> A = vector.Vector([1,2])
%>>> A
%Vector([1, 2])
%>>> A.compo
%[1, 2]
%>>>
%\end{ex}%Don't indent!
  \method
  \subsubsection{getRing -- get ring object}\linkedtwo{finitefield}{FinitePrimeFieldElement}{getRing}
   \func{getRing}{\param{self}}{\out{FinitePrimeField}}\\
   \spacing
   % document of basic document
   \quad Return an instance of FinitePrimeField to which the element belongs.
   %\spacing
   % added document
   %\quad \negok Note that this function returns integer only.\\
   %\spacing
   % input, output document
   %\quad \param{a} must be int, long or rational.Integer.\\
%
  \subsubsection{order -- order of multiplicative group}\linkedtwo{finitefield}{FinitePrimeFieldElement}{order}
   \func{order}{\param{self}}{\out{integer}}\\
   \spacing
   % document of basic document
   \quad Find and return the order of the element in the multiplicative group of $\mathbb{F}_p$.
   %\spacing
   % added document
   %\quad \negok Note that this function returns integer only.\\
   %\spacing
   % input, output document
   %\quad \param{a} must be int, long or rational.Integer.\\
%
%
%\begin{ex}
%>>> A = module.HogeClass((1,2))
%>>> A.hogemethod1(2)
%(2, 4)
%>>>
%\end{ex}%Don't indent!
\C
 \subsection{ExtendedField -- extended field of finite field}\linkedone{finitefield}{ExtendedField}
 ExtendedField is a class for finite field, whose cardinality $q = p^n$ with a prime $p$ and $n>1$. It is usually called $\mathbb{F}_q$ or $\mathrm{GF}(q)$.

 The class is a subclass of \linkingone{finitefield}{FiniteField}.

   \initialize
   \func{ExtendedField}
        {\hiki{basefield}{FiniteField},\ 
          \hiki{modulus}{FiniteFieldPolynomial}} 
        {\out{ExtendedField}}\\
   \spacing
   % document of basic document
   \quad Create a field extension $\mathtt{basefield}[X]/(\mathtt{modulus}(X))$.

FinitePrimeField instance with the given \param{characteristic}.
   % added document
   %\spacing
   % input, output document
   The \param{modulus} has to be an irreducible polynomial with coefficients in the \param{basefield}.
   \begin{at}
    \item[zero]\linkedtwo{finitefield}{ExtendedField}{zero}:\\ It expresses the additive unit 0. (read only)
    \item[one]\linkedtwo{finitefield}{ExtendedField}{one}:\\ It expresses the multiplicative unit 1. (read only)
   \end{at}
   \begin{op}
     \verb|F==G| & equality or not.\\
     \verb|x in F| & membership test.\\
     \verb|card(F)| & Cardinality of the field. \\
     \verb|repr(F)| & representation string.\\
     \verb|str(F)| & string.\\
   \end{op} 
%\begin{ex}
%>>> A = vector.Vector([1,2])
%>>> A
%Vector([1, 2])
%>>> A.compo
%[1, 2]
%>>>
%\end{ex}%Don't indent!
  \method
  \subsubsection{createElement -- create element of extended field}\linkedtwo{finitefield}{ExtendedField}{createElement}
   \func{createElement}{\param{self},\ \hiki{seed}{extended element seed}}{\out{ExtendedFieldElement}}\\
   \spacing
   % document of basic document
   \quad Create an element of the field from seed. The result is an instance of \linkingone{finitefield}{ExtendedFieldElement}.

   %\spacing
   % added document
   %\quad \negok Note that this function returns integer only.\\
   %\spacing
   % input, output document
   \quad The \param{seed} can be:
 \begin{itemize}
 \item a \linkingone{poly.uniutil}{FinitePrimeFieldPolynomial}
 \item an integer, which will be expanded in card(\param{basefield}) and interpreted as a polynomial. 
 \item \param{basefield} element. 
 \item a list of basefield elements interpreted as a polynomial coefficient.
 \end{itemize}
%
  \subsubsection{getCharacteristic -- get characteristic}\linkedtwo{finitefield}{ExtendedField}{getCharacteristic}
   \func{getCharacteristic}{\param{self}}{\out{integer}}\\
   \spacing
   % document of basic document
   \quad Return the characteristic of the field.
   %\spacing
   % added document
   %\quad \negok Note that this function returns integer only.\\
   %\spacing
   % input, output document
   %\quad \param{a} must be int, long or rational.Integer.\\
%
  \subsubsection{issubring -- subring test}\linkedtwo{finitefield}{ExtendedField}{issubring}
   \func{issubring}{\param{self},\ \hiki{other}{\linkingone{ring}{Ring}}}{\out{bool}}\\
   \spacing
   % document of basic document
   \quad Report whether another ring contains the field as subring.
   %\spacing
   % added document
   %\quad \negok Note that this function returns integer only.\\
   %\spacing
   % input, output document
   %\quad if \param{as\_column} is True, try to create column matrix.\\
%
  \subsubsection{issuperring -- superring test}\linkedtwo{finitefield}{ExtendedField}{issuperring}
   \func{issuperring}{\param{self},\ \hiki{other}{\linkingone{ring}{Ring}}}{\out{bool}}\\
   \spacing
   % document of basic document
   \quad Report whether the field is a superring of another ring.
   %\spacing
   % added document
   %\quad \negok Note that this function returns integer only.\\
   %\spacing
   % input, output document
   %\quad if \param{as\_column} is True, try to create column matrix.\\
%
  \subsubsection{primitive\_element -- generator of multiplicative group}\linkedtwo{finitefield}{ExtendedField}{primitive\_element}
   \func{primitive\_element}{\param{self}}{\out{ExtendedFieldElement}}\\
   \spacing
   % document of basic document
   \quad Return a primitive element of the field, i.e., a generator of the multiplicative group.
   \spacing
   % added document
   %\quad \negok Note that this function returns integer only.\\
   %\spacing
   % input, output document
   %\quad if \param{as\_column} is True, try to create column matrix.\\
%\begin{ex}
%>>> A = module.HogeClass((1,2))
%>>> A.hogemethod1(2)
%(2, 4)
%>>>
%\end{ex}%Don't indent!
\C
 \subsection{ExtendedFieldElement -- element of finite field}\linkedone{finitefield}{ExtendedFieldElement}
 ExtendedFieldElement is a class for an element of $F_q$. 

 The class is a subclass of \linkingone{finitefield}{FiniteFieldElement}.

  \initialize
  \func{ExtendedFieldElement}
       {\hiki{representative}{FiniteFieldPolynomial},\ 
       \hiki{field}{ExtendedField}}
       {\out{ExtendedFieldElement}}\\
  \spacing
  % document of basic document
  \quad Create an element of the finite extended field.
  % added document
  \spacing
  % input, output document
  The argument \param{representative} must be an \linkingone{poly.uniutil}{FiniteFieldPolynomial} has same \param{basefield}.
  Another argument \param{field} must be an instance of ExtendedField.

  %\begin{at}
  %  \item[compo]\linkedtwo{vector}{Vector}{compo}:\\ It expresses component of Vector.
  %\end{at}
  \begin{op}
    \verb|a+b| & addition.\\
    \verb|a-b| & subtraction.\\
    \verb|a*b| & multiplication.\\
    \verb|a/b| & inverse multiplication.\\
    \verb|a**n,pow(a,n)| & power.\\
    \verb|-a| & negation.\\
    \verb|+a| & make a copy.\\
    \verb|a==b| & equality test.\\
    \verb|a!=b| & inequality test.\\
    \verb|repr(a)| & return representation string.\\
    \verb|str(a)| & return string.\\
  \end{op} 
%\begin{ex}
%>>> A = vector.Vector([1,2])
%>>> A
%Vector([1, 2])
%>>> A.compo
%[1, 2]
%>>>
%\end{ex}%Don't indent!
  \method
  \subsubsection{getRing -- get ring object}\linkedtwo{finitefield}{FinitePrimeFieldElement}{getRing}
   \func{getRing}{\param{self}}{\out{FinitePrimeField}}\\
   \spacing
   % document of basic document
   \quad Return an instance of FinitePrimeField to which the element belongs.
   %\spacing
   % added document
   %\quad \negok Note that this function returns integer only.\\
   %\spacing
   % input, output document
   %\quad \param{a} must be int, long or rational.Integer.\\
%
  \subsubsection{inverse -- inverse element}\linkedtwo{finitefield}{ExtendedFieldElement}{inverse}
   \func{inverse}{\param{self}}{\out{ExtendedFieldElement}}\\
   \spacing
   % document of basic document
   \quad Return the inverse element.
   %\spacing
   % added document
   %\quad \negok Note that this function returns integer only.\\
   %\spacing
   % input, output document
   %\quad \param{a} must be int, long or rational.Integer.\\
%
%%   \subsubsection{order -- order of multiplicative group}\linkedtwo{finitefield}{FinitePrimeFieldElement}{order}
%%    \func{order}{\param{self}}{\out{integer}}\\
%%    \spacing
%%    % document of basic document
%%    \quad Find and return the order of the element in the multiplicative group of $F_p$.
%%    %\spacing
%%    % added document
%%    %\quad \negok Note that this function returns integer only.\\
%%    %\spacing
%%    % input, output document
%%    %\quad \param{a} must be int, long or rational.Integer.\\
%
%
%\begin{ex}
%>>> A = module.HogeClass((1,2))
%>>> A.hogemethod1(2)
%(2, 4)
%>>>
%\end{ex}%Don't indent!
\C

%---------- end document ---------- %

\bibliographystyle{jplain}%use jbibtex
\bibliography{nzmath_references}

\end{document}

