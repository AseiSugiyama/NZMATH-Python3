\documentclass{report}

\documentclass{report}

%%%%%%%%%%%%%%%%%%%%%%%%%%%%%%%%%%%%%%%%%%%%%%%%%%%%%%%%%%%%%
%
% macros for nzmath manual
%
%%%%%%%%%%%%%%%%%%%%%%%%%%%%%%%%%%%%%%%%%%%%%%%%%%%%%%%%%%%%%
\usepackage{amssymb,amsmath}
\usepackage{color}
\usepackage[dvipdfm,bookmarks=true,bookmarksnumbered=true,%
 pdftitle={NZMATH Users Manual},%
 pdfsubject={Manual for NZMATH Users},%
 pdfauthor={NZMATH Development Group},%
 pdfkeywords={TeX; dvipdfmx; hyperref; color;},%
 colorlinks=true]{hyperref}
\usepackage{fancybox}
\usepackage[T1]{fontenc}
%
\newcommand{\DS}{\displaystyle}
\newcommand{\C}{\clearpage}
\newcommand{\NO}{\noindent}
\newcommand{\negok}{$\dagger$}
\newcommand{\spacing}{\vspace{1pt}\\ }
% software macros
\newcommand{\nzmathzero}{{\footnotesize $\mathbb{N}\mathbb{Z}$}\texttt{MATH}}
\newcommand{\nzmath}{{\nzmathzero}\ }
\newcommand{\pythonzero}{$\mbox{\texttt{Python}}$}
\newcommand{\python}{{\pythonzero}\ }
% link macros
\newcommand{\linkingzero}[1]{{\bf \hyperlink{#1}{#1}}}%module
\newcommand{\linkingone}[2]{{\bf \hyperlink{#1.#2}{#2}}}%module,class/function etc.
\newcommand{\linkingtwo}[3]{{\bf \hyperlink{#1.#2.#3}{#3}}}%module,class,method
\newcommand{\linkedzero}[1]{\hypertarget{#1}{}}
\newcommand{\linkedone}[2]{\hypertarget{#1.#2}{}}
\newcommand{\linkedtwo}[3]{\hypertarget{#1.#2.#3}{}}
\newcommand{\linktutorial}[1]{\href{http://docs.python.org/tutorial/#1}{#1}}
\newcommand{\linktutorialone}[2]{\href{http://docs.python.org/tutorial/#1}{#2}}
\newcommand{\linklibrary}[1]{\href{http://docs.python.org/library/#1}{#1}}
\newcommand{\linklibraryone}[2]{\href{http://docs.python.org/library/#1}{#2}}
\newcommand{\pythonhp}{\href{http://www.python.org/}{\python website}}
\newcommand{\nzmathwiki}{\href{http://nzmath.sourceforge.net/wiki/}{{\nzmathzero}Wiki}}
\newcommand{\nzmathsf}{\href{http://sourceforge.net/projects/nzmath/}{\nzmath Project Page}}
\newcommand{\nzmathtnt}{\href{http://tnt.math.se.tmu.ac.jp/nzmath/}{\nzmath Project Official Page}}
% parameter name
\newcommand{\param}[1]{{\tt #1}}
% function macros
\newcommand{\hiki}[2]{{\tt #1}:\ {\em #2}}
\newcommand{\hikiopt}[3]{{\tt #1}:\ {\em #2}=#3}

\newdimen\hoge
\newdimen\truetextwidth
\newcommand{\func}[3]{%
\setbox0\hbox{#1(#2)}
\hoge=\wd0
\truetextwidth=\textwidth
\advance \truetextwidth by -2\oddsidemargin
\ifdim\hoge<\truetextwidth % short form
{\bf \colorbox{skyyellow}{#1(#2)\ $\to$ #3}}
%
\else % long form
\fcolorbox{skyyellow}{skyyellow}{%
   \begin{minipage}{\textwidth}%
   {\bf #1(#2)\\ %
    \qquad\quad   $\to$\ #3}%
   \end{minipage}%
   }%
\fi%
}

\newcommand{\out}[1]{{\em #1}}
\newcommand{\initialize}{%
  \paragraph{\large \colorbox{skyblue}{Initialize (Constructor)}}%
    \quad\\ %
    \vspace{3pt}\\
}
\newcommand{\method}{\C \paragraph{\large \colorbox{skyblue}{Methods}}}
% Attribute environment
\newenvironment{at}
{%begin
\paragraph{\large \colorbox{skyblue}{Attribute}}
\quad\\
\begin{description}
}%
{%end
\end{description}
}
% Operation environment
\newenvironment{op}
{%begin
\paragraph{\large \colorbox{skyblue}{Operations}}
\quad\\
\begin{table}[h]
\begin{center}
\begin{tabular}{|l|l|}
\hline
operator & explanation\\
\hline
}%
{%end
\hline
\end{tabular}
\end{center}
\end{table}
}
% Examples environment
\newenvironment{ex}%
{%begin
\paragraph{\large \colorbox{skyblue}{Examples}}
\VerbatimEnvironment
\renewcommand{\EveryVerbatim}{\fontencoding{OT1}\selectfont}
\begin{quote}
\begin{Verbatim}
}%
{%end
\end{Verbatim}
\end{quote}
}
%
\definecolor{skyblue}{cmyk}{0.2, 0, 0.1, 0}
\definecolor{skyyellow}{cmyk}{0.1, 0.1, 0.5, 0}
%
%\title{NZMATH User Manual\\ {\large{(for version 1.0)}}}
%\date{}
%\author{}
\begin{document}
%\maketitle
%
\setcounter{tocdepth}{3}
\setcounter{secnumdepth}{3}


\tableofcontents
\C

\chapter{Functions}


%---------- start document ---------- %
 \section{squarefree -- Squarefreeness tests}\linkedzero{squarefree}

There are two method groups.
A function in one group raises \linkingone{squarefree}{Undetermined} when it cannot determine squarefreeness.
A function in another group returns {\tt None} in such cases.
The latter group of functions have ``\_ternary'' suffix on their names.
We refer a set \(\{{\tt True}, {\tt False}, {\tt None}\}\) as {\it ternary}\linkedone{squarefree}{ternary}.

The parameter type {\it integer}\linkedone{squarefree}{integer} means either {\it int}, {\it long} or \linkingone{rational}{Integer}.

This module provides an exception class.
\begin{description}
  \item[Undetermined]:\ Report undetermined state of calculation.
    The exception will be raised by
    \linkingone{squarefree}{lenstra} or
    \linkingone{squarefree}{trivial\_test}.
\end{description}

\subsection{Definition}

  We define squarefreeness as:\\
  \(n\) is squarefree \(\iff\) there is no prime \(p\) whose square divides \(n\).

\vspace{1em}
\noindent Examples:
  \begin{itemize}
  \item \(0\) is non-squarefree because any square of prime can divide \(0\).
  \item \(1\) is squarefree because there is no prime dividing \(1\).
  \item \(2\), \(3\), \(5\), and any other primes are squarefree.
  \item \(4\), \(8\), \(9\), \(12\), \(16\) are non-squarefree composites.
  \item \(6\), \(10\), \(14\), \(15\), \(21\) are squarefree composites.
\end{itemize}

 \subsection{lenstra -- Lenstra's condition}\linkedone{squarefree}{lenstra}
 \func{lenstra}{\hiki{n}{integer}}{\out{bool}}\\
 \spacing
 % document of basic document
 \quad If return value is True, \param{n} is squarefree.  Otherwise, the
 squarefreeness is still unknown and \linkingone{squarefree}{Undetermined} is raised.
 The algorithm is based on~\cite{Lenstra1979}. \\
 \spacing
 % added document
 \negok The condition is so strong that it seems \param{n} has to be a
 prime or a Carmichael number to satisfy it.\\
 \spacing
 % input, output document
 \quad Input parameter \param{n} ought to be an odd \linkingone{squarefree}{integer}.
 % 
 \subsection{trial\_division -- trial division}\linkedone{squarefree}{trial\_division}
 \func{trial\_division}{\hiki{n}{integer}}{\out{bool}}\\
 \spacing
 % document of basic document
 \quad Check whether \param{n} is squarefree or not. \\
 \spacing
 % added document
 The method is a kind of trial division and inefficient for large numbers. \\
 \spacing
 % input, output document
 \quad Input parameter \param{n} ought to be an \linkingone{squarefree}{integer}.
% 
 \subsection{trivial\_test -- trivial tests}\linkedone{squarefree}{trivial\_test}
 \func{trivial\_test}{\hiki{n}{integer}}{\out{bool}}\\
 \spacing
 % document of basic document
 \quad Check whether \param{n} is squarefree or not.  If the squarefreeness is still unknown, then \linkingone{squarefree}{Undetermined} is raised. \\
 \spacing
 % added document
 This method do anything but factorization including Lenstra's method. \\
 \spacing
 % input, output document
 \quad Input parameter \param{n} ought to be an odd \linkingone{squarefree}{integer}.
% 
 \subsection{viafactor -- via factorization}\linkedone{squarefree}{viafactor}
 \func{viafactor}{\hiki{n}{integer}}{\out{bool}}\\
 \spacing
 % document of basic document
 \quad Check whether \param{n} is squarefree or not. \\
 \spacing
 % added document
 It is obvious that if one knows the prime factorization of the number, he/she can tell whether the number is squarefree or not. \\
 \spacing
 % input, output document
 \quad Input parameter \param{n} ought to be an \linkingone{squarefree}{integer}.
% 
 \subsection{viadecomposition -- via partial factorization}\linkedone{squarefree}{viadecomposition}
 \func{viadecomposition}{\hiki{n}{integer}}{\out{bool}}\\
 \spacing
 % document of basic document
 \quad Test the squarefreeness of \param{n}.
 The return value is either one of {\tt True} or {\tt False};
 {\tt None} never be returned. \\
 \spacing
 % added document
 The method uses partial factorization into squarefree parts,
 if such partial factorization is possible.  In other cases,
 It completely factor \param{n} by trial division.
 \spacing
 % input, output document
 \quad Input parameter \param{n} ought to be an \linkingone{squarefree}{integer}.
% 
 \subsection{lenstra\_ternary -- Lenstra's condition, ternary version}\linkedone{squarefree}{lenstra\_ternary}
 \func{lenstra\_ternary}{\hiki{n}{integer}}{\out{ternary}}\\
 \spacing
 % document of basic document
 \quad Test the squarefreeness of \param{n}. The return value is one of the ternary logical constants.  If return value is {\tt True}, \param{n} is squarefree.  Otherwise, the squarefreeness is still unknown and {\tt None} is returned. \\
 \spacing
 % added document
 \negok The condition is so strong that it seems \param{n} has to be a
 prime or a Carmichael number to satisfy it.\\
 This is a ternary version of \linkingone{squarefree}{lenstra}. \\
 \spacing
 % input, output document
 \quad Input parameter \param{n} ought to be an odd \linkingone{squarefree}{integer}.
 % 
 \subsection{trivial\_test\_ternary -- trivial tests, ternary version}\linkedone{squarefree}{trivial\_test\_ternary}
 \func{trivial\_test\_ternary}{\hiki{n}{integer}}{\out{ternary}}\\
 \spacing
 % document of basic document
 \quad Test the squarefreeness of \param{n}.
 The return value is one of the ternary logical constants. \\
 \spacing
 % added document
 The method uses a series of trivial tests including \linkingone{squarefree}{lenstra\_ternary}. \\
 This is a ternary version of \linkingone{squarefree}{trivial\_test}. \\
 \spacing
 % input, output document
 \quad Input parameter \param{n} ought to be an \linkingone{squarefree}{integer}.
% 
 \subsection{trial\_division\_ternary  -- trial division, ternary version}\linkedone{squarefree}{trial\_division\_ternary}
 \func{trial\_division\_ternary}{\hiki{n}{integer}}{\out{ternary}}\\
 \spacing
 % document of basic document
 \quad Test the squarefreeness of \param{n}.
 The return value is either one of {\tt True} or {\tt False};
 {\tt None} never be returned. \\
 \spacing
 % added document
 The method is a kind of trial division. \\
 This is a ternary version of \linkingone{squarefree}{trial\_division}.\\
 \spacing
 % input, output document
 \quad Input parameter \param{n} ought to be an \linkingone{squarefree}{integer}.
% 
 \subsection{viafactor\_ternary -- via factorization, ternary version}\linkedone{squarefree}{viafactor\_ternary}
 \func{viafactor\_ternary}{\hiki{n}{integer}}{\out{ternary}}\\
 \spacing
 % document of basic document
 \quad Just for symmetry, this function is defined as an alias of \linkingone{squarefree}{viafactor}. \\
 \spacing
 % added document
 \spacing
 % input, output document
 \quad Input parameter \param{n} ought to be an \linkingone{squarefree}{integer}.
% 
\C

%---------- end document ---------- %

\bibliographystyle{jplain}%use jbibtex
\bibliography{nzmath_references}

\end{document}

