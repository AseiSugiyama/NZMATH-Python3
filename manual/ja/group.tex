\documentclass{report}

%%%%%%%%%%%%%%%%%%%%%%%%%%%%%%%%%%%%%%%%%%%%%%%%%%%%%%%%%%%%%
%
% macros for nzmath manual
%
%%%%%%%%%%%%%%%%%%%%%%%%%%%%%%%%%%%%%%%%%%%%%%%%%%%%%%%%%%%%%
\usepackage{amssymb,amsmath}
\usepackage{color}
\usepackage[dvipdfm,bookmarks=true,bookmarksnumbered=true,%
 pdftitle={NZMATH Users Manual},%
 pdfsubject={Manual for NZMATH Users},%
 pdfauthor={NZMATH Development Group},%
 pdfkeywords={TeX; dvipdfmx; hyperref; color;},%
 colorlinks=true]{hyperref}
\usepackage{fancybox}
\usepackage[T1]{fontenc}
%
\newcommand{\DS}{\displaystyle}
\newcommand{\C}{\clearpage}
\newcommand{\NO}{\noindent}
\newcommand{\negok}{$\dagger$}
\newcommand{\spacing}{\vspace{1pt}\\ }
% software macros
\newcommand{\nzmathzero}{{\footnotesize $\mathbb{N}\mathbb{Z}$}\texttt{MATH}}
\newcommand{\nzmath}{{\nzmathzero}\ }
\newcommand{\pythonzero}{$\mbox{\texttt{Python}}$}
\newcommand{\python}{{\pythonzero}\ }
% link macros
\newcommand{\linkingzero}[1]{{\bf \hyperlink{#1}{#1}}}%module
\newcommand{\linkingone}[2]{{\bf \hyperlink{#1.#2}{#2}}}%module,class/function etc.
\newcommand{\linkingtwo}[3]{{\bf \hyperlink{#1.#2.#3}{#3}}}%module,class,method
\newcommand{\linkedzero}[1]{\hypertarget{#1}{}}
\newcommand{\linkedone}[2]{\hypertarget{#1.#2}{}}
\newcommand{\linkedtwo}[3]{\hypertarget{#1.#2.#3}{}}
\newcommand{\linktutorial}[1]{\href{http://docs.python.org/tutorial/#1}{#1}}
\newcommand{\linktutorialone}[2]{\href{http://docs.python.org/tutorial/#1}{#2}}
\newcommand{\linklibrary}[1]{\href{http://docs.python.org/library/#1}{#1}}
\newcommand{\linklibraryone}[2]{\href{http://docs.python.org/library/#1}{#2}}
\newcommand{\pythonhp}{\href{http://www.python.org/}{\python website}}
\newcommand{\nzmathwiki}{\href{http://nzmath.sourceforge.net/wiki/}{{\nzmathzero}Wiki}}
\newcommand{\nzmathsf}{\href{http://sourceforge.net/projects/nzmath/}{\nzmath Project Page}}
\newcommand{\nzmathtnt}{\href{http://tnt.math.se.tmu.ac.jp/nzmath/}{\nzmath Project Official Page}}
% parameter name
\newcommand{\param}[1]{{\tt #1}}
% function macros
\newcommand{\hiki}[2]{{\tt #1}:\ {\em #2}}
\newcommand{\hikiopt}[3]{{\tt #1}:\ {\em #2}=#3}

\newdimen\hoge
\newdimen\truetextwidth
\newcommand{\func}[3]{%
\setbox0\hbox{#1(#2)}
\hoge=\wd0
\truetextwidth=\textwidth
\advance \truetextwidth by -2\oddsidemargin
\ifdim\hoge<\truetextwidth % short form
{\bf \colorbox{skyyellow}{#1(#2)\ $\to$ #3}}
%
\else % long form
\fcolorbox{skyyellow}{skyyellow}{%
   \begin{minipage}{\textwidth}%
   {\bf #1(#2)\\ %
    \qquad\quad   $\to$\ #3}%
   \end{minipage}%
   }%
\fi%
}

\newcommand{\out}[1]{{\em #1}}
\newcommand{\initialize}{%
  \paragraph{\large \colorbox{skyblue}{Initialize (Constructor)}}%
    \quad\\ %
    \vspace{3pt}\\
}
\newcommand{\method}{\C \paragraph{\large \colorbox{skyblue}{Methods}}}
% Attribute environment
\newenvironment{at}
{%begin
\paragraph{\large \colorbox{skyblue}{Attribute}}
\quad\\
\begin{description}
}%
{%end
\end{description}
}
% Operation environment
\newenvironment{op}
{%begin
\paragraph{\large \colorbox{skyblue}{Operations}}
\quad\\
\begin{table}[h]
\begin{center}
\begin{tabular}{|l|l|}
\hline
operator & explanation\\
\hline
}%
{%end
\hline
\end{tabular}
\end{center}
\end{table}
}
% Examples environment
\newenvironment{ex}%
{%begin
\paragraph{\large \colorbox{skyblue}{Examples}}
\VerbatimEnvironment
\renewcommand{\EveryVerbatim}{\fontencoding{OT1}\selectfont}
\begin{quote}
\begin{Verbatim}
}%
{%end
\end{Verbatim}
\end{quote}
}
%
\definecolor{skyblue}{cmyk}{0.2, 0, 0.1, 0}
\definecolor{skyyellow}{cmyk}{0.1, 0.1, 0.5, 0}
%
%\title{NZMATH User Manual\\ {\large{(for version 1.0)}}}
%\date{}
%\author{}
\begin{document}
%\maketitle
%
\setcounter{tocdepth}{3}
\setcounter{secnumdepth}{3}


\tableofcontents
\C

\chapter{Classes}


%---------- start document ---------- %
 \section{group -- algorithms for finite groups}\linkedzero{group}
 \begin{itemize}
   \item {\bf Classes}
   \begin{itemize}
     \item \linkingone{group}{Group}
     \item \linkingone{group}{GroupElement}
     \item \linkingone{group}{GenerateGroup}
     \item \linkingone{group}{AbelianGenerate}
   \end{itemize}
 \end{itemize}

\C

 \subsection{\negok Group -- group structure}\linkedone{group}{Group}
 \initialize
  \func{Group}{\hiki{value}{class},\ \hikiopt{operation}{int}{-1}}{Group}\\
  \spacing
  % document of basic document
  \quad Create an object which wraps \param{value} (typically a ring or a field)
  only to expose its group structure. \\
  \spacing
  % added document
  \quad The instance has methods defined for (abstract) group.
  For example, \linkingtwo{group}{Group}{identity} returns the identity element of the group from wrapped \param{value}.\\
  \spacing
  % input, output document
  \quad \param{value} must be an instance of a class expresses group structure.
  \param{operation} must be 0 or 1;
  If \param{operation} is 0, \param{value} is regarded as the additive group.
  On the other hand, if \param{operation} is 1, \param{value} is considered as the multiplicative group.
  The default value of \param{operation} is 0.\\
  \negok You can input an instance of \linkingone{group}{Group} itself as \param{value}.
  In this case, the default value of \param{operation} is the attribute \linkingtwo{group}{Group}{operation} of the instance.
  \begin{at}
    \item[entity]\linkedtwo{group}{Group}{entity}:\\ The wrapped object.
    \item[operation]\linkedtwo{group}{Group}{operation}:\\ It expresses the mode of operation;
$0$ means additive, while $1$ means multiplicative.
  \end{at}
  \begin{op}
    \verb+A==B+ & Return whether A and B are equal or not.\\
    \verb+A!=B+ & Check whether A and B are not equal.\\
    \verb+repr(A)+ & representation\\
    \verb+str(A)+ & simple representation\\
  \end{op} 
\begin{ex}
>>> G1=group.Group(finitefield.FinitePrimeField(37), 1)
>>> print G1
F_37
>>> G2=group.Group(intresidue.IntegerResidueClassRing(6), 0)
>>> print G2
Z/6Z
\end{ex}%Don't indent!
  \method
  \subsubsection{setOperation -- change operation}\linkedtwo{group}{Group}{setOperation}
   \func{setOperation}{\param{self},\ \hiki{operation}{int}}{(None)}\\
   \spacing
   % document of basic document
   \quad Change group type to additive ($0$) or multiplicative ($1$).\\
   \spacing
   % added document
   %\spacing
   % input, output document
   \quad \param{operation} must be 0 or 1.\\
 \subsubsection{\negok createElement -- generate a GroupElement instance}\linkedtwo{group}{Group}{createElement}
   \func{createElement}{\param{self},\ \param{*value}}{\out{GroupElement}}\\
   \spacing
   % document of basic document
   \quad Return \linkingone{group}{GroupElement} object whose group is \param{self}, initialized with \param{value}.\\
   \spacing
   % added document
   \quad \negok This method calls \param{self}.\linkingtwo{group}{Group}{entity}.createElement.\\
   \spacing
   % input, output document
   \quad \param{value} must fit the form of argument for \param{self}.\linkingtwo{group}{Group}{entity}.createElement.\\
  \subsubsection{\negok identity -- identity element}\linkedtwo{group}{Group}{identity}
   \func{identity}{\param{self}}{\param{GroupElement}}\\
   \spacing
   % document of basic document
   \quad Return identity element (unit) of group.\\
   \spacing
   % added document
   \quad Return zero (additive) or one (multiplicative) corresponding to \linkingtwo{group}{Group}{operation}.
   \negok This method calls \param{self}.\linkingtwo{group}{Group}{entity}.identity or \linkingtwo{group}{Group}{entity} does not have the attribute then returns one or zero.
   \spacing
   % input, output document
  \subsubsection{grouporder -- order of the group}\linkedtwo{group}{Group}{grouporder}
   \func{grouporder}{\param{self}}{\param{long}}\\
   \spacing
   % document of basic document
   \quad Return group order (cardinality) of \param{self}.\\
   \spacing
   % added document
   \quad \negok This method calls \param{self}.\linkingtwo{group}{Group}{entity}.grouporder, card or \_\_len\_\_.\\
   We assume that the group is finite, so returned value is expected as some long integer.
   If the group is infinite, we do not define the type of output by the method.
   \spacing
   % input, output document
\begin{ex}
>>> G1=group.Group(finitefield.FinitePrimeField(37), 1)
>>> G1.grouporder()
36
>>> G1.setOperation(0)
>>> print G1.identity()
FinitePrimeField,0 in F_37
>>> G1.grouporder()
37
\end{ex}%Don't indent!
\C

 \subsection{GroupElement -- elements of group structure}\linkedone{group}{GroupElement}
 \initialize
  \func{GroupElement}{\hiki{value}{class},\ \hikiopt{operation}{int}{-1}}{GroupElement}\\
  \spacing
  % document of basic document
  \quad Create an object which wraps \param{value} (typically a ring element or
  a field element) to make it behave as an element of group.\\
  \spacing 
  % added document
  \quad The instance has methods defined for an (abstract) element of group.
  For example, \linkingtwo{group}{GroupElement}{inverse} returns the inverse element of \param{value} as the element of group object.\\
  \spacing
  % input, output document
  \quad \param{value} must be an instance of a class expresses an element of group structure.
  \param{operation} must be $0$ or $1$;
  If \param{operation} is $0$, \param{value} is regarded as the additive group.
  On the other hand, if \param{operation} is $1$, \param{value} is considered as the multiplicative group.
  The default value of \param{operation} is $0$.\\
  \negok You can input an instance of \linkingone{group}{GroupElement} itself as \param{value}.
  In this case, the default value of \param{operation} is the attribute \linkingtwo{group}{GroupElement}{operation} of the instance.
  \begin{at}
    \item[entity]\linkedtwo{group}{GroupElement}{entity}:\\ The wrapped object.
    \item[set]\linkedtwo{group}{GroupElement}{set}:\\ It is an instance of \linkingone{group}{Group}, which expresses the group to which \param{self} belongs.
    \item[operation]\linkedtwo{group}{GroupElement}{operation}:\\ It expresses the mode of operation;
$0$ means additive, while $1$ means multiplicative.
  \end{at}
  \begin{op}
    \verb+A==B+ & Return whether A and B are equal or not.\\
    \verb+A!=B+ & Check whether A and B are not equal.\\
    \verb+A.ope(B)+ & Basic operation (additive $+$, multiplicative $*$)\\
    \verb+A.ope2(n)+ & Extended operation (additive $*$, multiplicative $**$)\\
    \verb+A.inverse()+\linkedtwo{group}{GroupElement}{inverse} & Return the inverse element of \param{self}\\
    \verb+repr(A)+ & representation\\
    \verb+str(A)+ & simple representation\\
  \end{op} 
\begin{ex}
>>> G1=group.GroupElement(finitefield.FinitePrimeFieldElement(18, 37), 1)
>>> print G1
FinitePrimeField,18 in F_37
>>> G2=group.Group(intresidue.IntegerResidueClass(3, 6), 0)
IntegerResidueClass(3, 6)
\end{ex}%Don't indent!
  \method
  \subsubsection{setOperation -- change operation}\linkedtwo{group}{GroupElement}{setOperation}
   \func{setOperation}{\param{self},\ \hiki{operation}{int}}{(None)}\\
   \spacing
   % document of basic document
   \quad Change group type to additive ($0$) or multiplicative ($1$).\\
   \spacing
   % added document
   %\spacing
   % input, output document
   \quad \param{operation} must be $0$ or $1$.\\
 \subsubsection{\negok getGroup -- generate a Group instance}\linkedtwo{group}{GroupElement}{getGroup}
   \func{getGroup}{\param{self}}{\out{Group}}\\
   \spacing
   % document of basic document
   \quad Return \linkingone{group}{Group} object to which \param{self} belongs.\\
   \spacing
   % added document
   \quad \negok This method calls \param{self}.\linkingtwo{group}{GroupElement}{entity}.getRing or getGroup.\\
   \negok In an initialization of \linkingone{group}{GroupElement}, the attribute \linkingtwo{group}{GroupElement}{set} is set as the value returned from the method.\\
   \spacing
   % input, output document
  \subsubsection{order -- order by factorization method}\linkedtwo{group}{GroupElement}{order}
   \func{order}{\param{self}}{\param{long}}\\
   \spacing
   % document of basic document
   \quad Return the order of \param{self}.\\
   \spacing
   % added document
   \quad \negok This method uses the factorization of order of group.\\
   \negok We assume that the group is finite, so returned value is expected as some long integer.
   \negok If the group is infinite, the method would raise an error or return an invalid value.\\
   \spacing
   % input, output document
  \subsubsection{t\_order -- order by baby-step giant-step}\linkedtwo{group}{GroupElement}{t\_order}
   \func{t\_order}{\param{self},\ \hikiopt{v}{int}{2}}{\param{long}}\\
   \spacing
   % document of basic document
   \quad Return the order of \param{self}.\\
   \spacing
   % added document
   \quad \negok This method uses Terr's baby-step giant-step algorithm.\\
   This method does not use the order of group.
   You can put number of baby-step to \param{v}.
   \negok We assume that the group is finite, so returned value is expected as some long integer.
   \negok If the group is infinite, the method would raise an error or return an invalid value.\\
   \spacing
   % input, output document
   \quad \param{v} must be some int integer.\\
\begin{ex}
>>> G1=group.GroupElement(finitefield.FinitePrimeFieldElement(18, 37), 1)
>>> G1.order()
36
>>> G1.t_order()
36
\end{ex}%Don't indent!
\C

 \subsection{\negok GenerateGroup -- group structure with generator}\linkedone{group}{GenerateGroup}
 \initialize
  \func{GenerateGroup}{\hiki{value}{class},\ \hikiopt{operation}{int}{-1}}{GroupElement}\\
  \spacing
  % document of basic document
  \quad Create an object which is generated by \param{value} as the element of group structure. \\
  \spacing
  % added document
  \quad This initializes a group `including' the group elements, not a group with generators, now.
  We do not recommend using this module now.
  The instance has methods defined for an (abstract) element of group.
  For example, \linkingtwo{group}{GroupElement}{inverse} returns the inverse element of \param{value} as the element of group object.\\
  The class inherits the class \linkingone{group}{Group}.\\
  \spacing
  % input, output document
  \quad \param{value} must be a list of generators.
  Each generator should be an instance of a class expresses an element of group structure.
  \param{operation} must be $0$ or $1$;
  If \param{operation} is $0$, \param{value} is regarded as the additive group.
  On the other hand, if \param{operation} is $1$, \param{value} is considered as the multiplicative group.
  The default value of \param{operation} is $0$.\\
\begin{ex}
>>> G1=group.GenerateGroup([intresidue.IntegerResidueClass(2, 20),
... intresidue.IntegerResidueClass(6, 20)])
>>> G1.identity()
intresidue.IntegerResidueClass(0, 20)
\end{ex}%Don't indent!
\C

 \subsection{AbelianGenerate -- abelian group structure with generator}\linkedone{group}{AbelianGenerate}
 \initialize
  \quad The class inherits the class \linkingone{group}{GenerateGroup}.\\
  
  \subsubsection{relationLattice -- relation between generators}\linkedtwo{group}{AbelianGenerate}{relationLattice}
   \func{relationLattice}{\param{self}}{\out{\linkingone{matrix}{Matrix}}}\\
   \spacing
   % document of basic document
   \quad Return a list of relation lattice basis as a square matrix each of whose column vector is a relation basis.\\
   \spacing
   % added document
   \quad The relation basis, $V$ satisfies that $\prod_{i} \mbox{generator}_i V_i=1$.
   \spacing
   % input, output document
  \subsubsection{computeStructure -- abelian group structure}\linkedtwo{group}{AbelianGenerate}{computeStructure}
   \func{computeStructure}{\param{self}}{\param{tuple}}\\
   \spacing
   % document of basic document
   \quad Compute finite abelian group structure.\\
   \spacing
   % added document
   \quad If \param{self} $G \simeq \oplus_i <h_i>$, return [($h_1$, ord($h_1$)),..($h_n$, ord($h_n$))] and $^\# G$,
   where $<h_i>$ is a cyclic group with the generator $h_i$.\\
   \spacing
   % input, output document
   \quad The output is a tuple which has two elements;
   the first element is a list which elements are a list of $h_i$ and its order,
   on the other hand, the second element is the order of the group.
\begin{ex}
>>> G=AbelianGenerate([intresidue.IntegerResidueClass(2, 20),
... intresidue.IntegerResidueClass(6, 20)])
>>> G.relationLattice()
10 7
 0 1
>>> G.computeStructure()
([IntegerResidueClassRing,IntegerResidueClass(2, 20), 10)], 10L)
\end{ex}%Don't indent!
\C
%---------- end document ---------- %

\bibliographystyle{jplain}%use jbibtex
\bibliography{nzmath_references}

\end{document}

