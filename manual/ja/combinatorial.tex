\documentclass{report}

\documentclass{report}

%%%%%%%%%%%%%%%%%%%%%%%%%%%%%%%%%%%%%%%%%%%%%%%%%%%%%%%%%%%%%
%
% macros for nzmath manual
%
%%%%%%%%%%%%%%%%%%%%%%%%%%%%%%%%%%%%%%%%%%%%%%%%%%%%%%%%%%%%%
\usepackage{amssymb,amsmath}
\usepackage{color}
\usepackage[dvipdfm,bookmarks=true,bookmarksnumbered=true,%
 pdftitle={NZMATH Users Manual},%
 pdfsubject={Manual for NZMATH Users},%
 pdfauthor={NZMATH Development Group},%
 pdfkeywords={TeX; dvipdfmx; hyperref; color;},%
 colorlinks=true]{hyperref}
\usepackage{fancybox}
\usepackage[T1]{fontenc}
%
\newcommand{\DS}{\displaystyle}
\newcommand{\C}{\clearpage}
\newcommand{\NO}{\noindent}
\newcommand{\negok}{$\dagger$}
\newcommand{\spacing}{\vspace{1pt}\\ }
% software macros
\newcommand{\nzmathzero}{{\footnotesize $\mathbb{N}\mathbb{Z}$}\texttt{MATH}}
\newcommand{\nzmath}{{\nzmathzero}\ }
\newcommand{\pythonzero}{$\mbox{\texttt{Python}}$}
\newcommand{\python}{{\pythonzero}\ }
% link macros
\newcommand{\linkingzero}[1]{{\bf \hyperlink{#1}{#1}}}%module
\newcommand{\linkingone}[2]{{\bf \hyperlink{#1.#2}{#2}}}%module,class/function etc.
\newcommand{\linkingtwo}[3]{{\bf \hyperlink{#1.#2.#3}{#3}}}%module,class,method
\newcommand{\linkedzero}[1]{\hypertarget{#1}{}}
\newcommand{\linkedone}[2]{\hypertarget{#1.#2}{}}
\newcommand{\linkedtwo}[3]{\hypertarget{#1.#2.#3}{}}
\newcommand{\linktutorial}[1]{\href{http://docs.python.org/tutorial/#1}{#1}}
\newcommand{\linktutorialone}[2]{\href{http://docs.python.org/tutorial/#1}{#2}}
\newcommand{\linklibrary}[1]{\href{http://docs.python.org/library/#1}{#1}}
\newcommand{\linklibraryone}[2]{\href{http://docs.python.org/library/#1}{#2}}
\newcommand{\pythonhp}{\href{http://www.python.org/}{\python website}}
\newcommand{\nzmathwiki}{\href{http://nzmath.sourceforge.net/wiki/}{{\nzmathzero}Wiki}}
\newcommand{\nzmathsf}{\href{http://sourceforge.net/projects/nzmath/}{\nzmath Project Page}}
\newcommand{\nzmathtnt}{\href{http://tnt.math.se.tmu.ac.jp/nzmath/}{\nzmath Project Official Page}}
% parameter name
\newcommand{\param}[1]{{\tt #1}}
% function macros
\newcommand{\hiki}[2]{{\tt #1}:\ {\em #2}}
\newcommand{\hikiopt}[3]{{\tt #1}:\ {\em #2}=#3}

\newdimen\hoge
\newdimen\truetextwidth
\newcommand{\func}[3]{%
\setbox0\hbox{#1(#2)}
\hoge=\wd0
\truetextwidth=\textwidth
\advance \truetextwidth by -2\oddsidemargin
\ifdim\hoge<\truetextwidth % short form
{\bf \colorbox{skyyellow}{#1(#2)\ $\to$ #3}}
%
\else % long form
\fcolorbox{skyyellow}{skyyellow}{%
   \begin{minipage}{\textwidth}%
   {\bf #1(#2)\\ %
    \qquad\quad   $\to$\ #3}%
   \end{minipage}%
   }%
\fi%
}

\newcommand{\out}[1]{{\em #1}}
\newcommand{\initialize}{%
  \paragraph{\large \colorbox{skyblue}{Initialize (Constructor)}}%
    \quad\\ %
    \vspace{3pt}\\
}
\newcommand{\method}{\C \paragraph{\large \colorbox{skyblue}{Methods}}}
% Attribute environment
\newenvironment{at}
{%begin
\paragraph{\large \colorbox{skyblue}{Attribute}}
\quad\\
\begin{description}
}%
{%end
\end{description}
}
% Operation environment
\newenvironment{op}
{%begin
\paragraph{\large \colorbox{skyblue}{Operations}}
\quad\\
\begin{table}[h]
\begin{center}
\begin{tabular}{|l|l|}
\hline
operator & explanation\\
\hline
}%
{%end
\hline
\end{tabular}
\end{center}
\end{table}
}
% Examples environment
\newenvironment{ex}%
{%begin
\paragraph{\large \colorbox{skyblue}{Examples}}
\VerbatimEnvironment
\renewcommand{\EveryVerbatim}{\fontencoding{OT1}\selectfont}
\begin{quote}
\begin{Verbatim}
}%
{%end
\end{Verbatim}
\end{quote}
}
%
\definecolor{skyblue}{cmyk}{0.2, 0, 0.1, 0}
\definecolor{skyyellow}{cmyk}{0.1, 0.1, 0.5, 0}
%
%\title{NZMATH User Manual\\ {\large{(for version 1.0)}}}
%\date{}
%\author{}
\begin{document}
%\maketitle
%
\setcounter{tocdepth}{3}
\setcounter{secnumdepth}{3}


\tableofcontents
\C

\chapter{Functions}


%---------- start document ---------- %
 \section{combinatorial -- combinatorial functions}\linkedzero{combinatorial}
%
  \subsection{binomial -- binomial coefficient}\linkedone{combinatorial}{binomial}
   \func{binomial}
   {%
     \hiki{n}{integer},\ %
     \hiki{m}{integer}
   }{%
     \out{integer}
   }\\
   \spacing
   % document of basic document
   \quad Return the binomial coefficient for \param{n} and \param{m}.
   In other words, $\displaystyle{\frac{\param{n} !}{(\param{n} - \param{m}) ! \param{m} !}}$.\\
   \spacing
   % added document
   \negok For convenience, {\tt binomial(n, n+i)} returns \(0\) for positive \(i\), and {\tt binomial(0,0)} returns \(1\).\\
   \spacing
   % input, output document
   \quad \param{n} must be a positive integer and \param{m} must be
   a non-negative integer. \\

  \subsection{combinationIndexGenerator -- iterator for combinations}\linkedone{combinatorial}{combinationIndexGenerator}

   \func{combinationIndexGenerator}{%
     \hiki{n}{integer},\ %
     \hiki{m}{integer}
   }{%
     \out{iterator}
   }\\
   \spacing
   % document of basic document
   \quad Return an iterator which generates indices of \param{m}
   element subsets of \param{n} element set.\\
   \spacing 
   \quad {\tt combination\_index\_generator}\linkedone{combinatorial}{combination\_index\_generator}
   is an alias of {\tt combinationIndexGenerator}.\\

  \subsection{factorial -- factorial}\linkedone{combinatorial}{factorial}
   \func{factorial}{%
     \hiki{n}{integer}
   }{%
     \out{integer}
   }\\
   \spacing
   % document of basic document
   \quad Return \(\param{n}!\) for non-negative integer \param{n}.\\

  \subsection{permutationGenerator -- iterator for permutation}\linkedone{combinatorial}{permutationGenerator}
  \func{permutationGenerator}{%
    \hiki{n}{integer}
  }{%
    \out{iterator}
  }
   \spacing
   \quad Generate all permutations of \param{n} elements as list iterator.\\
   % 
   \spacing 
   \quad The number of generated list is \param{n}'s
   \linkingone{combinatorial}{factorial}, so be careful to use
   big \param{n}.\\
   \spacing 
   \quad {\tt permutation\_generator}\linkedone{combinatorial}{permutation\_generator}
   is an alias of {\tt permutationGenerator}.\\

  \subsection{fallingfactorial -- the falling factorial}\linkedone{combinatorial}{fallingfactorial}
   \func{fallingfactorial}{%
     \hiki{n}{integer},\ %
     \hiki{m}{integer}
   }{%
     \out{integer}
   }\\
   \spacing
   % document of basic document
   \quad Return the falling factorial; \param{n} to the \param{m} falling,
   i.e. \(n(n-1)\cdots(n-m+1)\).\\

  \subsection{risingfactorial -- the rising factorial}\linkedone{combinatorial}{risingfactorial}
   \func{risingfactorial}{%
     \hiki{n}{integer},\ %
     \hiki{m}{integer}
   }{%
     \out{integer}
   }\\
   \spacing
   % document of basic document
   \quad Return the rising factorial; \param{n} to the \param{m} rising,
   i.e.\, \(n(n+1)\cdots(n+m-1)\).\\

  \subsection{multinomial -- the multinomial coefficient}\linkedone{combinatorial}{multinomial}
   \func{multinomial}{%
     \hiki{n}{integer},\ %
     \hiki{parts}{list}
   }{%
     \out{integer}
   }\\
   \spacing
   % document of basic document
   \quad Return the multinomial coefficient.\\
   \spacing
   % input, output document
   \quad \param{parts} must be a sequence of natural numbers and the sum of elements in \param{parts} should be equal to \param{n}.\\

  \subsection{bernoulli -- the Bernoulli number}\linkedone{combinatorial}{bernoulli}
   \func{bernoulli}{%
     \hiki{n}{integer}
   }{%
     \out{Rational}
   }\\
   \spacing
   % document of basic document
   \quad Return the \param{n}-th Bernoulli number.\\

  \subsection{catalan -- the Catalan number}\linkedone{combinatorial}{catalan}
   \func{catalan}{%
     \hiki{n}{integer}
   }{%
     \out{integer}
   }\\
   \spacing
   % document of basic document
   \quad Return the \param{n}-th Catalan number.\\

  \subsection{euler -- the Euler number}\linkedone{combinatorial}{euler}
   \func{euler}{%
     \hiki{n}{integer}
   }{%
     \out{integer}
   }\\
   \spacing
   % document of basic document
   \quad Return the \param{n}-th Euler number.\\

  \subsection{bell -- the Bell number}\linkedone{combinatorial}{bell}
   \func{bell}{%
     \hiki{n}{integer}
   }{%
     \out{integer}
   }\\
   \spacing
   % document of basic document
   \quad Return the \param{n}-th Bell number.\\
   \spacing
   % added document
   \quad The Bell number \(b\) is defined by:
   \[b(n) = \sum_{i=0}^{n} S(n, i),\ \]
   where \(S\) denotes Stirling number of the second kind (\linkingone{combinatorial}{stirling2}).\\

  \subsection{stirling1 -- Stirling number of the first kind}\linkedone{combinatorial}{stirling1}
   \func{stirling1}{%
     \hiki{n}{integer},\ %
     \hiki{m}{integer}
   }{%
     \out{integer}
   }\\
   \spacing
   % document of basic document
   \quad Return Stirling number of the first kind.\\
   \spacing
   % added document
   \quad Let \(s\) denote the Stirling number and \((x)_n\) the falling factorial, then
   \[(x)_n = \sum_{i=0}^{n} s(n,\ i) x^i. \]\\
   \(s\) satisfies the recurrence relation:
   \[s(n,\ m) = s(n-1,\ m-1) - (n-1)s(n-1,\ m)\ .\]\\

  \subsection{stirling2 -- Stirling number of the second kind}\linkedone{combinatorial}{stirling2}
   \func{stirling2}{%
     \hiki{n}{integer},\ %
     \hiki{m}{integer}
   }{%
     \out{integer}
   }\\
   \spacing
   % document of basic document
   \quad Return Stirling number of the second kind.\\
   \spacing
   % added document
   \quad Let \(S\) denote the Stirling number, \((x)_i\) falling factorial, then:
   \[x^n = \sum_{i=0}^{n} S(n,\ i) (x)_i\]
   \(S\) satisfies:
   \[S(n,\ m) = S(n-1,\ m-1) + m S(n-1,\ m)\]\\

  \subsection{partition\_number -- the number of partitions}\linkedone{combinatorial}{partition\_number}
   \func{partition\_number}{%
     \hiki{n}{integer}
   }{%
     \out{integer}
   }\\
   \spacing
   % document of basic document
   \quad Return the number of partitions of \param{n}.\\

  \subsection{partitionGenerator -- iterator for partition}\linkedone{combinatorial}{partitionGenerator}
   \func{partitionGenerator}{%
     \hiki{n}{integer},\ %
     \hikiopt{maxi}{integer}{0}
   }{%
     \out{iterator}
   }\\
   \spacing
   % document of basic document
   \quad Return an iterator which generates partitions of \param{n}.\\
   % input, output document
   \spacing
   \quad If \param{maxi} is given, then summands are limited not to exceed \param{maxi}.\\
   \quad The number of partitions (given by
   \linkingone{combinatorial}{partition\_number}) grows exponentially,
   so be careful to use big \param{n}.\\
   \spacing 
   \quad {\tt partition\_generator}\linkedone{combinatorial}{partition\_generator}
   is an alias of {\tt partitionGenerator}.\\

  \subsection{partition\_conjugate -- the conjugate of partition}\linkedone{combinatorial}{partition\_conjugate}
   \func{partition\_conjugate}{%
     \hiki{partition}{tuple}
   }{%
     \out{tuple}
   }\\
   \spacing
   % document of basic document
   \quad Return the conjugate of \param{partition}.\\

\begin{ex}
>>> combinatorial.binomial(5, 2)
10L
>>> combinatorial.factorial(3)
6L
>>> combinatorial.fallingfactorial(7, 3) == 7 * 6 * 5
True
>>> combinatorial.risingfactorial(7, 3) == 7 * 8 * 9
True
>>> combinatorial.multinomial(7, [2, 2, 3])
210L
>>> for idx in combinatorial.combinationIndexGenerator(5, 3):
...     print idx
...
[0, 1, 2]
[0, 1, 3]
[0, 1, 4]
[0, 2, 3]
[0, 2, 4]
[0, 3, 4]
[1, 2, 3]
[1, 2, 4]
[1, 3, 4]
[2, 3, 4]
>>> for part in combinatorial.partitionGenerator(5):
...     print part
...
(5,)
(4, 1)
(3, 2)
(3, 1, 1)
(2, 2, 1)
(2, 1, 1, 1)
(1, 1, 1, 1, 1)
>>> combinatorial.partition_number(5)
7
>>> def limited_summands(n, maxi):
...     "partition with limited number of summands"
...     for part in combinatorial.partitionGenerator(n, maxi):
...         yield combinatorial.partition_conjugate(part)
...
>>> for part in limited_summands(5, 3):
...     print part
...
(2, 2, 1)
(3, 1, 1)
(3, 2)
(4, 1)
(5,)
\end{ex}%Don't indent!(indent causes an error.)

\C

%---------- end document ---------- %

\bibliographystyle{jplain}%use jbibtex
\bibliography{nzmath_references}

\end{document}

