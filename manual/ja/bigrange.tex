\documentclass{report}

%%%%%%%%%%%%%%%%%%%%%%%%%%%%%%%%%%%%%%%%%%%%%%%%%%%%%%%%%%%%%
%
% macros for nzmath manual
%
%%%%%%%%%%%%%%%%%%%%%%%%%%%%%%%%%%%%%%%%%%%%%%%%%%%%%%%%%%%%%
\usepackage{amssymb,amsmath}
\usepackage{color}
\usepackage[dvipdfm,bookmarks=true,bookmarksnumbered=true,%
 pdftitle={NZMATH Users Manual},%
 pdfsubject={Manual for NZMATH Users},%
 pdfauthor={NZMATH Development Group},%
 pdfkeywords={TeX; dvipdfmx; hyperref; color;},%
 colorlinks=true]{hyperref}
\usepackage{fancybox}
\usepackage[T1]{fontenc}
%
\newcommand{\DS}{\displaystyle}
\newcommand{\C}{\clearpage}
\newcommand{\NO}{\noindent}
\newcommand{\negok}{$\dagger$}
\newcommand{\spacing}{\vspace{1pt}\\ }
% software macros
\newcommand{\nzmathzero}{{\footnotesize $\mathbb{N}\mathbb{Z}$}\texttt{MATH}}
\newcommand{\nzmath}{{\nzmathzero}\ }
\newcommand{\pythonzero}{$\mbox{\texttt{Python}}$}
\newcommand{\python}{{\pythonzero}\ }
% link macros
\newcommand{\linkingzero}[1]{{\bf \hyperlink{#1}{#1}}}%module
\newcommand{\linkingone}[2]{{\bf \hyperlink{#1.#2}{#2}}}%module,class/function etc.
\newcommand{\linkingtwo}[3]{{\bf \hyperlink{#1.#2.#3}{#3}}}%module,class,method
\newcommand{\linkedzero}[1]{\hypertarget{#1}{}}
\newcommand{\linkedone}[2]{\hypertarget{#1.#2}{}}
\newcommand{\linkedtwo}[3]{\hypertarget{#1.#2.#3}{}}
\newcommand{\linktutorial}[1]{\href{http://docs.python.org/tutorial/#1}{#1}}
\newcommand{\linktutorialone}[2]{\href{http://docs.python.org/tutorial/#1}{#2}}
\newcommand{\linklibrary}[1]{\href{http://docs.python.org/library/#1}{#1}}
\newcommand{\linklibraryone}[2]{\href{http://docs.python.org/library/#1}{#2}}
\newcommand{\pythonhp}{\href{http://www.python.org/}{\python website}}
\newcommand{\nzmathwiki}{\href{http://nzmath.sourceforge.net/wiki/}{{\nzmathzero}Wiki}}
\newcommand{\nzmathsf}{\href{http://sourceforge.net/projects/nzmath/}{\nzmath Project Page}}
\newcommand{\nzmathtnt}{\href{http://tnt.math.se.tmu.ac.jp/nzmath/}{\nzmath Project Official Page}}
% parameter name
\newcommand{\param}[1]{{\tt #1}}
% function macros
\newcommand{\hiki}[2]{{\tt #1}:\ {\em #2}}
\newcommand{\hikiopt}[3]{{\tt #1}:\ {\em #2}=#3}

\newdimen\hoge
\newdimen\truetextwidth
\newcommand{\func}[3]{%
\setbox0\hbox{#1(#2)}
\hoge=\wd0
\truetextwidth=\textwidth
\advance \truetextwidth by -2\oddsidemargin
\ifdim\hoge<\truetextwidth % short form
{\bf \colorbox{skyyellow}{#1(#2)\ $\to$ #3}}
%
\else % long form
\fcolorbox{skyyellow}{skyyellow}{%
   \begin{minipage}{\textwidth}%
   {\bf #1(#2)\\ %
    \qquad\quad   $\to$\ #3}%
   \end{minipage}%
   }%
\fi%
}

\newcommand{\out}[1]{{\em #1}}
\newcommand{\initialize}{%
  \paragraph{\large \colorbox{skyblue}{Initialize (Constructor)}}%
    \quad\\ %
    \vspace{3pt}\\
}
\newcommand{\method}{\C \paragraph{\large \colorbox{skyblue}{Methods}}}
% Attribute environment
\newenvironment{at}
{%begin
\paragraph{\large \colorbox{skyblue}{Attribute}}
\quad\\
\begin{description}
}%
{%end
\end{description}
}
% Operation environment
\newenvironment{op}
{%begin
\paragraph{\large \colorbox{skyblue}{Operations}}
\quad\\
\begin{table}[h]
\begin{center}
\begin{tabular}{|l|l|}
\hline
operator & explanation\\
\hline
}%
{%end
\hline
\end{tabular}
\end{center}
\end{table}
}
% Examples environment
\newenvironment{ex}%
{%begin
\paragraph{\large \colorbox{skyblue}{Examples}}
\VerbatimEnvironment
\renewcommand{\EveryVerbatim}{\fontencoding{OT1}\selectfont}
\begin{quote}
\begin{Verbatim}
}%
{%end
\end{Verbatim}
\end{quote}
}
%
\definecolor{skyblue}{cmyk}{0.2, 0, 0.1, 0}
\definecolor{skyyellow}{cmyk}{0.1, 0.1, 0.5, 0}
%
%\title{NZMATH User Manual\\ {\large{(for version 1.0)}}}
%\date{}
%\author{}
\begin{document}
%\maketitle
%
\setcounter{tocdepth}{3}
\setcounter{secnumdepth}{3}


\tableofcontents
\C

\chapter{Functions}


%---------- start document ---------- %
 \section{bigrange -- range-like generator functions}\linkedzero{bigrange}
%
  \subsection{count -- count up}\linkedone{bigrange}{count}
   \func{count}
   {%
     \hikiopt{n}{integer}{0}
   }{%
     {\em iterator}
   }\\
   \spacing
   % document of basic document
   \quad \param{n}�܂Ő����グ��B .
   \linklibrary{itertools}.\linklibraryone{itertools\#count}{count}�Q�ƁB\\
   \spacing
   % added document
   %\spacing
   % input, output document
   \quad \param{n} must be int, long or rational.Integer.\\

  \subsection{range -- range-like iterator}\linkedone{bigrange}{range}
   \func{range}
   {%
     \hiki{start}{integer},\ %
     \hikiopt{stop}{integer}{None},\ %
     \hikiopt{step}{integer}{1}
   }{%
     {\em iterator}
   }\\
   \spacing
   % document of basic document
   \quad ���{���Ή���range�ł���B\\
   \spacing
   % added document
   \quad  It can generate more than
   \linklibrary{sys}.\linklibraryone{sys\#maxint}{maxint} elements,
   which is the limitation of the \linklibraryone{functions\#range}{range}
   built-in function.\\
   \spacing
   % input, output document
   \quad The argument names do not correspond to their roles, but
   users are familiar with the
   \linklibraryone{functions\#range}{range} built-in function of
   \python and understand the semantics.
   Note that the output is not a list.\\
%
\begin{ex}
>>> range(1, 100, 3) # built-in
[1, 4, 7, 10, 13, 16, 19, 22, 25, 28, 31, 34, 37, 40, 43, 46,
 49, 52, 55, 58, 61, 64, 67, 70, 73, 76, 79, 82, 85, 88, 91,
 94, 97]
>>> bigrange.range(1, 100, 3)
<generator object at 0x18f8c8>
\end{ex}%Don't indent!(indent causes an error.)

  \subsection{arithmetic\_progression -- arithmetic progression iterator}\linkedone{bigrange}{arithmetic\_progression}
   \func{arithmetic\_progression}
   {%
     \hiki{init}{integer},\ %
     \hiki{difference}{integer}
   }{%
     {\em iterator}
   }\\
   \spacing
   % document of basic document
   \quad Return an iterator which generates an arithmetic progression
   starting form \param{init} and \param{difference} step.\\

  \subsection{geometric\_progression -- geometric progression iterator}\linkedone{bigrange}{geometric\_progression}
   \func{geometric\_progression}
   {%
     \hiki{init}{integer},\ %
     \hiki{ratio}{integer}
   }{%
     {\em iterator}
   }\\
   \spacing
   % document of basic document
   \quad Return an iterator which generates a geometric progression
   starting form \param{init} and multiplying \param{ratio}.\\

  \subsection{multirange -- multiple range iterator}\linkedone{bigrange}{multirange}
   \func{multirange}
   {%
     \hiki{triples}{list of range triples}
   }{%
     {\em iterator}
   }\\
   \spacing
   % document of basic document
   \quad Return an iterator over Cartesian product of elements of ranges.\\
   \spacing
   % added document
   \quad Be cautious that using multirange usually means you are
   trying to do brute force looping.\\
   \spacing
   % input / output
   \quad The range triples may be doubles {\tt (start, stop)} or single {\tt (stop,)},
   but they have to be always tuples.\\
   
%
\begin{ex}
>>> bigrange.multirange([(1, 10, 3), (1, 10, 4)])
<generator object at 0x18f968>
>>> list(_)
[(1, 1), (1, 5), (1, 9), (4, 1), (4, 5), (4, 9), (7, 1),
 (7, 5), (7, 9)]
\end{ex}%Don't indent!(indent causes an error.)

  \subsection{multirange\_restrictions -- multiple range iterator with restrictions}\linkedone{bigrange}{multirange\_restrictions}
   \func{multirange\_restrictions}
   {%
     \hiki{triples}{list of range triples},\ %
     \hiki{**kwds}{keyword arguments}
   }{%
     {\em iterator}
   }\\
   \spacing
   % document of basic document
   \quad {\tt multirange\_restrictions} is an iterator similar to the
   {\tt multirange} but putting restrictions on each ranges.\\
   \spacing
   % added document
   \quad Restrictions are specified by keyword arguments: \param{ascending},
   \param{descending}, \param{strictly\_ascending} and
   \param{strictly\_descending}.

   A restriction \param{ascending}, for example, is a sequence that
   specifies the indices where the number emitted by the range should
   be greater than or equal to the number at the previous index.
   Other restrictions \param{descending}, \param{strictly\_ascending}
   and \param{strictly\_descending} are similar.  Compare the examples
   below and of \linkingone{bigrange}{multirange}.

%
\begin{ex}
>>> bigrange.multirange_restrictions([(1, 10, 3), (1, 10, 4)], ascending=(1,))
<generator object at 0x18f978>
>>> list(_)
[(1, 1), (1, 5), (1, 9), (4, 5), (4, 9), (7, 9)]
\end{ex}%Don't indent!(indent causes an error.)
\C
%---------- end document ---------- %

\bibliographystyle{jplain}%use jbibtex
\bibliography{nzmath_references}

\end{document}

