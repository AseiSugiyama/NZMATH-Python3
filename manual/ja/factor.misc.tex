\documentclass{report}

\documentclass{report}

%%%%%%%%%%%%%%%%%%%%%%%%%%%%%%%%%%%%%%%%%%%%%%%%%%%%%%%%%%%%%
%
% macros for nzmath manual
%
%%%%%%%%%%%%%%%%%%%%%%%%%%%%%%%%%%%%%%%%%%%%%%%%%%%%%%%%%%%%%
\usepackage{amssymb,amsmath}
\usepackage{color}
\usepackage[dvipdfm,bookmarks=true,bookmarksnumbered=true,%
 pdftitle={NZMATH Users Manual},%
 pdfsubject={Manual for NZMATH Users},%
 pdfauthor={NZMATH Development Group},%
 pdfkeywords={TeX; dvipdfmx; hyperref; color;},%
 colorlinks=true]{hyperref}
\usepackage{fancybox}
\usepackage[T1]{fontenc}
%
\newcommand{\DS}{\displaystyle}
\newcommand{\C}{\clearpage}
\newcommand{\NO}{\noindent}
\newcommand{\negok}{$\dagger$}
\newcommand{\spacing}{\vspace{1pt}\\ }
% software macros
\newcommand{\nzmathzero}{{\footnotesize $\mathbb{N}\mathbb{Z}$}\texttt{MATH}}
\newcommand{\nzmath}{{\nzmathzero}\ }
\newcommand{\pythonzero}{$\mbox{\texttt{Python}}$}
\newcommand{\python}{{\pythonzero}\ }
% link macros
\newcommand{\linkingzero}[1]{{\bf \hyperlink{#1}{#1}}}%module
\newcommand{\linkingone}[2]{{\bf \hyperlink{#1.#2}{#2}}}%module,class/function etc.
\newcommand{\linkingtwo}[3]{{\bf \hyperlink{#1.#2.#3}{#3}}}%module,class,method
\newcommand{\linkedzero}[1]{\hypertarget{#1}{}}
\newcommand{\linkedone}[2]{\hypertarget{#1.#2}{}}
\newcommand{\linkedtwo}[3]{\hypertarget{#1.#2.#3}{}}
\newcommand{\linktutorial}[1]{\href{http://docs.python.org/tutorial/#1}{#1}}
\newcommand{\linktutorialone}[2]{\href{http://docs.python.org/tutorial/#1}{#2}}
\newcommand{\linklibrary}[1]{\href{http://docs.python.org/library/#1}{#1}}
\newcommand{\linklibraryone}[2]{\href{http://docs.python.org/library/#1}{#2}}
\newcommand{\pythonhp}{\href{http://www.python.org/}{\python website}}
\newcommand{\nzmathwiki}{\href{http://nzmath.sourceforge.net/wiki/}{{\nzmathzero}Wiki}}
\newcommand{\nzmathsf}{\href{http://sourceforge.net/projects/nzmath/}{\nzmath Project Page}}
\newcommand{\nzmathtnt}{\href{http://tnt.math.metro-u.ac.jp/nzmath/}{\nzmath Project Official Page}}
% parameter name
\newcommand{\param}[1]{{\tt #1}}
% function macros
\newcommand{\hiki}[2]{{\tt #1}:\ {\em #2}}
\newcommand{\hikiopt}[3]{{\tt #1}:\ {\em #2}=#3}

\newdimen\hoge
\newdimen\truetextwidth
\newcommand{\func}[3]{%
\setbox0\hbox{#1(#2)}
\hoge=\wd0
\truetextwidth=\textwidth
\advance \truetextwidth by -2\oddsidemargin
\ifdim\hoge<\truetextwidth % short form
{\bf \colorbox{skyyellow}{#1(#2)\ $\to$ #3}}
%
\else % long form
\fcolorbox{skyyellow}{skyyellow}{%
   \begin{minipage}{\textwidth}%
   {\bf #1(#2)\\ %
    \qquad\quad   $\to$\ #3}%
   \end{minipage}%
   }%
\fi%
}

\newcommand{\out}[1]{{\em #1}}
\newcommand{\initialize}{%
  \paragraph{\large \colorbox{skyblue}{Initialize (Constructor)}}%
    \quad\\ %
    \vspace{3pt}\\
}
\newcommand{\method}{\C \paragraph{\large \colorbox{skyblue}{Methods}}}
% Attribute environment
\newenvironment{at}
{%begin
\paragraph{\large \colorbox{skyblue}{Attribute}}
\quad\\
\begin{description}
}%
{%end
\end{description}
}
% Operation environment
\newenvironment{op}
{%begin
\paragraph{\large \colorbox{skyblue}{Operations}}
\quad\\
\begin{table}[h]
\begin{center}
\begin{tabular}{|l|l|}
\hline
operator & explanation\\
\hline
}%
{%end
\hline
\end{tabular}
\end{center}
\end{table}
}
% Examples environment
\newenvironment{ex}%
{%begin
\paragraph{\large \colorbox{skyblue}{Examples}}
\VerbatimEnvironment
\renewcommand{\EveryVerbatim}{\fontencoding{OT1}\selectfont}
\begin{quote}
\begin{Verbatim}
}%
{%end
\end{Verbatim}
\end{quote}
}
%
\definecolor{skyblue}{cmyk}{0.2, 0, 0.1, 0}
\definecolor{skyyellow}{cmyk}{0.1, 0.1, 0.5, 0}
%
%\title{NZMATH User Manual\\ {\large{(for version 1.0)}}}
%\date{}
%\author{}
\begin{document}
%\maketitle
%
\setcounter{tocdepth}{3}
\setcounter{secnumdepth}{3}


\tableofcontents
\C

\chapter{Classes}


%---------- start document ---------- %
 \section{factor.misc -- miscellaneous functions related factoring}\linkedzero{factor.misc}
 \begin{itemize}
   \item {\bf Functions}
     \begin{itemize}
       \item \linkingone{factor.misc}{allDivisors}
       \item \linkingone{factor.misc}{primeDivisors}
       \item \linkingone{factor.misc}{primePowerTest}
       \item \linkingone{factor.misc}{squarePart}
     \end{itemize}
   \item {\bf Classes}
   \begin{itemize}
     \item \linkingone{factor.misc}{FactoredInteger}
   \end{itemize}
 \end{itemize}
%
  \subsection{allDivisors -- all divisors}\linkedone{factor.misc}{allDivisors}
   \func{allDivisors}{\hiki{n}{integer}}{\out{list}}\\
   \spacing
   % document of basic document
   \quad Return all factors divide \param{n} as a list.\\
%
  \subsection{primeDivisors -- prime divisors}\linkedone{factor.misc}{primeDivisors}
   \func{primeDivisors}{\hiki{n}{integer}}{\out{list}}\\
   \spacing
   % document of basic document
   \quad Return all prime factors divide \param{n} as a list.\\
%
  \subsection{primePowerTest -- prime power test}\linkedone{factor.misc}{primePowerTest}
   \func{primePowerTest}{\hiki{n}{integer}}{(\out{integer},\ \out{integer})}\\
   \spacing
   \quad Judge whether \param{n} is of the form \(p^k\) with a prime \(p\) or not.
   If it is true, then {\tt (p, k)} will be returned, otherwise {\tt (n, 0)}.\\
   \spacing
   \quad This function is based on Algo. 1.7.5 in \cite{Cohen1}.\\
%
  \subsection{squarePart -- square part}\linkedone{factor.misc}{squarePart}
   \func{squarePart}{\hiki{n}{integer}}{\out{integer}}\\
   \spacing
   \quad Return the largest integer whose square divides \param{n}.
%
\begin{ex}
>>> factor.misc.allDivisors(1001)
[1, 7, 11, 13L, 77, 91L, 143L, 1001L]
>>> factor.misc.primeDivisors(100)
[2, 5]
>>> factor.misc.primePowerTest(128)
(2, 7)
>>> factor.misc.squarePart(128)
8L
\end{ex}%Don't indent!(indent causes an error.)
\C

  \subsection{FactoredInteger -- integer with its factorization}\linkedone{factor.misc}{FactoredInteger}
  \initialize
   \func{FactoredInteger}{\hiki{integer}{integer},\ \hikiopt{factors}{dict}{\{\}}}{\out{FactoredInteger}}\\
   \spacing
   \quad Integer with its factorization information.\\
   \spacing
   \quad If \param{factors} is given, it is a dict of type
   {\tt {prime:exponent}} and the product of \(prime^{exponent}\)
   is equal to the \param{integer}. Otherwise, factorization is
   carried out in initialization.\\

   \func{from\_partial\_factorization}{\param{cls},\ \hiki{integer}{integer},\ \hiki{partial}{dict}}{\out{FactoredInteger}}\\
   \spacing
   \quad A class method to create a new \linkingone{factor.misc}{FactoredInteger} object from partial factorization information \param{partial}.\\
   \begin{op}
     \verb+F * G+ & multiplication (other operand can be an int)\\
     \verb+F ** n+ & powering\\
     \verb+F == G+ & equal\\
     \verb+F != G+ & not equal\\
     \verb+F % G+ & remainder (the result is an int)\\
     \verb+F // G+ & same as \linkingtwo{factor.misc}{FactoredInteger}{exact\_division} method\\
     \verb+str(F)+ & string\\
     \verb+int(F)+ & convert to Python integer (forgetting factorization)\\
   \end{op}
   \method
   \subsubsection{is\_divisible\_by}\linkedtwo{factor.misc}{FactoredInteger}{is\_divisible\_by}
   \func{is\_divisible\_by}{\param{self},\ \hiki{other}{integer/\linkingone{factor.misc}{FactoredInteger}}}{\out{bool}}\\
   \spacing
   \quad Return True if \param{other} divides \param{self}.\\

   \subsubsection{exact\_division}\linkedtwo{factor.misc}{FactoredInteger}{exact\_division}
   \func{exact\_division}{\param{self},\ \hiki{other}{integer/\linkingone{factor.misc}{FactoredInteger}}}{\out{\linkingone{factor.misc}{FactoredInteger}}}\\
   \spacing
   \quad Divide by \param{other}. The \param{other} must divide \param{self}.\\

   \subsubsection{divisors}\linkedtwo{factor.misc}{FactoredInteger}{divisors}
   \func{divisors}{\param{self}}{\out{list}}\\
   \spacing
   \quad Return all divisors as a list.\\

   \subsubsection{proper\_divisors}\linkedtwo{factor.misc}{FactoredInteger}{proper\_divisors}
   \func{proper\_divisors}{\param{self}}{\out{list}}\\
   \spacing
   \quad Return all proper divisors (i.e. divisors excluding \(1\) and
   \param{self}) as a list.\\

   \subsubsection{prime\_divisors}\linkedtwo{factor.misc}{FactoredInteger}{prime\_divisors}
   \func{prime\_divisors}{\param{self}}{\out{list}}\\
   \spacing
   \quad Return all prime divisors as a list.\\

   \subsubsection{square\_part}\linkedtwo{factor.misc}{FactoredInteger}{square\_part}
   \func{square\_part}{\param{self},\ \hikiopt{asfactored}{bool}{False}}{\out{integer/\linkingone{factor.misc}{FactoredInteger object}}}\\
   \spacing
   \quad Return the largest integer whose square divides \param{self}.\\
   \spacing
   \quad If an optional argument \param{asfactored} is true,
   then the result is also a \linkingone{factor.misc}{FactoredInteger object}. (default is False)\\

   \subsubsection{squarefree\_part}\linkedtwo{factor.misc}{FactoredInteger}{squarefree\_part}
   \func{squarefree\_part}{\param{self},\ \hikiopt{asfactored}{bool}{False}}{\out{integer/\linkingone{factor.misc}{FactoredInteger object}}}\\
   \spacing
   \quad Return the largest squarefree integer which divides \param{self}.\\
   \spacing
   \quad If an optional argument \param{asfactored} is true,
   then the result is also a \linkingone{factor.misc}{FactoredInteger object} object. (default is False)\\

   \subsubsection{copy}\linkedtwo{factor.misc}{FactoredInteger}{copy}
   \func{copy}{\param{self}}{\out{\linkingone{factor.misc}{FactoredInteger object}}}\\
   \spacing
   \quad Return a copy of the object.\\

%---------- end document ---------- %

\bibliographystyle{jplain}%use jbibtex
\bibliography{nzmath_references}

\end{document}

