\documentclass{report}

%%%%%%%%%%%%%%%%%%%%%%%%%%%%%%%%%%%%%%%%%%%%%%%%%%%%%%%%%%%%%
%
% macros for nzmath manual
%
%%%%%%%%%%%%%%%%%%%%%%%%%%%%%%%%%%%%%%%%%%%%%%%%%%%%%%%%%%%%%
\usepackage{amssymb,amsmath}
\usepackage{color}
\usepackage[dvipdfm,bookmarks=true,bookmarksnumbered=true,%
 pdftitle={NZMATH Users Manual},%
 pdfsubject={Manual for NZMATH Users},%
 pdfauthor={NZMATH Development Group},%
 pdfkeywords={TeX; dvipdfmx; hyperref; color;},%
 colorlinks=true]{hyperref}
\usepackage{fancybox}
\usepackage[T1]{fontenc}
%
\newcommand{\DS}{\displaystyle}
\newcommand{\C}{\clearpage}
\newcommand{\NO}{\noindent}
\newcommand{\negok}{$\dagger$}
\newcommand{\spacing}{\vspace{1pt}\\ }
% software macros
\newcommand{\nzmathzero}{{\footnotesize $\mathbb{N}\mathbb{Z}$}\texttt{MATH}}
\newcommand{\nzmath}{{\nzmathzero}\ }
\newcommand{\pythonzero}{$\mbox{\texttt{Python}}$}
\newcommand{\python}{{\pythonzero}\ }
% link macros
\newcommand{\linkingzero}[1]{{\bf \hyperlink{#1}{#1}}}%module
\newcommand{\linkingone}[2]{{\bf \hyperlink{#1.#2}{#2}}}%module,class/function etc.
\newcommand{\linkingtwo}[3]{{\bf \hyperlink{#1.#2.#3}{#3}}}%module,class,method
\newcommand{\linkedzero}[1]{\hypertarget{#1}{}}
\newcommand{\linkedone}[2]{\hypertarget{#1.#2}{}}
\newcommand{\linkedtwo}[3]{\hypertarget{#1.#2.#3}{}}
\newcommand{\linktutorial}[1]{\href{http://docs.python.org/tutorial/#1}{#1}}
\newcommand{\linktutorialone}[2]{\href{http://docs.python.org/tutorial/#1}{#2}}
\newcommand{\linklibrary}[1]{\href{http://docs.python.org/library/#1}{#1}}
\newcommand{\linklibraryone}[2]{\href{http://docs.python.org/library/#1}{#2}}
\newcommand{\pythonhp}{\href{http://www.python.org/}{\python website}}
\newcommand{\nzmathwiki}{\href{http://nzmath.sourceforge.net/wiki/}{{\nzmathzero}Wiki}}
\newcommand{\nzmathsf}{\href{http://sourceforge.net/projects/nzmath/}{\nzmath Project Page}}
\newcommand{\nzmathtnt}{\href{http://tnt.math.se.tmu.ac.jp/nzmath/}{\nzmath Project Official Page}}
% parameter name
\newcommand{\param}[1]{{\tt #1}}
% function macros
\newcommand{\hiki}[2]{{\tt #1}:\ {\em #2}}
\newcommand{\hikiopt}[3]{{\tt #1}:\ {\em #2}=#3}

\newdimen\hoge
\newdimen\truetextwidth
\newcommand{\func}[3]{%
\setbox0\hbox{#1(#2)}
\hoge=\wd0
\truetextwidth=\textwidth
\advance \truetextwidth by -2\oddsidemargin
\ifdim\hoge<\truetextwidth % short form
{\bf \colorbox{skyyellow}{#1(#2)\ $\to$ #3}}
%
\else % long form
\fcolorbox{skyyellow}{skyyellow}{%
   \begin{minipage}{\textwidth}%
   {\bf #1(#2)\\ %
    \qquad\quad   $\to$\ #3}%
   \end{minipage}%
   }%
\fi%
}

\newcommand{\out}[1]{{\em #1}}
\newcommand{\initialize}{%
  \paragraph{\large \colorbox{skyblue}{Initialize (Constructor)}}%
    \quad\\ %
    \vspace{3pt}\\
}
\newcommand{\method}{\C \paragraph{\large \colorbox{skyblue}{Methods}}}
% Attribute environment
\newenvironment{at}
{%begin
\paragraph{\large \colorbox{skyblue}{Attribute}}
\quad\\
\begin{description}
}%
{%end
\end{description}
}
% Operation environment
\newenvironment{op}
{%begin
\paragraph{\large \colorbox{skyblue}{Operations}}
\quad\\
\begin{table}[h]
\begin{center}
\begin{tabular}{|l|l|}
\hline
operator & explanation\\
\hline
}%
{%end
\hline
\end{tabular}
\end{center}
\end{table}
}
% Examples environment
\newenvironment{ex}%
{%begin
\paragraph{\large \colorbox{skyblue}{Examples}}
\VerbatimEnvironment
\renewcommand{\EveryVerbatim}{\fontencoding{OT1}\selectfont}
\begin{quote}
\begin{Verbatim}
}%
{%end
\end{Verbatim}
\end{quote}
}
%
\definecolor{skyblue}{cmyk}{0.2, 0, 0.1, 0}
\definecolor{skyyellow}{cmyk}{0.1, 0.1, 0.5, 0}
%
%\title{NZMATH User Manual\\ {\large{(for version 1.0)}}}
%\date{}
%\author{}
\begin{document}
%\maketitle
%
\setcounter{tocdepth}{3}
\setcounter{secnumdepth}{3}


\tableofcontents
\C

\chapter{Classes}


%---------- start document ---------- %
 \section{poly.ring -- polynomial rings}\linkedzero{poly.ring}
 \begin{itemize}
   \item {\bf Classes}
   \begin{itemize}
     \item \linkingone{poly.ring}{PolynomialRing}
     \item \linkingone{poly.ring}{RationalFunctionField}
     \item \linkingone{poly.ring}{PolynomialIdeal}
   \end{itemize}
 \end{itemize}

\C

 \subsection{PolynomialRing -- ring of polynomials}\linkedone{poly.ring}{PolynomialRing}
 A class for uni-/multivariate polynomial rings.
 A subclass of \linkingone{ring}{CommutativeRing}.

 \initialize
 \func{PolynomialRing}{%
   \hiki{coeffring}{CommutativeRing},\ %
   \hikiopt{number\_of\_variables}{integer}{1}}{\out{PolynomialRing}}\\
 \spacing
 \quad \param{coeffring} is the ring of coefficients.
 \param{number\_of\_variables} is the number of variables.
 If its value is greater than \(1\), the ring is for multivariate polynomials.
  \begin{at}
    \item[zero]\linkedtwo{poly.ring}{PolynomialRing}{zero}:\\ zero of the ring.
    \item[one]\linkedtwo{poly.ring}{PolynomialRing}{one}:\\ one of the ring.
  \end{at}
 \method
 \subsubsection{getInstance -- classmethod}\linkedtwo{poly.ring}{PolynomialRing}{getInstance}
 \func{getInstance}{%
   \hiki{coeffring}{CommutativeRing},\ %
   \hiki{number\_of\_variables}{integer}}{\out{PolynomialRing}}\\
 \spacing
 \quad return the instance of polynomial ring with coefficient ring
 \param{coeffring} and number of variables \param{number\_of\_variables}.

 \subsubsection{getCoefficientRing}\linkedtwo{poly.ring}{PolynomialRing}{getCoefficientRing}
 \func{getCoefficientRing}{}{CommutativeRing}

 \subsubsection{getQuotientField}\linkedtwo{poly.ring}{PolynomialRing}{getQuotientField}
 \func{getQuotientField}{}{Field}

 \subsubsection{issubring}\linkedtwo{poly.ring}{PolynomialRing}{issubring}
 \func{issubring}{\hiki{other}{Ring}}{\out{bool}}

 \subsubsection{issuperring}\linkedtwo{poly.ring}{PolynomialRing}{issuperring}
 \func{issuperring}{\hiki{other}{Ring}}{\out{bool}}

 \subsubsection{getCharacteristic}\linkedtwo{poly.ring}{PolynomialRing}{getCharacteristic}
 \func{getCharacteristic}{}{\out{integer}}

 \subsubsection{createElement}\linkedtwo{poly.ring}{PolynomialRing}{createElement}
 \func{createElement}{seed}{\out{polynomial}}\\
 \quad Return a polynomial. \param{seed} can be a polynomial,
 an element of coefficient ring, or any other data suited for the first
 argument of uni-/multi-variate polynomials.

 \subsubsection{gcd}\linkedtwo{poly.ring}{PolynomialRing}{gcd}
 \func{gcd}{a, b}{\out{polynomial}}\\
 \quad Return the greatest common divisor of given polynomials (if possible).
 The polynomials must be in the polynomial ring.
 If the coefficient ring is a field, the result is monic.

 \subsubsection{isdomain}\linkedtwo{poly.ring}{PolynomialRing}{isdomain}
 \subsubsection{iseuclidean}\linkedtwo{poly.ring}{PolynomialRing}{iseuclidean}
 \subsubsection{isnoetherian}\linkedtwo{poly.ring}{PolynomialRing}{isnoetherian}
 \subsubsection{ispid}\linkedtwo{poly.ring}{PolynomialRing}{ispid}
 \subsubsection{isufd}\linkedtwo{poly.ring}{PolynomialRing}{isufd}

 Inherited from \linkingone{ring}{CommutativeRing}.

 \subsection{RationalFunctionField -- field of rational functions}\linkedone{poly.ring}{RationalFunctionField}
  \initialize
  \func{RationalFunctionField}{%
    \hiki{field}{Field},\ 
    \hiki{number\_of\_variables}{integer}}{%
    \out{RationalFunctionField}}\\
  \spacing
  \quad A class for fields of rational functions. 
  It is a subclass of \linkingone{ring}{QuotientField}.\\
  \spacing
  \quad \param{field} is the field of coefficients, which should be a
  \linkingone{ring}{Field} object.
  \param{number\_of\_variables} is the number of variables.\\
  \spacing
  \begin{at}
    \item[zero]\linkedtwo{poly.ring}{RationalFunctionField}{zero}:\\ zero of the field.
    \item[one]\linkedtwo{poly.ring}{RationalFunctionField}{one}:\\ one of the field.
  \end{at}
%
  \method
  \subsubsection{getInstance -- classmethod}\linkedtwo{poly.ring}{RationalFunctionField}{getInstance}
  \func{getInstance}{%
    \hiki{coefffield}{Field},\ %
    \hiki{number\_of\_variables}{integer}}{\out{RationalFunctionField}}\\
  \spacing
  \quad return the instance of {\tt RationalFunctionField} with coefficient
  field \param{coefffield} and number of variables \param{number\_of\_variables}.

  \subsubsection{createElement}\linkedtwo{poly.ring}{RationalFunctionField}{createElement}
  \func{createElement}{*\hiki{seedarg}{list}, **\hiki{seedkwd}{dict}}{\out{RationalFunction}}\\

  \subsubsection{getQuotientField}\linkedtwo{poly.ring}{RationalFunctionField}{getQuotientField}
  \func{getQuotientField}{}{\out{Field}}

  \subsubsection{issubring}\linkedtwo{poly.ring}{RationalFunctionField}{issubring}
  \func{issubring}{\hiki{other}{Ring}}{\out{bool}}\\

  \subsubsection{issuperring}\linkedtwo{poly.ring}{RationalFunctionField}{issuperring}
  \func{issuperring}{\hiki{other}{Ring}}{\out{bool}}\\

  \subsubsection{unnest}\linkedtwo{poly.ring}{RationalFunctionField}{unnest}
  \func{unnest}{}{\out{RationalFunctionField}}\\
  \spacing
  \quad If self is a nested {\tt RationalFunctionField} i.e. its
  coefficient field is also a {\tt RationalFunctionField}, then the
  method returns one level unnested {\tt RationalFunctionField}.\\
  \quad For example:
\begin{ex}
>>> RationalFunctionField(RationalFunctionField(Q, 1), 1).unnest()
RationalFunctionField(Q, 2)
\end{ex}

  \subsubsection{gcd}\linkedtwo{poly.ring}{RationalFunctionField}{gcd}
  \func{gcd}{\hiki{a}{RationalFunction},\ \hiki{b}{RationalFunction}}{%
  \out{RationalFunction}}\\
  \spacing
  \quad Inherited from \linkingone{ring}{Field}.

 \subsubsection{isdomain}\linkedtwo{poly.ring}{RationalFunctionField}{isdomain}
 \subsubsection{iseuclidean}\linkedtwo{poly.ring}{RationalFunctionField}{iseuclidean}
 \subsubsection{isnoetherian}\linkedtwo{poly.ring}{RationalFunctionField}{isnoetherian}
 \subsubsection{ispid}\linkedtwo{poly.ring}{RationalFunctionField}{ispid}
 \subsubsection{isufd}\linkedtwo{poly.ring}{RationalFunctionField}{isufd}

 Inherited from \linkingone{ring}{CommutativeRing}.

  \subsection{PolynomialIdeal -- ideal of polynomial ring}\linkedone{poly.ring}{PolynomialIdeal}

  A subclass of \linkingone{ring}{Ideal} represents ideals of polynomial rings.
  
  \initialize
  \func{PolynomialIdeal}{%
    \hiki{generators}{list},\ %
    \hiki{polyring}{PolynomialRing}}{\out{PolynomialIdeal}}\\
  \spacing
  \quad Create an object represents an ideal in a polynomial ring
  \param{polyring} generated by \param{generators}.
  \begin{op}
    \verb/in/ & membership test\\
    \verb/==/ & same ideal?\\
    \verb/!=/ & different ideal?\\
    \verb/+/ & addition\\
    \verb/*/ & multiplication\\
  \end{op}
  \method
  \subsubsection{reduce}\linkedtwo{poly.ring}{PolynomialIdeal}{reduce}
  \func{reduce}{\hiki{element}{polynomial}}{\out{polynomial}}\\
  \spacing
  \quad Modulo \param{element} by the ideal.

  \subsubsection{issubset}\linkedtwo{poly.ring}{PolynomialIdeal}{issubset}
  \func{issubset}{\hiki{other}{set}}{\out{bool}}\\

  \subsubsection{issuperset}\linkedtwo{poly.ring}{PolynomialIdeal}{issuperset}
  \func{issuperset}{\hiki{other}{set}}{\out{bool}}\\

\C

%---------- end document ---------- %

\bibliographystyle{jplain}%use jbibtex
\bibliography{nzmath_references}

\end{document}

