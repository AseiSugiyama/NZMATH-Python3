\documentclass{report}

%%%%%%%%%%%%%%%%%%%%%%%%%%%%%%%%%%%%%%%%%%%%%%%%%%%%%%%%%%%%%
%
% macros for nzmath manual
%
%%%%%%%%%%%%%%%%%%%%%%%%%%%%%%%%%%%%%%%%%%%%%%%%%%%%%%%%%%%%%
\usepackage{amssymb,amsmath}
\usepackage{color}
\usepackage[dvipdfm,bookmarks=true,bookmarksnumbered=true,%
 pdftitle={NZMATH Users Manual},%
 pdfsubject={Manual for NZMATH Users},%
 pdfauthor={NZMATH Development Group},%
 pdfkeywords={TeX; dvipdfmx; hyperref; color;},%
 colorlinks=true]{hyperref}
\usepackage{fancybox}
\usepackage[T1]{fontenc}
%
\newcommand{\DS}{\displaystyle}
\newcommand{\C}{\clearpage}
\newcommand{\NO}{\noindent}
\newcommand{\negok}{$\dagger$}
\newcommand{\spacing}{\vspace{1pt}\\ }
% software macros
\newcommand{\nzmathzero}{{\footnotesize $\mathbb{N}\mathbb{Z}$}\texttt{MATH}}
\newcommand{\nzmath}{{\nzmathzero}\ }
\newcommand{\pythonzero}{$\mbox{\texttt{Python}}$}
\newcommand{\python}{{\pythonzero}\ }
% link macros
\newcommand{\linkingzero}[1]{{\bf \hyperlink{#1}{#1}}}%module
\newcommand{\linkingone}[2]{{\bf \hyperlink{#1.#2}{#2}}}%module,class/function etc.
\newcommand{\linkingtwo}[3]{{\bf \hyperlink{#1.#2.#3}{#3}}}%module,class,method
\newcommand{\linkedzero}[1]{\hypertarget{#1}{}}
\newcommand{\linkedone}[2]{\hypertarget{#1.#2}{}}
\newcommand{\linkedtwo}[3]{\hypertarget{#1.#2.#3}{}}
\newcommand{\linktutorial}[1]{\href{http://docs.python.org/tutorial/#1}{#1}}
\newcommand{\linktutorialone}[2]{\href{http://docs.python.org/tutorial/#1}{#2}}
\newcommand{\linklibrary}[1]{\href{http://docs.python.org/library/#1}{#1}}
\newcommand{\linklibraryone}[2]{\href{http://docs.python.org/library/#1}{#2}}
\newcommand{\pythonhp}{\href{http://www.python.org/}{\python website}}
\newcommand{\nzmathwiki}{\href{http://nzmath.sourceforge.net/wiki/}{{\nzmathzero}Wiki}}
\newcommand{\nzmathsf}{\href{http://sourceforge.net/projects/nzmath/}{\nzmath Project Page}}
\newcommand{\nzmathtnt}{\href{http://tnt.math.se.tmu.ac.jp/nzmath/}{\nzmath Project Official Page}}
% parameter name
\newcommand{\param}[1]{{\tt #1}}
% function macros
\newcommand{\hiki}[2]{{\tt #1}:\ {\em #2}}
\newcommand{\hikiopt}[3]{{\tt #1}:\ {\em #2}=#3}

\newdimen\hoge
\newdimen\truetextwidth
\newcommand{\func}[3]{%
\setbox0\hbox{#1(#2)}
\hoge=\wd0
\truetextwidth=\textwidth
\advance \truetextwidth by -2\oddsidemargin
\ifdim\hoge<\truetextwidth % short form
{\bf \colorbox{skyyellow}{#1(#2)\ $\to$ #3}}
%
\else % long form
\fcolorbox{skyyellow}{skyyellow}{%
   \begin{minipage}{\textwidth}%
   {\bf #1(#2)\\ %
    \qquad\quad   $\to$\ #3}%
   \end{minipage}%
   }%
\fi%
}

\newcommand{\out}[1]{{\em #1}}
\newcommand{\initialize}{%
  \paragraph{\large \colorbox{skyblue}{Initialize (Constructor)}}%
    \quad\\ %
    \vspace{3pt}\\
}
\newcommand{\method}{\C \paragraph{\large \colorbox{skyblue}{Methods}}}
% Attribute environment
\newenvironment{at}
{%begin
\paragraph{\large \colorbox{skyblue}{Attribute}}
\quad\\
\begin{description}
}%
{%end
\end{description}
}
% Operation environment
\newenvironment{op}
{%begin
\paragraph{\large \colorbox{skyblue}{Operations}}
\quad\\
\begin{table}[h]
\begin{center}
\begin{tabular}{|l|l|}
\hline
operator & explanation\\
\hline
}%
{%end
\hline
\end{tabular}
\end{center}
\end{table}
}
% Examples environment
\newenvironment{ex}%
{%begin
\paragraph{\large \colorbox{skyblue}{Examples}}
\VerbatimEnvironment
\renewcommand{\EveryVerbatim}{\fontencoding{OT1}\selectfont}
\begin{quote}
\begin{Verbatim}
}%
{%end
\end{Verbatim}
\end{quote}
}
%
\definecolor{skyblue}{cmyk}{0.2, 0, 0.1, 0}
\definecolor{skyyellow}{cmyk}{0.1, 0.1, 0.5, 0}
%
%\title{NZMATH User Manual\\ {\large{(for version 1.0)}}}
%\date{}
%\author{}
\begin{document}
%\maketitle
%
\setcounter{tocdepth}{3}
\setcounter{secnumdepth}{3}


\tableofcontents
\C

\chapter{Classes}


%---------- start document ---------- %
 \section{poly.multiutil -- utilities for multivariate polynomials}\linkedzero{poly.multiutil}
 \begin{itemize}
   \item {\bf Classes}
     \begin{itemize}
     \item \linkingone{poly.multiutil}{RingPolynomial}
     \item \linkingone{poly.multiutil}{DomainPolynomial}
     \item \linkingone{poly.multiutil}{UniqueFactorizationDomainPolynomial}
     \item OrderProvider
     \item NestProvider
     \item PseudoDivisionProvider
     \item GcdProvider
     \item RingElementProvider
     \end{itemize}
   \item {\bf Functions}
     \begin{itemize}
     \item \linkingone{poly.multiutil}{polynomial}
     \end{itemize}
 \end{itemize}

\C

 \subsection{RingPolynomial}\linkedone{poly.multiutil}{RingPolynomial}
 General polynomial with commutative ring coefficients.

 \initialize
   \func{RingPolynomial}{%
    \hiki{coefficients}{terminit},\ %
    **\hiki{keywords}{dict}}{\out{RingPolynomial}}\\
  \spacing
  % document of basic document
  \quad The \param{keywords} must include:
  \begin{description}
  \item[coeffring] a commutative ring ({\it CommutativeRing})
  \item[number\_of\_variables] the number of variables({\it integer})
  \item[order] term order ({\it TermOrder})
  \end{description}
  \quad This class inherits \linkingone{poly.multivar}{BasicPolynomial},
  \linkingone{poly.multiutil}{OrderProvider},
  \linkingone{poly.multiutil}{NestProvider} and
  \linkingone{poly.multiutil}{RingElementProvider}.
%
  \begin{at}
    \item[order]\linkedtwo{poly.multiutil}{RingPolynomial}{order}:\\
      term order.
  \end{at}
%
  \method
  \subsubsection{getRing}\linkedtwo{poly.multiutil}{RingPolynomial}{getRing}
  \func{getRing}{\param{self}}{\out{Ring}}\\
  \spacing
  \quad Return an object of a subclass of {\tt Ring},
  to which the polynomial belongs.\\
  (This method overrides the definition in RingElementProvider)

  \subsubsection{getCoefficientRing}\linkedtwo{poly.multiutil}{RingPolynomial}{getCoefficientRing}
  \func{getCoefficientRing}{\param{self}}{\out{Ring}}\\
  \spacing
  \quad Return an object of a subclass of {\tt Ring},
  to which the all coefficients belong.\\
  (This method overrides the definition in RingElementProvider)

  \subsubsection{leading\_variable}\linkedtwo{poly.multiutil}{RingPolynomial}{leading\_variable}
  \func{leading\_variable}{\param{self}}{\out{integer}}\\
  \spacing
  \quad Return the position of the leading variable
  (the leading term among all total degree one terms).\\
  The leading term varies with term orders, so does the result. The
  term order can be specified via the attribute {\tt order}.\\
  (This method is inherited from NestProvider)

  \subsubsection{nest}\linkedtwo{poly.multiutil}{RingPolynomial}{nest}
  \func{nest}{\param{self},\ \hiki{outer}{integer},\ \hiki{coeffring}{CommutativeRing}}{\out{polynomial}}\\
  \spacing
  \quad Nest the polynomial by extracting \param{outer} variable
  at the given position.\\
  (This method is inherited from NestProvider)

  \subsubsection{unnest}\linkedtwo{poly.multiutil}{RingPolynomial}{unnest}
  \func{nest}{\param{self},\ \hiki{q}{polynomial},\ \hiki{outer}{integer},\ \hiki{coeffring}{CommutativeRing}}{\out{polynomial}}\\
  \spacing
  \quad Unnest the nested polynomial \param{q} by inserting \param{outer}
  variable at the given position.\\
  (This method is inherited from NestProvider)

 \subsection{DomainPolynomial}\linkedone{poly.multiutil}{DomainPolynomial}
 Polynomial with domain coefficients.
 \initialize
   \func{DomainPolynomial}{%
    \hiki{coefficients}{terminit},\ %
    **\hiki{keywords}{dict}}{\out{DomainPolynomial}}\\
  \spacing
  % document of basic document
  \quad The \param{keywords} must include:
  \begin{description}
  \item[coeffring] a commutative ring ({\it CommutativeRing})
  \item[number\_of\_variables] the number of variables({\it integer})
  \item[order] term order ({\it TermOrder})
  \end{description}
  \quad This class inherits \linkingone{poly.multiutil}{RingPolynomial} and
  \linkingone{poly.multiutil}{PseudoDivisionProvider}.
%
  \begin{op}
    \verb+f / g+ & division (result is a rational function)\\
  \end{op}
  \method

  \subsubsection{pseudo\_divmod}\linkedtwo{poly.multiutil}{DomainPolynomial}{pseudo\_divmod}
  \func{pseudo\_divmod}{\param{self},\ \hiki{other}{polynomial}}{\out{polynomial}}\\
  \spacing
  \quad Return \(Q\), \(R\) polynomials such that:
  \[d^{deg(self) - deg(other) + 1} self = other \times Q + R\]
  w.r.t. a fixed variable, where \(d\) is the leading coefficient of 
  \param{other}.\\
  \quad The leading coefficient varies with term orders,
  so does the result. The term order can be specified via
  the attribute {\tt order}.\\
  (This method is inherited from PseudoDivisionProvider.)

  \subsubsection{pseudo\_floordiv}\linkedtwo{poly.multiutil}{DomainPolynomial}{pseudo\_floordiv}
  \func{pseudo\_floordiv}{\param{self},\ \hiki{other}{polynomial}}{\out{polynomial}}\\
  \spacing
  \quad Return a polynomial \(Q\) such that
  \[d^{deg(self) - deg(other) + 1} self = other \times Q + R\]
  w.r.t. a fixed variable, where \(d\) is the leading coefficient of 
  \param{other} and \(R\) is a polynomial.\\

  The leading coefficient varies with term orders, so does the result.
  The term order can be specified via the attribute {\tt order}.\\
  (This method is inherited from PseudoDivisionProvider.)

  \subsubsection{pseudo\_mod}\linkedtwo{poly.multiutil}{DomainPolynomial}{pseudo\_mod}
  \func{pseudo\_mod}{\param{self},\ \hiki{other}{polynomial}}{\out{polynomial}}\\
  \quad Return a polynomial \(R\) such that
  \[ d^{deg(self) - deg(other) + 1} \times self = other \times Q + R\]
  where \(d\) is the leading coefficient of \param{other} and \(Q\)
  a polynomial.\\
  \quad The leading coefficient varies with term orders,
  so does the result. The term order can be specified via
  the attribute {\tt order}.\\
  (This method is inherited from PseudoDivisionProvider.)

  \subsubsection{exact\_division}\linkedtwo{poly.multiutil}{DomainPolynomial}{exact\_division}
  \func{exact\_division}{\param{self},\ \hiki{other}{polynomial}}{\out{polynomial}}\\
  \spacing
  \quad Return quotient of exact division.\\
  (This method is inherited from PseudoDivisionProvider.)

 \subsection{UniqueFactorizationDomainPolynomial}\linkedone{poly.multiutil}{UniqueFactorizationDomainPolynomial}
 Polynomial with unique factorization domain (UFD) coefficients.
 \initialize
   \func{UniqueFactorizationDomainPolynomial}{%
    \hiki{coefficients}{terminit},\ %
    **\hiki{keywords}{dict}}{\out{UniqueFactorizationDomainPolynomial}}\\
  \spacing
  % document of basic document
  \quad The \param{keywords} must include:
  \begin{description}
  \item[coeffring] a commutative ring ({\it CommutativeRing})
  \item[number\_of\_variables] the number of variables({\it integer})
  \item[order] term order ({\it TermOrder})
  \end{description}
  \quad This class inherits \linkingone{poly.multiutil}{DomainPolynomial} and
  \linkingone{poly.multiutil}{GcdProvider}.
  \method
  \subsubsection{gcd}\linkedtwo{poly.multiutil}{UniqueFactorizationDomainPolynomial}{gcd}
  \func{gcd}{\param{self},\ \hiki{other}{polynomial}}{\out{polynomial}}\\
  \quad Return gcd. The nested polynomials' gcd is used.\\
  (This method is inherited from GcdProvider.)

  \subsubsection{resultant}\linkedtwo{poly.multiutil}{UniqueFactorizationDomainPolynomial}{resultant}
  \func{resultant}{\param{self},\ \hiki{other}{polynomial},\ \hiki{var}{integer}}{\out{polynomial}}\\
  \quad Return resultant of two polynomials of the same ring,
  with respect to the variable specified by its position \param{var}.

% Functions
 \subsection{polynomial -- factory function for various polynomials}\linkedone{poly.multiutil}{polynomial}
  \func{polynomial}{\hiki{coefficients}{terminit},\ \hiki{coeffring}{CommutativeRing}, \hikiopt{number\_of\_variables}{integer}{None}}{\out{polynomial}}\\
   \spacing
   % document of basic document
   \quad Return a polynomial.\\
   \spacing
   \quad \negok One can override the way to choose a polynomial type
   from a coefficient ring, by setting:\\
   {\tt special\_ring\_table[coeffring\_type] = polynomial\_type}\\
   before the function call.

\C

%---------- end document ---------- %

\bibliographystyle{jplain}%use jbibtex
\bibliography{nzmath_references}

\end{document}

