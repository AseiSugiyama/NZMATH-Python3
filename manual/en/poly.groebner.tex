\documentclass{report}

%%%%%%%%%%%%%%%%%%%%%%%%%%%%%%%%%%%%%%%%%%%%%%%%%%%%%%%%%%%%%
%
% macros for nzmath manual
%
%%%%%%%%%%%%%%%%%%%%%%%%%%%%%%%%%%%%%%%%%%%%%%%%%%%%%%%%%%%%%
\usepackage{amssymb,amsmath}
\usepackage{color}
\usepackage[dvipdfm,bookmarks=true,bookmarksnumbered=true,%
 pdftitle={NZMATH Users Manual},%
 pdfsubject={Manual for NZMATH Users},%
 pdfauthor={NZMATH Development Group},%
 pdfkeywords={TeX; dvipdfmx; hyperref; color;},%
 colorlinks=true]{hyperref}
\usepackage{fancybox}
\usepackage[T1]{fontenc}
%
\newcommand{\DS}{\displaystyle}
\newcommand{\C}{\clearpage}
\newcommand{\NO}{\noindent}
\newcommand{\negok}{$\dagger$}
\newcommand{\spacing}{\vspace{1pt}\\ }
% software macros
\newcommand{\nzmathzero}{{\footnotesize $\mathbb{N}\mathbb{Z}$}\texttt{MATH}}
\newcommand{\nzmath}{{\nzmathzero}\ }
\newcommand{\pythonzero}{$\mbox{\texttt{Python}}$}
\newcommand{\python}{{\pythonzero}\ }
% link macros
\newcommand{\linkingzero}[1]{{\bf \hyperlink{#1}{#1}}}%module
\newcommand{\linkingone}[2]{{\bf \hyperlink{#1.#2}{#2}}}%module,class/function etc.
\newcommand{\linkingtwo}[3]{{\bf \hyperlink{#1.#2.#3}{#3}}}%module,class,method
\newcommand{\linkedzero}[1]{\hypertarget{#1}{}}
\newcommand{\linkedone}[2]{\hypertarget{#1.#2}{}}
\newcommand{\linkedtwo}[3]{\hypertarget{#1.#2.#3}{}}
\newcommand{\linktutorial}[1]{\href{http://docs.python.org/tutorial/#1}{#1}}
\newcommand{\linktutorialone}[2]{\href{http://docs.python.org/tutorial/#1}{#2}}
\newcommand{\linklibrary}[1]{\href{http://docs.python.org/library/#1}{#1}}
\newcommand{\linklibraryone}[2]{\href{http://docs.python.org/library/#1}{#2}}
\newcommand{\pythonhp}{\href{http://www.python.org/}{\python website}}
\newcommand{\nzmathwiki}{\href{http://nzmath.sourceforge.net/wiki/}{{\nzmathzero}Wiki}}
\newcommand{\nzmathsf}{\href{http://sourceforge.net/projects/nzmath/}{\nzmath Project Page}}
\newcommand{\nzmathtnt}{\href{http://tnt.math.se.tmu.ac.jp/nzmath/}{\nzmath Project Official Page}}
% parameter name
\newcommand{\param}[1]{{\tt #1}}
% function macros
\newcommand{\hiki}[2]{{\tt #1}:\ {\em #2}}
\newcommand{\hikiopt}[3]{{\tt #1}:\ {\em #2}=#3}

\newdimen\hoge
\newdimen\truetextwidth
\newcommand{\func}[3]{%
\setbox0\hbox{#1(#2)}
\hoge=\wd0
\truetextwidth=\textwidth
\advance \truetextwidth by -2\oddsidemargin
\ifdim\hoge<\truetextwidth % short form
{\bf \colorbox{skyyellow}{#1(#2)\ $\to$ #3}}
%
\else % long form
\fcolorbox{skyyellow}{skyyellow}{%
   \begin{minipage}{\textwidth}%
   {\bf #1(#2)\\ %
    \qquad\quad   $\to$\ #3}%
   \end{minipage}%
   }%
\fi%
}

\newcommand{\out}[1]{{\em #1}}
\newcommand{\initialize}{%
  \paragraph{\large \colorbox{skyblue}{Initialize (Constructor)}}%
    \quad\\ %
    \vspace{3pt}\\
}
\newcommand{\method}{\C \paragraph{\large \colorbox{skyblue}{Methods}}}
% Attribute environment
\newenvironment{at}
{%begin
\paragraph{\large \colorbox{skyblue}{Attribute}}
\quad\\
\begin{description}
}%
{%end
\end{description}
}
% Operation environment
\newenvironment{op}
{%begin
\paragraph{\large \colorbox{skyblue}{Operations}}
\quad\\
\begin{table}[h]
\begin{center}
\begin{tabular}{|l|l|}
\hline
operator & explanation\\
\hline
}%
{%end
\hline
\end{tabular}
\end{center}
\end{table}
}
% Examples environment
\newenvironment{ex}%
{%begin
\paragraph{\large \colorbox{skyblue}{Examples}}
\VerbatimEnvironment
\renewcommand{\EveryVerbatim}{\fontencoding{OT1}\selectfont}
\begin{quote}
\begin{Verbatim}
}%
{%end
\end{Verbatim}
\end{quote}
}
%
\definecolor{skyblue}{cmyk}{0.2, 0, 0.1, 0}
\definecolor{skyyellow}{cmyk}{0.1, 0.1, 0.5, 0}
%
%\title{NZMATH User Manual\\ {\large{(for version 1.0)}}}
%\date{}
%\author{}
\begin{document}
%\maketitle
%
\setcounter{tocdepth}{3}
\setcounter{secnumdepth}{3}


\tableofcontents
\C

\chapter{Functions}



%---------- start document ---------- %
 \section{poly.groebner -- Gr\"obner Basis}\linkedzero{poly.groebner}
 The groebner module is for computing Gr\"obner bases for multivariate polynomial ideals.


 This module uses the following types:
 \begin{description}
   \item[polynomial]\linkedone{poly.groebner}{polynomial}:\\
     \param{polynomial} is the polynomial generated by function \linkingone{poly.multiutil}{polynomial}. 
   \item[order]\linkedone{poly.groebner}{order}:\\
     \param{order} is the order on terms of polynomials. 
 \end{description}

%
  \subsection{buchberger -- na\"ive algorithm for obtaining Gr\"obner basis}\linkedone{poly.groebner}{buchberger}
   \func{buchberger}
   {%
     \hiki{generating}{list},\ %
     \hiki{order}{order}%
   }{%
     \out{[polynomials]}%
   }\\
   \spacing
   % document of basic document
   Return a Gr\"obner basis of the ideal generated by given generating
   set of polynomials with respect to the \param{order}.\\
   \spacing
   % added document
   Be careful, this implementation is very naive.\\
   \spacing
   % input, output document
   The argument \param{generating} is a list of \linkingone{poly.multiutil}{Polynomial};
   the argument \param{order} is an order.
   %\quad \\
%
  \subsection{normal\_strategy -- normal algorithm for obtaining Gr\"obner basis}\linkedone{poly.groebner}{normal\_strategy}
   \func{normal\_strategy}
   {%
     \hiki{generating}{list},\ %
     \hiki{order}{order}%
   }
   {%
     \out{[polynomials]}}\\
   \spacing
   % document of basic document
   Return a Gr\"obner basis of the ideal generated by given generating
   set of polynomials with respect to the \param{order}.\\
   \spacing
   % added document
    This function uses the `normal strategy'. \\
   \spacing
   % input, output document
   The argument \param{generating} is a list of \linkingone{poly.multiutil}{Polynomial};
   the argument \param{order} is an order.
   \\
%
  \subsection{reduce\_groebner -- reduce Gr\"obner basis}\linkedone{poly.groebner}{reduce\_groebner}
   \func{reduce\_groebner}
   {%
     \hiki{gbasis}{list},\ 
     \hiki{order}{order}
   }
   {\out{[polynomials]}}\\
   \spacing
   % document of basic document
   Return the reduced Gr\"obner basis constructed from a Gr\"obner basis.\\
   \spacing
   % added document
   The output satisfies that:
   \begin{itemize}
   \item \(\operatorname{lb}(f)\) divides \(\operatorname{lb}(g)\) 
     \(\Rightarrow\) \(g\) is not in reduced Gr\"obner basis, and
   \item monic.
   \end{itemize}
   %\spacing
   % input, output document
   The argument \param{gbasis} is a list of polynomials, a Gr\"obner basis (not merely a generating set).\\
%
  \subsection{s\_polynomial -- S-polynomial}\linkedone{poly.groebner}{s\_polynomial}
   \func{s\_polynomial}
   {%
     \hiki{f}{polynomial},\ 
     \hiki{g}{polynomial},
     \hiki{order}{order}}
   {\out{[polynomials]}}\\
   \spacing
   % document of basic document
   Return S-polynomial of \param{f} and \param{g}
   with respect to the \param{order}.\\
   \spacing
   % added document
   \[S(f, g) = (\operatorname{lc}(g)*T/\operatorname{lb}(f))*f - (\operatorname{lc}(f)*T/\operatorname{lb}(g))*g,\]
    where \(T = \operatorname{lcm}(\operatorname{lb}(f),\ \operatorname{lb}(g))\).
   %\spacing
   % input, output document
%
%
\begin{ex}
>>> f = multiutil.polynomial({(1,0):2, (1,1):1},rational.theRationalField, 2)
>>> g = multiutil.polynomial({(0,1):-2, (1,1):1},rational.theRationalField, 2)
>>> lex = termorder.lexicographic_order
>>> groebner.s_polynomial(f, g, lex)
UniqueFactorizationDomainPolynomial({(1, 0): 2, (0, 1): 2})
>>> gb = groebner.normal_strategy([f, g], lex)
>>> for gb_poly in gb:
...     print gb_poly
...
UniqueFactorizationDomainPolynomial({(1, 1): 1, (1, 0): 2})
UniqueFactorizationDomainPolynomial({(1, 1): 1, (0, 1): -2})
UniqueFactorizationDomainPolynomial({(1, 0): 2, (0, 1): 2})
UniqueFactorizationDomainPolynomial({(0, 2): -2, (0, 1): -4.0})
>>> gb_red = groebner.reduce_groebner(gb, lex)
>>> for gb_poly in gb_red:
...     print gb_poly
...
UniqueFactorizationDomainPolynomial({(1, 0): Rational(1, 1), (0, 1): Rational(1, 1)})
UniqueFactorizationDomainPolynomial({(0, 2): Rational(1, 1), (0, 1): 2.0})
\end{ex}

%---------- end document ---------- %

\bibliographystyle{jplain}%use jbibtex
\bibliography{nzmath_references}

\end{document}

