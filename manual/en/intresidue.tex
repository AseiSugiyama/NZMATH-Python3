\documentclass{report}

%%%%%%%%%%%%%%%%%%%%%%%%%%%%%%%%%%%%%%%%%%%%%%%%%%%%%%%%%%%%%
%
% macros for nzmath manual
%
%%%%%%%%%%%%%%%%%%%%%%%%%%%%%%%%%%%%%%%%%%%%%%%%%%%%%%%%%%%%%
\usepackage{amssymb,amsmath}
\usepackage{color}
\usepackage[dvipdfm,bookmarks=true,bookmarksnumbered=true,%
 pdftitle={NZMATH Users Manual},%
 pdfsubject={Manual for NZMATH Users},%
 pdfauthor={NZMATH Development Group},%
 pdfkeywords={TeX; dvipdfmx; hyperref; color;},%
 colorlinks=true]{hyperref}
\usepackage{fancybox}
\usepackage[T1]{fontenc}
%
\newcommand{\DS}{\displaystyle}
\newcommand{\C}{\clearpage}
\newcommand{\NO}{\noindent}
\newcommand{\negok}{$\dagger$}
\newcommand{\spacing}{\vspace{1pt}\\ }
% software macros
\newcommand{\nzmathzero}{{\footnotesize $\mathbb{N}\mathbb{Z}$}\texttt{MATH}}
\newcommand{\nzmath}{{\nzmathzero}\ }
\newcommand{\pythonzero}{$\mbox{\texttt{Python}}$}
\newcommand{\python}{{\pythonzero}\ }
% link macros
\newcommand{\linkingzero}[1]{{\bf \hyperlink{#1}{#1}}}%module
\newcommand{\linkingone}[2]{{\bf \hyperlink{#1.#2}{#2}}}%module,class/function etc.
\newcommand{\linkingtwo}[3]{{\bf \hyperlink{#1.#2.#3}{#3}}}%module,class,method
\newcommand{\linkedzero}[1]{\hypertarget{#1}{}}
\newcommand{\linkedone}[2]{\hypertarget{#1.#2}{}}
\newcommand{\linkedtwo}[3]{\hypertarget{#1.#2.#3}{}}
\newcommand{\linktutorial}[1]{\href{http://docs.python.org/tutorial/#1}{#1}}
\newcommand{\linktutorialone}[2]{\href{http://docs.python.org/tutorial/#1}{#2}}
\newcommand{\linklibrary}[1]{\href{http://docs.python.org/library/#1}{#1}}
\newcommand{\linklibraryone}[2]{\href{http://docs.python.org/library/#1}{#2}}
\newcommand{\pythonhp}{\href{http://www.python.org/}{\python website}}
\newcommand{\nzmathwiki}{\href{http://nzmath.sourceforge.net/wiki/}{{\nzmathzero}Wiki}}
\newcommand{\nzmathsf}{\href{http://sourceforge.net/projects/nzmath/}{\nzmath Project Page}}
\newcommand{\nzmathtnt}{\href{http://tnt.math.se.tmu.ac.jp/nzmath/}{\nzmath Project Official Page}}
% parameter name
\newcommand{\param}[1]{{\tt #1}}
% function macros
\newcommand{\hiki}[2]{{\tt #1}:\ {\em #2}}
\newcommand{\hikiopt}[3]{{\tt #1}:\ {\em #2}=#3}

\newdimen\hoge
\newdimen\truetextwidth
\newcommand{\func}[3]{%
\setbox0\hbox{#1(#2)}
\hoge=\wd0
\truetextwidth=\textwidth
\advance \truetextwidth by -2\oddsidemargin
\ifdim\hoge<\truetextwidth % short form
{\bf \colorbox{skyyellow}{#1(#2)\ $\to$ #3}}
%
\else % long form
\fcolorbox{skyyellow}{skyyellow}{%
   \begin{minipage}{\textwidth}%
   {\bf #1(#2)\\ %
    \qquad\quad   $\to$\ #3}%
   \end{minipage}%
   }%
\fi%
}

\newcommand{\out}[1]{{\em #1}}
\newcommand{\initialize}{%
  \paragraph{\large \colorbox{skyblue}{Initialize (Constructor)}}%
    \quad\\ %
    \vspace{3pt}\\
}
\newcommand{\method}{\C \paragraph{\large \colorbox{skyblue}{Methods}}}
% Attribute environment
\newenvironment{at}
{%begin
\paragraph{\large \colorbox{skyblue}{Attribute}}
\quad\\
\begin{description}
}%
{%end
\end{description}
}
% Operation environment
\newenvironment{op}
{%begin
\paragraph{\large \colorbox{skyblue}{Operations}}
\quad\\
\begin{table}[h]
\begin{center}
\begin{tabular}{|l|l|}
\hline
operator & explanation\\
\hline
}%
{%end
\hline
\end{tabular}
\end{center}
\end{table}
}
% Examples environment
\newenvironment{ex}%
{%begin
\paragraph{\large \colorbox{skyblue}{Examples}}
\VerbatimEnvironment
\renewcommand{\EveryVerbatim}{\fontencoding{OT1}\selectfont}
\begin{quote}
\begin{Verbatim}
}%
{%end
\end{Verbatim}
\end{quote}
}
%
\definecolor{skyblue}{cmyk}{0.2, 0, 0.1, 0}
\definecolor{skyyellow}{cmyk}{0.1, 0.1, 0.5, 0}
%
%\title{NZMATH User Manual\\ {\large{(for version 1.0)}}}
%\date{}
%\author{}
\begin{document}
%\maketitle
%
\setcounter{tocdepth}{3}
\setcounter{secnumdepth}{3}


\tableofcontents
\C

\chapter{Classes}


%---------- start document ---------- %
 \section{intresidue -- integer residue}\linkedzero{intresidue}
intresidue module provides integer residue classes or $\mathbf{Z}/m\mathbf{Z}$.

 \begin{itemize}
   \item {\bf Classes}
   \begin{itemize}
     \item \linkingone{intresidue}{IntegerResidueClass}
     \item \linkingone{intresidue}{IntegerResidueClassRing}
   \end{itemize}
   %\item {\bf Functions}
   %  \begin{itemize}
   %    \item \linkingone{rational}{innerProduct}
   %  \end{itemize}
 \end{itemize}

\C

 \subsection{IntegerResidueClass -- integer residue class}\linkedone{intresidue}{IntegerResidueClass}
 
 This class is a subclass of \linkingone{ring}{CommutativeRingElement}.

  \initialize
  \func{IntegerResidueClass}
       {\hiki{representative}{integer},\ 
       \hiki{modulus}{integer}}
       {\out{Integer}}\\
  \spacing
  % document of basic document
  \quad Create a residue class of modulus with residue representative.
  % added document
  \spacing
  % input, output document
  \param{modulus} must be positive integer.
  %\begin{at}
  %  \item[compo]\linkedtwo{vector}{Vector}{compo}:\\ It expresses component of Vector.
  %\end{at}
  \begin{op}
    \verb|a+b| & addition.\\
    \verb|a-b| & subtraction.\\
    \verb|a*b| & multiplication.\\
    \verb|a/b| & division.\\
    \verb|a**i,pow(a,i)| & power.\\
    \verb|-a| & negation.\\
    \verb|+a| & make a copy.\\
    \verb|a==b| & equality or not.\\
    \verb|a!=b| & inequality or not.\\
    \verb|repr(a)| & return representation string.\\
    \verb|str(a)| & return string.\\
  \end{op} 
%\begin{ex}
%>>> A = vector.Vector([1,2])
%>>> A
%Vector([1, 2])
%>>> A.compo
%[1, 2]
%>>>
%\end{ex}%Don't indent!
  \method
  \subsubsection{getRing -- get ring object}\linkedtwo{intresidue}{IntegerResidueClassRing}{getRing}
   \func{getRing}{\param{self}}{\out{IntegerResidueClassRing}}\\
   \spacing
   % document of basic document
   \quad Return a ring to which it belongs.
   %\spacing
   % added document
   %\quad \negok Note that this function returns integer only.\\
   %\spacing
   % input, output document
   %\quad \param{a} must be int, long or rational.Integer.\\
%
  \subsubsection{getResidue -- get residue}\linkedtwo{intresidue}{IntegerResidueClassRing}{getResidue}
   \func{getResidue}{\param{self}}{\out{integer}}\\
   \spacing
   % document of basic document
   \quad Return the value of residue.
   %\spacing
   % added document
   %\quad \negok Note that this function returns integer only.\\
   %\spacing
   % input, output document
   %\quad \param{a} must be int, long or rational.Integer.\\
%
  \subsubsection{getModulus -- get modulus}\linkedtwo{intresidue}{IntegerResidueClassRing}{getModulus}
   \func{getModulus}{\param{self}}{\out{integer}}\\
   \spacing
   % document of basic document
   \quad Return the value of modulus.
   %\spacing
   % added document
   %\quad \negok Note that this function returns integer only.\\
   %\spacing
   % input, output document
   %\quad \param{a} must be int, long or rational.Integer.\\
%
  \subsubsection{inverse -- inverse element}\linkedtwo{intresidue}{IntegerResidueClassRing}{inverse}
   \func{inverse}{\param{self}}{\out{IntegerResidueClass}}\\
   \spacing
   % document of basic document
   \quad Return the inverse element if it is invertible. Otherwise raise ValueError.
   %\spacing
   % added document
   %\quad \negok Note that this function returns integer only.\\
   %\spacing
   % input, output document
   %\quad \param{a} must be int, long or rational.Integer.\\
%
  \subsubsection{minimumAbsolute -- minimum absolute representative}\linkedtwo{intresidue}{IntegerResidueClassRing}{minimumAbsolute}
   \func{minimumAbsolute}{\param{self}}{\out{\linkingone{rational}{Integer}}}\\
   \spacing
   % document of basic document
   \quad  Return the minimum absolute representative integer of the residue class.
   %\spacing
   % added document
   %\quad \negok Note that this function returns integer only.\\
   %\spacing
   % input, output document
   %\quad \param{a} must be int, long or rational.Integer.\\
%
  \subsubsection{minimumNonNegative -- smallest non-negative representative}\linkedtwo{intresidue}{IntegerResidueClassRing}{minimumNonNegative}
   \func{minimumNonNegative}{\param{self}}{\out{\linkingone{rational}{Integer}}}\\
   \spacing
   % document of basic document
   \quad Return the smallest non-negative representative element of the residue class.
   %\spacing
   % added document
   \quad \negok this method has an alias, named toInteger.\\
   %\spacing
   % input, output document
   %\quad \param{a} must be int, long or rational.Integer.\\
%
%\begin{ex}
%>>> A = module.HogeClass((1,2))
%>>> A.hogemethod1(2)
%(2, 4)
%>>>
%\end{ex}%Don't indent!
\C
 \subsection{IntegerResidueClassRing -- ring of integer residue}\linkedone{intresidue}{IntegerResidueClassRing}
 The class is for rings of integer residue classes.

 This class is a subclass of \linkingone{ring}{CommutativeRing}.


  \initialize
  \func{IntegerResidueClassRing}{\hiki{modulus}{integer}}{\out{IntegerResidueClassRing}}\\
  \spacing
  % document of basic document
  \quad Create an instance of IntegerResidueClassRing. 
  % added document
  The argument \param{modulus} = $m$ specifies an ideal $m\mathbb{Z}$.
  % \spacing
  % input, output document
  %See \linkingone{module}{point} for \param{point}.
  \begin{at}
    \item[zero]\linkedtwo{integer}{IntegerRing}{zero}:\\ It expresses The additive unit 0. (read only)
    \item[one]\linkedtwo{integer}{IntegerRing}{one}:\\ It expresses The multiplicative unit 1. (read only)
  \end{at}
  \begin{op}
  %  \verb|+| & Vector sum.\\
  %  \verb|-| & Vector subtraction.\\
  %  \verb|*| & Scalar multiplication.\\
  %  \verb|//| & Scalar division.\\
  %  \verb|-(unary)| & element negation.\\
    \verb|R==A| & ring equality.\\
  %  \verb|!=| & inequality or not.\\
  %  \verb+V[i]+ & Return the coefficient of i-th element of Vector.\\
  %  \verb+V[i] = c+ & Replace the coefficient of i-th element of Vector by c.\\
    \verb|card(R)| & return cardinality. See also \linkingzero{compatibility} module.\\
    \verb|e in R| & return whether an element is in or not.\\
    \verb|repr(R)| & return representation string.\\
    \verb|str(R)| & return string.\\
  \end{op} 
%\begin{ex}
%>>> A = vector.Vector([1,2])
%>>> A
%Vector([1, 2])
%>>> A.compo
%[1, 2]
%>>>
%\end{ex}%Don't indent!
  \method
  \subsubsection{createElement -- create IntegerResidueClass object}\linkedtwo{intresidue}{IntegerResidueClassRing}{createElement}
   \func{createElement}{\param{self},\ \hiki{seed}{integer}}{\out{Integer}}\\
   \spacing
   % document of basic document
   \quad Return an IntegerResidueClass instance with \param{seed}. 
   %\spacing
   % added document
   %\quad \negok Note that this function returns integer only.\\
   %\spacing
   % input, output document
   %\quad \\
%
  \subsubsection{isfield -- field test}\linkedtwo{intresidue}{IntegerResidueClassRing}{isfield}
   \func{isfield}{\param{self}}{\out{bool}}\\
   \spacing
   % document of basic document
   \quad Return True if the modulus is prime, False if not. Since a finite domain is a field, other ring property tests are merely aliases of isfield; they are isdomain, iseuclidean, isnoetherian, ispid, isufd.
   % added document
   %\quad \negok Note that this function returns integer only.\\
   %\spacing
   % input, output document
   %\quad if \param{as\_column} is True, try to create column matrix.\\
%
  \subsubsection{getInstance -- get instance of IntegerResidueClassRing}\linkedtwo{intresidue}{IntegerResidueClassRing}{getInstance}
   \func{getInstance}{\param{cls},\ \hiki{modulus}{integer}}{\out{IntegerResidueClass}}\\
   \spacing
   % document of basic document
   \quad Return an instance of the class of specified modulus. Since this is a class method, use it as:

\verb|IntegerResidueClassRing.getInstance(3)|

to create a $\mathbb{Z}/3\mathbb{Z}$ object, for example.
%\begin{ex}
%>>> A = module.HogeClass((1,2))
%>>> A.hogemethod1(2)
%(2, 4)
%>>>
%\end{ex}%Don't indent!

\C

%---------- end document ---------- %

\bibliographystyle{jplain}%use jbibtex
\bibliography{nzmath_references}

\end{document}

