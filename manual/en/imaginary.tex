%%%%%%%%%%%%%%%%%%%%%%%%%%%%%%%%%%%%%%%%%%%%%%%%%%%%%%%%%%%%%
%
% macros for nzmath manual
%
%%%%%%%%%%%%%%%%%%%%%%%%%%%%%%%%%%%%%%%%%%%%%%%%%%%%%%%%%%%%%
\usepackage{amssymb,amsmath}
\usepackage{color}
\usepackage[dvipdfm,bookmarks=true,bookmarksnumbered=true,%
 pdftitle={NZMATH Users Manual},%
 pdfsubject={Manual for NZMATH Users},%
 pdfauthor={NZMATH Development Group},%
 pdfkeywords={TeX; dvipdfmx; hyperref; color;},%
 colorlinks=true]{hyperref}
\usepackage{fancybox}
\usepackage[T1]{fontenc}
%
\newcommand{\DS}{\displaystyle}
\newcommand{\C}{\clearpage}
\newcommand{\NO}{\noindent}
\newcommand{\negok}{$\dagger$}
\newcommand{\spacing}{\vspace{1pt}\\ }
% software macros
\newcommand{\nzmathzero}{{\footnotesize $\mathbb{N}\mathbb{Z}$}\texttt{MATH}}
\newcommand{\nzmath}{{\nzmathzero}\ }
\newcommand{\pythonzero}{$\mbox{\texttt{Python}}$}
\newcommand{\python}{{\pythonzero}\ }
% link macros
\newcommand{\linkingzero}[1]{{\bf \hyperlink{#1}{#1}}}%module
\newcommand{\linkingone}[2]{{\bf \hyperlink{#1.#2}{#2}}}%module,class/function etc.
\newcommand{\linkingtwo}[3]{{\bf \hyperlink{#1.#2.#3}{#3}}}%module,class,method
\newcommand{\linkedzero}[1]{\hypertarget{#1}{}}
\newcommand{\linkedone}[2]{\hypertarget{#1.#2}{}}
\newcommand{\linkedtwo}[3]{\hypertarget{#1.#2.#3}{}}
\newcommand{\linktutorial}[1]{\href{http://docs.python.org/tutorial/#1}{#1}}
\newcommand{\linktutorialone}[2]{\href{http://docs.python.org/tutorial/#1}{#2}}
\newcommand{\linklibrary}[1]{\href{http://docs.python.org/library/#1}{#1}}
\newcommand{\linklibraryone}[2]{\href{http://docs.python.org/library/#1}{#2}}
\newcommand{\pythonhp}{\href{http://www.python.org/}{\python website}}
\newcommand{\nzmathwiki}{\href{http://nzmath.sourceforge.net/wiki/}{{\nzmathzero}Wiki}}
\newcommand{\nzmathsf}{\href{http://sourceforge.net/projects/nzmath/}{\nzmath Project Page}}
\newcommand{\nzmathtnt}{\href{http://tnt.math.se.tmu.ac.jp/nzmath/}{\nzmath Project Official Page}}
% parameter name
\newcommand{\param}[1]{{\tt #1}}
% function macros
\newcommand{\hiki}[2]{{\tt #1}:\ {\em #2}}
\newcommand{\hikiopt}[3]{{\tt #1}:\ {\em #2}=#3}

\newdimen\hoge
\newdimen\truetextwidth
\newcommand{\func}[3]{%
\setbox0\hbox{#1(#2)}
\hoge=\wd0
\truetextwidth=\textwidth
\advance \truetextwidth by -2\oddsidemargin
\ifdim\hoge<\truetextwidth % short form
{\bf \colorbox{skyyellow}{#1(#2)\ $\to$ #3}}
%
\else % long form
\fcolorbox{skyyellow}{skyyellow}{%
   \begin{minipage}{\textwidth}%
   {\bf #1(#2)\\ %
    \qquad\quad   $\to$\ #3}%
   \end{minipage}%
   }%
\fi%
}

\newcommand{\out}[1]{{\em #1}}
\newcommand{\initialize}{%
  \paragraph{\large \colorbox{skyblue}{Initialize (Constructor)}}%
    \quad\\ %
    \vspace{3pt}\\
}
\newcommand{\method}{\C \paragraph{\large \colorbox{skyblue}{Methods}}}
% Attribute environment
\newenvironment{at}
{%begin
\paragraph{\large \colorbox{skyblue}{Attribute}}
\quad\\
\begin{description}
}%
{%end
\end{description}
}
% Operation environment
\newenvironment{op}
{%begin
\paragraph{\large \colorbox{skyblue}{Operations}}
\quad\\
\begin{table}[h]
\begin{center}
\begin{tabular}{|l|l|}
\hline
operator & explanation\\
\hline
}%
{%end
\hline
\end{tabular}
\end{center}
\end{table}
}
% Examples environment
\newenvironment{ex}%
{%begin
\paragraph{\large \colorbox{skyblue}{Examples}}
\VerbatimEnvironment
\renewcommand{\EveryVerbatim}{\fontencoding{OT1}\selectfont}
\begin{quote}
\begin{Verbatim}
}%
{%end
\end{Verbatim}
\end{quote}
}
%
\definecolor{skyblue}{cmyk}{0.2, 0, 0.1, 0}
\definecolor{skyyellow}{cmyk}{0.1, 0.1, 0.5, 0}
%
%\title{NZMATH User Manual\\ {\large{(for version 1.0)}}}
%\date{}
%\author{}
\begin{document}
%\maketitle
%
\setcounter{tocdepth}{3}
\setcounter{secnumdepth}{3}


\tableofcontents
\C

\chapter{Classes}


%---------- start document ---------- %
 \section{imaginary -- complex numbers and its functions}\linkedzero{imaginary}
The module {\tt imaginary} provides complex numbers. The functions provided are mainly corresponding to the \linklibrary{cmath} standard module.

 \begin{itemize}
   \item {\bf Classes}
   \begin{itemize}
     \item \linkingone{imaginary}{ComplexField}
     \item \linkingone{imaginary}{Complex}
     \item \negok \linkingone{imaginary}{ExponentialPowerSeries}
     \item \negok \linkingone{imaginary}{AbsoluteError}
     \item \negok \linkingone{imaginary}{RelativeError}
   \end{itemize}
   \item {\bf Functions}
     \begin{itemize}
       \item \linkingone{imaginary}{exp}
       \item \linkingone{imaginary}{expi}
       \item \linkingone{imaginary}{log}
       \item \linkingone{imaginary}{sin}
       \item \linkingone{imaginary}{cos}
       \item \linkingone{imaginary}{tan}
       \item \linkingone{imaginary}{sinh}
       \item \linkingone{imaginary}{cosh}
       \item \linkingone{imaginary}{tanh}
       \item \linkingone{imaginary}{atanh}
       \item \linkingone{imaginary}{sqrt}

     \end{itemize}
 \end{itemize}

This module also provides following constants:
\begin{description}
   \item[e]\linkedone{imaginary}{e}:\\
     This constant is obsolete (Ver 1.1.0).
   \item[pi]\linkedone{imaginary}{pi}:\\
     This constant is obsolete (Ver 1.1.0).
   \item[j]\linkedone{imaginary}{j}:\\
     \param{j} is the imaginary unit.
   \item[theComplexField]\linkedone{imaginary}{theComplexField}:\\
     \param{theComplexField} is the instance of \linkingone{imaginary}{ComplexField}.
 \end{description}

\C
 \subsection{ComplexField -- field of complex numbers}\linkedone{imaginary}{ComplexField}
 The class is for the field of complex numbers. The class has the single instance \linkingone{imaginary}{theComplexField}.

 This class is a subclass of \linkingone{ring}{Field}.

  \initialize
  \func{ComplexField}{}{\out{ComplexField}}\\
  \spacing
  % document of basic document
  \quad Create an instance of ComplexField. 
  % added document
  You may not want to create an instance, since there is already \linkingone{imaginary}{theComplexField}.
  % \spacing
  % input, output document
  %See \linkingone{module}{point} for \param{point}.
  \begin{at}
    \item[zero]\linkedtwo{imaginary}{ComplexField}{zero}:\\ It expresses The additive unit 0. (read only)
    \item[one]\linkedtwo{imaginary}{ComplexField}{one}:\\ It expresses The multiplicative unit 1. (read only)
  \end{at}
  \begin{op}
  %  \verb|+| & Vector sum.\\
  %  \verb|-| & Vector subtraction.\\
  %  \verb|*| & Scalar multiplication.\\
  %  \verb|//| & Scalar division.\\
  %  \verb|-(unary)| & element negation.\\
  %  \verb|==| & equality or not.\\
  %  \verb|!=| & inequality or not.\\
  %  \verb+V[i]+ & Return the coefficient of i-th element of Vector.\\
  %  \verb+V[i] = c+ & Replace the coefficient of i-th element of Vector by c.\\
  %  \verb|len| & return length of \linkingtwo{vector}{Vector}{compo}.\\
    \verb|in| & membership test; return whether an element is in or not.\\
    \verb|repr| & return representation string.\\
    \verb|str| & return string.\\
  \end{op} 
%\begin{ex}
%>>> A = vector.Vector([1,2])
%>>> A
%Vector([1, 2])
%>>> A.compo
%[1, 2]
%>>>
%\end{ex}%Don't indent!
  \method
   \subsubsection{createElement -- create Imaginary object}\linkedtwo{imaginary}{ComplexField}{createElement}
    \func{createElement}{\param{self},\ \hiki{seed}{integer}}{\out{Integer}}\\
    \spacing
    % document of basic document
    \quad Return a Complex object with \param{seed}. 
    \spacing
    % added document
    %\quad \negok Note that this function returns integer only.\\
    \spacing
    % input, output document
    \quad \param{seed} must be complex or numbers having embedding to complex.\\
%
%
  \subsubsection{getCharacteristic -- get characteristic}\linkedtwo{imaginary}{ComplexField}{getCharacteristic}
   \func{getCharacteristic}{\param{self}}{\out{integer}}\\
   \spacing
   % document of basic document
   \quad Return the characteristic, zero.
   %\spacing
   % added document
   %\quad \negok Note that this function returns integer only.\\
   %\spacing
   % input, output document
   %\quad \param{a} must be int, long or rational.Integer.\\
%
  \subsubsection{issubring -- subring test}\linkedtwo{imaginary}{ComplexField}{issubring}
   \func{issubring}{\param{self},\ \hiki{aRing}{\linkingone{ring}{Ring}}}{\out{bool}}\\
   \spacing
   % document of basic document
   \quad Report whether another ring contains the complex field as subring.
   \spacing
   % added document
   %\quad \negok Note that this function returns integer only.\\
   %\spacing
   % input, output document
   %\quad if \param{as\_column} is True, try to create column matrix.\\
%
  \subsubsection{issuperring -- superring test}\linkedtwo{imaginary}{ComplexField}{issuperring}
   \func{issuperring}{\param{self},\ \hiki{aRing}{\linkingone{ring}{Ring}}}{\out{bool}}\\
   \spacing
   % document of basic document
   \quad Report whether the complex field contains another ring as subring.
   \spacing
   % added document
   %\quad \negok Note that this function returns integer only.\\
   %\spacing
   % input, output document
   %\quad if \param{as\_column} is True, try to create column matrix.\\
%\begin{ex}
%>>> A = module.HogeClass((1,2))
%>>> A.hogemethod1(2)
%(2, 4)
%>>>
%\end{ex}%Don't indent!

\C
 \subsection{Complex -- a complex number}\linkedone{imaginary}{Complex}
 Complex is a class of complex number. Each instance has a coupled numbers; real and imaginary part of the number.

 This class is a subclass of \linkingone{ring}{FieldElement}.

 All implemented operators in this class are delegated to complex type. 
  \initialize
  \func{Complex}
       {\hiki{re}{number}
        \hikiopt{im}{number}{0}
       }
       {\out{Imaginary}}\\
  \spacing
  % document of basic document
  \quad Create a complex number.
  % added document
  \spacing
  % input, output document
  \param{re} can be either real or complex number. If \param{re} is real and \param{im} is not given, then its imaginary part is zero. 
  \begin{at}
    \item[real]\linkedtwo{imaginary}{Complex}{real}:\\ It expresses the real part of complex number.
    \item[imag]\linkedtwo{imaginary}{Complex}{imag}:\\ It expresses the imaginary part of complex number.
  \end{at}
  %\begin{op}
  %  \verb|+| & Vector sum.\\
  %  \verb|-| & Vector subtraction.\\
  %  \verb|*| & Scalar multiplication.\\
  %  \verb|//| & Scalar division.\\
  %  \verb|-(unary)| & element negation.\\
  %  \verb|==| & equality or not.\\
  %  \verb|!=| & inequality or not.\\
  %  \verb+V[i]+ & Return the coefficient of i-th element of Vector.\\
  %  \verb+V[i] = c+ & Replace the coefficient of i-th element of Vector by c.\\
  %  \verb|len| & return length of \linkingtwo{vector}{Vector}{compo}.\\
  %  \verb|repr| & return representation string.\\
  %  \verb|str| & return string of \linkingtwo{vector}{Vector}{compo}.\\
  %\end{op} 
%\begin{ex}
%>>> A = vector.Vector([1,2])
%>>> A
%Vector([1, 2])
%>>> A.compo
%[1, 2]
%>>>
%\end{ex}%Don't indent!
  \method
  \subsubsection{getRing -- get ring object}\linkedtwo{imaginary}{Complex}{getRing}
   \func{getRing}{\param{self}}{\out{ComplexField}}\\
   \spacing
   % document of basic document
   \quad Return the complex field instance.
   %\spacing
   % added document
   %\quad \negok Note that this function returns integer only.\\
   %\spacing
   % input, output document
   %\quad \param{a} must be int, long or rational.Integer.\\
%
  \subsubsection{arg -- argument of complex}\linkedtwo{imaginary}{Complex}{arg}
   \func{arg}{\param{self}}{\out{radian}}\\
   \spacing
   % document of basic document
   \quad Return the angle between the x-axis and the number in the Gaussian plane.
   %\spacing
   % added document
   %\quad \negok Note that this function returns integer only.\\
   %\spacing
   % input, output document
   \quad \out{radian} must be Float.\\
%
  \subsubsection{conjugate -- complex conjugate}\linkedtwo{imaginary}{Complex}{conjugate}
   \func{conjugate}{\param{self}}{\out{Complex}}\\
   \spacing
   % document of basic document
   \quad Return the complex conjugate of the number.
   %\spacing
   % added document
   %\quad \negok Note that this function returns integer only.\\
   %\spacing
   % input, output document
   %\quad \out{radian} must be Float.\\
%
  \subsubsection{copy -- copied number}\linkedtwo{imaginary}{Complex}{copy}
   \func{copy}{\param{self}}{\out{Complex}}\\
   \spacing
   % document of basic document
   \quad Return the copy of the number itself.
   %\spacing
   % added document
   %\quad \negok Note that this function returns integer only.\\
   %\spacing
   % input, output document
   %\quad \out{radian} must be Float.\\
%
  \subsubsection{inverse -- complex inverse}\linkedtwo{imaginary}{Complex}{inverse}
   \func{inverse}{\param{self}}{\out{Complex}}\\
   \spacing
   % document of basic document
   \quad Return the inverse of the number.
   \spacing
   % added document
   %\quad \negok Note that this function returns integer only.\\
   %\spacing
   % input, output document
   \quad If the number is zero, ZeroDivisionError is raised.
%
%\begin{ex}
%>>> A = module.HogeClass((1,2))
%>>> A.hogemethod1(2)
%(2, 4)
%>>>
%\end{ex}%Don't indent!
\C
 \subsection{ExponentialPowerSeries -- exponential power series}\linkedone{imaginary}{ExponentialPowerSeries}
  This class is obsolete (Ver 1.1.0).

 \subsection{AbsoluteError -- absolute error}\linkedone{imaginary}{AbsoluteError}
  This class is obsolete (Ver 1.1.0).

 \subsection{RelativeError -- relative error}\linkedone{imaginary}{RelativeError}
  This class is obsolete (Ver 1.1.0).

  \subsection{exp(function) -- exponential value}\linkedone{imaginary}{exp}
   This function is obsolete (Ver 1.1.0).

  \subsection{expi(function) -- imaginary exponential value}\linkedone{imaginary}{expi}
   This function is obsolete (Ver 1.1.0).

  \subsection{log(function) -- logarithm}\linkedone{imaginary}{log}
   This function is obsolete (Ver 1.1.0).

  \subsection{sin(function) -- sine function}\linkedone{imaginary}{sin}
   This function is obsolete (Ver 1.1.0).

  \subsection{cos(function) -- cosine function}\linkedone{imaginary}{cos}
   This function is obsolete (Ver 1.1.0).

  \subsection{tan(function) -- tangent function}\linkedone{imaginary}{tan}
   This function is obsolete (Ver 1.1.0).

  \subsection{sinh(function) -- hyperbolic sine function}\linkedone{imaginary}{sinh}
   This function is obsolete (Ver 1.1.0).

  \subsection{cosh(function) -- hyperbolic cosine function}\linkedone{imaginary}{cosh}
   This function is obsolete (Ver 1.1.0).

  \subsection{tanh(function) -- hyperbolic tangent function}\linkedone{imaginary}{tanh}
   This function is obsolete (Ver 1.1.0).

  \subsection{atanh(function) -- hyperbolic arc tangent function}\linkedone{imaginary}{atanh}
   This function is obsolete (Ver 1.1.0).

  \subsection{sqrt(function) -- square root}\linkedone{imaginary}{sqrt}
   This function is obsolete (Ver 1.1.0).

%
% \begin{ex}
% >>> A = vector.Vector([1,2,3])
% >>> B = vector.Vector([2,1,0])
% >>> vector.innerProduct(A,B)
% 4
% >>>
% \end{ex}%Don't indent!
\C

%---------- end document ---------- %

\bibliographystyle{jplain}%use jbibtex
\bibliography{nzmath_references}

\end{document}

