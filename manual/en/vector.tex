%%%%%%%%%%%%%%%%%%%%%%%%%%%%%%%%%%%%%%%%%%%%%%%%%%%%%%%%%%%%%
%
% macros for nzmath manual
%
%%%%%%%%%%%%%%%%%%%%%%%%%%%%%%%%%%%%%%%%%%%%%%%%%%%%%%%%%%%%%
\usepackage{amssymb,amsmath}
\usepackage{color}
\usepackage[dvipdfm,bookmarks=true,bookmarksnumbered=true,%
 pdftitle={NZMATH Users Manual},%
 pdfsubject={Manual for NZMATH Users},%
 pdfauthor={NZMATH Development Group},%
 pdfkeywords={TeX; dvipdfmx; hyperref; color;},%
 colorlinks=true]{hyperref}
\usepackage{fancybox}
\usepackage[T1]{fontenc}
%
\newcommand{\DS}{\displaystyle}
\newcommand{\C}{\clearpage}
\newcommand{\NO}{\noindent}
\newcommand{\negok}{$\dagger$}
\newcommand{\spacing}{\vspace{1pt}\\ }
% software macros
\newcommand{\nzmathzero}{{\footnotesize $\mathbb{N}\mathbb{Z}$}\texttt{MATH}}
\newcommand{\nzmath}{{\nzmathzero}\ }
\newcommand{\pythonzero}{$\mbox{\texttt{Python}}$}
\newcommand{\python}{{\pythonzero}\ }
% link macros
\newcommand{\linkingzero}[1]{{\bf \hyperlink{#1}{#1}}}%module
\newcommand{\linkingone}[2]{{\bf \hyperlink{#1.#2}{#2}}}%module,class/function etc.
\newcommand{\linkingtwo}[3]{{\bf \hyperlink{#1.#2.#3}{#3}}}%module,class,method
\newcommand{\linkedzero}[1]{\hypertarget{#1}{}}
\newcommand{\linkedone}[2]{\hypertarget{#1.#2}{}}
\newcommand{\linkedtwo}[3]{\hypertarget{#1.#2.#3}{}}
\newcommand{\linktutorial}[1]{\href{http://docs.python.org/tutorial/#1}{#1}}
\newcommand{\linktutorialone}[2]{\href{http://docs.python.org/tutorial/#1}{#2}}
\newcommand{\linklibrary}[1]{\href{http://docs.python.org/library/#1}{#1}}
\newcommand{\linklibraryone}[2]{\href{http://docs.python.org/library/#1}{#2}}
\newcommand{\pythonhp}{\href{http://www.python.org/}{\python website}}
\newcommand{\nzmathwiki}{\href{http://nzmath.sourceforge.net/wiki/}{{\nzmathzero}Wiki}}
\newcommand{\nzmathsf}{\href{http://sourceforge.net/projects/nzmath/}{\nzmath Project Page}}
\newcommand{\nzmathtnt}{\href{http://tnt.math.se.tmu.ac.jp/nzmath/}{\nzmath Project Official Page}}
% parameter name
\newcommand{\param}[1]{{\tt #1}}
% function macros
\newcommand{\hiki}[2]{{\tt #1}:\ {\em #2}}
\newcommand{\hikiopt}[3]{{\tt #1}:\ {\em #2}=#3}

\newdimen\hoge
\newdimen\truetextwidth
\newcommand{\func}[3]{%
\setbox0\hbox{#1(#2)}
\hoge=\wd0
\truetextwidth=\textwidth
\advance \truetextwidth by -2\oddsidemargin
\ifdim\hoge<\truetextwidth % short form
{\bf \colorbox{skyyellow}{#1(#2)\ $\to$ #3}}
%
\else % long form
\fcolorbox{skyyellow}{skyyellow}{%
   \begin{minipage}{\textwidth}%
   {\bf #1(#2)\\ %
    \qquad\quad   $\to$\ #3}%
   \end{minipage}%
   }%
\fi%
}

\newcommand{\out}[1]{{\em #1}}
\newcommand{\initialize}{%
  \paragraph{\large \colorbox{skyblue}{Initialize (Constructor)}}%
    \quad\\ %
    \vspace{3pt}\\
}
\newcommand{\method}{\C \paragraph{\large \colorbox{skyblue}{Methods}}}
% Attribute environment
\newenvironment{at}
{%begin
\paragraph{\large \colorbox{skyblue}{Attribute}}
\quad\\
\begin{description}
}%
{%end
\end{description}
}
% Operation environment
\newenvironment{op}
{%begin
\paragraph{\large \colorbox{skyblue}{Operations}}
\quad\\
\begin{table}[h]
\begin{center}
\begin{tabular}{|l|l|}
\hline
operator & explanation\\
\hline
}%
{%end
\hline
\end{tabular}
\end{center}
\end{table}
}
% Examples environment
\newenvironment{ex}%
{%begin
\paragraph{\large \colorbox{skyblue}{Examples}}
\VerbatimEnvironment
\renewcommand{\EveryVerbatim}{\fontencoding{OT1}\selectfont}
\begin{quote}
\begin{Verbatim}
}%
{%end
\end{Verbatim}
\end{quote}
}
%
\definecolor{skyblue}{cmyk}{0.2, 0, 0.1, 0}
\definecolor{skyyellow}{cmyk}{0.1, 0.1, 0.5, 0}
%
%\title{NZMATH User Manual\\ {\large{(for version 1.0)}}}
%\date{}
%\author{}
\begin{document}
%\maketitle
%
\setcounter{tocdepth}{3}
\setcounter{secnumdepth}{3}


\tableofcontents
\C

\chapter{Classes}


%---------- start document ---------- %
 \section{vector -- vector object and arithmetic}\linkedzero{vector}
 \begin{itemize}
   \item {\bf Classes}
   \begin{itemize}
     \item \linkingone{vector}{Vector}
   \end{itemize}
   \item {\bf Functions}
     \begin{itemize}
       \item \linkingone{vector}{innerProduct}
     \end{itemize}
 \end{itemize}

This module provides an exception class.
\begin{description}
  \item[VectorSizeError]:\ Report vector size is invalid. (Mainly for operations with two vectors.)
\end{description}

\C

 \subsection{Vector -- vector class}\linkedone{vector}{Vector}
 Vector is a class for vector.
  \initialize
  \func{Vector}{\hiki{compo}{list}}{\out{Vector}}\\
  \spacing
  % document of basic document
  \quad Create Vector object from \param{compo}.
  % added document
  %
  % \spacing
  % input, output document
  \param{compo} must be a list of elements which are an integer or an instance of \linkingone{ring}{RingElement}.
  \begin{at}
    \item[compo]\linkedtwo{vector}{Vector}{compo}:\\ It expresses component of vector.
  \end{at}
  \begin{op}
    \verb|u+v| & Vector sum.\\
    \verb|u-v| & Vector subtraction.\\
    \verb|A*v| & Multiplication vector with matrix\\
    \verb|a*v| & or scalar multiplication.\\
    \verb|v//a| & Scalar division.\\
    \verb|v%n| & Reduction each elements of \linkingtwo{vector}{Vector}{compo}\\
    \verb|-v| & element negation.\\
    \verb|u==v| & equality.\\
    \verb|u!=v| & inequality.\\
    \verb+v[i]+ & Return the coefficient of i-th element of Vector.\\
    \verb+v[i] = c+ & Replace the coefficient of i-th element of Vector by c.\\
    \verb|len(v)| & return length of \linkingtwo{vector}{Vector}{compo}.\\
    \verb|repr(v)| & return representation string.\\
    \verb|str(v)| & return string of \linkingtwo{vector}{Vector}{compo}.\\
  \end{op}
  Note that index is 1-origin, which is standard in mathematics field.
\begin{ex}
>>> A = vector.Vector([1, 2])
>>> A
Vector([1, 2])
>>> A.compo
[1, 2]
>>> B = vector.Vector([2, 1])
>>> A + B
Vector([3, 3])
>>> A % 2
Vector([1, 0])
>>> A[1]
1
>>> len(B)
2
\end{ex}%Don't indent!
  \method
  \subsubsection{copy -- copy itself}\linkedtwo{vector}{Vector}{copy}
   \func{copy}{\param{self}}{\out{Vector}}\\
   \spacing
   % document of basic document
   \quad Return copy of \param{self}.\\
   \spacing
   % added document
   %\quad \negok Note that this function returns integer only.\\
   %\spacing
   % input, output document
   %\quad \param{a} must be int, long or rational.Integer.\\
  \subsubsection{set -- set other compo}\linkedtwo{vector}{Vector}{set}
   \func{set}{\param{self},\ \hiki{compo}{list}}{(None)}\\
   \spacing
   % document of basic document
   \quad Substitute \linkingtwo{vector}{Vector}{compo} with \param{compo}.\\
   \spacing
   % added document
   %\quad \negok Note that this function returns integer only.\\
   %\spacing
   % input, output document
   %\quad \param{a} must be int, long or rational.Integer.\\
  \subsubsection{indexOfNoneZero -- first non-zero coordinate}\linkedtwo{vector}{Vector}{indexOfNoneZero}
   \func{indexOfNoneZero}{\param{self}}{integer}\\
   \spacing
   % document of basic document
   \quad Return the first index of non-zero element of \param{self}.\linkingtwo{vector}{Vector}{compo}.\\
   \spacing
   % added document
   \quad \negok Raise ValueError if all elements of \linkingtwo{vector}{Vector}{compo} are zero.\\
   \spacing
   % input, output document
   %\quad \param{a} must be int, long or rational.Integer.\\
  \subsubsection{toMatrix -- convert to Matrix object}\linkedtwo{vector}{Vector}{toMatrix}
   \func{toMatrix}{\param{self},\ \hikiopt{as\_column}{bool}{False}}{\out{Matrix}}\\
   \spacing
   % document of basic document
   \quad Return \linkingone{matrix}{Matrix} object using \linkingone{matrix}{createMatrix} function.\\
   \spacing
   % added document
   %\quad \negok Note that this function returns integer only.\\
   %\spacing
   % input, output document
   \quad If \param{as\_column} is True, create the column matrix with \param{self}.
   Otherwise, create the row matrix.\\
\begin{ex}
>>> A = vector.Vector([0, 4, 5])
>>> A.indexOfNoneZero()
2
>>> print A.toMatrix()
0 4 5
>>> print A.toMatrix()
0
4
5
\end{ex}%Don't indent!
\C
  \subsection{innerProduct(function) -- inner product}\linkedone{vector}{innerProduct}
  \func{innerProduct}{\hiki{bra}{Vector}, \ \hiki{ket}{Vector}}{\out{RingElement}}\\
   \spacing
   % document of basic document
   \quad Return the inner product of \param{bra} and \param{ket}.\\
   \spacing
   % added document
   \quad The function supports Hermitian inner product for elements in the complex number field.\\
   \spacing
   % input, output document
   \quad \negok Note that the returned value depends on type of elements.\\
\begin{ex}
>>> A = vector.Vector([1, 2, 3])
>>> B = vector.Vector([2, 1, 0])
>>> vector.innerProduct(A, B)
4
>>> C = vector.Vector([1+1j, 2+2j, 3+3j])
>>> vector.innerProduct(C, C)
(28+0j)
\end{ex}%Don't indent!
\C

%---------- end document ---------- %

\bibliographystyle{jplain}%use jbibtex
\bibliography{nzmath_references}

\end{document}

