%%%%%%%%%%%%%%%%%%%%%%%%%%%%%%%%%%%%%%%%%%%%%%%%%%%%%%%%%%%%%
%
% macros for nzmath manual
%
%%%%%%%%%%%%%%%%%%%%%%%%%%%%%%%%%%%%%%%%%%%%%%%%%%%%%%%%%%%%%
\usepackage{amssymb,amsmath}
\usepackage{color}
\usepackage[dvipdfm,bookmarks=true,bookmarksnumbered=true,%
 pdftitle={NZMATH Users Manual},%
 pdfsubject={Manual for NZMATH Users},%
 pdfauthor={NZMATH Development Group},%
 pdfkeywords={TeX; dvipdfmx; hyperref; color;},%
 colorlinks=true]{hyperref}
\usepackage{fancybox}
\usepackage[T1]{fontenc}
%
\newcommand{\DS}{\displaystyle}
\newcommand{\C}{\clearpage}
\newcommand{\NO}{\noindent}
\newcommand{\negok}{$\dagger$}
\newcommand{\spacing}{\vspace{1pt}\\ }
% software macros
\newcommand{\nzmathzero}{{\footnotesize $\mathbb{N}\mathbb{Z}$}\texttt{MATH}}
\newcommand{\nzmath}{{\nzmathzero}\ }
\newcommand{\pythonzero}{$\mbox{\texttt{Python}}$}
\newcommand{\python}{{\pythonzero}\ }
% link macros
\newcommand{\linkingzero}[1]{{\bf \hyperlink{#1}{#1}}}%module
\newcommand{\linkingone}[2]{{\bf \hyperlink{#1.#2}{#2}}}%module,class/function etc.
\newcommand{\linkingtwo}[3]{{\bf \hyperlink{#1.#2.#3}{#3}}}%module,class,method
\newcommand{\linkedzero}[1]{\hypertarget{#1}{}}
\newcommand{\linkedone}[2]{\hypertarget{#1.#2}{}}
\newcommand{\linkedtwo}[3]{\hypertarget{#1.#2.#3}{}}
\newcommand{\linktutorial}[1]{\href{http://docs.python.org/tutorial/#1}{#1}}
\newcommand{\linktutorialone}[2]{\href{http://docs.python.org/tutorial/#1}{#2}}
\newcommand{\linklibrary}[1]{\href{http://docs.python.org/library/#1}{#1}}
\newcommand{\linklibraryone}[2]{\href{http://docs.python.org/library/#1}{#2}}
\newcommand{\pythonhp}{\href{http://www.python.org/}{\python website}}
\newcommand{\nzmathwiki}{\href{http://nzmath.sourceforge.net/wiki/}{{\nzmathzero}Wiki}}
\newcommand{\nzmathsf}{\href{http://sourceforge.net/projects/nzmath/}{\nzmath Project Page}}
\newcommand{\nzmathtnt}{\href{http://tnt.math.se.tmu.ac.jp/nzmath/}{\nzmath Project Official Page}}
% parameter name
\newcommand{\param}[1]{{\tt #1}}
% function macros
\newcommand{\hiki}[2]{{\tt #1}:\ {\em #2}}
\newcommand{\hikiopt}[3]{{\tt #1}:\ {\em #2}=#3}

\newdimen\hoge
\newdimen\truetextwidth
\newcommand{\func}[3]{%
\setbox0\hbox{#1(#2)}
\hoge=\wd0
\truetextwidth=\textwidth
\advance \truetextwidth by -2\oddsidemargin
\ifdim\hoge<\truetextwidth % short form
{\bf \colorbox{skyyellow}{#1(#2)\ $\to$ #3}}
%
\else % long form
\fcolorbox{skyyellow}{skyyellow}{%
   \begin{minipage}{\textwidth}%
   {\bf #1(#2)\\ %
    \qquad\quad   $\to$\ #3}%
   \end{minipage}%
   }%
\fi%
}

\newcommand{\out}[1]{{\em #1}}
\newcommand{\initialize}{%
  \paragraph{\large \colorbox{skyblue}{Initialize (Constructor)}}%
    \quad\\ %
    \vspace{3pt}\\
}
\newcommand{\method}{\C \paragraph{\large \colorbox{skyblue}{Methods}}}
% Attribute environment
\newenvironment{at}
{%begin
\paragraph{\large \colorbox{skyblue}{Attribute}}
\quad\\
\begin{description}
}%
{%end
\end{description}
}
% Operation environment
\newenvironment{op}
{%begin
\paragraph{\large \colorbox{skyblue}{Operations}}
\quad\\
\begin{table}[h]
\begin{center}
\begin{tabular}{|l|l|}
\hline
operator & explanation\\
\hline
}%
{%end
\hline
\end{tabular}
\end{center}
\end{table}
}
% Examples environment
\newenvironment{ex}%
{%begin
\paragraph{\large \colorbox{skyblue}{Examples}}
\VerbatimEnvironment
\renewcommand{\EveryVerbatim}{\fontencoding{OT1}\selectfont}
\begin{quote}
\begin{Verbatim}
}%
{%end
\end{Verbatim}
\end{quote}
}
%
\definecolor{skyblue}{cmyk}{0.2, 0, 0.1, 0}
\definecolor{skyyellow}{cmyk}{0.1, 0.1, 0.5, 0}
%
%\title{NZMATH User Manual\\ {\large{(for version 1.0)}}}
%\date{}
%\author{}
\begin{document}
%\maketitle
%
\setcounter{tocdepth}{3}
\setcounter{secnumdepth}{3}


\tableofcontents
\C

\chapter{Classes}


%---------- start document ---------- %
 \section{factor.util -- utilities for factorization}\linkedzero{factor.util}
 \begin{itemize}
   \item {\bf Classes}
   \begin{itemize}
     \item \linkingone{factor.util}{FactoringInteger}
     \item \linkingone{factor.util}{FactoringMethod}
   \end{itemize}
 \end{itemize}

 This module uses following type:
 \begin{description}
   \item[factorlist]\linkedone{factor.util}{factorlist}:\\
     \param{factorlist} is a list which consists of pairs {\tt (base, index)}.
     Each pair means \(base^{index}\).
     The product of those terms expresses whole prime factorization.
 \end{description}

\C

 \subsection{FactoringInteger -- keeping track of factorization}\linkedone{factor.util}{FactoringInteger}
 \initialize
  \func{FactoringInteger}{\hiki{number}{integer}}{\out{FactoringInteger}}\\
  \spacing
  % document of basic document
  \quad This is the base class for factoring integers.\\
  \spacing
  % added document
  \quad  \param{number} is stored in the attribute \linkingtwo{factor.util}{FactoringInteger}{number}. The factors will be stored in the attribute \linkingtwo{factor.util}{FactoringInteger}{factors}, and primality of factors will be tracked in the attribute \linkingtwo{factor.util}{FactoringInteger}{primality}.\\
  \spacing
  % input/output document
  \quad The given \param{number} must be a composite number.\\
  \begin{at}
    \item[number]\linkedtwo{factor.util}{FactoringInteger}{number}:\\ The composite number.
    \item[factors]\linkedtwo{factor.util}{FactoringInteger}{factors}:\\ Factors known at the time being referred.
    \item[primality]\linkedtwo{factor.util}{FactoringInteger}{primality}:\\ A dictionary of primality information of known factors.
      {\tt True} if the factor is prime, {\tt False} composite, or {\tt None} undetermined.
  \end{at}
%   \begin{op}
%     \verb+A==B+ & Return whether M and N are equal or not.\\
%   \end{op} 
% \begin{ex}
% >>> A = factor.util.FactoringInteger((1,2))
% >>> print A
% (1, 2)
% >>> A.point
% (1, 2)
% >>>
% \end{ex}%Don't indent!
  \method
  \subsubsection{getNextTarget -- next target}\linkedtwo{factor.util}{FactoringInteger}{getNextTarget}
   \func{getNextTarget}{\param{self},\ \hikiopt{cond}{function}{{\tt None}}}{\out{integer}}\\
   \spacing
   % document of basic document
   \quad Return the next target which meets \param{cond}.\\
   \spacing
   % added document
   If \param{cond} is not specified, then the next target is a composite (or undetermined) factor of \linkingtwo{factor.util}{FactoringInteger}{number}.\\
   \spacing
   % input, output document
   \quad \param{cond} should be a binary predicate whose arguments are base and index.\\
   \quad If there is no target factor, \linklibraryone{exceptions\#exceptions.LookupError}{LookupError} will be raised.\\
%
 \subsubsection{getResult -- result of factorization}\linkedtwo{factor.util}{FactoringInteger}{getResult}
   \func{getResult}{\param{self}}{\out{\linkingone{factor.util}{factors}}}\\
   \spacing
   % document of basic document
   \quad Return the currently known factorization of the \linkingtwo{factor.util}{FactoringInteger}{number}.\\
%
 \subsubsection{register -- register a new factor}\linkedtwo{factor.util}{FactoringInteger}{register}
   \func{register}{\param{self},\ \hiki{divisor}{integer},\ \hikiopt{isprime}{bool}{{\tt None}}}{}\\
   \spacing
   Register a \param{divisor} of the \linkingtwo{factor.util}{FactoringInteger}{number} if the \param{divisor} is a true divisor of the number.\\
   \spacing
   %added document
   \quad The number is divided by the \param{divisor} as many times as possible.\\
   \spacing
   % input/output document
   The optional argument \param{isprime} tells the primality of the
   \param{divisor} (default to undetermined).\\
%
 \subsubsection{sortFactors -- sort factors}\linkedtwo{factor.util}{FactoringInteger}{sortFactors}
   \func{sortFactors}{\param{self}}{}
   \spacing
   \quad Sort factors list.\\
   \spacing
    % added document
    \quad This affects the result of \linkingtwo{factor.util}{FactoringInteger}{getResult}.\\
%
\begin{ex}
>>> A = factor.util.FactoringInteger(100)
>>> A.getNextTarget()
100
>>> A.getResult()
[(100, 1)]
>>> A.register(5, True)
>>> A.getResult()
[(5, 2), (4, 1)]
>>> A.sortFactors()
>>> A.getResult()
[(4, 1), (5, 2)
>>> A.primality
{4: None, 5: True}
>>> A.getNextTarget()
4
\end{ex}%Don't indent!
\C

 \subsection{FactoringMethod -- method of factorization}\linkedone{factor.util}{FactoringMethod}
 \initialize
  \func{FactoringMethod}{}{\out{FactoringMethod}}\\
  \spacing
  % document of basic document
  \quad Base class of factoring methods.\\
  \spacing
  % added document
  \quad All methods defined in \linkingzero{factor.methods} are
  implemented as derived classes of this class. The method which users may call is  \linkingtwo{factor.util}{FactoringMethod}{factor} only. 
  Other methods are explained for future implementers of a new factoring method.\\
  \method
  \subsubsection{factor -- do factorization}\linkedtwo{factor.util}{FactoringMethod}{factor}
   \func{factor}{\param{self},\ %
     \hiki{number}{integer},\ %
     \hikiopt{return\_type}{str}{'list'},\ %
     \hikiopt{need\_sort}{bool}{False}
   }{\out{\linkingone{factor.util}{factorlist}}}\\
   \spacing
   % document of basic document
   \quad Return the factorization of the given positive integer \param{number}.
   \spacing
   % input, output document
   \quad The default returned type is a \linkingone{factor.util}{factorlist}.\\
   \quad A keyword option \param{return\_type} can be as the following:
   \begin{enumerate}
   \item {\tt 'list'} for default type (\linkingone{factor.util}{factorlist}).
   \item {\tt 'tracker'} for \linkingone{factor.util}{FactoringInteger}.
   \end{enumerate}
   \quad Another keyword option \param{need\_sort} is Boolean:
   {\tt True} to sort the result.
   This should be specified with {\tt return\_type='list'}.\\
%
  \subsubsection{\negok continue\_factor -- continue factorization}\linkedtwo{factor.util}{FactoringMethod}{continue\_factor}
  \func{continue\_factor}{\param{self},\ %
    \hiki{tracker}{\linkingone{factor.util}{FactoringInteger}},\ %
    \hikiopt{return\_type}{str}{'tracker'},\ %
    \hikiopt{primeq}{func}{\linkingone{prime}{primeq}}
  }{\out{\linkingone{factor.util}{FactoringInteger}}}\\
  \spacing
  \quad Continue factoring of the given \param{tracker} and return the
  result of factorization.\\
  \spacing
  \quad The default returned type is \linkingone{factor.util}{FactoringInteger},
  but if \param{return\_type} is specified as {\tt 'list'}
  then it returns \linkingone{factor.util}{factorlist}.
  The primality is judged by a function specified in \param{primeq}
  optional keyword argument, which default is \linkingone{prime}{primeq}.\\
%
  \subsubsection{\negok find -- find a factor}\linkedtwo{factor.util}{FactoringMethod}{find}
  \func{find}{\param{self},\ %
    \hiki{target}{integer},\ %
    **\param{options}
  }{\out{integer}}\\
  \spacing
  \quad Find a factor from the \param{target} number.\\
  \spacing
  \quad This method has to be overridden, or \linkingtwo{factor.util}{FactoringMethod}{factor} method should be overridden not to call this method.\\
%
  \subsubsection{\negok generate -- generate prime factors}\linkedtwo{factor.util}{FactoringMethod}{generate}
  \func{generate}{\param{self},\ %
    \hiki{target}{integer},\ %
    **\param{options}
  }{\out{integer}}\\
  \spacing
 % basic document
  \quad Generate prime factors of the \param{target} number with their valuations.\\
 \spacing
 % added document
  \quad The method may terminate with yielding {\tt (1, 1)}
  to indicate the factorization is incomplete.\\
  This method has to be overridden, or \linkingtwo{factor.util}{FactoringMethod}{factor} method should be overridden not to call this method.\\
\C

%---------- end document ---------- %

\bibliographystyle{jplain}%use jbibtex
\bibliography{nzmath_references}

\end{document}

