\documentclass{report}

%%%%%%%%%%%%%%%%%%%%%%%%%%%%%%%%%%%%%%%%%%%%%%%%%%%%%%%%%%%%%
%
% macros for nzmath manual
%
%%%%%%%%%%%%%%%%%%%%%%%%%%%%%%%%%%%%%%%%%%%%%%%%%%%%%%%%%%%%%
\usepackage{amssymb,amsmath}
\usepackage{color}
\usepackage[dvipdfm,bookmarks=true,bookmarksnumbered=true,%
 pdftitle={NZMATH Users Manual},%
 pdfsubject={Manual for NZMATH Users},%
 pdfauthor={NZMATH Development Group},%
 pdfkeywords={TeX; dvipdfmx; hyperref; color;},%
 colorlinks=true]{hyperref}
\usepackage{fancybox}
\usepackage[T1]{fontenc}
%
\newcommand{\DS}{\displaystyle}
\newcommand{\C}{\clearpage}
\newcommand{\NO}{\noindent}
\newcommand{\negok}{$\dagger$}
\newcommand{\spacing}{\vspace{1pt}\\ }
% software macros
\newcommand{\nzmathzero}{{\footnotesize $\mathbb{N}\mathbb{Z}$}\texttt{MATH}}
\newcommand{\nzmath}{{\nzmathzero}\ }
\newcommand{\pythonzero}{$\mbox{\texttt{Python}}$}
\newcommand{\python}{{\pythonzero}\ }
% link macros
\newcommand{\linkingzero}[1]{{\bf \hyperlink{#1}{#1}}}%module
\newcommand{\linkingone}[2]{{\bf \hyperlink{#1.#2}{#2}}}%module,class/function etc.
\newcommand{\linkingtwo}[3]{{\bf \hyperlink{#1.#2.#3}{#3}}}%module,class,method
\newcommand{\linkedzero}[1]{\hypertarget{#1}{}}
\newcommand{\linkedone}[2]{\hypertarget{#1.#2}{}}
\newcommand{\linkedtwo}[3]{\hypertarget{#1.#2.#3}{}}
\newcommand{\linktutorial}[1]{\href{http://docs.python.org/tutorial/#1}{#1}}
\newcommand{\linktutorialone}[2]{\href{http://docs.python.org/tutorial/#1}{#2}}
\newcommand{\linklibrary}[1]{\href{http://docs.python.org/library/#1}{#1}}
\newcommand{\linklibraryone}[2]{\href{http://docs.python.org/library/#1}{#2}}
\newcommand{\pythonhp}{\href{http://www.python.org/}{\python website}}
\newcommand{\nzmathwiki}{\href{http://nzmath.sourceforge.net/wiki/}{{\nzmathzero}Wiki}}
\newcommand{\nzmathsf}{\href{http://sourceforge.net/projects/nzmath/}{\nzmath Project Page}}
\newcommand{\nzmathtnt}{\href{http://tnt.math.se.tmu.ac.jp/nzmath/}{\nzmath Project Official Page}}
% parameter name
\newcommand{\param}[1]{{\tt #1}}
% function macros
\newcommand{\hiki}[2]{{\tt #1}:\ {\em #2}}
\newcommand{\hikiopt}[3]{{\tt #1}:\ {\em #2}=#3}

\newdimen\hoge
\newdimen\truetextwidth
\newcommand{\func}[3]{%
\setbox0\hbox{#1(#2)}
\hoge=\wd0
\truetextwidth=\textwidth
\advance \truetextwidth by -2\oddsidemargin
\ifdim\hoge<\truetextwidth % short form
{\bf \colorbox{skyyellow}{#1(#2)\ $\to$ #3}}
%
\else % long form
\fcolorbox{skyyellow}{skyyellow}{%
   \begin{minipage}{\textwidth}%
   {\bf #1(#2)\\ %
    \qquad\quad   $\to$\ #3}%
   \end{minipage}%
   }%
\fi%
}

\newcommand{\out}[1]{{\em #1}}
\newcommand{\initialize}{%
  \paragraph{\large \colorbox{skyblue}{Initialize (Constructor)}}%
    \quad\\ %
    \vspace{3pt}\\
}
\newcommand{\method}{\C \paragraph{\large \colorbox{skyblue}{Methods}}}
% Attribute environment
\newenvironment{at}
{%begin
\paragraph{\large \colorbox{skyblue}{Attribute}}
\quad\\
\begin{description}
}%
{%end
\end{description}
}
% Operation environment
\newenvironment{op}
{%begin
\paragraph{\large \colorbox{skyblue}{Operations}}
\quad\\
\begin{table}[h]
\begin{center}
\begin{tabular}{|l|l|}
\hline
operator & explanation\\
\hline
}%
{%end
\hline
\end{tabular}
\end{center}
\end{table}
}
% Examples environment
\newenvironment{ex}%
{%begin
\paragraph{\large \colorbox{skyblue}{Examples}}
\VerbatimEnvironment
\renewcommand{\EveryVerbatim}{\fontencoding{OT1}\selectfont}
\begin{quote}
\begin{Verbatim}
}%
{%end
\end{Verbatim}
\end{quote}
}
%
\definecolor{skyblue}{cmyk}{0.2, 0, 0.1, 0}
\definecolor{skyyellow}{cmyk}{0.1, 0.1, 0.5, 0}
%
%\title{NZMATH User Manual\\ {\large{(for version 1.0)}}}
%\date{}
%\author{}
\begin{document}
%\maketitle
%
\setcounter{tocdepth}{3}
\setcounter{secnumdepth}{3}


\tableofcontents
\C

\chapter{Classes}


%---------- start document ---------- %
 \section{poly.uniutil -- univariate utilities}\linkedzero{poly.uniutil}
 \begin{itemize}
   \item {\bf Classes}
   \begin{itemize}
     \item \linkingone{poly.uniutil}{RingPolynomial}
     \item \linkingone{poly.uniutil}{DomainPolynomial}
     \item \linkingone{poly.uniutil}{UniqueFactorizationDomainPolynomial}
     \item \linkingone{poly.uniutil}{IntegerPolynomial}
     \item \linkingone{poly.uniutil}{FieldPolynomial}
     \item \linkingone{poly.uniutil}{FinitePrimeFieldPolynomial}
     \item OrderProvider
     \item DivisionProvider
     \item PseudoDivisionProvider
     \item ContentProvider
     \item SubresultantGcdProvider
     \item PrimeCharacteristicFunctionsProvider
     \item VariableProvider
     \item RingElementProvider
   \end{itemize}
   \item {\bf Functions}
     \begin{itemize}
       \item \linkingone{poly.uniutil}{polynomial}
     \end{itemize}
 \end{itemize}

\C

 \subsection{RingPolynomial -- polynomial over commutative ring}\linkedone{poly.uniutil}{RingPolynomial}

 \initialize
  \func{RingPolynomial}{%
    \hiki{coefficients}{terminit},\ %
    \hiki{coeffring}{CommutativeRing},\ %
    **\hiki{keywords}{dict}}{\out{RingPolynomial object}}\\
  \spacing
  % document of basic document
  \quad Initialize a polynomial over the given commutative ring \param{coeffring}.\\
  \spacing
  % added document
  \quad This class inherits from \linkingone{poly.univar}{SortedPolynomial},
  \linkingone{poly.uniutil}{OrderProvider} and \linkingone{poly.uniutil}{RingElementProvider}.\\
  \spacing
  % input, output document
  \quad The type of the \param{coefficients} is \linkingone{poly.formalsum}{terminit}.
  \param{coeffring} is an instance of descendant of \linkingone{ring}{CommutativeRing}.\\
  \method
  \subsubsection{getRing}\linkedtwo{poly.uniutil}{RingPolynomial}{getRing}
  \func{getRing}{\param{self}}{\out{Ring}}\\
  \spacing
  \quad Return an object of a subclass of {\tt Ring},
  to which the polynomial belongs.\\
  (This method overrides the definition in RingElementProvider)

  \subsubsection{getCoefficientRing}\linkedtwo{poly.uniutil}{RingPolynomial}{getCoefficientRing}
  \func{getCoefficientRing}{\param{self}}{\out{Ring}}\\
  \spacing
  \quad Return an object of a subclass of {\tt Ring},
  to which the all coefficients belong.\\
  (This method overrides the definition in RingElementProvider)

  \subsubsection{shift\_degree\_to}\linkedtwo{poly.uniutil}{RingPolynomial}{shift\_degree\_to}
  \func{shift\_degree\_to}{\param{self}, \hiki{degree}{integer}}{\out{polynomial}}\\
  \spacing
  \quad Return polynomial whose degree is the given \param{degree}.
  More precisely, let \(f(X) = a_0 + ... + a_n X^n\), then
  {\tt f.shift\_degree\_to(m)} returns:
  \begin{itemize}
  \item zero polynomial, if f is zero polynomial
  \item \(a_{n-m} + ... + a_n X^m\), if \(0 \leq m < n\)
  \item \(a_0 X^{m-n} + ... + a_n X^m\), if \(m \geq n\)
  \end{itemize}
  (This method is inherited from OrderProvider)

  \subsubsection{split\_at}\linkedtwo{poly.uniutil}{RingPolynomial}{split\_at}
  \func{split\_at}{\param{self},\ \hiki{degree}{integer}}{\out{polynomial}}\\
  \spacing
  \quad Return tuple of two polynomials, which are split at the
  given degree. The term of the given degree, if exists, belongs to
  the lower degree polynomial.\\
  (This method is inherited from OrderProvider)  

 \subsection{DomainPolynomial -- polynomial over domain}\linkedone{poly.uniutil}{DomainPolynomial}
 \initialize
  \func{DomainPolynomial}{%
    \hiki{coefficients}{terminit},\ %
    \hiki{coeffring}{CommutativeRing},\ %
    **\hiki{keywords}{dict}}{\out{DomainPolynomial object}}\\
  \spacing
  % document of basic document
  \quad Initialize a polynomial over the given domain \param{coeffring}.\\
  \spacing
  % added document
  \quad In addition to the basic polynomial operations,
  it has pseudo division methods.\\
  \spacing
  \quad This class inherits \linkingone{poly.uniutil}{RingPolynomial} and
  \linkingone{poly.uniutil}{PseudoDivisionProvider}.
  \spacing
  % input, output document
  \quad The type of the \param{coefficients} is \linkingone{poly.formalsum}{terminit}.
  \param{coeffring} is an instance of descendant of \linkingone{ring}{CommutativeRing} which satisfies {\tt coeffring.isdomain()}.\\
  \spacing
  \method
  \subsubsection{pseudo\_divmod}\linkedtwo{poly.uniutil}{DomainPolynomial}{pseudo\_divmod}
  \func{pseudo\_divmod}{\param{self},\ \hiki{other}{polynomial}}{\out{tuple}}\\
  \spacing
  \quad Return a tuple {\tt (Q, R)},
  where \(Q\), \(R\) are polynomials such that:
  \[ d^{deg(f) - deg(other) + 1} f = other \times Q + R,\]
  where \(d\) is the leading coefficient of \param{other}.\\
  (This method is inherited from PseudoDivisionProvider)

  \subsubsection{pseudo\_floordiv}\linkedtwo{poly.uniutil}{DomainPolynomial}{pseudo\_floordiv}
  \func{pseudo\_floordiv}{\param{self},\ \hiki{other}{polynomial}}{\out{polynomial}}\\
  \spacing
  \quad Return a polynomial \(Q\) such that:
  \[ d^{deg(f) - deg(other) + 1} f = other \times Q + R,\]
  where \(d\) is the leading coefficient of \param{other}.\\
  (This method is inherited from PseudoDivisionProvider)

  \subsubsection{pseudo\_mod}\linkedtwo{poly.uniutil}{DomainPolynomial}{pseudo\_mod}
  \func{pseudo\_mod}{\param{self},\ \hiki{other}{polynomial}}{\out{polynomial}}\\
  \spacing
  \quad Return a polynomial \(R\) such that:
  \[ d^{deg(f) - deg(other) + 1} f = other \times Q + R,\]
  where \(d\) is the leading coefficient of \param{other}.\\
  (This method is inherited from PseudoDivisionProvider)

  \subsubsection{exact\_division}\linkedtwo{poly.uniutil}{DomainPolynomial}{exact\_division}
  \func{exact\_division}{\param{self},\ \hiki{other}{polynomial}}{\out{polynomial}}\\
  \spacing
  \quad Return quotient of exact division.\\
  (This method is inherited from PseudoDivisionProvider)

  \subsubsection{scalar\_exact\_division}\linkedtwo{poly.uniutil}{DomainPolynomial}{scalar\_exact\_division}
  \func{scalar\_exact\_division}{\param{self},\ \hiki{scale}{CommutativeRingElement}}{\out{polynomial}}\\
  \spacing
  \quad Return quotient by \param{scale} which can divide each
  coefficient exactly.\\
  (This method is inherited from PseudoDivisionProvider)

  \subsubsection{discriminant}\linkedtwo{poly.uniutil}{DomainPolynomial}{discriminant}
  \func{discriminant}{\param{self}}{\out{CommutativeRingElement}}\\
  \spacing
  \quad Return discriminant of the polynomial.

  \subsubsection{to\_field\_polynomial}\linkedtwo{poly.uniutil}{DomainPolynomial}{to\_field\_polynomial}
  \func{to\_field\_polynomial}{\param{self}}{\out{FieldPolynomial}}\\
  \spacing
  \quad Return a {\tt FieldPolynomial} object obtained by embedding
  the polynomial ring over the domain \(D\) to over the quotient
  field of \(D\).

 \subsection{UniqueFactorizationDomainPolynomial -- polynomial over UFD}\linkedone{poly.uniutil}{UniqueFactorizationDomainPolynomial}
 \initialize
  \func{UniqueFactorizationDomainPolynomial}{%
    \hiki{coefficients}{terminit},\ %
    \hiki{coeffring}{CommutativeRing},\ %
    **\hiki{keywords}{dict}}{\out{UniqueFactorizationDomainPolynomial object}}\\
  \spacing
  % document of basic document
  \quad Initialize a polynomial over the given UFD \param{coeffring}.\\
  \spacing
  % added document
  \quad This class inherits from \linkingone{poly.uniutil}{DomainPolynomial},
  \linkingone{poly.uniutil}{SubresultantGcdProvider} and \linkingone{poly.uniutil}{ContentProvider}.\\
  \spacing
  % input, output document
  \quad The type of the \param{coefficients} is \linkingone{poly.formalsum}{terminit}.
  \param{coeffring} is an instance of descendant of \linkingone{ring}{CommutativeRing} which satisfies {\tt coeffring.isufd()}.

  \subsubsection{content}\linkedtwo{poly.uniutil}{UniqueFactorizationDomainPolynomial}{content}
  \func{content}{\param{self}}{\out{CommutativeRingElement}}\\
  \spacing
  \quad Return content of the polynomial.\\
  (This method is inherited from ContentProvider)

  \subsubsection{primitive\_part}\linkedtwo{poly.uniutil}{UniqueFactorizationDomainPolynomial}{primitive\_part}
  \func{primitive\_part}{\param{self}}{\out{UniqueFactorizationDomainPolynomial}}\\
  \spacing
  \quad Return the primitive part of the polynomial.\\
  (This method is inherited from ContentProvider)

  \subsubsection{subresultant\_gcd}\linkedtwo{poly.uniutil}{UniqueFactorizationDomainPolynomial}{subresultant\_gcd}
  \func{subresultant\_gcd}{\param{self},\ \hiki{other}{polynomial}}{\out{UniqueFactorizationDomainPolynomial}}\\
  \spacing
  \quad Return the greatest common divisor of given polynomials.
  They must be in the polynomial ring and its coefficient ring must be a UFD.\\
  (This method is inherited from SubresultantGcdProvider)\\
  Reference: \cite{Cohen1}{Algorithm 3.3.1}

  \subsubsection{subresultant\_extgcd}\linkedtwo{poly.uniutil}{UniqueFactorizationDomainPolynomial}{subresultant\_extgcd}
  \func{subresultant\_extgcd}{\param{self},\ \hiki{other}{polynomial}}{\out{tuple}}\\
  \spacing
  \quad Return {\tt (A, B, P)} s.t. \(A\times self + B \times other=P\),
  where \(P\) is the greatest common divisor of given polynomials.
  They must be in the polynomial ring and its coefficient ring must be a UFD.\\
  Reference: \cite{Kida}{p.18}\\
  (This method is inherited from SubresultantGcdProvider)

  \subsubsection{resultant}\linkedtwo{poly.uniutil}{UniqueFactorizationDomainPolynomial}{resultant}
  \func{resultant}{\param{self}, \hiki{other}{polynomial}}{\out{polynomial}}\\
  \quad Return the resultant of \param{self} and \param{other}.\\
  (This method is inherited from SubresultantGcdProvider)

 \subsection{IntegerPolynomial -- polynomial over ring of rational integers}\linkedone{poly.uniutil}{IntegerPolynomial}
 \initialize
  \func{IntegerPolynomial}{%
    \hiki{coefficients}{terminit},\ %
    \hiki{coeffring}{CommutativeRing},\ %
    **\hiki{keywords}{dict}}{\out{IntegerPolynomial object}}\\
  \spacing
  % document of basic document
  \quad Initialize a polynomial over the given commutative ring \param{coeffring}.\\
  \spacing
  % added document
  \quad This class is required because special initialization must be
  done for built-in int/long.\\
  \spacing
  \quad This class inherits from \linkingone{poly.uniutil}{UniqueFactorizationDomainPolynomial}.\\
  \spacing
  % input, output document
  \quad The type of the \param{coefficients} is \linkingone{poly.formalsum}{terminit}.
  \param{coeffring} is an instance of \linkingone{rational}{IntegerRing}.
  You have to give the rational integer ring, though it seems redundant.

 \subsection{FieldPolynomial -- polynomial over field}\linkedone{poly.uniutil}{FieldPolynomial}
 \initialize
  \func{FieldPolynomial}{%
    \hiki{coefficients}{terminit},\ %
    \hiki{coeffring}{Field},\ %
    **\hiki{keywords}{dict}}{\out{FieldPolynomial object}}\\
  \spacing
  % document of basic document
  \quad Initialize a polynomial over the given field \param{coeffring}.\\
  \spacing
  % added document
  \quad Since the polynomial ring over field is a Euclidean domain,
  it provides divisions.\\
  \spacing
  \quad This class inherits from \linkingone{poly.uniutil}{RingPolynomial},
  \linkingone{poly.uniutil}{DivisionProvider} and \linkingone{poly.uniutil}{ContentProvider}.\\
  \spacing
  % input, output document
  \quad The type of the \param{coefficients} is \linkingone{poly.formalsum}{terminit}.
  \param{coeffring} is an instance of descendant of \linkingone{ring}{Field}.\\
%
  \begin{op}
    \verb+f // g+ & quotient of floor division\\
    \verb+f % g+ & remainder\\
    \verb+divmod(f, g)+ & quotient and remainder\\
    \verb+f / g+ & division in rational function field\\
  \end{op}
  \method

  \subsubsection{content}\linkedtwo{poly.uniutil}{FieldPolynomial}{content}
  \func{content}{\param{self}}{\out{FieldElement}}\\
  \spacing
  \quad Return content of the polynomial.\\
  (This method is inherited from ContentProvider)

  \subsubsection{primitive\_part}\linkedtwo{poly.uniutil}{FieldPolynomial}{primitive\_part}
  \func{primitive\_part}{\param{self}}{\out{polynomial}}\\
  \spacing
  \quad Return the primitive part of the polynomial.\\
  (This method is inherited from ContentProvider)

  \subsubsection{mod}\linkedtwo{poly.uniutil}{FieldPolynomial}{mod}
  \func{mod}{\param{self},\ \hiki{dividend}{polynomial}}{\out{polynomial}}\\
  \spacing
  \quad Return \(dividend \bmod self\).\\
  (This method is inherited from DivisionProvider)

  \subsubsection{scalar\_exact\_division}\linkedtwo{poly.uniutil}{FieldPolynomial}{scalar\_exact\_division}
  \func{scalar\_exact\_division}{\param{self},\ \hiki{scale}{FieldElement}}{\out{polynomial}}\\
  \spacing
  \quad Return quotient by \param{scale} which can divide each
  coefficient exactly.\\
  (This method is inherited from DivisionProvider)

  \subsubsection{gcd}\linkedtwo{poly.uniutil}{FieldPolynomial}{gcd}
  \func{gcd}{\param{self},\ \hiki{other}{polynomial}}{\out{polynomial}}\\
  \spacing
  \quad Return a greatest common divisor of self and other.\\
  \spacing
  \quad Returned polynomial is always monic.\\
  (This method is inherited from DivisionProvider)

  \subsubsection{extgcd}\linkedtwo{poly.uniutil}{FieldPolynomial}{extgcd}
  \func{extgcd}{\param{self},\ \hiki{other}{polynomial}}{\out{tuple}}\\
  \spacing
  \quad Return a tuple {\tt (u, v, d)}; they are the greatest common
  divisor \(d\) of two polynomials \param{self} and \param{other} and
  \(u\), \(v\) such that
  \[ d = self \times u + other \times v\]
  \spacing
  See \linkingone{gcd}{extgcd}.\\
  (This method is inherited from DivisionProvider)

 \subsection{FinitePrimeFieldPolynomial -- polynomial over finite prime field}\linkedone{poly.uniutil}{FinitePrimeFieldPolynomial}
 \initialize
  \func{FinitePrimeFieldPolynomial}{%
    \hiki{coefficients}{terminit},\ %
    \hiki{coeffring}{FinitePrimeField},\ %
    **\hiki{keywords}{dict}}{\out{FinitePrimeFieldPolynomial object}}\\
  \spacing
  % document of basic document
  \quad Initialize a polynomial over the given commutative ring \param{coeffring}.\\
  \spacing
  % added document
  \quad This class inherits from \linkingone{poly.uniutil}{FieldPolynomial} and
  \linkingone{poly.uniutil}{PrimeCharacteristicFunctionsProvider}.\\
  \spacing
  % input, output document
  \quad The type of the \param{coefficients} is \linkingone{poly.formalsum}{terminit}.
  \param{coeffring} is an instance of descendant of \linkingone{finitefield}{FinitePrimeField}.
  \method
  \subsubsection{mod\_pow -- powering with modulus}\linkedtwo{poly.uniutil}{FinitePrimeFieldPolynomial}{mod\_pow}
  \func{mod\_pow}{\param{self},\ \hiki{polynom}{polynomial}, \hiki{index}{integer}}{\out{polynomial}}\\
  \spacing
  \quad Return \(polynom ^ {index} \bmod self\).\\
  \spacing
  \quad Note that \param{self} is the modulus.\\
  (This method is inherited from PrimeCharacteristicFunctionsProvider)

  \subsubsection{pthroot}\linkedtwo{poly.uniutil}{FinitePrimeFieldPolynomial}{pthroot}
  \func{pthroot}{\param{self}}{\out{polynomial}}\\
  \spacing
  \quad Return a polynomial obtained by sending \(X^p\) to \(X\),
  where \(p\) is the characteristic. If the polynomial does not consist
  of \(p\)-th powered terms only, result is nonsense.\\
  (This method is inherited from PrimeCharacteristicFunctionsProvider)

  \subsubsection{squarefree\_decomposition}\linkedtwo{poly.uniutil}{FinitePrimeFieldPolynomial}{squarefree\_decomposition}
  \func{squarefree\_decomposition}{\param{self}}{\out{dict}}\\
  \spacing
  \quad Return the square free decomposition of the polynomial.\\
  \spacing
  \quad The return value is a dict whose keys are integers and values are
  corresponding powered factors.  For example, If
\begin{ex}
>>> A = A1 * A2**2
>>> A.squarefree_decomposition()
{1: A1, 2: A2}.
\end{ex}%Don't indent!
  (This method is inherited from PrimeCharacteristicFunctionsProvider)

  \subsubsection{distinct\_degree\_decomposition}\linkedtwo{poly.uniutil}{FinitePrimeFieldPolynomial}{distinct\_degree\_decomposition}
  \func{distinct\_degree\_decomposition}{\param{self}}{\out{dict}}\\
  \spacing
  \quad Return the distinct degree factorization of the polynomial.\\
  \spacing
  \quad The return value is a dict whose keys are integers and values are
  corresponding product of factors of the degree. For example, if
  \(A = A1 \times A2\), and all irreducible factors of \(A1\) having
  degree \(1\) and all irreducible factors of \(A2\) having degree \(2\),
  then the result is: {\tt \{1: A1, 2: A2\}}.\\

  \quad The given polynomial must be square free, and its coefficient
  ring must be a finite field.\\
  (This method is inherited from PrimeCharacteristicFunctionsProvider)

  \subsubsection{split\_same\_degrees}\linkedtwo{poly.uniutil}{FinitePrimeFieldPolynomial}{split\_same\_degrees}
  \func{split\_same\_degrees}{\param{self}, \hiki{degree}}{\out{list}}\\
  \spacing
  \quad Return the irreducible factors of the polynomial.\\
  \spacing
  \quad The polynomial must be a product of irreducible factors of
  the given degree.\\
  (This method is inherited from PrimeCharacteristicFunctionsProvider)

  \subsubsection{factor}\linkedtwo{poly.uniutil}{FinitePrimeFieldPolynomial}{factor}
  \func{factor}{\param{self}}{\out{list}}\\
  \spacing
  \quad Factor the polynomial.\\
  \spacing
  \quad The returned value is a list of tuples whose first component
  is a factor and second component is its multiplicity.\\
  (This method is inherited from PrimeCharacteristicFunctionsProvider)

  \subsubsection{isirreducible}\linkedtwo{poly.uniutil}{FinitePrimeFieldPolynomial}{isirreducible}
  \func{isirreducible}{\param{self}}{\out{bool}}\\
  \quad If the polynomial is irreducible return {\tt True},
  otherwise {\tt False}.\\
  (This method is inherited from PrimeCharacteristicFunctionsProvider)


  \subsection{polynomial -- factory function for various polynomials}\linkedone{poly.uniutil}{polynomial}
  \func{polynomial}{\hiki{coefficients}{terminit},\ \hiki{coeffring}{CommutativeRing}}{\out{polynomial}}\\
   \spacing
   % document of basic document
   \quad Return a polynomial.\\
   \spacing
   \quad \negok One can override the way to choose a polynomial type
   from a coefficient ring, by setting:\\
   {\tt special\_ring\_table[coeffring\_type] = polynomial\_type}\\
   before the function call.
\C

%---------- end document ---------- %

\bibliographystyle{jplain}%use jbibtex
\bibliography{nzmath_references}

\end{document}

