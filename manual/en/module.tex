\documentclass{report}

\documentclass{report}

%%%%%%%%%%%%%%%%%%%%%%%%%%%%%%%%%%%%%%%%%%%%%%%%%%%%%%%%%%%%%
%
% macros for nzmath manual
%
%%%%%%%%%%%%%%%%%%%%%%%%%%%%%%%%%%%%%%%%%%%%%%%%%%%%%%%%%%%%%
\usepackage{amssymb,amsmath}
\usepackage{color}
\usepackage[dvipdfm,bookmarks=true,bookmarksnumbered=true,%
 pdftitle={NZMATH Users Manual},%
 pdfsubject={Manual for NZMATH Users},%
 pdfauthor={NZMATH Development Group},%
 pdfkeywords={TeX; dvipdfmx; hyperref; color;},%
 colorlinks=true]{hyperref}
\usepackage{fancybox}
\usepackage[T1]{fontenc}
%
\newcommand{\DS}{\displaystyle}
\newcommand{\C}{\clearpage}
\newcommand{\NO}{\noindent}
\newcommand{\negok}{$\dagger$}
\newcommand{\spacing}{\vspace{1pt}\\ }
% software macros
\newcommand{\nzmathzero}{{\footnotesize $\mathbb{N}\mathbb{Z}$}\texttt{MATH}}
\newcommand{\nzmath}{{\nzmathzero}\ }
\newcommand{\pythonzero}{$\mbox{\texttt{Python}}$}
\newcommand{\python}{{\pythonzero}\ }
% link macros
\newcommand{\linkingzero}[1]{{\bf \hyperlink{#1}{#1}}}%module
\newcommand{\linkingone}[2]{{\bf \hyperlink{#1.#2}{#2}}}%module,class/function etc.
\newcommand{\linkingtwo}[3]{{\bf \hyperlink{#1.#2.#3}{#3}}}%module,class,method
\newcommand{\linkedzero}[1]{\hypertarget{#1}{}}
\newcommand{\linkedone}[2]{\hypertarget{#1.#2}{}}
\newcommand{\linkedtwo}[3]{\hypertarget{#1.#2.#3}{}}
\newcommand{\linktutorial}[1]{\href{http://docs.python.org/tutorial/#1}{#1}}
\newcommand{\linktutorialone}[2]{\href{http://docs.python.org/tutorial/#1}{#2}}
\newcommand{\linklibrary}[1]{\href{http://docs.python.org/library/#1}{#1}}
\newcommand{\linklibraryone}[2]{\href{http://docs.python.org/library/#1}{#2}}
\newcommand{\pythonhp}{\href{http://www.python.org/}{\python website}}
\newcommand{\nzmathwiki}{\href{http://nzmath.sourceforge.net/wiki/}{{\nzmathzero}Wiki}}
\newcommand{\nzmathsf}{\href{http://sourceforge.net/projects/nzmath/}{\nzmath Project Page}}
\newcommand{\nzmathtnt}{\href{http://tnt.math.se.tmu.ac.jp/nzmath/}{\nzmath Project Official Page}}
% parameter name
\newcommand{\param}[1]{{\tt #1}}
% function macros
\newcommand{\hiki}[2]{{\tt #1}:\ {\em #2}}
\newcommand{\hikiopt}[3]{{\tt #1}:\ {\em #2}=#3}

\newdimen\hoge
\newdimen\truetextwidth
\newcommand{\func}[3]{%
\setbox0\hbox{#1(#2)}
\hoge=\wd0
\truetextwidth=\textwidth
\advance \truetextwidth by -2\oddsidemargin
\ifdim\hoge<\truetextwidth % short form
{\bf \colorbox{skyyellow}{#1(#2)\ $\to$ #3}}
%
\else % long form
\fcolorbox{skyyellow}{skyyellow}{%
   \begin{minipage}{\textwidth}%
   {\bf #1(#2)\\ %
    \qquad\quad   $\to$\ #3}%
   \end{minipage}%
   }%
\fi%
}

\newcommand{\out}[1]{{\em #1}}
\newcommand{\initialize}{%
  \paragraph{\large \colorbox{skyblue}{Initialize (Constructor)}}%
    \quad\\ %
    \vspace{3pt}\\
}
\newcommand{\method}{\C \paragraph{\large \colorbox{skyblue}{Methods}}}
% Attribute environment
\newenvironment{at}
{%begin
\paragraph{\large \colorbox{skyblue}{Attribute}}
\quad\\
\begin{description}
}%
{%end
\end{description}
}
% Operation environment
\newenvironment{op}
{%begin
\paragraph{\large \colorbox{skyblue}{Operations}}
\quad\\
\begin{table}[h]
\begin{center}
\begin{tabular}{|l|l|}
\hline
operator & explanation\\
\hline
}%
{%end
\hline
\end{tabular}
\end{center}
\end{table}
}
% Examples environment
\newenvironment{ex}%
{%begin
\paragraph{\large \colorbox{skyblue}{Examples}}
\VerbatimEnvironment
\renewcommand{\EveryVerbatim}{\fontencoding{OT1}\selectfont}
\begin{quote}
\begin{Verbatim}
}%
{%end
\end{Verbatim}
\end{quote}
}
%
\definecolor{skyblue}{cmyk}{0.2, 0, 0.1, 0}
\definecolor{skyyellow}{cmyk}{0.1, 0.1, 0.5, 0}
%
%\title{NZMATH User Manual\\ {\large{(for version 1.0)}}}
%\date{}
%\author{}
\begin{document}
%\maketitle
%
\setcounter{tocdepth}{3}
\setcounter{secnumdepth}{3}


\tableofcontents
\C

\chapter{Functions}


%---------- start document ---------- %
 \section{module -- module/ideal with HNF}\linkedzero{module}
 \begin{itemize}
   \item {\bf Classes}
   \begin{itemize}
     \item \linkingone{module}{Submodule}
     \item \linkingone{module}{Module}
     \item \linkingone{module}{Ideal}
     \item \linkingone{module}{Ideal\_with\_generator}
   \end{itemize}
 \end{itemize}

\C

 \subsection{Submodule -- submodule as matrix representation}\linkedone{module}{Submodule}
 \initialize
  \func{Submodule}{\hiki{row}{integer},\ \hiki{column}{integer},\ \hikiopt{compo}{compo}{0},\ \hikiopt{coeff\_ring}{CommutativeRing}{0},\ \hikiopt{ishnf}{True/False}{None}}{\out{Submodule}}\\
  \spacing
  % document of basic document
  \quad Create a submodule with matrix representation.\\
  \spacing
  % added document
  \quad Submodule is subclass of \linkingone{matrix}{RingMatrix}.\\
  We assume that \param{coeff\_ring} is a PID (principal ideal domain).
  Then, we have the HNF(hermite normal form) corresponding to a matrices.\\
  \spacing
  %
  % input, output document
  If \param{ishnf} is True, we assume that the input matrix is a HNF.
  \begin{at}
   \item[ishnf] If the matrix is a HNF, then \param{ishnf} should be True, otherwise False.
  \end{at}
  \method
  \subsubsection{getGenerators -- generator of module}\linkedtwo{module}{Submodule}{getGenerators}
   \func{getGenerators}{\param{self}}{\out{list}}\\
   \spacing
   % document of basic document
   \quad Return a (current) generator of the module \param{self}.\\
   \spacing
   % added document
   %
   % input, output document
   \quad Return the list of vectors consisting of a generator.\\
 \subsubsection{isSubmodule -- Check whether submodule of self}\linkedtwo{module}{Submodule}{isSubmodule}
   \func{isSubmodule}{\param{self},\ \hiki{other}{Submodule}}{\out{True/False}}\\
   \spacing
   % document of basic document
   \quad Return True if the submodule instance is a submodule of the \param{other}, or False otherwise.\\
   \spacing
   % added document
   %
   %\spacing
   % input, output document
   %
   \subsubsection{isEqual -- Check whether self and other are same module}\linkedtwo{module}{Submodule}{isEqual}
   \func{isEqual}{\param{self},\ \hiki{other}{Submodule}}{\out{True/False}}\\
   \spacing
   % document of basic document
   \quad Return True if the submodule instance is \param{other} as module, or False otherwise.\\.\\
   \spacing
   % added document
   \quad You should use the method for equality test of module, not matrix.
   For equality test of matrix simply, use \param{self}$==$\param{other}.
   \\
   % input, output document
   %
   \subsubsection{isContain -- Check whether other is in self}\linkedtwo{module}{Submodule}{isContains}
   \func{isContains}{\param{self},\ \hiki{other}{vector.Vector}}{\out{True/False}}\\
   \spacing
   % document of basic document
   \quad Determine whether \param{other} is in \param{self} or not.\\.\\
   \spacing
   % added document
   \quad If you want to represent \param{other} as linear combination with the HNF generator of \param{self}, use \linkingtwo{module}{Submodule}{represent\_element}.
   \\
   % input, output document
   %
   \subsubsection{toHNF - change to HNF}\linkedtwo{matrix}{Submodule}{toHNF}
   \func{toHNF}{\param{self}}{\out{(None)}}\\
   \spacing
   % document of basic document
   \quad Rewrite \param{self} to HNF (hermite normal form), and set True to its \param{ishnf}.\\
   \spacing
   % added document
   \quad Note that HNF do not always give basis of \param{self}.(i.e. HNF may be redundant.)\\
   \spacing
   % input, output document
   %\quad
  \subsubsection{sumOfSubmodules - sum as submodule}\linkedtwo{matrix}{Submodule}{sumOfSubmodules}
   \func{sumOfSubmodules}{\param{self},\ \hiki{other}{Submodule}}{\out{Submodule}}\\
   \spacing
   % document of basic document
   \quad Return a module which is sum of two subspaces.\\
   \spacing
   % added document
   %\quad
   %\spacing
   % input, output document
   %\quad
  \subsubsection{intersectionOfSubmodules - intersection as submodule}\linkedtwo{matrix}{Submodule}{intersectionOfSubmodules}
   \func{intersectionOfSubmodules}{\param{self},\ \hiki{other}{Submodule}}{\out{Submodule}}\\
   \spacing
   % document of basic document
   \quad Return a module which is intersection of two subspaces.\\
   \spacing
   % added document
   %\quad
   %\spacing
   % input, output document
   %\quad
 \subsubsection{represent\_element -- represent element as linear combination}\linkedtwo{module}{Submodule}{represent\_element}
   \func{represent\_element}{\param{self},\ \hiki{other}{vector.Vector}}{\out{vector.Vector/False}}\\
   \spacing
   % document of basic document
   \quad Represent \param{other} as a linear combination with HNF generators.\\
   \spacing
   % added document
   \quad If \param{other} not in \param{self}, return False.
   Note that this method calls \linkingtwo{module}{Submodule}{toHNF}.\\
   \spacing
   % input, output document
   \quad The method returns coefficients as an instance of \linkingone{vector}{Vector}.
 \subsubsection{linear\_combination -- compute linear combination}\linkedtwo{module}{Submodule}{linear\_combination}
   \func{linear\_combination}{\param{self},\ \hiki{coeff}{list}}{\out{vector.Vector}}\\
   \spacing
   % document of basic document
   \quad For given $\mathbf{Z}$-coefficients \param{coeff}, 
         return a vector corresponding to a linear combination of (current) basis.\\
   \spacing
   % added document
   %
   % input, output document
   \quad \param{coeff} must be a list of instances in \linkingone{ring}{RingElement} whose size is the column of \param{self}.
\begin{ex}
>>> A = module.Submodule(4, 3, [1,2,3]+[4,5,6]+[7,8,9]+[10,11,12])
>>> A.toHNF()
>>> print A
9 1
6 1
3 1
0 1
>>> A.getGenerator
[Vector([9L, 6L, 3L, 0L]), Vector([1L, 1L, 1L, 1L])]
>>> V = vector.Vector([10,7,4,1])
>>> A.represent_element(V)
Vector([1L, 1L])
>>> V == A.linear_combination([1,1])
True
>>> B = module.Submodule(4, 1, [1,2,3,4])
>>> C = module.Submodule(4, 2, [2,-4]+[4,-3]+[6,-2]+[8,-1])
>>> print B.intersectionOfSubmodules(C)
2
4
6
8
\end{ex}%Don't indent!
\C
  \subsection{fromMatrix(class function) - create submodule}\linkedtwo{module}{Submodule}{fromMatrix}
  \func{fromMatrix}{\param{cls},\ \hiki{mat}{RingMatrix},\ \hikiopt{ishnf}{True/False}{None}}{\out{Submodule}}\\
   \spacing
   % document of basic document
   \quad Create a Submodule instance from a matrix instance \param{mat}, whose class can be any of subclasses of Matrix.\\
   \spacing
   % added document
   \quad Please use this method if you want a Submodule instance for sure.\\
   \spacing
   % input, output document
   %
\C

\subsection{Module - module over a number field}\linkedone{module}{Module}
 \initialize
  \func{Module}{\hiki{pair\_mat\_repr}{list/matrix},\ \hiki{number\_field}{algfield.NumberField},\ \hikiopt{base}{list/matrix.SquareMatrix}{None},\ \hikiopt{ishnf}{bool}{False}}{\out{Module}}\\
  \spacing
  % document of basic document
  \quad Create a new module object over a number field.\\
  \spacing
  % added document
  \quad A module is a finitely generated sub $\mathbf{Z}$-module. 
  Note that we do not assume rank of a module is deg(number\_field).\\
  We represent a module as generators respect to base module over $\mathbf{Z}[\theta]$, where $\theta$ is a solution of \param{number\_field}.\linkingtwo{algfield}{BasicAlgNumber}{polynomial}.\\
  \spacing
  % input, output document
  \quad \param{pair\_mat\_repr} should be one of the following form:
  \begin{itemize}
  \item $[M,\ d]$, 
   where $M$ is a list of integral tuple/vectors whose size is the degree of \param{number\_field} and
         $d$ is a denominator.
  \item $[M,\ d]$,
   where $M$ is an integral matrix whose the number of row is the degree of \param{number\_field} and
         $d$ is a denominator.
  \item a rational matrix whose the number of row is the degree of \param{number\_field}.
  \end{itemize}
  Also, \param{base} should be one of the following form:
  \begin{itemize}
  \item \linkedone{module}{base} a list of rational tuple/vectors whose size is the degree of \param{number\_field}
  \item a square non-singular rational matrix whose size is the degree of \param{number\_field}
  \end{itemize}
  The module is internally represented as $\frac{1}{d}M$ with respect to \linkingtwo{module}{Module}{base},
  where $d$ is \linkingtwo{module}{Module}{denominator} and $M$ is \linkingtwo{module}{Module}{mat\_repr}.
  If \param{ishnf} is True, we assume that \param{mat\_repr} is a HNF.\\
  \begin{at}
  \item[mat\_repr]\linkedtwo{module}{Module}{mat\_repr}: an instance of \linkingone{module}{Submodule} $M$ whose size is the degree of \param{number\_field}
  \item[denominator]\linkedtwo{module}{Module}{denominator}: an integer $d$
  \item[base]\linkedtwo{module}{Module}{base}: a square non-singular rational matrix whose size is the degree of \param{number\_field}
  \item[number\_field]\linkedtwo{module}{Module}{number\_field}: the number field over which the module is defined
  \end{at}
\begin{op}
    \verb+M==N+ & Return whether \param{M} and \param{N} are equal or not as module.\\
    \verb+c in M+ & Check whether some element of \param{M} equals \param{c}.\\
    \verb|M+N| & Return the sum of \param{M} and \param{N} as module.\\
    \verb+M*N+ & Return the product of \param{M} and \param{N} as the ideal computation. \\
               &  \param{N} must be module or scalar(i.e. an element of \linkingtwo{module}{Module}{number\_field}).\\
               &  If you want to compute the intersection of $M$ and $N$, see \linkingtwo{module}{Module}{intersect}.\\
    \verb+M**c+ & Return \param{M} to \param{c} based on the ideal multiplication.\\
    \verb+repr(M)+ & Return the repr string of the module \param{M}.\\
    \verb+str(M)+ & Return the str string of the module \param{M}.\\
  \end{op}
\begin{ex}
>>> F = algfield.NumberField([2,0,1])
>>> M_1 = module.Module([matrix.RingMatrix(2,2,[1,0]+[0,2]), 2], F)
>>> M_2 = module.Module([matrix.RingMatrix(2,2,[2,0]+[0,5]), 3], F)
>>> print M_1
([1, 0]+[0, 2], 2)
 over
([1L, 0L]+[0L, 1L], NumberField([2, 0, 1]))
>>> print M_1 + M_2
([1L, 0L]+[0L, 2L], 6)
 over
([Rational(1, 1), Rational(0, 1)]+[Rational(0, 1), Rational(1, 1)], 
NumberField([2, 0, 1]))
>>> print M_1 * 2
([1L, 0L]+[0L, 2L], 1L)
 over
([Rational(1, 1), Rational(0, 1)]+[Rational(0, 1), Rational(1, 1)], 
NumberField([2, 0, 1]))
>>> print M_1 * M_2
([2L, 0L]+[0L, 1L], 6L)
 over
([Rational(1, 1), Rational(0, 1)]+[Rational(0, 1), Rational(1, 1)], 
NumberField([2, 0, 1]))
>>> print M_1 ** 2
([1L, 0L]+[0L, 2L], 4L)
 over
([Rational(1, 1), Rational(0, 1)]+[Rational(0, 1), Rational(1, 1)], 
NumberField([2, 0, 1]))
\end{ex}%Don't indent!
\method
  \subsubsection{toHNF - change to hermite normal form(HNF)}\linkedtwo{module}{Module}{toHNF}
   \func{toHNF}{\param{self}}{\out{(None)}}\\
   \spacing
   % document of basic document
   \quad Change \param{self}.\linkingtwo{module}{Module}{mat\_repr} to the hermite normal form(HNF).\\
   \spacing
   % added document
   %\quad
   %\spacing
   % input, output document
   %\quad
  \subsubsection{copy - create copy}\linkedtwo{module}{Module}{copy}
   \func{copy}{\param{self}}{\out{Module}}\\
   \spacing
   % document of basic document
   \quad Create copy of \param{self}.\\
   \spacing
   % added document
   %\quad
   %\spacing
   % input, output document
   %\quad
  \subsubsection{intersect - intersection}\linkedtwo{module}{Module}{intersect}
   \func{intersect}{\param{self},\ \hiki{other}{Module}}{\out{Module}}\\
   \spacing
   % document of basic document
   \quad Return intersection of \param{self} and \param{other}.\\
   \spacing
   % added document
   %\quad
   %\spacing
   % input, output document
   %\quad 
  \subsubsection{issubmodule - Check submodule}\linkedtwo{module}{Module}{issubmodule}
   \func{submodule}{\param{self},\ \hiki{other}{Module}}{\out{True/False}}\\
   \spacing
   % document of basic document
   \quad Check \param{self} is submodule of \param{other}.\\
   \spacing
   % added document
   %\quad
   %\spacing
   % input, output document
   %\quad 
  \subsubsection{issupermodule - Check supermodule}\linkedtwo{module}{Module}{issupermodule}
   \func{supermodule}{\param{self},\ \hiki{other}{Module}}{\out{True/False}}\\
   \spacing
   % document of basic document
   \quad Check \param{self} is supermodule of \param{other}.\\
   \spacing
   % added document
   %\quad
   %\spacing
   % input, output document
   %\quad
  \subsubsection{represent\_element - Represent as linear combination}\linkedtwo{module}{Module}{represent\_element}
   \func{represent\_element}{\param{self},\ \hiki{other}{algfield.BasicAlgNumber}}{\out{list/False}}\\
   \spacing
   % document of basic document
   \quad Represent \param{other} as a linear combination with generators of \param{self}.
        If \param{other} is not in \param{self}, return False.\\
   \spacing
   % added document
   \quad Note that we do not assume \param{self}.\linkingtwo{module}{Module}{mat\_repr} is HNF.
   \spacing
   % input, output document
   \quad
   The output is a list of integers if \param{other} is in \param{self}.\\
   \spacing
  \subsubsection{change\_base\_module - Change base}\linkedtwo{module}{Module}{change\_base\_module}
   \func{change\_base\_module}{\param{self},\ \hiki{other\_base}{list/matrix.RingSquareMatrix}}{\out{Module}}\\
   \spacing
   % document of basic document
   \quad Return the module which is equal to \param{self} respect to \param{other\_base}.\\
   \spacing
   % added document
   %\quad
   %\spacing
   % input, output document
   \quad
   \param{other\_base} follows the form \linkingone{module}{base}.\\
  \subsubsection{index - size of module}\linkedtwo{module}{Module}{index}
   \func{index}{\param{self}}{\out{rational.Rational}}\\
   \spacing
   % document of basic document
   \quad Return the order of a residue group over \param{self}.\linkingtwo{module}{Module}{base}.
         That is, return $[M:N]$ if $N \subset M$ or ${[N:M]}^{-1}$ if $M subset N$,
         where $M$ is the module \param{self} and $N$ is the module corresponding to \param{self}.\linkingtwo{module}{Module}{base}. \\
   \spacing
   % added document
   %\quad
   %\spacing
   % input, output document
   %\quad
  \subsubsection{smallest\_rational - a $\mathbf{Z}$-generator in the rational field}\linkedtwo{module}{Module}{smallest\_rational}
   \func{smallest\_rational}{\param{self}}{\out{rational.Rational}}\\
   \spacing
   % document of basic document
   \quad Return the $\mathbf{Z}$-generator of intersection of the module \param{self} and the rational field.\\
   \spacing
   % added document
   %\quad
   %\spacing
   % input, output document
   %\quad
 \begin{ex}
>>> F = algfield.NumberField([1,0,2])
>>> M_1=module.Module([matrix.RingMatrix(2,2,[1,0]+[0,2]), 2], F)
>>> M_2=module.Module([matrix.RingMatrix(2,2,[2,0]+[0,5]), 3], F)
>>> print M_1.intersect(M_2)
([2L, 0L]+[0L, 5L], 1L)
 over
([Rational(1, 1), Rational(0, 1)]+[Rational(0, 1), Rational(1, 1)],
 NumberField([2, 0, 1]))
>>> M_1.represent_element( F.createElement( [[2,4], 1] ) )
[4L, 4L]
>>> print M_1.change_base_module( matrix.FieldSquareMatrix(2, 2, [1,0]+[0,1]) / 2 )
([1L, 0L]+[0L, 2L], 1L)
 over
([Rational(1, 2), Rational(0, 1)]+[Rational(0, 1), Rational(1, 2)],
 NumberField([2, 0, 1]))
>>> M_2.index()
Rational(10, 9)
>>> M_2.smallest_rational()
Rational(2, 3)
\end{ex}%Don't indent!
\C

\subsection{Ideal - ideal over a number field}\linkedone{module}{Ideal}
 \initialize
  \func{Ideal}{\hiki{pair\_mat\_repr}{list/matrix},\ \hiki{number\_field}{algfield.NumberField},\ \hikiopt{base}{list/matrix.SquareMatrix}{None},\ \hikiopt{ishnf}{bool}{False}}{\out{Ideal}}\\
  \spacing
  % document of basic document
  \quad Create a new ideal object over a number field.\\
  \spacing
  % added document
  \quad Ideal is subclass of \linkingone{module}{Module}.\\
  \spacing
  % input, output document
  \quad Refer to initialization of \linkingone{module}{Module}.\\
%\begin{ex}
%\end{ex}%Don't indent!
\C
\method

\subsubsection{inverse -- inverse}\linkedtwo{module}{Ideal}{inverse}
   \func{inverse}{\param{self}}{\out{Ideal}}\\
   \spacing
   % document of basic document
   \quad Return the inverse ideal of \param{self}.\\
   \spacing
   % added document
   \quad This method calls \param{self}.\linkingtwo{module}{Module}{number\_field}.\linkingtwo{algfield}{NumberField}{integer\_ring}.\\
   \spacing
   % input, output document
   %\quad \\
 \subsubsection{issubideal -- Check subideal}\linkedtwo{module}{Ideal}{issubideal}
   \func{issubideal}{\param{self},\ \hiki{other}{Ideal}}{\out{Ideal}}\\
   \spacing
   % document of basic document
   \quad Check \param{self} is subideal of \param{other}.\\
   \spacing
   % added document
   %\quad 
   %\spacing
   % input, output document
   %\quad \\
  \subsubsection{issuperideal -- Check superideal}\linkedtwo{module}{Ideal}{issuoerideal}
   \func{issuperideal}{\param{self},\ \hiki{other}{Ideal}}{\out{Ideal}}\\
   \spacing
   % document of basic document
   \quad Check \param{self} is superideal of \param{other}.\\
   \spacing
   % added document
   %\quad 
   %\spacing
   % input, output document
   %\quad \\
  \subsubsection{gcd -- greatest common divisor}\linkedtwo{module}{Ideal}{gcd}
   \func{gcd}{\param{self},\ \hiki{other}{Ideal}}{\out{Ideal}}\\
   \spacing
   % document of basic document
   \quad Return the greatest common divisor(gcd) of \param{self} and \param{other} as ideal.\\
   \spacing
   % added document
   \quad This method simply executes \param{self}$+$\param{other}.\\
   \spacing
   % input, output document
   %\quad \\
  \subsubsection{lcm -- least common multiplier}\linkedtwo{module}{Ideal}{lcm}
   \func{lcm}{\param{self},\ \hiki{other}{Ideal}}{\out{Ideal}}\\
   \spacing
   % document of basic document
   \quad Return the least common multiplier(lcm) of \param{self} and \param{other} as ideal.\\
   \spacing
   % added document
   \quad This method simply calls the method \linkingtwo{module}{Module}{intersect}.\\
   \spacing
   % input, output document
   %\quad \\
%  \subsubsection{twoElementRepresentation -- Represent as two element}\linkedtwo{module}{Ideal}{twoElementRepresentation}
%   \func{twoElementRepresentation}{\param{self}}{\out{Ideal}}\\
%   \spacing
%   % document of basic document
%   \quad Return the ideal which is \param{self} represented with only two elements.\\
%   \spacing
%   % added document
%   %\quad \\
%   %\spacing
%   % input, output document
%   %\quad \\
  \subsubsection{norm -- norm}\linkedtwo{module}{Ideal}{norm}
   \func{norm}{\param{self}}{\out{rational.Rational}}\\
   \spacing
   % document of basic document
   \quad Return the norm of \param{self}.\\
   \spacing
   % added document
   \quad This method calls \param{self}.\linkingtwo{module}{Module}{number\_field}.\linkingtwo{algfield}{NumberField}{integer\_ring}.\\
   \spacing
   % input, output document
   %\quad \\
  \subsubsection{isIntegral -- Check integral}\linkedtwo{module}{Ideal}{isIntegral}
   \func{isIntegral}{\param{self}}{\out{True/False}}\\
   \spacing
   % document of basic document
   \quad Determine whether \param{self} is an integral ideal or not.\\
   \spacing
   % added document
   %\quad \\
   %\spacing
   % input, output document
   %\quad \\
%  \subsubsection{isPrime -- Check primality}\linkedtwo{module}{Ideal}{isPrime}
%   \func{isPrime}{\param{self}}{\out{True/False}}\\
%   \spacing
%   % document of basic document
%   \quad Determine whether \param{self} is a prime ideal or not.\\
%   \spacing
%   % added document
%   %\quad \\
%   %\spacing
%   % input, output document
%   %\quad \\
\begin{ex}
>>> M = module.Ideal([matrix.RingMatrix(2, 2, [1,0]+[0,2]), 2], F)
>>> print M.inverse()
([-2L, 0L]+[0L, 2L], 1L)
 over
([Rational(1, 1), Rational(0, 1)]+[Rational(0, 1), Rational(1, 1)],
 NumberField([2, 0, 1]))
>>> print M * M.inverse()
([1L, 0L]+[0L, 1L], 1L)
 over
([Rational(1, 1), Rational(0, 1)]+[Rational(0, 1), Rational(1, 1)],
 NumberField([2, 0, 1]))
>>> M.norm()
Rational(1, 2)
>>> M.isIntegral()
False
\end{ex}%Don't indent!
\C

\subsection{Ideal\_with\_generator - ideal with generator}\linkedone{module}{Ideal\_with\_generator}
 \initialize
  \func{Ideal\_with\_generator}{\hiki{generator}{list}}{\out{Ideal\_with\_generator}}\\
  \spacing
  % document of basic document
  \quad Create a new ideal given as a generator.\\
  \spacing
  % added document
  %\quad 
  %\spacing
  % input, output document
  \quad \param{generator} is a list of instances in \linkingone{algfield}{BasicAlgNumber}, which represent generators, over a same number field.
  \begin{at}
  \item[generator]\linkedtwo{module}{Ideal\_with\_generator}{generator}: generators of the ideal
  \item[number\_field]\linkedtwo{module}{Ideal\_with\_generator}{number\_field}: the number field over which generators are defined
  \end{at}
\begin{op}
    \verb+M==N+ & Return whether \param{M} and \param{N} are equal or not as module.\\
    \verb+c in M+ & Check whether some element of \param{M} equals \param{c}.\\
    \verb|M+N| & Return the sum of \param{M} and \param{N} as ideal with generators.\\
    \verb+M*N+ & Return the product of \param{M} and \param{N} as ideal with generators. \\
    \verb+M**c+ & Return \param{M} to \param{c} based on the ideal multiplication.\\
    \verb+repr(M)+ & Return the repr string of the ideal \param{M}.\\
    \verb+str(M)+ & Return the str string of the ideal \param{M}.\\
  \end{op}
\begin{ex}
>>> F = algfield.NumberField([2,0,1])
>>> M_1 = module.Ideal_with_generator([
 F.createElement([[1,0], 2]), F.createElement([[0,1], 1]) 
])
>>> M_2 = module.Ideal_with_generator([
 F.createElement([[2,0], 3]), F.createElement([[0,5], 3]) 
])
>>> print M_1
[BasicAlgNumber([[1, 0], 2], [2, 0, 1]), BasicAlgNumber([[0, 1], 1], [2, 0, 1])]
>>> print M_1 + M_2
[BasicAlgNumber([[1, 0], 2], [2, 0, 1]), BasicAlgNumber([[0, 1], 1], [2, 0, 1]),
 BasicAlgNumber([[2, 0], 3], [2, 0, 1]), BasicAlgNumber([[0, 5], 3], [2, 0, 1])]
>>> print M_1 * M_2
[BasicAlgNumber([[1L, 0L], 3L], [2, 0, 1]), BasicAlgNumber([[0L, 5L], 6], [2, 0, 1]), 
BasicAlgNumber([[0L, 2L], 3], [2, 0, 1]), BasicAlgNumber([[-10L, 0L], 3], [2, 0, 1])]
>>> print M_1 ** 2
[BasicAlgNumber([[1L, 0L], 4], [2, 0, 1]), BasicAlgNumber([[0L, 1L], 2], [2, 0, 1]), 
BasicAlgNumber([[0L, 1L], 2], [2, 0, 1]), BasicAlgNumber([[-2L, 0L], 1], [2, 0, 1])]
\end{ex}%Don't indent!
\method
  \subsubsection{copy - create copy}\linkedtwo{module}{Ideal\_with\_generator}{copy}
   \func{copy}{\param{self}}{\out{Ideal\_with\_generator}}\\
   \spacing
   % document of basic document
   \quad Create copy of \param{self}.\\
   \spacing
   % added document
   %\quad
   %\spacing
   % input, output document
   %\quad
  \subsubsection{to\_HNFRepresentation - change to ideal with HNF}\linkedtwo{module}{Ideal\_with\_generator}{to\_HNFRepresentation}
   \func{to\_HNFRepresentation}{\param{self}}{\out{Ideal}}\\
   \spacing
   % document of basic document
   \quad Transform \param{self} to the corresponding ideal as HNF(hermite normal form) representation.\\
   \spacing
   % added document
   %\quad
   %\spacing
   % input, output document
   %\quad
  \subsubsection{twoElementRepresentation - Represent with two element}\linkedtwo{module}{Ideal\_with\_generator}{twoElementRepresentation}
   \func{twoElementRepresentation}{\param{self}}{\out{Ideal\_with\_generator}}\\
   \spacing
   % document of basic document
   \quad Transform \param{self} to the corresponding ideal as HNF(hermite normal form) representation.\\
   \spacing
   % added document
   \quad  If \param{self} is not a prime ideal, this method is not efficient.\\
   \spacing 
   % input, output document
   %\quad
  \subsubsection{smallest\_rational - a $\mathbf{Z}$-generator in the rational field}\linkedtwo{module}{Ideal\_with\_generator}{smallest\_rational}
   \func{smallest\_rational}{\param{self}}{\out{rational.Rational}}\\
   \spacing
   % document of basic document
   \quad Return the $\mathbf{Z}$-generator of intersection of the module \param{self} and the rational field.\\
   \spacing
   % added document
   \quad This method calls \linkingtwo{module}{Ideal\_with\_generator}{to\_HNFRepresentation}.\\
   \spacing
   % input, output document
   %\quad
\subsubsection{inverse -- inverse}\linkedtwo{module}{Ideal\_with\_generator}{inverse}
   \func{inverse}{\param{self}}{\out{Ideal}}\\
   \spacing
   % document of basic document
   \quad Return the inverse ideal of \param{self}.\\
   \spacing
   % added document
   \quad This method calls \linkingtwo{module}{Ideal\_with\_generator}{to\_HNFRepresentation}.\\
   \spacing
   % input, output document
   %\quad \\
  \subsubsection{norm -- norm}\linkedtwo{module}{Ideal\_with\_generator}{norm}
   \func{norm}{\param{self}}{\out{rational.Rational}}\\
   \spacing
   % document of basic document
   \quad Return the norm of \param{self}.\\
   \spacing
   % added document
   \quad This method calls \linkingtwo{module}{Ideal\_with\_generator}{to\_HNFRepresentation}.\\
   \spacing
   % input, output document
   %\quad \\
  \subsubsection{intersect - intersection}\linkedtwo{module}{Ideal\_with\_generator}{intersection}
   \func{intersect}{\param{self},\ \hiki{other}{Ideal\_with\_generator}}{\out{Ideal}}\\
   \spacing
   % document of basic document
   \quad Return intersection of \param{self} and \param{other}.\\
   \spacing
   % added document
   \quad This method calls \linkingtwo{module}{Ideal\_with\_generator}{to\_HNFRepresentation}.\\
   \spacing
   % input, output document
   %\quad
 \subsubsection{issubideal -- Check subideal}\linkedtwo{module}{Ideal\_with\_generator}{issubideal}
   \func{issubideal}{\param{self},\ \hiki{other}{Ideal\_with\_generator}}{\out{Ideal}}\\
   \spacing
   % document of basic document
   \quad Check \param{self} is subideal of \param{other}.\\
   \spacing
   % added document
   \quad This method calls \linkingtwo{module}{Ideal\_with\_generator}{to\_HNFRepresentation}.\\
   \spacing
   % input, output document
   %\quad \\
  \subsubsection{issuperideal -- Check superideal}\linkedtwo{module}{Ideal\_with\_generator}{issuoerideal}
   \func{issuperideal}{\param{self},\ \hiki{other}{Ideal\_with\_generator}}{\out{Ideal}}\\
   \spacing
   % document of basic document
   \quad This method calls \linkingtwo{module}{Ideal\_with\_generator}{to\_HNFRepresentation}.\\
   \spacing
   % added document
   %\quad 
   %\spacing
   % input, output document
   %\quad \\
\begin{ex}
>>> M = module.Ideal_with_generator([
F.createElement([[2,0], 3]), F.createElement([[0,2], 3]), F.createElement([[1,0], 3])
])
>>> print M.to_HNFRepresentation()
([2L, 0L, 0L, -4L, 1L, 0L]+[0L, 2L, 2L, 0L, 0L, 1L], 3L)
 over
([1L, 0L]+[0L, 1L], NumberField([2, 0, 1]))
>>> print M.twoElementRepresentation()
[BasicAlgNumber([[1L, 0], 3], [2, 0, 1]), BasicAlgNumber([[3, 2], 3], [2, 0, 1])]
>>> M.norm()
Rational(1, 9)
\end{ex}%Don't indent!
\C

%---------- end document ---------- %

\bibliographystyle{jplain}%use jbibtex
\bibliography{nzmath_references}

\end{document}

