\documentclass{report}

\documentclass{report}

%%%%%%%%%%%%%%%%%%%%%%%%%%%%%%%%%%%%%%%%%%%%%%%%%%%%%%%%%%%%%
%
% macros for nzmath manual
%
%%%%%%%%%%%%%%%%%%%%%%%%%%%%%%%%%%%%%%%%%%%%%%%%%%%%%%%%%%%%%
\usepackage{amssymb,amsmath}
\usepackage{color}
\usepackage[dvipdfm,bookmarks=true,bookmarksnumbered=true,%
 pdftitle={NZMATH Users Manual},%
 pdfsubject={Manual for NZMATH Users},%
 pdfauthor={NZMATH Development Group},%
 pdfkeywords={TeX; dvipdfmx; hyperref; color;},%
 colorlinks=true]{hyperref}
\usepackage{fancybox}
\usepackage[T1]{fontenc}
%
\newcommand{\DS}{\displaystyle}
\newcommand{\C}{\clearpage}
\newcommand{\NO}{\noindent}
\newcommand{\negok}{$\dagger$}
\newcommand{\spacing}{\vspace{1pt}\\ }
% software macros
\newcommand{\nzmathzero}{{\footnotesize $\mathbb{N}\mathbb{Z}$}\texttt{MATH}}
\newcommand{\nzmath}{{\nzmathzero}\ }
\newcommand{\pythonzero}{$\mbox{\texttt{Python}}$}
\newcommand{\python}{{\pythonzero}\ }
% link macros
\newcommand{\linkingzero}[1]{{\bf \hyperlink{#1}{#1}}}%module
\newcommand{\linkingone}[2]{{\bf \hyperlink{#1.#2}{#2}}}%module,class/function etc.
\newcommand{\linkingtwo}[3]{{\bf \hyperlink{#1.#2.#3}{#3}}}%module,class,method
\newcommand{\linkedzero}[1]{\hypertarget{#1}{}}
\newcommand{\linkedone}[2]{\hypertarget{#1.#2}{}}
\newcommand{\linkedtwo}[3]{\hypertarget{#1.#2.#3}{}}
\newcommand{\linktutorial}[1]{\href{http://docs.python.org/tutorial/#1}{#1}}
\newcommand{\linktutorialone}[2]{\href{http://docs.python.org/tutorial/#1}{#2}}
\newcommand{\linklibrary}[1]{\href{http://docs.python.org/library/#1}{#1}}
\newcommand{\linklibraryone}[2]{\href{http://docs.python.org/library/#1}{#2}}
\newcommand{\pythonhp}{\href{http://www.python.org/}{\python website}}
\newcommand{\nzmathwiki}{\href{http://nzmath.sourceforge.net/wiki/}{{\nzmathzero}Wiki}}
\newcommand{\nzmathsf}{\href{http://sourceforge.net/projects/nzmath/}{\nzmath Project Page}}
\newcommand{\nzmathtnt}{\href{http://tnt.math.metro-u.ac.jp/nzmath/}{\nzmath Project Official Page}}
% parameter name
\newcommand{\param}[1]{{\tt #1}}
% function macros
\newcommand{\hiki}[2]{{\tt #1}:\ {\em #2}}
\newcommand{\hikiopt}[3]{{\tt #1}:\ {\em #2}=#3}

\newdimen\hoge
\newdimen\truetextwidth
\newcommand{\func}[3]{%
\setbox0\hbox{#1(#2)}
\hoge=\wd0
\truetextwidth=\textwidth
\advance \truetextwidth by -2\oddsidemargin
\ifdim\hoge<\truetextwidth % short form
{\bf \colorbox{skyyellow}{#1(#2)\ $\to$ #3}}
%
\else % long form
\fcolorbox{skyyellow}{skyyellow}{%
   \begin{minipage}{\textwidth}%
   {\bf #1(#2)\\ %
    \qquad\quad   $\to$\ #3}%
   \end{minipage}%
   }%
\fi%
}

\newcommand{\out}[1]{{\em #1}}
\newcommand{\initialize}{%
  \paragraph{\large \colorbox{skyblue}{Initialize (Constructor)}}%
    \quad\\ %
    \vspace{3pt}\\
}
\newcommand{\method}{\C \paragraph{\large \colorbox{skyblue}{Methods}}}
% Attribute environment
\newenvironment{at}
{%begin
\paragraph{\large \colorbox{skyblue}{Attribute}}
\quad\\
\begin{description}
}%
{%end
\end{description}
}
% Operation environment
\newenvironment{op}
{%begin
\paragraph{\large \colorbox{skyblue}{Operations}}
\quad\\
\begin{table}[h]
\begin{center}
\begin{tabular}{|l|l|}
\hline
operator & explanation\\
\hline
}%
{%end
\hline
\end{tabular}
\end{center}
\end{table}
}
% Examples environment
\newenvironment{ex}%
{%begin
\paragraph{\large \colorbox{skyblue}{Examples}}
\VerbatimEnvironment
\renewcommand{\EveryVerbatim}{\fontencoding{OT1}\selectfont}
\begin{quote}
\begin{Verbatim}
}%
{%end
\end{Verbatim}
\end{quote}
}
%
\definecolor{skyblue}{cmyk}{0.2, 0, 0.1, 0}
\definecolor{skyyellow}{cmyk}{0.1, 0.1, 0.5, 0}
%
%\title{NZMATH User Manual\\ {\large{(for version 1.0)}}}
%\date{}
%\author{}
\begin{document}
%\maketitle
%
\setcounter{tocdepth}{3}
\setcounter{secnumdepth}{3}


\tableofcontents
\C

\chapter{Classes}


%---------- start document ---------- %
 \section{poly.formalsum -- formal sum}\linkedzero{poly.formalsum}
 \begin{itemize}
   \item {\bf Classes}
   \begin{itemize}
     \item \negok\linkingone{poly.formalsum}{FormalSumContainerInterface}
     \item \linkingone{poly.formalsum}{DictFormalSum}
     \item \negok \linkingone{poly.formalsum}{ListFormalSum}
   \end{itemize}
 \end{itemize}

 The formal sum is mathematically a finite sum of terms,
 A term consists of two parts: coefficient and base.
 All coefficients in a formal sum are in a common ring,
 while bases are arbitrary.

 Two formal sums can be added in the following way.
 If there are terms with common base, they are fused into a new
 term with the same base and coefficients added.

 A coefficient can be looked up from the base. If the specified base
 does not appear in the formal sum, it is null.

 We refer the following for convenience as {\tt terminit}:
 \begin{description}
   \item[terminit]\linkedone{poly.formalsum}{terminit}:\\
     \param{terminit} means one of types to initialize
     \linklibraryone{stdtypes\#dict}{dict}.  The dictionary
     constructed from it will be considered as a mapping from bases to
     coefficients.
 \end{description}

\paragraph{Note for beginner}
You may need USE only \linkingone{poly.formalsum}{DictFormalSum},
but may have to READ the description of
\linkingone{poly.formalsum}{FormalSumContainerInterface} because
interface (all method names and their semantics) is defined in it.


\C
%
 \subsection{FormalSumContainerInterface -- interface class}\linkedone{poly.formalsum}{FormalSumContainerInterface}
  \initialize
  Since the interface is an abstract class, do not instantiate.\\
  \spacing
  % document of basic document
  \quad The interface defines what ``formal sum'' is.
  Derived classes must provide the following operations and methods.
  \begin{op}
    \verb/f + g/ & addition\\
    \verb/f - g/ & subtraction\\
    \verb/-f/ & negation\\
    \verb/+f/ & new copy\\
    \verb/f * a, a * f/ & multiplication by scalar {\tt a}\\
    \verb/f == g/ & equality\\
    \verb/f != g/ & inequality\\
    \verb/f[b]/	& get coefficient corresponding to a base {\tt b}\\
    \verb/b in f/ & return whether base {\tt b} is in {\tt f}\\
    \verb/len(f)/ & number of terms\\
    \verb/hash(f)/ & hash\\
  \end{op}
  \method
  \subsubsection{construct\_with\_default -- copy-constructing}\linkedtwo{poly.formalsum}{FormalSumContainerInterface}{construct\_with\_default}
   \func{construct\_with\_default}{\param{self},\ %
   \hiki{maindata}{terminit}}{\out{FormalSumContainerInterface}}\\
   \spacing
   % document of basic document
   \quad Create a new formal sum of the same class with \param{self},
   with given only the \param{maindata} and use copy of \param{self}'s
   data if necessary.
   \spacing
%
  \subsubsection{iterterms -- iterator of terms}\linkedtwo{poly.formalsum}{FormalSumContainerInterface}{iterterms}
   \func{iterterms}{\param{self}}{\out{iterator}}\\
   \spacing
   % document of basic document
   \quad Return an iterator of the terms.
   \spacing
   % input, output document
   \quad Each term yielded from iterators is a {\tt (base, coefficient)} pair.\\
 \subsubsection{itercoefficients -- iterator of coefficients}\linkedtwo{poly.formalsum}{FormalSumContainerInterface}{itercoefficients}
   \func{itercoefficients}{\param{self}}{\out{iterator}}\\
   \spacing
   % document of basic document
   \quad Return an iterator of the coefficients.\\
 \subsubsection{iterbases -- iterator of bases}\linkedtwo{poly.formalsum}{FormalSumContainerInterface}{iterbases}
   \func{iterbases}{\param{self}}{\out{iterator}}\\
   \spacing
   % document of basic document
   \quad Return an iterator of the bases.\\
  \subsubsection{terms -- list of terms}\linkedtwo{poly.formalsum}{FormalSumContainerInterface}{terms}
   \func{terms}{\param{self}}{\out{list}}\\
   \spacing
   % document of basic document
   \quad Return a list of the terms.
   \spacing
   % input, output document
   \quad Each term in returned lists is a {\tt (base, coefficient)} pair.\\
 \subsubsection{coefficients -- list of coefficients}\linkedtwo{poly.formalsum}{FormalSumContainerInterface}{coefficients}
   \func{coefficients}{\param{self}}{\out{list}}\\
   \spacing
   % document of basic document
   \quad Return a list of the coefficients.\\
 \subsubsection{bases -- list of bases}\linkedtwo{poly.formalsum}{FormalSumContainerInterface}{bases}
   \func{bases}{\param{self}}{\out{list}}\\
   \spacing
   % document of basic document
   \quad Return a list of the bases.\\
  \subsubsection{terms\_map -- list of terms}\linkedtwo{poly.formalsum}{FormalSumContainerInterface}{terms\_map}
   \func{terms\_map}{\param{self},\ %
   \hiki{func}{function}}{\out{FormalSumContainerInterface}}\\
   \spacing
   % document of basic document
   \quad Map on terms, i.e., create a new formal sum by applying \param{func}
   to each term.
   \spacing
   % input, output document
   \quad \param{func} has to accept two parameters {\tt base} and
   {\tt coefficient}, then return a new term pair.
 \subsubsection{coefficients\_map -- list of coefficients}\linkedtwo{poly.formalsum}{FormalSumContainerInterface}{coefficients\_map}
   \func{coefficients\_map}{\param{self}}{\out{FormalSumContainerInterface}}\\
   \spacing
   % document of basic document
   \quad Map on coefficients, i.e., create a new formal sum by applying
   \param{func} to each coefficient.\\
   \spacing
   % input, output document
   \quad \param{func} has to accept one parameters {\tt coefficient},
   then return a new coefficient.
 \subsubsection{bases\_map -- list of bases}\linkedtwo{poly.formalsum}{FormalSumContainerInterface}{bases\_map}
   \func{bases\_map}{\param{self}}{\out{FormalSumContainerInterface}}\\
   \spacing
   % document of basic document
   \quad Map on bases, i.e., create a new formal sum by applying
   \param{func} to each base.\\
   \spacing
   % input, output document
   \quad \param{func} has to accept one parameters {\tt base},
   then return a new base.

\C
%
 \subsection{DictFormalSum -- formal sum implemented with dictionary}\linkedone{poly.formalsum}{DictFormalSum}
  % document of basic document
  A formal sum implementation based on {\tt dict}.\\
  \spacing
  % added document
  \quad This class inherits \linkingone{poly.formalsum}{FormalSumContainerInterface}.
  All methods of the interface are implemented.
 \initialize
  \func{DictFormalSum}{\hiki{args}{terminit},\ %
  \hikiopt{defaultvalue}{RingElement}{None}}{\out{DictFormalSum}}\\
  \spacing
  \quad See \linkingone{poly.formalsum}{terminit} for type of \param{args}.
  It makes a mapping from bases to coefficients.\\
  \quad The optional argument \param{defaultvalue} is the default value for
  {\tt \_\_getitem\_\_}, i.e., if there is no term with the specified
  base, a look up attempt returns the \param{defaultvalue}.  It is,
  thus, an element of the ring to which other coefficients belong.

 \subsection{ListFormalSum -- formal sum implemented with list}\linkedone{poly.formalsum}{ListFormalSum}
  % document of basic document
   A formal sum implementation based on list.\\
  \spacing
  % added document
  \quad This class inherits \linkingone{poly.formalsum}{FormalSumContainerInterface}.
  All methods of the interface are implemented.
 \initialize
  \func{ListFormalSum}{\hiki{args}{terminit},\ %
  \hikiopt{defaultvalue}{RingElement}{None}}{\out{ListFormalSum}}\\
  \spacing
  \quad See \linkingone{poly.formalsum}{terminit} for type of \param{args}.
  It makes a mapping from bases to coefficients.\\
  \quad The optional argument \param{defaultvalue} is the default value for
  {\tt \_\_getitem\_\_}, i.e., if there is no term with the specified
  base, a look up attempt returns the \param{defaultvalue}.  It is,
  thus, an element of the ring to which other coefficients belong.
\C
%---------- end document ---------- %

\bibliographystyle{jplain}%use jbibtex
\bibliography{nzmath_references}

\end{document}

