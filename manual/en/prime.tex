\documentclass{report}

%%%%%%%%%%%%%%%%%%%%%%%%%%%%%%%%%%%%%%%%%%%%%%%%%%%%%%%%%%%%%
%
% macros for nzmath manual
%
%%%%%%%%%%%%%%%%%%%%%%%%%%%%%%%%%%%%%%%%%%%%%%%%%%%%%%%%%%%%%
\usepackage{amssymb,amsmath}
\usepackage{color}
\usepackage[dvipdfm,bookmarks=true,bookmarksnumbered=true,%
 pdftitle={NZMATH Users Manual},%
 pdfsubject={Manual for NZMATH Users},%
 pdfauthor={NZMATH Development Group},%
 pdfkeywords={TeX; dvipdfmx; hyperref; color;},%
 colorlinks=true]{hyperref}
\usepackage{fancybox}
\usepackage[T1]{fontenc}
%
\newcommand{\DS}{\displaystyle}
\newcommand{\C}{\clearpage}
\newcommand{\NO}{\noindent}
\newcommand{\negok}{$\dagger$}
\newcommand{\spacing}{\vspace{1pt}\\ }
% software macros
\newcommand{\nzmathzero}{{\footnotesize $\mathbb{N}\mathbb{Z}$}\texttt{MATH}}
\newcommand{\nzmath}{{\nzmathzero}\ }
\newcommand{\pythonzero}{$\mbox{\texttt{Python}}$}
\newcommand{\python}{{\pythonzero}\ }
% link macros
\newcommand{\linkingzero}[1]{{\bf \hyperlink{#1}{#1}}}%module
\newcommand{\linkingone}[2]{{\bf \hyperlink{#1.#2}{#2}}}%module,class/function etc.
\newcommand{\linkingtwo}[3]{{\bf \hyperlink{#1.#2.#3}{#3}}}%module,class,method
\newcommand{\linkedzero}[1]{\hypertarget{#1}{}}
\newcommand{\linkedone}[2]{\hypertarget{#1.#2}{}}
\newcommand{\linkedtwo}[3]{\hypertarget{#1.#2.#3}{}}
\newcommand{\linktutorial}[1]{\href{http://docs.python.org/tutorial/#1}{#1}}
\newcommand{\linktutorialone}[2]{\href{http://docs.python.org/tutorial/#1}{#2}}
\newcommand{\linklibrary}[1]{\href{http://docs.python.org/library/#1}{#1}}
\newcommand{\linklibraryone}[2]{\href{http://docs.python.org/library/#1}{#2}}
\newcommand{\pythonhp}{\href{http://www.python.org/}{\python website}}
\newcommand{\nzmathwiki}{\href{http://nzmath.sourceforge.net/wiki/}{{\nzmathzero}Wiki}}
\newcommand{\nzmathsf}{\href{http://sourceforge.net/projects/nzmath/}{\nzmath Project Page}}
\newcommand{\nzmathtnt}{\href{http://tnt.math.se.tmu.ac.jp/nzmath/}{\nzmath Project Official Page}}
% parameter name
\newcommand{\param}[1]{{\tt #1}}
% function macros
\newcommand{\hiki}[2]{{\tt #1}:\ {\em #2}}
\newcommand{\hikiopt}[3]{{\tt #1}:\ {\em #2}=#3}

\newdimen\hoge
\newdimen\truetextwidth
\newcommand{\func}[3]{%
\setbox0\hbox{#1(#2)}
\hoge=\wd0
\truetextwidth=\textwidth
\advance \truetextwidth by -2\oddsidemargin
\ifdim\hoge<\truetextwidth % short form
{\bf \colorbox{skyyellow}{#1(#2)\ $\to$ #3}}
%
\else % long form
\fcolorbox{skyyellow}{skyyellow}{%
   \begin{minipage}{\textwidth}%
   {\bf #1(#2)\\ %
    \qquad\quad   $\to$\ #3}%
   \end{minipage}%
   }%
\fi%
}

\newcommand{\out}[1]{{\em #1}}
\newcommand{\initialize}{%
  \paragraph{\large \colorbox{skyblue}{Initialize (Constructor)}}%
    \quad\\ %
    \vspace{3pt}\\
}
\newcommand{\method}{\C \paragraph{\large \colorbox{skyblue}{Methods}}}
% Attribute environment
\newenvironment{at}
{%begin
\paragraph{\large \colorbox{skyblue}{Attribute}}
\quad\\
\begin{description}
}%
{%end
\end{description}
}
% Operation environment
\newenvironment{op}
{%begin
\paragraph{\large \colorbox{skyblue}{Operations}}
\quad\\
\begin{table}[h]
\begin{center}
\begin{tabular}{|l|l|}
\hline
operator & explanation\\
\hline
}%
{%end
\hline
\end{tabular}
\end{center}
\end{table}
}
% Examples environment
\newenvironment{ex}%
{%begin
\paragraph{\large \colorbox{skyblue}{Examples}}
\VerbatimEnvironment
\renewcommand{\EveryVerbatim}{\fontencoding{OT1}\selectfont}
\begin{quote}
\begin{Verbatim}
}%
{%end
\end{Verbatim}
\end{quote}
}
%
\definecolor{skyblue}{cmyk}{0.2, 0, 0.1, 0}
\definecolor{skyyellow}{cmyk}{0.1, 0.1, 0.5, 0}
%
%\title{NZMATH User Manual\\ {\large{(for version 1.0)}}}
%\date{}
%\author{}
\begin{document}
%\maketitle
%
\setcounter{tocdepth}{3}
\setcounter{secnumdepth}{3}


\tableofcontents
\C

\chapter{Functions}


%---------- start document ---------- %
 \section{prime -- primality test , prime generation}\linkedzero{prime}
%
  \subsection{trialDivision -- trial division test}\linkedone{prime}{trialDivision}
   \func{trialDivision}
   {\hiki{n}{integer},\ \hikiopt{bound}{integer/float}{0}}{\out{True/False}}\\
   \spacing
   % document of basic document
   \quad Trial division primality test for an odd natural number.\\
   \spacing
   % added document
   %\spacing
   % input, output document
   \quad \param{bound} is a search bound of primes. 
   If it returns \(1\) under the condition that \param{bound} is given and 
   less than the square root of \param{n}, 
   it only means there is no prime factor less than \param{bound}.
%
  \subsection{spsp -- strong pseudo-prime test}\linkedone{prime}{spsp}
   \func{spsp}{\hiki{n}{integer},\ \hiki{base}{integer},\ \hikiopt{s}{integer}{None},\ \hikiopt{t}{integer}{None}}{\out{True/False}}\\
   \spacing
   % document of basic document
   \quad Strong Pseudo-Prime test on base \param{base}.\\
   \spacing
   % added document
   %\quad 
   %\spacing
   % input, output document
   \quad \param{s} and \param{t} are the numbers such that $n-1 = 2^\param{s}\param{t}$ and \param{t} is odd.
%
 \subsection{smallSpsp -- strong pseudo-prime test for small number}\linkedone{prime}{smallSpsp}
   \func{smallSpsp}{\hiki{n}{integer},\ \hikiopt{s}{integer}{None},\ \hikiopt{t}{integer}{None}}{\out{True/False}}\\
   \spacing
   % document of basic document
   \quad Strong Pseudo-Prime test for integer \param{n} less than $10^{12}$.\\
   \spacing
   % added document
   \quad $4$ spsp tests are sufficient to determine whether an integer less than $10^{12}$ is prime or not.
   \spacing
   % input, output document
   \quad \param{s} and \param{t} are the numbers such that $n-1 = 2^\param{s}\param{t}$ and \param{t} is odd.
%
  \subsection{miller -- Miller's primality test}\linkedone{prime}{miller}
   \func{miller}
   {\hiki{n}{integer}}{\out{True/False}}\\
   \spacing
   % document of basic document
   \quad Miller's primality test.\\
   \spacing
   % added document
   \quad This test is valid under GRH. See \linkingzero{config}.
   \spacing
   % input, output document
   %\quad 
%
  \subsection{millerRabin -- Miller-Rabin primality test}\linkedone{prime}{millerRabin}
   \func{millerRabin}
   {\hiki{n}{integer},\ \hikiopt{times}{integer}{20}}{\out{True/False}}\\
   \spacing
   % document of basic document
   \quad Miller's primality test.\\
   \spacing
   % added document
   \quad The difference from \linkingone{prime}{miller} is that 
   the Miller-Rabin method uses fast but probabilistic algorithm.
   On the other hand, \linkingone{prime}{miller} employs deterministic
   algorithm valid under GRH.
   \spacing
   % input, output document
   \quad \param{times} (default to $20$) is the number of repetition.
   The error probability is at most $4^{-\param{times}}$.
%
 \subsection{lpsp -- Lucas test}\linkedone{prime}{lpsp}
   \func{lpsp}
   {\hiki{n}{integer},\ \hiki{a}{integer},\ \hiki{b}{integer}}{\out{True/False}}\\
   \spacing
   % document of basic document
   \quad Lucas Pseudo-Prime test.\\
   \spacing
   % added document
   \quad Return True if \param{n} is a Lucas pseudo-prime of parameters \param{a}, \param{b},
    i.e. with respect to $x^2-\param{a}x+\param{b}$.
   \spacing
   % input, output document
   %\quad 
%
 \subsection{fpsp -- Frobenius test}\linkedone{prime}{fpsp}
   \func{fpsp}
   {\hiki{n}{integer},\ \hiki{a}{integer},\ \hiki{b}{integer}}{\out{True/False}}\\
   \spacing
   % document of basic document
   \quad Frobenius Pseudo-Prime test.\\
   \spacing
   % added document
   \quad Return True if \param{n} is a Frobenius pseudo-prime of parameters \param{a}, \param{b},
    i.e. with respect to $x^2-\param{a}x+\param{b}$.
   \spacing
   % input, output document
   %\quad 
%
 \subsection{by\_primitive\_root -- Lehmer's test}\linkedone{prime}{by\_primitive\_root}
   \func{by\_primitive\_root}
   {\hiki{n}{integer},\ \hiki{divisors}{sequence}}{\out{True/False}}\\
   \spacing
   % document of basic document
   \quad Lehmer's primality test~\cite{Lehmer1927}.\\
   \spacing
   % added document
   \quad Return True iff \param{n} is prime.\\
    The method proves the primality of \param{n} by existence of a primitive
    root.
   \spacing
   % input, output document
   \quad \param{divisors} is a sequence (list, tuple, etc.) of prime divisors
   of $n - 1$.
   %\quad 
%
 \subsection{full\_euler -- Brillhart \& Selfridge's test}\linkedone{prime}{full\_euler}
   \func{full\_euler}
   {\hiki{n}{integer},\ \hiki{divisors}{sequence}}{\out{True/False}}\\
   \spacing
   % document of basic document
   \quad Brillhart \& Selfridge's primality test~\cite{BS1967}.\\
   \spacing
   % added document
   \quad Return True iff \param{n} is prime.\\
    The method proves the primality of \param{n} by the equality
    $\varphi(n) = n - 1$, where $\varphi$ denotes the Euler totient
    (see \linkingone{multiplicative}{euler}).
    It requires a sequence of all prime divisors of $n - 1$.
   \spacing
   % input, output document
   \quad \param{divisors} is a sequence (list, tuple, etc.) of prime divisors
   of $n - 1$.
   %\quad 
   \quad
%
 \subsection{apr -- Jacobi sum test}\linkedone{prime}{apr}
   \func{apr}
   {\hiki{n}{integer}}{\out{True/False}}\\
   \spacing
   % document of basic document
   \quad APR (Adleman-Pomerance-Rumery) primality test or the Jacobi sum test.\\
   \spacing
   % added document
   %\quad 
   %\spacing
   % input, output document
   \quad Assuming \param{n} has no prime factors less than $32$.
    Assuming \param{n} is spsp (strong pseudo-prime) for several bases.
%
 \subsection{primeq -- primality test automatically}\linkedone{prime}{primeq}
   \func{primeq}
   {\hiki{n}{integer}}{\out{True/False}}\\
   \spacing
   % document of basic document
   \quad A convenient function for primality test.\\
   \spacing
   % added document
   \quad It uses one of \linkingone{prime}{trialDivision}, \linkingone{prime}{smallSpsp} or \linkingone{prime}{apr} depending on the size of \param{n}.
   \spacing
   % input, output document
   %\quad 
%
%\subsection{bigprimeq -- primality test automatically}\linkedone{prime}{bigprimeq}
%   \func{bigprimeq}
%   {\hiki{z}{integer}}{\out{True/False}}\\
%   \spacing
   % document of basic document
%   \quad Giving up rigorous proof of primality, return True for a probable prime.
%   \spacing
   % added document
%   \quad 
%   \spacing
   % input, output document
   %\quad 
%
 \subsection{prime -- $n$-th prime number}\linkedone{prime}{prime}
   \func{prime}
   {\hiki{n}{integer}}{\out{integer}}\\
   \spacing
   % document of basic document
   \quad Return the \param{n}-th prime number.\\
   \spacing
   % added document
   %\quad 
   %\spacing
   % input, output document
   %\quad 
%
 \subsection{nextPrime -- generate next prime}\linkedone{prime}{nextPrime}
   \func{nextPrime}
   {\hiki{n}{integer}}{\out{integer}}\\
   \spacing
   % document of basic document
   \quad Return the smallest prime bigger than the given integer \param{n}.
   \spacing
   % added document
   %\quad 
   %\spacing
   % input, output document
   %\quad 
%
 \subsection{randPrime -- generate random prime}\linkedone{prime}{randPrime}
   \func{randPrime}
   {\hiki{n}{integer}}{\out{integer}}\\
   \spacing
   % document of basic document
   \quad Return a random \param{n}-digits prime.\\
   \spacing
   % added document
   %\quad 
   %\spacing
   % input, output document
   %\quad 
%
 \subsection{generator -- generate primes}\linkedone{prime}{generator}
   \func{generator}
   {(None)}{\out{generator}}\\
   \spacing
   % document of basic document
   \quad Generate primes from $2$ to $\infty$ (as generator).\\
   \spacing
   % added document
   %\quad 
   %\spacing
   % input, output document
   %\quad 
%
 \subsection{generator\_eratosthenes -- generate primes using Eratosthenes sieve}\linkedone{prime}{generator\_eratosthenes}
   \func{generator\_eratosthenes}
   {\hiki{n}{integer}}{\out{generator}}\\
   \spacing
   % document of basic document
   \quad Generate primes up to \param{n} using Eratosthenes sieve.\\
   \spacing
   % added document
   %\quad 
   %\spacing
   % input, output document
   %\quad 
%
 \subsection{primonial -- product of primes}\linkedone{prime}{primonial}
   \func{primonial}
   {\hiki{p}{integer}}{\out{integer}}\\
   \spacing
   % document of basic document
   \quad Return the product
   \begin{equation*}
   \prod_{q \in \mathbb{P}_{\le \param{p}}} q=2\cdot 3\cdot 5\cdots \param{p}\ .
   \end{equation*}
   \spacing
   % added document
   %\quad 
   %\spacing
   % input, output document
   %\quad 
%
 \subsection{properDivisors -- proper divisors}\linkedone{prime}{properDivisors}
   \func{properDivisors}
   {\hiki{n}{integer}}{\out{list}}\\
   \spacing
   % document of basic document
   \quad Return proper divisors of \param{n} (all divisors of \param{n} excluding $1$ and \param{n}).\\
   \spacing
   % added document
   \quad  It is only useful for a product of small primes.
   Use \linkingtwo{factor.misc}{FactoredInteger}{proper\_divisors} in a more
   general case.
   \spacing
   % input, output document
   \quad The output is the list of all proper divisors.\\
%
 \subsection{primitive\_root -- primitive root}\linkedone{prime}{primitive\_root}
   \func{primitive\_root}
   {\hiki{p}{integer}}{\out{integer}}\\
   \spacing
   % document of basic document
   \quad Return a primitive root of \param{p}.\\
   \spacing
   % added document
   %\quad  
   %\spacing
   % input, output document
   \quad \param{p} must be an odd prime.
%
 \subsection{Lucas\_chain -- Lucas sequence}\linkedone{prime}{Lucas\_chain}
   \func{Lucas\_chain}
   {\hiki{n}{integer},\ \hiki{f}{function},\ \hiki{g}{function},\ \hiki{x\_0}{integer},\ \hiki{x\_1}{integer}}{(\out{integer},\ \out{integer})}\\
   \spacing
   % document of basic document
   \quad Return the value of ($x_n$,\ $x_{n+1}$) for the sequnce $\{ x_i \}$ defined as:\\
   \begin{eqnarray*}
      x_{2i} = \param{f}(x_i)\\
      x_{2i+1} = \param{g}(x_i, x_{i+1})\ ,
   \end{eqnarray*}
   where the initial values \param{x\_0},\ \param{x\_1}.\\
   \spacing
   % added document
   %\quad  
   %\spacing
   % input, output document
   \quad \param{f} is the function which can be input as $1$-ary integer.
   \param{g} is the function which can be input as $2$-ary integer.\\
%
\begin{ex}
>>> prime.primeq(131)
True
>>> prime.primeq(133)
False
>>> g = prime.generator()
>>> g.next()
2
>>> g.next()
3
>>> prime.prime(10)
29
>>> prime.nextPrime(100)
101
>>> prime.primitive_root(23)
5
\end{ex}%Don't indent!(indent causes an error.)
\C

%---------- end document ---------- %

\bibliographystyle{jplain}%use jbibtex
\bibliography{nzmath_references}

\end{document}

