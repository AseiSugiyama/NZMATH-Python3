\documentclass{report}

%%%%%%%%%%%%%%%%%%%%%%%%%%%%%%%%%%%%%%%%%%%%%%%%%%%%%%%%%%%%%
%
% macros for nzmath manual
%
%%%%%%%%%%%%%%%%%%%%%%%%%%%%%%%%%%%%%%%%%%%%%%%%%%%%%%%%%%%%%
\usepackage{amssymb,amsmath}
\usepackage{color}
\usepackage[dvipdfm,bookmarks=true,bookmarksnumbered=true,%
 pdftitle={NZMATH Users Manual},%
 pdfsubject={Manual for NZMATH Users},%
 pdfauthor={NZMATH Development Group},%
 pdfkeywords={TeX; dvipdfmx; hyperref; color;},%
 colorlinks=true]{hyperref}
\usepackage{fancybox}
\usepackage[T1]{fontenc}
%
\newcommand{\DS}{\displaystyle}
\newcommand{\C}{\clearpage}
\newcommand{\NO}{\noindent}
\newcommand{\negok}{$\dagger$}
\newcommand{\spacing}{\vspace{1pt}\\ }
% software macros
\newcommand{\nzmathzero}{{\footnotesize $\mathbb{N}\mathbb{Z}$}\texttt{MATH}}
\newcommand{\nzmath}{{\nzmathzero}\ }
\newcommand{\pythonzero}{$\mbox{\texttt{Python}}$}
\newcommand{\python}{{\pythonzero}\ }
% link macros
\newcommand{\linkingzero}[1]{{\bf \hyperlink{#1}{#1}}}%module
\newcommand{\linkingone}[2]{{\bf \hyperlink{#1.#2}{#2}}}%module,class/function etc.
\newcommand{\linkingtwo}[3]{{\bf \hyperlink{#1.#2.#3}{#3}}}%module,class,method
\newcommand{\linkedzero}[1]{\hypertarget{#1}{}}
\newcommand{\linkedone}[2]{\hypertarget{#1.#2}{}}
\newcommand{\linkedtwo}[3]{\hypertarget{#1.#2.#3}{}}
\newcommand{\linktutorial}[1]{\href{http://docs.python.org/tutorial/#1}{#1}}
\newcommand{\linktutorialone}[2]{\href{http://docs.python.org/tutorial/#1}{#2}}
\newcommand{\linklibrary}[1]{\href{http://docs.python.org/library/#1}{#1}}
\newcommand{\linklibraryone}[2]{\href{http://docs.python.org/library/#1}{#2}}
\newcommand{\pythonhp}{\href{http://www.python.org/}{\python website}}
\newcommand{\nzmathwiki}{\href{http://nzmath.sourceforge.net/wiki/}{{\nzmathzero}Wiki}}
\newcommand{\nzmathsf}{\href{http://sourceforge.net/projects/nzmath/}{\nzmath Project Page}}
\newcommand{\nzmathtnt}{\href{http://tnt.math.se.tmu.ac.jp/nzmath/}{\nzmath Project Official Page}}
% parameter name
\newcommand{\param}[1]{{\tt #1}}
% function macros
\newcommand{\hiki}[2]{{\tt #1}:\ {\em #2}}
\newcommand{\hikiopt}[3]{{\tt #1}:\ {\em #2}=#3}

\newdimen\hoge
\newdimen\truetextwidth
\newcommand{\func}[3]{%
\setbox0\hbox{#1(#2)}
\hoge=\wd0
\truetextwidth=\textwidth
\advance \truetextwidth by -2\oddsidemargin
\ifdim\hoge<\truetextwidth % short form
{\bf \colorbox{skyyellow}{#1(#2)\ $\to$ #3}}
%
\else % long form
\fcolorbox{skyyellow}{skyyellow}{%
   \begin{minipage}{\textwidth}%
   {\bf #1(#2)\\ %
    \qquad\quad   $\to$\ #3}%
   \end{minipage}%
   }%
\fi%
}

\newcommand{\out}[1]{{\em #1}}
\newcommand{\initialize}{%
  \paragraph{\large \colorbox{skyblue}{Initialize (Constructor)}}%
    \quad\\ %
    \vspace{3pt}\\
}
\newcommand{\method}{\C \paragraph{\large \colorbox{skyblue}{Methods}}}
% Attribute environment
\newenvironment{at}
{%begin
\paragraph{\large \colorbox{skyblue}{Attribute}}
\quad\\
\begin{description}
}%
{%end
\end{description}
}
% Operation environment
\newenvironment{op}
{%begin
\paragraph{\large \colorbox{skyblue}{Operations}}
\quad\\
\begin{table}[h]
\begin{center}
\begin{tabular}{|l|l|}
\hline
operator & explanation\\
\hline
}%
{%end
\hline
\end{tabular}
\end{center}
\end{table}
}
% Examples environment
\newenvironment{ex}%
{%begin
\paragraph{\large \colorbox{skyblue}{Examples}}
\VerbatimEnvironment
\renewcommand{\EveryVerbatim}{\fontencoding{OT1}\selectfont}
\begin{quote}
\begin{Verbatim}
}%
{%end
\end{Verbatim}
\end{quote}
}
%
\definecolor{skyblue}{cmyk}{0.2, 0, 0.1, 0}
\definecolor{skyyellow}{cmyk}{0.1, 0.1, 0.5, 0}
%
%\title{NZMATH User Manual\\ {\large{(for version 1.0)}}}
%\date{}
%\author{}
\begin{document}
%\maketitle
%
\setcounter{tocdepth}{3}
\setcounter{secnumdepth}{3}


\tableofcontents
\C

\chapter{Classes}


%---------- start document ---------- %
 \section{elliptic -- elliptic class object}\linkedzero{elliptic}
 \begin{itemize}
   \item {\bf Classes}
   \begin{itemize}
     \item \linkingone{elliptic}{ECGeneric}
     \item \linkingone{elliptic}{ECoverQ}
     \item \linkingone{elliptic}{ECoverGF}
   \end{itemize}
   \item {\bf Functions}
     \begin{itemize}
       \item \linkingone{elliptic}{EC}
     \end{itemize}
 \end{itemize}

 This module using following type:
 \begin{description}
   \item[weierstrassform]\linkedone{elliptic}{weierstrassform}:\\
     \param{weierstrassform} is a list ($a_1, a_2, a_3, a_4, a_6$) or ($a_4, a_6$), it represents $E:y^2+a_1xy+a_3y=x^3+a_2x^2+a_4x+a_6$ or $E:y^2=x^3+a_4x+a_6$, respectively.
   \item[infpoint]\linkedone{elliptic}{infpoint}:\\
     \param{infpoint} is the list {\tt [0]}, which represents infinite point on the elliptic curve.
   \item[point]\linkedone{elliptic}{point}:\\
     \param{point} is two-dimensional coordinate list [\param{x}, \param{y}] or \linkingone{elliptic}{infpoint}.
 \end{description}

\C

 \subsection{\negok ECGeneric -- generic elliptic curve class}\linkedone{elliptic}{ECGeneric}
\initialize
  \func{ECGeneric}{
    \hiki{coefficient}{\linkingone{elliptic}{weierstrassform}},\
    \hikiopt{basefield}{Field}{None}
  }{
    \out{ECGeneric}
  }\\
  \spacing
  % document of basic document
  \quad Create an elliptic curve object.\\
  \spacing
  % added document
  \quad The class is for the definition of elliptic curves over general fields.
  Instead of using this class directly, we recommend that you call \linkingone{elliptic}{EC}.\\
  \negok The class precomputes the following values.
  \begin{itemize}
    \item shorter form:\ $y^2=b_2 x^3+b_4x^2+b_6x+b_8$
    \item shortest form:\ $y^2=x^3+c_4x+c_6$
    \item discriminant
    \item j-invariant
  \end{itemize}
  \quad\\
  % input, output document
  \quad All elements of \param{coefficient} must be in \param{basefield}.\\
  See \linkingone{elliptic}{weierstrassform} for more information about \param{coefficient}.
  If discriminant of \param{self} equals $0$, it raises ValueError.
  \begin{at}
    \item[basefield]\linkedtwo{elliptic}{ECGeneric}{basefield}:\\ It expresses the field which each coordinate of all points in \param{self} is on.
(This means not only \param{self} is defined over \param{basefield}.)
    \item[ch]\linkedtwo{elliptic}{ECGeneric}{ch}:\\ It expresses the characteristic of \param{basefield}.
    \item[infpoint]\linkedtwo{elliptic}{ECGeneric}{infpoint}:\\ It expresses infinity point (i.e. [0]).
    \item[a1, a2, a3, a4, a6]\linkedtwo{elliptic}{ECGeneric}{a1, a2, a3, a4, a6}:\\ It expresses the coefficients \param{a1}, \param{a2}, \param{a3}, \param{a4}, \param{a6}.
    \item[b2, b4, b6, b8]\linkedtwo{elliptic}{ECGeneric}{b2, b4, b6, b8}:\\ It expresses the coefficients \param{b2}, \param{b4}, \param{b6}, \param{b8}.
    \item[c4, c6]\linkedtwo{elliptic}{ECGeneric}{c4, c6}:\\ It expresses the coefficients \param{c4}, \param{c6}.
    \item[disc]\linkedtwo{elliptic}{ECGeneric}{disc}:\\ It expresses the discriminant of \param{self}.
    \item[j]\linkedtwo{elliptic}{ECGeneric}{j}:\\ It expresses the j-invariant of \param{self}.
    \item[coefficient]\linkedtwo{elliptic}{ECGeneric}{coefficient}:\\ It expresses the \linkingone{elliptic}{weierstrassform} of \param{self}.
\end{at}
  %\begin{op}
  %\end{op}
  \method
  \subsubsection{simple -- simplify the curve coefficient}\linkedtwo{elliptic}{ECGeneric}{simple}
   \func{simple}{\param{self}}{\out{\linkingone{elliptic}{ECGeneric}}}\\
   \spacing
   % document of basic document
   \quad Return elliptic curve corresponding to the short Weierstrass form of \param{self} by changing the coordinates.\\
   %\spacing
   % added document
   %\quad \negok Note that this function returns integer only.\\
   %\spacing
   % input, output document
   %\quad \param{a} must be int, long or rational.Integer.\\
%
  \subsubsection{changeCurve -- change the curve by coordinate change}\linkedtwo{elliptic}{ECGeneric}{changeCurve}
   \func{changeCurve}{\param{self},\ \hiki{V}{list}}{\out{\linkingone{elliptic}{ECGeneric}}}\\
   \spacing
   % document of basic document
   \quad Return elliptic curve corresponding to the curve obtained by some coordinate change $x=u^2x'+r,\ y=u^3y'+su^2x'+t$.\\
   \spacing
   % added document
   \quad For $u \ne 0$, the coordinate change gives some curve which is \linkingtwo{elliptic}{ECGeneric}{basefield}-isomorphic to \param{self}.\\
   \spacing
   % input, output document
   \quad $V$ must be a list of the form $[u,r,s,t]$, where $u,r,s,t$ are in \linkingtwo{elliptic}{ECGeneric}{basefield}.\\
%
  \subsubsection{changePoint -- change coordinate of point on the curve}\linkedtwo{elliptic}{ECGeneric}{changePoint}
   \func{changePoint}{\param{self},\ \hiki{P}{\linkingone{elliptic}{point}},\ \hiki{V}{list}}{\out{\linkingone{elliptic}{point}}}\\
   \spacing
   % document of basic document
   \quad Return the point corresponding to the point obtained by the coordinate change $x'=(x-r) u^{-2},\ y'=(y-s(x-r)+t)u^{-3}$.\\
   \spacing
   % added document
   \quad Note that the inverse coordinate change is $x=u^2x'+r,\ y=u^3y'+su^2x'+t$.See \linkingtwo{elliptic}{ECGeneric}{changeCurve}.\\
   \spacing
   % input, output document
   \quad $V$ must be a list of the form $[u,r,s,t]$, where $u,r,s,t$ are in \linkingtwo{elliptic}{ECGeneric}{basefield}.$u$ must be non-zero.\\
%
  \subsubsection{coordinateY -- Y-coordinate from X-coordinate}\linkedtwo{elliptic}{ECGeneric}{coordinateY}
   \func{coordinateY}{\param{self},\ \hiki{x}{\linkingone{ring}{FieldElement}}}{\out{\linkingone{ring}{FieldElement} / False}}\\
   \spacing
   % document of basic document
   \quad Return Y-coordinate of the point on \param{self}  whose X-coordinate is \param{x}.\\
   \spacing
   % added document
   %\quad \negok Note that this function returns integer only.\\
   %\spacing
   % input, output document
   \quad The output would be one Y-coordinate (if a coordinate is found).
   If such a Y-coordinate does not exist,  it returns False.\\
%
  \subsubsection{whetherOn -- Check point is on curve }\linkedtwo{elliptic}{ECGeneric}{whetherOn}
   \func{whetherOn}{\param{self},\ \hiki{P}{\linkingone{elliptic}{point}}}{\out{bool}}\\
   \spacing
   % document of basic document
   \quad Check whether the point \param{P} is on \param{self} or not.\\
   \spacing
   % added document
   %\quad \negok Note that this function returns integer only.\\
   %\spacing
   % input, output document
   %\quad Output \param{Y-coordinate} is either one Y-coordinate or 2-component vector containing the Y-coordinates. 
%
  \subsubsection{add -- Point addition on the curve}\linkedtwo{elliptic}{ECGeneric}{add}
   \func{add}{\param{self},\ \hiki{P}{\linkingone{elliptic}{point}},\ \hiki{Q}{\linkingone{elliptic}{point}}}{\out{\linkingone{elliptic}{point}}}\\
   \spacing
   % document of basic document
   \quad Return the sum of the point \param{P} and \param{Q} on \param{self}.\\
   \spacing
   % added document
   %\quad \negok Note that this function returns integer only.\\
   %\spacing
   % input, output document
   %\quad Output \param{Y-coordinate} is either one Y-coordinate or 2-component vector containing the Y-coordinates. 
%
  \subsubsection{sub -- Point subtraction on the curve}\linkedtwo{elliptic}{ECGeneric}{sub}
   \func{sub}{\param{self},\ \hiki{P}{\linkingone{elliptic}{point}},\ \hiki{Q}{\linkingone{elliptic}{point}}}{\out{\linkingone{elliptic}{point}}}\\
   \spacing
   % document of basic document
   \quad Return the subtraction of the point \param{P} from \param{Q} on \param{self}.\\
   \spacing
   % added document
   %\quad \negok Note that this function returns integer only.\\
   %\spacing
   % input, output document
   %\quad Output \param{Y-coordinate} is either one Y-coordinate or 2-component vector containing the Y-coordinates. 
%
  \subsubsection{mul -- Scalar point multiplication on the curve}\linkedtwo{elliptic}{ECGeneric}{mul}
   \func{mul}{\param{self},\ \hiki{k}{integer},\ \hiki{P}{\linkingone{elliptic}{point}}}{\out{\linkingone{elliptic}{point}}}\\
   \spacing
   % document of basic document
   \quad Return the scalar multiplication of the point \param{P} by a scalar \param{k} on \param{self}.\\
   \spacing
   % added document
   %\quad \negok Note that this function returns integer only.\\
   %\spacing
   % input, output document
   %\quad Output \param{Y-coordinate} is either one Y-coordinate or 2-component vector containing the Y-coordinates. 
%
  \subsubsection{divPoly -- division polynomial}\linkedtwo{elliptic}{ECGeneric}{divPoly}
   \func{divPoly}{\param{self},\ \hikiopt{m}{integer}{None}}{\out{\linkingone{poly.uniutil}{FieldPolynomial}/(\hiki{f}{list}, \hiki{H}{integer})}}\\
   \spacing
   % document of basic document
   \quad Return the division polynomial.\\
   \spacing
   % added document
   %\quad \negok Note that this function returns integer only.\\
   %\spacing
   % input, output document
   \quad  If \param{m} is odd, this method returns the usual division polynomial. If \param{m} is even, return the quotient of the usual division polynomial by $2y + a_1 x + a_3$.\\
 \negok If \param{m} is not specified (i.e. \param{m=None}), then return $(\param{f},\ \param{H})$.
 $\param{H}$ is the least prime satisfying $\prod_{2\le l \le H,\ l:prime} l > 4\sqrt{q}$, where $q$ is the order of \linkingtwo{elliptic}{ECGeneric}{basefield}.
 $\param{f}$ is the list of $k$-division polynomials up to $k \le \param{H}$.
 These are used for Schoof's algorithm.\\
%
\C
 \subsection{ECoverQ -- elliptic curve over rational field}\linkedone{elliptic}{ECoverQ}
 The class is for elliptic curves over the rational field $\mathbb{Q}$ (\linkingone{rational}{RationalField} in nzmath.\linkingzero{rational}).\\
 The class is a subclass of \linkingone{elliptic}{ECGeneric}.

\initialize
  \func{ECoverQ}{\hiki{coefficient}{\linkingone{elliptic}{weierstrassform}}}{\out{\linkingone{elliptic}{ECoverQ}}}\\
  \spacing
  % document of basic document
  \quad Create elliptic curve over the rational field.\\
  \spacing
  % added document
  %
  % \spacing
  % input, output document
  \quad All elements of \param{coefficient} must be integer or \linkingone{rational}{Rational}. \\
  See \linkingone{elliptic}{weierstrassform} for more information about \param{coefficient}.\\
\begin{ex}
>>> E = elliptic.ECoverQ([ratinal.Rational(1, 2), 3])
>>> print E.disc
-3896/1
>>> print E.j
1728/487
\end{ex}%Don't indent!
  \method
  \subsubsection{point -- obtain random point on curve}\linkedtwo{elliptic}{ECoverQ}{point}
   \func{point}{\param{self},\ \hikiopt{limit}{integer}{1000}}{\out{\linkingone{elliptic}{point}}}\\
   \spacing
   % document of basic document
   \quad Return a random point on \param{self}.\\
   \spacing
   % added document
   \quad  \param{limit} expresses the time of trying to choose points.
   If failed, raise ValueError.
   \negok Because it is difficult to search the rational point over the rational field, it might raise error with high frequency.\\
   %\spacing
   % input, output document
   %\quad \param{a} must be int, long or rational.Integer.\\
%
\begin{ex}
>>> print E.changeCurve([1, 2, 3, 4])
y ** 2 + 6/1 * x * y + 8/1 * y = x ** 3 - 3/1 * x ** 2 - 23/2 * x - 4/1
>>> E.divPoly(3)
FieldPolynomial([(0, Rational(-1, 4)), (1, Rational(36, 1)), (2, Rational(3, 1)
), (4, Rational(3, 1))], RationalField())
\end{ex}%Don't indent!
\C
 \subsection{ECoverGF -- elliptic curve over finite field}\linkedone{elliptic}{ECoverGF}
 The class is for elliptic curves over a finite field, denoted by $\mathbb{F}_q$ (\linkingone{finitefield}{FiniteField} and its subclasses in nzmath).\\
 The class is a subclass of \linkingone{elliptic}{ECGeneric}.
\initialize
  \func{ECoverGF}{
    \hiki{coefficient}{\linkingone{elliptic}{weierstrassform}},\ 
    \hiki{basefield}{\linkingone{finitefield}{FiniteField}}
    }{
    \out{\linkingone{elliptic}{ECoverGF}}
  }\\
  \spacing
  % document of basic document
  \quad Create elliptic curve over a finite field.\\
  \spacing
  % added document
  %
  % \spacing
  % input, output document
  \quad All elements of \param{coefficient} must be in \param{basefield}.
  \param{basefield} should be an instance of \linkingone{finitefield}{FiniteField}.\\
  See \linkingone{elliptic}{weierstrassform} for more information about \param{coefficient}.
\begin{ex}
>>> E = elliptic.ECoverGF([2, 5], finitefield.FinitePrimeField(11))
>>> print E.j
7 in F_11
>>> E.whetherOn([8, 4])
True
>>> E.add([3, 4], [9, 9])
[FinitePrimeFieldElement(0, 11), FinitePrimeFieldElement(4, 11)]
>>> E.mul(5, [9, 9])
[FinitePrimeFieldElement(0, 11)]
\end{ex}%Don't indent!
  \method
  \subsubsection{point -- find random point on curve}\linkedtwo{elliptic}{ECoverGF}{point}
   \func{point}{\param{self}}{\out{\linkingone{elliptic}{point}}}\\
   \spacing
   % document of basic document
   \quad Return a random point on \param{self}.\\
   \spacing
   % added document
   \quad This method uses a probabilistic algorithm.\\
   %\spacing
   % input, output document
   %\quad \param{a} must be int, long or rational.Integer.\\
%
  \subsubsection{naive -- Frobenius trace by naive method}\linkedtwo{elliptic}{ECoverGF}{naive}
   \func{naive}{\param{self}}{\out{integer}}\\
   \spacing
   % document of basic document
   \quad Return Frobenius trace $t$ by a naive method.\\
   \spacing
   % added document
   \quad \negok The function counts up the Legendre symbols of all rational points on \param{self}.\\
   Frobenius trace of the curve is $t$ such that $\#E(\mathbb{F}_q)=q+1-t$, where $\#E(\mathbb{F}_q)$ stands for the number of points on \param{self} over \param{self}.\linkingtwo{elliptic}{ECGeneric}{basefield} $\mathbb{F}_q$. \\
   \spacing
   % input, output document
   \quad The characteristic of \param{self}.\linkingtwo{elliptic}{ECGeneric}{basefield} must not be $2$ nor $3$.\\
%
  \subsubsection{Shanks\_Mestre -- Frobenius trace by Shanks and Mestre method}\linkedtwo{elliptic}{ECoverGF}{Shanks\_Mestre}
   \func{Shanks\_Mestre}{\param{self}}{\out{integer}}\\
   \spacing
   % document of basic document
   \quad Return Frobenius trace $t$ by Shanks and Mestre method.\\
   \spacing
   % added document
   \quad \negok This uses the method proposed by Shanks and Mestre.
   \negok See Algorithm 7.5.3 of \cite{Pomerance} for more information about the algorithm.\\
   Frobenius trace of the curve is $t$ such that $\#E(\mathbb{F}_q)=q+1-t$, where $\#E(\mathbb{F}_q)$ stands for the number of points on \param{self} over \param{self}.\linkingtwo{elliptic}{ECGeneric}{basefield} $\mathbb{F}_q$. \\
   \spacing
   % input, output document
   \quad \param{self}.\linkingtwo{elliptic}{ECGeneric}{basefield} must be an instance of \linkingone{finitefield}{FinitePrimeField}.\\
%
  \subsubsection{Schoof -- Frobenius trace by Schoof's method}\linkedtwo{elliptic}{ECoverGF}{Schoof}
   \func{Schoof}{\param{self}}{\out{integer}}\\
   \spacing
   % document of basic document
   \quad Return Frobenius trace $t$ by Schoof's method.\\
   \spacing
   % added document
   \quad \negok This uses the method proposed by Schoof.\\
   Frobenius trace of the curve is $t$ such that $\#E(\mathbb{F}_q)=q+1-t$, where $\#E(\mathbb{F}_q)$ stands for the number of points on \param{self} over \param{self}.\linkingtwo{elliptic}{ECGeneric}{basefield} $\mathbb{F}_q$. \\
   \spacing
   % input, output document
%
  \subsubsection{trace -- Frobenius trace}\linkedtwo{elliptic}{ECoverGF}{trace}
   \func{trace}{\param{self},\ \hikiopt{r}{integer}{None}}{\out{integer}}\\
   \spacing
   % document of basic document
   \quad Return Frobenius trace $t$.\\
   \spacing
   % added document
   \quad Frobenius trace of the curve is $t$ such that $\#E(\mathbb{F}_q)=q+1-t$, where $\#E(\mathbb{F}_q)$ stands for the number of points on \param{self} over \param{self}.\linkingtwo{elliptic}{ECGeneric}{basefield} $\mathbb{F}_q$. \\
   If positive $r$ given, it returns $q^r+1-\#E(\mathbb{F}_{q^r})$.\\
   \negok The method selects algorithms by investigating \param{self}.\linkingtwo{elliptic}{ECGeneric}{ch} when \param{self}.\linkingtwo{elliptic}{ECGeneric}{basefield} is an instance of \linkingone{finitefield}{
FinitePrimeField}.
   If \param{ch}<1000, the method uses \linkingtwo{elliptic}{ECoverGF}{naive}. 
   If $10^4<\param{ch}<10^{30}$, the method uses \linkingtwo{elliptic}{ECoverGF}{Shanks\_Mestre}. 
   Otherwise, it uses \linkingtwo{elliptic}{ECoverGF}{Schoof}.\\
   \spacing
   % input, output document
   \quad The parameter $r$ must be positive integer.\\
%
  \subsubsection{order -- order of group of rational points on the curve}\linkedtwo{elliptic}{ECoverGF}{order}
   \func{order}{\param{self},\ \hikiopt{r}{integer}{None}}{\out{integer}}\\
   \spacing
   % document of basic document
   \quad Return order $\#E(\mathbb{F}_q)=q+1-t$. \\
   \spacing
   % added document
   \quad If positive $r$ given, this computes $\#E({\mathbb{F}_q}^r)$ instead. \\
   \negok On the computation of Frobenius trace $t$, the method calls \linkingtwo{elliptic}{ECoverGF}{trace}.\\
   \spacing
   % input, output document
   \quad The parameter $r$ must be positive integer.\\
%
  \subsubsection{pointorder -- order of point on the curve}\linkedtwo{elliptic}{ECoverGF}{pointorder}
   \func{pointorder}{\param{self},\ \hiki{P}{\linkingone{elliptic}{point}},\ \hikiopt{ord\_factor}{list}{None}}{\out{integer}}\\
   \spacing
   % document of basic document
   \quad Return order of a point \param{P}. \\
   \spacing
   % added document
   \quad \negok The method uses factorization of \linkingtwo{elliptic}{ECoverGF}{order}.\\
   If \param{ord\_factor} is given, computation of factorizing the order of \param{self} is omitted and it applies \param{ord\_factor} instead.\\
   \spacing
   % input, output document
   %\quad 
%
  \subsubsection{TatePairing -- Tate Pairing}\linkedtwo{elliptic}{ECoverGF}{TatePairing}
   \func{TatePairing}
        {\param{self},\ 
          \hiki{m}{integer},\ 
          \hiki{P}{\linkingone{elliptic}{point}},\ 
          \hiki{Q}{\linkingone{elliptic}{point}} 
        }
        {\out{\linkingone{finitefield}{FiniteFieldElement}}}\\
   \spacing
   % document of basic document
   \quad Return Tate-Lichetenbaum pairing ${\langle P,\ Q\rangle}_m$.\\
   \spacing
   % added document
   \quad \negok The method uses Miller's algorithm.\\
   The image of the Tate pairing is $\mathbb{F}_q^*/{\mathbb{F}_q^*}^m$, but the method returns an element of $\mathbb{F}_q$, so the value is not uniquely defined.
   If uniqueness is needed, use \linkingtwo{elliptic}{ECoverGF}{TatePairing\_Extend}.\\
   \spacing
   % input, output document
   \quad The point \param{P} has to be a \param{m}-torsion point (i.e. $mP=$[0]).
   Also, the number \param{m} must divide \linkingtwo{elliptic}{ECoverGF}{order}.\\
%
  \subsubsection{TatePairing\_Extend -- Tate Pairing with final exponentiation}\linkedtwo{elliptic}{ECoverGF}{TatePairing\_Extend}
   \func{TatePairing\_Extend}
        {\param{self},\ 
          \hiki{m}{integer},\ 
          \hiki{P}{\linkingone{elliptic}{point}},\ 
          \hiki{Q}{\linkingone{elliptic}{point}} 
        }
        {\out{\linkingone{finitefield}{FiniteFieldElement}}}\\
   \spacing
   % document of basic document
   \quad Return Tate Pairing with final exponentiation, i.e. ${{\langle P,\ Q\rangle}_m}^{(q-1)/m}$.\\
   \spacing
   % added document
   \quad \negok The method calls \linkingtwo{elliptic}{ECoverGF}{TatePairing}.\\
   \spacing
   % input, output document
   \quad The point \param{P} has to be a \param{m}-torsion point (i.e. $mP=$[0]). Also the number \param{m} must divide \linkingtwo{elliptic}{ECoverGF}{order}.\\
   The output is in the group generated by $m$-th root of unity in ${\mathbb{F}_q^*}$.
%
  \subsubsection{WeilPairing -- Weil Pairing }\linkedtwo{elliptic}{ECoverGF}{WeilPairing}
   \func{WeilPairing}
        {\param{self},\ 
          \hiki{m}{integer},\ 
          \hiki{P}{\linkingone{elliptic}{point}},\ 
          \hiki{Q}{\linkingone{elliptic}{point}} 
        }
        {\out{\linkingone{finitefield}{FiniteFieldElement}}}\\
   \spacing
   % document of basic document
   \quad Return Weil pairing  $e_m(P,\ Q)$.\\
   \spacing
   % added document
   \quad \negok The method uses Miller's algorithm.\\
   \spacing
   % input, output document
   \quad The points \param{P} and \param{Q} has to be a \param{m}-torsion point (i.e. $mP=mQ=$[0]).
   Also, the number \param{m} must divide \linkingtwo{elliptic}{ECoverGF}{order}.

   The output is in the group generated by $m$-th root of unity in ${\mathbb{F}_q^*}$.
%
  \subsubsection{BSGS -- point order by Baby-Step and Giant-Step}\linkedtwo{elliptic}{ECoverGF}{BSGS}
   \func{BSGS}
        {\param{self},\ 
          \hiki{P}{\linkingone{elliptic}{point}}
        }
        {\out{integer}}\\
   \spacing
   % document of basic document
   \quad Return order of point $P$ by Baby-Step and Giant-Step method.\\
   \spacing
   % added document
   \quad \negok See \cite{Washington} for more information about the algorithm.\\
   %\spacing
   % input, output document
   %\quad
%
  \subsubsection{DLP\_BSGS -- solve Discrete Logarithm Problem by Baby-Step and Giant-Step}\linkedtwo{elliptic}{ECoverGF}{DLP\_BSGS}
   \func{DLP\_BSGS}
        {\param{self},\ 
          \hiki{n}{integer},\ 
          \hiki{P}{\linkingone{elliptic}{point}},\ 
          \hiki{Q}{\linkingone{elliptic}{point}}
        }
        {\out{\hiki{m}{integer}}}\\
   \spacing
   % document of basic document
   \quad Return \param{m} such that $Q=mP$ by Baby-Step and Giant-Step method.\\
   \spacing
   % added document
   %\spacing
   % input, output document
   \quad The points \param{P} and \param{Q} has to be a \param{n}-torsion point (i.e. $nP=nQ=$[0]). Also, the number \param{n} must divide \linkingtwo{elliptic}{ECoverGF}{order}.\\
   The output \param{m} is an integer.\\
%
  \subsubsection{structure -- structure of group of rational points}\linkedtwo{elliptic}{ECoverGF}{structure}
   \func{structure}
        {\param{self}}
        {\out{\hiki{structure}{tuple}}}\\
   \spacing
   % document of basic document
   \quad Return the group structure of \param{self}.\\
   \spacing
   % added document
   \quad  The structure of $E(\mathbb{F}_q)$ is represented as $\mathbb{Z}/d\mathbb{Z} \times \mathbb{Z}/n\mathbb{Z}$.
   The method uses \linkingtwo{elliptic}{ECoverGF}{WeilPairing}.\\
   \spacing
   % input, output document
   \quad The output \param{structure} is a tuple of positive two integers \param{(d,\  n)}. \param{d} divides \param{n}.
%
  \subsubsection{issupersingular -- check supersingular curve}\linkedtwo{elliptic}{ECoverGF}{issupersingular}
   \func{structure}
        {\param{self}}
        {\out{bool}}\\
   \spacing
   % document of basic document
   \quad Check whether \param{self} is a supersingular curve or not.\\
   \spacing
   % added document
   %\quad  
   %\spacing
   % input, output document
   %\quad
%
\begin{ex}
>>> E=nzmath.elliptic.ECoverGF([2, 5], nzmath.finitefield.FinitePrimeField(11))
>>> E.whetherOn([0, 4])
True
>>> print E.coordinateY(3)
4 in F_11
>>> E.trace()
2
>>> E.order()
10
>>> E.pointorder([3, 4])
10L
>>> E.TatePairing(10, [3, 4], [9, 9])
FinitePrimeFieldElement(3, 11)
>>> E.DLP_BSGS(10, [3, 4], [9, 9])
6
\end{ex}%Don't indent!
\C
  \subsection{EC(function)}\linkedone{elliptic}{EC}
  \func{EC}
       {\hiki{coefficient}{\linkingone{elliptic}{weierstrassform}},\ 
        \hiki{basefield}{\linkingone{ring}{Field}}}
       {\out{\linkingone{elliptic}{ECGeneric}}}\\
   \spacing
   \quad Create an elliptic curve object.\\
   \spacing
   % added document
   % \quad \negok
   % \spacing
   % input, output document
   \quad All elements of \param{coefficient} must be in \param{basefield}.\\
   \param{basefield} must be \linkingone{rational}{RationalField} or \linkingone{finitefield}{FiniteField} or their subclasses. 
  See also \linkingone{elliptic}{weierstrassform} for \param{coefficient}.
%\begin{ex}
%>>> module.hogefunction()
%(0, 0)
%>>>
%\end{ex}%Don't indent!
\C

%---------- end document ---------- %

\bibliographystyle{jplain}%use jbibtex
\bibliography{nzmath_references}

\end{document}

