\documentclass{report}

%%%%%%%%%%%%%%%%%%%%%%%%%%%%%%%%%%%%%%%%%%%%%%%%%%%%%%%%%%%%%
%
% macros for nzmath manual
%
%%%%%%%%%%%%%%%%%%%%%%%%%%%%%%%%%%%%%%%%%%%%%%%%%%%%%%%%%%%%%
\usepackage{amssymb,amsmath}
\usepackage{color}
\usepackage[dvipdfm,bookmarks=true,bookmarksnumbered=true,%
 pdftitle={NZMATH Users Manual},%
 pdfsubject={Manual for NZMATH Users},%
 pdfauthor={NZMATH Development Group},%
 pdfkeywords={TeX; dvipdfmx; hyperref; color;},%
 colorlinks=true]{hyperref}
\usepackage{fancybox}
\usepackage[T1]{fontenc}
%
\newcommand{\DS}{\displaystyle}
\newcommand{\C}{\clearpage}
\newcommand{\NO}{\noindent}
\newcommand{\negok}{$\dagger$}
\newcommand{\spacing}{\vspace{1pt}\\ }
% software macros
\newcommand{\nzmathzero}{{\footnotesize $\mathbb{N}\mathbb{Z}$}\texttt{MATH}}
\newcommand{\nzmath}{{\nzmathzero}\ }
\newcommand{\pythonzero}{$\mbox{\texttt{Python}}$}
\newcommand{\python}{{\pythonzero}\ }
% link macros
\newcommand{\linkingzero}[1]{{\bf \hyperlink{#1}{#1}}}%module
\newcommand{\linkingone}[2]{{\bf \hyperlink{#1.#2}{#2}}}%module,class/function etc.
\newcommand{\linkingtwo}[3]{{\bf \hyperlink{#1.#2.#3}{#3}}}%module,class,method
\newcommand{\linkedzero}[1]{\hypertarget{#1}{}}
\newcommand{\linkedone}[2]{\hypertarget{#1.#2}{}}
\newcommand{\linkedtwo}[3]{\hypertarget{#1.#2.#3}{}}
\newcommand{\linktutorial}[1]{\href{http://docs.python.org/tutorial/#1}{#1}}
\newcommand{\linktutorialone}[2]{\href{http://docs.python.org/tutorial/#1}{#2}}
\newcommand{\linklibrary}[1]{\href{http://docs.python.org/library/#1}{#1}}
\newcommand{\linklibraryone}[2]{\href{http://docs.python.org/library/#1}{#2}}
\newcommand{\pythonhp}{\href{http://www.python.org/}{\python website}}
\newcommand{\nzmathwiki}{\href{http://nzmath.sourceforge.net/wiki/}{{\nzmathzero}Wiki}}
\newcommand{\nzmathsf}{\href{http://sourceforge.net/projects/nzmath/}{\nzmath Project Page}}
\newcommand{\nzmathtnt}{\href{http://tnt.math.se.tmu.ac.jp/nzmath/}{\nzmath Project Official Page}}
% parameter name
\newcommand{\param}[1]{{\tt #1}}
% function macros
\newcommand{\hiki}[2]{{\tt #1}:\ {\em #2}}
\newcommand{\hikiopt}[3]{{\tt #1}:\ {\em #2}=#3}

\newdimen\hoge
\newdimen\truetextwidth
\newcommand{\func}[3]{%
\setbox0\hbox{#1(#2)}
\hoge=\wd0
\truetextwidth=\textwidth
\advance \truetextwidth by -2\oddsidemargin
\ifdim\hoge<\truetextwidth % short form
{\bf \colorbox{skyyellow}{#1(#2)\ $\to$ #3}}
%
\else % long form
\fcolorbox{skyyellow}{skyyellow}{%
   \begin{minipage}{\textwidth}%
   {\bf #1(#2)\\ %
    \qquad\quad   $\to$\ #3}%
   \end{minipage}%
   }%
\fi%
}

\newcommand{\out}[1]{{\em #1}}
\newcommand{\initialize}{%
  \paragraph{\large \colorbox{skyblue}{Initialize (Constructor)}}%
    \quad\\ %
    \vspace{3pt}\\
}
\newcommand{\method}{\C \paragraph{\large \colorbox{skyblue}{Methods}}}
% Attribute environment
\newenvironment{at}
{%begin
\paragraph{\large \colorbox{skyblue}{Attribute}}
\quad\\
\begin{description}
}%
{%end
\end{description}
}
% Operation environment
\newenvironment{op}
{%begin
\paragraph{\large \colorbox{skyblue}{Operations}}
\quad\\
\begin{table}[h]
\begin{center}
\begin{tabular}{|l|l|}
\hline
operator & explanation\\
\hline
}%
{%end
\hline
\end{tabular}
\end{center}
\end{table}
}
% Examples environment
\newenvironment{ex}%
{%begin
\paragraph{\large \colorbox{skyblue}{Examples}}
\VerbatimEnvironment
\renewcommand{\EveryVerbatim}{\fontencoding{OT1}\selectfont}
\begin{quote}
\begin{Verbatim}
}%
{%end
\end{Verbatim}
\end{quote}
}
%
\definecolor{skyblue}{cmyk}{0.2, 0, 0.1, 0}
\definecolor{skyyellow}{cmyk}{0.1, 0.1, 0.5, 0}
%
%\title{NZMATH User Manual\\ {\large{(for version 1.0)}}}
%\date{}
%\author{}
\begin{document}
%\maketitle
%
\setcounter{tocdepth}{3}
\setcounter{secnumdepth}{3}


\tableofcontents
\C

\chapter{Classes}


%---------- start document ---------- %
 \section{poly.multivar -- multivariate polynomial}\linkedzero{poly.multivar}
 \begin{itemize}
   \item {\bf Classes}
   \begin{itemize}
     \item \negok \linkingone{poly.multivar}{PolynomialInterface}
     \item \negok \linkingone{poly.multivar}{BasicPolynomial}
     \item \linkingone{poly.multivar}{TermIndices}
   \end{itemize}
 \end{itemize}

\C

 \subsection{PolynomialInterface -- base class for all multivariate polynomials}\linkedone{poly.multivar}{PolynomialInterface}
  Since the interface is an abstract class, do not instantiate.\\

  % No Documentation Yet
%
 \subsection{BasicPolynomial -- basic implementation of polynomial}\linkedone{poly.multivar}{BasicPolynomial}
  Basic polynomial data type.

  % No Documentation Yet
%
 \subsection{TermIndices -- Indices of terms of multivariate polynomials}\linkedone{poly.multivar}{TermIndices}
  It is a tuple-like object.
  \initialize
  \func{TermIndices}{\hiki{indices}{tuple}}{\out{TermIndices}}\\
  \spacing
  \quad The constructor does not check the validity of indices, such
  as integerness, nonnegativity, etc.
  \begin{op}
    \verb/t == u/ & equality\\
    \verb/t != u/ & inequality\\
    \verb/t + u/ & (componentwise) addition\\
    \verb/t - u/ & (componentwise) subtraction\\
    \verb/t * a/ & (componentwise) multiplication by scalar {\tt a}\\
    \verb/t <= u, t < u, t >= u, t > u/ & ordering\\
    \verb/t[k]/ & {\tt k}-th index\\
    \verb/len(t)/ & length\\
    \verb/iter(t)/ & iterator\\
    \verb/hash(t)/ & hash\\
  \end{op}
  \method
  \subsubsection{pop}\linkedtwo{poly.multivar}{TermIndices}{pop}
   \func{pop}{\param{self},\ \hiki{pos}{integer}}{(\out{integer}, \out{TermIndices})}\\
   \spacing
   % document of basic document
   \quad Return the index at pos and a new TermIndices object as the
   omitting-the-pos indices.

  \subsubsection{gcd}\linkedtwo{poly.multivar}{TermIndices}{gcd}
  \func{gcd}{\param{self},\ \hiki{other}{TermIndices}}{\out{TermIndices}}\\
  \quad Return the ``gcd'' of two indices.

  \subsubsection{lcm}\linkedtwo{poly.multivar}{TermIndices}{lcm}
  \func{lcm}{\param{self},\ \hiki{other}{TermIndices}}{\out{TermIndices}}\\
  \quad Return the ``lcm'' of two indices.

\C

%---------- end document ---------- %

\bibliographystyle{jplain}%use jbibtex
\bibliography{nzmath_references}

\end{document}

