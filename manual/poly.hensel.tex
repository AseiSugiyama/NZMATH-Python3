\documentclass{report}

%%%%%%%%%%%%%%%%%%%%%%%%%%%%%%%%%%%%%%%%%%%%%%%%%%%%%%%%%%%%%
%
% macros for nzmath manual
%
%%%%%%%%%%%%%%%%%%%%%%%%%%%%%%%%%%%%%%%%%%%%%%%%%%%%%%%%%%%%%
\usepackage{amssymb,amsmath}
\usepackage{color}
\usepackage[dvipdfm,bookmarks=true,bookmarksnumbered=true,%
 pdftitle={NZMATH Users Manual},%
 pdfsubject={Manual for NZMATH Users},%
 pdfauthor={NZMATH Development Group},%
 pdfkeywords={TeX; dvipdfmx; hyperref; color;},%
 colorlinks=true]{hyperref}
\usepackage{fancybox}
\usepackage[T1]{fontenc}
%
\newcommand{\DS}{\displaystyle}
\newcommand{\C}{\clearpage}
\newcommand{\NO}{\noindent}
\newcommand{\negok}{$\dagger$}
\newcommand{\spacing}{\vspace{1pt}\\ }
% software macros
\newcommand{\nzmathzero}{{\footnotesize $\mathbb{N}\mathbb{Z}$}\texttt{MATH}}
\newcommand{\nzmath}{{\nzmathzero}\ }
\newcommand{\pythonzero}{$\mbox{\texttt{Python}}$}
\newcommand{\python}{{\pythonzero}\ }
% link macros
\newcommand{\linkingzero}[1]{{\bf \hyperlink{#1}{#1}}}%module
\newcommand{\linkingone}[2]{{\bf \hyperlink{#1.#2}{#2}}}%module,class/function etc.
\newcommand{\linkingtwo}[3]{{\bf \hyperlink{#1.#2.#3}{#3}}}%module,class,method
\newcommand{\linkedzero}[1]{\hypertarget{#1}{}}
\newcommand{\linkedone}[2]{\hypertarget{#1.#2}{}}
\newcommand{\linkedtwo}[3]{\hypertarget{#1.#2.#3}{}}
\newcommand{\linktutorial}[1]{\href{http://docs.python.org/tutorial/#1}{#1}}
\newcommand{\linktutorialone}[2]{\href{http://docs.python.org/tutorial/#1}{#2}}
\newcommand{\linklibrary}[1]{\href{http://docs.python.org/library/#1}{#1}}
\newcommand{\linklibraryone}[2]{\href{http://docs.python.org/library/#1}{#2}}
\newcommand{\pythonhp}{\href{http://www.python.org/}{\python website}}
\newcommand{\nzmathwiki}{\href{http://nzmath.sourceforge.net/wiki/}{{\nzmathzero}Wiki}}
\newcommand{\nzmathsf}{\href{http://sourceforge.net/projects/nzmath/}{\nzmath Project Page}}
\newcommand{\nzmathtnt}{\href{http://tnt.math.se.tmu.ac.jp/nzmath/}{\nzmath Project Official Page}}
% parameter name
\newcommand{\param}[1]{{\tt #1}}
% function macros
\newcommand{\hiki}[2]{{\tt #1}:\ {\em #2}}
\newcommand{\hikiopt}[3]{{\tt #1}:\ {\em #2}=#3}

\newdimen\hoge
\newdimen\truetextwidth
\newcommand{\func}[3]{%
\setbox0\hbox{#1(#2)}
\hoge=\wd0
\truetextwidth=\textwidth
\advance \truetextwidth by -2\oddsidemargin
\ifdim\hoge<\truetextwidth % short form
{\bf \colorbox{skyyellow}{#1(#2)\ $\to$ #3}}
%
\else % long form
\fcolorbox{skyyellow}{skyyellow}{%
   \begin{minipage}{\textwidth}%
   {\bf #1(#2)\\ %
    \qquad\quad   $\to$\ #3}%
   \end{minipage}%
   }%
\fi%
}

\newcommand{\out}[1]{{\em #1}}
\newcommand{\initialize}{%
  \paragraph{\large \colorbox{skyblue}{Initialize (Constructor)}}%
    \quad\\ %
    \vspace{3pt}\\
}
\newcommand{\method}{\C \paragraph{\large \colorbox{skyblue}{Methods}}}
% Attribute environment
\newenvironment{at}
{%begin
\paragraph{\large \colorbox{skyblue}{Attribute}}
\quad\\
\begin{description}
}%
{%end
\end{description}
}
% Operation environment
\newenvironment{op}
{%begin
\paragraph{\large \colorbox{skyblue}{Operations}}
\quad\\
\begin{table}[h]
\begin{center}
\begin{tabular}{|l|l|}
\hline
operator & explanation\\
\hline
}%
{%end
\hline
\end{tabular}
\end{center}
\end{table}
}
% Examples environment
\newenvironment{ex}%
{%begin
\paragraph{\large \colorbox{skyblue}{Examples}}
\VerbatimEnvironment
\renewcommand{\EveryVerbatim}{\fontencoding{OT1}\selectfont}
\begin{quote}
\begin{Verbatim}
}%
{%end
\end{Verbatim}
\end{quote}
}
%
\definecolor{skyblue}{cmyk}{0.2, 0, 0.1, 0}
\definecolor{skyyellow}{cmyk}{0.1, 0.1, 0.5, 0}
%
%\title{NZMATH User Manual\\ {\large{(for version 1.0)}}}
%\date{}
%\author{}
\begin{document}
%\maketitle
%
\setcounter{tocdepth}{3}
\setcounter{secnumdepth}{3}


\tableofcontents
\C

\chapter{Classes}


%---------- start document ---------- %
 \section{poly.hensel -- Hensel lift}\linkedzero{poly.hensel}
 \begin{itemize}
   \item {\bf Classes}
   \begin{itemize}
     \item \negok \linkingone{poly.hensel}{HenselLiftPair}
     \item \negok \linkingone{poly.hensel}{HenselLiftMulti}
     \item \negok \linkingone{poly.hensel}{HenselLiftSimultaneously}
   \end{itemize}
   \item {\bf Functions}
     \begin{itemize}
       \item \linkingone{poly.hensel}{lift\_upto}
     \end{itemize}
 \end{itemize}

 In this module document, {\em polynomial} means integer polynomial.
\C

 \subsection{HenselLiftPair -- Hensel lift for a pair}\linkedone{poly.hensel}{HenselLiftPair}
 \initialize
  \func{HenselLiftPair}{%
    \hiki{f}{polynomial},
    \hiki{a1}{polynomial},
    \hiki{a2}{polynomial},
    \hiki{u1}{polynomial},
    \hiki{u2}{polynomial},
    \hiki{p}{integer},
    \hikiopt{q}{integer}{p}}{\out{HenselLiftPair}}\\
  \spacing
  % document of basic document
  \quad This object keeps integer polynomial pair which will be lifted by Hensel's lemma.
  % added document
  %
  \spacing
  % input, output document
  \quad The argument should satisfy the following preconditions:
  \begin{itemize}
  \item \param{f}, \param{a1} and \param{a2} are monic
  \item {\tt \param{f} == \param{a1}*\param{a2} (mod \param{q})}
  \item {\tt \param{a1}*\param{u1} + \param{a2}*\param{u2} == 1 (mod \param{p})}
  \item \param{p} divides \param{q} and both are positive
  \end{itemize}
  \func{from\_factors}{%
    \hiki{f}{polynomial},
    \hiki{a1}{polynomial},
    \hiki{a2}{polynomial},
    \hiki{p}{integer}}{\out{HenselLiftPair}}\\
  \spacing
  \quad This is a class method to create and return an instance of {\tt HenselLiftPair}.
  You do not have to precompute {\tt u1} and {\tt u2} for the default constructor; they will be prepared for you from other arguments.\\
  \spacing
  % input, output document
  \quad The argument should satisfy the following preconditions:
  \begin{itemize}
  \item \param{f}, \param{a1} and \param{a2} are monic
  \item {\tt \param{f} == \param{a1}*\param{a2} (mod \param{p})}
  \item \param{p} is prime
  \end{itemize}
  \begin{at}
    \item[point]\linkedtwo{poly.hensel}{HenselLiftPair}{factors}:\\
      factors {\tt a1} and {\tt a2} as a list.
  \end{at}
  \method
  \subsubsection{lift -- lift one step}\linkedtwo{poly.hensel}{HenselLiftPair}{lift}
  \func{lift}{\param{self}}{}\\
  \spacing
  \quad Lift polynomials by so-called the quadratic method.
  \subsubsection{lift\_factors -- lift {\tt a1} and {\tt a2}}\linkedtwo{poly.hensel}{HenselLiftPair}{lift\_factors}
   \func{lift\_factors}{\param{self}}{}\\
   \spacing
   % document of basic document
   \quad Update factors by lifted integer coefficient polynomials {\tt Ai}'s:
   \begin{itemize}
   \item {\tt f == A1 * A2 (mod p * q)}
   \item {\tt Ai == ai (mod q)} \((i = 1, 2)\)
   \end{itemize}
   Moreover, {\tt q} is updated to {\tt p * q}.
   \spacing
   % added document
   \quad \negok The preconditions which should be automatically satisfied:
   \begin{itemize}
   \item {\tt f == a1*a2 (mod q)}
   \item {\tt a1*u1 + a2*u2 == 1 (mod p)}
   \item {\tt p} divides {\tt q}
   \end{itemize}
   \subsubsection{lift\_ladder -- lift {\tt u1} and {\tt u2}}\linkedtwo{poly.hensel}{HenselLiftPair}{lift\_ladder}
   \func{lift\_ladder}{\param{self}}{}\\
   \spacing
   % document of basic document
   \quad Update {\tt u1} and {\tt u2} with {\tt U1} and {\tt U2}:
   \begin{itemize}
   \item {\tt a1*U1 + a2*U2 == 1 (mod p**2)}
   \item {\tt Ui == ui (mod p)} \((i = 1, 2)\)
   \end{itemize}
   Then, update {\tt p} to {\tt p**2}.
   \spacing
   % added document
   \quad \negok The preconditions which should be automatically satisfied:
   \begin{itemize}
   \item {\tt a1*u1 + a2*u2 == 1 (mod p)}
   \end{itemize}

\subsection{HenselLiftMulti -- Hensel lift for multiple polynomials}\linkedone{poly.hensel}{HenselLiftMulti}
 \initialize
  \func{HenselLiftMulti}{%
    \hiki{f}{polynomial},
    \hiki{factors}{list},
    \hiki{ladder}{tuple},
    \hiki{p}{integer},
    \hikiopt{q}{integer}{p}}{\out{HenselLiftMulti}}\\
  \spacing
  % document of basic document
  \quad This object keeps integer polynomial factors which will be lifted by Hensel's lemma.
  If the number of factors is just two, then you should use \linkingone{poly.hensel}{HenselLiftPair}.
  % added document
  %
  \spacing
  % input, output document
  \quad \param{factors} is a list of polynomials;
  we refer those polynomials as {\tt a1}, {\tt a2}, \(\ldots\)
  \param{ladder} is a tuple of two lists {\tt sis} and {\tt tis},
  both lists consist polynomials.
  We refer polynomials in {\tt sis} as {\tt s1}, {\tt s2}, \(\ldots\),
  and those in {\tt tis} as {\tt t1}, {\tt t2}, \(\ldots\)
  Moreover, we define {\tt bi} as the product of {\tt aj}'s for
  \(i < j\).
  \quad The argument should satisfy the following preconditions:
  \begin{itemize}
  \item \param{f} and all of \param{factors} are monic
  \item {\tt \param{f} == \param{a1}*...*\param{ar} (mod \param{q})}
  \item {\tt ai*si + bi*ti == 1 (mod \param{p})} \((i = 1,2,\ldots,r)\)
  \item \param{p} divides \param{q} and both are positive
  \end{itemize}
%
  \func{from\_factors}{%
    \hiki{f}{polynomial},
    \hiki{factors}{list},
    \hiki{p}{integer}}{\out{HenselLiftMulti}}\\
  \spacing
  \quad This is a class method to create and return an instance of {\tt HenselLiftMulti}.
  You do not have to precompute {\tt ladder} for the default constructor; they will be prepared for you from other arguments.\\
  \spacing
  % input, output document
  \quad The argument should satisfy the following preconditions:
  \begin{itemize}
  \item \param{f} and all of \param{factors} are monic
  \item {\tt \param{f} == \param{a1}*...*\param{ar} (mod \param{q})}
  \item \param{p} is prime
  \end{itemize}
  \begin{at}
    \item[point]\linkedtwo{poly.hensel}{HenselLiftMulti}{factors}:\\
      factors {\tt ai}s as a list.
  \end{at}
  \method
  \subsubsection{lift -- lift one step}\linkedtwo{poly.hensel}{HenselLiftMulti}{lift}
  \func{lift}{\param{self}}{}\\
  \spacing
  \quad Lift polynomials by so-called the quadratic method.
  \subsubsection{lift\_factors -- lift factors}\linkedtwo{poly.hensel}{HenselLiftMulti}{lift\_factors}
  \func{lift\_factors}{\param{self}}{}\\
  \spacing
  % document of basic document
  \quad Update factors by lifted integer coefficient polynomials {\tt Ai}s:
  \begin{itemize}
  \item {\tt f == A1*...*Ar (mod p * q)}
  \item {\tt Ai == ai (mod q)} \((i = 1, \ldots, r)\)
  \end{itemize}
  Moreover, {\tt q} is updated to {\tt p * q}.
  \spacing
  % added document
  \quad \negok The preconditions which should be automatically satisfied:
  \begin{itemize}
  \item {\tt f == a1*...*ar (mod q)}
  \item {\tt ai*si + bi*ti == 1 (mod p)} \((i = 1,\ldots, r)\)
  \item {\tt p} divides {\tt q}
  \end{itemize}
  \subsubsection{lift\_ladder -- lift {\tt u1} and {\tt u2}}\linkedtwo{poly.hensel}{HenselLiftMulti}{lift\_ladder}
  \func{lift\_ladder}{\param{self}}{}\\
  \spacing
  % document of basic document
  \quad Update {\tt si}s and {\tt ti}s with {\tt Si}s and {\tt Ti}s:
  \begin{itemize}
  \item {\tt a1*Si + bi*Ti == 1 (mod p**2)}
  \item {\tt Si == si (mod p)} \((i = 1, \ldots, r)\)
  \item {\tt Ti == ti (mod p)} \((i = 1, \ldots, r)\)
  \end{itemize}
  Then, update {\tt p} to {\tt p**2}.
  \spacing
  % added document
  \quad \negok The preconditions which should be automatically satisfied:
  \begin{itemize}
  \item {\tt ai*si + bi*ti == 1 (mod p)} \((i = 1,\ldots, r)\)
  \end{itemize}
%

\subsection{HenselLiftSimultaneously}\linkedone{poly.hensel}{HenselLiftSimultaneously}

  The method explained in~\cite{ColEnc}.\\
  \quad \negok Keep these invariants:
  \begin{itemize}
  \item     {\tt ai}s, {\tt pi} and {\tt gi}s are monic
  \item     {\tt f == g1*...*gr (mod p)}
  \item     {\tt f == d0 + d1*p + d2*p**2 +...+ dk*p**k}
  \item     {\tt hi == g(i+1)*...*gr}
  \item     {\tt 1 == gi*si + hi*ti (mod p)} \((i = 1 ,\ldots, r)\)
  \item     \(\deg\)({\tt si}) \(<\) \(\deg\)({\tt hi}),
    \(\deg\)({\tt ti}) \(<\) \(\deg\)({\tt gi}) \((i = 1 ,\ldots, r)\)
  \item     {\tt p} divides {\tt q}
  \item     {\tt f == l1*...*lr (mod q/p)}
  \item     {\tt f == a1*...*ar (mod q)}
  \item     {\tt ui == ai*yi + bi*zi (mod p)} \((i = 1, \ldots, r)\)
  \end{itemize}

 \initialize
  \func{HenselLiftSimultaneously}{%
    \hiki{target}{polynomial},
    \hiki{factors}{list},
    \hiki{cofactors}{list},
    \hiki{bases}{list},
    \hiki{p}{integer}}{\out{HenselLiftSimultaneously}}\\
  \spacing
  % document of basic document
  \quad This object keeps integer polynomial factors which will be lifted by Hensel's lemma.\\
  \spacing
  \quad {\tt f = \param{target}}, {\tt gi} in \param{factors},
  {\tt hi}s in \param{cofactors} and {\tt si}s and {\tt ti}s are in \param{bases}.
%
  \func{from\_factors}{%
    \hiki{target}{polynomial},\ %
    \hiki{factors}{list},\ %
    \hiki{p}{integer},\ %
    \hikiopt{ubound}{integer}{\linklibraryone{sys\#maxint}{sys.maxint}}}{%
    \out{HenselLiftSimultaneously}}
  \spacing
  % document of basic document
  \quad This is a class method to create and return an instance of {\tt HenselLiftSimultaneously}, whose factors are lifted by \linkingone{poly.hensel}{HenselLiftMulti} upto \param{ubound} if it is smaller than {\tt sys.maxint}, or upto {\tt sys.maxint} otherwise.
  You do not have to precompute auxiliary polynomials for the default
  constructor; they will be prepared for you from other arguments.\\
  \spacing
  \quad {\tt f = \param{target}}, {\tt gi}s in \param{factors}.
%
  \method
  \subsubsection{lift -- lift one step}\linkedtwo{poly.hensel}{HenselLiftSimultaneously}{lift}
  \func{lift}{\param{self}}{}\\
  \spacing
  The lift. You should call this method only.
  \subsubsection{first\_lift -- the first step}\linkedtwo{poly.hensel}{HenselLiftSimultaneously}{first\_lift}
  \func{first\_lift}{\param{self}}{}\\
  \spacing
  \quad Start lifting.\\
  {\tt f == l1*l2*...*lr (mod p**2)}\\
  Initialize {\tt di}s, {\tt ui}s, {\tt yi}s and {\tt zi}s.
  Update {\tt ai}s, {\tt bi}s.
  Then, update {\tt q} with {\tt p**2}.
  \subsubsection{general\_lift -- next step}\linkedtwo{poly.hensel}{HenselLiftSimultaneously}{general\_lift}
  \func{general\_lift}{\param{self}}{}\\
  \spacing
  \quad Continue lifting.\\
  {\tt f == a1*a2*...*ar (mod p*q)}\\
  Initialize {\tt ai}s, {\tt ubi}s, {\tt yi}s and {\tt zi}s.
  Then, update {\tt q} with {\tt p*q.}

  \subsection{lift\_upto -- main function}\linkedone{poly.hensel}{lift\_upto}
  \func{lift\_upto}{\param{self},\ %
  \hiki{target}{polynomial},\  %
  \hiki{factors}{list},\ %
  \hiki{p}{integer},\ %
  \hiki{bound}{integer}}{\out{tuple}}\\
\spacing
\quad Hensel lift \param{factors} mod \param{p} of \param{target} upto \param{bound}
and return {\tt factors} mod {\tt q} and the {\tt q} itself.\\
\quad These preconditions should be satisfied:
\begin{itemize}
\item \param{target} is monic.
\item {\tt \param{target} == product(\param{factors}) mod \param{p}}
\end{itemize}
\quad The result {\tt (factors, q)} satisfies the following postconditions:
\begin{itemize}
\item there exist \(k\) s.t. {\tt q == \param{p}**k >= \param{bound}} and
\item {\tt \param{target} == product(factors) mod q}
\end{itemize}

\C

%---------- end document ---------- %

\bibliographystyle{jplain}%use jbibtex
\bibliography{nzmath_references}

\end{document}

